%% latex-leseansicht-vorspann.tex
%% Vorspann für die Leseansicht.
%% Lädt die gemeinsame Datei latex-vorspann.tex mit nicht gesetztem Schalter.

\newif\ifkorrekturansicht
\korrekturansichtfalse

\input{../tex-inputs/latex-vorspann}


         
         \renewcommand{\erwaehntePersonen}{Personen: Max Eugen Burckhard, Felix Salten, Otto Paul Schinnerer, Olga Schnitzler}
         \renewcommand{\erwaehnteInstitutionen}{Institutionen: Niederösterreichische Statthalterei}
         \renewcommand{\erwaehnteOrte}{Orte: Wien}
         \renewcommand{\erwaehnteWerke}{Werke: Excentric, Jugend, Reigen. Zehn Dialoge}
               \section[Hermann Bahr an Arthur Schnitzler, 9. 11. 1903]{ Hermann Bahr an Arthur Schnitzler, 9. 11. 1903}\nopagebreak\mylabel{v}\rehead{ }\begin{ledgroupsized}[t]{13cm}\normalsize\beginnumbering \toendnotes[C]{\smallbreak\pagebreak[2]} \Standort{CUL, Schnitzler, B 5b.}
\physDesc{Brief, 1 Blatt, 2 Seiten, 1336 Zeichen
\newline{}Handschrift: schwarze Tinte, deutsche Kurrent
\newline{}Ordnung: mit Bleistift von unbekannter Hand nummeriert:
                                    »102« }\buchAbdrucke{\weitereDrucke{Hermann Bahr, Arthur Schnitzler: \emph{Briefwechsel, Aufzeichnungen, Dokumente (1891–1931)}. Hg. Kurt Ifkovits und Martin Anton Müller. Göttingen: \emph{Wallstein} 2018, S. 277–278.} }\toendnotes[C]{\smallbreak}\pstart
           \raggedleft{}{\pb}9. 11. 03\pend
           \pstart\center{}Lieber Arthur!\pend\pstart
           Ich habe geſtern Dein »Excentric\pwindex{Schnitzler, Arthur 15.05.1862 – 21.10.1931@\textsc{Schnitzler, Arthur} (15.05.1862 – 21.10.1931), \emph{Schriftsteller, Mediziner}!Excentric16. 07. 1902@\strich\emph{Excentric} {[}16. 07. 1902{]}|pw}« vorgeleſen und
               die Leute haben über das liebenswürdige Fräulein de la Roſière ſo gebrüllt, daß ich
               wirklich bisweilen eine Minute lang warten mußte, bis ſie ſich ſo weit gefaßt hatten,
               mich wieder anzuhören. Die Geſchichte iſt köſtlich und zum Vorleſen ideal. Ich
               ſchicke Dir das Heft mit derſelben Poſt zurück, ich habe mir die betr. Nummer der
                  \label{K_L01337_1v}\edtext{Jugend\pwindex{?? Werk@Nicht ermittelte Verfasserinnen und Verfasser!Jugend1896 – 1940@\emph{Jugend} {[}1896 – 1940{]}|pw}}{\lemma{\textnormal{\emph{Jugend}}}\Cendnote{\textnormal{Arthur Schnitzler\pwindex{Schnitzler, Arthur 15.05.1862 – 21.10.1931@\textsc{Schnitzler, Arthur} (15.05.1862 – 21.10.1931), \emph{Schriftsteller, Mediziner}|pwk}: \emph{Excentric}\pwindex{Schnitzler, Arthur 15.05.1862 – 21.10.1931@\textsc{Schnitzler, Arthur} (15.05.1862 – 21.10.1931), \emph{Schriftsteller, Mediziner}!Excentric16. 07. 1902@\strich\emph{Excentric} {[}16. 07. 1902{]}|pwk}. In: \emph{Jugend}\pwindex{?? Werk@Nicht ermittelte Verfasserinnen und Verfasser!Jugend1896 – 1940@\emph{Jugend} {[}1896 – 1940{]}|pwk},
                     Jg. 7, Nr. 30, {[}16.{]} 7. 1902, S. 492–496.}}}\label{K_L01337_1h} bereits
               verſchafft.\pend
           \pstart
           Noch etwas, ganz aufrichtig. Da Du keine Sitze von mir verlangt haſt, habe ich Dir
               keine \substVorne{}\textsuperscript{\textcolor{gray}{×}}\substDazwischen{}g\substHinten{}eſchickt, weil mir das von mir immer so furchtbar aufdringlich vorkommt,
               Jemandem ungebeten Sitze zu ſchicken, der dann am End erſt ſeine Köchin anflehen muß,
               ſie zu benützen.\pend
           \pstart
           {\pb}Anbei findeſt Du den \label{K_L01337_2v}\edtext{Rek\textcolor{gray}{o}urs, der am 5. d. der Statthalterei\orgindex{Niederoesterreichische Statthalterei@Niederösterreichische Statthalterei|pw} überreicht worden iſt}{\lemma{\textnormal{\emph{Rekours, … iſt}}}\Cendnote{\textnormal{Vgl. Schnitzler\pwindex{Schnitzler, Arthur 15.05.1862 – 21.10.1931@\textsc{Schnitzler, Arthur} (15.05.1862 – 21.10.1931), \emph{Schriftsteller, Mediziner}|pwk} an Otto P. Schinnerer\pwindex{Schinnerer, Otto Paul 1890-11-05 – 1942-11-07@\textsc{Schinnerer, Otto Paul} (1890-11-05 – 1942-11-07), \emph{Wissenschaftler}|pwk}, 6. 2. 1930, in A. S.
                        \emph{Briefe} II,660–664.}}}\label{K_L01337_2h}. Er iſt von mir
               mit Burckhard\pwindex{Burckhard, Max Eugen 14.07.1854 – 16.03.1912@\textsc{Burckhard, Max Eugen} (14.07.1854 – 16.03.1912), \emph{Schriftsteller, Wissenschaftler, Theaterleiter}|pw} berathen und dann von dieſem
               verfaßt worden, was aber, nach ſeinem Wunſch, nicht bekannt werden ſoll. Verſuche,
               den Rekurs in irgend eine Wiener\oindex{Wien@\textbf{Wien}|pw} Zeitung zu
               bringen, ſind durchaus misglückt. Überlege, ob Du ihn eventuell der nächſten Auflage
               des Reigens\pwindex{Schnitzler, Arthur 15.05.1862 – 21.10.1931@\textsc{Schnitzler, Arthur} (15.05.1862 – 21.10.1931), \emph{Schriftsteller, Mediziner}!Reigen. Zehn Dialoge1900@\strich\emph{Reigen. Zehn Dialoge} {[}1900{]}|pw} vordrucken würdeſt. Sag aber nur
               offen Nein, wenn es Dir nicht paßt.\pend
           \pstart
           \label{K_L01337_3v}\edtext{Salten\pwindex{Salten, Felix 06.09.1869 – 08.10.1945@\textsc{Salten, Felix} (06.09.1869 – 08.10.1945), \emph{Schriftsteller, Journalist}|pw} tuſt Du glaub ich unrecht}{\lemma{\textnormal{\emph{Salten … unrecht}}}\Cendnote{\textnormal{Das könnte auf ein verlorenes
                  Korrespondenzstück hinweisen; zum Inhalt siehe die Antwort Schnitzlers\pwindex{Schnitzler, Arthur 15.05.1862 – 21.10.1931@\textsc{Schnitzler, Arthur} (15.05.1862 – 21.10.1931), \emph{Schriftsteller, Mediziner}|pwk}.}}}\label{K_L01337_3h}. Du mußt nur doch die für ihn
               unglaublich heikle und gefährliche Situation bedenken, in der er geſchrieben hat.
               Aber darüber mündlich.\pend
           \pstart
           Mit den beſten Grüßen an Deine Frau\pwindex{Schnitzler, Olga 17.01.1882 – 13.01.1970@\textsc{Schnitzler, Olga} (17.01.1882 – 13.01.1970), \emph{Schauspielerin, Sängerin}|pwv}{\\[\baselineskip]}herzlichſt Dein{\\[\baselineskip]}\spacefill\mbox{Hermann}\pend
           \leftskip=0em{}
         
         \endnumbering\mylabel{h}\end{ledgroupsized}  \newcommand{\dateiname}{L01337}\newcommand{\titel}{Hermann Bahr an Arthur Schnitzler, 9. 11. 1903}\newcommand{\editorInnen}{ Kurt Ifkovits,  Martin Anton Müller}%% latex-leseansicht-abspann.tex
%% Abspann für die Leseansicht.
%% Der Schalter \ifkorrekturansicht ist bereits durch den Vorspann gesetzt.

%% latex-abspann.tex
%% Gemeinsamer Abspann für Korrekturansicht und Leseansicht.
%% Setzt den Schalter \ifkorrekturansicht voraus (gesetzt in den
%% einbindenden Dateien latex-korrekturansicht-abspann.tex bzw.
%% latex-leseansicht-abspann.tex).
%% ---------------------------------------------------------------

\normalsize

% Das esempio-Environment wird nur in der Leseansicht benötigt
\ifkorrekturansicht\else
\newenvironment{esempio}[3]%
{
    \vspace{1.5ex}
    \rlap{\underline{#1}}
    \par
    \setlength{\parindent}{0cm}
    \nopagebreak
    \leftskip=#2cm
    \rightskip=#3cm
}
{
    \par
}
\fi

\doendnotes{C}
\bigskip
\vfill

\clearpage

\footnotesize

\ifkorrekturansicht
  \lohead{\textsc{register}}
\fi

% theindex-Environment neu definieren ohne reledmac
\makeatletter
\renewenvironment{theindex}{%
  \ifkorrekturansicht
    \section*{\indexname}%
  \else
    \subsubsection*{Index der erwähnten Entitäten}%
  \fi
  \setlength{\parindent}{0pt}%
  \setlength{\parskip}{0pt plus 0.3pt}%
  \let\item\@idxitem
}{%
  \ifkorrekturansicht\clearpage\fi
}
\makeatother

\IfFileExists{\jobname-pw.ind}{\input{\jobname-pw.ind}}{}

% Quellenangabe nur in der Leseansicht
\ifkorrekturansicht\else
% Fallback-Definitionen, falls die .tex-Datei \titel etc. nicht gesetzt hat
\providecommand{\titel}{}
\providecommand{\editorInnen}{}
\providecommand{\dateiname}{\jobname}

\vspace{3cm}

\vfill

\footnotesize
\textsc{Quelle}: \titel. Herausgegeben von {\editorInnen}. In: \emph{Arthur Schnitzler: Briefwechsel mit Autorinnen und Autoren}.
 Digitale Edition, https://schnitzler-briefe.acdh.oeaw.ac.at/{\dateiname}.html (Stand \today)
\fi

\end{document}


      