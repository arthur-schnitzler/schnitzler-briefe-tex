%% latex-korrekturansicht-vorspann.tex
%% Vorspann für die Korrekturansicht.
%% Lädt die gemeinsame Datei latex-vorspann.tex mit gesetztem Schalter.

\newif\ifkorrekturansicht
\korrekturansichttrue

\input{../tex-inputs/latex-vorspann}


\section[Hermann Bahr an Arthur Schnitzler, 9. 11. 1903]{L01337 Hermann Bahr an Arthur Schnitzler, 9. 11. 1903}
\nopagebreak\mylabel{L01337v}
\rehead{ }\normalsize\beginnumbering\briefempfaengerindex{Schnitzler, Arthur@\textsc{Schnitzler, Arthur}!zzzBahr, Hermann@\emph{von Hermann Bahr}!1903-11-091@{9. 11. 1903}|(be}
\toendnotes[C]{\smallbreak\pagebreak[2]}\Standort{CUL, Schnitzler, B 5b.}
\physDesc{Brief, 1 Blatt, 2 Seiten, 1336 Zeichen
\newline{}Handschrift: schwarze Tinte, deutsche Kurrent
\newline{}Ordnung: mit Bleistift von unbekannter Hand nummeriert: »102« }
\buchAbdrucke{\weitereDrucke{Hermann Bahr, Arthur Schnitzler: \emph{Briefwechsel, Aufzeichnungen, Dokumente (1891–1931)}. Göttingen: \emph{Wallstein} 2018, S. 277–278.} }\toendnotes[C]{\smallbreak}
\pstart
           \raggedleft{}{\pb}9. 11. 03\pend
           
\pstart\center{}Lieber Arthur!\pend\vspace{0.5em}
\pstart
           Ich habe geſtern Dein »Excentric\pwindex{Excentric@\emph{Excentric}|pw}« vorgeleſen und
               die Leute haben über das liebenswürdige Fräulein de la Roſière ſo gebrüllt, daß ich
               wirklich bisweilen eine Minute lang warten mußte, bis ſie ſich ſo weit gefaßt hatten,
               mich wieder anzuhören. Die Geſchichte iſt köſtlich und zum Vorleſen ideal. Ich
               ſchicke Dir das Heft mit derſelben Poſt zurück, ich habe mir die betr. Nummer der
                  \label{K_L01337-1v}\edtext{Jugend\pwindex{Jugend@\emph{Jugend}|pw}}{\lemma{\textnormal{\emph{Jugend}}}\Cendnote{\textnormal{Arthur Schnitzler: \emph{Excentric}\pwindex{Excentric@\emph{Excentric}|pwk}. In: \emph{Jugend}\pwindex{Jugend@\emph{Jugend}|pwk},
                     Jg. 7, Nr. 30, {[}16.{]} 7. 1902, S. 492–496.}}}\label{K_L01337-1} bereits
               verſchafft.\pend
           
\pstart
           Noch etwas, ganz aufrichtig. Da Du keine Sitze von mir verlangt haſt, habe ich Dir
               keine \substVorne{}\textsuperscript{\textcolor{gray}{×}}\substDazwischen{}g\substHinten{}eſchickt, weil mir das von mir immer so furchtbar aufdringlich vorkommt,
               Jemandem ungebeten Sitze zu ſchicken, der dann am End erſt ſeine Köchin anflehen muß,
               ſie zu benützen.\pend
           
\pstart
           {\pb}Anbei findeſt Du den \label{K_L01337-2v}\edtext{Rek\textcolor{gray}{o}urs, der am 5. d. der Statthalterei\orgindex{Niederoesterreichische Statthalterei@Niederösterreichische Statthalterei|pw} überreicht worden iſt}{\lemma{\textnormal{\emph{Rekours, … iſt}}}\Cendnote{\textnormal{Vgl. Schnitzler an Otto P. Schinnerer\pwindex{Schinnerer, Otto Paul 1890-11-05 – 1942-11-07@\textsc{Schinnerer, Otto Paul} (1890-11-05 – 1942-11-07), \emph{Literaturwissenschaftler/Literaturwissenschaftlerin}|pwk}, 6. 2. 1930, in A. S. 
                     \emph{Briefe 1913–1931}, S. 660–664.}}}\label{K_L01337-2}. Er iſt von mir
               mit Burckhard\pwindex{Burckhard, Max Eugen 14.07.1854 – 16.03.1912@\textsc{Burckhard, Max Eugen} (14.07.1854 – 16.03.1912), \emph{Schriftsteller/Schriftstellerin, Rechtswissenschaftler/Rechtswissenschaftlerin, Theaterleiter/Theaterleiterin}|pw} berathen und dann von dieſem
               verfaßt worden, was aber, nach ſeinem Wunſch, nicht bekannt werden ſoll. Verſuche,
               den Rekurs in irgend eine Wiener\oindex{Wien@\textbf{Wien}, \emph{A.ADM2}|pw} Zeitung zu
               bringen, ſind durchaus misglückt. Überlege, ob Du ihn eventuell der nächſten Auflage
               des Reigens\pwindex{Reigen. Zehn Dialoge@\emph{Reigen. Zehn Dialoge}|pw} vordrucken würdeſt. Sag aber nur
               offen Nein, wenn es Dir nicht paßt.\pend
           
\pstart
           \label{K_L01337-3v}\edtext{Salten\pwindex{Salten, Felix 06.09.1869 – 08.10.1945@\textsc{Salten, Felix} (06.09.1869 – 08.10.1945), \emph{Schriftsteller/Schriftstellerin, Journalist/Journalistin, Chefredakteur/Chefredakteurin}|pw} tuſt Du glaub ich unrecht}{\lemma{\textnormal{\emph{Salten … unrecht}}}\Cendnote{\textnormal{Das dürfte auf ein verlorenes
                  Korrespondenzstück hinweisen, in dem Schnitzler seine Verärgerung über Saltens\pwindex{Salten, Felix 06.09.1869 – 08.10.1945@\textsc{Salten, Felix} (06.09.1869 – 08.10.1945), \emph{Schriftsteller/Schriftstellerin, Journalist/Journalistin, Chefredakteur/Chefredakteurin}|pwk}{ } Feuilleton \emph{Arthur Schnitzler und sein »Reigen«}\pwindex{Arthur Schnitzler und sein »Reigen«@\emph{Arthur Schnitzler und sein »Reigen«}|pwk} zum Ausdruck gebracht
                  hat (Felix Salten\pwindex{Salten, Felix 06.09.1869 – 08.10.1945@\textsc{Salten, Felix} (06.09.1869 – 08.10.1945), \emph{Schriftsteller/Schriftstellerin, Journalist/Journalistin, Chefredakteur/Chefredakteurin}|pwk}: \emph{Arthur Schnitzler und sein »Reigen«}\pwindex{Arthur Schnitzler und sein »Reigen«@\emph{Arthur Schnitzler und sein »Reigen«}|pwk} (In: \emph{Die Zeit}\pwindex{Zeit@\emph{Die Zeit}|pwk}, Jg. 2, Nr. 398, 7. 11. 1903, Morgenblatt, S. 1–2). Vgl. Arthur Schnitzler an Felix Salten, 7. 11. 1903.}}}\label{K_L01337-3}. Du mußt nur
               doch die für ihn unglaublich heikle und gefährliche Situation bedenken, in der er
               geſchrieben hat. Aber darüber mündlich.\pend
           
\pstart
           Mit den beſten Grüßen an Deine Frau\pwindex{Schnitzler, Olga 17.01.1882 – 13.01.1970@\textsc{Schnitzler, Olga} (17.01.1882 – 13.01.1970), \emph{Schauspieler/Schauspielerin, Sänger/Sängerin}|pwv}{\\[\baselineskip]}herzlichſt Dein{\\[\baselineskip]}\spacefill\mbox{Hermann}\pend
           \leftskip=0em{}\selectlanguage{ngerman}\endnumbering\briefempfaengerindex{Schnitzler, Arthur@\textsc{Schnitzler, Arthur}!zzzBahr, Hermann@\emph{von Hermann Bahr}!1903-11-091@{9. 11. 1903}|)be}\mylabel{L01337h}  \normalsize

\doendnotes{C}
\bigskip
\vfill

\clearpage

\footnotesize

\lohead{\textsc{register}}

% Definiere theindex-Environment komplett neu ohne reledmac
\makeatletter
\renewenvironment{theindex}{%
  \section*{\indexname}%
  \setlength{\parindent}{0pt}%
  \setlength{\parskip}{0pt plus 0.3pt}%
  \let\item\@idxitem
}{%
  \clearpage
}
\makeatother

\IfFileExists{\jobname-pw.ind}{\input{\jobname-pw.ind}}{}

\end{document}

      