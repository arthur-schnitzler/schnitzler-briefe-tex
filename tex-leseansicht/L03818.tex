%% latex-korrekturansicht-vorspann.tex
%% Vorspann für die Korrekturansicht.
%% Lädt die gemeinsame Datei latex-vorspann.tex mit gesetztem Schalter.

\newif\ifkorrekturansicht
\korrekturansichttrue

\input{../tex-inputs/latex-vorspann}


\section[Sigmund Freud an Arthur Schnitzler, 8. 3. 1926]{L03818 Sigmund Freud an Arthur Schnitzler, 8. 3. 1926}
\nopagebreak\mylabel{L03818v}
\rehead{ }\normalsize\beginnumbering\briefempfaengerindex{Schnitzler, Arthur@\textsc{Schnitzler, Arthur}!zzzFreud, Sigmund@\emph{von Sigmund Freud}!1926-03-081@{8. 3. 1926}|(be}
\toendnotes[C]{\smallbreak\pagebreak[2]}\Standort{CUL, Schnitzler, B 31.}
\physDesc{Brief, 1 Blatt, 2 Seiten, 568 Zeichen
\newline{}Handschrift: , deutsche Kurrent}\toendnotes[C]{\smallbreak}
\pstart
           \raggedleft{}{\pb}8. 3. 26\pend
           
\pstart
           \textcolor{gray}{\textbf{PROF. D\textsuperscript{R.} FREUD }}\hfill \textcolor{gray}{\textbf{WIEN IX., BERGGASSE 19\oindex{Berggasse 19@\textbf{Berggasse 19}, \emph{Wohngebäude (K.WHS)}|pw}}}\pend
           
\pstart{}Verehrteſter!\pend\vspace{0.5em}
\pstart
           Ich war Ihnen noch nie ſo nah. Ich hauſe \label{K_L03818-1v}\edtext{im Sanatorium\oindex{Cottage-Sanatorium fuer Nerven- und Stoffwechselkranke@\textbf{Cottage-Sanatorium für Nerven- und Stoffwechselkranke}, \emph{Sanatorium (K.SAN)}|pwv}}{\lemma{\textnormal{\emph{im Sanatorium}}}\Cendnote{\textnormal{Vom 5. 3. bis zum
                     2. 4. 1926 hielt sich Sigmund
                     Freud\pwindex{Freud, Sigmund 06.05.1856 – 23.09.1939@\textsc{Freud, Sigmund} (06.05.1856 – 23.09.1939), \emph{Psychoanalytiker/Psychoanalytikerin}|pwk} im Cottage-Sanatorium\oindex{Cottage-Sanatorium fuer Nerven- und Stoffwechselkranke@\textbf{Cottage-Sanatorium für Nerven- und Stoffwechselkranke}, \emph{Sanatorium (K.SAN)}|pwk} in der
                     Sternwartestraße 74\oindex{Sternwartestrasse 74@\textbf{Sternwartestraße 74}, \emph{Gebäude (K.GBD)}|pwk} auf. Schnitzler besuchte ihn dort zwei Mal, vgl. A. S.: \emph{Tagebuch}, 12. 3. 1926, und A. S.: \emph{Tagebuch}, 26. 3. 1926.}}}\label{K_L03818-1} in Ihrer Straße\oindex{Sternwartestrasse@\textbf{Sternwartestraße}, \emph{R.ST}|pw} u mache auf Wunſch der Interniſten
               Herztherapie, befinde mich aber ſubjektiv recht wol. \pend
           
\pstart
           Infolge eines früheren Verſäumnißes kann ich mich heute in Einem für
               zwei Ihrer Geſchenke\textcolor{red}{\textsuperscript{\textbf{KEY}}} bedanken. Die \label{K_L03818-2v}\edtext{begleitende Brochüre\pwindex{Hemmung, Symptom und Angst@\emph{Hemmung, Symptom und Angst}|pwv}}{\lemma{\textnormal{\emph{begleitende Brochüre}}}\Cendnote{\textnormal{Schnitzlers{ }\emph{Tagebuch}\pwindex{Tagebuch@\emph{Tagebuch}|pwk}eintrag bestätigt den Erhalt von und
                  die Beschäftigung mit Freuds\pwindex{Freud, Sigmund 06.05.1856 – 23.09.1939@\textsc{Freud, Sigmund} (06.05.1856 – 23.09.1939), \emph{Psychoanalytiker/Psychoanalytikerin}|pwk}{ }Text\pwindex{Hemmung, Symptom und Angst@\emph{Hemmung, Symptom und Angst}|pwkv} (\emph{Hemmung, Symptom und Angst}\pwindex{Hemmung, Symptom und Angst@\emph{Hemmung, Symptom und Angst}|pwk}. Leipzig\oindex{Leipzig@\textbf{Leipzig}, \emph{P.PPLA3}|pwk}, Wien\oindex{Wien@\textbf{Wien}, \emph{A.ADM2}|pwk}, Zürich\oindex{Zuerich@\textbf{Zürich}, \emph{P.PPLA}|pwk}: \emph{Internationaler Psychoanalytischer Verlag}\orgindex{Internationaler Psychoanalytischer Verlag@Internationaler Psychoanalytischer Verlag|pwk}{ }1926), vgl. A. S.: \emph{Tagebuch}, 9. 3. 1926.}}}\label{K_L03818-2} ſoll in keiner Weiſe eine Revanche ſein, ſie iſt eben nur meine letzte
                  Publikation\pwindex{Hemmung, Symptom und Angst@\emph{Hemmung, Symptom und Angst}|pwv} – vielleicht in
               jedem Sinne – ſonst aber recht \strikeout{untereſ} unintereſſant
               und beſonders für Sie unwichtig. \pend
           
\pstart
           Troſt, daß Sie sie ja weder zu leſen noch ſich darüber {\pb}zu äußern brauchen.\pend
           
\pstart
           Mit herzl Gruß{\\[\baselineskip]}\spacefill\mbox{Ihr Freud}\pend
           \leftskip=0em{}\selectlanguage{ngerman}\endnumbering\briefempfaengerindex{Schnitzler, Arthur@\textsc{Schnitzler, Arthur}!zzzFreud, Sigmund@\emph{von Sigmund Freud}!1926-03-081@{8. 3. 1926}|)be}\mylabel{L03818h}
\begin{anhang}
\end{anhang}\normalsize

\doendnotes{C}
\bigskip
\vfill

\clearpage

\footnotesize

\lohead{\textsc{register}}

% Definiere theindex-Environment komplett neu ohne reledmac
\makeatletter
\renewenvironment{theindex}{%
  \section*{\indexname}%
  \setlength{\parindent}{0pt}%
  \setlength{\parskip}{0pt plus 0.3pt}%
  \let\item\@idxitem
}{%
  \clearpage
}
\makeatother

\IfFileExists{\jobname-pw.ind}{\input{\jobname-pw.ind}}{}

\end{document}

      