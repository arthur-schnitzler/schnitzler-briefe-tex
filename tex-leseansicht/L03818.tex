%% latex-leseansicht-vorspann.tex
%% Vorspann für die Leseansicht.
%% Lädt die gemeinsame Datei latex-vorspann.tex mit nicht gesetztem Schalter.

\newif\ifkorrekturansicht
\korrekturansichtfalse

\input{../tex-inputs/latex-vorspann}


\section[Sigmund Freud an Arthur Schnitzler, 8. 3. 1926]{L03818 Sigmund Freud an Arthur Schnitzler, 8. 3. 1926}
\nopagebreak\mylabel{L03818v}
\rehead{ }\normalsize\beginnumbering\briefempfaengerindex{Schnitzler, Arthur@\textsc{Schnitzler, Arthur}!zzzFreud, Sigmund@\emph{von Sigmund Freud}!1926-03-081@{8. 3. 1926}|(be}
\toendnotes[C]{\smallbreak\pagebreak[2]}
\correspDesc{Versand  durch Sigmund Freud am 8. 3. 1926 in Wien
\newline{}Erhalt  durch Arthur Schnitzler im Zeitraum [8. 3. 1926
                  – 11. 3. 1926?] in Wien}\toendnotes[C]{\smallbreak}
\Standort{CUL, Schnitzler, B 31.}
\physDesc{Brief, 1 Blatt, 2 Seiten, 568 Zeichen
\newline{}Handschrift: blaue Tinte, deutsche Kurrent}
\buchAbdrucke{\weitereDrucke{1) Sigmund Freud: \emph{Briefe an Arthur Schnitzler.}Herausgegeben von Henry Schnitzler In: \emph{Neue deutsche Rundschau}, Jg. 66 (Januar 1955) Nr. 1, S. 99.} \weitereDrucke{2) Sigmund Freud: \emph{Sigmund Freud Edition. Digitale historisch-kritische
                        Gesamtausgabe}. Herausgegeben von Christine Diercks, Arkadi Blatow und Elisabeth Skale. (2014–2025) \url{https://www.freudedition.net/briefe/freud-sigmund/schnitzler-arthur/1926/03/08}.} }\toendnotes[C]{\smallbreak}
\pstart
           \raggedleft{}{\pb}8.\,3.\,26\pend
           
\pstart
           \textcolor{gray}{\textbf{PROF. D\textsuperscript{R.} FREUD }}\hfill \textcolor{gray}{\textbf{WIEN IX., BERGGASSE 19\oindex{Wien@\textbf{Wien}!IX., Alsergrund@\textbf{IX., Alsergrund}!Berggasse 19@\textbf{Berggasse 19}, \emph{Wohngebäude}|pw}}}\pend
           
\pstart{}Verehrteſter!\pend\vspace{0.5em}
\pstart
           Ich war Ihnen noch nie ſo nah. Ich hauſe \label{K_L03818-1v}\edtext{im Sanatorium\oindex{Wien@\textbf{Wien}!XVIII., Währing@\textbf{XVIII., Währing}!Cottage-Sanatorium für Nerven- und Stoffwechselkranke@\textbf{Cottage-Sanatorium für Nerven- und Stoffwechselkranke}, \emph{Sanatorium}|pwv}}{\lemma{\textnormal{\emph{im Sanatorium}}}\Cendnote{\textnormal{Vom 5. 3. 1926 bis zum
                     2. 4. 1926 hielt sich Sigmund
                     Freud\pwindex{Freud, Sigmund 6.\,5.\,1856 Pribor – 23.\,9.\,1939 London@\textsc{Freud, Sigmund} (6.\,5.\,1856 Pribor – 23.\,9.\,1939 London), \emph{Psychoanalytiker}|pwk} im Cottage-Sanatorium\oindex{Wien@\textbf{Wien}!XVIII., Währing@\textbf{XVIII., Währing}!Cottage-Sanatorium für Nerven- und Stoffwechselkranke@\textbf{Cottage-Sanatorium für Nerven- und Stoffwechselkranke}, \emph{Sanatorium}|pwk} in der
                     Sternwartestraße 74\oindex{Wien@\textbf{Wien}!XVIII., Währing@\textbf{XVIII., Währing}!Sternwartestraße 74@\textbf{Sternwartestraße 74}, \emph{Gebäude}|pwk} auf. Schnitzler besuchte ihn dort zwei Mal, vgl. A. S.: \emph{Tagebuch}, 12. 3. 1926, und 26. 3. 1926.}}}\label{K_L03818-1} in Ihrer Straße\oindex{Wien@\textbf{Wien}!XVIII., Währing@\textbf{XVIII., Währing}!Sternwartestraße@\textbf{Sternwartestraße}, \emph{Straße}|pw} u mache auf Wunſch der Interniſten
               Herztherapie, befinde mich aber ſubjektiv recht wol.\pend
           
\pstart
           Infolge eines früheren Verſäumnißes kann ich mich heute in Einem für
               zwei Ihrer \label{K_L03818-2v}\edtext{Geſchenke\pwindex{Schnitzler, Arthur 15.\,5.\,1862 Wien – 21.\,10.\,1931 ebd.@\textsc{Schnitzler, Arthur} (15.\,5.\,1862 Wien – 21.\,10.\,1931 ebd.), \emph{Schriftsteller, Mediziner}!Gang zum Weiher. Dramatische Dichtung@\strich\emph{Der Gang zum Weiher. Dramatische Dichtung}|pwv}\pwindex{Schnitzler, Arthur 15.\,5.\,1862 Wien – 21.\,10.\,1931 ebd.@\textsc{Schnitzler, Arthur} (15.\,5.\,1862 Wien – 21.\,10.\,1931 ebd.), \emph{Schriftsteller, Mediziner}!Frau des Richters. Novelle@\strich\emph{Die Frau des Richters. Novelle}|pwv}}{\lemma{\textnormal{\emph{Geschenke}}}\Cendnote{\textnormal{Anfang März erschien \emph{Der Gang zum
                     Weiher}\pwindex{Schnitzler, Arthur 15.\,5.\,1862 Wien – 21.\,10.\,1931 ebd.@\textsc{Schnitzler, Arthur} (15.\,5.\,1862 Wien – 21.\,10.\,1931 ebd.), \emph{Schriftsteller, Mediziner}!Gang zum Weiher. Dramatische Dichtung@\strich\emph{Der Gang zum Weiher. Dramatische Dichtung}|pwk}. Beim zweiten Werk dürfte es sich um \emph{Die Frau des Richters}\pwindex{Schnitzler, Arthur 15.\,5.\,1862 Wien – 21.\,10.\,1931 ebd.@\textsc{Schnitzler, Arthur} (15.\,5.\,1862 Wien – 21.\,10.\,1931 ebd.), \emph{Schriftsteller, Mediziner}!Frau des Richters. Novelle@\strich\emph{Die Frau des Richters. Novelle}|pwk} handeln.}}}\label{K_L03818-2} bedanken. Die
                  \label{K_L03818-3v}\edtext{begleitende Brochüre\pwindex{Freud, Sigmund 6.\,5.\,1856 Pribor – 23.\,9.\,1939 London@\textsc{Freud, Sigmund} (6.\,5.\,1856 Pribor – 23.\,9.\,1939 London), \emph{Psychoanalytiker}!Hemmung, Symptom und Angst@\strich\emph{Hemmung, Symptom und Angst}|pwv}}{\lemma{\textnormal{\emph{begleitende Brochüre}}}\Cendnote{\textnormal{Schnitzlers{ }\emph{Tagebuch}\pwindex{Schnitzler, Arthur 15.\,5.\,1862 Wien – 21.\,10.\,1931 ebd.@\textsc{Schnitzler, Arthur} (15.\,5.\,1862 Wien – 21.\,10.\,1931 ebd.), \emph{Schriftsteller, Mediziner}!Tagebuch@\strich\emph{Tagebuch}|pwk}-Eintrag vom 9. 3. 1926 bestätigt den Erhalt von und
                  die Beschäftigung mit Freuds\pwindex{Freud, Sigmund 6.\,5.\,1856 Pribor – 23.\,9.\,1939 London@\textsc{Freud, Sigmund} (6.\,5.\,1856 Pribor – 23.\,9.\,1939 London), \emph{Psychoanalytiker}|pwk} Text \emph{Hemmung, Symptom und Angst}\pwindex{Freud, Sigmund 6.\,5.\,1856 Pribor – 23.\,9.\,1939 London@\textsc{Freud, Sigmund} (6.\,5.\,1856 Pribor – 23.\,9.\,1939 London), \emph{Psychoanalytiker}!Hemmung, Symptom und Angst@\strich\emph{Hemmung, Symptom und Angst}|pwk}. (Leipzig\oindex{Leipzig@\textbf{Leipzig}, \emph{Hauptstadt}|pwk}, Wien\oindex{Wien@\textbf{Wien}, \emph{Verwaltungsgebiet}|pwk}, Zürich\oindex{Zürich@\textbf{Zürich}|pwk}: \emph{Internationaler Psychoanalytischer Verlag}\orgindex{Internationaler Psychoanalytischer Verlag@Internationaler Psychoanalytischer Verlag|pwk}{ }1926).}}}\label{K_L03818-3} ſoll in keiner Weiſe eine Revanche{ }ſein, ſie iſt eben nur meine letzte
                  Publikation\pwindex{Freud, Sigmund 6.\,5.\,1856 Pribor – 23.\,9.\,1939 London@\textsc{Freud, Sigmund} (6.\,5.\,1856 Pribor – 23.\,9.\,1939 London), \emph{Psychoanalytiker}!Hemmung, Symptom und Angst@\strich\emph{Hemmung, Symptom und Angst}|pwv} – vielleicht in
               jedem Sinne –{ }ſonst aber recht \strikeout{untereſ} unintereſſant
               und beſonders für Sie unwichtig.\pend
           
\pstart
           Troſt, daß Sie sie ja weder zu leſen noch ſich darüber {\pb}zu äußern brauchen.\pend
           
\pstart
           Mit herzl Gruß{\\[\baselineskip]}\spacefill\mbox{Ihr Freud}\pend
           \leftskip=0em{}\selectlanguage{ngerman}\endnumbering\briefempfaengerindex{Schnitzler, Arthur@\textsc{Schnitzler, Arthur}!zzzFreud, Sigmund@\emph{von Sigmund Freud}!1926-03-081@{8. 3. 1926}|)be}\mylabel{L03818h}
\begin{anhang}
\end{anhang}\newcommand{\dateiname}{L03818}\newcommand{\titel}{Sigmund Freud an Arthur Schnitzler, 8. 3. 1926}\newcommand{\editorInnen}{Selma Jahnke und Martin Anton Müller}%% latex-leseansicht-abspann.tex
%% Abspann für die Leseansicht.
%% Der Schalter \ifkorrekturansicht ist bereits durch den Vorspann gesetzt.

%% latex-abspann.tex
%% Gemeinsamer Abspann für Korrekturansicht und Leseansicht.
%% Setzt den Schalter \ifkorrekturansicht voraus (gesetzt in den
%% einbindenden Dateien latex-korrekturansicht-abspann.tex bzw.
%% latex-leseansicht-abspann.tex).
%% ---------------------------------------------------------------

\normalsize

% Das esempio-Environment wird nur in der Leseansicht benötigt
\ifkorrekturansicht\else
\newenvironment{esempio}[3]%
{
    \vspace{1.5ex}
    \rlap{\underline{#1}}
    \par
    \setlength{\parindent}{0cm}
    \nopagebreak
    \leftskip=#2cm
    \rightskip=#3cm
}
{
    \par
}
\fi

\doendnotes{C}
\bigskip
\vfill

\clearpage

\footnotesize

\ifkorrekturansicht
  \lohead{\textsc{register}}
\fi

% theindex-Environment neu definieren ohne reledmac
\makeatletter
\renewenvironment{theindex}{%
  \ifkorrekturansicht
    \section*{\indexname}%
  \else
    \subsubsection*{Index der erwähnten Entitäten}%
  \fi
  \setlength{\parindent}{0pt}%
  \setlength{\parskip}{0pt plus 0.3pt}%
  \let\item\@idxitem
}{%
  \ifkorrekturansicht\clearpage\fi
}
\makeatother

\IfFileExists{\jobname-pw.ind}{\input{\jobname-pw.ind}}{}

% Quellenangabe nur in der Leseansicht
\ifkorrekturansicht\else
% Fallback-Definitionen, falls die .tex-Datei \titel etc. nicht gesetzt hat
\providecommand{\titel}{}
\providecommand{\editorInnen}{}
\providecommand{\dateiname}{\jobname}

\vspace{3cm}

\vfill

\footnotesize
\textsc{Quelle}: \titel. Herausgegeben von {\editorInnen}. In: \emph{Arthur Schnitzler: Briefwechsel mit Autorinnen und Autoren}.
 Digitale Edition, https://schnitzler-briefe.acdh.oeaw.ac.at/{\dateiname}.html (Stand \today)
\fi

\end{document}


