%% latex-leseansicht-vorspann.tex
%% Vorspann für die Leseansicht.
%% Lädt die gemeinsame Datei latex-vorspann.tex mit nicht gesetztem Schalter.

\newif\ifkorrekturansicht
\korrekturansichtfalse

\input{../tex-inputs/latex-vorspann}


\section[Arthur Schnitzler an Georg Brandes, 8. 5. 1899]{L00915 Arthur Schnitzler an Georg Brandes, 8. 5. 1899}
\nopagebreak\mylabel{L00915v}
\rehead{ }\normalsize\beginnumbering\briefempfaengerindex{Brandes, Georg@\textsc{Brandes, Georg}!zzzSchnitzler, Arthur@\emph{von Arthur Schnitzler}!1899-05-081@{8. 5. 1899}|(be}
\toendnotes[C]{\smallbreak\pagebreak[2]}
\correspDesc{Versand  durch Arthur Schnitzler am 8. 5. 1899 in Wien
\newline{}Erhalt  durch Georg Brandes im Zeitraum [9. 5. 1899
                  – 13. 5. 1899?] in Kopenhagen}\toendnotes[C]{\smallbreak}
\Standort{Kopenhagen, Det Kongelige Bibliotek, Georg Brandes Arkiv, box 125.}
\physDesc{Brief, 1 Blatt, 4 Seiten, 1294 Zeichen
\newline{}Handschrift: schwarze Tinte, deutsche Kurrent
\newline{}Ordnung: mit Bleistift von unbekannter Hand nummeriert:
                                    »15« und datiert: »8/5 99« und nummeriert: »15.« }
\buchAbdrucke{\weitereDrucke{1) Georg Brandes, Arthur Schnitzler: \emph{Ein Briefwechsel}. Herausgegeben von Kurt Bergel. Bern: \emph{Francke} 1956, S. 75.} \weitereDrucke{2) Arthur Schnitzler: \emph{Briefe 1875–1912}. Herausgegeben von Therese Nickl und Heinrich Schnitzler. Frankfurt am Main: \emph{S. Fischer} 1981, S. 370–371.} }\toendnotes[C]{\smallbreak}
\pstart{}{\pb}Lieber und verehrter Herr Brandes,\pend\vspace{0.5em}
\pstart
           zugleich mit dieſem Brief geht ein neues Buch\pwindex{Schnitzler, Arthur 15.\,5.\,1862 Wien – 21.\,10.\,1931 ebd.@\textsc{Schnitzler, Arthur} (15.\,5.\,1862 Wien – 21.\,10.\,1931 ebd.), \emph{Schriftsteller, Mediziner}!grüne Kakadu – Paracelsus – Die Gefährtin. Drei Einakter@\strich\emph{Der grüne Kakadu – Paracelsus – Die Gefährtin. Drei Einakter}|pwv} an Sie ab, das 3 Einakter von mir enthält. Sie werden{ }ſchon ziemlich viel gegeben und insbeſondere der »Kakadu\pwindex{Schnitzler, Arthur 15.\,5.\,1862 Wien – 21.\,10.\,1931 ebd.@\textsc{Schnitzler, Arthur} (15.\,5.\,1862 Wien – 21.\,10.\,1931 ebd.), \emph{Schriftsteller, Mediziner}!grüne Kakadu. Groteske in einem Akt@\strich\emph{Der grüne Kakadu. Groteske in einem Akt}|pw}« amüſirt die Leute{ }ſehr. –\pend
           
\pstart
           – Weiter ka{\geminationn} ich Ihnen heute kaum was{ }ſagen. Vor{ }ſieben
               Wochen iſt das Geſchöpf\pwindex{Reinhard, Marie 13.\,3.\,1871 Wien – 18.\,3.\,1899 ebd.@\textsc{Reinhard, Marie} (13.\,3.\,1871 Wien – 18.\,3.\,1899 ebd.), \emph{Gesangspädagogin}|pwv}
               begraben worden, das ich von allen {\pb}Menſchen der
               Erde am liebſten gehabt habe, meine Geliebte, Freundin und Braut – die durch mehr als
               vier Jahre meinem Leben{ }ſeinen ganzen Sinn und{ }ſeine ganze Freude gegeben hat, – und{ }ſeither dämmere ich hin, aber exiſtire kaum mehr. Aus der Fülle der Geſundheit und
               Jugend hat{ }ſie eine blödſinnige und tückiſche Krankheit innerhalb zweier Tage ins
               Grab geriſſen, und ich habe{ }ſie{ }ſterben geſehen, bei vollem Bewußt{\pb}ſein{ }ſterben geſehn. Bitte{ }ſagen Sie mir kein Wort
               darüber. Ich mußte es Ihnen aber{ }ſagen. –\pend
           
\pstart
           Jener däniſche\oindex{Dänemark@\textbf{Dänemark}|pw}{ }Schriftſteller\pwindex{Larsen, Karl 28.\,7.\,1860 Rendsburg – 11.\,7.\,1931 Kopenhagen@\textsc{Larsen, Karl} (28.\,7.\,1860 Rendsburg – 11.\,7.\,1931 Kopenhagen), \emph{Schriftsteller}|pwv} hat{ }ſich bei
               mir nicht blicken laſſen. Allerdings war ich einige Male von Wien\oindex{Wien@\textbf{Wien}, \emph{Verwaltungsgebiet}|pw} abweſend. Laſſen Sie mich recht bald hören wie es Ihnen
               geht, ob Sie endgiltig geſund{ }ſind und wie Sie mit Ihren Plänen für den Sommer{ }ſtehn. –\pend
           
\pstart
           Paul Goldmann\pwindex{Goldmann, Paul 31.\,1.\,1865 Breslau – 25.\,9.\,1935 Wien@\textsc{Goldmann, Paul} (31.\,1.\,1865 Breslau – 25.\,9.\,1935 Wien), \emph{Schriftsteller, Journalist}|pw} iſt wieder in Frankfurt\oindex{Frankfurt am Main@\textbf{Frankfurt am Main}, \emph{Hauptstadt}|pw} und reiſt viel für{ }ſein Blatt\orgindex{Frankfurter Zeitung@Frankfurter Zeitung|pwv}.\pend
           
\pstart
           {\pb}Richard Beer Hofmann\pwindex{Beer-Hofmann, Richard 11.\,7.\,1866 Wien – 26.\,9.\,1945 New York City@\textsc{Beer-Hofmann, Richard} (11.\,7.\,1866 Wien – 26.\,9.\,1945 New York City), \emph{Schriftsteller}|pw} hat zwei Kinder, Mirjam\pwindex{Beer-Hofmann, Mirjam 4.\,9.\,1897 Wien – 24.\,12.\,1984 New York City@\textsc{Beer-Hofmann, Mirjam} (4.\,9.\,1897 Wien – 24.\,12.\,1984 New York City)|pw} und Naemi\strikeout{e}\pwindex{Beer-Hofmann, Naëmah 20.\,12.\,1898 Wien – 10.\,11.\,1971 New York City@\textsc{Beer-Hofmann, Naëmah} (20.\,12.\,1898 Wien – 10.\,11.\,1971 New York City)|pw}, und befindet{ }ſich wohl.\pend
           
\pstart
           Ich grüße Sie von Herzen als Ihr{\\[\baselineskip]}treuergebener
                  \spacefill\mbox{ArthSchnitzler}\pend
           \leftskip=0em{}
\pstart
           Wien\oindex{Wien@\textbf{Wien}, \emph{Verwaltungsgebiet}|pw}{ }8. 5. 99.\pend
           \selectlanguage{ngerman}\endnumbering\briefempfaengerindex{Brandes, Georg@\textsc{Brandes, Georg}!zzzSchnitzler, Arthur@\emph{von Arthur Schnitzler}!1899-05-081@{8. 5. 1899}|)be}\mylabel{L00915h}  \newcommand{\dateiname}{L00915}\newcommand{\titel}{Arthur Schnitzler an Georg Brandes, 8. 5. 1899}\newcommand{\editorInnen}{Martin Anton Müller und Gerd-Hermann Susen}%% latex-leseansicht-abspann.tex
%% Abspann für die Leseansicht.
%% Der Schalter \ifkorrekturansicht ist bereits durch den Vorspann gesetzt.

%% latex-abspann.tex
%% Gemeinsamer Abspann für Korrekturansicht und Leseansicht.
%% Setzt den Schalter \ifkorrekturansicht voraus (gesetzt in den
%% einbindenden Dateien latex-korrekturansicht-abspann.tex bzw.
%% latex-leseansicht-abspann.tex).
%% ---------------------------------------------------------------

\normalsize

% Das esempio-Environment wird nur in der Leseansicht benötigt
\ifkorrekturansicht\else
\newenvironment{esempio}[3]%
{
    \vspace{1.5ex}
    \rlap{\underline{#1}}
    \par
    \setlength{\parindent}{0cm}
    \nopagebreak
    \leftskip=#2cm
    \rightskip=#3cm
}
{
    \par
}
\fi

\doendnotes{C}
\bigskip
\vfill

\clearpage

\footnotesize

\ifkorrekturansicht
  \lohead{\textsc{register}}
\fi

% theindex-Environment neu definieren ohne reledmac
\makeatletter
\renewenvironment{theindex}{%
  \ifkorrekturansicht
    \section*{\indexname}%
  \else
    \subsubsection*{Index der erwähnten Entitäten}%
  \fi
  \setlength{\parindent}{0pt}%
  \setlength{\parskip}{0pt plus 0.3pt}%
  \let\item\@idxitem
}{%
  \ifkorrekturansicht\clearpage\fi
}
\makeatother

\IfFileExists{\jobname-pw.ind}{\input{\jobname-pw.ind}}{}

% Quellenangabe nur in der Leseansicht
\ifkorrekturansicht\else
% Fallback-Definitionen, falls die .tex-Datei \titel etc. nicht gesetzt hat
\providecommand{\titel}{}
\providecommand{\editorInnen}{}
\providecommand{\dateiname}{\jobname}

\vspace{3cm}

\vfill

\footnotesize
\textsc{Quelle}: \titel. Herausgegeben von {\editorInnen}. In: \emph{Arthur Schnitzler: Briefwechsel mit Autorinnen und Autoren}.
 Digitale Edition, https://schnitzler-briefe.acdh.oeaw.ac.at/{\dateiname}.html (Stand \today)
\fi

\end{document}


