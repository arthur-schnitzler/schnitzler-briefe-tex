%% latex-leseansicht-vorspann.tex
%% Vorspann für die Leseansicht.
%% Lädt die gemeinsame Datei latex-vorspann.tex mit nicht gesetztem Schalter.

\newif\ifkorrekturansicht
\korrekturansichtfalse

\input{../tex-inputs/latex-vorspann}


         
         \newcommand{\erwaehntePersonen}{Personen: Paul Goldmann, Fedor Mamroth, Andrea Mantegna}
         \newcommand{\erwaehnteOrte}{Orte: Frankgasse, Genua, IX., Alsergrund, Italien, Mailand, Pegli, Wien, Österreich}
         \newcommand{\erwaehnteWerke}{Werke: Madonna mit den Cherubin}
               \section[Richard Beer-Hofmann an Arthur Schnitzler, 15. 2. 1900]{ Richard Beer-Hofmann an Arthur Schnitzler, 15. 2. 1900}\nopagebreak\mylabel{v}\rehead{ }\begin{ledgroupsized}[t]{13cm}\normalsize\beginnumbering \toendnotes[C]{\smallbreak\pagebreak[2]} \Standort{CUL, Schnitzler, B 8.}
\physDesc{Bildpostkarte
\newline{}Handschrift: blauer Buntstift, lateinische Kurrent\newline{}Versand: 1) Stempel: »\nobreak{}\oindex{Pegli@\textbf{Pegli}|pwk}Pegli (Genova\oindex{Genua@\textbf{Genua}|pw}), 15 2 00\nobreak{}«.   2) Stempel: »\nobreak{}\oindex{IX., Alsergrund@\textbf{IX., Alsergrund}|pwk}Wien 9/3, 17. 2. 00, 8.V\nobreak{}«. \newline{}Ordnung: mit Bleistift von unbekannter Hand nummeriert: »149« }\buchAbdrucke{\weitereDrucke{Arthur Schnitzler, Richard Beer-Hofmann: \emph{Briefwechsel 1891–1931}. Hg. Konstanze Fliedl. Wien, Zürich: \emph{Europaverlag} 1992, S. 140.} }\pstart{}{\pb}\textcolor{gray}{\textbf{A}}n D\textsuperscript{r} Arthur
                  Schnitzler\pend{}\pstart{}Wien\oindex{Wien@\textbf{Wien}|pw}\pend{}\pstart{}IX Frankgasse 1\oindex{Frankgasse@\textbf{Frankgasse}|pw}\pend{}\pstart{}Austria\oindex{Oesterreich@\textbf{Österreich}|pw}\pend{}{\bigskip}\pstart
           \noindent{}\centering{}\textcolor{gray}{\textbf{{\pb}La Vergine col Figlio\pwindex{Mantegna, Andrea 1431 – 13.09.1506@\textsc{Mantegna, Andrea} (1431 – 13.09.1506), \emph{Maler, Künstler, Kupferstecher}!Madonna mit den Cherubin1485@\strich\emph{Madonna mit den Cherubin} {[}1485{]}|pw} (A. Mantegna\pwindex{Mantegna, Andrea 1431 – 13.09.1506@\textsc{Mantegna, Andrea} (1431 – 13.09.1506), \emph{Maler, Künstler, Kupferstecher}|pw})}}\pend
           \pstart
           \noindent{}{\pb}\textcolor{gray}{\textbf{Milano\oindex{Mailand@\textbf{Mailand}|pw}}}\hfill Pegli\oindex{Pegli@\textbf{Pegli}|pw}{ }15/II 1900\pend
           \pstart
           Lieber Arthur! Ich möchte wissen, I. Was Sie machen, II. Wie Paul\pwindex{Goldmann, Paul 31.01.1865 – 25.09.1935@\textsc{Goldmann, Paul} (31.01.1865 – 25.09.1935), \emph{Schriftsteller, Journalist}|pw}s Adresse ist. III Ob sein Onkel Fedor M.\pwindex{Mamroth, Fedor 21.02.1851 – 25.06.1907@\textsc{Mamroth, Fedor} (21.02.1851 – 25.06.1907), \emph{Journalist, Kritiker}|pw} heißt. IV Ob Sie nach Italien\oindex{Italien@\textbf{Italien}|pw} gehen. Ich grüße Sie herzlich\pend
           \pstart Ihr\spacefill\mbox{Richard.}\pend{}
         
         \endnumbering\mylabel{h}\end{ledgroupsized}  \newcommand{\dateiname}{L01013}\newcommand{\titel}{Richard Beer-Hofmann an Arthur Schnitzler, 15. 2. 1900}\newcommand{\editorInnen}{Martin Anton Müller und Gerd-Hermann Susen}%% latex-leseansicht-abspann.tex
%% Abspann für die Leseansicht.
%% Der Schalter \ifkorrekturansicht ist bereits durch den Vorspann gesetzt.

%% latex-abspann.tex
%% Gemeinsamer Abspann für Korrekturansicht und Leseansicht.
%% Setzt den Schalter \ifkorrekturansicht voraus (gesetzt in den
%% einbindenden Dateien latex-korrekturansicht-abspann.tex bzw.
%% latex-leseansicht-abspann.tex).
%% ---------------------------------------------------------------

\normalsize

% Das esempio-Environment wird nur in der Leseansicht benötigt
\ifkorrekturansicht\else
\newenvironment{esempio}[3]%
{
    \vspace{1.5ex}
    \rlap{\underline{#1}}
    \par
    \setlength{\parindent}{0cm}
    \nopagebreak
    \leftskip=#2cm
    \rightskip=#3cm
}
{
    \par
}
\fi

\doendnotes{C}
\bigskip
\vfill

\clearpage

\footnotesize

\ifkorrekturansicht
  \lohead{\textsc{register}}
\fi

% theindex-Environment neu definieren ohne reledmac
\makeatletter
\renewenvironment{theindex}{%
  \ifkorrekturansicht
    \section*{\indexname}%
  \else
    \subsubsection*{Index der erwähnten Entitäten}%
  \fi
  \setlength{\parindent}{0pt}%
  \setlength{\parskip}{0pt plus 0.3pt}%
  \let\item\@idxitem
}{%
  \ifkorrekturansicht\clearpage\fi
}
\makeatother

\IfFileExists{\jobname-pw.ind}{\input{\jobname-pw.ind}}{}

% Quellenangabe nur in der Leseansicht
\ifkorrekturansicht\else
% Fallback-Definitionen, falls die .tex-Datei \titel etc. nicht gesetzt hat
\providecommand{\titel}{}
\providecommand{\editorInnen}{}
\providecommand{\dateiname}{\jobname}

\vspace{3cm}

\vfill

\footnotesize
\textsc{Quelle}: \titel. Herausgegeben von {\editorInnen}. In: \emph{Arthur Schnitzler: Briefwechsel mit Autorinnen und Autoren}.
 Digitale Edition, https://schnitzler-briefe.acdh.oeaw.ac.at/{\dateiname}.html (Stand \today)
\fi

\end{document}


      