%% latex-korrekturansicht-vorspann.tex
%% Vorspann für die Korrekturansicht.
%% Lädt die gemeinsame Datei latex-vorspann.tex mit gesetztem Schalter.

\newif\ifkorrekturansicht
\korrekturansichttrue

\input{../tex-inputs/latex-vorspann}


\section[Richard Beer-Hofmann an Arthur Schnitzler, 15. 2. 1900]{L01013 Richard Beer-Hofmann an Arthur Schnitzler, 15. 2. 1900}
\nopagebreak\mylabel{L01013v}
\rehead{ }\normalsize\beginnumbering\briefempfaengerindex{Schnitzler, Arthur@\textsc{Schnitzler, Arthur}!zzzBeer-Hofmann, Richard@\emph{von Richard Beer-Hofmann}!1900-02-151@{15. 2. 1900}|(be}
\toendnotes[C]{\smallbreak\pagebreak[2]}\Standort{CUL, Schnitzler, B 8.}
\physDesc{Bildpostkarte, 234 Zeichen
\newline{}Handschrift: blauer Buntstift, lateinische Kurrent
\newline{}Versand: 1) Stempel: »\nobreak{}\oindex{Pegli@\textbf{Pegli}, \emph{P.PPLX}|pwk}Pegli (Genova), 15 2 00\nobreak{}«.   2) Stempel: »\nobreak{}\oindex{IX., Alsergrund@\textbf{IX., Alsergrund}, \emph{A.ADM3}|pwk}Wien 9/3, 17. 2. 00, 8.V\nobreak{}«. 
\newline{}Ordnung: mit Bleistift von unbekannter Hand nummeriert: »149« }
\buchAbdrucke{\weitereDrucke{Arthur Schnitzler, Richard Beer-Hofmann: \emph{Briefwechsel 1891–1931}. Wien, Zürich: \emph{Europaverlag} 1992, S. 140.} }\pstart{}{\pb}\textcolor{gray}{\textbf{A}}n D\textsuperscript{r} Arthur
                  Schnitzler\pend{}\pstart{}Wien\oindex{Wien@\textbf{Wien}, \emph{A.ADM2}|pw}\pend{}\pstart{}IX Frankgasse 1\oindex{Frankgasse 1@\textbf{Frankgasse 1}, \emph{Wohngebäude (K.WHS)}|pw}\pend{}\pstart{}Austria\oindex{Oesterreich@\textbf{Österreich}, \emph{A.PCLI}|pw}\pend{}{\bigskip}
\pstart
           \noindent{}\centering{}{\pb}\textcolor{gray}{\textbf{La Vergine col Figlio\pwindex{Madonna mit den Cherubin@\emph{Madonna mit den Cherubin}|pw} (A. Mantegna\pwindex{Mantegna, Andrea 1431 – 13.09.1506@\textsc{Mantegna, Andrea} (1431 – 13.09.1506), \emph{Maler/Malerin, Künstler/Künstlerin, Kupferstecher/Kupferstecherin}|pw})}}\pend
           
\pstart
           \textcolor{gray}{\textbf{Milano\oindex{Mailand@\textbf{Mailand}, \emph{P.PPLA}|pw}}}\pend
           \vspace{1em}
\pstart
           \raggedleft{}{\pb}Pegli\oindex{Pegli@\textbf{Pegli}, \emph{P.PPLX}|pw}{ }15/II 1900\pend
           \vspace{0.5em}
\pstart
           Lieber Arthur! Ich möchte wissen, I. Was Sie machen, II. Wie Pauls\pwindex{Goldmann, Paul 31.01.1865 – 25.09.1935@\textsc{Goldmann, Paul} (31.01.1865 – 25.09.1935), \emph{Schriftsteller/Schriftstellerin, Journalist/Journalistin}|pw} Adresse ist. III Ob sein Onkel Fedor M.\pwindex{Mamroth, Fedor 21.02.1851 – 25.06.1907@\textsc{Mamroth, Fedor} (21.02.1851 – 25.06.1907), \emph{Journalist/Journalistin, Kritiker/Kritikerin}|pw} heißt. IV Ob Sie nach Italien\oindex{Italien@\textbf{Italien}, \emph{A.PCLI}|pw} gehen. Ich grüße Sie herzlich\pend
           \pstart Ihr\spacefill\mbox{Richard.}\pend{}\selectlanguage{ngerman}\endnumbering\briefempfaengerindex{Schnitzler, Arthur@\textsc{Schnitzler, Arthur}!zzzBeer-Hofmann, Richard@\emph{von Richard Beer-Hofmann}!1900-02-151@{15. 2. 1900}|)be}\mylabel{L01013h}  \normalsize

\doendnotes{C}
\bigskip
\vfill

\clearpage

\footnotesize

\lohead{\textsc{register}}

% Definiere theindex-Environment komplett neu ohne reledmac
\makeatletter
\renewenvironment{theindex}{%
  \section*{\indexname}%
  \setlength{\parindent}{0pt}%
  \setlength{\parskip}{0pt plus 0.3pt}%
  \let\item\@idxitem
}{%
  \clearpage
}
\makeatother

\IfFileExists{\jobname-pw.ind}{\input{\jobname-pw.ind}}{}

\end{document}

      