%% latex-korrekturansicht-vorspann.tex
%% Vorspann für die Korrekturansicht.
%% Lädt die gemeinsame Datei latex-vorspann.tex mit gesetztem Schalter.

\newif\ifkorrekturansicht
\korrekturansichttrue

\input{../tex-inputs/latex-vorspann}


\section[Elsa Plessner an Arthur Schnitzler, 15. 11. 1915]{L03729 Elsa Plessner an Arthur Schnitzler, 15. 11. 1915}
\nopagebreak\mylabel{L03729v}
\rehead{ }\normalsize\beginnumbering\briefempfaengerindex{Schnitzler, Arthur@\textsc{Schnitzler, Arthur}!zzzPlessner, Elsa@\emph{von Elsa Plessner}!1915-11-151@{15. 11. 1915}|(be}
\toendnotes[C]{\smallbreak\pagebreak[2]}\Standort{DLA, A:Schnitzler, HS.1985.1.419.}
\physDesc{Brief,  Blätter, 4 Seiten, 2717 Zeichen
\newline{}Handschrift: , lateinische Kurrent
\newline{}Schnitzler: 1) sechs Unterstreichungen  2) beschriftet: »Plessner«}\toendnotes[C]{\smallbreak}
\pstart
           {\pb}\textcolor{gray}{\textbf{PENSION}}\hfill \textcolor{gray}{\textbf{Telephon 51151}}\pend
           
\pstart
           \textcolor{gray}{\textbf{\uuline{SERNO\oindex{Pension Serno@\textbf{Pension Serno}, \emph{Beherbergungsgebäude (K.BHB)}|pw}}}}\pend
           
\pstart
           \textcolor{gray}{\textbf{MÜNCHEN}}\oindex{Muenchen@\textbf{München}, \emph{P.PPLA}|pw}\hfill \textcolor{gray}{\textbf{Den}}{ }15. XI. \textcolor{gray}{\textbf{191}}5\pend
           
\pstart
           \textcolor{gray}{\textbf{Theresienstraße}}\oindex{Pension Serno@\textbf{Pension Serno}, \emph{Beherbergungsgebäude (K.BHB)}|pw}\pend
           
\pstart
           \textcolor{gray}{\textbf{78 I. und II. Stock}}\oindex{Pension Serno@\textbf{Pension Serno}, \emph{Beherbergungsgebäude (K.BHB)}|pw}\pend
           
\pstart{}Verehrter Herr Doctor!\pend\vspace{0.5em}
\pstart
           Man sagt mir, dass Sie in den nächsten Wochen hierher nach München\oindex{Muenchen@\textbf{München}, \emph{P.PPLA}|pw} kommen werden, \label{K_L03729-1v}\edtext{zur Aufführung}{\lemma{\textnormal{\emph{zur Aufführung}}}\Cendnote{\textnormal{Die \emph{Theaterpremiere von \emph{Der einsame Weg}\pwindex{einsame Weg. Schauspiel in fuenf Akten@\emph{Der einsame Weg. Schauspiel in fünf Akten}|pwk}}\eventindex{Kammerspiele Muenchen@\textbf{Kammerspiele München}!Premiere von Der Einsame Weg, 27.11.1915@Premiere von Der Einsame Weg, 27.11.1915|pwk} von Arthur Schnitzler fand am 27. 11. 1915 in den \emph{Münchner
                     Kammerspielen}\orgindex{Muenchner Kammerspiele@Münchner Kammerspiele|pwk} in München\oindex{Muenchen@\textbf{München}, \emph{P.PPLA}|pwk} statt, vgl. Richard Elchinger\pwindex{Elchinger, Richard 1883-05-02 – 1955-10-21@\textsc{Elchinger, Richard} (1883-05-02 – 1955-10-21), \emph{Theaterkritiker/Theaterkritikerin}|pwk}: \emph{Der einsame Weg. Schauspiel von Artur Schnitzler. Erste
                        Aufführung in den Kammerspielen am 27. November}\pwindex{einsame Weg. Schauspiel von Artur Schnitzler. Erste Auffuehrung
                  in den Kammerspielen am 27. November@\emph{Der einsame Weg. Schauspiel von Artur Schnitzler. Erste Aufführung in den Kammerspielen am 27. November}|pwk}. In: \emph{Münchner neueste Nachrichten}\pwindex{Muenchner Neueste Nachrichten@\emph{Münchner Neueste Nachrichten}|pwk}, Jg. 68, Nr. 610, S. 
                     2. Schnitzler reiste nicht dazu
                  an. }}}\label{K_L03729-1} des »einsamen Weg\pwindex{einsame Weg. Schauspiel in fuenf Akten@\emph{Der einsame Weg. Schauspiel in fünf Akten}|pw}« in den Kammerspielen\orgindex{Muenchner Kammerspiele@Münchner Kammerspiele|pw}. Dadurch sehe ich mich in die
               Nothwendigkeit versetzt, nach geraumer Zeit wieder einmal ein Schreiben an Sie,
               verehrter Herr Doctor, zu richten, – warum, werden Sie sofort einsehen.\pend
           
\pstart
           Mit der vor Kurzem erfolgten, endlichen \label{K_L03729-2v}\edtext{Auflösung meiner Ehe}{\lemma{\textnormal{\emph{Auflösung meiner Ehe}}}\Cendnote{\textnormal{Elsa Plessner war
                  seit dem 22. 4. 1903 mit Wilhelm
                     Ginsberg\pwindex{Ginsberg, Wilhelm 1880-06-06 – 1960-05-13@\textsc{Ginsberg, Wilhelm} (1880-06-06 – 1960-05-13), \emph{Kaufmann/Kauffrau, Mediziner/Medizinerin}|pwk} verheiratet gewesen, vgl. \emph{Theaterzeitung}. In: \emph{Illustrirtes Wiener Extrablatt}\pwindex{Illustrirtes Wiener Extrablatt@\emph{Illustrirtes Wiener Extrablatt}|pwk} (Abendausgabe), 32. Jg., Nr. 109,
                     22. 4. 1903, S. 3.}}}\label{K_L03729-2} ist für mich jeder innere und äußere Grund
               fortgefallen, der mich verhindern {\pb}konnte, wieder als
               Bühnenschriftstellerin in der Öffentlichkeit zu erscheinen. Daher habe ich auf
               Anrathen eines kleinen, wie mir scheint, recht urtheilsfähigen Freundeskreises, ein
               dreiactiges Schauspiel »das erste Capitel\pwindex{erste Kapitel. Schauspiel in drei Akten@\emph{Das erste Kapitel. Schauspiel in drei Akten}|pw}« den
                  Münchner Kammerspielen\orgindex{Muenchner Kammerspiele@Münchner Kammerspiele|pw} eingereicht. Diese Arbeit\pwindex{erste Kapitel. Schauspiel in drei Akten@\emph{Das erste Kapitel. Schauspiel in drei Akten}|pwv}, noch aus meiner
               Mädchenzeit stammend, ist fast die einzige meiner literarischen Jugendsünden, die
               begangen zu haben ich nicht bereue, und die meinem, seit vierzehn Jahren einigermaßen
               gereiften Urtheil heute noch wertvoll erscheint. Sie selbst, verehrter Herr Doctor,
               haben sie, wie alle meine Arbeiten \label{K_L03729-3v}\edtext{unmittelbar nach der Entstehung}{\lemma{\textnormal{\emph{unmittelbar … Entstehung}}}\Cendnote{\textnormal{Plessner\pwindex{Plessner, Elsa 22.08.1875 – 01.05.1932@\textsc{Plessner, Elsa} (22.08.1875 – 01.05.1932), \emph{Schriftsteller/Schriftstellerin}|pwk} schickte Schnitzler das Schauspiel\pwindex{erste Kapitel. Schauspiel in drei Akten@\emph{Das erste Kapitel. Schauspiel in drei Akten}|pwkv} mit ihrem Brief vom 9. 1. 1900.}}}\label{K_L03729-3} gelesen,
               und in einem, in meinem Besitz befindlichen \label{K_L03729-4v}\edtext{Briefe an mich}{\lemma{\textnormal{\emph{Briefe an mich}}}\Cendnote{\textnormal{nicht überliefert}}}\label{K_L03729-4}{ }{\pb}zu meinem größten Stolz als »unendlich fein« gelobt. – Das erste Cap.\pwindex{erste Kapitel. Schauspiel in drei Akten@\emph{Das erste Kapitel. Schauspiel in drei Akten}|pw} ist nun in einer leichten Überarbeitung – die
               nichts geschädigt hat, was an dem Stück\pwindex{erste Kapitel. Schauspiel in drei Akten@\emph{Das erste Kapitel. Schauspiel in drei Akten}|pwv} lobenswert war – Herrn Dir. Ziegel\pwindex{Ziegel, Erich 1876-08-26 – 1950-11-30@\textsc{Ziegel, Erich} (1876-08-26 – 1950-11-30), \emph{Theaterleiter/Theaterleiterin, Regisseur/Regisseurin, Schauspieler/Schauspielerin}|pw} zugesandt worden und ich bitte Sie \uline{nicht} um Ihre Fürsprache sondern ich fühle mich verpflichtet, Ihnen
               mitzutheilen, dass ich in dem \label{K_L03729-5v}\edtext{Begleitbrief}{\lemma{\textnormal{\emph{Begleitbrief}}}\Cendnote{\textnormal{nicht
                  überliefert}}}\label{K_L03729-5} an Herrn Ziegel\pwindex{Ziegel, Erich 1876-08-26 – 1950-11-30@\textsc{Ziegel, Erich} (1876-08-26 – 1950-11-30), \emph{Theaterleiter/Theaterleiterin, Regisseur/Regisseurin, Schauspieler/Schauspielerin}|pw}
               folgenden Paßus schrieb: »Arthur Schnitzler,
               der die Arbeit\pwindex{erste Kapitel. Schauspiel in drei Akten@\emph{Das erste Kapitel. Schauspiel in drei Akten}|pwv} in einer
               früheren Form kannte, bezeichnete sie mir als, unendlich fein«. –\pend
           
\pstart
           Ich musste – verschollen, wie ich als Schriftstellerin bin –, einen Eideshelfer von
               Gewicht zu Hilfe rufen, damit man über das literarische Niveau des unbekannten
               Einsenders einigermaßen im Klaren sei. –\pend
           
\pstart
           Es ist daher sehr leicht möglich, dass Herr Dir. Ziegel\pwindex{Ziegel, Erich 1876-08-26 – 1950-11-30@\textsc{Ziegel, Erich} (1876-08-26 – 1950-11-30), \emph{Theaterleiter/Theaterleiterin, Regisseur/Regisseurin, Schauspieler/Schauspielerin}|pw} sich an Sie, verehrter Herr Doctor, mit einer dies{\pb}bezüglichen Frage wendet, wenn Sie hier sind. Um Ihnen nun die Verlegenheit zu
               ersparen, wenn Sie sich, wie leicht denkbar, nicht mehr an das »erste Cap.\pwindex{erste Kapitel. Schauspiel in drei Akten@\emph{Das erste Kapitel. Schauspiel in drei Akten}|pw}« und Ihr damaliges Urtheil erinnern, eine
               Verlegenheit, aus der für mich eine peinliche folgenschwere Blamage entstehen könnte,
               erlaube ich mir, diesen Brief an Sie.\pend
           
\pstart
           Ihr damaliges Urtheil war für mich von entscheidender Bedeutung, was nicht hindert,
               dass Autor u. Stück\pwindex{erste Kapitel. Schauspiel in drei Akten@\emph{Das erste Kapitel. Schauspiel in drei Akten}|pwv} Ihrem
               Gedächtins gänzlich entschwunden sein könnten.\pend
           
\pstart
           Ich hoffe, Sie sind nicht böse, dass ich mich ohne Ihr Vorwissen unter Ihren Schutz
               stellte, und dass ich Sie hiermit vielmals bitte, mich gegebenenfalls nicht zu
               desavouiren.\pend
           
\pstart
           Indem ich Sie bitte, Ihrer Frau Gemahlin\pwindex{Schnitzler, Olga 17.01.1882 – 13.01.1970@\textsc{Schnitzler, Olga} (17.01.1882 – 13.01.1970), \emph{Schauspieler/Schauspielerin, Sänger/Sängerin}|pwv} meinen verbindlichsten Gruß zu übermitteln mit
               vorzüglicher Hochachtung{\\[\baselineskip]}\spacefill\mbox{Elsa Ginsberg-Plessner}\pend
           \leftskip=0em{}\selectlanguage{ngerman}\endnumbering\briefempfaengerindex{Schnitzler, Arthur@\textsc{Schnitzler, Arthur}!zzzPlessner, Elsa@\emph{von Elsa Plessner}!1915-11-151@{15. 11. 1915}|)be}\mylabel{L03729h}
\begin{anhang}
\end{anhang}\normalsize

\doendnotes{C}
\bigskip
\vfill

\clearpage

\footnotesize

\lohead{\textsc{register}}

% Definiere theindex-Environment komplett neu ohne reledmac
\makeatletter
\renewenvironment{theindex}{%
  \section*{\indexname}%
  \setlength{\parindent}{0pt}%
  \setlength{\parskip}{0pt plus 0.3pt}%
  \let\item\@idxitem
}{%
  \clearpage
}
\makeatother

\IfFileExists{\jobname-pw.ind}{\input{\jobname-pw.ind}}{}

\end{document}

      