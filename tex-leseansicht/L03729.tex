%% latex-leseansicht-vorspann.tex
%% Vorspann für die Leseansicht.
%% Lädt die gemeinsame Datei latex-vorspann.tex mit nicht gesetztem Schalter.

\newif\ifkorrekturansicht
\korrekturansichtfalse

\input{../tex-inputs/latex-vorspann}


\section[Elsa Ginsberg-Plessner an Arthur Schnitzler, 15. 11. 1915]{L03729 Elsa Ginsberg-Plessner an Arthur Schnitzler, 15. 11. 1915}
\nopagebreak\mylabel{L03729v}
\rehead{ }\normalsize\beginnumbering\briefempfaengerindex{Schnitzler, Arthur@\textsc{Schnitzler, Arthur}!zzzPlessner, Elsa@\emph{von Elsa Plessner}!1915-11-151@{15. 11. 1915}|(be}
\toendnotes[C]{\smallbreak\pagebreak[2]}
\correspDesc{Versand  durch Elsa Plessner am 15. 11. 1915 in München
\newline{}Erhalt  durch Arthur Schnitzler im Zeitraum [16. 11. 1915 – 20. 11. 1915?] in Wien}\toendnotes[C]{\smallbreak}
\Standort{DLA, A:Schnitzler, HS.1985.1.419.}
\physDesc{Brief, 1 Blatt, 4 Seiten, 2721 Zeichen
\newline{}Handschrift: schwarze Tinte, lateinische Kurrent
\newline{}Schnitzler: 1) mit roter Tinte sechs Unterstreichungen  2) mit Bleistift beschriftet: »Plessner«}\toendnotes[C]{\smallbreak}
\pstart
           {\pb}\textcolor{gray}{\textbf{PENSION}}\hfill \textcolor{gray}{\textbf{Telephon 51151}}\pend
           
\pstart
           \textcolor{gray}{\textbf{\uuline{SERNO\oindex{Pension Serno@\textbf{Pension Serno}, \emph{Beherbergungsgebäude}|pw}}}}\pend
           
\pstart
           \textcolor{gray}{\textbf{MÜNCHEN}}\oindex{München@\textbf{München}|pw}\hfill \textcolor{gray}{\textbf{DEN}}{ }15. XI. \textcolor{gray}{\textbf{191}}5\pend
           
\pstart
           \textcolor{gray}{\textbf{Theresienstraße}}\oindex{Theresienstraße 78@\textbf{Theresienstraße 78}, \emph{Wohngebäude}|pw}\pend
           
\pstart
           \textcolor{gray}{\textbf{78 I. und II. Stock}}\oindex{Pension Serno@\textbf{Pension Serno}, \emph{Beherbergungsgebäude}|pw}\pend
           
\pstart{}Verehrter Herr Doctor!\pend\vspace{0.5em}
\pstart
           Man sagt mir, dass Sie in den nächsten Wochen hierher nach München\oindex{München@\textbf{München}|pw} kommen werden, zur \label{K_L03729-1v}\edtext{Aufführung\eventindex{Kammerspiele München@\textbf{Kammerspiele München}!Premiere von Der Einsame Weg, 27.11.1915@Premiere von Der Einsame Weg, 27.11.1915|pwv} des »einsamen Weg\pwindex{Schnitzler, Arthur 15. 5. 1862 Wien – 21. 10. 1931 ebd.@\textsc{Schnitzler, Arthur} (15. 5. 1862 Wien – 21. 10. 1931 ebd.), \emph{Schriftsteller, Mediziner}!einsame Weg. Schauspiel in fünf Akten@\strich\emph{Der einsame Weg. Schauspiel in fünf Akten}|pw}« in den
                  Kammerspielen\orgindex{Münchner Kammerspiele@Münchner Kammerspiele|pw}}{\lemma{\textnormal{\emph{Aufführung … Kammerspielen}}}\Cendnote{\textnormal{Die Premiere von \emph{Der einsame Weg}\pwindex{Schnitzler, Arthur 15. 5. 1862 Wien – 21. 10. 1931 ebd.@\textsc{Schnitzler, Arthur} (15. 5. 1862 Wien – 21. 10. 1931 ebd.), \emph{Schriftsteller, Mediziner}!einsame Weg. Schauspiel in fünf Akten@\strich\emph{Der einsame Weg. Schauspiel in fünf Akten}|pwk}\eventindex{Kammerspiele München@\textbf{Kammerspiele München}!Premiere von Der Einsame Weg, 27.11.1915@Premiere von Der Einsame Weg, 27.11.1915|pwk} von Arthur Schnitzler fand am 27. 11. 1915 an den \emph{Münchner Kammerspielen}\orgindex{Münchner Kammerspiele@Münchner Kammerspiele|pwk} in München\oindex{München@\textbf{München}|pwk}
                  statt. Vgl. Richard Elchinger\pwindex{Elchinger, Richard 2.\,5.\,1883 – 21.\,10.\,1955@\textsc{Elchinger, Richard} (2.\,5.\,1883 – 21.\,10.\,1955), \emph{Theaterkritiker}|pwk}: \emph{Der einsame Weg. Schauspiel von Artur
                        Schnitzler. Erste Aufführung in den Kammerspielen am 27. November}\pwindex{Elchinger, Richard 2.\,5.\,1883 – 21.\,10.\,1955@\textsc{Elchinger, Richard} (2.\,5.\,1883 – 21.\,10.\,1955), \emph{Theaterkritiker}!einsame Weg. Schauspiel von Artur Schnitzler. Erste Aufführung in den Kammerspielen am 27. November@\strich\emph{Der einsame Weg. Schauspiel von Artur Schnitzler. Erste Aufführung in den Kammerspielen am 27. November}|pwk}. In:
                        \emph{Münchner neueste Nachrichten}\pwindex{Münchner Neueste Nachrichten@\emph{Münchner Neueste Nachrichten}|pwk}, Jg. 68,
                     Nr. 610, S. 2. Schnitzler reiste
                  nicht an.}}}\label{K_L03729-1}. Dadurch sehe ich mich in die Nothwendigkeit versetzt, nach
               geraumer Zeit wieder einmal ein Schreiben an Sie, verehrter Herr Doctor, zu richten –
               warum, werden Sie sofort einsehen.\pend
           
\pstart
           Mit der vor Kurzem erfolgten, endlichen \label{K_L03729-2v}\edtext{Auflösung meiner Ehe\pwindex{Ginsberg, Wilhelm 6.\,6.\,1880 Berlin – 13.\,5.\,1960 Gent@\textsc{Ginsberg, Wilhelm} (6.\,6.\,1880 Berlin – 13.\,5.\,1960 Gent), \emph{Kaufmann, Mediziner}|pwv}}{\lemma{\textnormal{\emph{Auflösung meiner Ehe}}}\Cendnote{\textnormal{Die Hochzeit zwischen ihr und 
                  Wilhelm
                     Ginsberg\pwindex{Ginsberg, Wilhelm 6.\,6.\,1880 Berlin – 13.\,5.\,1960 Gent@\textsc{Ginsberg, Wilhelm} (6.\,6.\,1880 Berlin – 13.\,5.\,1960 Gent), \emph{Kaufmann, Mediziner}|pwk}
                  hatte am 22. 4. 1903 in Wien\oindex{Wien@\textbf{Wien}, \emph{Verwaltungsgebiet}|pwk} stattgefunden (vgl.
                     \emph{Theaterzeitung}. In: \emph{Illustrirtes Wiener Extrablatt}\pwindex{Illustrirtes Wiener Extrablatt@\emph{Illustrirtes Wiener Extrablatt}|pwk} (Abendausgabe), 32. Jg., Nr. 109,
                     22. 4. 1903, S. 3). Sie bekamen einen Sohn\pwindex{\textcolor{red}{\textsuperscript{XXXX indx1}}|pwkv}
                  und der Ehemann schloss ein Medizinstudium ab. Näheres zur Auflösung der Ehe ist nicht bekannt.}}}\label{K_L03729-2} ist für mich jeder innere und äußere Grund
               fortgefallen, der mich verhindern {\pb}konnte, wieder als
               Bühnenschriftstellerin in der Öffentlichkeit zu erscheinen. Daher habe ich auf
               Anrathen eines kleinen, wie mir scheint, recht urtheilsfähigen Freundeskreises, ein
               dreiactiges Schauspiel »das erste Capitel\pwindex{Plessner, Elsa 22.\,8.\,1875 Wien – 7.\,5.\,1932 Alicante@\textsc{Plessner, Elsa} (22.\,8.\,1875 Wien – 7.\,5.\,1932 Alicante), \emph{Schriftstellerin}!erste Kapitel. Schauspiel in drei Akten@\strich\emph{Das erste Kapitel. Schauspiel in drei Akten}|pw}« den
                  Münchner Kammerspielen\orgindex{Münchner Kammerspiele@Münchner Kammerspiele|pw} eingereicht. Diese Arbeit\pwindex{Plessner, Elsa 22.\,8.\,1875 Wien – 7.\,5.\,1932 Alicante@\textsc{Plessner, Elsa} (22.\,8.\,1875 Wien – 7.\,5.\,1932 Alicante), \emph{Schriftstellerin}!erste Kapitel. Schauspiel in drei Akten@\strich\emph{Das erste Kapitel. Schauspiel in drei Akten}|pwv}, noch aus meiner
               Mädchenzeit stammend, ist fast die einzige meiner literarischen Jugendsünden, die
               begangen zu haben ich nicht bereue, und die meinem, seit vierzehn Jahren einigermaßen
               gereiften Urtheil heute noch wertvoll erscheint. Sie selbst, verehrter Herr Doctor,
               haben sie, wie alle meine Arbeiten \label{K_L03729-3v}\edtext{unmittelbar nach der Entstehung}{\lemma{\textnormal{\emph{unmittelbar … Entstehung}}}\Cendnote{\textnormal{Plessner\pwindex{Plessner, Elsa 22.\,8.\,1875 Wien – 7.\,5.\,1932 Alicante@\textsc{Plessner, Elsa} (22.\,8.\,1875 Wien – 7.\,5.\,1932 Alicante), \emph{Schriftstellerin}|pwk} schickte Schnitzler das Schauspiel\pwindex{Plessner, Elsa 22.\,8.\,1875 Wien – 7.\,5.\,1932 Alicante@\textsc{Plessner, Elsa} (22.\,8.\,1875 Wien – 7.\,5.\,1932 Alicante), \emph{Schriftstellerin}!erste Kapitel. Schauspiel in drei Akten@\strich\emph{Das erste Kapitel. Schauspiel in drei Akten}|pwkv} mit ihrem Brief vom XXXX Auszeichnungsfehler: Dokument L03723 nicht gefunden.}}}\label{K_L03729-3} gelesen,
               und in einem, in meinem Besitz befindlichen \label{K_L03729-4v}\edtext{Briefe an mich}{\lemma{\textnormal{\emph{Briefe an mich}}}\Cendnote{\textnormal{nicht überliefert}}}\label{K_L03729-4}{ }{\pb}zu meinem größten Stolz als »unendlich fein« gelobt. –
                  Das erste Cap.\pwindex{Plessner, Elsa 22.\,8.\,1875 Wien – 7.\,5.\,1932 Alicante@\textsc{Plessner, Elsa} (22.\,8.\,1875 Wien – 7.\,5.\,1932 Alicante), \emph{Schriftstellerin}!erste Kapitel. Schauspiel in drei Akten@\strich\emph{Das erste Kapitel. Schauspiel in drei Akten}|pw} ist nun in einer leichten
               Überarbeitung – die nichts geschädigt hat, was an dem Stück\pwindex{Plessner, Elsa 22.\,8.\,1875 Wien – 7.\,5.\,1932 Alicante@\textsc{Plessner, Elsa} (22.\,8.\,1875 Wien – 7.\,5.\,1932 Alicante), \emph{Schriftstellerin}!erste Kapitel. Schauspiel in drei Akten@\strich\emph{Das erste Kapitel. Schauspiel in drei Akten}|pwv} lobenswert war – Herrn Dir. Ziegel\pwindex{Ziegel, Erich 26.\,8.\,1876 Schwerin – 30.\,11.\,1950 München@\textsc{Ziegel, Erich} (26.\,8.\,1876 Schwerin – 30.\,11.\,1950 München), \emph{Theaterleiter, Regisseur, Schauspieler}|pw} zugesandt worden und ich bitte Sie \uline{nicht} um Ihre Fürsprache, sondern ich fühle mich
               verpflichtet, Ihnen mitzutheilen, dass ich in dem \label{K_L03729-5v}\edtext{Begleitbrief}{\lemma{\textnormal{\emph{Begleitbrief}}}\Cendnote{\textnormal{nicht überliefert}}}\label{K_L03729-5} an Herrn Ziegel\pwindex{Ziegel, Erich 26.\,8.\,1876 Schwerin – 30.\,11.\,1950 München@\textsc{Ziegel, Erich} (26.\,8.\,1876 Schwerin – 30.\,11.\,1950 München), \emph{Theaterleiter, Regisseur, Schauspieler}|pw}
               folgenden Passus schrieb: »Arthur Schnitzler,
               der die Arbeit\pwindex{Plessner, Elsa 22.\,8.\,1875 Wien – 7.\,5.\,1932 Alicante@\textsc{Plessner, Elsa} (22.\,8.\,1875 Wien – 7.\,5.\,1932 Alicante), \emph{Schriftstellerin}!erste Kapitel. Schauspiel in drei Akten@\strich\emph{Das erste Kapitel. Schauspiel in drei Akten}|pwv} in einer
               früheren Form kannte, bezeichnete sie mir als ›unendlich fein{[}‹{]}«. – –\pend
           
\pstart
           Ich musste – verschollen, wie ich als Schriftstellerin bin –, einen Eideshelfer von
               Gewicht zu Hilfe rufen, damit man über das literarische Niveau des unbekannten
               Einsenders einigermaßen im Klaren sei. –\pend
           
\pstart
           Es ist daher sehr leicht möglich, dass Herr Dir. Ziegel\pwindex{Ziegel, Erich 26.\,8.\,1876 Schwerin – 30.\,11.\,1950 München@\textsc{Ziegel, Erich} (26.\,8.\,1876 Schwerin – 30.\,11.\,1950 München), \emph{Theaterleiter, Regisseur, Schauspieler}|pw} sich an Sie, verehrter Herr Doctor, mit einer dies{\pb}bezüglichen Frage wendet, wenn Sie hier sind. Um Ihnen
               nun die Verlegenheit zu ersparen, wenn Sie sich, wie leicht denkbar, nicht mehr an
               das »erste Cap.\pwindex{Plessner, Elsa 22.\,8.\,1875 Wien – 7.\,5.\,1932 Alicante@\textsc{Plessner, Elsa} (22.\,8.\,1875 Wien – 7.\,5.\,1932 Alicante), \emph{Schriftstellerin}!erste Kapitel. Schauspiel in drei Akten@\strich\emph{Das erste Kapitel. Schauspiel in drei Akten}|pw}« und Ihr damaliges Urtheil
               erinnern, eine Verlegenheit, aus der für mich eine peinliche, folgenschwere Blamage
               entstehen könnte, erlaube ich mir diesen Brief an Sie.\pend
           
\pstart
           Ihr damaliges Urtheil war für mich von entscheidender Bedeutung, was nicht hindert,
               dass Autor u. Stück\pwindex{Plessner, Elsa 22.\,8.\,1875 Wien – 7.\,5.\,1932 Alicante@\textsc{Plessner, Elsa} (22.\,8.\,1875 Wien – 7.\,5.\,1932 Alicante), \emph{Schriftstellerin}!erste Kapitel. Schauspiel in drei Akten@\strich\emph{Das erste Kapitel. Schauspiel in drei Akten}|pwv} Ihrem
               Gedächtnis gänzlich entschwunden sein könnten.\pend
           
\pstart
           Ich hoffe, Sie sind nicht böse, dass ich mich ohne Ihr Vorwissen unter Ihren Schutz
               stellte, und dass ich Sie hiermit vielmals bitte, mich gegebenenfalls nicht zu
               desavouiren.\pend
           
\pstart
           Indem ich Sie bitte, Ihrer Frau Gemahlin\pwindex{Schnitzler, Olga 17.\,1.\,1882 Wien – 13.\,1.\,1970 Lugano@\textsc{Schnitzler, Olga} (17.\,1.\,1882 Wien – 13.\,1.\,1970 Lugano), \emph{Schauspielerin, Sängerin}|pwv} meinen verbindlichsten Gruß zu übermitteln{\\[\baselineskip]}mit
               vorzüglicher Hochachtung{\\[\baselineskip]}\spacefill\mbox{Elsa Ginsberg-Plessner.}\pend
           \leftskip=0em{}\selectlanguage{ngerman}\endnumbering\briefempfaengerindex{Schnitzler, Arthur@\textsc{Schnitzler, Arthur}!zzzPlessner, Elsa@\emph{von Elsa Plessner}!1915-11-151@{15. 11. 1915}|)be}\mylabel{L03729h}  \newcommand{\dateiname}{L03729}\newcommand{\titel}{Elsa Ginsberg-Plessner an Arthur Schnitzler, 15. 11. 1915}\newcommand{\editorInnen}{Selma Jahnke und Martin Anton Müller}%% latex-leseansicht-abspann.tex
%% Abspann für die Leseansicht.
%% Der Schalter \ifkorrekturansicht ist bereits durch den Vorspann gesetzt.

%% latex-abspann.tex
%% Gemeinsamer Abspann für Korrekturansicht und Leseansicht.
%% Setzt den Schalter \ifkorrekturansicht voraus (gesetzt in den
%% einbindenden Dateien latex-korrekturansicht-abspann.tex bzw.
%% latex-leseansicht-abspann.tex).
%% ---------------------------------------------------------------

\normalsize

% Das esempio-Environment wird nur in der Leseansicht benötigt
\ifkorrekturansicht\else
\newenvironment{esempio}[3]%
{
    \vspace{1.5ex}
    \rlap{\underline{#1}}
    \par
    \setlength{\parindent}{0cm}
    \nopagebreak
    \leftskip=#2cm
    \rightskip=#3cm
}
{
    \par
}
\fi

\doendnotes{C}
\bigskip
\vfill

\clearpage

\footnotesize

\ifkorrekturansicht
  \lohead{\textsc{register}}
\fi

% theindex-Environment neu definieren ohne reledmac
\makeatletter
\renewenvironment{theindex}{%
  \ifkorrekturansicht
    \section*{\indexname}%
  \else
    \subsubsection*{Index der erwähnten Entitäten}%
  \fi
  \setlength{\parindent}{0pt}%
  \setlength{\parskip}{0pt plus 0.3pt}%
  \let\item\@idxitem
}{%
  \ifkorrekturansicht\clearpage\fi
}
\makeatother

\IfFileExists{\jobname-pw.ind}{\input{\jobname-pw.ind}}{}

% Quellenangabe nur in der Leseansicht
\ifkorrekturansicht\else
% Fallback-Definitionen, falls die .tex-Datei \titel etc. nicht gesetzt hat
\providecommand{\titel}{}
\providecommand{\editorInnen}{}
\providecommand{\dateiname}{\jobname}

\vspace{3cm}

\vfill

\footnotesize
\textsc{Quelle}: \titel. Herausgegeben von {\editorInnen}. In: \emph{Arthur Schnitzler: Briefwechsel mit Autorinnen und Autoren}.
 Digitale Edition, https://schnitzler-briefe.acdh.oeaw.ac.at/{\dateiname}.html (Stand \today)
\fi

\end{document}


