%% latex-leseansicht-vorspann.tex
%% Vorspann für die Leseansicht.
%% Lädt die gemeinsame Datei latex-vorspann.tex mit nicht gesetztem Schalter.

\newif\ifkorrekturansicht
\korrekturansichtfalse

\input{../tex-inputs/latex-vorspann}


         
         \renewcommand{\erwaehntePersonen}{Personen: Friedrich Michael Fels}
         \renewcommand{\erwaehnteInstitutionen}{Institutionen: S. Fischer Verlag}
         \renewcommand{\erwaehnteOrte}{Orte: Berlin, Frankgasse 1, I., Innere Stadt, IX., Alsergrund, Wien}
         \renewcommand{\erwaehnteWerke}{Werke: Die Gegenwart. Zeitschrift für Literatur, Wirtschaftsleben und Kunst, Sterben. Novelle, Sterben. Novelle von Arthur Schnitzler}
               \section[Friedrich M. Fels an Arthur Schnitzler, 23. 4. 1895]{ Friedrich M. Fels an Arthur Schnitzler, 23. 4. 1895}\nopagebreak\mylabel{v}\rehead{ }\begin{ledgroupsized}[t]{13cm}\normalsize\beginnumbering\briefempfaengerindex{Schnitzler, Arthur@\textsc{Schnitzler, Arthur}!zzzFels, Friedrich Michael@\emph{von Friedrich Michael Fels}!1895-04-231@{23. 4. 1895}|(be} \toendnotes[C]{\smallbreak\pagebreak[2]} \Standort{DLA, A:Schnitzler, HS.NZ85.1.2956.}
\physDesc{Kartenbrief, 252 Zeichen
\newline{}Handschrift: schwarze Tinte, lateinische Kurrent
\newline{}Versand: 1) Stempel: »\nobreak{}\oindex{I., Innere Stadt@\textbf{I., Innere Stadt}|pwk}Wien 1/1, 23. 5. 1895, 1–N\nobreak{}«.   2) Stempel: »\nobreak{}\oindex{IX., Alsergrund@\textbf{IX., Alsergrund}|pwk}Wien 9/3, 2\textcolor{gray}{3}. 5. 1895, 3, Bestellt\nobreak{}«. 
\newline{}Schnitzler: mit Bleistift datiert: »23/4 95« und nummeriert: »21« }\toendnotes[C]{\smallbreak}\pstart{}{\pb}Herrn Dr. Arthur Schnitzler\pend{}\pstart{}Wien\oindex{Wien@\textbf{Wien}|pw}\pend{}\pstart{}IX, Frankgaſse 1\oindex{Frankgasse 1@\textbf{Frankgasse 1}|pw}\pend{}{\bigskip}\pstart{}{\pb}Lieber Dr. Schnitzler,\pend\pstart
           In der Gegenwart\pwindex{?? Werk@Nicht ermittelte Verfasserinnen und Verfasser!Gegenwart. Zeitschrift fuer Literatur, Wirtschaftsleben und Kunst1871 – 1931@\emph{Die Gegenwart. Zeitschrift für Literatur, Wirtschaftsleben und Kunst} {[}1871 – 1931{]}|pw} vom 20. d.{ }ſteht eine \label{K_L00430-1v}\edtext{Besprechung\pwindex{?? Werk@Nicht ermittelte Verfasserinnen und Verfasser!Sterben. Novelle von Arthur Schnitzler20. 4. 1895@\emph{Sterben. Novelle von Arthur Schnitzler} {[}20. 4. 1895{]}|pwv}}{\lemma{\textnormal{\emph{Besprechung}}}\Cendnote{\textnormal{»\so{Sterben}\pwindex{Schnitzler, Arthur 15.05.1862 – 21.10.1931@\textsc{Schnitzler, Arthur} (15.05.1862 – 21.10.1931), \emph{Schriftsteller, Mediziner}!Sterben. Novelle1894-10-01 – 1894-12-01@\strich\emph{Sterben. Novelle} {[}1894-10-01 – 1894-12-01{]}|pw}. Novelle von \so{Arthur Schnitzler}. (Berlin\oindex{Berlin@\textbf{Berlin}|pw}, S. Fischer\orgindex{S. Fischer Verlag@S. Fischer Verlag|pw}.) Ein trauriges und peinliches, aber ein feines und
                     bedeutendes Werk eines echten Künstlers. Das Sterben eines Schwindsüchtigen
                     wird geschildert, sein Ringen mit Leben und Tod, das langsame Scheiden von der
                     Geliebten, die Empörung, das Aufbäumen, der Todeskampf. Jeder Zug ist
                     beobachtet und wahr, nichts übertrieben, dem Dramatischen wird discret aus dem
                     Wege gegangen, wie jeglichem schildernden Naturalismus. Das Zuständliche ist
                     knapp, die Menschen ohne Individualisirung gezeichnet – sie haben auch keinen
                     Familiennamen –, aber die Seelenanalyse ist voll feiner Züge, so daß uns weder
                     Ekel noch Schauer erfaßt und die rein menschliche Theilnahme bis zuletzt rege
                     bleibt.« ([O. V.], in: \emph{Die
                        Gegenwart}\pwindex{?? Werk@Nicht ermittelte Verfasserinnen und Verfasser!Gegenwart. Zeitschrift fuer Literatur, Wirtschaftsleben und Kunst1871 – 1931@\emph{Die Gegenwart. Zeitschrift für Literatur, Wirtschaftsleben und Kunst} {[}1871 – 1931{]}|pwk}, Bd. 47, Nr. 16, 20. 4. 1895,
                  S. 255.)}}}\label{K_L00430-1h} Ihrer Novelle\pwindex{Schnitzler, Arthur 15.05.1862 – 21.10.1931@\textsc{Schnitzler, Arthur} (15.05.1862 – 21.10.1931), \emph{Schriftsteller, Mediziner}!Sterben. Novelle1894-10-01 – 1894-12-01@\strich\emph{Sterben. Novelle} {[}1894-10-01 – 1894-12-01{]}|pwv}, sehr knapp und \uline{sehr} anerke{\geminationn}enend, dabei sehr vernünftig – ungefähr so, wie wir
               selbst darüber schreiben würden.\pend
           \pstart
           Herzlichst{\\[\baselineskip]}\spacefill\mbox{Fels}\pend
           \leftskip=0em{}
         
         \endnumbering\mylabel{h}\end{ledgroupsized}  \newcommand{\dateiname}{L00430}\newcommand{\titel}{Friedrich M. Fels an Arthur Schnitzler, 23. 4. 1895}\newcommand{\editorInnen}{Martin Anton Müller und Gerd-Hermann Susen}%% latex-leseansicht-abspann.tex
%% Abspann für die Leseansicht.
%% Der Schalter \ifkorrekturansicht ist bereits durch den Vorspann gesetzt.

%% latex-abspann.tex
%% Gemeinsamer Abspann für Korrekturansicht und Leseansicht.
%% Setzt den Schalter \ifkorrekturansicht voraus (gesetzt in den
%% einbindenden Dateien latex-korrekturansicht-abspann.tex bzw.
%% latex-leseansicht-abspann.tex).
%% ---------------------------------------------------------------

\normalsize

% Das esempio-Environment wird nur in der Leseansicht benötigt
\ifkorrekturansicht\else
\newenvironment{esempio}[3]%
{
    \vspace{1.5ex}
    \rlap{\underline{#1}}
    \par
    \setlength{\parindent}{0cm}
    \nopagebreak
    \leftskip=#2cm
    \rightskip=#3cm
}
{
    \par
}
\fi

\doendnotes{C}
\bigskip
\vfill

\clearpage

\footnotesize

\ifkorrekturansicht
  \lohead{\textsc{register}}
\fi

% theindex-Environment neu definieren ohne reledmac
\makeatletter
\renewenvironment{theindex}{%
  \ifkorrekturansicht
    \section*{\indexname}%
  \else
    \subsubsection*{Index der erwähnten Entitäten}%
  \fi
  \setlength{\parindent}{0pt}%
  \setlength{\parskip}{0pt plus 0.3pt}%
  \let\item\@idxitem
}{%
  \ifkorrekturansicht\clearpage\fi
}
\makeatother

\IfFileExists{\jobname-pw.ind}{\input{\jobname-pw.ind}}{}

% Quellenangabe nur in der Leseansicht
\ifkorrekturansicht\else
% Fallback-Definitionen, falls die .tex-Datei \titel etc. nicht gesetzt hat
\providecommand{\titel}{}
\providecommand{\editorInnen}{}
\providecommand{\dateiname}{\jobname}

\vspace{3cm}

\vfill

\footnotesize
\textsc{Quelle}: \titel. Herausgegeben von {\editorInnen}. In: \emph{Arthur Schnitzler: Briefwechsel mit Autorinnen und Autoren}.
 Digitale Edition, https://schnitzler-briefe.acdh.oeaw.ac.at/{\dateiname}.html (Stand \today)
\fi

\end{document}


      