%% latex-korrekturansicht-vorspann.tex
%% Vorspann für die Korrekturansicht.
%% Lädt die gemeinsame Datei latex-vorspann.tex mit gesetztem Schalter.

\newif\ifkorrekturansicht
\korrekturansichttrue

\input{../tex-inputs/latex-vorspann}


\section[Arthur Schnitzler an Georg Brandes, 8. 6. 1899]{L00923 Arthur Schnitzler an Georg Brandes, 8. 6. 1899}
\nopagebreak\mylabel{L00923v}
\rehead{ }\normalsize\beginnumbering\briefempfaengerindex{Brandes, Georg@\textsc{Brandes, Georg}!zzzSchnitzler, Arthur@\emph{von Arthur Schnitzler}!1899-06-081@{8. 6. 1899}|(be}
\toendnotes[C]{\smallbreak\pagebreak[2]}\Standort{Kopenhagen, Det Kongelige Bibliotek, Georg Brandes Arkiv, box 125.}
\physDesc{Briefkarte, 613 Zeichen
\newline{}Handschrift: schwarze Tinte, deutsche Kurrent
\newline{}Ordnung: mit Bleistift von unbekannter Hand nummeriert:
                                    »17.« }
\buchAbdrucke{\weitereDrucke{Georg Brandes, Arthur Schnitzler: \emph{Ein Briefwechsel}. Bern: \emph{Francke} 1956, S. 77–78.} }\toendnotes[C]{\smallbreak}
\pstart
           \raggedleft{}{\pb}8. 6. 99.\pend
           \vspace{0.5em}
\pstart
           Verehrteſter Herr Brandes, eine Bitte diesmal, deren Erfüllg Ihnen
               hoffentlich nicht allzu viel Mühe macht. Ein Herr \label{K_L00923-1v}\edtext{\textsc{Soutif}\pwindex{Soutif, Emile @\textsc{Soutif, Émile}, \emph{Lehrer/Lehrerin}|pw}}{\lemma{\textnormal{\emph{Soutif}}}\Cendnote{\textnormal{Die Übersetzung ist nicht überliefert.
                  Über Émile Soutif\pwindex{Soutif, Emile @\textsc{Soutif, Émile}, \emph{Lehrer/Lehrerin}|pwk} ist nur der Eintrag im
                     \emph{Adreßbuch für Dresden und Vororte}\pwindex{Adressbuch fuer Dresden und Vororte@\emph{Adreßbuch für Dresden und Vororte}|pwk} (1899, Theil I, S. 580.) bekannt, in dem er als
                     »Lehrer d. franz. Sprache u. Literat.« ausgewiesen ist.}}}\label{K_L00923-1}
               hat eine Überſetzg »des \uline{grünen Kakadu}\pwindex{gruene Kakadu. Groteske in einem Akt@\emph{Der grüne Kakadu. Groteske in einem Akt}|pw}« ins franzöſiſche an \textsc{\uline{Antoine}}\pwindex{Antoine, Andre 1858-01-31 – 1943-10-23@\textsc{Antoine, André} (1858-01-31 – 1943-10-23), \emph{Theaterleiter/Theaterleiterin, Schauspieler/Schauspielerin}|pw} in Paris\oindex{Paris@\textbf{Paris}, \emph{P.PPLC}|pw} geſchickt. Ich weiſs nun kaum, ob
                  \textsc{Antoine}\pwindex{Antoine, Andre 1858-01-31 – 1943-10-23@\textsc{Antoine, André} (1858-01-31 – 1943-10-23), \emph{Theaterleiter/Theaterleiterin, Schauspieler/Schauspielerin}|pw} meinen Namen kennt. Wenn \uline{Sie} aber ihm ein {\pb}Wort ſchreiben, er ſolle das Ding aufmerkſam
               durchleſen, ſo thut er’s gewiß. Alſo daſs Sie ihm ſagen: »Leſen Sie den ›\textsc{Peroquet vert}\pwindex{gruene Kakadu. Groteske in einem Akt@\emph{Der grüne Kakadu. Groteske in einem Akt}|pw}‹«– bitte ich Sie; – nichts anderes, keine »Empfehlung« – oder dergleichen.\pend
           
\pstart
           Es iſt doch nicht zu unbeſcheiden, hoff ich?\pend
           
\pstart
           Sind Sie nun endlich außer Bett? Und wohl – und heiter? Ihr treuer \spacefill\mbox{Arthur
                  Schnitzler}\pend
           \selectlanguage{ngerman}\endnumbering\briefempfaengerindex{Brandes, Georg@\textsc{Brandes, Georg}!zzzSchnitzler, Arthur@\emph{von Arthur Schnitzler}!1899-06-081@{8. 6. 1899}|)be}\mylabel{L00923h}  \normalsize

\doendnotes{C}
\bigskip
\vfill

\clearpage

\footnotesize

\lohead{\textsc{register}}

% Definiere theindex-Environment komplett neu ohne reledmac
\makeatletter
\renewenvironment{theindex}{%
  \section*{\indexname}%
  \setlength{\parindent}{0pt}%
  \setlength{\parskip}{0pt plus 0.3pt}%
  \let\item\@idxitem
}{%
  \clearpage
}
\makeatother

\IfFileExists{\jobname-pw.ind}{\input{\jobname-pw.ind}}{}

\end{document}

      