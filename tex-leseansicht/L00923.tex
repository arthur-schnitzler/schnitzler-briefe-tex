%% latex-leseansicht-vorspann.tex
%% Vorspann für die Leseansicht.
%% Lädt die gemeinsame Datei latex-vorspann.tex mit nicht gesetztem Schalter.

\newif\ifkorrekturansicht
\korrekturansichtfalse

\input{../tex-inputs/latex-vorspann}


\section[Arthur Schnitzler an Georg Brandes, 8. 6. 1899]{L00923 Arthur Schnitzler an Georg Brandes, 8. 6. 1899}
\nopagebreak\mylabel{L00923v}
\rehead{ }\normalsize\beginnumbering\briefempfaengerindex{Brandes, Georg@\textsc{Brandes, Georg}!zzzSchnitzler, Arthur@\emph{von Arthur Schnitzler}!1899-06-081@{8. 6. 1899}|(be}
\toendnotes[C]{\smallbreak\pagebreak[2]}
\correspDesc{Versand  durch Arthur Schnitzler am 8. 6. 1899 in Wien
\newline{}Erhalt  durch Georg Brandes im Zeitraum [8. 6. 1899
                  – 12. 6. 1899?] \textbf{Ort fehlend} }\toendnotes[C]{\smallbreak}
\Standort{Kopenhagen, Det Kongelige Bibliotek, Georg Brandes Arkiv, box 125.}
\physDesc{Briefkarte, 613 Zeichen
\newline{}Handschrift: schwarze Tinte, deutsche Kurrent
\newline{}Ordnung: mit Bleistift von unbekannter Hand nummeriert:
                                    »17.« }
\buchAbdrucke{\weitereDrucke{Georg Brandes, Arthur Schnitzler: \emph{Ein Briefwechsel}. Herausgegeben von Kurt Bergel. Bern: \emph{Francke} 1956, S. 77–78.} }\toendnotes[C]{\smallbreak}
\pstart
           \raggedleft{}{\pb}8. 6. 99.\pend
           \vspace{0.5em}
\pstart
           Verehrteſter Herr Brandes, eine Bitte diesmal, deren Erfüllg Ihnen
               hoffentlich nicht allzu viel Mühe macht. Ein Herr \label{K_L00923-1v}\edtext{\textsc{Soutif}\pwindex{Soutif, Émile @\textsc{Soutif, Émile}, \emph{Lehrer}|pw}}{\lemma{\textnormal{\emph{Soutif}}}\Cendnote{\textnormal{Die Übersetzung ist nicht überliefert.
                  Über Émile Soutif\pwindex{Soutif, Émile @\textsc{Soutif, Émile}, \emph{Lehrer}|pwk} ist nur der Eintrag im
                     \emph{Adreßbuch für Dresden und Vororte}\pwindex{Adreßbuch für Dresden und Vororte@\emph{Adreßbuch für Dresden und Vororte}|pwk} (1899, Theil I, S. 580.) bekannt, in dem er als
                     »Lehrer d. franz. Sprache u. Literat.« ausgewiesen ist.}}}\label{K_L00923-1}
               hat eine Überſetzg »des \uline{grünen Kakadu}\pwindex{Schnitzler, Arthur 15.\,5.\,1862 Wien – 21.\,10.\,1931 ebd.@\textsc{Schnitzler, Arthur} (15.\,5.\,1862 Wien – 21.\,10.\,1931 ebd.), \emph{Schriftsteller, Mediziner}!grüne Kakadu. Groteske in einem Akt@\strich\emph{Der grüne Kakadu. Groteske in einem Akt}|pw}« ins franzöſiſche an \textsc{\uline{Antoine}}\pwindex{Antoine, André 31.\,1.\,1858 Limoges – 23.\,10.\,1943 Le Pouliguen@\textsc{Antoine, André} (31.\,1.\,1858 Limoges – 23.\,10.\,1943 Le Pouliguen), \emph{Theaterleiter, Schauspieler}|pw} in Paris\oindex{Paris@\textbf{Paris}, \emph{Hauptstadt}|pw} geſchickt. Ich weiſs nun kaum, ob
                  \textsc{Antoine}\pwindex{Antoine, André 31.\,1.\,1858 Limoges – 23.\,10.\,1943 Le Pouliguen@\textsc{Antoine, André} (31.\,1.\,1858 Limoges – 23.\,10.\,1943 Le Pouliguen), \emph{Theaterleiter, Schauspieler}|pw} meinen Namen kennt. Wenn \uline{Sie} aber ihm ein {\pb}Wort{ }ſchreiben, er{ }ſolle das Ding aufmerkſam
               durchleſen,{ }ſo thut er’s gewiß. Alſo daſs Sie ihm{ }ſagen: »Leſen Sie den ›\textsc{Peroquet vert}\pwindex{Schnitzler, Arthur 15.\,5.\,1862 Wien – 21.\,10.\,1931 ebd.@\textsc{Schnitzler, Arthur} (15.\,5.\,1862 Wien – 21.\,10.\,1931 ebd.), \emph{Schriftsteller, Mediziner}!grüne Kakadu. Groteske in einem Akt@\strich\emph{Der grüne Kakadu. Groteske in einem Akt}|pw}‹«– bitte ich Sie; – nichts anderes, keine »Empfehlung« – oder dergleichen.\pend
           
\pstart
           Es iſt doch nicht zu unbeſcheiden, hoff ich?\pend
           
\pstart
           Sind Sie nun endlich außer Bett? Und wohl – und heiter? Ihr treuer \spacefill\mbox{Arthur
                  Schnitzler}\pend
           \selectlanguage{ngerman}\endnumbering\briefempfaengerindex{Brandes, Georg@\textsc{Brandes, Georg}!zzzSchnitzler, Arthur@\emph{von Arthur Schnitzler}!1899-06-081@{8. 6. 1899}|)be}\mylabel{L00923h}  \newcommand{\dateiname}{L00923}\newcommand{\titel}{Arthur Schnitzler an Georg Brandes, 8. 6. 1899}\newcommand{\editorInnen}{Martin Anton Müller und Gerd-Hermann Susen}%% latex-leseansicht-abspann.tex
%% Abspann für die Leseansicht.
%% Der Schalter \ifkorrekturansicht ist bereits durch den Vorspann gesetzt.

%% latex-abspann.tex
%% Gemeinsamer Abspann für Korrekturansicht und Leseansicht.
%% Setzt den Schalter \ifkorrekturansicht voraus (gesetzt in den
%% einbindenden Dateien latex-korrekturansicht-abspann.tex bzw.
%% latex-leseansicht-abspann.tex).
%% ---------------------------------------------------------------

\normalsize

% Das esempio-Environment wird nur in der Leseansicht benötigt
\ifkorrekturansicht\else
\newenvironment{esempio}[3]%
{
    \vspace{1.5ex}
    \rlap{\underline{#1}}
    \par
    \setlength{\parindent}{0cm}
    \nopagebreak
    \leftskip=#2cm
    \rightskip=#3cm
}
{
    \par
}
\fi

\doendnotes{C}
\bigskip
\vfill

\clearpage

\footnotesize

\ifkorrekturansicht
  \lohead{\textsc{register}}
\fi

% theindex-Environment neu definieren ohne reledmac
\makeatletter
\renewenvironment{theindex}{%
  \ifkorrekturansicht
    \section*{\indexname}%
  \else
    \subsubsection*{Index der erwähnten Entitäten}%
  \fi
  \setlength{\parindent}{0pt}%
  \setlength{\parskip}{0pt plus 0.3pt}%
  \let\item\@idxitem
}{%
  \ifkorrekturansicht\clearpage\fi
}
\makeatother

\IfFileExists{\jobname-pw.ind}{\input{\jobname-pw.ind}}{}

% Quellenangabe nur in der Leseansicht
\ifkorrekturansicht\else
% Fallback-Definitionen, falls die .tex-Datei \titel etc. nicht gesetzt hat
\providecommand{\titel}{}
\providecommand{\editorInnen}{}
\providecommand{\dateiname}{\jobname}

\vspace{3cm}

\vfill

\footnotesize
\textsc{Quelle}: \titel. Herausgegeben von {\editorInnen}. In: \emph{Arthur Schnitzler: Briefwechsel mit Autorinnen und Autoren}.
 Digitale Edition, https://schnitzler-briefe.acdh.oeaw.ac.at/{\dateiname}.html (Stand \today)
\fi

\end{document}


