%% latex-leseansicht-vorspann.tex
%% Vorspann für die Leseansicht.
%% Lädt die gemeinsame Datei latex-vorspann.tex mit nicht gesetztem Schalter.

\newif\ifkorrekturansicht
\korrekturansichtfalse

\input{../tex-inputs/latex-vorspann}


               \section[Arthur Schnitzler an Georg Brandes, 8. 6. 1899]{ Arthur Schnitzler an Georg Brandes, 8. 6. 1899}\nopagebreak\mylabel{v}\rehead{ }\begin{ledgroupsized}[t]{13cm}\normalsize\beginnumbering\briefempfaengerindex{Brandes, Georg@\textsc{Brandes, Georg}!zzzSchnitzler, Arthur@\emph{von Arthur Schnitzler}!1899-06-081@{8. 6. 1899}|(be} \toendnotes[C]{\smallbreak\pagebreak[2]} \Standort{Kopenhagen, Det Kongelige Bibliotek, Georg Brandes Arkiv, box 125.}
\physDesc{Briefkarte
\newline{}Handschrift: schwarze Tinte, deutsche Kurrent\newline{}Ordnung: mit Bleistift von unbekannter Hand nummeriert: »17.« }\buchAbdrucke{\weitereDrucke{Georg Brandes, Arthur Schnitzler: \emph{Ein Briefwechsel}. Hg. Kurt Bergel. Bern: \emph{Francke} 1956, S. 77–78.} }\toendnotes[C]{\smallbreak}\pstart
           \raggedleft{}{\pb}8. 6. 99.\pend
           \pstart
           Verehrteſter Herr Brandes, eine Bitte diesmal, deren Erfüllg
                    Ihnen hoffentlich nicht allzu viel Mühe macht. Ein Herr \label{K_L00923_1v}\edtext{\textsc{Soutif}\pwindex{Soutif, Emile @\textsc{Soutif, Émile}, \emph{Lehrer}|pw}}{\lemma{\textnormal{\emph{Soutif}}}\Cendnote{\textnormal{Die Übersetzung ist nicht überliefert.
                        Über Émile Soutif\pwindex{Soutif, Emile @\textsc{Soutif, Émile}, \emph{Lehrer}|pwk} ist nur der Eintrag im
                            \emph{Adreßbuch für Dresden und Vororte}\pwindex{?? Werk@Nicht ermittelte Verfasserinnen und Verfasser!Adressbuch fuer Dresden und Vororte1897 – 1914@\emph{Adreßbuch für Dresden und Vororte} {[}1897 – 1914{]}|pwk} (1899, Theil I, S. 580.) bekannt, in dem er als
                            »Lehrer d. franz. Sprache u. Literat.« ausgewiesen
                        ist.}}}\label{K_L00923_1h} hat eine Überſetzg »des \uline{grünen Kakadu}\pwindex{Schnitzler, Arthur 15.05.1862 – 21.10.1931@\textsc{Schnitzler, Arthur} (15.05.1862 – 21.10.1931), \emph{Schriftsteller, Mediziner}!gruene Kakadu. Groteske in einem Akt1.3.1899 – 1.3.1899@\strich\emph{Der grüne Kakadu. Groteske in einem Akt} {[}1.3.1899 – 1.3.1899{]}|pw}« ins franzöſiſche an \textsc{\uline{Antoine}}\pwindex{Antoine, Andre 31.01.1858 – 23.10.1943@\textsc{Antoine, André} (31.01.1858 – 23.10.1943), \emph{Theaterleiter, Schauspieler}|pw} in Paris\oindex{Paris@\textbf{Paris}|pw} geſchickt. Ich weiſs
                    nun kaum, ob \textsc{Antoine}\pwindex{Antoine, Andre 31.01.1858 – 23.10.1943@\textsc{Antoine, André} (31.01.1858 – 23.10.1943), \emph{Theaterleiter, Schauspieler}|pw} meinen Namen kennt. Wenn \uline{Sie} aber
                    ihm ein {\pb}Wort ſchreiben, er ſolle das
                    Ding aufmerkſam durchleſen, ſo thut er’s gewiß. Alſo daſs Sie ihm ſagen: »Leſen
                    Sie den ›\textsc{Peroquet vert}\pwindex{Schnitzler, Arthur 15.05.1862 – 21.10.1931@\textsc{Schnitzler, Arthur} (15.05.1862 – 21.10.1931), \emph{Schriftsteller, Mediziner}!gruene Kakadu. Groteske in einem Akt1.3.1899 – 1.3.1899@\strich\emph{Der grüne Kakadu. Groteske in einem Akt} {[}1.3.1899 – 1.3.1899{]}|pw}‹«– bitte ich Sie; – nichts anderes, keine »Empfehlung« – oder
                    dergleichen.\pend
           \pstart
           Es iſt doch nicht zu unbeſcheiden, hoff ich?\pend
           \pstart
           Sind Sie nun endlich außer Bett? Und wohl – und heiter? Ihr treuer \spacefill\mbox{Arthur
                        Schnitzler}\pend
                     \endnumbering\briefempfaengerindex{Brandes, Georg@\textsc{Brandes, Georg}!zzzSchnitzler, Arthur@\emph{von Arthur Schnitzler}!1899-06-081@{8. 6. 1899}|)be}\mylabel{h}\end{ledgroupsized}  \newcommand{\dateiname}{L00923}\newcommand{\titel}{Arthur Schnitzler an Georg Brandes, 8. 6. 1899}\newcommand{\editorInnen}{Martin Anton Müller und Gerd-Hermann Susen}
            \footnotesize
\begin{ledgroupsized}[t]{11.5cm}
\doendnotes{C}
\end{ledgroupsized}
         %% latex-leseansicht-abspann.tex
%% Abspann für die Leseansicht.
%% Der Schalter \ifkorrekturansicht ist bereits durch den Vorspann gesetzt.

%% latex-abspann.tex
%% Gemeinsamer Abspann für Korrekturansicht und Leseansicht.
%% Setzt den Schalter \ifkorrekturansicht voraus (gesetzt in den
%% einbindenden Dateien latex-korrekturansicht-abspann.tex bzw.
%% latex-leseansicht-abspann.tex).
%% ---------------------------------------------------------------

\normalsize

% Das esempio-Environment wird nur in der Leseansicht benötigt
\ifkorrekturansicht\else
\newenvironment{esempio}[3]%
{
    \vspace{1.5ex}
    \rlap{\underline{#1}}
    \par
    \setlength{\parindent}{0cm}
    \nopagebreak
    \leftskip=#2cm
    \rightskip=#3cm
}
{
    \par
}
\fi

\doendnotes{C}
\bigskip
\vfill

\clearpage

\footnotesize

\ifkorrekturansicht
  \lohead{\textsc{register}}
\fi

% theindex-Environment neu definieren ohne reledmac
\makeatletter
\renewenvironment{theindex}{%
  \ifkorrekturansicht
    \section*{\indexname}%
  \else
    \subsubsection*{Index der erwähnten Entitäten}%
  \fi
  \setlength{\parindent}{0pt}%
  \setlength{\parskip}{0pt plus 0.3pt}%
  \let\item\@idxitem
}{%
  \ifkorrekturansicht\clearpage\fi
}
\makeatother

\IfFileExists{\jobname-pw.ind}{\input{\jobname-pw.ind}}{}

% Quellenangabe nur in der Leseansicht
\ifkorrekturansicht\else
% Fallback-Definitionen, falls die .tex-Datei \titel etc. nicht gesetzt hat
\providecommand{\titel}{}
\providecommand{\editorInnen}{}
\providecommand{\dateiname}{\jobname}

\vspace{3cm}

\vfill

\footnotesize
\textsc{Quelle}: \titel. Herausgegeben von {\editorInnen}. In: \emph{Arthur Schnitzler: Briefwechsel mit Autorinnen und Autoren}.
 Digitale Edition, https://schnitzler-briefe.acdh.oeaw.ac.at/{\dateiname}.html (Stand \today)
\fi

\end{document}


      