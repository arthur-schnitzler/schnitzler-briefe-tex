%% latex-leseansicht-vorspann.tex
%% Vorspann für die Leseansicht.
%% Lädt die gemeinsame Datei latex-vorspann.tex mit nicht gesetztem Schalter.

\newif\ifkorrekturansicht
\korrekturansichtfalse

\input{../tex-inputs/latex-vorspann}


\section[Paul Goldmann an Arthur Schnitzler, 18. 8. [1893]]{L02712 Paul Goldmann an Arthur Schnitzler, 18. 8. [1893]}
\nopagebreak\mylabel{L02712v}
\rehead{ }\normalsize\beginnumbering\briefempfaengerindex{Schnitzler, Arthur@\textsc{Schnitzler, Arthur}!zzzGoldmann, Paul@\emph{von Paul Goldmann}!1893-08-182@{18. 8. [1893]}|(be}
\toendnotes[C]{\smallbreak\pagebreak[2]}
\correspDesc{Versand  durch Paul Goldmann am 18. 8. [1893] in Paris
\newline{}Erhalt  durch Arthur Schnitzler im Zeitraum [19. 8. 1893
                  – 23. 8. 1893?] in Wien}\toendnotes[C]{\smallbreak}
\Standort{DLA, A:Schnitzler, HS.NZ85.1.3163.}
\physDesc{Brief, 2 Blätter, 8 Seiten, 4377 Zeichen
\newline{}Handschrift: schwarze Tinte, deutsche Kurrent
\newline{}Schnitzler: 1) mit Bleistift das Jahr »93« vermerkt  2) mit rotem Buntstift zwei Unterstreichungen}\toendnotes[C]{\smallbreak}
\pstart
           {\pb}\textcolor{gray}{\textbf{\textbf{Frankfurter Zeitung\orgindex{Frankfurter Zeitung@Frankfurter Zeitung|pw}.}}}\pend
           
\pstart
           \textcolor{gray}{\textbf{\textbf{(\begin{otherlanguage}{french}Gazette de Francfort\end{otherlanguage}\orgindex{Frankfurter Zeitung@Frankfurter Zeitung|pw}.)}}}\pend
           
\pstart
           \textcolor{gray}{\textbf{\begin{otherlanguage}{french}Directeur\end{otherlanguage}{ }\textbf{M. L. Sonnemann\pwindex{Sonnemann, Leopold 29.\,10.\,1831 Höchberg – 30.\,10.\,1909 Frankfurt am Main@\textsc{Sonnemann, Leopold} (29.\,10.\,1831 Höchberg – 30.\,10.\,1909 Frankfurt am Main), \emph{Journalist, Herausgeber}|pw}.}}}\hfill \textsc{Paris\oindex{Paris@\textbf{Paris}, \emph{Hauptstadt}|pw}}, 18. August.\pend
           
\pstart
           \begin{otherlanguage}{french}\textcolor{gray}{\textbf{Journal politique, financier,}}\end{otherlanguage}\pend
           
\pstart
           \begin{otherlanguage}{french}\textcolor{gray}{\textbf{commercial et litteraire.}}\end{otherlanguage}\pend
           
\pstart
           \begin{otherlanguage}{french}\textcolor{gray}{\textbf{\textbf{Paraissant trois fois par jour}}}\end{otherlanguage}\pend
           
\pstart
           \begin{otherlanguage}{french}\textcolor{gray}{\textbf{\textbf{Bureaux à Paris\oindex{Paris@\textbf{Paris}, \emph{Hauptstadt}|pw}:}}}\end{otherlanguage}\pend
           
\pstart
           \begin{otherlanguage}{french}\textcolor{gray}{\textbf{\textbf{rue Richelieu 75\oindex{rue Richelieu@\textbf{rue Richelieu}, \emph{Straße}|pw}.}}}\end{otherlanguage}\pend
           
\pstart\center{}Mein lieber Arthur!\pend\vspace{0.5em}
\pstart
           Ich habe Dir nicht{ }ſofort geantwortet, weil ich erſt die Antwort des H. \textsc{Sonnemann}\pwindex{Sonnemann, Leopold 29.\,10.\,1831 Höchberg – 30.\,10.\,1909 Frankfurt am Main@\textsc{Sonnemann, Leopold} (29.\,10.\,1831 Höchberg – 30.\,10.\,1909 Frankfurt am Main), \emph{Journalist, Herausgeber}|pw}, meines Chefs\pwindex{Sonnemann, Leopold 29.\,10.\,1831 Höchberg – 30.\,10.\,1909 Frankfurt am Main@\textsc{Sonnemann, Leopold} (29.\,10.\,1831 Höchberg – 30.\,10.\,1909 Frankfurt am Main), \emph{Journalist, Herausgeber}|pwv},
               betreffend meinen Urlaub abwarten und Dir Beſtimmtes über meine Reiſepläne mittheilen
               wollte. Bis jetzt iſt noch nichts gekommen, und ich will nun die Antwort auf Deine
               lieben Zeilen nicht länger verſchieben. Aus der Verzögerung der Antwort des Chefs\pwindex{Sonnemann, Leopold 29.\,10.\,1831 Höchberg – 30.\,10.\,1909 Frankfurt am Main@\textsc{Sonnemann, Leopold} (29.\,10.\,1831 Höchberg – 30.\,10.\,1909 Frankfurt am Main), \emph{Journalist, Herausgeber}|pwv}{ }ſchließe ich, daß meine
               Bitte um{ }ſofortige Beurlaubung nicht bewilligt werden und daß ich genöthigt werden
               dürfte, bis nach den \label{K_L02712-1v}\edtext{Stichwahlen}{\lemma{\textnormal{\emph{Stichwahlen}}}\Cendnote{\textnormal{In Frankreich\oindex{Frankreich@\textbf{Frankreich}|pwk} wurde am 20. 8. 1893 ein neues
                  Parlament gewählt. Am 3. 9. 1893 gewann Jean Casimir-Perier\pwindex{Casimir-Perier, Jean 8.\,11.\,1847 Paris – 11.\,3.\,1907 ebd.@\textsc{Casimir-Perier, Jean} (8.\,11.\,1847 Paris – 11.\,3.\,1907 ebd.), \emph{Politiker, Präsident}|pwk} die Stichwahl gegen Georges Clemenceau\pwindex{Clemenceau, Georges 28.\,9.\,1841 Mouilleron-en-Pareds – 24.\,11.\,1929 Paris@\textsc{Clemenceau, Georges} (28.\,9.\,1841 Mouilleron-en-Pareds – 24.\,11.\,1929 Paris), \emph{Politiker}|pwk}.}}}\label{K_L02712-1}{ }{\pb}– 3. September – zu
               bleiben. Dann komme ich höchſtwahrſcheinlich im Lauf des September nach \textsc{Salzburg\oindex{Salzburg@\textbf{Salzburg}, \emph{Verwaltungsgebiet}|pw}}, und falls Du \label{K_L02712-2v}\edtext{verreiſt}{\lemma{\textnormal{\emph{verreist}}}\Cendnote{\textnormal{Im Sommer, nach dem 18. 8. 1893, verreiste Schnitzler
                  vom 22. 8. 1893 bis
                  zum 31. 8. 1893 nach
                     Tirol\oindex{Tirol@\textbf{Tirol}, \emph{Land}|pwk}, Südtirol\oindex{Südtirol@\textbf{Südtirol}, \emph{Verwaltungsgebiet}|pwk}, Italien\oindex{Italien@\textbf{Italien}|pwk}, Kärnten\oindex{Kärnten@\textbf{Kärnten}, \emph{Land}|pwk}, Niederösterreich\oindex{Niederösterreich@\textbf{Niederösterreich}, \emph{Land}|pwk} und in die Steiermark\oindex{Steiermark@\textbf{Steiermark}, \emph{Land}|pwk}. Am 5. 9. 1893 und vom 9. 9. 1893 bis 11. 9. 1893 war Schnitzler außerdem in Reichenau an der
                     Rax\oindex{Reichenau an der Rax@\textbf{Reichenau an der Rax}, \emph{Verwaltungsgebiet}|pwk}, vom 16. 9. 1893 bis 19. 9. 1893 in Salzburg\oindex{Salzburg@\textbf{Salzburg}, \emph{Verwaltungsgebiet}|pwk}, wo er
                  jedenfalls am 17. 9. 1893 und 18. 9. 1893{ }Goldmann\pwindex{Goldmann, Paul 31.\,1.\,1865 Breslau – 25.\,9.\,1935 Wien@\textsc{Goldmann, Paul} (31.\,1.\,1865 Breslau – 25.\,9.\,1935 Wien), \emph{Schriftsteller, Journalist}|pwk} traf. Ein damit einhergehendes
                  Zusammentreffen mit Hugo von Hofmannsthal\pwindex{Hofmannsthal, Hugo von 1.\,2.\,1874 Wien – 15.\,7.\,1929 Rodaun@\textsc{Hofmannsthal, Hugo von} (1.\,2.\,1874 Wien – 15.\,7.\,1929 Rodaun), \emph{Schriftsteller}|pwk}
                  und Richard Beer-Hofmann\pwindex{Beer-Hofmann, Richard 11.\,7.\,1866 Wien – 26.\,9.\,1945 New York City@\textsc{Beer-Hofmann, Richard} (11.\,7.\,1866 Wien – 26.\,9.\,1945 New York City), \emph{Schriftsteller}|pwk} ist nicht
                  bekannt.}}}\label{K_L02712-2}, bitte ich Dich, mir jetzt noch raſch eine Adreſſe mitzutheilen,
               wo Dich ein Telegramm oder ein Brief von mir erreichen kann. Ich kann Dir gar nicht{ }ſagen, wie unendlich ich mich auf ein Wiederſehen mit Dir freue. Aber ich bitte Dich
               nochmals dringend, Dich auf Enttäuſchungen vorzubereiten. Ich habe mich nicht zu
               meinem Vortheil verändert.\pend
           
\pstart
           Was Du{ }ſonſt über die Beziehungen zwiſchen Dir und mir{ }ſchreibſt, iſt lieb und gut
               und hat mir aufrichtig wohlgethan. Aber wenn Du einen Ton des Zweifels bei {\pb}mir bemerkſt – ich glaube allerdings, Du haſt
               Unrecht, – trägſt Du nicht auch eine Schuld? Denk’ Dir nur, was Du mir während dieſer
               Jahre geſchrieben haſt und was nicht. Du haſt mich einzig und allein an Deinem
               literariſchen Leben theilnehmen laſſen. Aber von Deinem Perſönlichen, was mir doch
               bei allem Intereſſe für das Erſte das unendlich Werthvollere iſt, weiß ich rein gar
               nichts mehr. Höchſtens hier und da eine Andeutung, es{ }ſei Dir unmöglich, über{ }ſolche
               Dinge zu{ }ſchreiben. Und da ich weiß, daß Du mir ähnlich biſt, und da ich mich kenne,
               wie ich das Wort »unmöglich« gebrauche, weil es{ }ſchöner klingt als »unbequem«, {\pb}wie es doch eigentlich heißen{ }ſollte, –{ }ſo habe ich
               manchmal Reflexionen darüber gemacht – nicht bittere, aber{ }ſchmerzliche. Nun, das{ }ſoll{ }ſich wohl Alles jetzt wieder ausgleichen\textcolor{gray}{.} Auch Deine
               Bitterkeit gegen mich. Denn bei aller Feinheit des Taktes, bei alle\substVorne{}\textsuperscript{\textcolor{gray}{n}}\substDazwischen{}m\substHinten{} noblen Wunſch,{ }ſie zurückzudrängen, klingt{ }ſie in Deinen Briefen durch, und
               ich glaube, immer zu leſen: Nicht einmal \label{K_L02712-3v}\edtext{eine Beſprechung}{\lemma{\textnormal{\emph{eine Besprechung}}}\Cendnote{\textnormal{von \emph{Anatol}\pwindex{Schnitzler, Arthur 15.\,5.\,1862 Wien – 21.\,10.\,1931 ebd.@\textsc{Schnitzler, Arthur} (15.\,5.\,1862 Wien – 21.\,10.\,1931 ebd.), \emph{Schriftsteller, Mediziner}!Anatol@\strich\emph{Anatol}|pwk}}}}\label{K_L02712-3} in der Frankfurter Zeitung\pwindex{Frankfurter Zeitung@\emph{Frankfurter Zeitung}|pw} hat er mir
               geliefert! Da habe ich wirklich große Schuld. Ich weiß wohl, daß ich nicht gekonnt
               habe. Aber wenn ich{ }ſo zurückdenke, habe ich keine Ahnung, wie das{ }ſo eigentlich {\pb}gekommen iſt. Ich meine, es war doch viel
               Willensſchwäche von meiner Seite dabei. Aber auch darüber wollen wir reden. Über
               Deine{ }ſonſtigen Autoren-Leiden, mein liebſter Arthur, \strikeout{f\textcolor{gray}{×}\-\textcolor{gray}{×}} haſt Du keinen Grund, Dich beſonders traurig zu fühlen. Das gehört dazu, ich{ }ſchwöre es Dir, und iſt nur eine zurückzulegende Etape. In \textsc{Paris\oindex{Paris@\textbf{Paris}, \emph{Hauptstadt}|pw}} iſt doch das geiſtige Leben noch ganz anders entwickelt als in Deutſchland\oindex{Deutschland@\textbf{Deutschland}|pw} und Öſterreich\oindex{Österreich@\textbf{Österreich}|pw}, ich meine in Bezug auf die \label{K_L02712-4v}\edtext{Zahl der jährlich geſchriebenen {\pb}und gedruckten Werke}{\lemma{\textnormal{\emph{Zahl … Werke}}}\Cendnote{\textnormal{Einen internationalen Vergleich der jährlichen Drucke ermöglicht eine Statistik\pwindex{Vermischtes@\emph{Vermischtes}|pwkv} aus dem Jahr 1895: »{[}Es{]} existieren zur Zeit 3985 Papierfabriken auf der Erde,
                     deren Gesammtproduktion sich auf 7904 Millionen Buch im Jahre beläuft. Die
                     Hälfte dieses riesigen Papiermaterials verbraucht die Buchdruckerei,
                     während 600 Millionen Buch auf die Zeitungen entfallen. Per Kopf berechnet
                     verbraucht der Engländ\oindex{England@\textbf{England}, \emph{Land}|pwv}er
                     von allen Nationen am meisten Papier, nämlich 11 ½ Buch im Durchschnitt pro
                     Jahr. Nach ihm kommt der Amerika\oindex{Vereinigte Staaten von Amerika [USA]@\textbf{Vereinigte Staaten von Amerika [USA]}|pw}ner mit
                      10 ¼ Buch pro Jahr und Kopf. Hierauf der Deutſch\oindex{Deutschland@\textbf{Deutschland}|pwv}e mit 8 und der Fran\oindex{Frankreich@\textbf{Frankreich}|pwv}zose mit 7 ½  Buch. Weitaus weniger konsumiren Oesterreich\oindex{Österreich@\textbf{Österreich}|pw} und Italien\oindex{Italien@\textbf{Italien}|pw} an Papier, da bei beiden Nationen die
                     durchschnittliche Ziffer pro Jahr und Kopf nur 3 ½ Buch beträgt. Zum Schluß
                     kommt der Mexik\oindex{Mexiko@\textbf{Mexiko}|pwv}aner
                     mit 2, der Spanie\oindex{Spanien@\textbf{Spanien}|pwv}r
                     mit 1 ½ und als letzter der Russ\oindex{Russland@\textbf{Russland}|pwv}e mit gar nur 1 ⅝ Buch Papier, welches pro Jahr auf den Einwohner
                     entfällt.« ([O. V.]: \emph{Vermischtes}\pwindex{Vermischtes@\emph{Vermischtes}|pwk}. In: \emph{Vorwärts}\pwindex{Vorwärts@\emph{Vorwärts}|pwk},
                     Jg. 12, Nr. 191, 17. 8. 1895,
                  S. 7.)}}}\label{K_L02712-4}. Und was ich da{ }ſo über Dummheit und Gemeinheit von Verlegern
               erzählen höre. Ein anderes Beiſpiel: Hier lebt \label{K_L02712-5v}\edtext{\textsc{Knut Hamsun\pwindex{Hamsun, Knut 4.\,8.\,1859 Lom – 19.\,2.\,1952 Grimstad@\textsc{Hamsun, Knut} (4.\,8.\,1859 Lom – 19.\,2.\,1952 Grimstad), \emph{Schriftsteller, Nobelpreisträger}|pw}}}{\lemma{\textnormal{\emph{Knut Hamsun}}}\Cendnote{\textnormal{Durch seinen Roman Hunger\pwindex{Hamsun, Knut 4.\,8.\,1859 Lom – 19.\,2.\,1952 Grimstad@\textsc{Hamsun, Knut} (4.\,8.\,1859 Lom – 19.\,2.\,1952 Grimstad), \emph{Schriftsteller, Nobelpreisträger}!Sult@\strich\emph{Sult}|pwkv} (norweg. \emph{Sult}\pwindex{Hamsun, Knut 4.\,8.\,1859 Lom – 19.\,2.\,1952 Grimstad@\textsc{Hamsun, Knut} (4.\,8.\,1859 Lom – 19.\,2.\,1952 Grimstad), \emph{Schriftsteller, Nobelpreisträger}!Sult@\strich\emph{Sult}|pwk}, 1890) berühmt geworden, lebte Knut Hamsun\pwindex{Hamsun, Knut 4.\,8.\,1859 Lom – 19.\,2.\,1952 Grimstad@\textsc{Hamsun, Knut} (4.\,8.\,1859 Lom – 19.\,2.\,1952 Grimstad), \emph{Schriftsteller, Nobelpreisträger}|pwk} zwischen 1893 und
                     1895 an der Adresse 8 rue de
                     Vaurigard\oindex{rue de Vaugirard@\textbf{rue de Vaugirard}, \emph{Straße}|pwk} in Paris\oindex{Paris@\textbf{Paris}, \emph{Hauptstadt}|pwk}.}}}\label{K_L02712-5}, deſſen
               glänzendes Talent Du doch kennſt. Seit Jahresfriſt muß er mit zwei neuen Romanen\pwindex{Hamsun, Knut 4.\,8.\,1859 Lom – 19.\,2.\,1952 Grimstad@\textsc{Hamsun, Knut} (4.\,8.\,1859 Lom – 19.\,2.\,1952 Grimstad), \emph{Schriftsteller, Nobelpreisträger}!Neue Erde. Roman@\strich\emph{Neue Erde. Roman}|pwuv}\pwindex{Hamsun, Knut 4.\,8.\,1859 Lom – 19.\,2.\,1952 Grimstad@\textsc{Hamsun, Knut} (4.\,8.\,1859 Lom – 19.\,2.\,1952 Grimstad), \emph{Schriftsteller, Nobelpreisträger}!Mysterien. Roman@\strich\emph{Mysterien. Roman}|pwuv}, deren \strikeout{Ein\textcolor{gray}{e}}{ }\label{K_L02712-6v}\edtext{einen}{\lemma{\textnormal{\emph{einen}}}\Cendnote{\textnormal{nicht rekonstruierbar}}}\label{K_L02712-6} mein Onkel\pwindex{Mamroth, Fedor 21.\,2.\,1851 Breslau – 25.\,6.\,1907 Frankfurt am Main@\textsc{Mamroth, Fedor} (21.\,2.\,1851 Breslau – 25.\,6.\,1907 Frankfurt am Main), \emph{Journalist, Kritiker}|pwv} geſehen hat und auch als höchſt
               bedeutend bezeichnet – er hat ihn aus demſelben Grunde nicht drucken können wie den
                  \label{K_L02712-7v}\edtext{Deinen\pwindex{Schnitzler, Arthur 15.\,5.\,1862 Wien – 21.\,10.\,1931 ebd.@\textsc{Schnitzler, Arthur} (15.\,5.\,1862 Wien – 21.\,10.\,1931 ebd.), \emph{Schriftsteller, Mediziner}!Sterben. Novelle@\strich\emph{Sterben. Novelle}|pwv}}{\lemma{\textnormal{\emph{Deinen}}}\Cendnote{\textnormal{Siehe XXXX Auszeichnungsfehler: Dokument L00216 nicht gefunden. }}}\label{K_L02712-7} – muß alſo
               bei allen deutſch\oindex{Deutschland@\textbf{Deutschland}|pwv}en Verlegern
               hauſiren gehen, findet nicht \uline{einen}, lebt nur durch
               die Wohlthat zweier \label{K_L02712-8v}\edtext{\textsc{Mäcene\pwindex{Langen, Albert 8.\,7.\,1869 Antwerpen – 30.\,4.\,1909 München@\textsc{Langen, Albert} (8.\,7.\,1869 Antwerpen – 30.\,4.\,1909 München), \emph{Verleger}|pwuv}}}{\lemma{\textnormal{\emph{Mäcene}}}\Cendnote{\textnormal{Es dürfte sich um Willy Gretor\pwindex{Gretor, Willy 16.\,7.\,1868 Kaliningrad – 31.\,7.\,1923 Kopenhagen@\textsc{Gretor, Willy} (16.\,7.\,1868 Kaliningrad – 31.\,7.\,1923 Kopenhagen), \emph{Maler, Kunstagent, Kunsthändler}|pwk} und Albert
                     Langen\pwindex{Langen, Albert 8.\,7.\,1869 Antwerpen – 30.\,4.\,1909 München@\textsc{Langen, Albert} (8.\,7.\,1869 Antwerpen – 30.\,4.\,1909 München), \emph{Verleger}|pwk} handeln. Langen\pwindex{Langen, Albert 8.\,7.\,1869 Antwerpen – 30.\,4.\,1909 München@\textsc{Langen, Albert} (8.\,7.\,1869 Antwerpen – 30.\,4.\,1909 München), \emph{Verleger}|pwk} hatte zuerst
                  dem \emph{S. Fischer Verlag}\orgindex{S. Fischer Verlag@S. Fischer Verlag|pwk} eine Kostenbeteiligung
                  für den Abdruck von Hamsuns\pwindex{Hamsun, Knut 4.\,8.\,1859 Lom – 19.\,2.\,1952 Grimstad@\textsc{Hamsun, Knut} (4.\,8.\,1859 Lom – 19.\,2.\,1952 Grimstad), \emph{Schriftsteller, Nobelpreisträger}|pwk}{ }\emph{Mysterien}\pwindex{Hamsun, Knut 4.\,8.\,1859 Lom – 19.\,2.\,1952 Grimstad@\textsc{Hamsun, Knut} (4.\,8.\,1859 Lom – 19.\,2.\,1952 Grimstad), \emph{Schriftsteller, Nobelpreisträger}!Mysterien. Roman@\strich\emph{Mysterien. Roman}|pwk} angeboten und, nach der Ablehnung, dafür
                     1894 einen eigenen Verlag\orgindex{Albert Langen@Albert Langen|pwkv}
                  gegründet. Im \emph{Albert Langen Verlag}\orgindex{Albert Langen@Albert Langen|pwk} erschien im selben Jahr auch der Roman \emph{Neue Erde}\pwindex{Hamsun, Knut 4.\,8.\,1859 Lom – 19.\,2.\,1952 Grimstad@\textsc{Hamsun, Knut} (4.\,8.\,1859 Lom – 19.\,2.\,1952 Grimstad), \emph{Schriftsteller, Nobelpreisträger}!Neue Erde. Roman@\strich\emph{Neue Erde. Roman}|pwk}.}}}\label{K_L02712-8} und wird{ }ſeine Bücher\pwindex{Hamsun, Knut 4.\,8.\,1859 Lom – 19.\,2.\,1952 Grimstad@\textsc{Hamsun, Knut} (4.\,8.\,1859 Lom – 19.\,2.\,1952 Grimstad), \emph{Schriftsteller, Nobelpreisträger}!Neue Erde. Roman@\strich\emph{Neue Erde. Roman}|pwuv}\pwindex{Hamsun, Knut 4.\,8.\,1859 Lom – 19.\,2.\,1952 Grimstad@\textsc{Hamsun, Knut} (4.\,8.\,1859 Lom – 19.\,2.\,1952 Grimstad), \emph{Schriftsteller, Nobelpreisträger}!Mysterien. Roman@\strich\emph{Mysterien. Roman}|pwuv} nur
               publiciren können, wenn ihm die Letzteren\pwindex{Langen, Albert 8.\,7.\,1869 Antwerpen – 30.\,4.\,1909 München@\textsc{Langen, Albert} (8.\,7.\,1869 Antwerpen – 30.\,4.\,1909 München), \emph{Verleger}|pwuv}{ }{\pb}Geld leihen, um{ }ſie im Selbſtverlag\orgindex{Albert Langen@Albert Langen|pwv} erſcheinen zu laſſen. Dein \textsc{Anatol\pwindex{Schnitzler, Arthur 15.\,5.\,1862 Wien – 21.\,10.\,1931 ebd.@\textsc{Schnitzler, Arthur} (15.\,5.\,1862 Wien – 21.\,10.\,1931 ebd.), \emph{Schriftsteller, Mediziner}!Anatol@\strich\emph{Anatol}|pw}} wird meiner Anſicht nach{ }ſehr gekauft werden, wenn Du erſt einen \label{K_L02712-9v}\edtext{Bühnenerfolg haben wirſt}{\lemma{\textnormal{\emph{Bühnenerfolg haben wirst}}}\Cendnote{\textnormal{Die erste vollständige Aufführung des \emph{Anatol}\pwindex{Schnitzler, Arthur 15.\,5.\,1862 Wien – 21.\,10.\,1931 ebd.@\textsc{Schnitzler, Arthur} (15.\,5.\,1862 Wien – 21.\,10.\,1931 ebd.), \emph{Schriftsteller, Mediziner}!Anatol@\strich\emph{Anatol}|pwk}-Zyklus fand erst am 3. 12. 1910 statt
                  (doppelte Uraufführung\eventindex{Volkstheater@\textbf{Volkstheater}!Uraufführung von Anatol, 3.12.1910@Uraufführung von Anatol, 3.12.1910|pwkv}\eventindex{Lessing-Theater@\textbf{Lessing-Theater}!Uraufführung (II) von Anatol, 3.12.1910@Uraufführung (II) von Anatol, 3.12.1910|pwkv} am \emph{Lessing-Theater}\orgindex{Lessing-Theater@Lessing-Theater|pwk} in
                     Berlin\oindex{Berlin@\textbf{Berlin}, \emph{Hauptstadt}|pwk} und am \emph{Deutschen Volkstheater}\orgindex{Volkstheater@Volkstheater|pwk} in Wien\oindex{Wien@\textbf{Wien}, \emph{Verwaltungsgebiet}|pwk}). Neue Auflagen des Zyklus\pwindex{Schnitzler, Arthur 15.\,5.\,1862 Wien – 21.\,10.\,1931 ebd.@\textsc{Schnitzler, Arthur} (15.\,5.\,1862 Wien – 21.\,10.\,1931 ebd.), \emph{Schriftsteller, Mediziner}!Anatol@\strich\emph{Anatol}|pwkv} gab es jedoch schon ab 1895 bei \emph{S. Fischer}\orgindex{S. Fischer Verlag@S. Fischer Verlag|pwk}.}}}\label{K_L02712-9}.
                  \label{K_L02712-10v}\edtext{\textsc{Sudermanns\pwindex{Sudermann, Hermann 30.\,9.\,1857 Macikai – 21.\,11.\,1928 Berlin@\textsc{Sudermann, Hermann} (30.\,9.\,1857 Macikai – 21.\,11.\,1928 Berlin), \emph{Schriftsteller}|pw}} Romane}{\lemma{\textnormal{\emph{Sudermanns Romane}}}\Cendnote{\textnormal{Hermann Sudermann\pwindex{Sudermann, Hermann 30.\,9.\,1857 Macikai – 21.\,11.\,1928 Berlin@\textsc{Sudermann, Hermann} (30.\,9.\,1857 Macikai – 21.\,11.\,1928 Berlin), \emph{Schriftsteller}|pwk} wagte bereits in den
                     1870er-Jahren erste literarische Versuche,
                  veröffentlichte jedoch erst 1886 die Novellensammlung \emph{Im Zwielicht}\pwindex{Sudermann, Hermann 30.\,9.\,1857 Macikai – 21.\,11.\,1928 Berlin@\textsc{Sudermann, Hermann} (30.\,9.\,1857 Macikai – 21.\,11.\,1928 Berlin), \emph{Schriftsteller}!Im Zwielicht. Zwanglose Geschichten@\strich\emph{Im Zwielicht. Zwanglose Geschichten}|pwk} und 1887
                  seinen ersten Roman \emph{Frau Sorge}\pwindex{Sudermann, Hermann 30.\,9.\,1857 Macikai – 21.\,11.\,1928 Berlin@\textsc{Sudermann, Hermann} (30.\,9.\,1857 Macikai – 21.\,11.\,1928 Berlin), \emph{Schriftsteller}!Frau Sorge@\strich\emph{Frau Sorge}|pwk}. Einen großen
                  Erfolg feierte dann das am 29. 11. 1889 am \emph{Lessing-Theater}\orgindex{Lessing-Theater@Lessing-Theater|pwk} uraufgeführte Stück \emph{Die Ehre}\pwindex{Sudermann, Hermann 30.\,9.\,1857 Macikai – 21.\,11.\,1928 Berlin@\textsc{Sudermann, Hermann} (30.\,9.\,1857 Macikai – 21.\,11.\,1928 Berlin), \emph{Schriftsteller}!Ehre. Schauspiel in 4 Akten@\strich\emph{Die Ehre. Schauspiel in 4 Akten}|pwk}.}}}\label{K_L02712-10} haben{ }ſich Jahre lang
               unbeachtet herumgeſeilt, und jetzt kann man nicht genug davon kriegen. Alſo nur ein
               wenig Geduld, liebſter Freund, und Alles wird gehen. Eine \label{K_L02712-11v}\edtext{Aufführung\pwindex{Schnitzler, Arthur 15.\,5.\,1862 Wien – 21.\,10.\,1931 ebd.@\textsc{Schnitzler, Arthur} (15.\,5.\,1862 Wien – 21.\,10.\,1931 ebd.), \emph{Schriftsteller, Mediziner}!Märchen. Schauspiel in drei Aufzügen@\strich\emph{Das Märchen. Schauspiel in drei Aufzügen}|pwuv} im
                  Volkstheater\orgindex{Volkstheater@Volkstheater|pw}}{\lemma{\textnormal{\emph{Aufführung im Volkstheater}}}\Cendnote{\textnormal{Obzwar bislang von \emph{Anatol}\pwindex{Schnitzler, Arthur 15.\,5.\,1862 Wien – 21.\,10.\,1931 ebd.@\textsc{Schnitzler, Arthur} (15.\,5.\,1862 Wien – 21.\,10.\,1931 ebd.), \emph{Schriftsteller, Mediziner}!Anatol@\strich\emph{Anatol}|pwk} die Rede war, dürfte Goldmann\pwindex{Goldmann, Paul 31.\,1.\,1865 Breslau – 25.\,9.\,1935 Wien@\textsc{Goldmann, Paul} (31.\,1.\,1865 Breslau – 25.\,9.\,1935 Wien), \emph{Schriftsteller, Journalist}|pwk} nunmehr von \emph{Das
                     Märchen}\pwindex{Schnitzler, Arthur 15.\,5.\,1862 Wien – 21.\,10.\,1931 ebd.@\textsc{Schnitzler, Arthur} (15.\,5.\,1862 Wien – 21.\,10.\,1931 ebd.), \emph{Schriftsteller, Mediziner}!Märchen. Schauspiel in drei Aufzügen@\strich\emph{Das Märchen. Schauspiel in drei Aufzügen}|pwk} sprechen. Es wurde am 1. 9. 1893 vom \emph{Volkstheater}\orgindex{Volkstheater@Volkstheater|pwk} in Wien\oindex{Wien@\textbf{Wien}, \emph{Verwaltungsgebiet}|pwk} zur Aufführung
                  angenommen. Am 1. 12. 1893 kam es zur Uraufführung\eventindex{Volkstheater@\textbf{Volkstheater}!Uraufführung von Das Märchen, 1.12.1893@Uraufführung von Das Märchen, 1.12.1893|pwkv}.}}}\label{K_L02712-11} würde ich an Deiner
               Stelle nur annehmen, wenn das Stück\pwindex{Schnitzler, Arthur 15.\,5.\,1862 Wien – 21.\,10.\,1931 ebd.@\textsc{Schnitzler, Arthur} (15.\,5.\,1862 Wien – 21.\,10.\,1931 ebd.), \emph{Schriftsteller, Mediziner}!Märchen. Schauspiel in drei Aufzügen@\strich\emph{Das Märchen. Schauspiel in drei Aufzügen}|pwv} bereits in Deutſchland\oindex{Deutschland@\textbf{Deutschland}|pw} geſpielt
               wäre. Denn in \textsc{Wien\oindex{Wien@\textbf{Wien}, \emph{Verwaltungsgebiet}|pw}} zum überhaupt erſten Mal geſpielt zu werden, bei dieſer irrſinnig dummen Kritik
                  {\pb}und noch dazu in dieſem vollſtändig \label{K_L02712-12v}\edtext{unkünſtleriſch geleiteten Theater\orgindex{Volkstheater@Volkstheater|pwv}}{\lemma{\textnormal{\emph{unkünstlerisch … Theater}}}\Cendnote{\textnormal{Von 1889 bis
                     1905 war Emerich von
                     Bukovics\pwindex{Bukovics, Emerich von 28.\,2.\,1844 Wien – 4.\,7.\,1905 ebd.@\textsc{Bukovics, Emerich von} (28.\,2.\,1844 Wien – 4.\,7.\,1905 ebd.), \emph{Journalist, Theaterleiter}|pwk} Leiter des \emph{Volkstheaters}\orgindex{Volkstheater@Volkstheater|pwk}.}}}\label{K_L02712-12}, würde ich nicht für zuträglich halten. Die Hauptſache
               iſt, die Berlin\oindex{Berlin@\textbf{Berlin}, \emph{Hauptstadt}|pw}er Aufführung zu beſchleunigen,
               und auch darüber wollen wir gemeinſam Rath halten.\pend
           
\pstart
           Grüß’ Dich Gott, mein lieber Arthur! Auf hoffentlich baldiges Wiederſehen!\pend
           
\pstart
           Dein treuer {\\[\baselineskip]}\spacefill\mbox{Paul Goldmnn}\pend
           \leftskip=0em{}
\pstart
           Wenn Du es{ }ſo machen könnteſt, daß ich auch \textsc{Loris\pwindex{Hofmannsthal, Hugo von 1.\,2.\,1874 Wien – 15.\,7.\,1929 Rodaun@\textsc{Hofmannsthal, Hugo von} (1.\,2.\,1874 Wien – 15.\,7.\,1929 Rodaun), \emph{Schriftsteller}|pw}} und \textsc{Richard\pwindex{Beer-Hofmann, Richard 11.\,7.\,1866 Wien – 26.\,9.\,1945 New York City@\textsc{Beer-Hofmann, Richard} (11.\,7.\,1866 Wien – 26.\,9.\,1945 New York City), \emph{Schriftsteller}|pw}}{ }ſehe,{ }ſo wäre das ganz beſonders herrlich. \textsc{Loris\pwindex{Hofmannsthal, Hugo von 1.\,2.\,1874 Wien – 15.\,7.\,1929 Rodaun@\textsc{Hofmannsthal, Hugo von} (1.\,2.\,1874 Wien – 15.\,7.\,1929 Rodaun), \emph{Schriftsteller}|pw}} hat in der Frkf. Ztg.\pwindex{Frankfurter Zeitung@\emph{Frankfurter Zeitung}|pw} ein \label{K_L02712-13v}\edtext{stupendes Feuilleton\pwindex{Hofmannsthal, Hugo von 1.\,2.\,1874 Wien – 15.\,7.\,1929 Rodaun@\textsc{Hofmannsthal, Hugo von} (1.\,2.\,1874 Wien – 15.\,7.\,1929 Rodaun), \emph{Schriftsteller}!Gabriele d’Annunzio@\strich\emph{Gabriele d’Annunzio}|pwv}}{\lemma{\textnormal{\emph{stupendes Feuilleton}}}\Cendnote{\textnormal{Loris\pwindex{Hofmannsthal, Hugo von 1.\,2.\,1874 Wien – 15.\,7.\,1929 Rodaun@\textsc{Hofmannsthal, Hugo von} (1.\,2.\,1874 Wien – 15.\,7.\,1929 Rodaun), \emph{Schriftsteller}|pwk}: \emph{Gabriele d’Annunzio}\pwindex{Hofmannsthal, Hugo von 1.\,2.\,1874 Wien – 15.\,7.\,1929 Rodaun@\textsc{Hofmannsthal, Hugo von} (1.\,2.\,1874 Wien – 15.\,7.\,1929 Rodaun), \emph{Schriftsteller}!Gabriele d’Annunzio@\strich\emph{Gabriele d’Annunzio}|pwk}. In: \emph{Frankfurter Zeitung}\pwindex{Frankfurter Zeitung@\emph{Frankfurter Zeitung}|pwk}, Jg. 37, Nr. 219,
                        9. 8. 1893, Erstes Morgenblatt, S. 1–3. Darin erörterte
                     Hugo von Hofmannsthal\pwindex{Hofmannsthal, Hugo von 1.\,2.\,1874 Wien – 15.\,7.\,1929 Rodaun@\textsc{Hofmannsthal, Hugo von} (1.\,2.\,1874 Wien – 15.\,7.\,1929 Rodaun), \emph{Schriftsteller}|pwk} den Begriff der
                  (literarischen) »Moderne« am Beispiel von Gabriele d’Annunzio\pwindex{D’Annunzio, Gabriele 12.\,3.\,1863 Pescara – 1.\,3.\,1938 Cargnacco@\textsc{D’Annunzio, Gabriele} (12.\,3.\,1863 Pescara – 1.\,3.\,1938 Cargnacco), \emph{Schriftsteller}|pwk}. Goldmann\pwindex{Goldmann, Paul 31.\,1.\,1865 Breslau – 25.\,9.\,1935 Wien@\textsc{Goldmann, Paul} (31.\,1.\,1865 Breslau – 25.\,9.\,1935 Wien), \emph{Schriftsteller, Journalist}|pwk}
                  dürfte der Aufsatz\pwindex{Hofmannsthal, Hugo von 1.\,2.\,1874 Wien – 15.\,7.\,1929 Rodaun@\textsc{Hofmannsthal, Hugo von} (1.\,2.\,1874 Wien – 15.\,7.\,1929 Rodaun), \emph{Schriftsteller}!Gabriele d’Annunzio@\strich\emph{Gabriele d’Annunzio}|pwkv} vor
                  allem aufgrund der darin enthaltenen kontra-naturalistischen Ausführungen
                  missfallen haben.}}}\label{K_L02712-13} gehabt.\pend
           \selectlanguage{ngerman}\endnumbering\briefempfaengerindex{Schnitzler, Arthur@\textsc{Schnitzler, Arthur}!zzzGoldmann, Paul@\emph{von Paul Goldmann}!1893-08-182@{18. 8. [1893]}|)be}\mylabel{L02712h}  \newcommand{\dateiname}{L02712}\newcommand{\titel}{Paul Goldmann an Arthur Schnitzler, 18. 8. [1893]}\newcommand{\editorInnen}{Martin Anton Müller und Laura Untner}%% latex-leseansicht-abspann.tex
%% Abspann für die Leseansicht.
%% Der Schalter \ifkorrekturansicht ist bereits durch den Vorspann gesetzt.

%% latex-abspann.tex
%% Gemeinsamer Abspann für Korrekturansicht und Leseansicht.
%% Setzt den Schalter \ifkorrekturansicht voraus (gesetzt in den
%% einbindenden Dateien latex-korrekturansicht-abspann.tex bzw.
%% latex-leseansicht-abspann.tex).
%% ---------------------------------------------------------------

\normalsize

% Das esempio-Environment wird nur in der Leseansicht benötigt
\ifkorrekturansicht\else
\newenvironment{esempio}[3]%
{
    \vspace{1.5ex}
    \rlap{\underline{#1}}
    \par
    \setlength{\parindent}{0cm}
    \nopagebreak
    \leftskip=#2cm
    \rightskip=#3cm
}
{
    \par
}
\fi

\doendnotes{C}
\bigskip
\vfill

\clearpage

\footnotesize

\ifkorrekturansicht
  \lohead{\textsc{register}}
\fi

% theindex-Environment neu definieren ohne reledmac
\makeatletter
\renewenvironment{theindex}{%
  \ifkorrekturansicht
    \section*{\indexname}%
  \else
    \subsubsection*{Index der erwähnten Entitäten}%
  \fi
  \setlength{\parindent}{0pt}%
  \setlength{\parskip}{0pt plus 0.3pt}%
  \let\item\@idxitem
}{%
  \ifkorrekturansicht\clearpage\fi
}
\makeatother

\IfFileExists{\jobname-pw.ind}{\input{\jobname-pw.ind}}{}

% Quellenangabe nur in der Leseansicht
\ifkorrekturansicht\else
% Fallback-Definitionen, falls die .tex-Datei \titel etc. nicht gesetzt hat
\providecommand{\titel}{}
\providecommand{\editorInnen}{}
\providecommand{\dateiname}{\jobname}

\vspace{3cm}

\vfill

\footnotesize
\textsc{Quelle}: \titel. Herausgegeben von {\editorInnen}. In: \emph{Arthur Schnitzler: Briefwechsel mit Autorinnen und Autoren}.
 Digitale Edition, https://schnitzler-briefe.acdh.oeaw.ac.at/{\dateiname}.html (Stand \today)
\fi

\end{document}


