%% latex-korrekturansicht-vorspann.tex
%% Vorspann für die Korrekturansicht.
%% Lädt die gemeinsame Datei latex-vorspann.tex mit gesetztem Schalter.

\newif\ifkorrekturansicht
\korrekturansichttrue

\input{../tex-inputs/latex-vorspann}


\section[ Paul Goldmann an Arthur Schnitzler, 29. 6. 1903]{L03376 Paul Goldmann an Arthur Schnitzler, 29. 6. 1903}
\nopagebreak\mylabel{L03376v}
\rehead{ }\normalsize\beginnumbering\briefempfaengerindex{Schnitzler, Arthur@\textsc{Schnitzler, Arthur}!zzzGoldmann, Paul@\emph{von Paul Goldmann}!1903-06-291@{29. 6. 1903}|(be}
\toendnotes[C]{\smallbreak\pagebreak[2]}\Standort{DLA, A:Schnitzler, HS.NZ85.1.3173.}
\physDesc{Postkarte, 322 Zeichen
\newline{}Handschrift: 1) blaue Tinte, deutsche Kurrent\hspace{1em}2) blaue Tinte, lateinische Kurrent (\noindent{}Adresse)\hspace{1em}
\newline{}Versand: Stempel: »\nobreak{}\oindex{Berlin@\textbf{Berlin}, \emph{P.PPLC}|pwk}Berlin S. W. 11, 29. 6. 03., 4–5N.\nobreak{}«. Stempel: »\nobreak{}\oindex{IX., Alsergrund@\textbf{IX., Alsergrund}, \emph{A.ADM3}|pwk}9/3 Wien 72, 30. 6. 03, 9. V, Bestellt\nobreak{}«.  
\newline{}Schnitzler: mit Bleistift das Jahr »903« vermerkt }\toendnotes[C]{\smallbreak}\pstart{}{\pb}Herrn\pend{}\pstart{}Dr. Arthur Schnitzler\pend{}\pstart{}Wien\oindex{Wien@\textbf{Wien}, \emph{A.ADM2}|pw}\pend{}\pstart{}IX. Frankgaſse 1\oindex{Frankgasse 1@\textbf{Frankgasse 1}, \emph{Wohngebäude (K.WHS)}|pw}.\pend{}{\bigskip}\vspace{1em}
\pstart
           {\pb}Berlin\oindex{Berlin@\textbf{Berlin}, \emph{P.PPLC}|pw}, 29. Juni.\pend
           
\pstart{}Mein lieber Freund,\pend\vspace{0.5em}
\pstart
           Ich leſe eben, daß der »Reigen\pwindex{Reigen. Zehn Dialoge@\emph{Reigen. Zehn Dialoge}|pw}« bereits in
                  \label{K_L03376-1v}\edtext{achter Auflage}{\lemma{\textnormal{\emph{achter Auflage}}}\Cendnote{\textnormal{Das heißt zu dieser Zeit, dass von der Ausgabe\pwindex{Reigen. Zehn Dialoge@\emph{Reigen. Zehn Dialoge}|pwkv} nunmehr 8000 Stück
                  gedruckt waren. Vgl. A. S.: \emph{Tagebuch}, 28. 6. 1903.}}}\label{K_L03376-1} erſchienen iſt. Ich beglückwünſche Dich zu dem Rieſenerfolg dieſes Buch\pwindex{Reigen. Zehn Dialoge@\emph{Reigen. Zehn Dialoge}|pwv}es, welches auch für mich
               zum Feinſten und Reizendſten gehört, das Du geſchrieben haſt.\pend
           
\pstart
           Herzlichſt Dein {\\[\baselineskip]}\spacefill\mbox{Paul Goldmnn}\pend
           \leftskip=0em{}\selectlanguage{ngerman}\endnumbering\briefempfaengerindex{Schnitzler, Arthur@\textsc{Schnitzler, Arthur}!zzzGoldmann, Paul@\emph{von Paul Goldmann}!1903-06-291@{29. 6. 1903}|)be}\mylabel{L03376h}  \normalsize

\doendnotes{C}
\bigskip
\vfill

\clearpage

\footnotesize

\lohead{\textsc{register}}

% Definiere theindex-Environment komplett neu ohne reledmac
\makeatletter
\renewenvironment{theindex}{%
  \section*{\indexname}%
  \setlength{\parindent}{0pt}%
  \setlength{\parskip}{0pt plus 0.3pt}%
  \let\item\@idxitem
}{%
  \clearpage
}
\makeatother

\IfFileExists{\jobname-pw.ind}{\input{\jobname-pw.ind}}{}

\end{document}

      