%% latex-korrekturansicht-vorspann.tex
%% Vorspann für die Korrekturansicht.
%% Lädt die gemeinsame Datei latex-vorspann.tex mit gesetztem Schalter.

\newif\ifkorrekturansicht
\korrekturansichttrue

\input{../tex-inputs/latex-vorspann}


\section[ Felix Salten an Arthur Schnitzler, {[}22. 2. 1922?{]}]{L03595 Felix Salten an Arthur Schnitzler, {[}22. 2. 1922?{]}}
\nopagebreak\mylabel{L03595v}
\rehead{ }\normalsize\beginnumbering\briefempfaengerindex{Schnitzler, Arthur@\textsc{Schnitzler, Arthur}!zzzSalten, Felix@\emph{von Felix Salten}!1922-02-221@{{[}22. 2. 1922?{]}}|(be}
\toendnotes[C]{\smallbreak\pagebreak[2]}\Standort{CUL, Schnitzler, B 89, B 2.}
\physDesc{Brief, 1 Blatt, 1 Seite, 640 Zeichen
\newline{}Handschrift: Bleistift, lateinische Kurrent
\newline{}Schnitzler: mit Bleistift womöglich Vermerk der Jahreszahl: »/22« 
\newline{}Ordnung: mit Bleistift von unbekannter Hand beschriftet: »?« }\toendnotes[C]{\smallbreak}
\pstart
           \raggedleft{}{\pb}\label{K_L03595-1v}\edtext{Dienstag}{\lemma{\textnormal{\emph{Dienstag}}}\Cendnote{\textnormal{Die Datierung des undatierten Korrespondenzstücks
                        gelingt durch Annäherung. Der eine überlieferte Brief von Ellis O. Jones\pwindex{Jones, Ellis O. 1873-12-13 – 1967-08-01@\textsc{Jones, Ellis O.} (1873-12-13 – 1967-08-01), \emph{Journalist/Journalistin, Herausgeber/Herausgeberin, Aktivist/Aktivistin}|pwk} an Schnitzler (\emph{DLA Marbach}, HS.1985.1.3581) ist datiert mit
                        »Nov 8 1921« und in Berlin\oindex{Berlin@\textbf{Berlin}, \emph{P.PPLC}|pwk} abgefasst. Aus ihm geht hervor, dass dieser
                        darauf hoffte, Schnitzler persönlich
                        kennenzulernen. Damit kann der Zeitraum des vorliegenden Korrespondenzstücks
                        nach vorne hin eingegrenzt werden. Die Verknüpfung von
                        »Generalprobe«, »Dienstag« und dem unsicher
                        gelesenen »Claudel« kann ferner als Hinweis auf die Generalprobe
                        des Stücks \emph{Der Tausch}\pwindex{Tausch. Drama in drei Akten@\emph{Der Tausch. Drama in drei Akten}|pwk} am Mittwoch, dem
                        23. 2. 1921
                        gelesen werden, die sowohl Schnitzler als
                        auch Salten\pwindex{Salten, Felix 06.09.1869 – 08.10.1945@\textsc{Salten, Felix} (06.09.1869 – 08.10.1945), \emph{Schriftsteller/Schriftstellerin, Journalist/Journalistin, Chefredakteur/Chefredakteurin}|pwk} besuchte. Ein Besuch Saltens\pwindex{Salten, Felix 06.09.1869 – 08.10.1945@\textsc{Salten, Felix} (06.09.1869 – 08.10.1945), \emph{Schriftsteller/Schriftstellerin, Journalist/Journalistin, Chefredakteur/Chefredakteurin}|pwk}
                  am selben Abend ist jedoch nicht belegt.}}}\label{K_L03595-1}.\pend
           
\pstart{}Lieber,\pend\vspace{0.5em}
\pstart
           ein M\textsuperscript{r}{ }Ellis O. Jones\pwindex{Jones, Ellis O. 1873-12-13 – 1967-08-01@\textsc{Jones, Ellis O.} (1873-12-13 – 1967-08-01), \emph{Journalist/Journalistin, Herausgeber/Herausgeberin, Aktivist/Aktivistin}|pw}, amerikanischer\oindex{Vereinigte Staaten von Amerika [USA]@\textbf{Vereinigte Staaten von Amerika [USA]}, \emph{A.PCLI}|pw} Journalist, Vertreter des Foreign Press Service\orgindex{Foreign Press Service@Foreign Press Service|pw}, kommt heute{ }Nachmittag \uline{um 3h}, durch eine Bekannte eingeführt, um mich
               zu interviewen. Er will, \label{K_L03595-2v}\edtext{mit der
               gleichen Absicht, auch zu Ihnen}{\lemma{\textnormal{\emph{mit … Ihnen}}}\Cendnote{\textnormal{Weder
                  Besuch noch Interview können nachgewiesen werden.}}}\label{K_L03595-2}. Ich weiß \uline{garnichts} von ihm, kann ihn weder empfehlen noch
               einführen, sondern habe es nur übernommen, die Anfrage an Sie weiterzugeben.
               Vielleicht laßen Sie zu mir her Bescheid sagen, ob Sie Herrn Jones\pwindex{Jones, Ellis O. 1873-12-13 – 1967-08-01@\textsc{Jones, Ellis O.} (1873-12-13 – 1967-08-01), \emph{Journalist/Journalistin, Herausgeber/Herausgeberin, Aktivist/Aktivistin}|pw} überhaupt und ob Sie ihn dann, wenn er von mir
               fortgeht oder sonst wann empfangen wollen.\pend
           \pstart Herzlichst Ihr \spacefill\mbox{Salten}\pend{}
\pstart
           \noindent{}Mir interessantes:\pend
           
\pstart
           Sind Sie heute, etwa nach dem Nachtmahl, frei?\pend
           
\pstart
           Morgen in der Generalprobe von 
                  \textcolor{gray}{Claudel}\pwindex{Tausch. Drama in drei Akten@\emph{Der Tausch. Drama in drei Akten}|pwv}\pwindex{Claudel, Paul 06.08.1868 – 23.02.1955@\textsc{Claudel, Paul} (06.08.1868 – 23.02.1955), \emph{Schriftsteller/Schriftstellerin}|pw}?\pend
           \selectlanguage{ngerman}\endnumbering\briefempfaengerindex{Schnitzler, Arthur@\textsc{Schnitzler, Arthur}!zzzSalten, Felix@\emph{von Felix Salten}!1922-02-221@{{[}22. 2. 1922?{]}}|)be}\mylabel{L03595h}  \normalsize

\doendnotes{C}
\bigskip
\vfill

\clearpage

\footnotesize

\lohead{\textsc{register}}

% Definiere theindex-Environment komplett neu ohne reledmac
\makeatletter
\renewenvironment{theindex}{%
  \section*{\indexname}%
  \setlength{\parindent}{0pt}%
  \setlength{\parskip}{0pt plus 0.3pt}%
  \let\item\@idxitem
}{%
  \clearpage
}
\makeatother

\IfFileExists{\jobname-pw.ind}{\input{\jobname-pw.ind}}{}

\end{document}

      