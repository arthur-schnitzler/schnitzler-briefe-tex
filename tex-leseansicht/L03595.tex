%% latex-leseansicht-vorspann.tex
%% Vorspann für die Leseansicht.
%% Lädt die gemeinsame Datei latex-vorspann.tex mit nicht gesetztem Schalter.

\newif\ifkorrekturansicht
\korrekturansichtfalse

\input{../tex-inputs/latex-vorspann}


         
         \renewcommand{\erwaehntePersonen}{Personen: Paul Claudel, Ellis O. Jones, Felix Salten}
         \renewcommand{\erwaehnteInstitutionen}{Institutionen: Foreign Press Service}
         \renewcommand{\erwaehnteOrte}{Orte: Berlin, Vereinigte Staaten von Amerika (USA), Wien}
         \renewcommand{\erwaehnteWerke}{Werke: Der Tausch. Drama in drei Akten}
               \section[ Felix Salten an Arthur Schnitzler, {[}22. 2. 1922?{]}]{ Felix Salten an Arthur Schnitzler, {[}22. 2. 1922?{]}}\nopagebreak\mylabel{v}\rehead{ }\begin{ledgroupsized}[t]{13cm}\normalsize\beginnumbering\briefempfaengerindex{Schnitzler, Arthur@\textsc{Schnitzler, Arthur}!zzzSalten, Felix@\emph{von Felix Salten}!1922-02-221@{{[}22. 2. 1922?{]}}|(be} \toendnotes[C]{\smallbreak\pagebreak[2]} \Standort{CUL, Schnitzler, B 89, B 2.}
\physDesc{Brief, 1 Blatt, 1 Seite, 640 Zeichen
\newline{}Handschrift: Bleistift, lateinische Kurrent
\newline{}Schnitzler: mit Bleistift womöglich Vermerk der Jahreszahl: »/22« 
\newline{}Ordnung: mit Bleistift von unbekannter Hand beschriftet: »?« }\toendnotes[C]{\smallbreak}\pstart
           \raggedleft{}{\pb}\label{K_L03595-1v}\edtext{Dienstag}{\lemma{\textnormal{\emph{Dienstag}}}\Cendnote{\textnormal{Die Datierung des undatierten Korrespondenzstücks
                        gelingt durch Annäherung. Der eine überlieferte Brief von Ellis O. Jones\pwindex{Jones, Ellis O. 1873-12-13 – 1967-08-01@\textsc{Jones, Ellis O.} (1873-12-13 – 1967-08-01), \emph{Journalist, Herausgeber, Aktivist}|pwk} an Schnitzler\pwindex{Schnitzler, Arthur 15.05.1862 – 21.10.1931@\textsc{Schnitzler, Arthur} (15.05.1862 – 21.10.1931), \emph{Schriftsteller, Mediziner}|pwk} (\emph{DLA Marbach}, HS.1985.1.3581) ist datiert mit
                        »Nov 8 1921« und in Berlin\oindex{Berlin@\textbf{Berlin}|pwk} abgefasst. Aus ihm geht hervor, dass dieser
                        darauf hoffte, Schnitzler\pwindex{Schnitzler, Arthur 15.05.1862 – 21.10.1931@\textsc{Schnitzler, Arthur} (15.05.1862 – 21.10.1931), \emph{Schriftsteller, Mediziner}|pwk} persönlich
                        kennenzulernen. Damit kann der Zeitraum des vorliegenden Korrespondenzstücks
                        nach vorne hin eingegrenzt werden. Die Verknüpfung von
                        »Generalprobe«, »Dienstag« und dem unsicher
                        gelesenen »Claudel« kann ferner als Hinweis auf die Generalprobe
                        des Stücks \emph{Der Tausch}\pwindex{Claudel, Paul 06.08.1868 – 23.02.1955@\textsc{Claudel, Paul} (06.08.1868 – 23.02.1955), \emph{Schriftsteller}!Tausch. Drama in drei Akten1920@\strich\emph{Der Tausch. Drama in drei Akten} {[}1920{]}|pwk} am Mittwoch, dem
                        23. 2. 1921
                        gelesen werden, die sowohl Schnitzler\pwindex{Schnitzler, Arthur 15.05.1862 – 21.10.1931@\textsc{Schnitzler, Arthur} (15.05.1862 – 21.10.1931), \emph{Schriftsteller, Mediziner}|pwk} als
                        auch Salten\pwindex{Salten, Felix 06.09.1869 – 08.10.1945@\textsc{Salten, Felix} (06.09.1869 – 08.10.1945), \emph{Schriftsteller, Journalist, Chefredakteur}|pwk} besuchte. Ein Besuch Saltens\pwindex{Salten, Felix 06.09.1869 – 08.10.1945@\textsc{Salten, Felix} (06.09.1869 – 08.10.1945), \emph{Schriftsteller, Journalist, Chefredakteur}|pwk}
                  am selben Abend ist jedoch nicht belegt.}}}\label{K_L03595-1h}.\pend
           \pstart{}Lieber,\pend\pstart
           ein M\textsuperscript{r}{ }Ellis O. Jones\pwindex{Jones, Ellis O. 1873-12-13 – 1967-08-01@\textsc{Jones, Ellis O.} (1873-12-13 – 1967-08-01), \emph{Journalist, Herausgeber, Aktivist}|pw}, amerikanischer\oindex{Vereinigte Staaten von Amerika (USA)@\textbf{Vereinigte Staaten von Amerika (USA)}|pw} Journalist, Vertreter des Foreign Press Service\orgindex{Foreign Press Service@Foreign Press Service|pw}, kommt heute{ }Nachmittag \uline{um 3h}, durch eine Bekannte eingeführt, um mich
               zu interviewen. Er will, \label{K_L03595-2v}\edtext{mit der
               gleichen Absicht, auch zu Ihnen}{\lemma{\textnormal{\emph{mit … Ihnen}}}\Cendnote{\textnormal{Weder
                  Besuch noch Interview können nachgewiesen werden.}}}\label{K_L03595-2h}. Ich weiß \uline{garnichts} von ihm, kann ihn weder empfehlen noch
               einführen, sondern habe es nur übernommen, die Anfrage an Sie weiterzugeben.
               Vielleicht laßen Sie zu mir her Bescheid sagen, ob Sie Herrn Jones\pwindex{Jones, Ellis O. 1873-12-13 – 1967-08-01@\textsc{Jones, Ellis O.} (1873-12-13 – 1967-08-01), \emph{Journalist, Herausgeber, Aktivist}|pw} überhaupt und ob Sie ihn dann, wenn er von mir
               fortgeht oder sonst wann empfangen wollen.\pend
           \pstart Herzlichst Ihr \spacefill\mbox{Salten}\pend{}\pstart
           \noindent{}Mir interessantes:\pend
           \pstart
           Sind Sie heute, etwa nach dem Nachtmahl, frei?\pend
           \pstart
           Morgen in der Generalprobe von 
                  \textcolor{gray}{Claudel}\pwindex{Claudel, Paul 06.08.1868 – 23.02.1955@\textsc{Claudel, Paul} (06.08.1868 – 23.02.1955), \emph{Schriftsteller}!Tausch. Drama in drei Akten1920@\strich\emph{Der Tausch. Drama in drei Akten} {[}1920{]}|pwv}\pwindex{Claudel, Paul 06.08.1868 – 23.02.1955@\textsc{Claudel, Paul} (06.08.1868 – 23.02.1955), \emph{Schriftsteller}|pw}?\pend
           
         
         \endnumbering\mylabel{h}\end{ledgroupsized}  \newcommand{\dateiname}{L03595}\newcommand{\titel}{Felix Salten an Arthur Schnitzler, [22. 2. 1922?]}\newcommand{\editorInnen}{Martin Anton Müller und Laura Untner}%% latex-leseansicht-abspann.tex
%% Abspann für die Leseansicht.
%% Der Schalter \ifkorrekturansicht ist bereits durch den Vorspann gesetzt.

%% latex-abspann.tex
%% Gemeinsamer Abspann für Korrekturansicht und Leseansicht.
%% Setzt den Schalter \ifkorrekturansicht voraus (gesetzt in den
%% einbindenden Dateien latex-korrekturansicht-abspann.tex bzw.
%% latex-leseansicht-abspann.tex).
%% ---------------------------------------------------------------

\normalsize

% Das esempio-Environment wird nur in der Leseansicht benötigt
\ifkorrekturansicht\else
\newenvironment{esempio}[3]%
{
    \vspace{1.5ex}
    \rlap{\underline{#1}}
    \par
    \setlength{\parindent}{0cm}
    \nopagebreak
    \leftskip=#2cm
    \rightskip=#3cm
}
{
    \par
}
\fi

\doendnotes{C}
\bigskip
\vfill

\clearpage

\footnotesize

\ifkorrekturansicht
  \lohead{\textsc{register}}
\fi

% theindex-Environment neu definieren ohne reledmac
\makeatletter
\renewenvironment{theindex}{%
  \ifkorrekturansicht
    \section*{\indexname}%
  \else
    \subsubsection*{Index der erwähnten Entitäten}%
  \fi
  \setlength{\parindent}{0pt}%
  \setlength{\parskip}{0pt plus 0.3pt}%
  \let\item\@idxitem
}{%
  \ifkorrekturansicht\clearpage\fi
}
\makeatother

\IfFileExists{\jobname-pw.ind}{\input{\jobname-pw.ind}}{}

% Quellenangabe nur in der Leseansicht
\ifkorrekturansicht\else
% Fallback-Definitionen, falls die .tex-Datei \titel etc. nicht gesetzt hat
\providecommand{\titel}{}
\providecommand{\editorInnen}{}
\providecommand{\dateiname}{\jobname}

\vspace{3cm}

\vfill

\footnotesize
\textsc{Quelle}: \titel. Herausgegeben von {\editorInnen}. In: \emph{Arthur Schnitzler: Briefwechsel mit Autorinnen und Autoren}.
 Digitale Edition, https://schnitzler-briefe.acdh.oeaw.ac.at/{\dateiname}.html (Stand \today)
\fi

\end{document}


      