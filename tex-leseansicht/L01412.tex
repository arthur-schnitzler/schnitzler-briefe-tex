%% latex-leseansicht-vorspann.tex
%% Vorspann für die Leseansicht.
%% Lädt die gemeinsame Datei latex-vorspann.tex mit nicht gesetztem Schalter.

\newif\ifkorrekturansicht
\korrekturansichtfalse

\input{../tex-inputs/latex-vorspann}


         \renewcommand{\erwaehnteWerke}{}
               \section[Arthur Schnitzler an Hugo von Hofmannsthal, 30. 6. 1904]{ Arthur Schnitzler an Hugo von Hofmannsthal, 30. 6. 1904}\nopagebreak\mylabel{v}\rehead{ }\begin{ledgroupsized}[t]{13cm}\normalsize\beginnumbering \toendnotes[C]{\smallbreak\pagebreak[2]} \Standort{FDH, Hs-30885,109.}
\physDesc{Kartenbrief
\newline{}Handschrift: schwarze Tinte, deutsche Kurrent\newline{}Versand: 1) Stempel: »\nobreak{}\oindex{XXXX Ortsangabe fehlt|pwk}18/1 Wien, 3{[}0. 6. 1904{]}\nobreak{}«.   2) Stempel: »\nobreak{}\oindex{XXXX Ortsangabe fehlt|pwk}Rodaun, 1{[}. 7. 1904{]}\nobreak{}«. }\buchAbdrucke{\weitereDrucke{Hugo von Hofmannsthal, Arthur Schnitzler: \emph{Briefwechsel}. Hg. Therese Nickl und Heinrich Schnitzler. Frankfurt am Main: \emph{S. Fischer} 1964, S. 190.} }\toendnotes[C]{\smallbreak}\pstart{}{\pb}Herrn Dr Hugo von Hofmannsthal\pend{}\pstart{}\textsc{\label{K_L01412_1v}\edtext{Rodaun\oindex{XXXX Ortsangabe fehlt|pw}}{\lemma{\textnormal{\emph{Rodaun}}}\Cendnote{\textnormal{Schnitzler\pwindex{\textcolor{red}{\textsuperscript{XXXX1 indx}}|pwk} begann die Zeile mit einem
                        »W«, das von einem »R« überschrieben wurde. Zur Sicherheit schrieb er am
                        oberen Rand noch einmal »Rodaun\oindex{XXXX Ortsangabe fehlt|pw}«.}}}\label{K_L01412_1h}}\pend{}\pstart{}\textsc{bei Liesing\oindex{XXXX Ortsangabe fehlt|pw}}\pend{}{\bigskip}\pstart
           \raggedleft{}{\pb}30. 6. 904\pend
           \pstart
           mein lieber Hugo, es geht mir noch recht gelb aber doch im ganzen
               beſſer, daſs Sie bald kommen wollen, iſt ſehr lieb, ich ſchlage Ihnen z. B. vor
                  Mittwoch{ }Mittag bei uns zu ſpeiſen, vielleicht ka{\geminationn}
               ich da auch ſchon ein bischen ſpaziren gehen. Für die »Kunſt\textcolor{red}{\textsuperscript{XXXX indx}}« ſchönen Dank. Antworten Sie recht bald. Auch jeder andre Tag geht
               natürlich.\pend
           \pstart
           Herzlichſt{\\[\baselineskip]}Ihr \spacefill\mbox{A.}\pend
           \leftskip=0em{}
         
         \endnumbering\mylabel{h}\end{ledgroupsized}  \newcommand{\dateiname}{L01412}\newcommand{\titel}{Arthur Schnitzler an Hugo von Hofmannsthal, 30. 6. 1904}\newcommand{\editorInnen}{Martin Anton Müller und Gerd-Hermann Susen}%% latex-leseansicht-abspann.tex
%% Abspann für die Leseansicht.
%% Der Schalter \ifkorrekturansicht ist bereits durch den Vorspann gesetzt.

%% latex-abspann.tex
%% Gemeinsamer Abspann für Korrekturansicht und Leseansicht.
%% Setzt den Schalter \ifkorrekturansicht voraus (gesetzt in den
%% einbindenden Dateien latex-korrekturansicht-abspann.tex bzw.
%% latex-leseansicht-abspann.tex).
%% ---------------------------------------------------------------

\normalsize

% Das esempio-Environment wird nur in der Leseansicht benötigt
\ifkorrekturansicht\else
\newenvironment{esempio}[3]%
{
    \vspace{1.5ex}
    \rlap{\underline{#1}}
    \par
    \setlength{\parindent}{0cm}
    \nopagebreak
    \leftskip=#2cm
    \rightskip=#3cm
}
{
    \par
}
\fi

\doendnotes{C}
\bigskip
\vfill

\clearpage

\footnotesize

\ifkorrekturansicht
  \lohead{\textsc{register}}
\fi

% theindex-Environment neu definieren ohne reledmac
\makeatletter
\renewenvironment{theindex}{%
  \ifkorrekturansicht
    \section*{\indexname}%
  \else
    \subsubsection*{Index der erwähnten Entitäten}%
  \fi
  \setlength{\parindent}{0pt}%
  \setlength{\parskip}{0pt plus 0.3pt}%
  \let\item\@idxitem
}{%
  \ifkorrekturansicht\clearpage\fi
}
\makeatother

\IfFileExists{\jobname-pw.ind}{\input{\jobname-pw.ind}}{}

% Quellenangabe nur in der Leseansicht
\ifkorrekturansicht\else
% Fallback-Definitionen, falls die .tex-Datei \titel etc. nicht gesetzt hat
\providecommand{\titel}{}
\providecommand{\editorInnen}{}
\providecommand{\dateiname}{\jobname}

\vspace{3cm}

\vfill

\footnotesize
\textsc{Quelle}: \titel. Herausgegeben von {\editorInnen}. In: \emph{Arthur Schnitzler: Briefwechsel mit Autorinnen und Autoren}.
 Digitale Edition, https://schnitzler-briefe.acdh.oeaw.ac.at/{\dateiname}.html (Stand \today)
\fi

\end{document}


      