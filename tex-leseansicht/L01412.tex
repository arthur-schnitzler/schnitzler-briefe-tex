%% latex-korrekturansicht-vorspann.tex
%% Vorspann für die Korrekturansicht.
%% Lädt die gemeinsame Datei latex-vorspann.tex mit gesetztem Schalter.

\newif\ifkorrekturansicht
\korrekturansichttrue

\input{../tex-inputs/latex-vorspann}


\section[Arthur Schnitzler an Hugo von Hofmannsthal, 30. 6. 1904]{L01412 Arthur Schnitzler an Hugo von Hofmannsthal, 30. 6. 1904}
\nopagebreak\mylabel{L01412v}
\rehead{ }\normalsize\beginnumbering\briefempfaengerindex{Hofmannsthal, Hugo von@\textsc{Hofmannsthal, Hugo von}!zzzSchnitzler, Arthur@\emph{von Arthur Schnitzler}!1904-06-301@{30. 6. 1904}|(be}
\toendnotes[C]{\smallbreak\pagebreak[2]}\Standort{FDH, Hs-30885,109.}
\physDesc{Kartenbrief, 402 Zeichen
\newline{}Handschrift: schwarze Tinte, deutsche Kurrent
\newline{}Versand: 1) Stempel: »\nobreak{}\oindex{XVIII., Waehring@\textbf{XVIII., Währing}, \emph{A.ADM3}|pwk}18/1 Wien, 3{[}0. 6. 1904{]}\nobreak{}«.   2) Stempel: »\nobreak{}\oindex{Rodaun@\textbf{Rodaun}, \emph{A.ADM4}|pwk}Rodaun, 1{[}. 7. 1904{]}\nobreak{}«. }
\buchAbdrucke{\weitereDrucke{Hugo von Hofmannsthal, Arthur Schnitzler: \emph{Briefwechsel}. Frankfurt am Main: \emph{S. Fischer} 1964, S. 190.} }\toendnotes[C]{\smallbreak}\pstart{}{\pb}Herrn Dr Hugo von Hofmannsthal\pend{}\pstart{}\textsc{\label{K_L01412-1v}\edtext{Rodaun\oindex{Rodaun@\textbf{Rodaun}, \emph{A.ADM4}|pw}}{\lemma{\textnormal{\emph{Rodaun}}}\Cendnote{\textnormal{Schnitzler begann die Zeile mit einem
                           »W«, das von einem »R« überschrieben wurde.
                        Zur Sicherheit schrieb er am oberen Rand noch einmal »Rodaun\oindex{Rodaun@\textbf{Rodaun}, \emph{A.ADM4}|pw}«.}}}\label{K_L01412-1}}\pend{}\pstart{}\textsc{bei Liesing\oindex{XXIII., Liesing@\textbf{XXIII., Liesing}, \emph{A.ADM3}|pw}}\pend{}{\bigskip}\vspace{1em}
\pstart
           \raggedleft{}{\pb}30. 6. 904\pend
           \vspace{0.5em}
\pstart
           mein lieber Hugo, es geht mir noch recht gelb aber doch im ganzen
               beſſer, daſs Sie bald kommen wollen, iſt ſehr lieb, ich ſchlage Ihnen z. B. vor
                  Mittwoch{ }Mittag bei uns zu ſpeiſen, vielleicht ka{\geminationn}
               ich da auch ſchon ein bischen ſpaziren gehen. Für die »Kunſt\pwindex{Kunst und Kuenstler@\emph{Kunst und Künstler}|pw}« ſchönen Dank. Antworten Sie recht bald. Auch jeder andre
               Tag geht natürlich.\pend
           
\pstart
           Herzlichſt{\\[\baselineskip]}Ihr \spacefill\mbox{A.}\pend
           \leftskip=0em{}\selectlanguage{ngerman}\endnumbering\briefempfaengerindex{Hofmannsthal, Hugo von@\textsc{Hofmannsthal, Hugo von}!zzzSchnitzler, Arthur@\emph{von Arthur Schnitzler}!1904-06-301@{30. 6. 1904}|)be}\mylabel{L01412h}  \normalsize

\doendnotes{C}
\bigskip
\vfill

\clearpage

\footnotesize

\lohead{\textsc{register}}

% Definiere theindex-Environment komplett neu ohne reledmac
\makeatletter
\renewenvironment{theindex}{%
  \section*{\indexname}%
  \setlength{\parindent}{0pt}%
  \setlength{\parskip}{0pt plus 0.3pt}%
  \let\item\@idxitem
}{%
  \clearpage
}
\makeatother

\IfFileExists{\jobname-pw.ind}{\input{\jobname-pw.ind}}{}

\end{document}

      