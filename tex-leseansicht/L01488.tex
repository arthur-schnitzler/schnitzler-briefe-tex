%% latex-korrekturansicht-vorspann.tex
%% Vorspann für die Korrekturansicht.
%% Lädt die gemeinsame Datei latex-vorspann.tex mit gesetztem Schalter.

\newif\ifkorrekturansicht
\korrekturansichttrue

\input{../tex-inputs/latex-vorspann}


\section[Arthur Schnitzler an Hugo von Hofmannsthal, 31. 12. 1904]{L01488 Arthur Schnitzler an Hugo von Hofmannsthal, 31. 12. 1904}
\nopagebreak\mylabel{L01488v}
\rehead{ }\normalsize\beginnumbering\briefempfaengerindex{Hofmannsthal, Hugo von@\textsc{Hofmannsthal, Hugo von}!zzzSchnitzler, Arthur@\emph{von Arthur Schnitzler}!1904-12-311@{31. 12. 1904}|(be}
\toendnotes[C]{\smallbreak\pagebreak[2]}\Standort{FDH, Hs-30885,119.}
\physDesc{Brief, 2 Blätter, 5 Seiten, 2206 Zeichen
\newline{}Handschrift: schwarze Tinte, deutsche Kurrent}
\buchAbdrucke{\weitereDrucke{Hugo von Hofmannsthal, Arthur Schnitzler: \emph{Briefwechsel}. Frankfurt am Main: \emph{S. Fischer} 1964, S. 209.} }\toendnotes[C]{\smallbreak}
\pstart
           \raggedleft{}{\pb}Wien\oindex{Wien@\textbf{Wien}, \emph{A.ADM2}|pw}, 31. 12. 904.\pend
           
\pstart{}lieber Hugo, \pend\vspace{0.5em}
\pstart
           ich habe Grunwald\pwindex{Grunwald, Willy 14.02.1870 – 1945-05-08@\textsc{Grunwald, Willy} (14.02.1870 – 1945-05-08), \emph{Theaterleiter/Theaterleiterin, Schauspieler/Schauspielerin}|pw} in Traumulus\pwindex{Traumulus@\emph{Traumulus}|pw} als problematiſchen Corpsſtudenten, in der Frau vom Meer\pwindex{Frau vom Meer. Schauspiel in fuenf Akten@\emph{Die Frau vom Meer. Schauspiel in fünf Akten}|pw} als Lyngſtrand\pwindex{Frau vom Meer. Schauspiel in fuenf Akten@\emph{Die Frau vom Meer. Schauspiel in fünf Akten}|pwv} und da{\geminationn} im Geyer\pwindex{Florian Geyer. Die Tragoedie des Bauernkrieges@\emph{Florian Geyer. Die Tragödie des Bauernkrieges}|pw} als {\dots} ich weiſs
               nicht mehr was geſehen, und Brahm\pwindex{Brahm, Otto 05.02.1856 – 28.11.1912@\textsc{Brahm, Otto} (05.02.1856 – 28.11.1912), \emph{Theaterleiter/Theaterleiterin, Regisseur/Regisseurin}|pw} weiſs, daſs
               ich ihn ſehr ſchätze und noch allerlei Möglichkeiten in ihm zu ſpüren glaube. Er iſt
               aber gewiſs keine ſehr reiche und keine ſehr ſtarke Natur und hat auch das
               geheimnisvolle nicht, das manche haben, ohne stark {\pb}und
               groß zu ſein; er iſt ſehr ſcharf umriſſen aber es iſt nicht viel Luft um ihn. Nun
               ſcheint es mir aber für den Jaffier\pwindex{gerettete Venedig. Trauerspiel in fuenf Aufzuegen@\emph{Das gerettete Venedig. Trauerspiel in fünf Aufzügen}|pwv} notwendig, daſs man in ſeiner Perſönlichkeit den vergang\damage{ene}n Zauber ahnt und ich glaube, ſo etwas überzeugend herauszubringen, ist \strikeout{dichteriſch}{ }ſchauſpieleriſch ebenſo ſchwer, ja an der Grenze
               des Möglichen wie dichteriſch. Ihnen iſt es nur dadurch (und doch nicht ganz)
               gelungen, daſs Sie zwei in ihrer Art außerordentliche Menſchen, den \textsc{Pierre}\pwindex{gerettete Venedig. Trauerspiel in fuenf Aufzuegen@\emph{Das gerettete Venedig. Trauerspiel in fünf Aufzügen}|pwv} und die \textsc{Belvidera}\pwindex{gerettete Venedig. Trauerspiel in fuenf Aufzuegen@\emph{Das gerettete Venedig. Trauerspiel in fünf Aufzügen}|pwv}, {\pb}einen, deſſen Weſen Muth, die andere, deren
               Weſen Hingebung, noch zu einer Zeit unter jenem Zauber ſtehen laſſen, da wir nichts
               mehr \substVorne{}\textsuperscript{davon be}\substDazwischen{}von ihm\substHinten{} angerührt werden – aber immerhin\textcolor{gray}{,} wir denken: Muſs das
               ein Kerl geweſen ſein – daſs die zwei gar nicht merken, wie wenig er es heute iſt! –
                  Mitterwurzer\pwindex{Mitterwurzer, Friedrich 16.10.1844 – 13.02.1897@\textsc{Mitterwurzer, Friedrich} (16.10.1844 – 13.02.1897), \emph{Schauspieler/Schauspielerin}|pw}, Kainz\pwindex{Kainz, Josef 02.01.1858 – 20.09.1910@\textsc{Kainz, Josef} (02.01.1858 – 20.09.1910), \emph{Schauspieler/Schauspielerin}|pw}, Baſſermann\pwindex{Bassermann, Albert 07.09.1867 – 15.05.1952@\textsc{Bassermann, Albert} (07.09.1867 – 15.05.1952), \emph{Schauspieler/Schauspielerin}|pw} wieder
               trügen dieſes »geweſene« wie einen Heiligenſchein von verſtäubten Schickſalen um ihr
               Haupt, einen Schein, der eben nur in Perſönlichkeitsatmosphäre ſichtbar {\pb}wird. Davon, mein ich, wird bei Grunwald\pwindex{Grunwald, Willy 14.02.1870 – 1945-05-08@\textsc{Grunwald, Willy} (14.02.1870 – 1945-05-08), \emph{Theaterleiter/Theaterleiterin, Schauspieler/Schauspielerin}|pw} nichts merklich ſein. Warum ich Ihnen das ſage weiſs
               ich eigentlich nicht – denn wenn \textsc{Bassermann}\pwindex{Bassermann, Albert 07.09.1867 – 15.05.1952@\textsc{Bassermann, Albert} (07.09.1867 – 15.05.1952), \emph{Schauspieler/Schauspielerin}|pw} abſolut nicht will, iſt G.\pwindex{Grunwald, Willy 14.02.1870 – 1945-05-08@\textsc{Grunwald, Willy} (14.02.1870 – 1945-05-08), \emph{Theaterleiter/Theaterleiterin, Schauspieler/Schauspielerin}|pw} gewiſs der
               einzige, der in Betracht kommt. Er wird ſetze ich voraus, die Rolle von der weibiſch
                  \strikeout{ja} – verwöhnten Seite her zu nehmen ſuchen, \strikeout{und als} ja, er wird vielleicht auch das hyſteriſch
               verlogene (es iſt eine Bezeichnung, kein Schimpf) in \substVorne{}\textsuperscript{\textcolor{gray}{×}\-\textcolor{gray}{×}\-\textcolor{gray}{×}\-\textcolor{gray}{×}\-\textcolor{gray}{×}\-\textcolor{gray}{×}\-\textcolor{gray}{×}\-\textcolor{gray}{×}\-\textcolor{gray}{×}\-\textcolor{gray}{×}}\substDazwischen{}lebhafterer\substHinten{} Weiſe herausbringen, als Sie wollten. Wie immer, – es {\pb}wird durch dieſe Beſetzung \strikeout{noch} mehr als je die Tragoedie von der Enttäuſchung des Pierre\pwindex{gerettete Venedig. Trauerspiel in fuenf Aufzuegen@\emph{Das gerettete Venedig. Trauerspiel in fünf Aufzügen}|pwv}, und vielleicht ko{\geminationm}t nun alles bei der Einſtudierg darauf an, mit dieſem
               Gleichgewichtsverhältnis von vornherein zu rechnen.\pend
           
\pstart
           Sie haben doch nun meine Karte aus Lueg\oindex{Lueg@\textbf{Lueg}, \emph{Teil eines besiedelten Ortes (A.BSOX)}|pw} bekommen?
               Wir ſind alſo Montag 2.{ }Abends 8{ }Hietzing, \textsc{Kuffner}\oindex{Ottakringer Braeu@\textbf{Ottakringer Bräu}, \emph{Bierhaus (K.BIR)}|pw}. Vielleicht iſt unſer \textsc{Charolais}\pwindex{Beer-Hofmann, Richard 1866-07-11 – 1945-09-26@\textsc{Beer-Hofmann, Richard} (1866-07-11 – 1945-09-26), \emph{Schriftsteller/Schriftstellerin}|pwv}\pwindex{Graf von Charolais. Ein Trauerspiel@\emph{Der Graf von Charolais. Ein Trauerspiel}|pw} doch ſchon hier und kommt?\pend
           
\pstart
           Herzlichſt Ihr{\\[\baselineskip]}\spacefill\mbox{A.}\pend
           \leftskip=0em{}\selectlanguage{ngerman}\endnumbering\briefempfaengerindex{Hofmannsthal, Hugo von@\textsc{Hofmannsthal, Hugo von}!zzzSchnitzler, Arthur@\emph{von Arthur Schnitzler}!1904-12-311@{31. 12. 1904}|)be}\mylabel{L01488h}  \normalsize

\doendnotes{C}
\bigskip
\vfill

\clearpage

\footnotesize

\lohead{\textsc{register}}

% Definiere theindex-Environment komplett neu ohne reledmac
\makeatletter
\renewenvironment{theindex}{%
  \section*{\indexname}%
  \setlength{\parindent}{0pt}%
  \setlength{\parskip}{0pt plus 0.3pt}%
  \let\item\@idxitem
}{%
  \clearpage
}
\makeatother

\IfFileExists{\jobname-pw.ind}{\input{\jobname-pw.ind}}{}

\end{document}

      