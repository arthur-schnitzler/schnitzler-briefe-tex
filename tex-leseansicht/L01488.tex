%% latex-leseansicht-vorspann.tex
%% Vorspann für die Leseansicht.
%% Lädt die gemeinsame Datei latex-vorspann.tex mit nicht gesetztem Schalter.

\newif\ifkorrekturansicht
\korrekturansichtfalse

\input{../tex-inputs/latex-vorspann}


\section[Arthur Schnitzler an Hugo von Hofmannsthal, 31. 12. 1904]{L01488 Arthur Schnitzler an Hugo von Hofmannsthal, 31. 12. 1904}
\nopagebreak\mylabel{L01488v}
\rehead{ }\normalsize\beginnumbering\briefempfaengerindex{Hofmannsthal, Hugo von@\textsc{Hofmannsthal, Hugo von}!zzzSchnitzler, Arthur@\emph{von Arthur Schnitzler}!1904-12-311@{31. 12. 1904}|(be}
\toendnotes[C]{\smallbreak\pagebreak[2]}
\correspDesc{Versand  durch Arthur Schnitzler am 31. 12. 1904 in Wien
\newline{}Erhalt  durch Hugo von Hofmannsthal im Zeitraum [31. 12. 1904 – 4. 1. 1905?] in Wien}\toendnotes[C]{\smallbreak}
\Standort{FDH, Hs-30885,119.}
\physDesc{Brief, 2 Blätter, 5 Seiten, 2206 Zeichen
\newline{}Handschrift: schwarze Tinte, deutsche Kurrent}
\buchAbdrucke{\weitereDrucke{Hugo von Hofmannsthal, Arthur Schnitzler: \emph{Briefwechsel}. Herausgegeben von Therese Nickl und Heinrich Schnitzler. Frankfurt am Main: \emph{S. Fischer} 1964, S. 209.} }\toendnotes[C]{\smallbreak}
\pstart
           \raggedleft{}{\pb}Wien\oindex{Wien@\textbf{Wien}, \emph{Verwaltungsgebiet}|pw}, 31. 12. 904.\pend
           
\pstart{}lieber Hugo,\pend\vspace{0.5em}
\pstart
           ich habe Grunwald\pwindex{Grunwald, Willy 14.\,2.\,1870 Lingen – 8.\,5.\,1945 Berlin@\textsc{Grunwald, Willy} (14.\,2.\,1870 Lingen – 8.\,5.\,1945 Berlin), \emph{Theaterleiter, Schauspieler}|pw} in Traumulus\pwindex{\textcolor{red}{\textsuperscript{XXXX indx1}}!Traumulus. Tragische Komödie@\strich\emph{Traumulus. Tragische Komödie}|pw}\pwindex{\textcolor{red}{\textsuperscript{XXXX indx1}}!Traumulus. Tragische Komödie@\strich\emph{Traumulus. Tragische Komödie}|pw} als problematiſchen Corpsſtudenten, in der Frau vom Meer\pwindex{\textcolor{red}{\textsuperscript{XXXX indx1}}!Frau vom Meer. Schauspiel in fünf Akten@\strich\emph{Die Frau vom Meer. Schauspiel in fünf Akten}|pw} als Lyngſtrand\pwindex{\textcolor{red}{\textsuperscript{XXXX indx1}}!Frau vom Meer. Schauspiel in fünf Akten@\strich\emph{Die Frau vom Meer. Schauspiel in fünf Akten}|pwv} und da{\geminationn} im Geyer\pwindex{\textcolor{red}{\textsuperscript{XXXX indx1}}!Florian Geyer. Die Tragödie des Bauernkrieges@\strich\emph{Florian Geyer. Die Tragödie des Bauernkrieges}|pw} als {\dots} ich weiſs
               nicht mehr was geſehen, und Brahm\pwindex{Brahm, Otto 5.\,2.\,1856 Hamburg – 28.\,11.\,1912 Berlin@\textsc{Brahm, Otto} (5.\,2.\,1856 Hamburg – 28.\,11.\,1912 Berlin), \emph{Theaterleiter, Regisseur}|pw} weiſs, daſs
               ich ihn{ }ſehr{ }ſchätze und noch allerlei Möglichkeiten in ihm zu{ }ſpüren glaube. Er iſt
               aber gewiſs keine{ }ſehr reiche und keine{ }ſehr{ }ſtarke Natur und hat auch das
               geheimnisvolle nicht, das manche haben, ohne stark {\pb}und
               groß zu{ }ſein; er iſt{ }ſehr{ }ſcharf umriſſen aber es iſt nicht viel Luft um ihn. Nun{ }ſcheint es mir aber für den Jaffier\pwindex{Hofmannsthal, Hugo von 1.\,2.\,1874 Wien – 15.\,7.\,1929 Rodaun@\textsc{Hofmannsthal, Hugo von} (1.\,2.\,1874 Wien – 15.\,7.\,1929 Rodaun), \emph{Schriftsteller}!gerettete Venedig. Trauerspiel in fünf Aufzügen@\strich\emph{Das gerettete Venedig. Trauerspiel in fünf Aufzügen}|pwv} notwendig, daſs man in{ }ſeiner Perſönlichkeit den vergang\damage{ene}n Zauber ahnt und ich glaube,{ }ſo etwas überzeugend herauszubringen, ist \strikeout{dichteriſch}{ }ſchauſpieleriſch ebenſo{ }ſchwer, ja an der Grenze
               des Möglichen wie dichteriſch. Ihnen iſt es nur dadurch (und doch nicht ganz)
               gelungen, daſs Sie zwei in ihrer Art außerordentliche Menſchen, den \textsc{Pierre}\pwindex{Hofmannsthal, Hugo von 1.\,2.\,1874 Wien – 15.\,7.\,1929 Rodaun@\textsc{Hofmannsthal, Hugo von} (1.\,2.\,1874 Wien – 15.\,7.\,1929 Rodaun), \emph{Schriftsteller}!gerettete Venedig. Trauerspiel in fünf Aufzügen@\strich\emph{Das gerettete Venedig. Trauerspiel in fünf Aufzügen}|pwv} und die \textsc{Belvidera}\pwindex{Hofmannsthal, Hugo von 1.\,2.\,1874 Wien – 15.\,7.\,1929 Rodaun@\textsc{Hofmannsthal, Hugo von} (1.\,2.\,1874 Wien – 15.\,7.\,1929 Rodaun), \emph{Schriftsteller}!gerettete Venedig. Trauerspiel in fünf Aufzügen@\strich\emph{Das gerettete Venedig. Trauerspiel in fünf Aufzügen}|pwv}, {\pb}einen, deſſen Weſen Muth, die andere, deren
               Weſen Hingebung, noch zu einer Zeit unter jenem Zauber{ }ſtehen laſſen, da wir nichts
               mehr \substVorne{}\textsuperscript{davon be}\substDazwischen{}von ihm\substHinten{} angerührt werden – aber immerhin\textcolor{gray}{,} wir denken: Muſs das
               ein Kerl geweſen{ }ſein – daſs die zwei gar nicht merken, wie wenig er es heute iſt! –
                  Mitterwurzer\pwindex{Mitterwurzer, Friedrich 16.\,10.\,1844 Dresden – 13.\,2.\,1897 Wien@\textsc{Mitterwurzer, Friedrich} (16.\,10.\,1844 Dresden – 13.\,2.\,1897 Wien), \emph{Schauspieler}|pw}, Kainz\pwindex{Kainz, Josef 2.\,1.\,1858 Mosonmagyaróvár – 20.\,9.\,1910 Wien@\textsc{Kainz, Josef} (2.\,1.\,1858 Mosonmagyaróvár – 20.\,9.\,1910 Wien), \emph{Schauspieler}|pw}, Baſſermann\pwindex{Bassermann, Albert 7.\,9.\,1867 Mannheim – 15.\,5.\,1952 Atlantischer Ozean@\textsc{Bassermann, Albert} (7.\,9.\,1867 Mannheim – 15.\,5.\,1952 Atlantischer Ozean), \emph{Schauspieler}|pw} wieder
               trügen dieſes »geweſene« wie einen Heiligenſchein von verſtäubten Schickſalen um ihr
               Haupt, einen Schein, der eben nur in Perſönlichkeitsatmosphäre{ }ſichtbar {\pb}wird. Davon, mein ich, wird bei Grunwald\pwindex{Grunwald, Willy 14.\,2.\,1870 Lingen – 8.\,5.\,1945 Berlin@\textsc{Grunwald, Willy} (14.\,2.\,1870 Lingen – 8.\,5.\,1945 Berlin), \emph{Theaterleiter, Schauspieler}|pw} nichts merklich{ }ſein. Warum ich Ihnen das{ }ſage weiſs
               ich eigentlich nicht – denn wenn \textsc{Bassermann}\pwindex{Bassermann, Albert 7.\,9.\,1867 Mannheim – 15.\,5.\,1952 Atlantischer Ozean@\textsc{Bassermann, Albert} (7.\,9.\,1867 Mannheim – 15.\,5.\,1952 Atlantischer Ozean), \emph{Schauspieler}|pw} abſolut nicht will, iſt G.\pwindex{Grunwald, Willy 14.\,2.\,1870 Lingen – 8.\,5.\,1945 Berlin@\textsc{Grunwald, Willy} (14.\,2.\,1870 Lingen – 8.\,5.\,1945 Berlin), \emph{Theaterleiter, Schauspieler}|pw} gewiſs der
               einzige, der in Betracht kommt. Er wird{ }ſetze ich voraus, die Rolle von der weibiſch
                  \strikeout{ja} – verwöhnten Seite her zu nehmen{ }ſuchen, \strikeout{und als} ja, er wird vielleicht auch das hyſteriſch
               verlogene (es iſt eine Bezeichnung, kein Schimpf) in \substVorne{}\textsuperscript{\textcolor{gray}{×}\-\textcolor{gray}{×}\-\textcolor{gray}{×}\-\textcolor{gray}{×}\-\textcolor{gray}{×}\-\textcolor{gray}{×}\-\textcolor{gray}{×}\-\textcolor{gray}{×}\-\textcolor{gray}{×}\-\textcolor{gray}{×}}\substDazwischen{}lebhafterer\substHinten{} Weiſe herausbringen, als Sie wollten. Wie immer, – es {\pb}wird durch dieſe Beſetzung \strikeout{noch} mehr als je die Tragoedie von der Enttäuſchung des Pierre\pwindex{Hofmannsthal, Hugo von 1.\,2.\,1874 Wien – 15.\,7.\,1929 Rodaun@\textsc{Hofmannsthal, Hugo von} (1.\,2.\,1874 Wien – 15.\,7.\,1929 Rodaun), \emph{Schriftsteller}!gerettete Venedig. Trauerspiel in fünf Aufzügen@\strich\emph{Das gerettete Venedig. Trauerspiel in fünf Aufzügen}|pwv}, und vielleicht ko{\geminationm}t nun alles bei der Einſtudierg darauf an, mit dieſem
               Gleichgewichtsverhältnis von vornherein zu rechnen.\pend
           
\pstart
           Sie haben doch nun meine Karte aus Lueg\oindex{Lueg@\textbf{Lueg}, \emph{Teil eines besiedelten Ortes}|pw} bekommen?
               Wir{ }ſind alſo Montag 2.{ }Abends 8{ }Hietzing, \textsc{Kuffner}\oindex{Wien@\textbf{Wien}!XIII., Hietzing@\textbf{XIII., Hietzing}!Ottakringer Bräu@\textbf{Ottakringer Bräu}, \emph{Bierhaus}|pw}. Vielleicht iſt unſer \textsc{Charolais}\pwindex{Beer-Hofmann, Richard 11.\,7.\,1866 Wien – 26.\,9.\,1945 New York City@\textsc{Beer-Hofmann, Richard} (11.\,7.\,1866 Wien – 26.\,9.\,1945 New York City), \emph{Schriftsteller}|pwv}\pwindex{Beer-Hofmann, Richard 11.\,7.\,1866 Wien – 26.\,9.\,1945 New York City@\textsc{Beer-Hofmann, Richard} (11.\,7.\,1866 Wien – 26.\,9.\,1945 New York City), \emph{Schriftsteller}!Graf von Charolais. Ein Trauerspiel@\strich\emph{Der Graf von Charolais. Ein Trauerspiel}|pw} doch{ }ſchon hier und kommt?\pend
           
\pstart
           Herzlichſt Ihr{\\[\baselineskip]}\spacefill\mbox{A.}\pend
           \leftskip=0em{}\selectlanguage{ngerman}\endnumbering\briefempfaengerindex{Hofmannsthal, Hugo von@\textsc{Hofmannsthal, Hugo von}!zzzSchnitzler, Arthur@\emph{von Arthur Schnitzler}!1904-12-311@{31. 12. 1904}|)be}\mylabel{L01488h}  \newcommand{\dateiname}{L01488}\newcommand{\titel}{Arthur Schnitzler an Hugo von Hofmannsthal, 31. 12. 1904}\newcommand{\editorInnen}{Martin Anton Müller und Gerd-Hermann Susen}%% latex-leseansicht-abspann.tex
%% Abspann für die Leseansicht.
%% Der Schalter \ifkorrekturansicht ist bereits durch den Vorspann gesetzt.

%% latex-abspann.tex
%% Gemeinsamer Abspann für Korrekturansicht und Leseansicht.
%% Setzt den Schalter \ifkorrekturansicht voraus (gesetzt in den
%% einbindenden Dateien latex-korrekturansicht-abspann.tex bzw.
%% latex-leseansicht-abspann.tex).
%% ---------------------------------------------------------------

\normalsize

% Das esempio-Environment wird nur in der Leseansicht benötigt
\ifkorrekturansicht\else
\newenvironment{esempio}[3]%
{
    \vspace{1.5ex}
    \rlap{\underline{#1}}
    \par
    \setlength{\parindent}{0cm}
    \nopagebreak
    \leftskip=#2cm
    \rightskip=#3cm
}
{
    \par
}
\fi

\doendnotes{C}
\bigskip
\vfill

\clearpage

\footnotesize

\ifkorrekturansicht
  \lohead{\textsc{register}}
\fi

% theindex-Environment neu definieren ohne reledmac
\makeatletter
\renewenvironment{theindex}{%
  \ifkorrekturansicht
    \section*{\indexname}%
  \else
    \subsubsection*{Index der erwähnten Entitäten}%
  \fi
  \setlength{\parindent}{0pt}%
  \setlength{\parskip}{0pt plus 0.3pt}%
  \let\item\@idxitem
}{%
  \ifkorrekturansicht\clearpage\fi
}
\makeatother

\IfFileExists{\jobname-pw.ind}{\input{\jobname-pw.ind}}{}

% Quellenangabe nur in der Leseansicht
\ifkorrekturansicht\else
% Fallback-Definitionen, falls die .tex-Datei \titel etc. nicht gesetzt hat
\providecommand{\titel}{}
\providecommand{\editorInnen}{}
\providecommand{\dateiname}{\jobname}

\vspace{3cm}

\vfill

\footnotesize
\textsc{Quelle}: \titel. Herausgegeben von {\editorInnen}. In: \emph{Arthur Schnitzler: Briefwechsel mit Autorinnen und Autoren}.
 Digitale Edition, https://schnitzler-briefe.acdh.oeaw.ac.at/{\dateiname}.html (Stand \today)
\fi

\end{document}


