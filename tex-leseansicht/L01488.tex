%% latex-leseansicht-vorspann.tex
%% Vorspann für die Leseansicht.
%% Lädt die gemeinsame Datei latex-vorspann.tex mit nicht gesetztem Schalter.

\newif\ifkorrekturansicht
\korrekturansichtfalse

\input{../tex-inputs/latex-vorspann}


         
         \renewcommand{\erwaehntePersonen}{Personen: Albert Bassermann, Richard Beer-Hofmann, Otto Brahm, Willy Grunwald, Hugo von Hofmannsthal, Josef Kainz, Friedrich Mitterwurzer}
         \renewcommand{\erwaehnteOrte}{Orte: Lueg am Wolfgangsee, Ottakringer Bräu, Wien}
         \renewcommand{\erwaehnteWerke}{Werke: Das gerettete Venedig. Trauerspiel in fünf Aufzügen, Der Graf von Charolais. Ein Trauerspiel, Die Frau vom Meer. Schauspiel in fünf Akten, Florian Geyer. Die Tragödie des Bauernkrieges, Traumulus}
               \section[Arthur Schnitzler an Hugo von Hofmannsthal, 31. 12. 1904]{ Arthur Schnitzler an Hugo von Hofmannsthal, 31. 12. 1904}\nopagebreak\mylabel{v}\rehead{ }\begin{ledgroupsized}[t]{13cm}\normalsize\beginnumbering \toendnotes[C]{\smallbreak\pagebreak[2]} \Standort{FDH, Hs-30885,119.}
\physDesc{Brief, 2 Blätter, 5 Seiten
\newline{}Handschrift: schwarze Tinte, deutsche Kurrent}\buchAbdrucke{\weitereDrucke{Hugo von Hofmannsthal, Arthur Schnitzler: \emph{Briefwechsel}. Hg. Therese Nickl und Heinrich Schnitzler. Frankfurt am Main: \emph{S. Fischer} 1964, S. 209.} }\toendnotes[C]{\smallbreak}\pstart
           \raggedleft{}{\pb}Wien\oindex{Wien@\textbf{Wien}|pw}, 31. 12. 904.\pend
           \pstart{}lieber Hugo, \pend\pstart
           ich habe Grunwald\pwindex{Grunwald, Willy 14.02.1870 – 1945-05-08@\textsc{Grunwald, Willy} (14.02.1870 – 1945-05-08), \emph{Theaterleiter, Schauspieler}|pw} in Traumulus\pwindex{\textcolor{red}{\textsuperscript{XXXX1 indx}}!Traumulus1904@\strich\emph{Traumulus} {[}1904{]}|pw}\pwindex{\textcolor{red}{\textsuperscript{XXXX1 indx}}!Traumulus1904@\strich\emph{Traumulus} {[}1904{]}|pw} als problematiſchen Corpsſtudenten, in der Frau vom Meer\pwindex{\textcolor{red}{\textsuperscript{XXXX1 indx}}!Frau vom Meer. Schauspiel in fuenf Akten1889@\strich\emph{Die Frau vom Meer. Schauspiel in fünf Akten} {[}1889{]}|pw} als Lyngſtrand\pwindex{\textcolor{red}{\textsuperscript{XXXX1 indx}}!Frau vom Meer. Schauspiel in fuenf Akten1889@\strich\emph{Die Frau vom Meer. Schauspiel in fünf Akten} {[}1889{]}|pwv} und da{\geminationn} im Geyer\pwindex{\textcolor{red}{\textsuperscript{XXXX1 indx}}!Florian Geyer. Die Tragoedie des Bauernkrieges1896@\strich\emph{Florian Geyer. Die Tragödie des Bauernkrieges} {[}1896{]}|pw} als {\dots} ich weiſs nicht mehr was
               geſehen, und Brahm\pwindex{Brahm, Otto 05.02.1856 – 28.11.1912@\textsc{Brahm, Otto} (05.02.1856 – 28.11.1912), \emph{Theaterleiter, Regisseur}|pw} weiſs, daſs ich ihn ſehr
               ſchätze und noch allerlei Möglichkeiten in ihm zu ſpüren glaube. Er iſt aber gewiſs
               keine ſehr reiche und keine ſehr ſtarke Natur und hat auch das geheimnisvolle nicht,
               das manche haben, ohne stark {\pb}und groß zu ſein; er iſt
               ſehr ſcharf umriſſen aber es iſt nicht viel Luft um ihn. Nun ſcheint es mir aber für
               den Jaffier\pwindex{Hofmannsthal, Hugo von 1874-02-01 – 1929-07-15@\textsc{Hofmannsthal, Hugo von} (1874-02-01 – 1929-07-15), \emph{Schriftsteller}!gerettete Venedig. Trauerspiel in fuenf Aufzuegen1905@\strich\emph{Das gerettete Venedig. Trauerspiel in fünf Aufzügen} {[}1905{]}|pwv} notwendig, daſs man
               in ſeiner Perſönlichkeit den vergang\damage{ene}n Zauber ahnt und ich glaube, ſo etwas überzeugend herauszubringen, ist \strikeout{dichteriſch}{ }ſchauſpieleriſch ebenſo ſchwer, ja an der Grenze
               des Möglichen wie dichteriſch. Ihnen iſt es nur dadurch (und doch nicht ganz)
               gelungen, daſs Sie zwei in ihrer Art außerordentliche Menſchen, den \textsc{Pierre}\pwindex{Hofmannsthal, Hugo von 1874-02-01 – 1929-07-15@\textsc{Hofmannsthal, Hugo von} (1874-02-01 – 1929-07-15), \emph{Schriftsteller}!gerettete Venedig. Trauerspiel in fuenf Aufzuegen1905@\strich\emph{Das gerettete Venedig. Trauerspiel in fünf Aufzügen} {[}1905{]}|pwv} und die \textsc{Belvidera}\pwindex{Hofmannsthal, Hugo von 1874-02-01 – 1929-07-15@\textsc{Hofmannsthal, Hugo von} (1874-02-01 – 1929-07-15), \emph{Schriftsteller}!gerettete Venedig. Trauerspiel in fuenf Aufzuegen1905@\strich\emph{Das gerettete Venedig. Trauerspiel in fünf Aufzügen} {[}1905{]}|pwv}, {\pb}einen, deſſen Weſen Muth, die andere, deren
               Weſen Hingebung, noch zu einer Zeit unter jenem Zauber ſtehen laſſen, da wir nichts
               mehr \substVorne{}\textsuperscript{davon be}{\allowbreak}\substDazwischen{}von ihm\substHinten{} angerührt werden – aber immerhin\textcolor{gray}{,} wir denken: Muſs das
               ein Kerl geweſen ſein – daſs die zwei gar nicht merken, wie wenig er es heute iſt! –
                  Mitterwurzer\pwindex{Mitterwurzer, Friedrich 16.10.1844 – 13.02.1897@\textsc{Mitterwurzer, Friedrich} (16.10.1844 – 13.02.1897), \emph{Schauspieler}|pw}, Kainz\pwindex{Kainz, Josef 02.01.1858 – 20.09.1910@\textsc{Kainz, Josef} (02.01.1858 – 20.09.1910), \emph{Schauspieler}|pw}, Baſſermann\pwindex{Bassermann, Albert 07.09.1867 – 15.05.1952@\textsc{Bassermann, Albert} (07.09.1867 – 15.05.1952), \emph{Schauspieler}|pw} wieder trügen dieſes
               »geweſene« wie einen Heiligenſchein von verſtäubten Schickſalen um ihr Haupt, einen
               Schein, der eben nur in Perſönlichkeitsatmosphäre ſichtbar {\pb}wird. Davon, mein ich, wird bei Grunwald\pwindex{Grunwald, Willy 14.02.1870 – 1945-05-08@\textsc{Grunwald, Willy} (14.02.1870 – 1945-05-08), \emph{Theaterleiter, Schauspieler}|pw} nichts merklich ſein. Warum ich Ihnen das ſage weiſs
               ich eigentlich nicht – denn wenn \textsc{Bassermann}\pwindex{Bassermann, Albert 07.09.1867 – 15.05.1952@\textsc{Bassermann, Albert} (07.09.1867 – 15.05.1952), \emph{Schauspieler}|pw} abſolut nicht will, iſt G.\pwindex{Grunwald, Willy 14.02.1870 – 1945-05-08@\textsc{Grunwald, Willy} (14.02.1870 – 1945-05-08), \emph{Theaterleiter, Schauspieler}|pw} gewiſs der
               einzige, der in Betracht kommt. Er wird ſetze ich voraus, die Rolle von der weibiſch
                  \strikeout{ja} – verwöhnten Seite her zu nehmen ſuchen, \strikeout{und als} ja, er wird vielleicht auch das hyſteriſch
               verlogene (es iſt eine Bezeichnung, kein Schimpf) in \substVorne{}\textsuperscript{\textcolor{gray}{×}\-\textcolor{gray}{×}\-\textcolor{gray}{×}\-\textcolor{gray}{×}\-\textcolor{gray}{×}\-\textcolor{gray}{×}\-\textcolor{gray}{×}\-\textcolor{gray}{×}\-\textcolor{gray}{×}\-\textcolor{gray}{×}}\substDazwischen{}lebhafterer\substHinten{} Weiſe herausbringen, als Sie wollten. Wie immer, – es {\pb}wird durch dieſe Beſetzung \strikeout{noch} mehr als je die Tragoedie von der Enttäuſchung des Pierre\pwindex{Hofmannsthal, Hugo von 1874-02-01 – 1929-07-15@\textsc{Hofmannsthal, Hugo von} (1874-02-01 – 1929-07-15), \emph{Schriftsteller}!gerettete Venedig. Trauerspiel in fuenf Aufzuegen1905@\strich\emph{Das gerettete Venedig. Trauerspiel in fünf Aufzügen} {[}1905{]}|pwv}, und vielleicht ko{\geminationm}t nun alles bei der Einſtudierg darauf an, mit dieſem
               Gleichgewichtsverhältnis von vornherein zu rechnen.\pend
           \pstart
           Sie haben doch nun meine Karte aus Lueg\oindex{Lueg am Wolfgangsee@\textbf{Lueg am Wolfgangsee}|pw} bekommen?
               Wir ſind alſo Montag 2.{ }Abends 8{ }Hietzing, \textsc{Kuffner}\oindex{Ottakringer Braeu@\textbf{Ottakringer Bräu}|pw}. Vielleicht iſt unſer \textsc{Charolais}\pwindex{Beer-Hofmann, Richard 1866-07-11 – 1945-09-26@\textsc{Beer-Hofmann, Richard} (1866-07-11 – 1945-09-26), \emph{Schriftsteller}|pwv}\pwindex{Beer-Hofmann, Richard 1866-07-11 – 1945-09-26@\textsc{Beer-Hofmann, Richard} (1866-07-11 – 1945-09-26), \emph{Schriftsteller}!Graf von Charolais. Ein Trauerspiel1904-12-23@\strich\emph{Der Graf von Charolais. Ein Trauerspiel} {[}1904-12-23{]}|pw} doch ſchon hier und kommt?\pend
           \pstart
           Herzlichſt Ihr{\\[\baselineskip]}\spacefill\mbox{A.}\pend
           \leftskip=0em{}
         
         \endnumbering\mylabel{h}\end{ledgroupsized}  \newcommand{\dateiname}{L01488}\newcommand{\titel}{Arthur Schnitzler an Hugo von Hofmannsthal, 31. 12. 1904}\newcommand{\editorInnen}{Martin Anton Müller und Gerd-Hermann Susen}%% latex-leseansicht-abspann.tex
%% Abspann für die Leseansicht.
%% Der Schalter \ifkorrekturansicht ist bereits durch den Vorspann gesetzt.

%% latex-abspann.tex
%% Gemeinsamer Abspann für Korrekturansicht und Leseansicht.
%% Setzt den Schalter \ifkorrekturansicht voraus (gesetzt in den
%% einbindenden Dateien latex-korrekturansicht-abspann.tex bzw.
%% latex-leseansicht-abspann.tex).
%% ---------------------------------------------------------------

\normalsize

% Das esempio-Environment wird nur in der Leseansicht benötigt
\ifkorrekturansicht\else
\newenvironment{esempio}[3]%
{
    \vspace{1.5ex}
    \rlap{\underline{#1}}
    \par
    \setlength{\parindent}{0cm}
    \nopagebreak
    \leftskip=#2cm
    \rightskip=#3cm
}
{
    \par
}
\fi

\doendnotes{C}
\bigskip
\vfill

\clearpage

\footnotesize

\ifkorrekturansicht
  \lohead{\textsc{register}}
\fi

% theindex-Environment neu definieren ohne reledmac
\makeatletter
\renewenvironment{theindex}{%
  \ifkorrekturansicht
    \section*{\indexname}%
  \else
    \subsubsection*{Index der erwähnten Entitäten}%
  \fi
  \setlength{\parindent}{0pt}%
  \setlength{\parskip}{0pt plus 0.3pt}%
  \let\item\@idxitem
}{%
  \ifkorrekturansicht\clearpage\fi
}
\makeatother

\IfFileExists{\jobname-pw.ind}{\input{\jobname-pw.ind}}{}

% Quellenangabe nur in der Leseansicht
\ifkorrekturansicht\else
% Fallback-Definitionen, falls die .tex-Datei \titel etc. nicht gesetzt hat
\providecommand{\titel}{}
\providecommand{\editorInnen}{}
\providecommand{\dateiname}{\jobname}

\vspace{3cm}

\vfill

\footnotesize
\textsc{Quelle}: \titel. Herausgegeben von {\editorInnen}. In: \emph{Arthur Schnitzler: Briefwechsel mit Autorinnen und Autoren}.
 Digitale Edition, https://schnitzler-briefe.acdh.oeaw.ac.at/{\dateiname}.html (Stand \today)
\fi

\end{document}


      