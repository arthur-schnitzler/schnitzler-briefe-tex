%% latex-leseansicht-vorspann.tex
%% Vorspann für die Leseansicht.
%% Lädt die gemeinsame Datei latex-vorspann.tex mit nicht gesetztem Schalter.

\newif\ifkorrekturansicht
\korrekturansichtfalse

\input{../tex-inputs/latex-vorspann}


\section[Hermann Bahr an Arthur Schnitzler, 1. 4. 1902]{L01216 Hermann Bahr an Arthur Schnitzler, 1. 4. 1902}
\nopagebreak\mylabel{L01216v}
\rehead{ }\normalsize\beginnumbering\briefempfaengerindex{Schnitzler, Arthur@\textsc{Schnitzler, Arthur}!zzzBahr, Hermann@\emph{von Hermann Bahr}!1902-04-012@{1. 4. 1902}|(be}
\toendnotes[C]{\smallbreak\pagebreak[2]}
\correspDesc{Versand  durch Hermann Bahr am 1. 4. 1902 in Wien
\newline{}Erhalt  durch Arthur Schnitzler im Zeitraum [1. 4. 1902
                  – 5. 4. 1902?] in Wien}\toendnotes[C]{\smallbreak}
\Standort{CUL, Schnitzler, B 5b.}
\physDesc{Brief, 1 Blatt, 2 Seiten, 646 Zeichen
\newline{}Handschrift: schwarze Tinte, deutsche Kurrent
\newline{}Schnitzler: mit Bleistift die Jahreszahl »902« ergänzt 
\newline{}Ordnung: mit Bleistift von unbekannter Hand nummeriert:
                                    »87« }
\buchAbdrucke{\weitereDrucke{Hermann Bahr, Arthur Schnitzler: \emph{Briefwechsel, Aufzeichnungen, Dokumente (1891–1931)}. Herausgegeben von Kurt Ifkovits und Martin Anton Müller. Göttingen: \emph{Wallstein} 2018, S. 228.} }\toendnotes[C]{\smallbreak}
\pstart
           \raggedleft{}{\pb}1. 4.\pend
           
\pstart\center{}Lieber Arthur!\pend\vspace{0.5em}
\pstart
           Die mir zugeſchickten Proben{ }ſind von jener heute{ }ſo weit verbreiteten
               Talentloſigkeit, die glaubt, es genüge einige Wendungen von »modernen« Autoren
               aufzuſchnappen, und gar nicht zu bemerken{ }ſcheint, daß{ }ſie gar nichts zu{ }ſagen hat.
               Dies{ }ſchließt nicht aus, daß der Ver{\pb}faſſer\pwindex{Modry, Gustav 1.\,9.\,1875 České Budějovice – 16.\,1.\,1928 Wien@\textsc{Modry, Gustav} (1.\,9.\,1875 České Budějovice – 16.\,1.\,1928 Wien), \emph{Rechtsanwalt}|pwv} vielleicht{ }ſich zum
               Journaliſten eignen könnte. Eine »Schmuck-Notiz« über Allerheiligen oder die
               Eröffnung oder Schließung eines Cafés oder eine{ }ſchöne Leich’ iſt ja ganz was
               anderes. Doch müßte man davon Proben{ }ſehen und wiſſen, was er{ }ſich unter »Journaliſt«
               (der er, wie Du{ }ſchreibſt, werden will) eigentlich denkt.\pend
           
\pstart
           Herzlichſt{\\[\baselineskip]}in Eile{\\[\baselineskip]}Dein alter{\\[\baselineskip]}\spacefill\mbox{Hermann}\pend
           \leftskip=0em{}\selectlanguage{ngerman}\endnumbering\briefempfaengerindex{Schnitzler, Arthur@\textsc{Schnitzler, Arthur}!zzzBahr, Hermann@\emph{von Hermann Bahr}!1902-04-012@{1. 4. 1902}|)be}\mylabel{L01216h}  \newcommand{\dateiname}{L01216}\newcommand{\titel}{Hermann Bahr an Arthur Schnitzler, 1. 4. 1902}\newcommand{\editorInnen}{Herausgegeben von Martin Anton Müller}%% latex-leseansicht-abspann.tex
%% Abspann für die Leseansicht.
%% Der Schalter \ifkorrekturansicht ist bereits durch den Vorspann gesetzt.

%% latex-abspann.tex
%% Gemeinsamer Abspann für Korrekturansicht und Leseansicht.
%% Setzt den Schalter \ifkorrekturansicht voraus (gesetzt in den
%% einbindenden Dateien latex-korrekturansicht-abspann.tex bzw.
%% latex-leseansicht-abspann.tex).
%% ---------------------------------------------------------------

\normalsize

% Das esempio-Environment wird nur in der Leseansicht benötigt
\ifkorrekturansicht\else
\newenvironment{esempio}[3]%
{
    \vspace{1.5ex}
    \rlap{\underline{#1}}
    \par
    \setlength{\parindent}{0cm}
    \nopagebreak
    \leftskip=#2cm
    \rightskip=#3cm
}
{
    \par
}
\fi

\doendnotes{C}
\bigskip
\vfill

\clearpage

\footnotesize

\ifkorrekturansicht
  \lohead{\textsc{register}}
\fi

% theindex-Environment neu definieren ohne reledmac
\makeatletter
\renewenvironment{theindex}{%
  \ifkorrekturansicht
    \section*{\indexname}%
  \else
    \subsubsection*{Index der erwähnten Entitäten}%
  \fi
  \setlength{\parindent}{0pt}%
  \setlength{\parskip}{0pt plus 0.3pt}%
  \let\item\@idxitem
}{%
  \ifkorrekturansicht\clearpage\fi
}
\makeatother

\IfFileExists{\jobname-pw.ind}{\input{\jobname-pw.ind}}{}

% Quellenangabe nur in der Leseansicht
\ifkorrekturansicht\else
% Fallback-Definitionen, falls die .tex-Datei \titel etc. nicht gesetzt hat
\providecommand{\titel}{}
\providecommand{\editorInnen}{}
\providecommand{\dateiname}{\jobname}

\vspace{3cm}

\vfill

\footnotesize
\textsc{Quelle}: \titel. Herausgegeben von {\editorInnen}. In: \emph{Arthur Schnitzler: Briefwechsel mit Autorinnen und Autoren}.
 Digitale Edition, https://schnitzler-briefe.acdh.oeaw.ac.at/{\dateiname}.html (Stand \today)
\fi

\end{document}


