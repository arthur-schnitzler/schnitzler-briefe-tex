\input{../tex-inputs/latex-pdf-vorspann}
\begin{center}
            \textcolor{red}{ENTWURF. ENTZIFFERUNG NOCH NICHT KORREKTURGELESEN}
                      \end{center}
            
               \section[Paul Goldmann an Arthur Schnitzler, 3. 7. 1894]{ Paul Goldmann an Arthur Schnitzler, 3. 7. 1894}\nopagebreak\mylabel{v}\rehead{ }\begin{ledgroupsized}[t]{13cm}\normalsize\beginnumbering\briefempfaengerindex{Schnitzler, Arthur@\textsc{Schnitzler, Arthur}!zzzGoldmann, Paul@\emph{von Paul Goldmann}!1894-07-031@{3. 7. {[}1894{]}}|(be} \toendnotes[C]{\smallbreak\pagebreak[2]} \Standort{DLA, A:Schnitzler, HS.NZ85.1.3164.}
\physDesc{Postkarte
\newline{}Handschrift: 1) schwarze Tinte, deutsche Kurrent\hspace{1em}2) schwarze Tinte, lateinische Kurrent (\noindent{}Adresse)\hspace{1em}\newline{}Versand: 1) Stempel: »\nobreak{}\oindex{Place de la Bourse@\textbf{Place de la Bourse}|pwk}Paris Pl. de la Bourse, \textcolor{gray}{3} Juil 94\nobreak{}«.  2) Stempel: »\nobreak{}\oindex{IX., Alsergrund@\textbf{IX., Alsergrund}|pwk}Wien 9/3 72, 5. 7. 94, 8.V, Bestellt\nobreak{}«. 
\newline{}Schnitzler: mit Bleistift das Datum »3/7 94« vermerkt }\toendnotes[C]{\smallbreak}\pstart{}
u                  {\pb}\begin{otherlanguage}{french}Autriche\end{otherlanguage}\oindex{Oesterreich@\textbf{Österreich}|pw}.\pend{}\pstart{}Herrn Dr. Arthur Schnitzler \pend{}\pstart{}IX. Frankgaße 1\oindex{Frankgasse@\textbf{Frankgasse}|pw}\pend{}\pstart{}Wien\oindex{Wien@\textbf{Wien}|pw}\pend{}{\bigskip}\pstart
           {\pb}\textsc{Paris\oindex{Paris@\textbf{Paris}|pw}}, 3. Juli.\pend
           \pstart{}Liebſter Freund,\pend\pstart
           Bitte ſchicke mir die Adreſſe\oindex{Frankgasse@\textbf{Frankgasse}|pwv}
               Deines Bruder\pwindex{Schnitzler, Julius 13.07.1865 – 29.06.1939@\textsc{Schnitzler, Julius} (13.07.1865 – 29.06.1939), \emph{Chirurg}|pwv}s oder \strikeout{ſei} des \label{K_L02629-2v}\edtext{Locales}{\lemma{\textnormal{\emph{Locales}}}\Cendnote{\textnormal{nicht identifizert}}}\label{K_L02629-2h},
               in dem er die \label{K_L02629-1v}\edtext{Hochzeit}{\lemma{\textnormal{\emph{Hochzeit}}}\Cendnote{\textnormal{Julius Schnitzler\pwindex{Schnitzler, Julius 13.07.1865 – 29.06.1939@\textsc{Schnitzler, Julius} (13.07.1865 – 29.06.1939), \emph{Chirurg}|pwk} und Helene Altmann\pwindex{Schnitzler, Helene 16.07.1871 – September 1941@\textsc{Schnitzler, Helene} (16.07.1871 – September 1941)|pwk} heirateten am 8. 7. 1894.}}}\label{K_L02629-1h} feiert.\pend
           \pstart
           Und warum ſchreibſt Du mir nicht? Herzlichſt {\\}Dein {\\}\spacefill\mbox{P. G.}\pend
           \endnumbering\briefempfaengerindex{Schnitzler, Arthur@\textsc{Schnitzler, Arthur}!zzzGoldmann, Paul@\emph{von Paul Goldmann}!1894-07-031@{3. 7. {[}1894{]}}|)be}\mylabel{h}\end{ledgroupsized}  \newcommand{\dateiname}{L02629}\newcommand{\titel}{Paul Goldmann an Arthur Schnitzler, 3. 7. 1894}\newcommand{\editorInnen}{Martin Anton Müller und Laura Untner}\input{../tex-inputs/latex-pdf-abspann}
      