%% latex-korrekturansicht-vorspann.tex
%% Vorspann für die Korrekturansicht.
%% Lädt die gemeinsame Datei latex-vorspann.tex mit gesetztem Schalter.

\newif\ifkorrekturansicht
\korrekturansichttrue

\input{../tex-inputs/latex-vorspann}


\section[ Felix Salten an Arthur Schnitzler, 22. 5. 1902]{L03330 Felix Salten an Arthur Schnitzler, 22. 5. 1902}
\nopagebreak\mylabel{L03330v}
\rehead{ }\normalsize\beginnumbering\briefempfaengerindex{Schnitzler, Arthur@\textsc{Schnitzler, Arthur}!zzzSalten, Felix@\emph{von Felix Salten}!1902-05-221@{22. 5. 1902}|(be}
\toendnotes[C]{\smallbreak\pagebreak[2]}\Standort{CUL, Schnitzler, B 89, A 2.}
\physDesc{Brief, 1 Blatt, 2 Seiten, 1534 Zeichen
\newline{}Handschrift: Bleistift, lateinische Kurrent
\newline{}Ordnung: mit Bleistift von unbekannter Hand nummeriert: »155« }\toendnotes[C]{\smallbreak}
\pstart
           \raggedleft{}{\pb}Florenz\oindex{Florenz@\textbf{Florenz}, \emph{P.PPLA}|pw}, 22. Mai 02.\pend
           \vspace{0.5em}
\pstart
           Lieber Arthur, eben las ich Ihre kleine \label{K_L03330-1v}\edtext{Novelle\pwindex{Daemmerseele@\emph{Dämmerseele}|pwv}}{\lemma{\textnormal{\emph{Novelle}}}\Cendnote{\textnormal{Arthur Schnitzler: \emph{Dämmerseele}\pwindex{Daemmerseele@\emph{Dämmerseele}|pwk}. In: \emph{Neue
                        Freie Presse}\pwindex{Neue Freie Presse@\emph{Neue Freie Presse}|pwk}, Nr. 13.553, 18. 5. 1902,
                     Morgenblatt, Pfingstbeilage, S. 31–33.}}}\label{K_L03330-1} in der »N. fr. Pr.\pwindex{Neue Freie Presse@\emph{Neue Freie Presse}|pw}« Ich glaube, das ist nicht blos an sich etwas
               Gutes, sondern auch ein Schritt weiter. Es ist alles Psychologische in eine knappe
               Gegenständlichkeit verlegt, und gut zusa{\geminationm}engefaßt. Meunier\pwindex{Meunier, Constantin 1831-04-12 – 1905-04-04@\textsc{Meunier, Constantin} (1831-04-12 – 1905-04-04), \emph{Maler/Malerin, Bildhauer/Bildhauerin}|pw} und Maupassant\pwindex{Maupassant, Guy de 05.08.1850 – 07.07.1893@\textsc{Maupassant, Guy de} (05.08.1850 – 07.07.1893), \emph{Schriftsteller/Schriftstellerin}|pw}. Und es ist wirklich »erzählt«. Ich finde neue Spuren \strikeout{\textcolor{gray}{darin},} und täusche mich hoffentlich nicht. Nebenbei: die
               ganze \label{K_L03330-2v}\edtext{Renate\pwindex{Geschichte der jungen Renate Fuchs@\emph{Die Geschichte der jungen Renate Fuchs}|pw}}{\lemma{\textnormal{\emph{Renate}}}\Cendnote{\textnormal{Jakob Wassermanns\pwindex{Wassermann, Jakob 10.03.1873 – 01.01.1934@\textsc{Wassermann, Jakob} (10.03.1873 – 01.01.1934), \emph{Schriftsteller/Schriftstellerin}|pwk}{ }\emph{Die Geschichte der jungen Renate Fuchs}\pwindex{Geschichte der jungen Renate Fuchs@\emph{Die Geschichte der jungen Renate Fuchs}|pwk} erschien 1900, die Buchausgabe ist auf 1901 vordatiert.}}}\label{K_L03330-2} liegt auch drin, im Extract,
               und eigentlich viel plastischer und aufrichtiger, obwol vorn und hinten alles fehlt.
               Der \label{K_L03330-3v}\edtext{Titel »Dämmerseele\pwindex{Daemmerseele@\emph{Dämmerseele}|pw}« scheint mir aber ganz verfehlt}{\lemma{\textnormal{\emph{Titel … verfehlt}}}\Cendnote{\textnormal{Da nur der Erstdruck \emph{Dämmerseele}\pwindex{Daemmerseele@\emph{Dämmerseele}|pwk} hieß, dürfte Schnitzler{ }Saltens\pwindex{Salten, Felix 06.09.1869 – 08.10.1945@\textsc{Salten, Felix} (06.09.1869 – 08.10.1945), \emph{Schriftsteller/Schriftstellerin, Journalist/Journalistin, Chefredakteur/Chefredakteurin}|pwk} Kritik ernst genommen haben. Die
                  erste Buchausgabe\pwindex{Daemmerseelen. Novellen@\emph{Dämmerseelen. Novellen}|pwkv} von 1907 verwendete \emph{Dämmerseelen}\pwindex{Daemmerseelen. Novellen@\emph{Dämmerseelen. Novellen}|pwk} als Gesamttitel, die betreffende Novelle\pwindex{Daemmerseele@\emph{Dämmerseele}|pwkv} wurde aber zu \emph{Die Fremde}\pwindex{Daemmerseele@\emph{Dämmerseele}|pwk} umbenannt.}}}\label{K_L03330-3}. – Geschrieben nimmt sich
               alles härter aus \textcolor{gray}{–} bitte – reduziren Sie also das Folgende auf die
               Wirkung des \uline{Gesagten}: Es ist ein Dörmann\pwindex{Doermann, Felix 29.05.1870 – 26.10.1928@\textsc{Dörmann, Felix} (29.05.1870 – 26.10.1928), \emph{Schriftsteller/Schriftstellerin}|pw} Titel, d. h. ein Versuch eine Gattung abzugrenzen, zu
               benennen, aber die Grenze und die Benennung sind nicht scharf, und dem Wort haftet
               eine leidige, ins Sentimentale gehende Weichheit an. Es liegt auch kaum die
               Notwendigkeit vor, durch den Titel etwas zu erklären; mit ihm selbst das Wort zu
               ergreifen. Und gerade mit diesem Titel ist alles in einer eigentlich hindernden und
               auch irreführenden Art vorweg genommen. Er ist vielleicht aus der Hofkirche\oindex{Hofkirche@\textbf{Hofkirche}, \emph{Kirche (K.KRC)}|pw} besser zu holen. Am besten aus der Einfachheit.
               Ganz außerordentlich ist der Schluß. Das geht in kurzer Wendung {\pb}zu einer beinahe dramatischen
               Höhe, jedesfalls zu einem weiten Ausblick. Nun bedaure ich es, dass ich noch nicht
               dazu kam, über die \label{K_L03330-4v}\edtext{Bertha Garlan\pwindex{Frau Bertha Garlan. Roman@\emph{Frau Bertha Garlan. Roman}|pw}}{\lemma{\textnormal{\emph{Bertha Garlan}}}\Cendnote{\textnormal{Das Erscheinen der Buchausgabe\pwindex{Frau Bertha Garlan. Roman@\emph{Frau Bertha Garlan. Roman}|pwkv} lag bereits
                  über 12 Monate zurück. Eine Besprechung verfasste Salten\pwindex{Salten, Felix 06.09.1869 – 08.10.1945@\textsc{Salten, Felix} (06.09.1869 – 08.10.1945), \emph{Schriftsteller/Schriftstellerin, Journalist/Journalistin, Chefredakteur/Chefredakteurin}|pwk} auch
               später nicht mehr.}}}\label{K_L03330-4} zu schreiben. Das will
               ich im Sommer nachholen. Jetzt war und bin
               ich eben sehr mit mir selbst beschäftigt.\pend
           
\pstart
           herzlichst Ihr {\\[\baselineskip]}\spacefill\mbox{Salten}\pend
           \leftskip=0em{}\selectlanguage{ngerman}\endnumbering\briefempfaengerindex{Schnitzler, Arthur@\textsc{Schnitzler, Arthur}!zzzSalten, Felix@\emph{von Felix Salten}!1902-05-221@{22. 5. 1902}|)be}\mylabel{L03330h}  \normalsize

\doendnotes{C}
\bigskip
\vfill

\clearpage

\footnotesize

\lohead{\textsc{register}}

% Definiere theindex-Environment komplett neu ohne reledmac
\makeatletter
\renewenvironment{theindex}{%
  \section*{\indexname}%
  \setlength{\parindent}{0pt}%
  \setlength{\parskip}{0pt plus 0.3pt}%
  \let\item\@idxitem
}{%
  \clearpage
}
\makeatother

\IfFileExists{\jobname-pw.ind}{\input{\jobname-pw.ind}}{}

\end{document}

      