%% latex-leseansicht-vorspann.tex
%% Vorspann für die Leseansicht.
%% Lädt die gemeinsame Datei latex-vorspann.tex mit nicht gesetztem Schalter.

\newif\ifkorrekturansicht
\korrekturansichtfalse

\input{../tex-inputs/latex-vorspann}


\section[ Felix Salten an Arthur Schnitzler, 22. 5. 1902]{L03330 Felix Salten an Arthur Schnitzler,  22. 5. 1902}
\nopagebreak\mylabel{L03330v}
\rehead{ }\normalsize\beginnumbering\briefempfaengerindex{Schnitzler, Arthur@\textsc{Schnitzler, Arthur}!zzzSalten, Felix@\emph{von Felix Salten}!1902-05-221@{22. 5. 1902}|(be}
\toendnotes[C]{\smallbreak\pagebreak[2]}
\correspDesc{Versand  durch Felix Salten am 22. 5. 1902 in Florenz
\newline{}Erhalt  durch Arthur Schnitzler im Zeitraum [23. 5. 1902
                  – 27. 5. 1902?] in Wien}\toendnotes[C]{\smallbreak}
\Standort{CUL, Schnitzler, B 89, A 2.}
\physDesc{Brief, 1 Blatt, 2 Seiten, 1534 Zeichen
\newline{}Handschrift: Bleistift, lateinische Kurrent
\newline{}Ordnung: mit Bleistift von unbekannter Hand nummeriert: »155« }\toendnotes[C]{\smallbreak}
\pstart
           \raggedleft{}{\pb}Florenz\oindex{Florenz@\textbf{Florenz}|pw}, 22. Mai 02.\pend
           \vspace{0.5em}
\pstart
           Lieber Arthur, eben las ich Ihre kleine \label{K_L03330-1v}\edtext{Novelle\pwindex{Schnitzler, Arthur 15.\,5.\,1862 Wien – 21.\,10.\,1931 ebd.@\textsc{Schnitzler, Arthur} (15.\,5.\,1862 Wien – 21.\,10.\,1931 ebd.), \emph{Schriftsteller, Mediziner}!Dämmerseele@\strich\emph{Dämmerseele}|pwv}}{\lemma{\textnormal{\emph{Novelle}}}\Cendnote{\textnormal{Arthur Schnitzler: \emph{Dämmerseele}\pwindex{Schnitzler, Arthur 15.\,5.\,1862 Wien – 21.\,10.\,1931 ebd.@\textsc{Schnitzler, Arthur} (15.\,5.\,1862 Wien – 21.\,10.\,1931 ebd.), \emph{Schriftsteller, Mediziner}!Dämmerseele@\strich\emph{Dämmerseele}|pwk}. In: \emph{Neue
                        Freie Presse}\pwindex{Neue Freie Presse@\emph{Neue Freie Presse}|pwk}, Nr. 13.553, 18. 5. 1902,
                     Morgenblatt, Pfingstbeilage, S. 31–33.}}}\label{K_L03330-1} in der »N. fr. Pr.\pwindex{Neue Freie Presse@\emph{Neue Freie Presse}|pw}« Ich glaube, das ist nicht blos an sich etwas
               Gutes, sondern auch ein Schritt weiter. Es ist alles Psychologische in eine knappe
               Gegenständlichkeit verlegt, und gut zusa{\geminationm}engefaßt. Meunier\pwindex{Meunier, Constantin 12.\,4.\,1831 Etterbeek – 4.\,4.\,1905 Ixelles@\textsc{Meunier, Constantin} (12.\,4.\,1831 Etterbeek – 4.\,4.\,1905 Ixelles), \emph{Maler, Bildhauer}|pw} und Maupassant\pwindex{Maupassant, Guy de 5.\,8.\,1850 Tourville-sur-Arques – 7.\,7.\,1893 Paris@\textsc{Maupassant, Guy de} (5.\,8.\,1850 Tourville-sur-Arques – 7.\,7.\,1893 Paris), \emph{Schriftsteller}|pw}. Und es ist wirklich »erzählt«. Ich finde neue Spuren \strikeout{\textcolor{gray}{darin},} und täusche mich hoffentlich nicht. Nebenbei: die
               ganze \label{K_L03330-2v}\edtext{Renate\pwindex{Wassermann, Jakob 10.\,3.\,1873 Fürth – 1.\,1.\,1934 Altaussee@\textsc{Wassermann, Jakob} (10.\,3.\,1873 Fürth – 1.\,1.\,1934 Altaussee), \emph{Schriftsteller}!Geschichte der jungen Renate Fuchs@\strich\emph{Die Geschichte der jungen Renate Fuchs}|pw}}{\lemma{\textnormal{\emph{Renate}}}\Cendnote{\textnormal{Jakob Wassermanns\pwindex{Wassermann, Jakob 10.\,3.\,1873 Fürth – 1.\,1.\,1934 Altaussee@\textsc{Wassermann, Jakob} (10.\,3.\,1873 Fürth – 1.\,1.\,1934 Altaussee), \emph{Schriftsteller}|pwk}{ }\emph{Die Geschichte der jungen Renate Fuchs}\pwindex{Wassermann, Jakob 10.\,3.\,1873 Fürth – 1.\,1.\,1934 Altaussee@\textsc{Wassermann, Jakob} (10.\,3.\,1873 Fürth – 1.\,1.\,1934 Altaussee), \emph{Schriftsteller}!Geschichte der jungen Renate Fuchs@\strich\emph{Die Geschichte der jungen Renate Fuchs}|pwk} erschien 1900, die Buchausgabe ist auf 1901 vordatiert.}}}\label{K_L03330-2} liegt auch drin, im Extract,
               und eigentlich viel plastischer und aufrichtiger, obwol vorn und hinten alles fehlt.
               Der \label{K_L03330-3v}\edtext{Titel »Dämmerseele\pwindex{Schnitzler, Arthur 15.\,5.\,1862 Wien – 21.\,10.\,1931 ebd.@\textsc{Schnitzler, Arthur} (15.\,5.\,1862 Wien – 21.\,10.\,1931 ebd.), \emph{Schriftsteller, Mediziner}!Dämmerseele@\strich\emph{Dämmerseele}|pw}« scheint mir aber ganz verfehlt}{\lemma{\textnormal{\emph{Titel … verfehlt}}}\Cendnote{\textnormal{Da nur der Erstdruck \emph{Dämmerseele}\pwindex{Schnitzler, Arthur 15.\,5.\,1862 Wien – 21.\,10.\,1931 ebd.@\textsc{Schnitzler, Arthur} (15.\,5.\,1862 Wien – 21.\,10.\,1931 ebd.), \emph{Schriftsteller, Mediziner}!Dämmerseele@\strich\emph{Dämmerseele}|pwk} hieß, dürfte Schnitzler{ }Saltens\pwindex{Salten, Felix 6.\,9.\,1869 Budapest – 8.\,10.\,1945 Zürich@\textsc{Salten, Felix} (6.\,9.\,1869 Budapest – 8.\,10.\,1945 Zürich), \emph{Schriftsteller, Journalist, Chefredakteur}|pwk} Kritik ernst genommen haben. Die
                  erste Buchausgabe\pwindex{Schnitzler, Arthur 15.\,5.\,1862 Wien – 21.\,10.\,1931 ebd.@\textsc{Schnitzler, Arthur} (15.\,5.\,1862 Wien – 21.\,10.\,1931 ebd.), \emph{Schriftsteller, Mediziner}!Dämmerseelen. Novellen@\strich\emph{Dämmerseelen. Novellen}|pwkv} von 1907 verwendete \emph{Dämmerseelen}\pwindex{Schnitzler, Arthur 15.\,5.\,1862 Wien – 21.\,10.\,1931 ebd.@\textsc{Schnitzler, Arthur} (15.\,5.\,1862 Wien – 21.\,10.\,1931 ebd.), \emph{Schriftsteller, Mediziner}!Dämmerseelen. Novellen@\strich\emph{Dämmerseelen. Novellen}|pwk} als Gesamttitel, die betreffende Novelle\pwindex{Schnitzler, Arthur 15.\,5.\,1862 Wien – 21.\,10.\,1931 ebd.@\textsc{Schnitzler, Arthur} (15.\,5.\,1862 Wien – 21.\,10.\,1931 ebd.), \emph{Schriftsteller, Mediziner}!Dämmerseele@\strich\emph{Dämmerseele}|pwkv} wurde aber zu \emph{Die Fremde}\pwindex{Schnitzler, Arthur 15.\,5.\,1862 Wien – 21.\,10.\,1931 ebd.@\textsc{Schnitzler, Arthur} (15.\,5.\,1862 Wien – 21.\,10.\,1931 ebd.), \emph{Schriftsteller, Mediziner}!Dämmerseele@\strich\emph{Dämmerseele}|pwk} umbenannt.}}}\label{K_L03330-3}. – Geschrieben nimmt sich
               alles härter aus \textcolor{gray}{–} bitte – reduziren Sie also das Folgende auf die
               Wirkung des \uline{Gesagten}: Es ist ein Dörmann\pwindex{Dörmann, Felix 29.\,5.\,1870 Wien – 26.\,10.\,1928 ebd.@\textsc{Dörmann, Felix} (29.\,5.\,1870 Wien – 26.\,10.\,1928 ebd.), \emph{Schriftsteller}|pw} Titel, d. h. ein Versuch eine Gattung abzugrenzen, zu
               benennen, aber die Grenze und die Benennung sind nicht scharf, und dem Wort haftet
               eine leidige, ins Sentimentale gehende Weichheit an. Es liegt auch kaum die
               Notwendigkeit vor, durch den Titel etwas zu erklären; mit ihm selbst das Wort zu
               ergreifen. Und gerade mit diesem Titel ist alles in einer eigentlich hindernden und
               auch irreführenden Art vorweg genommen. Er ist vielleicht aus der Hofkirche\oindex{Hofkirche@\textbf{Hofkirche}, \emph{Kirche}|pw} besser zu holen. Am besten aus der Einfachheit.
               Ganz außerordentlich ist der Schluß. Das geht in kurzer Wendung {\pb}zu einer beinahe dramatischen
               Höhe, jedesfalls zu einem weiten Ausblick. Nun bedaure ich es, dass ich noch nicht
               dazu kam, über die \label{K_L03330-4v}\edtext{Bertha Garlan\pwindex{Schnitzler, Arthur 15.\,5.\,1862 Wien – 21.\,10.\,1931 ebd.@\textsc{Schnitzler, Arthur} (15.\,5.\,1862 Wien – 21.\,10.\,1931 ebd.), \emph{Schriftsteller, Mediziner}!Frau Bertha Garlan. Roman@\strich\emph{Frau Bertha Garlan. Roman}|pw}}{\lemma{\textnormal{\emph{Bertha Garlan}}}\Cendnote{\textnormal{Das Erscheinen der Buchausgabe\pwindex{Schnitzler, Arthur 15.\,5.\,1862 Wien – 21.\,10.\,1931 ebd.@\textsc{Schnitzler, Arthur} (15.\,5.\,1862 Wien – 21.\,10.\,1931 ebd.), \emph{Schriftsteller, Mediziner}!Frau Bertha Garlan. Roman@\strich\emph{Frau Bertha Garlan. Roman}|pwkv} lag bereits
                  über 12 Monate zurück. Eine Besprechung verfasste Salten\pwindex{Salten, Felix 6.\,9.\,1869 Budapest – 8.\,10.\,1945 Zürich@\textsc{Salten, Felix} (6.\,9.\,1869 Budapest – 8.\,10.\,1945 Zürich), \emph{Schriftsteller, Journalist, Chefredakteur}|pwk} auch
               später nicht mehr.}}}\label{K_L03330-4} zu schreiben. Das will
               ich im Sommer nachholen. Jetzt war und bin
               ich eben sehr mit mir selbst beschäftigt.\pend
           
\pstart
           herzlichst Ihr {\\[\baselineskip]}\spacefill\mbox{Salten}\pend
           \leftskip=0em{}\selectlanguage{ngerman}\endnumbering\briefempfaengerindex{Schnitzler, Arthur@\textsc{Schnitzler, Arthur}!zzzSalten, Felix@\emph{von Felix Salten}!1902-05-221@{22. 5. 1902}|)be}\mylabel{L03330h}  \newcommand{\dateiname}{L03330}\newcommand{\titel}{Felix Salten an Arthur Schnitzler, 22. 5. 1902}\newcommand{\editorInnen}{Martin Anton Müller und Laura Untner}%% latex-leseansicht-abspann.tex
%% Abspann für die Leseansicht.
%% Der Schalter \ifkorrekturansicht ist bereits durch den Vorspann gesetzt.

%% latex-abspann.tex
%% Gemeinsamer Abspann für Korrekturansicht und Leseansicht.
%% Setzt den Schalter \ifkorrekturansicht voraus (gesetzt in den
%% einbindenden Dateien latex-korrekturansicht-abspann.tex bzw.
%% latex-leseansicht-abspann.tex).
%% ---------------------------------------------------------------

\normalsize

% Das esempio-Environment wird nur in der Leseansicht benötigt
\ifkorrekturansicht\else
\newenvironment{esempio}[3]%
{
    \vspace{1.5ex}
    \rlap{\underline{#1}}
    \par
    \setlength{\parindent}{0cm}
    \nopagebreak
    \leftskip=#2cm
    \rightskip=#3cm
}
{
    \par
}
\fi

\doendnotes{C}
\bigskip
\vfill

\clearpage

\footnotesize

\ifkorrekturansicht
  \lohead{\textsc{register}}
\fi

% theindex-Environment neu definieren ohne reledmac
\makeatletter
\renewenvironment{theindex}{%
  \ifkorrekturansicht
    \section*{\indexname}%
  \else
    \subsubsection*{Index der erwähnten Entitäten}%
  \fi
  \setlength{\parindent}{0pt}%
  \setlength{\parskip}{0pt plus 0.3pt}%
  \let\item\@idxitem
}{%
  \ifkorrekturansicht\clearpage\fi
}
\makeatother

\IfFileExists{\jobname-pw.ind}{\input{\jobname-pw.ind}}{}

% Quellenangabe nur in der Leseansicht
\ifkorrekturansicht\else
% Fallback-Definitionen, falls die .tex-Datei \titel etc. nicht gesetzt hat
\providecommand{\titel}{}
\providecommand{\editorInnen}{}
\providecommand{\dateiname}{\jobname}

\vspace{3cm}

\vfill

\footnotesize
\textsc{Quelle}: \titel. Herausgegeben von {\editorInnen}. In: \emph{Arthur Schnitzler: Briefwechsel mit Autorinnen und Autoren}.
 Digitale Edition, https://schnitzler-briefe.acdh.oeaw.ac.at/{\dateiname}.html (Stand \today)
\fi

\end{document}


