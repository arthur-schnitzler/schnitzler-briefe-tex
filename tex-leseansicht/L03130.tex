%% latex-leseansicht-vorspann.tex
%% Vorspann für die Leseansicht.
%% Lädt die gemeinsame Datei latex-vorspann.tex mit nicht gesetztem Schalter.

\newif\ifkorrekturansicht
\korrekturansichtfalse

\input{../tex-inputs/latex-vorspann}


\section[Felix Salten an Arthur Schnitzler, {[}24?. 10. 1893{]}]{L03130 Felix Salten an Arthur Schnitzler, {[}24?. 10. 1893{]}}
\nopagebreak\mylabel{L03130v}
\rehead{ }\normalsize\beginnumbering\briefempfaengerindex{Schnitzler, Arthur@\textsc{Schnitzler, Arthur}!zzzSalten, Felix@\emph{von Felix Salten}!1893-10-241@{{[}24?. 10. 1893{]}}|(be}
\toendnotes[C]{\smallbreak\pagebreak[2]}
\correspDesc{Versand  durch Felix Salten am [24?. 10. 1893] in Wien
\newline{}Erhalt  durch Arthur Schnitzler am [24?. 10. 1893] in Wien}\toendnotes[C]{\smallbreak}
\Standort{CUL, Schnitzler, B 89, A 1.}
\physDesc{Brief, 1 Blatt, 3 Seiten, 348 Zeichen
\newline{}Handschrift: Bleistift, lateinische Kurrent
\newline{}Schnitzler: 1) mit Bleistift datiert: »2\substVorne{}\textsuperscript{\textcolor{gray}{5}}\substDazwischen{}\textcolor{gray}{4}\substHinten{}/X 93«  2) mit Bleistift auf der vierten Seite vermerkt: »{\pb}Dr. v. Bogdanovits\pwindex{Bogdanovits, W. @\textsc{Bogdanovits, W.}, \emph{Mediziner}|pw}{ }Erzh. Karl\oindex{Wien@\textbf{Wien}!I., Innere Stadt@\textbf{I., Innere Stadt}!Hotel Erzherzog Karl [Wien]@\textbf{Hotel Erzherzog Karl [Wien]}, \emph{Hotel}|pw}{ }Kärnt.\oindex{Wien@\textbf{Wien}!I., Innere Stadt@\textbf{I., Innere Stadt}!Kärntner Straße@\textbf{Kärntner Straße}, \emph{Straße}|pw}«
\newline{}Ordnung: mit Bleistift von unbekannter Hand nummeriert: »33« }\toendnotes[C]{\smallbreak}
\pstart
           \noindent{}{\pb}lieber Arthur, vom Bureau\orgindex{»Phönix« Versicherung@»Phönix« Versicherung|pwv} musste ich nach Hause\oindex{Wien@\textbf{Wien}!IX., Alsergrund@\textbf{IX., Alsergrund}!Währinger Straße@\textbf{Währinger Straße}, \emph{Straße}|pwv}\oindex{Wien@\textbf{Wien}!XVIII., Währing@\textbf{XVIII., Währing}!Währinger Straße@\textbf{Währinger Straße}, \emph{Straße}|pwv} gehen, und liege im Bette. Bitte, seien Sie nicht \label{K_L03130-1v}\edtext{bös’}{\lemma{\textnormal{\emph{bös’}}}\Cendnote{\textnormal{Bezug unklar}}}\label{K_L03130-1}, aber mein Knie thut mir weh, sehr weh.
               Wenn Sie können, so {\pb}\label{K_L03130-2v}\edtext{schauen Sie im Lauf des Tages zu mir}{\lemma{\textnormal{\emph{schauen … mir}}}\Cendnote{\textnormal{Das kann als Indiz dafür genommen werden, dass die bei der Tagesziffer nicht
                  verlässlich lesbare Datierung durch Schnitzler stimmt, da er am 24. 10. 1893 bei Salten\pwindex{Salten, Felix 6.\,9.\,1869 Budapest – 8.\,10.\,1945 Zürich@\textsc{Salten, Felix} (6.\,9.\,1869 Budapest – 8.\,10.\,1945 Zürich), \emph{Schriftsteller, Journalist, Chefredakteur}|pwk}{ }zu Hause\oindex{Wien@\textbf{Wien}!IX., Alsergrund@\textbf{IX., Alsergrund}!Währinger Straße@\textbf{Währinger Straße}, \emph{Straße}|pwkv}\oindex{Wien@\textbf{Wien}!XVIII., Währing@\textbf{XVIII., Währing}!Währinger Straße@\textbf{Währinger Straße}, \emph{Straße}|pwkv} war.}}}\label{K_L03130-2}. Sind
               Sie bei diesem Brief \textcolor{gray}{gu}t! zu Hause\oindex{Wien@\textbf{Wien}!I., Innere Stadt@\textbf{I., Innere Stadt}!Kärntnerring 12/Bösendorferstraße 11@\textbf{Kärntnerring 12/Bösendorferstraße 11}, \emph{Wohngebäude}|pwv}, so senden Sie mir bitte irgend einen Roma\substVorne{}\textsuperscript{m}\substDazwischen{}n\substHinten{}, Korolenko\pwindex{Korolenko, Vladimir Galaktionovič 27.\,7.\,1853 – 25.\,12.\,1921@\textsc{Korolenko, Vladimir Galaktionovič} (27.\,7.\,1853 – 25.\,12.\,1921), \emph{Schriftsteller}|pw}, oder Jacobsen\pwindex{Jacobsen, Jens Peter 7.\,4.\,1847 Thisted – 30.\,4.\,1885 ebd.@\textsc{Jacobsen, Jens Peter} (7.\,4.\,1847 Thisted – 30.\,4.\,1885 ebd.), \emph{Schriftsteller}|pw} oder {\pb}so etwas. Auf
               Wiedersehen.\pend
           
\pstart
           Herzlichst {\\[\baselineskip]}Ihr {\\[\baselineskip]}\spacefill\mbox{Salten}\pend
           \leftskip=0em{}\selectlanguage{ngerman}\endnumbering\briefempfaengerindex{Schnitzler, Arthur@\textsc{Schnitzler, Arthur}!zzzSalten, Felix@\emph{von Felix Salten}!1893-10-241@{{[}24?. 10. 1893{]}}|)be}\mylabel{L03130h}  \newcommand{\dateiname}{L03130}\newcommand{\titel}{Felix Salten an Arthur Schnitzler, [24?. 10. 1893]}\newcommand{\editorInnen}{Martin Anton Müller und Laura Untner}%% latex-leseansicht-abspann.tex
%% Abspann für die Leseansicht.
%% Der Schalter \ifkorrekturansicht ist bereits durch den Vorspann gesetzt.

%% latex-abspann.tex
%% Gemeinsamer Abspann für Korrekturansicht und Leseansicht.
%% Setzt den Schalter \ifkorrekturansicht voraus (gesetzt in den
%% einbindenden Dateien latex-korrekturansicht-abspann.tex bzw.
%% latex-leseansicht-abspann.tex).
%% ---------------------------------------------------------------

\normalsize

% Das esempio-Environment wird nur in der Leseansicht benötigt
\ifkorrekturansicht\else
\newenvironment{esempio}[3]%
{
    \vspace{1.5ex}
    \rlap{\underline{#1}}
    \par
    \setlength{\parindent}{0cm}
    \nopagebreak
    \leftskip=#2cm
    \rightskip=#3cm
}
{
    \par
}
\fi

\doendnotes{C}
\bigskip
\vfill

\clearpage

\footnotesize

\ifkorrekturansicht
  \lohead{\textsc{register}}
\fi

% theindex-Environment neu definieren ohne reledmac
\makeatletter
\renewenvironment{theindex}{%
  \ifkorrekturansicht
    \section*{\indexname}%
  \else
    \subsubsection*{Index der erwähnten Entitäten}%
  \fi
  \setlength{\parindent}{0pt}%
  \setlength{\parskip}{0pt plus 0.3pt}%
  \let\item\@idxitem
}{%
  \ifkorrekturansicht\clearpage\fi
}
\makeatother

\IfFileExists{\jobname-pw.ind}{\input{\jobname-pw.ind}}{}

% Quellenangabe nur in der Leseansicht
\ifkorrekturansicht\else
% Fallback-Definitionen, falls die .tex-Datei \titel etc. nicht gesetzt hat
\providecommand{\titel}{}
\providecommand{\editorInnen}{}
\providecommand{\dateiname}{\jobname}

\vspace{3cm}

\vfill

\footnotesize
\textsc{Quelle}: \titel. Herausgegeben von {\editorInnen}. In: \emph{Arthur Schnitzler: Briefwechsel mit Autorinnen und Autoren}.
 Digitale Edition, https://schnitzler-briefe.acdh.oeaw.ac.at/{\dateiname}.html (Stand \today)
\fi

\end{document}


