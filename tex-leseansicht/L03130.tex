%% latex-korrekturansicht-vorspann.tex
%% Vorspann für die Korrekturansicht.
%% Lädt die gemeinsame Datei latex-vorspann.tex mit gesetztem Schalter.

\newif\ifkorrekturansicht
\korrekturansichttrue

\input{../tex-inputs/latex-vorspann}


\section[Felix Salten an Arthur Schnitzler, {[}24?. 10. 1893{]}]{L03130 Felix Salten an Arthur Schnitzler, {[}24?. 10. 1893{]}}
\nopagebreak\mylabel{L03130v}
\rehead{ }\normalsize\beginnumbering\briefempfaengerindex{Schnitzler, Arthur@\textsc{Schnitzler, Arthur}!zzzSalten, Felix@\emph{von Felix Salten}!1893-10-241@{{[}24?. 10. 1893{]}}|(be}
\toendnotes[C]{\smallbreak\pagebreak[2]}\Standort{CUL, Schnitzler, B 89, A 1.}
\physDesc{Brief, 1 Blatt, 3 Seiten, 348 Zeichen
\newline{}Handschrift: Bleistift, lateinische Kurrent
\newline{}Schnitzler: 1) mit Bleistift datiert: »2\substVorne{}\textsuperscript{\textcolor{gray}{5}}\substDazwischen{}\textcolor{gray}{4}\substHinten{}/X 93«  2) mit Bleistift auf der vierten Seite vermerkt: »{\pb}Dr. v. Bogdanovits\pwindex{Bogdanovits, W. @\textsc{Bogdanovits, W.}, \emph{Mediziner/Medizinerin}|pw}{ }Erzh. Karl\oindex{Hotel Erzherzog Karl [Wien]@\textbf{Hotel Erzherzog Karl [Wien]}, \emph{Hotel (K.HTL)}|pw}{ }Kärnt.\oindex{Kaerntner Strasse@\textbf{Kärntner Straße}, \emph{Straße (K.STR)}|pw}«
\newline{}Ordnung: mit Bleistift von unbekannter Hand nummeriert: »33« }\toendnotes[C]{\smallbreak}
\pstart
           \noindent{}{\pb}lieber Arthur, vom Bureau\orgindex{»Phoenix« Versicherung@»Phönix« Versicherung|pwv} musste ich nach Hause\oindex{Waehringer Strasse@\textbf{Währinger Straße}, \emph{Straße (K.STR)}|pwv} gehen, und liege im Bette. Bitte, seien Sie nicht \label{K_L03130-1v}\edtext{bös’}{\lemma{\textnormal{\emph{bös’}}}\Cendnote{\textnormal{Bezug unklar}}}\label{K_L03130-1}, aber mein Knie thut mir weh, sehr weh.
               Wenn Sie können, so {\pb}\label{K_L03130-2v}\edtext{schauen Sie im Lauf des Tages zu mir}{\lemma{\textnormal{\emph{schauen … mir}}}\Cendnote{\textnormal{Das kann als Indiz dafür genommen werden, dass die bei der Tagesziffer nicht
                  verlässlich lesbare Datierung durch Schnitzler stimmt, da er am 24. 10. 1893 bei Salten\pwindex{Salten, Felix 06.09.1869 – 08.10.1945@\textsc{Salten, Felix} (06.09.1869 – 08.10.1945), \emph{Schriftsteller/Schriftstellerin, Journalist/Journalistin, Chefredakteur/Chefredakteurin}|pwk}{ }zu Hause\oindex{Waehringer Strasse@\textbf{Währinger Straße}, \emph{Straße (K.STR)}|pwkv} war.}}}\label{K_L03130-2}. Sind
               Sie bei diesem Brief \textcolor{gray}{gu}t! zu Hause\oindex{Kaerntnerring 12/Boesendorferstrasse 11@\textbf{Kärntnerring 12/Bösendorferstraße 11}, \emph{Wohngebäude (K.WHS)}|pwv}, so senden Sie mir bitte irgend einen Roma\substVorne{}\textsuperscript{m}\substDazwischen{}n\substHinten{}, Korolenko\pwindex{Korolenko, Vladimir Galaktionovic 1853-07-27 – 1921-12-25@\textsc{Korolenko, Vladimir Galaktionovič} (1853-07-27 – 1921-12-25), \emph{Schriftsteller/Schriftstellerin}|pw}, oder Jacobsen\pwindex{Jacobsen, Jens Peter 07.04.1847 – 30.04.1885@\textsc{Jacobsen, Jens Peter} (07.04.1847 – 30.04.1885), \emph{Schriftsteller/Schriftstellerin}|pw} oder {\pb}so etwas. Auf
               Wiedersehen.\pend
           
\pstart
           Herzlichst {\\[\baselineskip]}Ihr {\\[\baselineskip]}\spacefill\mbox{Salten}\pend
           \leftskip=0em{}\selectlanguage{ngerman}\endnumbering\briefempfaengerindex{Schnitzler, Arthur@\textsc{Schnitzler, Arthur}!zzzSalten, Felix@\emph{von Felix Salten}!1893-10-241@{{[}24?. 10. 1893{]}}|)be}\mylabel{L03130h}  \normalsize

\doendnotes{C}
\bigskip
\vfill

\clearpage

\footnotesize

\lohead{\textsc{register}}

% Definiere theindex-Environment komplett neu ohne reledmac
\makeatletter
\renewenvironment{theindex}{%
  \section*{\indexname}%
  \setlength{\parindent}{0pt}%
  \setlength{\parskip}{0pt plus 0.3pt}%
  \let\item\@idxitem
}{%
  \clearpage
}
\makeatother

\IfFileExists{\jobname-pw.ind}{\input{\jobname-pw.ind}}{}

\end{document}

      