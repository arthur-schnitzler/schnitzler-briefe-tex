%% latex-leseansicht-vorspann.tex
%% Vorspann für die Leseansicht.
%% Lädt die gemeinsame Datei latex-vorspann.tex mit nicht gesetztem Schalter.

\newif\ifkorrekturansicht
\korrekturansichtfalse

\input{../tex-inputs/latex-vorspann}

\begin{center}
            \textcolor{red}{ENTWURF, NICHT FERTIG KORRIGIERT}
                      \end{center}
            
         
         \renewcommand{\erwaehntePersonen}{Personen: W. Bogdanovits, Jens Peter Jacobsen, Vladimir Galaktionovič Korolenko}
         \renewcommand{\erwaehnteOrte}{Orte: Hotel Erzherzog Karl, Kärntner Straße, Wien}
         \renewcommand{\erwaehnteWerke}{}
               \section[Felix Salten an Arthur Schnitzler, {[}24?. 10. 1893{]}]{ Felix Salten an Arthur Schnitzler, {[}24?. 10. 1893{]}}\nopagebreak\mylabel{v}\rehead{ }\begin{ledgroupsized}[t]{13cm}\normalsize\beginnumbering \toendnotes[C]{\smallbreak\pagebreak[2]} \Standort{CUL, Schnitzler, B 89, A 1.}
\physDesc{Brief, 1 Blatt, 3 Seiten
\newline{}Handschrift: Bleistift, lateinische Kurrent
\newline{}Schnitzler: 1) mit Bleistift datiert: »2\substVorne{}\textsuperscript{\textcolor{gray}{5}}\substDazwischen{}\textcolor{gray}{4}\substHinten{}/X 93«  2) mit Bleistift auf der vierten Seite: »{\pb} Dr. v. Bogdanovits\pwindex{Bogdanovits, W. @\textsc{Bogdanovits, W.}, \emph{Mediziner}|pw}{ }Erzh. Karl\oindex{Hotel Erzherzog Karl@\textbf{Hotel Erzherzog Karl}|pw}{ }Kärnt.\oindex{Kaerntner Strasse@\textbf{Kärntner Straße}|pw}«}\toendnotes[C]{\smallbreak}\pstart
           \noindent{}{\pb}Lieber Arthur, vom Bureau musste ich nach Hause gehen, und liege im
               Bette. Bitte, seien Sie nicht bös’, aber mein Knie thut mir weh, sehr weh. Wenn Sie
               können, so {\pb}\label{K_L03130-1v}\edtext{schauen Sie im Lauf des Tages zu
                  mir}{\lemma{\textnormal{\emph{schauen … mir}}}\Cendnote{\textnormal{Das kann als Indiz dafür genommen
                  werden, dass die bei der Tagesziffer nicht verlässlich lesbare Datierung durch Schnitzler\pwindex{Schnitzler, Arthur 15.05.1862 – 21.10.1931@\textsc{Schnitzler, Arthur} (15.05.1862 – 21.10.1931), \emph{Schriftsteller, Mediziner}|pwk} stimmt, da er am 24. 10. 1893 bei Salten\pwindex{Salten, Felix 06.09.1869 – 08.10.1945@\textsc{Salten, Felix} (06.09.1869 – 08.10.1945), \emph{Schriftsteller, Journalist}|pwk} zu Hause war.}}}\label{K_L03130-1h}. Sind Sie bei
               diesem Brief \textcolor{gray}{gu}t! zu Hause, so senden Sie mir bitte irgend einen
                  Roman\strikeout{\textcolor{gray}{×}}, Korolenko\pwindex{Korolenko, Vladimir Galaktionovic 1853-07-27 – 1921-12-25@\textsc{Korolenko, Vladimir Galaktionovič} (1853-07-27 – 1921-12-25), \emph{Schriftsteller}|pw}, oder Jacobsen\pwindex{Jacobsen, Jens Peter 07.04.1847 – 30.04.1885@\textsc{Jacobsen, Jens Peter} (07.04.1847 – 30.04.1885), \emph{Schriftsteller}|pw} oder {\pb}so etwas. Auf Wiedersehen. \pend
           \pstart
           Herzlichst{\\[\baselineskip]}Ihr \spacefill\mbox{Salten}\pend
           \leftskip=0em{}
         
         \endnumbering\mylabel{h}\end{ledgroupsized}\begin{anhang}\end{anhang}\newcommand{\dateiname}{L03130}\newcommand{\titel}{Felix Salten an Arthur Schnitzler, [24?. 10. 1893]}\newcommand{\editorInnen}{Martin Anton Müller und Laura Untner}%% latex-leseansicht-abspann.tex
%% Abspann für die Leseansicht.
%% Der Schalter \ifkorrekturansicht ist bereits durch den Vorspann gesetzt.

%% latex-abspann.tex
%% Gemeinsamer Abspann für Korrekturansicht und Leseansicht.
%% Setzt den Schalter \ifkorrekturansicht voraus (gesetzt in den
%% einbindenden Dateien latex-korrekturansicht-abspann.tex bzw.
%% latex-leseansicht-abspann.tex).
%% ---------------------------------------------------------------

\normalsize

% Das esempio-Environment wird nur in der Leseansicht benötigt
\ifkorrekturansicht\else
\newenvironment{esempio}[3]%
{
    \vspace{1.5ex}
    \rlap{\underline{#1}}
    \par
    \setlength{\parindent}{0cm}
    \nopagebreak
    \leftskip=#2cm
    \rightskip=#3cm
}
{
    \par
}
\fi

\doendnotes{C}
\bigskip
\vfill

\clearpage

\footnotesize

\ifkorrekturansicht
  \lohead{\textsc{register}}
\fi

% theindex-Environment neu definieren ohne reledmac
\makeatletter
\renewenvironment{theindex}{%
  \ifkorrekturansicht
    \section*{\indexname}%
  \else
    \subsubsection*{Index der erwähnten Entitäten}%
  \fi
  \setlength{\parindent}{0pt}%
  \setlength{\parskip}{0pt plus 0.3pt}%
  \let\item\@idxitem
}{%
  \ifkorrekturansicht\clearpage\fi
}
\makeatother

\IfFileExists{\jobname-pw.ind}{\input{\jobname-pw.ind}}{}

% Quellenangabe nur in der Leseansicht
\ifkorrekturansicht\else
% Fallback-Definitionen, falls die .tex-Datei \titel etc. nicht gesetzt hat
\providecommand{\titel}{}
\providecommand{\editorInnen}{}
\providecommand{\dateiname}{\jobname}

\vspace{3cm}

\vfill

\footnotesize
\textsc{Quelle}: \titel. Herausgegeben von {\editorInnen}. In: \emph{Arthur Schnitzler: Briefwechsel mit Autorinnen und Autoren}.
 Digitale Edition, https://schnitzler-briefe.acdh.oeaw.ac.at/{\dateiname}.html (Stand \today)
\fi

\end{document}


      