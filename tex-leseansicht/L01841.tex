%% latex-korrekturansicht-vorspann.tex
%% Vorspann für die Korrekturansicht.
%% Lädt die gemeinsame Datei latex-vorspann.tex mit gesetztem Schalter.

\newif\ifkorrekturansicht
\korrekturansichttrue

\input{../tex-inputs/latex-vorspann}


\section[Arthur Schnitzler an Albert Ehrenstein, 7. 5. 1909]{L01841 Arthur Schnitzler an Albert Ehrenstein, 7. 5. 1909}
\nopagebreak\mylabel{L01841v}
\rehead{ }\normalsize\beginnumbering\briefempfaengerindex{Ehrenstein, Albert@\textsc{Ehrenstein, Albert}!zzzSchnitzler, Arthur@\emph{von Arthur Schnitzler}!1909-05-071@{7. 5. 1909}|(be}
\toendnotes[C]{\smallbreak\pagebreak[2]}\Standort{Jerusalem, The National Library of Israel, ARC. Ms. Var. 306 1 118.}
\physDesc{Brief, 1 Blatt, 2 Seiten, 369 Zeichen
\newline{}Handschrift: schwarze Tinte, deutsche Kurrent}\toendnotes[C]{\smallbreak}
\pstart
           \raggedleft{}{\pb}7. 5. 09\pend
           
\pstart
           \textcolor{gray}{\textbf{Dr. Arthur Schnitzler}}{\\}\textcolor{gray}{\textbf{Wien XVIII. Spoettelgasse 7\oindex{Edmund-Weiss-Gasse 7@\textbf{Edmund-Weiß-Gasse 7}, \emph{Wohngebäude (K.WHS)}|pw}.}}\pend
           
\pstart{}lieber Herr Ehrenſtein,\pend\vspace{0.5em}
\pstart
           zur gefälligen Beruhigung: ich habe Sie Mittwoch nicht geſehen und
               überhaupt nicht, ſeit Sie bei mir waren.\pend
           
\pstart
           Aber es iſt wirklich (verzeihen Sie) kindiſch, ſich über ſolche Vorfälle zu
               excitiren. Arge Verſchwendung von Seelenkräften. Ihre \textsc{Manuscr.}\pwindex{Apaturien@\emph{Apaturien}|pwuv}\pwindex{Tubutsch@\emph{Tubutsch}|pwuv}\pwindex{Tod des Zehir eddin Muhammed Baber@\emph{Tod des Zehir eddin Muhammed Baber}|pwuv} noch nicht geleſen – {\pb}wegen intenſiver Arbeit. Nehmen Sie
               mirs nicht übel.\pend
           
\pstart
           Herzlich grüßend{\\[\baselineskip]}Ihr{\\[\baselineskip]}\spacefill\mbox{A. S.}\pend
           \leftskip=0em{}\selectlanguage{ngerman}\endnumbering\briefempfaengerindex{Ehrenstein, Albert@\textsc{Ehrenstein, Albert}!zzzSchnitzler, Arthur@\emph{von Arthur Schnitzler}!1909-05-071@{7. 5. 1909}|)be}\mylabel{L01841h}  \normalsize

\doendnotes{C}
\bigskip
\vfill

\clearpage

\footnotesize

\lohead{\textsc{register}}

% Definiere theindex-Environment komplett neu ohne reledmac
\makeatletter
\renewenvironment{theindex}{%
  \section*{\indexname}%
  \setlength{\parindent}{0pt}%
  \setlength{\parskip}{0pt plus 0.3pt}%
  \let\item\@idxitem
}{%
  \clearpage
}
\makeatother

\IfFileExists{\jobname-pw.ind}{\input{\jobname-pw.ind}}{}

\end{document}

      