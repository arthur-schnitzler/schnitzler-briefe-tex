\input{../tex-inputs/latex-pdf-vorspann}
\begin{center}
            \textcolor{red}{ENTWURF. ENTZIFFERUNG NOCH NICHT KORREKTURGELESEN}
                      \end{center}
            
               \section[Arthur Schnitzler an Albert Ehrenstein, 7. 5. 1909]{ Arthur Schnitzler an Albert Ehrenstein, 7. 5. 1909}\nopagebreak\mylabel{v}\rehead{ }\begin{ledgroupsized}[t]{13cm}\normalsize\beginnumbering\briefempfaengerindex{Ehrenstein, Albert@\textsc{Ehrenstein, Albert}!zzzSchnitzler, Arthur@\emph{von Arthur Schnitzler}!1909-05-071@{7. 5. 1909}|(be} \toendnotes[C]{\smallbreak\pagebreak[2]} \Standort{Jerusalem, The National Library of Israel, ARC. Ms. Var. 306 1 118.}
\physDesc{Brief, 1 Blatt, 2 Seiten
\newline{}Handschrift: schwarze Tinte, deutsche Kurrent}\toendnotes[C]{\smallbreak}\pstart
           \raggedleft{}{\pb}7. 5. 09\pend
           \pstart
           \textcolor{gray}{\textbf{Dr. Arthur Schnitzler}}{\\}\textcolor{gray}{\textbf{Wien XVIII. Spoettelgasse 7\oindex{Edmund-Weiss-Gasse@\textbf{Edmund-Weiß-Gasse}|pw}.}}\pend
           \pstart{}lieber Herr Ehrenſtein,\pend\pstart
           zur gefälligen Beruhigung: ich habe Sie Mittwoch nicht geſehen und
                    überhaupt nicht, ſeit Sie bei mir waren.\pend
           \pstart
           Aber es iſt wirklich (verzeihen Sie) kindiſch, ſich über ſolche Vorfälle zu
                    excitiren. Arge Verſchwendung von Seelenkräften. Ihre \textsc{Manuscr.}\pwindex{Ehrenstein, Albert 23.12.1886 – 08.04.1950@\textsc{Ehrenstein, Albert} (23.12.1886 – 08.04.1950), \emph{Schriftsteller}!Apaturien1912@\strich\emph{Apaturien} {[}1912{]}|pwuv}\pwindex{Ehrenstein, Albert 23.12.1886 – 08.04.1950@\textsc{Ehrenstein, Albert} (23.12.1886 – 08.04.1950), \emph{Schriftsteller}!Tubutsch1911@\strich\emph{Tubutsch} {[}1911{]}|pwuv}\pwindex{Ehrenstein, Albert 23.12.1886 – 08.04.1950@\textsc{Ehrenstein, Albert} (23.12.1886 – 08.04.1950), \emph{Schriftsteller}!Tod des Zehir eddin Muhammed Baber1912@\strich\emph{Tod des Zehir eddin Muhammed Baber} {[}1912{]}|pwuv} noch nicht geleſen – {\pb}wegen intenſiver Arbeit.
                    Nehmen Sie mirs nicht übel.\pend
           \pstart
           Herzlich grüßend{\\[\baselineskip]}Ihr{\\[\baselineskip]}\spacefill\mbox{A. S.}\pend
           \leftskip=0em{}\endnumbering\briefempfaengerindex{Ehrenstein, Albert@\textsc{Ehrenstein, Albert}!zzzSchnitzler, Arthur@\emph{von Arthur Schnitzler}!1909-05-071@{7. 5. 1909}|)be}\mylabel{h}\end{ledgroupsized}  \newcommand{\dateiname}{L01841}\newcommand{\titel}{Arthur Schnitzler an Albert Ehrenstein, 7. 5. 1909}\newcommand{\editorInnen}{Martin Anton Müller und Gerd-Hermann Susen}\input{../tex-inputs/latex-pdf-abspann}
      