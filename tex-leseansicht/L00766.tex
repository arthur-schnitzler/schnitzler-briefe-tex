%% latex-korrekturansicht-vorspann.tex
%% Vorspann für die Korrekturansicht.
%% Lädt die gemeinsame Datei latex-vorspann.tex mit gesetztem Schalter.

\newif\ifkorrekturansicht
\korrekturansichttrue

\input{../tex-inputs/latex-vorspann}


\section[Georg Brandes an Arthur Schnitzler, 25. 1. 1898]{L00766 Georg Brandes an Arthur Schnitzler, 25. 1. 1898}
\nopagebreak\mylabel{L00766v}
\rehead{ }\normalsize\beginnumbering\briefempfaengerindex{Schnitzler, Arthur@\textsc{Schnitzler, Arthur}!zzzBrandes, Georg@\emph{von Georg Brandes}!1898-01-251@{25. 1. 1898}|(be}
\toendnotes[C]{\smallbreak\pagebreak[2]}\Standort{New York, Leo Baeck Institute, AR–B.C.136,2.}
\physDesc{Kartenbrief, 221 Zeichen
\newline{}Handschrift: schwarze Tinte, lateinische Kurrent
\newline{}Versand: 1) Rohrpost  2) Stempel: »\nobreak{}\oindex{IX., Alsergrund@\textbf{IX., Alsergrund}, \emph{A.ADM3}|pwk}{[}Wien{]} 9/2, 25 1 98, \textcolor{gray}{8} 10\textcolor{gray}{V}\nobreak{}«. }
\buchAbdrucke{\weitereDrucke{Georg Brandes, Arthur Schnitzler: \emph{Ein Briefwechsel}. Bern: \emph{Francke} 1956, S. 66.} }\pstart{}{\pb}Herrn Dr. Arthur Schnitzler\pend{}\pstart{}Frankgasse 1\oindex{Frankgasse 1@\textbf{Frankgasse 1}, \emph{Wohngebäude (K.WHS)}|pw}\pend{}\pstart{}Wien IX\oindex{IX., Alsergrund@\textbf{IX., Alsergrund}, \emph{A.ADM3}|pw}\pend{}{\bigskip}\vspace{1em}
\pstart
           \centering{}{\pb}Residenz-Hotel\oindex{Residenzhotel@\textbf{Residenzhotel}, \emph{Hotel (K.HTL)}|pw}\pend
           
\pstart{}Liebster Herr Doctor\pend\vspace{0.5em}
\pstart
           Ich bin hier und würde mich freuen Sie zu sehen noch heute, wenn es geht. Sagen Sie
               mir ob und wann ich Ihnen willkommen bin.\pend
           
\pstart
           Ihr{\\[\baselineskip]}\spacefill\mbox{Georg Brandes}\pend
           \leftskip=0em{}\selectlanguage{ngerman}\endnumbering\briefempfaengerindex{Schnitzler, Arthur@\textsc{Schnitzler, Arthur}!zzzBrandes, Georg@\emph{von Georg Brandes}!1898-01-251@{25. 1. 1898}|)be}\mylabel{L00766h}  \normalsize

\doendnotes{C}
\bigskip
\vfill

\clearpage

\footnotesize

\lohead{\textsc{register}}

% Definiere theindex-Environment komplett neu ohne reledmac
\makeatletter
\renewenvironment{theindex}{%
  \section*{\indexname}%
  \setlength{\parindent}{0pt}%
  \setlength{\parskip}{0pt plus 0.3pt}%
  \let\item\@idxitem
}{%
  \clearpage
}
\makeatother

\IfFileExists{\jobname-pw.ind}{\input{\jobname-pw.ind}}{}

\end{document}

      