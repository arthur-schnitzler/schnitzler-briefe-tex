%% latex-leseansicht-vorspann.tex
%% Vorspann für die Leseansicht.
%% Lädt die gemeinsame Datei latex-vorspann.tex mit nicht gesetztem Schalter.

\newif\ifkorrekturansicht
\korrekturansichtfalse

\input{../tex-inputs/latex-vorspann}


\section[Theodor Herzl an Arthur Schnitzler, 21. 3. 1895]{L03854 Theodor Herzl an Arthur Schnitzler, 21. 3. 1895}
\nopagebreak\mylabel{L03854v}
\rehead{ }\normalsize\beginnumbering\briefempfaengerindex{Schnitzler, Arthur@\textsc{Schnitzler, Arthur}!zzzHerzl, Theodor@\emph{von Theodor Herzl}!1895-03-213@{21. 3. 1895}|(be}
\toendnotes[C]{\smallbreak\pagebreak[2]}
\correspDesc{Versand  durch Theodor Herzl am 21. 3. 1895 in Paris
\newline{}Erhalt  durch Arthur Schnitzler im Zeitraum [22. 3. 1895
                  – 26. 3. 1895?] in Wien}\toendnotes[C]{\smallbreak}
\Standort{CUL, Schnitzler, B 39.}
\physDesc{Brief, 1 Blatt, 1 Seite, 433 Zeichen
\newline{}Handschrift: schwarze Tinte, lateinische Kurrent
\newline{}Ordnung: mit Bleistift von unbekannter Hand nummeriert: »33« }
\buchAbdrucke{\weitereDrucke{Theodor Herzl: \emph{Briefe und autobiographische Notizen 1866–1895}. Bearbeitet von Johannes Wachten in Zusammenarbeit mit Chaya Harel, Daisy Tycho und Manfred Winkler. Berlin, Frankfurt am Main, Wien: \emph{Propyläen} 1983, S. 579–580 (Briefe und Tagebücher. Herausgegeben von Alex Bein, Hermann Greive, Moshe Schaerf, Julius H. Schoeps und Johannes Wachten, 1).} }\toendnotes[C]{\smallbreak}
\pstart
           \raggedleft{}{\pb}21. III. 95\pend
           
\pstart{}Lieber Freund,\pend\vspace{0.5em}
\pstart
           beiliegende höchst erfreuliche \label{K_L03854-1v}\edtext{Notiz}{\lemma{\textnormal{\emph{Notiz}}}\Cendnote{\textnormal{Die Beilage ist nicht
                  überliefert. Es handelte sich um eine postive Besprechung von Schnitzlers Novelle \emph{Sterben}\pwindex{Schnitzler, Arthur 15.\,5.\,1862 Wien – 21.\,10.\,1931 ebd.@\textsc{Schnitzler, Arthur} (15.\,5.\,1862 Wien – 21.\,10.\,1931 ebd.), \emph{Schriftsteller, Mediziner}!Sterben. Novelle@\strich\emph{Sterben. Novelle}|pwk} in der Rubrik »Au jour le jour«, in der der Schriftsteller dem
                     französischen\oindex{Frankreich@\textbf{Frankreich}|pwk} Publikum als
                  vielversprechende neue deutsche Stimme empfohlen wurde (P. I.: \emph{M. Arthur Schnitzler}. In: \emph{Journal des débats. Politique et littéraires}\pwindex{Journal des débats. Politiques et littéraires@\emph{Journal des débats. Politiques et littéraires}|pwk},
                     21. 3. 1895, S. 1).}}}\label{K_L03854-1} steht heute in den Débats\pwindex{Journal des débats. Politiques et littéraires@\emph{Journal des débats. Politiques et littéraires}|pwv}. Nur so weiter! Ich
               meine das Aeussere – mit dem Innern hat das nichts zu schaffen. Aber es stärkt beim
               Wandern, wenn man ab u. zu einen Schluck aus der Feldflasche des Erfolgs thun kann.\pend
           
\pstart
           Sonntag fahre ich nach Wien\oindex{Wien@\textbf{Wien}, \emph{Verwaltungsgebiet}|pw}, wenn
               nichts dazwischen kommt, bin Montag Abend dort, wohne Hörlgasse 12\oindex{Wien@\textbf{Wien}!IX., Alsergrund@\textbf{IX., Alsergrund}!Hörlgasse 12@\textbf{Hörlgasse 12}, \emph{Wohngebäude}|pw} bei meinen Eltern\pwindex{Herzl, Jakob 14.\,3.\,1837 Zemun – 9.\,6.\,1902 Wien@\textsc{Herzl, Jakob} (14.\,3.\,1837 Zemun – 9.\,6.\,1902 Wien), \emph{Bankdirektor, Großkaufmann}|pwv}\pwindex{Herzl, Jeanette 28.\,7.\,1836 Budapest – 20.\,2.\,1911 Wien@\textsc{Herzl, Jeanette} (28.\,7.\,1836 Budapest – 20.\,2.\,1911 Wien)|pwv} u. besuche Sie
                  \label{K_L03854-2v}\edtext{Dienstag Vormittag}{\lemma{\textnormal{\emph{Dienstag Vormittag}}}\Cendnote{\textnormal{Siehe A. S.: \emph{Tagebuch}, 21. 3. 1895.}}}\label{K_L03854-2}\pend
           \pstart Herzlich Ihr \spacefill\mbox{Th. H.}\pend{}\selectlanguage{ngerman}\endnumbering\briefempfaengerindex{Schnitzler, Arthur@\textsc{Schnitzler, Arthur}!zzzHerzl, Theodor@\emph{von Theodor Herzl}!1895-03-213@{21. 3. 1895}|)be}\mylabel{L03854h}
\begin{anhang}
\end{anhang}\newcommand{\dateiname}{L03854}\newcommand{\titel}{Theodor Herzl an Arthur Schnitzler, 21. 3. 1895}\newcommand{\editorInnen}{Selma Jahnke und Martin Anton Müller}%% latex-leseansicht-abspann.tex
%% Abspann für die Leseansicht.
%% Der Schalter \ifkorrekturansicht ist bereits durch den Vorspann gesetzt.

%% latex-abspann.tex
%% Gemeinsamer Abspann für Korrekturansicht und Leseansicht.
%% Setzt den Schalter \ifkorrekturansicht voraus (gesetzt in den
%% einbindenden Dateien latex-korrekturansicht-abspann.tex bzw.
%% latex-leseansicht-abspann.tex).
%% ---------------------------------------------------------------

\normalsize

% Das esempio-Environment wird nur in der Leseansicht benötigt
\ifkorrekturansicht\else
\newenvironment{esempio}[3]%
{
    \vspace{1.5ex}
    \rlap{\underline{#1}}
    \par
    \setlength{\parindent}{0cm}
    \nopagebreak
    \leftskip=#2cm
    \rightskip=#3cm
}
{
    \par
}
\fi

\doendnotes{C}
\bigskip
\vfill

\clearpage

\footnotesize

\ifkorrekturansicht
  \lohead{\textsc{register}}
\fi

% theindex-Environment neu definieren ohne reledmac
\makeatletter
\renewenvironment{theindex}{%
  \ifkorrekturansicht
    \section*{\indexname}%
  \else
    \subsubsection*{Index der erwähnten Entitäten}%
  \fi
  \setlength{\parindent}{0pt}%
  \setlength{\parskip}{0pt plus 0.3pt}%
  \let\item\@idxitem
}{%
  \ifkorrekturansicht\clearpage\fi
}
\makeatother

\IfFileExists{\jobname-pw.ind}{\input{\jobname-pw.ind}}{}

% Quellenangabe nur in der Leseansicht
\ifkorrekturansicht\else
% Fallback-Definitionen, falls die .tex-Datei \titel etc. nicht gesetzt hat
\providecommand{\titel}{}
\providecommand{\editorInnen}{}
\providecommand{\dateiname}{\jobname}

\vspace{3cm}

\vfill

\footnotesize
\textsc{Quelle}: \titel. Herausgegeben von {\editorInnen}. In: \emph{Arthur Schnitzler: Briefwechsel mit Autorinnen und Autoren}.
 Digitale Edition, https://schnitzler-briefe.acdh.oeaw.ac.at/{\dateiname}.html (Stand \today)
\fi

\end{document}


