\input{../tex-inputs/latex-pdf-vorspann}
\begin{center}
            \textcolor{red}{ENTWURF. ENTZIFFERUNG NOCH NICHT KORREKTURGELESEN}
                      \end{center}
            
               \section[Hugo von Hofmannsthal an Arthur Schnitzler, 27. 3. {[}1914{]}]{ Hugo von Hofmannsthal an Arthur Schnitzler, 27. 3. {[}1914{]}}\nopagebreak\mylabel{v}\rehead{ }\begin{ledgroupsized}[t]{13cm}\normalsize\beginnumbering\briefempfaengerindex{Schnitzler, Arthur@\textsc{Schnitzler, Arthur}!zzzHofmannsthal, Hugo von@\emph{von Hugo von Hofmannsthal}!1914-03-271@{27. 3. {[}1914{]}}|(be} \toendnotes[C]{\smallbreak\pagebreak[2]} \Standort{CUL, Schnitzler, B 43.}
\physDesc{Brief, 1 Blatt, 4 Seiten
\newline{}Handschrift: schwarze Tinte, deutsche Kurrent
\newline{}Schnitzler: mit Bleistift die Jahreszahl ergänzt: »914« und beschriftet: »Hugo« \newline{}Ordnung: 1) mit Bleistift von unbekannter Hand nummeriert: »\strikeout{335}« 2) mit Bleistift von unbekannter Hand nummeriert: »348«}\buchAbdrucke{\weitereDrucke{Hugo von Hofmannsthal, Arthur Schnitzler: \emph{Briefwechsel}. Hg. Therese Nickl und Heinrich Schnitzler. Frankfurt am Main: \emph{S. Fischer} 1964, S. 272–273.} }\toendnotes[C]{\smallbreak}\pstart
           \noindent{}\centering{}{\pb}\textcolor{gray}{\textbf{SÜDBAHN-HOTEL SEMMERING BEI WIEN\oindex{Suedbahnhotel@\textbf{Südbahnhotel}|pw}}}\pend
           \pstart
           \noindent{}\textcolor{gray}{\textbf{ERSTES HOTEL M. 350 ZIMMERN, GESCHÜTZTE, SCHÖNSTE U.
                        KLIMATISCH GÜNSTIGSTE LAGE AM SEMMERING\oindex{Semmering@\textbf{Semmering}|pw} MIT
                        AUSSICHT AUF RAX\oindex{Rax@\textbf{Rax}|pw}, SCHNEEBERG\oindex{Schneeberg@\textbf{Schneeberg}|pw}, EISENBAHNLINIE\orgindex{Suedbahnstrecke@Südbahnstrecke|pwv} ETC. K.K. HAUPTPOST, TELEGRAPHEN-
                        U. TELEPHONAMT IM HOTEL}}\pend
           \pstart
           \textcolor{gray}{\textbf{WINTERKURORT ERSTEN RANGES}}{[}.{]}{ }\textcolor{gray}{\textbf{GRÖSSTER UND VORNEHMSTER WINTERSPORTPLATZ ÖSTERREICHS\oindex{Oesterreich@\textbf{Österreich}|pw}. 2 STUNDEN EISENBAHNFAHRT VON WIEN\oindex{Wien@\textbf{Wien}|pw} UND GRAZ\oindex{Graz@\textbf{Graz}|pw}.}}\pend
           \pstart
           \centering{}\textcolor{gray}{\textbf{TELEGR.- U BRIEF-ADFR: SÜDBAHNHOTEL SEMMERING\oindex{Suedbahnhotel@\textbf{Südbahnhotel}|pw}, TELEPHON SEMMERING 5\oindex{Semmering@\textbf{Semmering}|pw}.}}\pend
           \pstart
           \noindent{}\raggedleft{}\textcolor{gray}{\textbf{Semmering\oindex{Semmering@\textbf{Semmering}|pw}, am}} 27 III.\pend
           \pstart{}mein lieber Arthur \pend\pstart
           Sie haben für den \textsc{Medardus}\pwindex{Schnitzler, Arthur 15.05.1862 – 21.10.1931@\textsc{Schnitzler, Arthur} (15.05.1862 – 21.10.1931), \emph{Schriftsteller, Mediziner}!junge Medardus. Dramatische Historie in einem Vorspiel und fuenf Aufzuegen1910-10-26@\strich\emph{Der junge Medardus. Dramatische Historie in einem Vorspiel und fünf Aufzügen} {[}1910-10-26{]}|pw} einen \label{K_L02167_1v}\edtext{Preis\orgindex{Raimund-Preis@Raimund-Preis|pwv}}{\lemma{\textnormal{\emph{Preis}}}\Cendnote{\textnormal{Die Zeitungen
                  berichteten am 27. 3. 1914 über die Zuerkennung an Schnitzler\pwindex{Schnitzler, Arthur 15.05.1862 – 21.10.1931@\textsc{Schnitzler, Arthur} (15.05.1862 – 21.10.1931), \emph{Schriftsteller, Mediziner}|pwk} und an Rudolf
                     Holzer\pwindex{Holzer, Rudolf 28.07.1875 – 17.07.1965@\textsc{Holzer, Rudolf} (28.07.1875 – 17.07.1965), \emph{Schriftsteller, Journalist}|pwk} für dessen Stück \emph{Gute
                  Mütter}\pwindex{Holzer, Rudolf 28.07.1875 – 17.07.1965@\textsc{Holzer, Rudolf} (28.07.1875 – 17.07.1965), \emph{Schriftsteller, Journalist}!Gute Muetter. Komoedie in drei Akten1913 – 1913@\strich\emph{Gute Mütter. Komödie in drei Akten} {[}1913 – 1913{]}|pwk}.}}}\label{K_L02167_1h} gekriegt, das wird Sie einen Augenblick oder einen Tag lang
               freuen, darum freuts mich auch und ich gratuliere Ihnen – aber vielleicht auch ohne
               dieſen Anlaſs {\pb}hätte ich Ihnen von
               hier geſchrieben, wo wir öfter beiſammen waren und miteinander viele Stunden
               ſpazierengegangen ſind.\pend
           \pstart
           Werden wir nicht ganz allmählich einander zu Schatten, lieber Arthur?\pend
           \pstart
           Und wie kommt es denn? woran liegt es denn?\hspace*{1.5em}Jahre
               und Jahre lang iſt die Aufforderung, einander zu ſehen \uline{immer} von mir, von uns gekommen, immer waren wir die Beſuchenden, die
               Vorſchlagenden – es iſt ganz unwillkürlich {\pb}geſchehen, aber auf einmal, in
               einer Weiſe die man ſich ſelbſt nicht erklären kann, kann in ſo etwas eine Ermüdung
                  ko{\geminationm}en, auf einmal kann man ſich \uline{fühlen} als der, der alleine an dem Draht zieht – man will es
               auch noch weitertun, man will nichts ändern, und doch hat ſich was geändert, man
               fühlts und weiß es kaum, weiß es und ſprichts nicht aus – {\pb}ſo will ichs einmal
               ausſprechen!\pend
           \pstart
           Ich habe eine ſchleppende, nicht gute Zeit hinter mir, hier oben iſts öde und rauh,
               aber doch iſt mir wohler.\pend
           \pstart
           Ich bleibe \label{K_L02167_2v}\edtext{vielleicht noch}{\lemma{\textnormal{\emph{vielleicht noch}}}\Cendnote{\textnormal{Er war um den 18. 3. 1914 angereist und blieb in etwa bis zum
                     4. 4. 1914.}}}\label{K_L02167_2h} 6–8 Tage. Daſs der Zufall es fügte, Sie kämen
               herauf {\dots}?\hspace*{1.5em}Etwa den
                  5–10 April bin ich ſicher wieder unten, den \label{K_L02167_3v}\edtext{10–15 fort}{\lemma{\textnormal{\emph{10–15 fort}}}\Cendnote{\textnormal{In dem Zeitraum machte er eine Reise mit dem Auto
                  durch Nieder-\oindex{Niederoesterreich@\textbf{Niederösterreich}|pwk} und Oberösterreich\oindex{Oberoesterreich@\textbf{Oberösterreich}|pwk}.}}}\label{K_L02167_3h}, wenns Wetter nicht zu unfreundlich iſt, vom
                     16\textsuperscript{ten} an ſicher wieder in Rodaun\oindex{Rodaun@\textbf{Rodaun}|pw}.\hspace*{1.5em}Vielleicht ſteh ich mich dann auch beſſer \strikeout{d} mit meinen Arbeiten, daſs ich Ihnen dann erzählen
               kann.\pend
           \pstart
           Ich habe Sie immer ſehr lieb.{\\[\baselineskip]}Ihr\spacefill\mbox{Hugo.}\pend
           \leftskip=0em{}\endnumbering\briefempfaengerindex{Schnitzler, Arthur@\textsc{Schnitzler, Arthur}!zzzHofmannsthal, Hugo von@\emph{von Hugo von Hofmannsthal}!1914-03-271@{27. 3. {[}1914{]}}|)be}\mylabel{h}\end{ledgroupsized}  \newcommand{\dateiname}{L02167}\newcommand{\titel}{Hugo von Hofmannsthal an Arthur Schnitzler, 27. 3. [1914]}\newcommand{\editorInnen}{Martin Anton Müller und Gerd-Hermann Susen}\input{../tex-inputs/latex-pdf-abspann}
      