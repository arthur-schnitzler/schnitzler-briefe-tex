%% latex-leseansicht-vorspann.tex
%% Vorspann für die Leseansicht.
%% Lädt die gemeinsame Datei latex-vorspann.tex mit nicht gesetztem Schalter.

\newif\ifkorrekturansicht
\korrekturansichtfalse

\input{../tex-inputs/latex-vorspann}


\section[Hugo von Hofmannsthal an Arthur Schnitzler, 27. 3. {[}1914{]}]{L02167 Hugo von Hofmannsthal an Arthur Schnitzler, 27. 3. [1914]}
\nopagebreak\mylabel{L02167v}
\rehead{ }\normalsize\beginnumbering\briefempfaengerindex{Schnitzler, Arthur@\textsc{Schnitzler, Arthur}!zzzHofmannsthal, Hugo von@\emph{von Hugo von Hofmannsthal}!1914-03-271@{27. 3. [1914]}|(be}
\toendnotes[C]{\smallbreak\pagebreak[2]}
\correspDesc{Versand  durch Hugo von Hofmannsthal am 27. 3. [1914] in Semmering
\newline{}Erhalt  durch Arthur Schnitzler im Zeitraum [28. 3. 1914
                  – 1. 4. 1914?] in Wien}\toendnotes[C]{\smallbreak}
\Standort{CUL, Schnitzler, B 43.}
\physDesc{Brief, 1 Blatt, 4 Seiten, 1461 Zeichen
\newline{}Handschrift: schwarze Tinte, deutsche Kurrent
\newline{}Schnitzler: mit Bleistift die Jahreszahl ergänzt: »914« und beschriftet: »Hugo« 
\newline{}Ordnung: 1) mit Bleistift von unbekannter Hand nummeriert: »\strikeout{335}«  2) mit Bleistift von unbekannter Hand nummeriert:
                                    »348«}
\buchAbdrucke{\weitereDrucke{Hugo von Hofmannsthal, Arthur Schnitzler: \emph{Briefwechsel}. Herausgegeben von Therese Nickl und Heinrich Schnitzler. Frankfurt am Main: \emph{S. Fischer} 1964, S. 272–273.} }\toendnotes[C]{\smallbreak}
\pstart
           \centering{}{\pb}\textcolor{gray}{\textbf{SÜDBAHN-HOTEL SEMMERING BEI WIEN\oindex{Südbahnhotel [Semmering]@\textbf{Südbahnhotel [Semmering]}, \emph{Hotel}|pw}}}\pend
           
\pstart
           \textcolor{gray}{\textbf{ERSTES HOTEL M. 350 ZIMMERN, GESCHÜTZTE, SCHÖNSTE U.
                        KLIMATISCH GÜNSTIGSTE LAGE AM SEMMERING\oindex{Semmering@\textbf{Semmering}, \emph{Verwaltungsgebiet}|pw}
                        MIT AUSSICHT AUF RAX\oindex{Rax@\textbf{Rax}, \emph{Berg}|pw}, SCHNEEBERG\oindex{Schneeberg@\textbf{Schneeberg}, \emph{Berg}|pw}, EISENBAHNLINIE\orgindex{Südbahnstrecke@Südbahnstrecke|pwv} ETC. K.K. HAUPTPOST, TELEGRAPHEN-
                        U. TELEPHONAMT IM HOTEL}}\pend
           
\pstart
           \textcolor{gray}{\textbf{WINTERKURORT ERSTEN RANGES}}{[}.{]}{ }\textcolor{gray}{\textbf{GRÖSSTER UND VORNEHMSTER WINTERSPORTPLATZ ÖSTERREICHS\oindex{Österreich@\textbf{Österreich}|pw}. 2 STUNDEN EISENBAHNFAHRT VON WIEN\oindex{Wien@\textbf{Wien}, \emph{Verwaltungsgebiet}|pw} UND GRAZ\oindex{Graz@\textbf{Graz}, \emph{Verwaltungsgebiet}|pw}.}}\pend
           
\pstart
           \centering{}\textcolor{gray}{\textbf{TELEGR.- U BRIEF-ADFR: SÜDBAHNHOTEL SEMMERING\oindex{Südbahnhotel [Semmering]@\textbf{Südbahnhotel [Semmering]}, \emph{Hotel}|pw}, TELEPHON SEMMERING 5\oindex{Semmering@\textbf{Semmering}, \emph{Verwaltungsgebiet}|pw}.}}\pend
           
\pstart
           \raggedleft{}\textcolor{gray}{\textbf{Semmering\oindex{Semmering@\textbf{Semmering}, \emph{Verwaltungsgebiet}|pw}, am}}{ }27 III.\pend
           
\pstart{}mein lieber Arthur\pend\vspace{0.5em}
\pstart
           Sie haben für den \textsc{Medardus}\pwindex{Schnitzler, Arthur 15.\,5.\,1862 Wien – 21.\,10.\,1931 ebd.@\textsc{Schnitzler, Arthur} (15.\,5.\,1862 Wien – 21.\,10.\,1931 ebd.), \emph{Schriftsteller, Mediziner}!junge Medardus. Dramatische Historie in einem Vorspiel und fünf Aufzügen@\strich\emph{Der junge Medardus. Dramatische Historie in einem Vorspiel und fünf Aufzügen}|pw} einen \label{K_L02167-1v}\edtext{Preis\orgindex{Raimund-Preis@Raimund-Preis|pwv}}{\lemma{\textnormal{\emph{Preis}}}\Cendnote{\textnormal{Die Zeitungen berichteten am
                     27. 3. 1914 über die Zuerkennung an Schnitzler und an Rudolf
                     Holzer\pwindex{Holzer, Rudolf 28.\,7.\,1875 Wien – 17.\,7.\,1965 ebd.@\textsc{Holzer, Rudolf} (28.\,7.\,1875 Wien – 17.\,7.\,1965 ebd.), \emph{Schriftsteller, Journalist}|pwk} für dessen Stück \emph{Gute
                  Mütter}\pwindex{Holzer, Rudolf 28.\,7.\,1875 Wien – 17.\,7.\,1965 ebd.@\textsc{Holzer, Rudolf} (28.\,7.\,1875 Wien – 17.\,7.\,1965 ebd.), \emph{Schriftsteller, Journalist}!Gute Mütter. Komödie in drei Akten@\strich\emph{Gute Mütter. Komödie in drei Akten}|pwk}.}}}\label{K_L02167-1} gekriegt, das wird Sie einen Augenblick oder einen Tag lang
               freuen, darum freuts mich auch und ich gratuliere Ihnen – aber vielleicht auch ohne
               dieſen Anlaſs {\pb}hätte ich Ihnen von
               hier geſchrieben, wo wir öfter beiſammen waren und miteinander viele Stunden{ }ſpazierengegangen{ }ſind.\pend
           
\pstart
           Werden wir nicht ganz allmählich einander zu Schatten, lieber Arthur?\pend
           
\pstart
           Und wie kommt es denn? woran liegt es denn?\hspace*{1.5em}Jahre
               und Jahre lang iſt die Aufforderung, einander zu{ }ſehen \uline{immer} von mir, von uns gekommen, immer waren wir die Beſuchenden, die
               Vorſchlagenden – es iſt ganz unwillkürlich {\pb}geſchehen, aber auf einmal, in
               einer Weiſe die man{ }ſich{ }ſelbſt nicht erklären kann, kann in{ }ſo etwas eine Ermüdung
                  ko{\geminationm}en, auf einmal kann man{ }ſich \uline{fühlen} als der, der alleine an dem Draht zieht – man will es
               auch noch weitertun, man will nichts ändern, und doch hat{ }ſich was geändert, man
               fühlts und weiß es kaum, weiß es und{ }ſprichts nicht aus – {\pb}ſo will ichs einmal
               ausſprechen!\pend
           
\pstart
           Ich habe eine{ }ſchleppende, nicht gute Zeit hinter mir, hier oben iſts öde und rauh,
               aber doch iſt mir wohler.\pend
           
\pstart
           Ich bleibe \label{K_L02167-2v}\edtext{vielleicht noch}{\lemma{\textnormal{\emph{vielleicht noch}}}\Cendnote{\textnormal{Er war um den 18. 3. 1914
                  angereist und blieb bis etwa 4. 4. 1914.}}}\label{K_L02167-2} 6–8 Tage. Daſs
               der Zufall es fügte, Sie kämen herauf {\dots}?\hspace*{1.5em}Etwa den 5–10 April bin ich{ }ſicher wieder unten, den \label{K_L02167-3v}\edtext{10–15 fort}{\lemma{\textnormal{\emph{10–15 fort}}}\Cendnote{\textnormal{In dem Zeitraum
                  machte er eine Reise mit dem Auto durch Nieder-\oindex{Niederösterreich@\textbf{Niederösterreich}, \emph{Land}|pwk} und Oberösterreich\oindex{Oberösterreich@\textbf{Oberösterreich}, \emph{Land}|pwk}.}}}\label{K_L02167-3},
               wenns Wetter nicht zu unfreundlich iſt, vom 16\textsuperscript{ten} an{ }ſicher wieder in Rodaun\oindex{Wien@\textbf{Wien}!XXIII., Liesing@\textbf{XXIII., Liesing}!Rodaun@\textbf{Rodaun}, \emph{Region}|pw}.\hspace*{1.5em}Vielleicht{ }ſteh ich mich dann auch beſſer \strikeout{d} mit meinen Arbeiten, daſs ich Ihnen dann erzählen
               kann.\pend
           
\pstart
           Ich habe Sie immer{ }ſehr lieb.{\\[\baselineskip]}Ihr\spacefill\mbox{Hugo.}\pend
           \leftskip=0em{}\selectlanguage{ngerman}\endnumbering\briefempfaengerindex{Schnitzler, Arthur@\textsc{Schnitzler, Arthur}!zzzHofmannsthal, Hugo von@\emph{von Hugo von Hofmannsthal}!1914-03-271@{27. 3. [1914]}|)be}\mylabel{L02167h}  \newcommand{\dateiname}{L02167}\newcommand{\titel}{Hugo von Hofmannsthal an Arthur Schnitzler, 27. 3. [1914]}\newcommand{\editorInnen}{Martin Anton Müller und Gerd-Hermann Susen}%% latex-leseansicht-abspann.tex
%% Abspann für die Leseansicht.
%% Der Schalter \ifkorrekturansicht ist bereits durch den Vorspann gesetzt.

%% latex-abspann.tex
%% Gemeinsamer Abspann für Korrekturansicht und Leseansicht.
%% Setzt den Schalter \ifkorrekturansicht voraus (gesetzt in den
%% einbindenden Dateien latex-korrekturansicht-abspann.tex bzw.
%% latex-leseansicht-abspann.tex).
%% ---------------------------------------------------------------

\normalsize

% Das esempio-Environment wird nur in der Leseansicht benötigt
\ifkorrekturansicht\else
\newenvironment{esempio}[3]%
{
    \vspace{1.5ex}
    \rlap{\underline{#1}}
    \par
    \setlength{\parindent}{0cm}
    \nopagebreak
    \leftskip=#2cm
    \rightskip=#3cm
}
{
    \par
}
\fi

\doendnotes{C}
\bigskip
\vfill

\clearpage

\footnotesize

\ifkorrekturansicht
  \lohead{\textsc{register}}
\fi

% theindex-Environment neu definieren ohne reledmac
\makeatletter
\renewenvironment{theindex}{%
  \ifkorrekturansicht
    \section*{\indexname}%
  \else
    \subsubsection*{Index der erwähnten Entitäten}%
  \fi
  \setlength{\parindent}{0pt}%
  \setlength{\parskip}{0pt plus 0.3pt}%
  \let\item\@idxitem
}{%
  \ifkorrekturansicht\clearpage\fi
}
\makeatother

\IfFileExists{\jobname-pw.ind}{\input{\jobname-pw.ind}}{}

% Quellenangabe nur in der Leseansicht
\ifkorrekturansicht\else
% Fallback-Definitionen, falls die .tex-Datei \titel etc. nicht gesetzt hat
\providecommand{\titel}{}
\providecommand{\editorInnen}{}
\providecommand{\dateiname}{\jobname}

\vspace{3cm}

\vfill

\footnotesize
\textsc{Quelle}: \titel. Herausgegeben von {\editorInnen}. In: \emph{Arthur Schnitzler: Briefwechsel mit Autorinnen und Autoren}.
 Digitale Edition, https://schnitzler-briefe.acdh.oeaw.ac.at/{\dateiname}.html (Stand \today)
\fi

\end{document}


