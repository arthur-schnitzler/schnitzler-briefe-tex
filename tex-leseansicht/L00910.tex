%% latex-leseansicht-vorspann.tex
%% Vorspann für die Leseansicht.
%% Lädt die gemeinsame Datei latex-vorspann.tex mit nicht gesetztem Schalter.

\newif\ifkorrekturansicht
\korrekturansichtfalse

\input{../tex-inputs/latex-vorspann}


\section[Arthur Schnitzler an Hugo von Hofmannsthal, 24. 3. 1899]{L00910 Arthur Schnitzler an Hugo von Hofmannsthal, 24. 3. 1899}
\nopagebreak\mylabel{L00910v}
\rehead{ }\normalsize\beginnumbering\briefempfaengerindex{Hofmannsthal, Hugo von@\textsc{Hofmannsthal, Hugo von}!zzzSchnitzler, Arthur@\emph{von Arthur Schnitzler}!1899-03-241@{{[}24. 3. 1899{]}}|(be}
\toendnotes[C]{\smallbreak\pagebreak[2]}
\correspDesc{Versand  durch Arthur Schnitzler am [24. 3. 1899] in Wien
\newline{}Erhalt  durch Hugo von Hofmannsthal im Zeitraum [25. 3. 1899
                  – 29. 3. 1899?] in Berlin}\toendnotes[C]{\smallbreak}
\Standort{FDH, Hs-30885,81.}
\physDesc{Brief, 1 Blatt, 3 Seiten, 805 Zeichen
\newline{}Handschrift: Bleistift, deutsche Kurrent}
\buchAbdrucke{\weitereDrucke{Hugo von Hofmannsthal, Arthur Schnitzler: \emph{Briefwechsel}. Herausgegeben von Therese Nickl und Heinrich Schnitzler. Frankfurt am Main: \emph{S. Fischer} 1964, S. 121.} }\toendnotes[C]{\smallbreak}
\pstart
           \raggedleft{}{\pb}24/3 99\pend
           \vspace{0.5em}
\pstart
           mein lieber Hugo, we{\geminationn} ich früher nach
                  Berlin\oindex{Berlin@\textbf{Berlin}, \emph{Hauptstadt}|pw} fahre,{ }ſo doch erſt
               Oſtern, mit meinem Bruder\pwindex{Schnitzler, Julius 13.\,7.\,1865 Wien – 29.\,6.\,1939 ebd.@\textsc{Schnitzler, Julius} (13.\,7.\,1865 Wien – 29.\,6.\,1939 ebd.), \emph{Chirurg}|pwv} (\uline{Chirurgencongreſs\orgindex{28. Congress der deutschen Gesellschaft für Chirurgie@28. Congress der deutschen Gesellschaft für Chirurgie|pw}}). Sagen Sie mir, wa{\geminationn} Sie wieder nach Wien\oindex{Wien@\textbf{Wien}, \emph{Verwaltungsgebiet}|pw} kommen. Vielleicht fahr ich morgen nach Graz\oindex{Graz@\textbf{Graz}, \emph{Verwaltungsgebiet}|pw}, dort{ }ſind jetzt ihre\pwindex{Reinhard, Marie 13.\,3.\,1871 Wien – 18.\,3.\,1899 ebd.@\textsc{Reinhard, Marie} (13.\,3.\,1871 Wien – 18.\,3.\,1899 ebd.), \emph{Gesangspädagogin}|pwv}{ }Eltern\pwindex{Reinhard, Carl 1.\,3.\,1868 Wien – 29.\,9.\,1904 ebd.@\textsc{Reinhard, Carl} (1.\,3.\,1868 Wien – 29.\,9.\,1904 ebd.), \emph{Kapellmeister}|pwv}\pwindex{Reinhard, Therese 13.\,12.\,1844 Wien – 25.\,3.\,1926 ebd.@\textsc{Reinhard, Therese} (13.\,12.\,1844 Wien – 25.\,3.\,1926 ebd.)|pwv}. Es brennt
               in mir weiter, ganz wie we{\geminationn} alles von dem {\pb}tobenden Schmerz aufgefreſſen werden{ }ſollte. Nie nie
               verſteht man es.\pend
           
\pstart
           Sie machen{ }ſich doch nichts daraus, dſs Ihre Stücke\pwindex{Hofmannsthal, Hugo von 1.\,2.\,1874 Wien – 15.\,7.\,1929 Rodaun@\textsc{Hofmannsthal, Hugo von} (1.\,2.\,1874 Wien – 15.\,7.\,1929 Rodaun), \emph{Schriftsteller}!Hochzeit der Sobeide@\strich\emph{Die Hochzeit der Sobeide}|pwv}\pwindex{Hofmannsthal, Hugo von 1.\,2.\,1874 Wien – 15.\,7.\,1929 Rodaun@\textsc{Hofmannsthal, Hugo von} (1.\,2.\,1874 Wien – 15.\,7.\,1929 Rodaun), \emph{Schriftsteller}!Abenteurer und die Sängerin oder Die Geschenke des Lebens@\strich\emph{Der Abenteurer und die Sängerin oder Die Geschenke des Lebens}|pwv} in B.\oindex{Berlin@\textbf{Berlin}, \emph{Hauptstadt}|pw} nicht gegangen{ }ſind; hoff ich.\pend
           
\pstart
           Wie{ }ſoll das mit meinen\pwindex{Schnitzler, Arthur 15.\,5.\,1862 Wien – 21.\,10.\,1931 ebd.@\textsc{Schnitzler, Arthur} (15.\,5.\,1862 Wien – 21.\,10.\,1931 ebd.), \emph{Schriftsteller, Mediziner}!grüne Kakadu – Paracelsus – Die Gefährtin. Drei Einakter@\strich\emph{Der grüne Kakadu – Paracelsus – Die Gefährtin. Drei Einakter}|pwv} in B.\oindex{Berlin@\textbf{Berlin}, \emph{Hauptstadt}|pw} werden. Jeder Satz iſt beinah eine
               gemeinſchaftliche Erinnerung – wie jeder Gedanke dieſer vier {\pb}Jahre, wie jedes Haus, jeder Stein, jeder Menſch in Wien\oindex{Wien@\textbf{Wien}, \emph{Verwaltungsgebiet}|pw}; wie meine ganze Existenz. –\pend
           
\pstart
           Schreiben Sie mir bitte wie Sie leben, wen Sie{ }ſehen.\pend
           
\pstart
           Ihr Vater\pwindex{Hofmannsthal, Hugo August von 21.\,12.\,1841 Wien – 8.\,12.\,1915 ebd.@\textsc{Hofmannsthal, Hugo August von} (21.\,12.\,1841 Wien – 8.\,12.\,1915 ebd.), \emph{Bankdirektor}|pwv} war bei mir, ich
               aber nicht zu Haus. Viel bin ich mit Guſt.
                  Schw.\pwindex{Schwarzkopf, Gustav 7.\,11.\,1853 Wien – 13.\,11.\,1939 ebd.@\textsc{Schwarzkopf, Gustav} (7.\,11.\,1853 Wien – 13.\,11.\,1939 ebd.), \emph{Schriftsteller}|pw} zuſa{\geminationm}en, auch mit Richard\pwindex{Beer-Hofmann, Richard 11.\,7.\,1866 Wien – 26.\,9.\,1945 New York City@\textsc{Beer-Hofmann, Richard} (11.\,7.\,1866 Wien – 26.\,9.\,1945 New York City), \emph{Schriftsteller}|pw}, Salten\pwindex{Salten, Felix 6.\,9.\,1869 Budapest – 8.\,10.\,1945 Zürich@\textsc{Salten, Felix} (6.\,9.\,1869 Budapest – 8.\,10.\,1945 Zürich), \emph{Schriftsteller, Journalist, Chefredakteur}|pw}.\pend
           
\pstart
           Von Herzen Ihr{\\[\baselineskip]}\spacefill\mbox{Arth}\pend
           \leftskip=0em{}\selectlanguage{ngerman}\endnumbering\briefempfaengerindex{Hofmannsthal, Hugo von@\textsc{Hofmannsthal, Hugo von}!zzzSchnitzler, Arthur@\emph{von Arthur Schnitzler}!1899-03-241@{{[}24. 3. 1899{]}}|)be}\mylabel{L00910h}  \newcommand{\dateiname}{L00910}\newcommand{\titel}{Arthur Schnitzler an Hugo von Hofmannsthal, 24. 3. 1899}\newcommand{\editorInnen}{Martin Anton Müller und Gerd-Hermann Susen}%% latex-leseansicht-abspann.tex
%% Abspann für die Leseansicht.
%% Der Schalter \ifkorrekturansicht ist bereits durch den Vorspann gesetzt.

%% latex-abspann.tex
%% Gemeinsamer Abspann für Korrekturansicht und Leseansicht.
%% Setzt den Schalter \ifkorrekturansicht voraus (gesetzt in den
%% einbindenden Dateien latex-korrekturansicht-abspann.tex bzw.
%% latex-leseansicht-abspann.tex).
%% ---------------------------------------------------------------

\normalsize

% Das esempio-Environment wird nur in der Leseansicht benötigt
\ifkorrekturansicht\else
\newenvironment{esempio}[3]%
{
    \vspace{1.5ex}
    \rlap{\underline{#1}}
    \par
    \setlength{\parindent}{0cm}
    \nopagebreak
    \leftskip=#2cm
    \rightskip=#3cm
}
{
    \par
}
\fi

\doendnotes{C}
\bigskip
\vfill

\clearpage

\footnotesize

\ifkorrekturansicht
  \lohead{\textsc{register}}
\fi

% theindex-Environment neu definieren ohne reledmac
\makeatletter
\renewenvironment{theindex}{%
  \ifkorrekturansicht
    \section*{\indexname}%
  \else
    \subsubsection*{Index der erwähnten Entitäten}%
  \fi
  \setlength{\parindent}{0pt}%
  \setlength{\parskip}{0pt plus 0.3pt}%
  \let\item\@idxitem
}{%
  \ifkorrekturansicht\clearpage\fi
}
\makeatother

\IfFileExists{\jobname-pw.ind}{\input{\jobname-pw.ind}}{}

% Quellenangabe nur in der Leseansicht
\ifkorrekturansicht\else
% Fallback-Definitionen, falls die .tex-Datei \titel etc. nicht gesetzt hat
\providecommand{\titel}{}
\providecommand{\editorInnen}{}
\providecommand{\dateiname}{\jobname}

\vspace{3cm}

\vfill

\footnotesize
\textsc{Quelle}: \titel. Herausgegeben von {\editorInnen}. In: \emph{Arthur Schnitzler: Briefwechsel mit Autorinnen und Autoren}.
 Digitale Edition, https://schnitzler-briefe.acdh.oeaw.ac.at/{\dateiname}.html (Stand \today)
\fi

\end{document}


