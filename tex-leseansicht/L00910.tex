%% latex-leseansicht-vorspann.tex
%% Vorspann für die Leseansicht.
%% Lädt die gemeinsame Datei latex-vorspann.tex mit nicht gesetztem Schalter.

\newif\ifkorrekturansicht
\korrekturansichtfalse

\input{../tex-inputs/latex-vorspann}


         
         \renewcommand{\erwaehntePersonen}{Personen: Richard Beer-Hofmann, Hugo von Hofmannsthal, Hugo August von Hofmannsthal, Marie Reinhard, Carl Reinhard, Therese Reinhard, Felix Salten, Julius Schnitzler, Gustav Schwarzkopf}
         \renewcommand{\erwaehnteInstitutionen}{Institutionen: 28. Congress der deutschen Gesellschaft für Chirurgie}
         \renewcommand{\erwaehnteOrte}{Orte: Berlin, Graz, Wien}
         \renewcommand{\erwaehnteWerke}{Werke: Der Abenteurer und die Sängerin oder Die Geschenke des Lebens, Der grüne Kakadu – Paracelsus – Die Gefährtin. Drei Einakter, Die Hochzeit der Sobeide}
               \section[Arthur Schnitzler an Hugo von Hofmannsthal, 24. 3. 1899]{ Arthur Schnitzler an Hugo von Hofmannsthal, 24. 3. 1899}\nopagebreak\mylabel{v}\rehead{ }\begin{ledgroupsized}[t]{13cm}\normalsize\beginnumbering \toendnotes[C]{\smallbreak\pagebreak[2]} \Standort{FDH, Hs-30885,81.}
\physDesc{Brief, 1 Blatt, 3 Seiten
\newline{}Handschrift: Bleistift, deutsche Kurrent}\buchAbdrucke{\weitereDrucke{Hugo von Hofmannsthal, Arthur Schnitzler: \emph{Briefwechsel}. Hg. Therese Nickl und Heinrich Schnitzler. Frankfurt am Main: \emph{S. Fischer} 1964, S. 121.} }\toendnotes[C]{\smallbreak}\pstart
           \raggedleft{}{\pb}24/3 99\pend
           \pstart
           mein lieber Hugo, we{\geminationn} ich früher
                    nach Berlin\oindex{Berlin@\textbf{Berlin}|pw} fahre, ſo doch erſt
                        Oſtern, mit meinem Bruder\pwindex{Schnitzler, Julius 13.07.1865 – 29.06.1939@\textsc{Schnitzler, Julius} (13.07.1865 – 29.06.1939), \emph{Chirurg}|pwv} (\uline{Chirurgencongreſs\orgindex{28. Congress der deutschen Gesellschaft fuer Chirurgie@28. Congress der deutschen Gesellschaft für Chirurgie|pw}}). Sagen Sie mir, wa{\geminationn} Sie wieder nach Wien\oindex{Wien@\textbf{Wien}|pw} kommen. Vielleicht fahr ich morgen nach Graz\oindex{Graz@\textbf{Graz}|pw}, dort ſind jetzt ihre\pwindex{Reinhard, Marie 1871-03-13 – 1899-03-18@\textsc{Reinhard, Marie} (1871-03-13 – 1899-03-18), \emph{Gesangspädagogin}|pwv}{ }Eltern\pwindex{Reinhard, Carl 01.03.1868 – 1904-09-29@\textsc{Reinhard, Carl} (01.03.1868 – 1904-09-29), \emph{Kapellmeister}|pwv}\pwindex{Reinhard, Therese 13.12.1844 – 25.03.1926@\textsc{Reinhard, Therese} (13.12.1844 – 25.03.1926)|pwv}. Es brennt
                    in mir weiter, ganz wie we{\geminationn} alles von dem {\pb}tobenden Schmerz aufgefreſſen werden ſollte. Nie nie
                    verſteht man es.\pend
           \pstart
           Sie machen ſich doch nichts daraus, dſs Ihre Stücke\pwindex{Hofmannsthal, Hugo von 1874-02-01 – 1929-07-15@\textsc{Hofmannsthal, Hugo von} (1874-02-01 – 1929-07-15), \emph{Schriftsteller}!Hochzeit der Sobeide1899-03-18@\strich\emph{Die Hochzeit der Sobeide} {[}1899-03-18{]}|pwv}\pwindex{Hofmannsthal, Hugo von 1874-02-01 – 1929-07-15@\textsc{Hofmannsthal, Hugo von} (1874-02-01 – 1929-07-15), \emph{Schriftsteller}!Abenteurer und die Saengerin oder Die Geschenke des Lebens18. 3. 1899@\strich\emph{Der Abenteurer und die Sängerin oder Die Geschenke des Lebens} {[}18. 3. 1899{]}|pwv} in B.\oindex{Berlin@\textbf{Berlin}|pw} nicht
                    gegangen ſind; hoff ich.\pend
           \pstart
           Wie ſoll das mit meinen\pwindex{Schnitzler, Arthur 15.05.1862 – 21.10.1931@\textsc{Schnitzler, Arthur} (15.05.1862 – 21.10.1931), \emph{Schriftsteller, Mediziner}!gruene Kakadu – Paracelsus – Die Gefaehrtin. Drei Einakter1898 – 1899@\strich\emph{Der grüne Kakadu – Paracelsus – Die Gefährtin. Drei Einakter} {[}1898 – 1899{]}|pwv} in
                        B.\oindex{Berlin@\textbf{Berlin}|pw} werden. Jeder Satz iſt beinah eine
                    gemeinſchaftliche Erinnerung – wie jeder Gedanke dieſer vier {\pb}Jahre, wie jedes Haus, jeder Stein, jeder Menſch in
                        Wien\oindex{Wien@\textbf{Wien}|pw}; wie meine ganze Existenz. –\pend
           \pstart
           Schreiben Sie mir bitte wie Sie leben, wen Sie ſehen.\pend
           \pstart
           Ihr Vater\pwindex{Hofmannsthal, Hugo August von 21.12.1841 – 08.12.1915@\textsc{Hofmannsthal, Hugo August von} (21.12.1841 – 08.12.1915), \emph{Bankdirektor}|pwv} war bei mir, ich
                    aber nicht zu Haus. Viel bin ich mit Guſt.
                        Schw.\pwindex{Schwarzkopf, Gustav 07.11.1853 – 13.11.1939@\textsc{Schwarzkopf, Gustav} (07.11.1853 – 13.11.1939), \emph{Schriftsteller}|pw} zuſa{\geminationm}en, auch mit Richard\pwindex{Beer-Hofmann, Richard 1866-07-11 – 1945-09-26@\textsc{Beer-Hofmann, Richard} (1866-07-11 – 1945-09-26), \emph{Schriftsteller}|pw}, Salten\pwindex{Salten, Felix 06.09.1869 – 08.10.1945@\textsc{Salten, Felix} (06.09.1869 – 08.10.1945), \emph{Schriftsteller, Journalist}|pw}.\pend
           \pstart
           Von Herzen Ihr{\\[\baselineskip]}\spacefill\mbox{Arth}\pend
           \leftskip=0em{}
         
         \endnumbering\mylabel{h}\end{ledgroupsized}  \newcommand{\dateiname}{L00910}\newcommand{\titel}{Arthur Schnitzler an Hugo von Hofmannsthal, 24. 3. 1899}\newcommand{\editorInnen}{Martin Anton Müller und Gerd-Hermann Susen}%% latex-leseansicht-abspann.tex
%% Abspann für die Leseansicht.
%% Der Schalter \ifkorrekturansicht ist bereits durch den Vorspann gesetzt.

%% latex-abspann.tex
%% Gemeinsamer Abspann für Korrekturansicht und Leseansicht.
%% Setzt den Schalter \ifkorrekturansicht voraus (gesetzt in den
%% einbindenden Dateien latex-korrekturansicht-abspann.tex bzw.
%% latex-leseansicht-abspann.tex).
%% ---------------------------------------------------------------

\normalsize

% Das esempio-Environment wird nur in der Leseansicht benötigt
\ifkorrekturansicht\else
\newenvironment{esempio}[3]%
{
    \vspace{1.5ex}
    \rlap{\underline{#1}}
    \par
    \setlength{\parindent}{0cm}
    \nopagebreak
    \leftskip=#2cm
    \rightskip=#3cm
}
{
    \par
}
\fi

\doendnotes{C}
\bigskip
\vfill

\clearpage

\footnotesize

\ifkorrekturansicht
  \lohead{\textsc{register}}
\fi

% theindex-Environment neu definieren ohne reledmac
\makeatletter
\renewenvironment{theindex}{%
  \ifkorrekturansicht
    \section*{\indexname}%
  \else
    \subsubsection*{Index der erwähnten Entitäten}%
  \fi
  \setlength{\parindent}{0pt}%
  \setlength{\parskip}{0pt plus 0.3pt}%
  \let\item\@idxitem
}{%
  \ifkorrekturansicht\clearpage\fi
}
\makeatother

\IfFileExists{\jobname-pw.ind}{\input{\jobname-pw.ind}}{}

% Quellenangabe nur in der Leseansicht
\ifkorrekturansicht\else
% Fallback-Definitionen, falls die .tex-Datei \titel etc. nicht gesetzt hat
\providecommand{\titel}{}
\providecommand{\editorInnen}{}
\providecommand{\dateiname}{\jobname}

\vspace{3cm}

\vfill

\footnotesize
\textsc{Quelle}: \titel. Herausgegeben von {\editorInnen}. In: \emph{Arthur Schnitzler: Briefwechsel mit Autorinnen und Autoren}.
 Digitale Edition, https://schnitzler-briefe.acdh.oeaw.ac.at/{\dateiname}.html (Stand \today)
\fi

\end{document}


      