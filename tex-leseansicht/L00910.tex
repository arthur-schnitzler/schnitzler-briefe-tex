%% latex-korrekturansicht-vorspann.tex
%% Vorspann für die Korrekturansicht.
%% Lädt die gemeinsame Datei latex-vorspann.tex mit gesetztem Schalter.

\newif\ifkorrekturansicht
\korrekturansichttrue

\input{../tex-inputs/latex-vorspann}


\section[Arthur Schnitzler an Hugo von Hofmannsthal, 24. 3. 1899]{L00910 Arthur Schnitzler an Hugo von Hofmannsthal, 24. 3. 1899}
\nopagebreak\mylabel{L00910v}
\rehead{ }\normalsize\beginnumbering\briefempfaengerindex{Hofmannsthal, Hugo von@\textsc{Hofmannsthal, Hugo von}!zzzSchnitzler, Arthur@\emph{von Arthur Schnitzler}!1899-03-241@{{[}24. 3. 1899{]}}|(be}
\toendnotes[C]{\smallbreak\pagebreak[2]}\Standort{FDH, Hs-30885,81.}
\physDesc{Brief, 1 Blatt, 3 Seiten, 805 Zeichen
\newline{}Handschrift: Bleistift, deutsche Kurrent}
\buchAbdrucke{\weitereDrucke{Hugo von Hofmannsthal, Arthur Schnitzler: \emph{Briefwechsel}. Frankfurt am Main: \emph{S. Fischer} 1964, S. 121.} }\toendnotes[C]{\smallbreak}
\pstart
           \raggedleft{}{\pb}24/3 99\pend
           \vspace{0.5em}
\pstart
           mein lieber Hugo, we{\geminationn} ich früher nach
                  Berlin\oindex{Berlin@\textbf{Berlin}, \emph{P.PPLC}|pw} fahre, ſo doch erſt
               Oſtern, mit meinem Bruder\pwindex{Schnitzler, Julius 13.07.1865 – 29.06.1939@\textsc{Schnitzler, Julius} (13.07.1865 – 29.06.1939), \emph{Chirurg/Chirurgin}|pwv} (\uline{Chirurgencongreſs\orgindex{28. Congress der deutschen Gesellschaft fuer Chirurgie@28. Congress der deutschen Gesellschaft für Chirurgie|pw}}). Sagen Sie mir, wa{\geminationn} Sie wieder nach Wien\oindex{Wien@\textbf{Wien}, \emph{A.ADM2}|pw} kommen. Vielleicht fahr ich morgen nach Graz\oindex{Graz@\textbf{Graz}, \emph{A.ADM2}|pw}, dort ſind jetzt ihre\pwindex{Reinhard, Marie 1871-03-13 – 1899-03-18@\textsc{Reinhard, Marie} (1871-03-13 – 1899-03-18), \emph{Gesangspädagoge/Gesangspädagogin}|pwv}{ }Eltern\pwindex{Reinhard, Carl 01.03.1868 – 1904-09-29@\textsc{Reinhard, Carl} (01.03.1868 – 1904-09-29), \emph{Kapellmeister/Kapellmeisterin}|pwv}\pwindex{Reinhard, Therese 13.12.1844 – 25.03.1926@\textsc{Reinhard, Therese} (13.12.1844 – 25.03.1926)|pwv}. Es brennt
               in mir weiter, ganz wie we{\geminationn} alles von dem {\pb}tobenden Schmerz aufgefreſſen werden ſollte. Nie nie
               verſteht man es.\pend
           
\pstart
           Sie machen ſich doch nichts daraus, dſs Ihre Stücke\pwindex{Hochzeit der Sobeide@\emph{Die Hochzeit der Sobeide}|pwv}\pwindex{Abenteurer und die Saengerin oder Die Geschenke des Lebens@\emph{Der Abenteurer und die Sängerin oder Die Geschenke des Lebens}|pwv} in B.\oindex{Berlin@\textbf{Berlin}, \emph{P.PPLC}|pw} nicht gegangen ſind; hoff ich.\pend
           
\pstart
           Wie ſoll das mit meinen\pwindex{gruene Kakadu – Paracelsus – Die Gefaehrtin. Drei Einakter@\emph{Der grüne Kakadu – Paracelsus – Die Gefährtin. Drei Einakter}|pwv} in B.\oindex{Berlin@\textbf{Berlin}, \emph{P.PPLC}|pw} werden. Jeder Satz iſt beinah eine
               gemeinſchaftliche Erinnerung – wie jeder Gedanke dieſer vier {\pb}Jahre, wie jedes Haus, jeder Stein, jeder Menſch in Wien\oindex{Wien@\textbf{Wien}, \emph{A.ADM2}|pw}; wie meine ganze Existenz. –\pend
           
\pstart
           Schreiben Sie mir bitte wie Sie leben, wen Sie ſehen.\pend
           
\pstart
           Ihr Vater\pwindex{Hofmannsthal, Hugo August von 21.12.1841 – 08.12.1915@\textsc{Hofmannsthal, Hugo August von} (21.12.1841 – 08.12.1915), \emph{Bankdirektor/Bankdirektorin}|pwv} war bei mir, ich
               aber nicht zu Haus. Viel bin ich mit Guſt.
                  Schw.\pwindex{Schwarzkopf, Gustav 07.11.1853 – 13.11.1939@\textsc{Schwarzkopf, Gustav} (07.11.1853 – 13.11.1939), \emph{Schriftsteller/Schriftstellerin}|pw} zuſa{\geminationm}en, auch mit Richard\pwindex{Beer-Hofmann, Richard 1866-07-11 – 1945-09-26@\textsc{Beer-Hofmann, Richard} (1866-07-11 – 1945-09-26), \emph{Schriftsteller/Schriftstellerin}|pw}, Salten\pwindex{Salten, Felix 06.09.1869 – 08.10.1945@\textsc{Salten, Felix} (06.09.1869 – 08.10.1945), \emph{Schriftsteller/Schriftstellerin, Journalist/Journalistin, Chefredakteur/Chefredakteurin}|pw}.\pend
           
\pstart
           Von Herzen Ihr{\\[\baselineskip]}\spacefill\mbox{Arth}\pend
           \leftskip=0em{}\selectlanguage{ngerman}\endnumbering\briefempfaengerindex{Hofmannsthal, Hugo von@\textsc{Hofmannsthal, Hugo von}!zzzSchnitzler, Arthur@\emph{von Arthur Schnitzler}!1899-03-241@{{[}24. 3. 1899{]}}|)be}\mylabel{L00910h}  \normalsize

\doendnotes{C}
\bigskip
\vfill

\clearpage

\footnotesize

\lohead{\textsc{register}}

% Definiere theindex-Environment komplett neu ohne reledmac
\makeatletter
\renewenvironment{theindex}{%
  \section*{\indexname}%
  \setlength{\parindent}{0pt}%
  \setlength{\parskip}{0pt plus 0.3pt}%
  \let\item\@idxitem
}{%
  \clearpage
}
\makeatother

\IfFileExists{\jobname-pw.ind}{\input{\jobname-pw.ind}}{}

\end{document}

      