%% latex-korrekturansicht-vorspann.tex
%% Vorspann für die Korrekturansicht.
%% Lädt die gemeinsame Datei latex-vorspann.tex mit gesetztem Schalter.

\newif\ifkorrekturansicht
\korrekturansichttrue

\input{../tex-inputs/latex-vorspann}


\section[Arthur Schnitzler an Georg Brandes, 4. 7. 1908]{L01779 Arthur Schnitzler an Georg Brandes, 4. 7. 1908}
\nopagebreak\mylabel{L01779v}
\rehead{ }\normalsize\beginnumbering\briefempfaengerindex{Brandes, Georg@\textsc{Brandes, Georg}!zzzSchnitzler, Arthur@\emph{von Arthur Schnitzler}!1908-07-041@{4. 7. 1908}|(be}
\toendnotes[C]{\smallbreak\pagebreak[2]}\Standort{Kopenhagen, Det Kongelige Bibliotek, Georg Brandes Arkiv, box 125.}
\physDesc{Brief, 2 Blätter, 8 Seiten, 4785 Zeichen
\newline{}Handschrift: schwarze Tinte, lateinische Kurrent
\newline{}Ordnung: 1) mit Bleistift das Datum auf beiden Blättern wiederholt: »4/7 08«  2) mit Bleistift von unbekannter Hand
                                    nummeriert: »29.«}
\buchAbdrucke{\weitereDrucke{1) Georg Brandes, Arthur Schnitzler: \emph{Ein Briefwechsel}. Bern: \emph{Francke} 1956, S. 96–98.} \weitereDrucke{2) Arthur Schnitzler: \emph{Briefe 1875–1912}. Frankfurt am Main: \emph{S. Fischer} 1981, S. 578–580.} }\toendnotes[C]{\smallbreak}
\pstart
           {\pb}\textcolor{gray}{\textbf{Dr. Arthur Schnitzler}}\hfill Seis am Schlern\oindex{Seis am Schlern@\textbf{Seis am Schlern}, \emph{P.PPL}|pw}{ }\introOben{}(Südtirol\oindex{Suedtirol@\textbf{Südtirol}, \emph{A.ADM2}|pw})\introOben{}\pend
           
\pstart
           \textcolor{gray}{\textbf{Wien XVIII. Spoettelgasse 7\oindex{Edmund-Weiss-Gasse 7@\textbf{Edmund-Weiß-Gasse 7}, \emph{Wohngebäude (K.WHS)}|pw}.}}\hfill 4/7 908\pend
           \vspace{0.5em}
\pstart
           verehrtester Herr Brandes, Sie haben wohl recht, dass in meinem Buch\pwindex{Weg ins Freie. Roman@\emph{Der Weg ins Freie. Roman}|pwv} zwei Romane enthalten
               sind, und dass künstlerisch geno{\geminationm}en der Zusa{\geminationm}enhang kein absolut notwendiger sein mag. (Ich tröste
               mich gleich damit, dass andre Autoren manchmal auch glauben, sie hätten einen Roman
               geschrieben – und es ist gar keiner) Schon während meiner Arbeit hab ich immer
               gefühlt, dass es so ko{\geminationm}en wird – aber ich konnte – oder
               wollte mir nicht helfen. Denn so sorgfältig das Buch componirt ist, es ist doch erst
               so recht ge{\pb}\uline{worden}, während ich es schrieb. Denken Sie, was
               eigentlich der Kern war, um den sich allmälig das ganze gruppirte: Eine Scene, in der
               ein thörichter Bruder den Geliebten seiner Schwester als »Verführer« zur Rede stellt
               und von ihm glänzend geschlagen wird. Es hätte damals, als mir dieser kleine Einfall
               kam, ein Stück werden sollen. (Dieser ganze Einfall ist jetzt in einem beinah
               überflüssigen Sätzchen des 5. Capitels enthalten.) Dann schwebte mir eine Novelle
               vor: ein junges Mädchen, das sich aus theoretischen Gründen zu einem Geliebten
               entschliesst und sich in ihre Stellung nicht hineinfinden kann. Dann spukte mir eine
               Komödie im Kopf, mit dem Titel {\pb}die Entrüsteten\pwindex{Weg ins Freie. Roman@\emph{Der Weg ins Freie. Roman}|pwv}, wofür schon die
               meisten Figuren, die sich jetzt im Roman vorfinden, feststanden, und noch einige
               andre. Nun dürfen Sie natürlich nicht glauben, dass ich diese Einfälle und Vorsätze
               sozusagen mit Absicht ineinander verschmolzen habe – sondern sie flossen ineinander,
               ganz ohne mein Zuthun – sodaß ich unmöglich daran hätte etwas ändern können. Ich habe
               nichts hineingestopft, weil ich eben Gelegenheit suchte, gewisse Ansichten oder
               Aphorismen anzubringen – sondern im Laufe der Erzählung, vielmehr schon während der
               Vorarbeiten, war jede Gestalt mit ihren Anschauungen dahingerückt, wo sie nun stehen
               geblieben ist. Mir war {\pb}das Verhältnis Georgs\pwindex{Weg ins Freie. Roman@\emph{Der Weg ins Freie. Roman}|pwv} zu seiner Geliebten
               immer geradeso wichtig wie seine Beziehung zu den verschiedentlichen Juden des Romans\pwindex{Weg ins Freie. Roman@\emph{Der Weg ins Freie. Roman}|pwv} – ich habe eben ein
               Lebensjahr des Freiherrn von
                  Wergenthin\pwindex{Weg ins Freie. Roman@\emph{Der Weg ins Freie. Roman}|pwv} geschildert, in dem er über allerlei Menschen und Probleme und
               über sich selbst ins Klare ko{\geminationm}t. Manche von diesen
               Problemen sind mir selbst allerdings erst im Laufe der Arbeit zu ihrer eigentlichen
               Bedeutung erstanden – obwohl sie ja von Anbeginn in den Geschehnissen enthalten
               waren; insbesondere das Problem der Schuld und der Verantwortung. Ganz flüchtig,
               gewissermaßen {\pb}wie ein Spaß, kam mir sogar der
               Gedanke, das Buch »Die Mörder« zu nennen oder »Die Schuldig-Unschuldigen« – (ein Spaß
               wie gesagt) – aber fiel es Ihnen nicht auch auf, wie sowohl Georg\pwindex{Weg ins Freie. Roman@\emph{Der Weg ins Freie. Roman}|pwv} als Heinrich Bermann\pwindex{Weg ins Freie. Roman@\emph{Der Weg ins Freie. Roman}|pwv} als Leo Golowski\pwindex{Weg ins Freie. Roman@\emph{Der Weg ins Freie. Roman}|pwv}{ }\introOben{}jeder\introOben{} ein Menschenleben auf dem Gewissen haben? Georg\pwindex{Weg ins Freie. Roman@\emph{Der Weg ins Freie. Roman}|pwv} metaphysisch oder in der
               Einbildung der Mörder seines ungeborenen Kindes – Heinrich\pwindex{Weg ins Freie. Roman@\emph{Der Weg ins Freie. Roman}|pwv} läßt seine Geliebte aus Eitelkeit – oder »Trägheit des Herzens\pwindex{Caspar Hauser oder Die Traegheit des Herzens@\emph{Caspar Hauser oder Die Trägheit des Herzens}|pw}« (um den Titel des neuen Wasserma{\geminationn}\pwindex{Wassermann, Jakob 10.03.1873 – 01.01.1934@\textsc{Wassermann, Jakob} (10.03.1873 – 01.01.1934), \emph{Schriftsteller/Schriftstellerin}|pw}’schen Romans zu citiren) zu {\pb}Grunde gehn –
                  Leo\pwindex{Weg ins Freie. Roman@\emph{Der Weg ins Freie. Roman}|pwv} bringt seinen Gegner im
               Duell um. (Und keinem von ihnen ist innerlich freier zu Muth, als dem, der \substVorne{}\textsuperscript{auch}\substDazwischen{}just\substHinten{} im \substVorne{}\textsuperscript{\uline{wahren}}\substDazwischen{}üblichen\substHinten{} Wortsinn getödtet hat!)\pend
           
\pstart
           – Was Sie an einer Stelle Ihres Briefes andeuten, ist mir auch in den Sinn geko{\geminationm}en: ob es nicht klüger, künstlerisch klüger gewesen
               wäre, Georg\pwindex{Weg ins Freie. Roman@\emph{Der Weg ins Freie. Roman}|pwv} zum Liebhaber
               einer Jüdin zu machen. Ich konnte nicht. Die Gestalt der Anna\pwindex{Weg ins Freie. Roman@\emph{Der Weg ins Freie. Roman}|pwv} stand von Anfang an eben so
               unwidersprechlich als katholisch da. Und es kam mir ja schliesslich nicht darauf an,
               irgendwas nachzuweisen: weder dass Christ und Jude sich nicht ver\substVorne{}\textsuperscript{\textcolor{gray}{bergen}}\substDazwischen{}tragen\substHinten{} – oder dass sie sich doch vertragen können – sondern ich wollte, ohne
               Tendenz, {\pb}Menschen und Beziehungen darstellen –
               die ich gesehn habe (ob in der Welt draußen oder in der Phantasie bliebe sich
               gleich.) \strikeout{Wie sich} Es freut mich so sehr, dass Sie
               innern Reichtum in dem Buch\pwindex{Weg ins Freie. Roman@\emph{Der Weg ins Freie. Roman}|pwv}
               finden. Dies Gefühl, ich will es gestehn, verliess mich selten während meiner Arbeit
               – und in diesem Gefühl verzieh ich mir mancherlei – vielleicht zu viel. Und \introOben{}–\introOben{} immer wieder in diesem selben Gefühl – war ich so
               niedergedrückt und hoffnungslos, dass ich sagte: Wie schön war dieser Roman, – eh ich
               ihn geschrieben habe! – Jetzt aber, da er fertig ist, schätz ich ihn höher als alles
               was ich bisher gemacht – und ich danke Ihnen herzlich für all das gute, das Sie mir
                  {\pb}darüber schreiben – und dank Ihnen noch mehr,
               dass Sie in meinen Sachen etwas verwandtes spüren. Was Ihre Freundschaft mir
               bedeutet, brauch ich Ihnen wohl nicht mehr zu sagen. Ich hoffe wir sehen uns wieder,
               und nicht in gar zu ferner Zeit. Kommen Sie denn gar nicht mehr nach Wien\oindex{Wien@\textbf{Wien}, \emph{A.ADM2}|pw}?\pend
           
\pstart
           Meine Frau\pwindex{Schnitzler, Olga 17.01.1882 – 13.01.1970@\textsc{Schnitzler, Olga} (17.01.1882 – 13.01.1970), \emph{Schauspieler/Schauspielerin, Sänger/Sängerin}|pwv} bittet mich, in
               guter Marienlyst\oindex{Marienlyst@\textbf{Marienlyst}, \emph{S.EST}|pw}er Erinnerung, Sie bestens zu
               grüßen. Wir haben keinen guten Winter hinter uns; meine Frau\pwindex{Schnitzler, Olga 17.01.1882 – 13.01.1970@\textsc{Schnitzler, Olga} (17.01.1882 – 13.01.1970), \emph{Schauspieler/Schauspielerin, Sänger/Sängerin}|pwv} hatte einen schweren Scharlach. Zwei
               Monate lang war das Kind\pwindex{Schnitzler, Heinrich 09.08.1902 – 12.07.1982@\textsc{Schnitzler, Heinrich} (09.08.1902 – 12.07.1982), \emph{Regisseur/Regisseurin, Schauspieler/Schauspielerin}|pwv}{ }\introOben{}deshalb\introOben{} außer Hause; seit dem Frühjahr sind wir viel
               herumgefahren; erst seit ein paar Tagen arbeit ich wieder was.\pend
           
\pstart
           In treuer Verehrung Ihr{\\[\baselineskip]}\spacefill\mbox{Arthur Schnitzler}\pend
           \leftskip=0em{}\selectlanguage{ngerman}\endnumbering\briefempfaengerindex{Brandes, Georg@\textsc{Brandes, Georg}!zzzSchnitzler, Arthur@\emph{von Arthur Schnitzler}!1908-07-041@{4. 7. 1908}|)be}\mylabel{L01779h}  \normalsize

\doendnotes{C}
\bigskip
\vfill

\clearpage

\footnotesize

\lohead{\textsc{register}}

% Definiere theindex-Environment komplett neu ohne reledmac
\makeatletter
\renewenvironment{theindex}{%
  \section*{\indexname}%
  \setlength{\parindent}{0pt}%
  \setlength{\parskip}{0pt plus 0.3pt}%
  \let\item\@idxitem
}{%
  \clearpage
}
\makeatother

\IfFileExists{\jobname-pw.ind}{\input{\jobname-pw.ind}}{}

\end{document}

      