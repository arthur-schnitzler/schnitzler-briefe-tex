\input{../tex-inputs/latex-pdf-vorspann}
\begin{center}
            \textcolor{red}{ENTWURF. ENTZIFFERUNG NOCH NICHT KORREKTURGELESEN}
                      \end{center}
            
               \section[Arthur Schnitzler an Hermann Bahr, 18. 11. 1911]{ Arthur Schnitzler an Hermann Bahr, 18. 11. 1911}\nopagebreak\mylabel{v}\rehead{ }\begin{ledgroupsized}[t]{13cm}\normalsize\beginnumbering\briefempfaengerindex{Bahr, Hermann@\textsc{Bahr, Hermann}!zzzSchnitzler, Arthur@\emph{von Arthur Schnitzler}!1911-11-181@{18. 11. 1911}|(be} \toendnotes[C]{\smallbreak\pagebreak[2]} \Standort{TMW, HS AM 60142 Ba.}
\physDesc{Bildpostkarte
\newline{}Handschrift: schwarze Tinte, deutsche Kurrent\newline{}Versand: 1) Stempel: »\nobreak{}\oindex{XIII., Hietzing@\textbf{XIII., Hietzing}|pwk}Wien 13 7, 18. XI. 11\nobreak{}«.  2) mit Bleistift von unbekannter Hand Postrayon »/9« zu
                                    »/7« verbessert, um eine
                                 Verwechslung mit dem namensgleichen Privatbeamten Hermann Bahr\pwindex{Bahr, Hermann 1858/1859 – 1939@\textsc{Bahr, Hermann} (1858/1859 – 1939), \emph{Privatbeamter}|pw} in der Töpfelgasse 7\oindex{Toepfelgasse@\textbf{Töpfelgasse}|pw} zu korrigieren}\buchAbdrucke{\weitereDrucke{1) \emph{18. 11. 1911, Abschrift.} In: Arthur Schnitzler: \emph{The Letters of Arthur Schnitzler to Hermann Bahr}. Edited, annotated, and with an introduction, by Donald G.
                        Daviau. Chapel Hill: \emph{The University of North Carolina Press} 1978, S. 109 (University of North Carolina studies in the Germanic languages
                        and literatures, 89).} \weitereDrucke{2) Hermann Bahr, Arthur Schnitzler: \emph{Briefwechsel, Aufzeichnungen, Dokumente (1891–1931)}. Hg. Kurt Ifkovits und Martin Anton Müller. Göttingen: \emph{Wallstein} 2018, S. 461.} }\toendnotes[C]{\smallbreak}\pstart{}{\pb}Herrn Hermann
                  Bahr\pend{}\pstart{}Wien XIII\oindex{XIII., Hietzing@\textbf{XIII., Hietzing}|pw}\pend{}\pstart{}\textsc{St. Veit\oindex{Ober Sankt Veit@\textbf{Ober Sankt Veit}|pw}}\pend{}\pstart{}\textsc{Veilissengasse\oindex{Veitlissengasse@\textbf{Veitlissengasse}|pw}}\pend{}{\bigskip}\pstart
           \noindent{}\centering{}\textcolor{gray}{\textbf{{\pb}Türkenschanz-Park\oindex{Tuerkenschanzpark@\textbf{Türkenschanzpark}|pw}}}\pend
           \pstart
           {\pb}Wien\oindex{Wien@\textbf{Wien}|pw}, 18. 11. 911.\pend
           \pstart
           herzlichen Dank, lieber Hermann, für dein und deiner verehrten Gattin\pwindex{Bahr-Mildenburg, Anna 29.11.1872 – 27.01.1947@\textsc{Bahr-Mildenburg, Anna} (29.11.1872 – 27.01.1947), \emph{Sängerin}|pwv}{ }Bayreuth\pwindex{Bahr, Hermann 19.07.1863 – 15.01.1934@\textsc{Bahr, Hermann} (19.07.1863 – 15.01.1934), \emph{Schriftsteller, Kritiker}!Bayreuth1911@\strich\emph{Bayreuth} {[}1911{]}|pw}\pwindex{Bahr-Mildenburg, Anna 29.11.1872 – 27.01.1947@\textsc{Bahr-Mildenburg, Anna} (29.11.1872 – 27.01.1947), \emph{Sängerin}!Bayreuth1911@\strich\emph{Bayreuth} {[}1911{]}|pw} Buch, das ich von einer Reiſe heimkehrend
               vorfinde u auf deſſen Lecture ich mich ſehr freue. Immer Dein \spacefill\mbox{Arthur}\pend
           \endnumbering\briefempfaengerindex{Bahr, Hermann@\textsc{Bahr, Hermann}!zzzSchnitzler, Arthur@\emph{von Arthur Schnitzler}!1911-11-181@{18. 11. 1911}|)be}\mylabel{h}\end{ledgroupsized}  \newcommand{\dateiname}{L02047}\newcommand{\titel}{Arthur Schnitzler an Hermann Bahr, 18. 11. 1911}\newcommand{\editorInnen}{ Kurt Ifkovits,  Martin Anton Müller}\input{../tex-inputs/latex-pdf-abspann}
      