%% latex-leseansicht-vorspann.tex
%% Vorspann für die Leseansicht.
%% Lädt die gemeinsame Datei latex-vorspann.tex mit nicht gesetztem Schalter.

\newif\ifkorrekturansicht
\korrekturansichtfalse

\input{../tex-inputs/latex-vorspann}


\section[Arthur Schnitzler an Hermann Bahr, 18. 11. 1911]{L02047 Arthur Schnitzler an Hermann Bahr, 18. 11. 1911}
\nopagebreak\mylabel{L02047v}
\rehead{ }\normalsize\beginnumbering\briefempfaengerindex{Bahr, Hermann@\textsc{Bahr, Hermann}!zzzSchnitzler, Arthur@\emph{von Arthur Schnitzler}!1911-11-181@{18. 11. 1911}|(be}
\toendnotes[C]{\smallbreak\pagebreak[2]}
\correspDesc{Versand  durch Arthur Schnitzler am 18. 11. 1911 in Wien
\newline{}Erhalt  durch Hermann Bahr im Zeitraum [18. 11. 1911 – 22. 11. 1911?] in Wien}\toendnotes[C]{\smallbreak}
\Standort{TMW, HS AM 60142 Ba.}
\physDesc{Bildpostkarte, 251 Zeichen
\newline{}Handschrift: schwarze Tinte, deutsche Kurrent
\newline{}Versand: 1) Stempel: »\nobreak{}\oindex{XIII., Hietzing@\textbf{XIII., Hietzing}, \emph{Verwaltungsgebiet}|pwk}Wien 13 7, 18. XI. 11\nobreak{}«.   2) mit Bleistift von unbekannter Hand Postrayon »/9«
                                 zu »/7« verbessert, um eine Verwechslung mit dem
                                 namensgleichen Privatbeamten Hermann Bahr\pwindex{Bahr, Hermann 1858/1859 – 1939 Wien@\textsc{Bahr, Hermann} (1858/1859 – 1939 Wien), \emph{Privatbeamter}|pw} in der Töpfelgasse 7\oindex{Wien@\textbf{Wien}!XIII., Hietzing@\textbf{XIII., Hietzing}!Töpfelgasse@\textbf{Töpfelgasse}, \emph{Straße}|pw} zu korrigieren}
\buchAbdrucke{\weitereDrucke{1) \emph{18. 11. 1911, Abschrift.} In: Arthur Schnitzler: \emph{The Letters of Arthur Schnitzler to Hermann Bahr}. Edited, annotated, and with an introduction, by Donald G. Daviau. Chapel Hill: \emph{The University of North Carolina Press} 1978, S. 109 (University of North Carolina studies in the Germanic languages
                        and literatures, 89).} \weitereDrucke{2) Hermann Bahr, Arthur Schnitzler: \emph{Briefwechsel, Aufzeichnungen, Dokumente (1891–1931)}. Herausgegeben von Kurt Ifkovits und Martin Anton Müller. Göttingen: \emph{Wallstein} 2018, S. 461.} }\toendnotes[C]{\smallbreak}\pstart{}{\pb}Herrn Hermann
                  Bahr\pend{}\pstart{}Wien XIII\oindex{XIII., Hietzing@\textbf{XIII., Hietzing}, \emph{Verwaltungsgebiet}|pw}\pend{}\pstart{}\textsc{St. Veit\oindex{Wien@\textbf{Wien}!XIII., Hietzing@\textbf{XIII., Hietzing}!Ober Sankt Veit@\textbf{Ober Sankt Veit}, \emph{Ehemaliger Ort}|pw}}\pend{}\pstart{}\textsc{Veilissengasse\oindex{Wien@\textbf{Wien}!XIII., Hietzing@\textbf{XIII., Hietzing}!Veitlissengasse@\textbf{Veitlissengasse}, \emph{Straße}|pw}}\pend{}{\bigskip}
\pstart
           \noindent{}\centering{}{\pb}\textcolor{gray}{\textbf{Türkenschanz-Park\oindex{Wien@\textbf{Wien}!XVIII., Währing@\textbf{XVIII., Währing}!Türkenschanzpark@\textbf{Türkenschanzpark}, \emph{Park}|pw}}}\pend
           \vspace{1em}
\pstart
           {\pb}Wien\oindex{Wien@\textbf{Wien}, \emph{Verwaltungsgebiet}|pw}, 18. 11. 911.\pend
           \vspace{0.5em}
\pstart
           herzlichen Dank, lieber Hermann, für dein und deiner verehrten Gattin\pwindex{Bahr-Mildenburg, Anna 29.\,11.\,1872 Wien – 27.\,1.\,1947 ebd.@\textsc{Bahr-Mildenburg, Anna} (29.\,11.\,1872 Wien – 27.\,1.\,1947 ebd.), \emph{Sängerin}|pwv}{ }Bayreuth\pwindex{Bahr, Hermann 19.\,7.\,1863 Linz – 15.\,1.\,1934 München@\textsc{Bahr, Hermann} (19.\,7.\,1863 Linz – 15.\,1.\,1934 München), \emph{Schriftsteller, Kritiker}!Bayreuth@\strich\emph{Bayreuth}|pw}\pwindex{Bahr-Mildenburg, Anna 29.\,11.\,1872 Wien – 27.\,1.\,1947 ebd.@\textsc{Bahr-Mildenburg, Anna} (29.\,11.\,1872 Wien – 27.\,1.\,1947 ebd.), \emph{Sängerin}!Bayreuth@\strich\emph{Bayreuth}|pw} Buch, das ich von einer Reiſe
               heimkehrend vorfinde u auf deſſen Lecture ich mich{ }ſehr freue. Immer Dein
                  \spacefill\mbox{Arthur}\pend
           \selectlanguage{ngerman}\endnumbering\briefempfaengerindex{Bahr, Hermann@\textsc{Bahr, Hermann}!zzzSchnitzler, Arthur@\emph{von Arthur Schnitzler}!1911-11-181@{18. 11. 1911}|)be}\mylabel{L02047h}  \newcommand{\dateiname}{L02047}\newcommand{\titel}{Arthur Schnitzler an Hermann Bahr, 18. 11. 1911}\newcommand{\editorInnen}{Herausgegeben von Martin Anton Müller}%% latex-leseansicht-abspann.tex
%% Abspann für die Leseansicht.
%% Der Schalter \ifkorrekturansicht ist bereits durch den Vorspann gesetzt.

%% latex-abspann.tex
%% Gemeinsamer Abspann für Korrekturansicht und Leseansicht.
%% Setzt den Schalter \ifkorrekturansicht voraus (gesetzt in den
%% einbindenden Dateien latex-korrekturansicht-abspann.tex bzw.
%% latex-leseansicht-abspann.tex).
%% ---------------------------------------------------------------

\normalsize

% Das esempio-Environment wird nur in der Leseansicht benötigt
\ifkorrekturansicht\else
\newenvironment{esempio}[3]%
{
    \vspace{1.5ex}
    \rlap{\underline{#1}}
    \par
    \setlength{\parindent}{0cm}
    \nopagebreak
    \leftskip=#2cm
    \rightskip=#3cm
}
{
    \par
}
\fi

\doendnotes{C}
\bigskip
\vfill

\clearpage

\footnotesize

\ifkorrekturansicht
  \lohead{\textsc{register}}
\fi

% theindex-Environment neu definieren ohne reledmac
\makeatletter
\renewenvironment{theindex}{%
  \ifkorrekturansicht
    \section*{\indexname}%
  \else
    \subsubsection*{Index der erwähnten Entitäten}%
  \fi
  \setlength{\parindent}{0pt}%
  \setlength{\parskip}{0pt plus 0.3pt}%
  \let\item\@idxitem
}{%
  \ifkorrekturansicht\clearpage\fi
}
\makeatother

\IfFileExists{\jobname-pw.ind}{\input{\jobname-pw.ind}}{}

% Quellenangabe nur in der Leseansicht
\ifkorrekturansicht\else
% Fallback-Definitionen, falls die .tex-Datei \titel etc. nicht gesetzt hat
\providecommand{\titel}{}
\providecommand{\editorInnen}{}
\providecommand{\dateiname}{\jobname}

\vspace{3cm}

\vfill

\footnotesize
\textsc{Quelle}: \titel. Herausgegeben von {\editorInnen}. In: \emph{Arthur Schnitzler: Briefwechsel mit Autorinnen und Autoren}.
 Digitale Edition, https://schnitzler-briefe.acdh.oeaw.ac.at/{\dateiname}.html (Stand \today)
\fi

\end{document}


