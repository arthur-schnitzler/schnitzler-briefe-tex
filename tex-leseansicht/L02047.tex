%% latex-korrekturansicht-vorspann.tex
%% Vorspann für die Korrekturansicht.
%% Lädt die gemeinsame Datei latex-vorspann.tex mit gesetztem Schalter.

\newif\ifkorrekturansicht
\korrekturansichttrue

\input{../tex-inputs/latex-vorspann}


\section[Arthur Schnitzler an Hermann Bahr, 18. 11. 1911]{L02047 Arthur Schnitzler an Hermann Bahr, 18. 11. 1911}
\nopagebreak\mylabel{L02047v}
\rehead{ }\normalsize\beginnumbering\briefempfaengerindex{Bahr, Hermann@\textsc{Bahr, Hermann}!zzzSchnitzler, Arthur@\emph{von Arthur Schnitzler}!1911-11-181@{18. 11. 1911}|(be}
\toendnotes[C]{\smallbreak\pagebreak[2]}\Standort{TMW, HS AM 60142 Ba.}
\physDesc{Bildpostkarte, 251 Zeichen
\newline{}Handschrift: schwarze Tinte, deutsche Kurrent
\newline{}Versand: 1) Stempel: »\nobreak{}\oindex{XIII., Hietzing@\textbf{XIII., Hietzing}, \emph{A.ADM3}|pwk}Wien 13 7, 18. XI. 11\nobreak{}«.   2) mit Bleistift von unbekannter Hand Postrayon »/9«
                                 zu »/7« verbessert, um eine Verwechslung mit dem
                                 namensgleichen Privatbeamten Hermann Bahr\pwindex{Bahr, Hermann 1858/1859 – 1939@\textsc{Bahr, Hermann} (1858/1859 – 1939), \emph{Privatbeamter/Privatbeamtin}|pw} in der Töpfelgasse 7\oindex{Toepfelgasse@\textbf{Töpfelgasse}, \emph{Straße (K.STR)}|pw} zu korrigieren}
\buchAbdrucke{\weitereDrucke{1) Arthur Schnitzler: \emph{The Letters of Arthur Schnitzler to Hermann Bahr}. Chapel Hill: \emph{The University of North Carolina Press} 1978, S. 109.} \weitereDrucke{2) Hermann Bahr, Arthur Schnitzler: \emph{Briefwechsel, Aufzeichnungen, Dokumente (1891–1931)}. Göttingen: \emph{Wallstein} 2018, S. 461.} }\toendnotes[C]{\smallbreak}\pstart{}{\pb}Herrn Hermann
                  Bahr\pend{}\pstart{}Wien XIII\oindex{XIII., Hietzing@\textbf{XIII., Hietzing}, \emph{A.ADM3}|pw}\pend{}\pstart{}\textsc{St. Veit\oindex{Ober Sankt Veit@\textbf{Ober Sankt Veit}, \emph{P.PPLX}|pw}}\pend{}\pstart{}\textsc{Veilissengasse\oindex{Veitlissengasse@\textbf{Veitlissengasse}, \emph{Straße (K.STR)}|pw}}\pend{}{\bigskip}
\pstart
           \noindent{}\centering{}{\pb}\textcolor{gray}{\textbf{Türkenschanz-Park\oindex{Tuerkenschanzpark@\textbf{Türkenschanzpark}, \emph{Park (K.PRK)}|pw}}}\pend
           \vspace{1em}
\pstart
           {\pb}Wien\oindex{Wien@\textbf{Wien}, \emph{A.ADM2}|pw}, 18. 11. 911.\pend
           \vspace{0.5em}
\pstart
           herzlichen Dank, lieber Hermann, für dein und deiner verehrten Gattin\pwindex{Bahr-Mildenburg, Anna 29.11.1872 – 27.01.1947@\textsc{Bahr-Mildenburg, Anna} (29.11.1872 – 27.01.1947), \emph{Sänger/Sängerin}|pwv}{ }Bayreuth\pwindex{Bayreuth@\emph{Bayreuth}|pw} Buch, das ich von einer Reiſe
               heimkehrend vorfinde u auf deſſen Lecture ich mich ſehr freue. Immer Dein
                  \spacefill\mbox{Arthur}\pend
           \selectlanguage{ngerman}\endnumbering\briefempfaengerindex{Bahr, Hermann@\textsc{Bahr, Hermann}!zzzSchnitzler, Arthur@\emph{von Arthur Schnitzler}!1911-11-181@{18. 11. 1911}|)be}\mylabel{L02047h}  \normalsize

\doendnotes{C}
\bigskip
\vfill

\clearpage

\footnotesize

\lohead{\textsc{register}}

% Definiere theindex-Environment komplett neu ohne reledmac
\makeatletter
\renewenvironment{theindex}{%
  \section*{\indexname}%
  \setlength{\parindent}{0pt}%
  \setlength{\parskip}{0pt plus 0.3pt}%
  \let\item\@idxitem
}{%
  \clearpage
}
\makeatother

\IfFileExists{\jobname-pw.ind}{\input{\jobname-pw.ind}}{}

\end{document}

      