%% latex-leseansicht-vorspann.tex
%% Vorspann für die Leseansicht.
%% Lädt die gemeinsame Datei latex-vorspann.tex mit nicht gesetztem Schalter.

\newif\ifkorrekturansicht
\korrekturansichtfalse

\input{../tex-inputs/latex-vorspann}


\section[Hugo von Hofmannsthal an Arthur Schnitzler, {[}29. 10. 1910{]}]{L01973 Hugo von Hofmannsthal an Arthur Schnitzler, {[}29. 10. 1910{]}}
\nopagebreak\mylabel{L01973v}
\rehead{ }\normalsize\beginnumbering\briefempfaengerindex{Schnitzler, Arthur@\textsc{Schnitzler, Arthur}!zzzHofmannsthal, Hugo von@\emph{von Hugo von Hofmannsthal}!1910-10-292@{{[}29. 10. 1910{]}}|(be}
\toendnotes[C]{\smallbreak\pagebreak[2]}
\correspDesc{Versand  durch Hugo von Hofmannsthal am [29. 10. 1910] in Hradec nad Moravicí
\newline{}Erhalt  durch Arthur Schnitzler im Zeitraum [30. 10. 1910 – 3. 11. 1910?] in Wien}\toendnotes[C]{\smallbreak}
\Standort{CUL, Schnitzler, B 43.}
\physDesc{Brief, 1 Blatt, 4 Seiten, 1064 Zeichen
\newline{}Handschrift: schwarze Tinte, deutsche Kurrent
\newline{}Schnitzler: mit Bleistift datiert: »29/X 910« und beschriftet: »\textsc{Hugo}« 
\newline{}Ordnung: 1) mit Bleistift von unbekannter Hand nummeriert: »\strikeout{307}«  2) mit Bleistift von unbekannter Hand nummeriert:
                                    »324«}
\buchAbdrucke{\weitereDrucke{Hugo von Hofmannsthal, Arthur Schnitzler: \emph{Briefwechsel}. Herausgegeben von Therese Nickl und Heinrich Schnitzler. Frankfurt am Main: \emph{S. Fischer} 1964, S. 255.} }\toendnotes[C]{\smallbreak}
\pstart
           {\pb}\textcolor{gray}{\textbf{SCHLOSS GRÄTZ\oindex{Schloss Grätz@\textbf{Schloss Grätz}, \emph{Schloss}|pw}}}\hfill \textcolor{gray}{\textbf{TELEGRAMME:}}\pend
           
\pstart
           \textcolor{gray}{\textbf{BEI{ }TROPPAU\oindex{Opava@\textbf{Opava}|pw}}}\hfill \textcolor{gray}{\textbf{GRÄTZ – SCHLESIEN\oindex{Hradec nad Moravicí@\textbf{Hradec nad Moravicí}, \emph{Verwaltungsgebiet}|pw}}}\pend
           
\pstart
           \raggedleft{}Samstag\pend
           
\pstart{}mein lieber Arthur\pend\vspace{0.5em}
\pstart
           Montag begebe ich mich von hier fort, nicht zu Fuß, bei Nacht und zornig
               wie Beethoven\pwindex{Beethoven, Ludwig van 17.\,12.\,1770 Bonn – 26.\,3.\,1827 Wien@\textsc{Beethoven, Ludwig van} (17.\,12.\,1770 Bonn – 26.\,3.\,1827 Wien), \emph{Komponist}|pw},{ }ſondern bei Tag, freundlich und
               in einem \textsc{Automobil}, auch wird mir auf dem Weg zwiſchen hier
               und Troppau\oindex{Opava@\textbf{Opava}|pw} nicht das Manuſcript der \textsc{Eroica}\pwindex{Beethoven, Ludwig van 17.\,12.\,1770 Bonn – 26.\,3.\,1827 Wien@\textsc{Beethoven, Ludwig van} (17.\,12.\,1770 Bonn – 26.\,3.\,1827 Wien), \emph{Komponist}!Symphonie Nr. 3 es-Dur op. 55 »Eroica«@\strich\emph{Symphonie Nr. 3 es-Dur op. 55 »Eroica«}|pw} aus {\pb}dem Mantel fallen und
               in einen kothigen Straßengraben rollen, weil ich es – leider! – nicht bei mir
               habe.\pend
           
\pstart
           Von Dienstag an bin ich dann in Rodaun\oindex{Wien@\textbf{Wien}!XXIII., Liesing@\textbf{XXIII., Liesing}!Rodaun@\textbf{Rodaun}, \emph{Region}|pw} und warte auf den Ruf, Euer Haus zum erſten Mal zu betreten und
               dieſer Stunde durch \label{K_L01973-1v}\edtext{Vorleſung}{\lemma{\textnormal{\emph{Vorlesung}}}\Cendnote{\textnormal{Siehe A. S.: \emph{Tagebuch}, 29. 11. 1910.
               }}}\label{K_L01973-1} des tiefſinnigen »Roſencavaliers\pwindex{Hofmannsthal, Hugo von 1.\,2.\,1874 Wien – 15.\,7.\,1929 Rodaun@\textsc{Hofmannsthal, Hugo von} (1.\,2.\,1874 Wien – 15.\,7.\,1929 Rodaun), \emph{Schriftsteller}!Rosenkavalier. Komödie für Musik@\strich\emph{Der Rosenkavalier. Komödie für Musik}|pw}« {\pb}eine höhere Weihe zu geben.\pend
           
\pstart
           Ich kann mir aber{ }ſehr wohl denken, daſs die Proben zum Medardus\pwindex{Schnitzler, Arthur 15.\,5.\,1862 Wien – 21.\,10.\,1931 ebd.@\textsc{Schnitzler, Arthur} (15.\,5.\,1862 Wien – 21.\,10.\,1931 ebd.), \emph{Schriftsteller, Mediziner}!junge Medardus. Dramatische Historie in einem Vorspiel und fünf Aufzügen@\strich\emph{Der junge Medardus. Dramatische Historie in einem Vorspiel und fünf Aufzügen}|pw}{ }ſehr hernehmend{ }ſind und Sie ein dringendes
               Bedürfnis haben, des Abends Ruhe zu finden, dann laſſen wir es halt bis
                  nachher.\hspace*{1.5em}Von Herzen Ihr\pend
           \pstart \spacefill\mbox{Hugo.}\pend{}
\pstart
           \noindent{}{\pb}\textsc{PS}. Ich möchte nicht gern mit einem Ihrer Kinder in
                  dauerndem Unfrieden leben, und da ich den Roman\pwindex{Schnitzler, Arthur 15.\,5.\,1862 Wien – 21.\,10.\,1931 ebd.@\textsc{Schnitzler, Arthur} (15.\,5.\,1862 Wien – 21.\,10.\,1931 ebd.), \emph{Schriftsteller, Mediziner}!Weg ins Freie. Roman@\strich\emph{Der Weg ins Freie. Roman}|pwv} damals halb zufällig halb abſichtlich in der
                  Eiſenbahn liegen laſſen habe,{ }ſo bitte ich Sie jetzt, wo zwei Jahre darüber
                  hingegangen{ }ſind, mir das Buch wieder einmal zu{ }ſchenken, wenn Sie ein
                  überflüſſiges Exemplar haben.\pend
           \selectlanguage{ngerman}\endnumbering\briefempfaengerindex{Schnitzler, Arthur@\textsc{Schnitzler, Arthur}!zzzHofmannsthal, Hugo von@\emph{von Hugo von Hofmannsthal}!1910-10-292@{{[}29. 10. 1910{]}}|)be}\mylabel{L01973h}  \newcommand{\dateiname}{L01973}\newcommand{\titel}{Hugo von Hofmannsthal an Arthur Schnitzler, [29. 10. 1910]}\newcommand{\editorInnen}{Martin Anton Müller und Gerd-Hermann Susen}%% latex-leseansicht-abspann.tex
%% Abspann für die Leseansicht.
%% Der Schalter \ifkorrekturansicht ist bereits durch den Vorspann gesetzt.

%% latex-abspann.tex
%% Gemeinsamer Abspann für Korrekturansicht und Leseansicht.
%% Setzt den Schalter \ifkorrekturansicht voraus (gesetzt in den
%% einbindenden Dateien latex-korrekturansicht-abspann.tex bzw.
%% latex-leseansicht-abspann.tex).
%% ---------------------------------------------------------------

\normalsize

% Das esempio-Environment wird nur in der Leseansicht benötigt
\ifkorrekturansicht\else
\newenvironment{esempio}[3]%
{
    \vspace{1.5ex}
    \rlap{\underline{#1}}
    \par
    \setlength{\parindent}{0cm}
    \nopagebreak
    \leftskip=#2cm
    \rightskip=#3cm
}
{
    \par
}
\fi

\doendnotes{C}
\bigskip
\vfill

\clearpage

\footnotesize

\ifkorrekturansicht
  \lohead{\textsc{register}}
\fi

% theindex-Environment neu definieren ohne reledmac
\makeatletter
\renewenvironment{theindex}{%
  \ifkorrekturansicht
    \section*{\indexname}%
  \else
    \subsubsection*{Index der erwähnten Entitäten}%
  \fi
  \setlength{\parindent}{0pt}%
  \setlength{\parskip}{0pt plus 0.3pt}%
  \let\item\@idxitem
}{%
  \ifkorrekturansicht\clearpage\fi
}
\makeatother

\IfFileExists{\jobname-pw.ind}{\input{\jobname-pw.ind}}{}

% Quellenangabe nur in der Leseansicht
\ifkorrekturansicht\else
% Fallback-Definitionen, falls die .tex-Datei \titel etc. nicht gesetzt hat
\providecommand{\titel}{}
\providecommand{\editorInnen}{}
\providecommand{\dateiname}{\jobname}

\vspace{3cm}

\vfill

\footnotesize
\textsc{Quelle}: \titel. Herausgegeben von {\editorInnen}. In: \emph{Arthur Schnitzler: Briefwechsel mit Autorinnen und Autoren}.
 Digitale Edition, https://schnitzler-briefe.acdh.oeaw.ac.at/{\dateiname}.html (Stand \today)
\fi

\end{document}


