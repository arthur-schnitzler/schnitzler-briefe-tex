%% latex-korrekturansicht-vorspann.tex
%% Vorspann für die Korrekturansicht.
%% Lädt die gemeinsame Datei latex-vorspann.tex mit gesetztem Schalter.

\newif\ifkorrekturansicht
\korrekturansichttrue

\input{../tex-inputs/latex-vorspann}


\section[Hugo von Hofmannsthal an Arthur Schnitzler, {[}29. 10. 1910{]}]{L01973 Hugo von Hofmannsthal an Arthur Schnitzler, {[}29. 10. 1910{]}}
\nopagebreak\mylabel{L01973v}
\rehead{ }\normalsize\beginnumbering\briefempfaengerindex{Schnitzler, Arthur@\textsc{Schnitzler, Arthur}!zzzHofmannsthal, Hugo von@\emph{von Hugo von Hofmannsthal}!1910-10-292@{{[}29. 10. 1910{]}}|(be}
\toendnotes[C]{\smallbreak\pagebreak[2]}\Standort{CUL, Schnitzler, B 43.}
\physDesc{Brief, 1 Blatt, 4 Seiten, 1064 Zeichen
\newline{}Handschrift: schwarze Tinte, deutsche Kurrent
\newline{}Schnitzler: mit Bleistift datiert: »29/X 910« und beschriftet: »\textsc{Hugo}« 
\newline{}Ordnung: 1) mit Bleistift von unbekannter Hand nummeriert: »\strikeout{307}«  2) mit Bleistift von unbekannter Hand nummeriert:
                                    »324«}
\buchAbdrucke{\weitereDrucke{Hugo von Hofmannsthal, Arthur Schnitzler: \emph{Briefwechsel}. Frankfurt am Main: \emph{S. Fischer} 1964, S. 255.} }\toendnotes[C]{\smallbreak}
\pstart
           {\pb}\textcolor{gray}{\textbf{SCHLOSS GRÄTZ\oindex{Schloss Graetz@\textbf{Schloss Grätz}, \emph{Schloss (K.SLS)}|pw}}}\hfill \textcolor{gray}{\textbf{TELEGRAMME:}}\pend
           
\pstart
           \textcolor{gray}{\textbf{BEI{ }TROPPAU\oindex{Opava@\textbf{Opava}, \emph{P.PPL}|pw}}}\hfill \textcolor{gray}{\textbf{GRÄTZ – SCHLESIEN\oindex{Hradec nad Moravicí@\textbf{Hradec nad Moravicí}, \emph{A.ADM3}|pw}}}\pend
           
\pstart
           \raggedleft{}Samstag\pend
           
\pstart{}mein lieber Arthur\pend\vspace{0.5em}
\pstart
           Montag begebe ich mich von hier fort, nicht zu Fuß, bei Nacht und zornig
               wie Beethoven\pwindex{Beethoven, Ludwig van 17.12.1770 – 26.03.1827@\textsc{Beethoven, Ludwig van} (17.12.1770 – 26.03.1827), \emph{Komponist/Komponistin}|pw}, ſondern bei Tag, freundlich und
               in einem \textsc{Automobil}, auch wird mir auf dem Weg zwiſchen hier
               und Troppau\oindex{Opava@\textbf{Opava}, \emph{P.PPL}|pw} nicht das Manuſcript der \textsc{Eroica}\pwindex{Symphonie Nr. 3 es-Dur op. 55 »Eroica«@\emph{Symphonie Nr. 3 es-Dur op. 55 »Eroica«}|pw} aus {\pb}dem Mantel fallen und
               in einen kothigen Straßengraben rollen, weil ich es – leider! – nicht bei mir
               habe.\pend
           
\pstart
           Von Dienstag an bin ich dann in Rodaun\oindex{Rodaun@\textbf{Rodaun}, \emph{A.ADM4}|pw} und warte auf den Ruf, Euer Haus zum erſten Mal zu betreten und
               dieſer Stunde durch \label{K_L01973-1v}\edtext{Vorleſung}{\lemma{\textnormal{\emph{Vorleſung}}}\Cendnote{\textnormal{Siehe A. S.: \emph{Tagebuch}, 29. 11. 1910.
               }}}\label{K_L01973-1} des tiefſinnigen »Roſencavaliers\pwindex{Rosenkavalier@\emph{Der Rosenkavalier}|pw}« {\pb}eine höhere Weihe zu geben.\pend
           
\pstart
           Ich kann mir aber ſehr wohl denken, daſs die Proben zum Medardus\pwindex{junge Medardus. Dramatische Historie in einem Vorspiel und fuenf Aufzuegen@\emph{Der junge Medardus. Dramatische Historie in einem Vorspiel und fünf Aufzügen}|pw}{ }ſehr hernehmend ſind und Sie ein dringendes
               Bedürfnis haben, des Abends Ruhe zu finden, dann laſſen wir es halt bis
                  nachher.\hspace*{1.5em}Von Herzen Ihr\pend
           \pstart \spacefill\mbox{Hugo.}\pend{}
\pstart
           \noindent{}{\pb}\textsc{PS}. Ich möchte nicht gern mit einem Ihrer Kinder in
                  dauerndem Unfrieden leben, und da ich den Roman\pwindex{Weg ins Freie. Roman@\emph{Der Weg ins Freie. Roman}|pwv} damals halb zufällig halb abſichtlich in der
                  Eiſenbahn liegen laſſen habe, ſo bitte ich Sie jetzt, wo zwei Jahre darüber
                  hingegangen ſind, mir das Buch wieder einmal zu ſchenken, wenn Sie ein
                  überflüſſiges Exemplar haben.\pend
           \selectlanguage{ngerman}\endnumbering\briefempfaengerindex{Schnitzler, Arthur@\textsc{Schnitzler, Arthur}!zzzHofmannsthal, Hugo von@\emph{von Hugo von Hofmannsthal}!1910-10-292@{{[}29. 10. 1910{]}}|)be}\mylabel{L01973h}  \normalsize

\doendnotes{C}
\bigskip
\vfill

\clearpage

\footnotesize

\lohead{\textsc{register}}

% Definiere theindex-Environment komplett neu ohne reledmac
\makeatletter
\renewenvironment{theindex}{%
  \section*{\indexname}%
  \setlength{\parindent}{0pt}%
  \setlength{\parskip}{0pt plus 0.3pt}%
  \let\item\@idxitem
}{%
  \clearpage
}
\makeatother

\IfFileExists{\jobname-pw.ind}{\input{\jobname-pw.ind}}{}

\end{document}

      