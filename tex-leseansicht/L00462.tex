%% latex-leseansicht-vorspann.tex
%% Vorspann für die Leseansicht.
%% Lädt die gemeinsame Datei latex-vorspann.tex mit nicht gesetztem Schalter.

\newif\ifkorrekturansicht
\korrekturansichtfalse

\input{../tex-inputs/latex-vorspann}

\begin{center}
            \textcolor{red}{ENTWURF. ENTZIFFERUNG NOCH NICHT KORREKTURGELESEN}
                      \end{center}
            
               \section[Arthur Schnitzler an Hugo von Hofmannsthal, 10. 7. 1895]{ Arthur Schnitzler an Hugo von Hofmannsthal, 10. 7. 1895}\nopagebreak\mylabel{v}\rehead{ }\begin{ledgroupsized}[t]{13cm}\normalsize\beginnumbering\briefempfaengerindex{Hofmannsthal, Hugo von@\textsc{Hofmannsthal, Hugo von}!zzzSchnitzler, Arthur@\emph{von Arthur Schnitzler}!1895-07-102@{10. 7. 1895}|(be} \toendnotes[C]{\smallbreak\pagebreak[2]} \Standort{FDH, Hs-30885,58.}
\physDesc{Brief, 2 Blätter, 7 Seiten
\newline{}Handschrift: schwarze Tinte, deutsche Kurrent}\buchAbdrucke{\weitereDrucke{1) Hugo von Hofmannsthal, Arthur Schnitzler: \emph{Briefwechsel}. Hg. Therese Nickl und Heinrich Schnitzler. Frankfurt am Main: \emph{S. Fischer} 1964, S. 54–56.} \weitereDrucke{2) Arthur Schnitzler: \emph{Briefe 1875–1912}. Hg. Therese Nickl und Heinrich Schnitzler. Frankfurt am Main: \emph{S. Fischer} 1981, S. 264–265.} }\toendnotes[C]{\smallbreak}\pstart
           \raggedleft{}{\pb}\textsc{Marienbad}\oindex{Marienbad@\textbf{Marienbad}|pw}{ }10/7 95.\pend
           \pstart{}Mein lieber Hugo,\pend\pstart
           ich bin in Prag\oindex{Prag@\textbf{Prag}|pw} geweſen, in \textsc{Karlsbad}\oindex{Karlsbad@\textbf{Karlsbad}|pw} und nun bin ich hier, wo ich wohl bis Ende der Woche oder Anfang
                    der nächſten bleiben werde. Dann erſcheine ich in Iſchl,
                            \textsc{Pension Petter}\oindex{Hotel und Pension Rudolfshoehe (Leopold Petter)@\textbf{Hotel und Pension Rudolfshöhe (Leopold Petter)}|pw}, wo ich hoffentlich eine Nachricht von Ihnen finden werde. Dieſe
                    Zeilen werden in einer Dachka{\geminationm}er, nein, eigentlich
                    in einem Dachſalon geſchrieben – zwei Fenſter mit eben ſovielen Ausſichten;
                    beide ſtehen offen und alles papierne {\pb}auf dem
                    Tiſch flattert und knittert. – Ich hab mich ſchon an manchem ſchönen freuen
                    können und fühle mich im ganzen wohl, ohne in irgend einem Augenblick zu einem
                    Hochgefühl geko{\geminationm}en zu ſein. In Prag\oindex{Prag@\textbf{Prag}|pw} das merkwürdigſte ein alter jüdiſcher Friedhof\oindex{Alter Juedischer Friedhof@\textbf{Alter Jüdischer Friedhof}|pwv}, der langſam verſinkt. Seit mehr als
                    hundert Jahren begräbt man dort nicht mehr, und die Grabſteine u. Sarkophage
                    werden langſam von der Erde eingeſchlürft. Einige ſind noch zur Hälfte über dem
                    Boden, von andern ſieht man gerade noch die oberſten Ränder. Alle dicht
                    aneinander, viele ſchief, manche gegen einander geneigt, ſich gegenſeitig {\pb}ſtützend. Darüber ſtille nicht ſehr hohe
                    tiefgrüne Bäume, mit ſo dichtem Laub, als wenn ſie alle zuſa{\geminationm}en ein Dach ſein wollten für dieſen Friedhof\oindex{Alter Juedischer Friedhof@\textbf{Alter Jüdischer Friedhof}|pwv}, der ſtirbt. – Die ethnographiſche Ausſtellung\orgindex{Cecho-slavische ethnographische Ausstellung@Čecho-slavische ethnographische Ausstellung|pw}: viel intereſſante
                    Stuben und Coſtüme. – Der Hradſchin\oindex{Hradcany@\textbf{Hradčany}|pw}, da
                    hat mir ein Führer erzählt, daſs man im Volk in Prag\oindex{Prag@\textbf{Prag}|pw} den Kronprinzen Rudolf\pwindex{Rudolf von Oesterreich-Ungarn 21.08.1858 – 30.01.1889@\textsc{Rudolf von Österreich-Ungarn} (21.08.1858 – 30.01.1889), \emph{Erzherzog, Kronprinz}|pw}
                    nicht für todt hält: ein Kutſcher hat ihn im Jahr 91{ }ſogar in die
                    Ausſtellung geführt, ganz beſti{\geminationm}t, er hat ihn
                    erkannt. – Ein Hofbedienſteter, der ſehr gemeſſen und höflich erläutert, und der
                    ſich, we{\geminationn} ihm was unhöfiſches paſſirt, ſchnell
                    wieder derfangt. Z. B. {\pb}wie er den Fenſterſturz
                    berichtet: »Hier hat man die drei in den Graben hinuntergeſchmiſſen, \textsc{reſpective} hinuntergeworfen«.\pend
           \pstart
           – In \textsc{Karlsbad}\oindex{Karlsbad@\textbf{Karlsbad}|pw} Wirkung der Curgäſte als Maſſe, wie jeder das ſeine beiträgt zum
                    Eindruck: Weltcurort; – aber man darf ſie nicht einzeln anſehn, we{\geminationn} man das große ſpüren will – denn dann ſind’s
                    Hochſtapler, Zuckerkranke, polniſche\oindex{Polen@\textbf{Polen}|pw} Juden, \label{K_L00462_1v}\edtext{Gigerln}{\lemma{\textnormal{\emph{Gigerln}}}\Cendnote{\textnormal{österreichisch Gigerl: Geck}}}\label{K_L00462_1h}, \textsc{Besesny}\pwindex{Bezecný, Josef von 05.02.1829 – 17.06.1904@\textsc{Bezecný, Josef von} (05.02.1829 – 17.06.1904), \emph{Theaterintendant}|pw}, \textsc{Broda}\pwindex{Broda, Moritz 23.03.1842 – 16.09.1910@\textsc{Broda, Moritz} (23.03.1842 – 16.09.1910), \emph{Schauspieler}|pw}, \textsc{Wilhelmine Sandrock}\pwindex{Sandrock, Wilhelmine 05.02.1861 – 29.11.1948@\textsc{Sandrock, Wilhelmine} (05.02.1861 – 29.11.1948), \emph{Schauspielerin}|pw} – allerdings auch Sonnenthal\pwindex{Sonnenthal, Adolf von 21.12.1834 – 04.04.1909@\textsc{Sonnenthal, Adolf von} (21.12.1834 – 04.04.1909), \emph{Schauspieler}|pw}
                    (Uebergang,), einige wirklich elegante Menſchen und ein paar entzückend ſchöne
                    Amerikanerinnen. – Ich bin aus \textsc{K}\oindex{Karlsbad@\textbf{Karlsbad}|pw}. {\pb}bald fort – man ka{\geminationn} dort nur 2 Tage oder 4 Wochen bleiben. – Hier,
                    in Marienbad\oindex{Marienbad@\textbf{Marienbad}|pw}, iſt es behaglicher, und die
                    Leute, die hier ſind, ſind nicht ſo ſtolz darauf, daſs ſie da ſind, wie in
                        \textsc{Karlsbad}\oindex{Karlsbad@\textbf{Karlsbad}|pw}. – Ein großer freundlicher Park, in dem hohe ſchöne Häuſer ſtehn,
                    die lauter Hotels ſind, und ringsherum beſcheidene Hügel, die ſich freuen, weil
                    man breite Wege zu ihnen hingeführt hat, und Wälder, die ſich freuen, weil ſo
                    brave dicke Menſchen in ihnen ſpazieren gehn; auch die Wirthe und Kellner {\pb}und Dienſtmänner lächeln hier; während ſie in
                        \textsc{K.}\oindex{Karlsbad@\textbf{Karlsbad}|pw} alle ſehr ernſt ſind und ihrer Würde nie vergeſſen können. – Hier
                    hab ich \textsc{Hänsel u
                        Grethel}\pwindex{\textcolor{red}{\textsuperscript{XXXX1 indx}}!Haensel und Grethel1893@\strich\emph{Hänsel und Grethel} {[}1893{]}|pw} im Theater geſehn, in \textsc{K.}\oindex{Karlsbad@\textbf{Karlsbad}|pw} den armen Jonathan\pwindex{\textcolor{red}{\textsuperscript{XXXX1 indx}}!arme Jonathan4.1.1890 – 4.1.1890@\strich\emph{Der arme Jonathan} {[}4.1.1890 – 4.1.1890{]}|pw}\pwindex{\textcolor{red}{\textsuperscript{XXXX1 indx}}!arme Jonathan4.1.1890 – 4.1.1890@\strich\emph{Der arme Jonathan} {[}4.1.1890 – 4.1.1890{]}|pw}\pwindex{\textcolor{red}{\textsuperscript{XXXX1 indx}}!arme Jonathan4.1.1890 – 4.1.1890@\strich\emph{Der arme Jonathan} {[}4.1.1890 – 4.1.1890{]}|pw}, in
                        Prag\oindex{Prag@\textbf{Prag}|pw} (böhmiſch\oindex{Boehmen@\textbf{Böhmen}|pw}) Dimitrij\pwindex{Dvořák, Antonín 1841-09-08 – 1904-05-01@\textsc{Dvořák, Antonín} (1841-09-08 – 1904-05-01), \emph{Komponist}!Dimitrij24. 3. 1894@\strich\emph{Dimitrij} {[}24. 3. 1894{]}|pw}, Oper v.
                        Dvorak\pwindex{Dvořák, Antonín 1841-09-08 – 1904-05-01@\textsc{Dvořák, Antonín} (1841-09-08 – 1904-05-01), \emph{Komponist}|pw} u. (deutſch) – \textsc{Attaché}\pwindex{\textcolor{red}{\textsuperscript{XXXX1 indx}}!Attache1874@\strich\emph{Ein Attaché} {[}1874{]}|pw}\pwindex{\textcolor{red}{\textsuperscript{XXXX1 indx}}!Attache1874@\strich\emph{Ein Attaché} {[}1874{]}|pw} mit \textsc{Hartmann}\pwindex{Hartmann, Ernst 08.01.1844 – 10.10.1911@\textsc{Hartmann, Ernst} (08.01.1844 – 10.10.1911), \emph{Schauspieler}|pw} u \textsc{Kallina}\pwindex{Kallina, Anna 31.03.1874 – 04.01.1948@\textsc{Kallina, Anna} (31.03.1874 – 04.01.1948), \emph{Schauspielerin}|pw} als Gäſten. –\pend
           \pstart
           Heut fahr ich nach \textsc{Franzensbad}\oindex{Franzensbad@\textbf{Franzensbad}|pw} hinüber.\pend
           \pstart
           Leben Sie wohl, ſagen Sie mir, wie Sie ſich befinden, ob Sie ſich i{\geminationm}er mehr nach dem Herbſt ſehnen und ſchreiben Sie
                    mir ſehr bald. Zum Arbeiten bin ich noch {\pb}nicht
                        geko{\geminationm}en; Sie? – Aber ich freu mich darauf, und
                    das iſt eigentlich viel beſſer.\pend
           \pstart Herzlichen Gruſs.\hspace*{2em}Ihr
                        \spacefill\mbox{Arthur}\pend{}\endnumbering\briefempfaengerindex{Hofmannsthal, Hugo von@\textsc{Hofmannsthal, Hugo von}!zzzSchnitzler, Arthur@\emph{von Arthur Schnitzler}!1895-07-102@{10. 7. 1895}|)be}\mylabel{h}\end{ledgroupsized}  \newcommand{\dateiname}{L00462}\newcommand{\titel}{Arthur Schnitzler an Hugo von Hofmannsthal, 10. 7. 1895}\newcommand{\editorInnen}{Martin Anton Müller und Gerd-Hermann Susen}%% latex-leseansicht-abspann.tex
%% Abspann für die Leseansicht.
%% Der Schalter \ifkorrekturansicht ist bereits durch den Vorspann gesetzt.

%% latex-abspann.tex
%% Gemeinsamer Abspann für Korrekturansicht und Leseansicht.
%% Setzt den Schalter \ifkorrekturansicht voraus (gesetzt in den
%% einbindenden Dateien latex-korrekturansicht-abspann.tex bzw.
%% latex-leseansicht-abspann.tex).
%% ---------------------------------------------------------------

\normalsize

% Das esempio-Environment wird nur in der Leseansicht benötigt
\ifkorrekturansicht\else
\newenvironment{esempio}[3]%
{
    \vspace{1.5ex}
    \rlap{\underline{#1}}
    \par
    \setlength{\parindent}{0cm}
    \nopagebreak
    \leftskip=#2cm
    \rightskip=#3cm
}
{
    \par
}
\fi

\doendnotes{C}
\bigskip
\vfill

\clearpage

\footnotesize

\ifkorrekturansicht
  \lohead{\textsc{register}}
\fi

% theindex-Environment neu definieren ohne reledmac
\makeatletter
\renewenvironment{theindex}{%
  \ifkorrekturansicht
    \section*{\indexname}%
  \else
    \subsubsection*{Index der erwähnten Entitäten}%
  \fi
  \setlength{\parindent}{0pt}%
  \setlength{\parskip}{0pt plus 0.3pt}%
  \let\item\@idxitem
}{%
  \ifkorrekturansicht\clearpage\fi
}
\makeatother

\IfFileExists{\jobname-pw.ind}{\input{\jobname-pw.ind}}{}

% Quellenangabe nur in der Leseansicht
\ifkorrekturansicht\else
% Fallback-Definitionen, falls die .tex-Datei \titel etc. nicht gesetzt hat
\providecommand{\titel}{}
\providecommand{\editorInnen}{}
\providecommand{\dateiname}{\jobname}

\vspace{3cm}

\vfill

\footnotesize
\textsc{Quelle}: \titel. Herausgegeben von {\editorInnen}. In: \emph{Arthur Schnitzler: Briefwechsel mit Autorinnen und Autoren}.
 Digitale Edition, https://schnitzler-briefe.acdh.oeaw.ac.at/{\dateiname}.html (Stand \today)
\fi

\end{document}


      