%% latex-korrekturansicht-vorspann.tex
%% Vorspann für die Korrekturansicht.
%% Lädt die gemeinsame Datei latex-vorspann.tex mit gesetztem Schalter.

\newif\ifkorrekturansicht
\korrekturansichttrue

\input{../tex-inputs/latex-vorspann}


\section[Arthur Schnitzler an Hugo von Hofmannsthal, 10. 7. 1895]{L00462 Arthur Schnitzler an Hugo von Hofmannsthal, 10. 7. 1895}
\nopagebreak\mylabel{L00462v}
\rehead{ }\normalsize\beginnumbering\briefempfaengerindex{Hofmannsthal, Hugo von@\textsc{Hofmannsthal, Hugo von}!zzzSchnitzler, Arthur@\emph{von Arthur Schnitzler}!1895-07-102@{10. 7. 1895}|(be}
\toendnotes[C]{\smallbreak\pagebreak[2]}\Standort{FDH, Hs-30885,58.}
\physDesc{Brief, 2 Blätter, 7 Seiten, 3187 Zeichen
\newline{}Handschrift: schwarze Tinte, deutsche Kurrent}
\buchAbdrucke{\weitereDrucke{1) Hugo von Hofmannsthal, Arthur Schnitzler: \emph{Briefwechsel}. Frankfurt am Main: \emph{S. Fischer} 1964, S. 54–56.} \weitereDrucke{2) Arthur Schnitzler: \emph{Briefe 1875–1912}. Frankfurt am Main: \emph{S. Fischer} 1981, S. 264–265.} }\toendnotes[C]{\smallbreak}
\pstart
           \raggedleft{}{\pb}\textsc{Marienbad}\oindex{Marienbad@\textbf{Marienbad}, \emph{P.PPL}|pw}{ }10/7 95.\pend
           
\pstart{}Mein lieber Hugo,\pend\vspace{0.5em}
\pstart
           ich bin in Prag\oindex{Prag@\textbf{Prag}, \emph{A.ADM1}|pw} geweſen, in \textsc{Karlsbad}\oindex{Karlsbad@\textbf{Karlsbad}, \emph{P.PPLA}|pw} und nun bin ich hier, wo ich wohl bis Ende der Woche oder Anfang der nächſten
               bleiben werde. Dann erſcheine ich in Iſchl, \textsc{Pension Petter}\oindex{Hotel und Pension Rudolfshoehe (Leopold Petter)@\textbf{Hotel und Pension Rudolfshöhe (Leopold Petter)}, \emph{Hotel (K.HTL)}|pw}, wo ich hoffentlich eine Nachricht von Ihnen finden werde. Dieſe Zeilen werden
               in einer Dachka{\geminationm}er, nein, eigentlich in einem Dachſalon
               geſchrieben – zwei Fenſter mit eben ſovielen Ausſichten; beide ſtehen offen und alles
               papierne {\pb}auf dem Tiſch flattert und knittert. – Ich hab
               mich ſchon an manchem ſchönen freuen können und fühle mich im ganzen wohl, ohne in
               irgend einem Augenblick zu einem Hochgefühl geko{\geminationm}en zu
               ſein. In Prag\oindex{Prag@\textbf{Prag}, \emph{A.ADM1}|pw} das merkwürdigſte ein alter jüdiſcher Friedhof\oindex{Alter Juedischer Friedhof@\textbf{Alter Jüdischer Friedhof}, \emph{Friedhof (K.FRH)}|pwv}, der
               langſam verſinkt. Seit mehr als hundert Jahren begräbt man dort nicht mehr, und die
               Grabſteine u. Sarkophage werden langſam von der Erde eingeſchlürft. Einige ſind noch
               zur Hälfte über dem Boden, von andern ſieht man gerade noch die oberſten Ränder. Alle
               dicht aneinander, viele ſchief, manche gegen einander geneigt, ſich gegenſeitig {\pb}ſtützend. Darüber ſtille nicht ſehr hohe tiefgrüne Bäume,
               mit ſo dichtem Laub, als wenn ſie alle zuſa{\geminationm}en ein Dach
               ſein wollten für dieſen Friedhof\oindex{Alter Juedischer Friedhof@\textbf{Alter Jüdischer Friedhof}, \emph{Friedhof (K.FRH)}|pwv}, der ſtirbt. – Die ethnographiſche
                  Ausſtellung\orgindex{Cecho-slavische ethnographische Ausstellung@Čecho-slavische ethnographische Ausstellung|pw}: viel intereſſante Stuben und Coſtüme. – Der Hradſchin\oindex{Prager Burg@\textbf{Prager Burg}, \emph{Gebäude (K.GBD)}|pw}, da hat mir ein Führer erzählt, daſs man im Volk in
                  Prag\oindex{Prag@\textbf{Prag}, \emph{A.ADM1}|pw} den Kronprinzen Rudolf\pwindex{Rudolf von Oesterreich-Ungarn 21.08.1858 – 30.01.1889@\textsc{Rudolf von Österreich-Ungarn} (21.08.1858 – 30.01.1889), \emph{Erzherzog/Erzherzogin, Kronprinz/Kronprinzessin}|pw} nicht für todt hält: ein Kutſcher hat ihn im Jahr
                  91{ }ſogar in die Ausſtellung geführt, ganz beſti{\geminationm}t, er hat ihn erkannt. – Ein Hofbedienſteter, der ſehr
               gemeſſen und höflich erläutert, und der ſich, we{\geminationn} ihm
               was unhöfiſches paſſirt, ſchnell wieder derfangt. Z. B. {\pb}wie er den Fenſterſturz berichtet: »Hier hat man die drei in den Graben
               hinuntergeſchmiſſen, \textsc{reſpective} hinuntergeworfen«.\pend
           
\pstart
           – In \textsc{Karlsbad}\oindex{Karlsbad@\textbf{Karlsbad}, \emph{P.PPLA}|pw} Wirkung der Curgäſte als Maſſe, wie jeder das ſeine beiträgt zum Eindruck:
               Weltcurort; – aber man darf ſie nicht einzeln anſehn, we{\geminationn} man das große ſpüren will – denn dann ſind’s Hochſtapler, Zuckerkranke, polniſche\oindex{Polen@\textbf{Polen}, \emph{A.PCLI}|pw} Juden, \label{K_L00462-1v}\edtext{Gigerln}{\lemma{\textnormal{\emph{Gigerln}}}\Cendnote{\textnormal{österreichisch Gigerl: Geck}}}\label{K_L00462-1}, \textsc{Besesny}\pwindex{Bezecný, Josef von 05.02.1829 – 17.06.1904@\textsc{Bezecný, Josef von} (05.02.1829 – 17.06.1904), \emph{Pianist/Pianistin, Theaterintendant/Theaterintendantin, Beamter/Beamte}|pw}, \textsc{Broda}\pwindex{Broda, Moritz 23.03.1842 – 16.09.1910@\textsc{Broda, Moritz} (23.03.1842 – 16.09.1910), \emph{Schauspieler/Schauspielerin}|pw}, \textsc{Wilhelmine Sandrock}\pwindex{Sandrock, Wilhelmine 05.02.1861 – 29.11.1948@\textsc{Sandrock, Wilhelmine} (05.02.1861 – 29.11.1948), \emph{Schauspieler/Schauspielerin}|pw} – allerdings auch Sonnenthal\pwindex{Sonnenthal, Adolf von 1834-12-21 – 1909-04-04@\textsc{Sonnenthal, Adolf von} (1834-12-21 – 1909-04-04), \emph{Schauspieler/Schauspielerin}|pw}
               (Uebergang,), einige wirklich elegante Menſchen und ein paar entzückend ſchöne
               Amerikanerinnen. – Ich bin aus \textsc{K}\oindex{Karlsbad@\textbf{Karlsbad}, \emph{P.PPLA}|pw}. {\pb}bald fort – man ka{\geminationn}
               dort nur 2 Tage oder 4 Wochen bleiben. – Hier, in Marienbad\oindex{Marienbad@\textbf{Marienbad}, \emph{P.PPL}|pw}, iſt es behaglicher, und die Leute, die hier ſind, ſind nicht ſo
               ſtolz darauf, daſs ſie da ſind, wie in \textsc{Karlsbad}\oindex{Karlsbad@\textbf{Karlsbad}, \emph{P.PPLA}|pw}. – Ein großer freundlicher Park, in dem hohe ſchöne Häuſer ſtehn, die lauter
               Hotels ſind, und ringsherum beſcheidene Hügel, die ſich freuen, weil man breite Wege
               zu ihnen hingeführt hat, und Wälder, die ſich freuen, weil ſo brave dicke Menſchen in
               ihnen ſpazieren gehn; auch die Wirthe und Kellner {\pb}und
               Dienſtmänner lächeln hier; während ſie in \textsc{K.}\oindex{Karlsbad@\textbf{Karlsbad}, \emph{P.PPLA}|pw} alle ſehr ernſt ſind und ihrer Würde nie vergeſſen können. – Hier hab ich \textsc{Hänsel u Grethel}\pwindex{Haensel und Gretel. Maerchenspiel in drei Bildern@\emph{Hänsel und Gretel. Märchenspiel in drei Bildern}|pw} im Theater geſehn, in \textsc{K.}\oindex{Karlsbad@\textbf{Karlsbad}, \emph{P.PPLA}|pw} den armen Jonathan\pwindex{arme Jonathan. Operette in drei Acten@\emph{Der arme Jonathan. Operette in drei Acten}|pw}, in Prag\oindex{Prag@\textbf{Prag}, \emph{A.ADM1}|pw} (böhmiſch\oindex{Boehmen@\textbf{Böhmen}, \emph{L.RGN}|pw}) Dimitrij\pwindex{Dimitrij@\emph{Dimitrij}|pw}, Oper v. Dvorak\pwindex{Dvořák, Antonín 1841-09-08 – 1904-05-01@\textsc{Dvořák, Antonín} (1841-09-08 – 1904-05-01), \emph{Komponist/Komponistin}|pw} u. (deutſch) – \textsc{Attaché}\pwindex{Attache. Lustspiel in 4 Acten@\emph{Ein Attaché. Lustspiel in 4 Acten}|pw} mit \textsc{Hartmann}\pwindex{Hartmann, Ernst 08.01.1844 – 10.10.1911@\textsc{Hartmann, Ernst} (08.01.1844 – 10.10.1911), \emph{Schauspieler/Schauspielerin}|pw} u \textsc{Kallina}\pwindex{Kallina, Anna 31.03.1874 – 04.01.1948@\textsc{Kallina, Anna} (31.03.1874 – 04.01.1948), \emph{Schauspieler/Schauspielerin}|pw} als Gäſten. –\pend
           
\pstart
           Heut fahr ich nach \textsc{Franzensbad}\oindex{Franzensbad@\textbf{Franzensbad}, \emph{P.PPL}|pw} hinüber.\pend
           
\pstart
           Leben Sie wohl, ſagen Sie mir, wie Sie ſich befinden, ob Sie ſich i{\geminationm}er mehr nach dem Herbſt ſehnen und ſchreiben Sie mir
               ſehr bald. Zum Arbeiten bin ich noch {\pb}nicht geko{\geminationm}en; Sie? – Aber ich freu mich darauf, und das iſt
               eigentlich viel beſſer.\pend
           \pstart Herzlichen Gruſs.\hspace*{2em}Ihr \spacefill\mbox{Arthur}\pend{}\selectlanguage{ngerman}\endnumbering\briefempfaengerindex{Hofmannsthal, Hugo von@\textsc{Hofmannsthal, Hugo von}!zzzSchnitzler, Arthur@\emph{von Arthur Schnitzler}!1895-07-102@{10. 7. 1895}|)be}\mylabel{L00462h}  \normalsize

\doendnotes{C}
\bigskip
\vfill

\clearpage

\footnotesize

\lohead{\textsc{register}}

% Definiere theindex-Environment komplett neu ohne reledmac
\makeatletter
\renewenvironment{theindex}{%
  \section*{\indexname}%
  \setlength{\parindent}{0pt}%
  \setlength{\parskip}{0pt plus 0.3pt}%
  \let\item\@idxitem
}{%
  \clearpage
}
\makeatother

\IfFileExists{\jobname-pw.ind}{\input{\jobname-pw.ind}}{}

\end{document}

      