%% latex-korrekturansicht-vorspann.tex
%% Vorspann für die Korrekturansicht.
%% Lädt die gemeinsame Datei latex-vorspann.tex mit gesetztem Schalter.

\newif\ifkorrekturansicht
\korrekturansichttrue

\input{../tex-inputs/latex-vorspann}


\section[Arthur Schnitzler an Richard Beer-Hofmann, 15. 6. 1911]{L02022 Arthur Schnitzler an Richard Beer-Hofmann, 15. 6. 1911}
\nopagebreak\mylabel{L02022v}
\rehead{ }\normalsize\beginnumbering\briefempfaengerindex{Beer-Hofmann, Richard@\textsc{Beer-Hofmann, Richard}!zzzSchnitzler, Arthur@\emph{von Arthur Schnitzler}!1911-06-151@{15. 6. 1911}|(be}
\toendnotes[C]{\smallbreak\pagebreak[2]}\Standort{YCGL, MSS 31.}
\physDesc{Brief, 1 Blatt, 2 Seiten, 345 Zeichen
\newline{}Handschrift: Bleistift, deutsche Kurrent}
\buchAbdrucke{\weitereDrucke{Arthur Schnitzler, Richard Beer-Hofmann: \emph{Briefwechsel 1891–1931}. Wien, Zürich: \emph{Europaverlag} 1992, S. 214.} }\toendnotes[C]{\smallbreak}
\pstart
           {\pb}\textcolor{gray}{\textbf{Dr. Arthur Schnitzler}}\hfill 15/6 911\pend
           
\pstart
           \textcolor{gray}{\textbf{Wien XVIII. Sternwartestrasse 71\oindex{Sternwartestrasse 71@\textbf{Sternwartestraße 71}, \emph{Wohngebäude (K.WHS)}|pw}}}\pend
           
\pstart{}lieber Richard,\pend\vspace{0.5em}
\pstart
           wollen Sie heute nach dem Nachtmahl, ſo um 9 etwa mit Ihrer Frau\pwindex{Beer-Hofmann, Paula 25.02.1879 – 30.10.1939@\textsc{Beer-Hofmann, Paula} (25.02.1879 – 30.10.1939)|pwv} herüberko{\geminationm}en ſo würde es uns freuen. Sie werden, bereits
               geſättigt, \textsc{Rosenbaum}\pwindex{Rosenbaum, Richard 04.11.1867 – 25.06.1942@\textsc{Rosenbaum, Richard} (04.11.1867 – 25.06.1942), \emph{Dramaturg/Dramaturgin, Verleger/Verlegerin}|pw} (nicht den\pwindex{Rostler, Karl 20.04.1872 – 31.01.1940@\textsc{Rostler, Karl} (20.04.1872 – 31.01.1940), \emph{Hotelportier/Hotelportierin}|pwv} vom Berg
                  \label{T_L02022-1v}\edtext{(Semmering\oindex{Semmering@\textbf{Semmering}, \emph{A.ADM3}|pw})}{\lemma{\textnormal{\emph{(Semmering)}}}\Cendnote{\textnormal{Schnitzler verwendet eckige
               Klammern.}}}\label{T_L02022-1}, ſondern \label{K_L02022-1v}\edtext{den vom Berger\pwindex{Berger, Alfred von 30.04.1853 – 24.08.1912@\textsc{Berger, Alfred von} (30.04.1853 – 24.08.1912), \emph{Schriftsteller/Schriftstellerin, Journalist/Journalistin, Theaterleiter/Theaterleiterin}|pw}}{\lemma{\textnormal{\emph{den vom Berger}}}\Cendnote{\textnormal{Richard Rosenbaum\pwindex{Rosenbaum, Richard 04.11.1867 – 25.06.1942@\textsc{Rosenbaum, Richard} (04.11.1867 – 25.06.1942), \emph{Dramaturg/Dramaturgin, Verleger/Verlegerin}|pwk} war beim \emph{Burgtheater}\orgindex{Burgtheater@Burgtheater|pwk} angestellt, dessen Direktor Alfred von Berger\pwindex{Berger, Alfred von 30.04.1853 – 24.08.1912@\textsc{Berger, Alfred von} (30.04.1853 – 24.08.1912), \emph{Schriftsteller/Schriftstellerin, Journalist/Journalistin, Theaterleiter/Theaterleiterin}|pwk} war.}}}\label{K_L02022-1}) ſamt \textsc{Towska}\pwindex{Rosenbaum, Kory Elisabeth 26.06.1868 – 28.01.1930@\textsc{Rosenbaum, Kory Elisabeth} (26.06.1868 – 28.01.1930), \emph{Schriftsteller/Schriftstellerin}|pw} vorfinden; der erſtere ſehr nett, die zweitere {\pb}mir noch wenig bekannt.\pend
           
\pstart
           Und wann reiſen Sie? Wir \label{K_L02022-2v}\edtext{gegen 26.}{\lemma{\textnormal{\emph{gegen 26.}}}\Cendnote{\textnormal{Zu der Reise kam es
                  nicht.}}}\label{K_L02022-2} – \textsc{Seis}\oindex{Seis am Schlern@\textbf{Seis am Schlern}, \emph{P.PPL}|pw}.\pend
           
\pstart
           Herzlichſt{\\[\baselineskip]}Ihr{\\[\baselineskip]}\spacefill\mbox{A.}\pend
           \leftskip=0em{}\selectlanguage{ngerman}\endnumbering\briefempfaengerindex{Beer-Hofmann, Richard@\textsc{Beer-Hofmann, Richard}!zzzSchnitzler, Arthur@\emph{von Arthur Schnitzler}!1911-06-151@{15. 6. 1911}|)be}\mylabel{L02022h}  \normalsize

\doendnotes{C}
\bigskip
\vfill

\clearpage

\footnotesize

\lohead{\textsc{register}}

% Definiere theindex-Environment komplett neu ohne reledmac
\makeatletter
\renewenvironment{theindex}{%
  \section*{\indexname}%
  \setlength{\parindent}{0pt}%
  \setlength{\parskip}{0pt plus 0.3pt}%
  \let\item\@idxitem
}{%
  \clearpage
}
\makeatother

\IfFileExists{\jobname-pw.ind}{\input{\jobname-pw.ind}}{}

\end{document}

      