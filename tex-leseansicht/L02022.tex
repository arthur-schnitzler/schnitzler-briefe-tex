%% latex-leseansicht-vorspann.tex
%% Vorspann für die Leseansicht.
%% Lädt die gemeinsame Datei latex-vorspann.tex mit nicht gesetztem Schalter.

\newif\ifkorrekturansicht
\korrekturansichtfalse

\input{../tex-inputs/latex-vorspann}


\section[Arthur Schnitzler an Richard Beer-Hofmann, 15. 6. 1911]{L02022 Arthur Schnitzler an Richard Beer-Hofmann, 15. 6. 1911}
\nopagebreak\mylabel{L02022v}
\rehead{ }\normalsize\beginnumbering\briefempfaengerindex{Beer-Hofmann, Richard@\textsc{Beer-Hofmann, Richard}!zzzSchnitzler, Arthur@\emph{von Arthur Schnitzler}!1911-06-151@{15. 6. 1911}|(be}
\toendnotes[C]{\smallbreak\pagebreak[2]}
\correspDesc{Versand  durch Arthur Schnitzler am 15. 6. 1911 in Wien
\newline{}Erhalt  durch Richard Beer-Hofmann am 15. 6. 1911 in Wien}\toendnotes[C]{\smallbreak}
\Standort{YCGL, MSS 31.}
\physDesc{Brief, 1 Blatt, 2 Seiten, 345 Zeichen
\newline{}Handschrift: Bleistift, deutsche Kurrent}
\buchAbdrucke{\weitereDrucke{Arthur Schnitzler, Richard Beer-Hofmann: \emph{Briefwechsel 1891–1931}. Herausgegeben von Konstanze Fliedl. Wien, Zürich: \emph{Europaverlag} 1992, S. 214.} }\toendnotes[C]{\smallbreak}
\pstart
           {\pb}\textcolor{gray}{\textbf{Dr. Arthur Schnitzler}}\hfill 15/6 911\pend
           
\pstart
           \textcolor{gray}{\textbf{Wien XVIII. Sternwartestrasse 71\oindex{Wien@\textbf{Wien}!XVIII., Währing@\textbf{XVIII., Währing}!Sternwartestraße 71@\textbf{Sternwartestraße 71}, \emph{Wohngebäude}|pw}}}\pend
           
\pstart{}lieber Richard,\pend\vspace{0.5em}
\pstart
           wollen Sie heute nach dem Nachtmahl,{ }ſo um 9 etwa mit Ihrer Frau\pwindex{Beer-Hofmann, Paula 25.\,2.\,1879 Wien – 30.\,10.\,1939 Zürich@\textsc{Beer-Hofmann, Paula} (25.\,2.\,1879 Wien – 30.\,10.\,1939 Zürich)|pwv} herüberko{\geminationm}en{ }ſo würde es uns freuen. Sie werden, bereits
               geſättigt, \textsc{Rosenbaum}\pwindex{Rosenbaum, Richard 4.\,11.\,1867 Žikov – 25.\,6.\,1942 Konzentrationslager Theresienstadt@\textsc{Rosenbaum, Richard} (4.\,11.\,1867 Žikov – 25.\,6.\,1942 Konzentrationslager Theresienstadt), \emph{Dramaturg, Verleger}|pw} (nicht den\pwindex{Rostler, Karl 20.\,4.\,1872 Wien – 31.\,1.\,1940 ebd.@\textsc{Rostler, Karl} (20.\,4.\,1872 Wien – 31.\,1.\,1940 ebd.), \emph{Hotelportier}|pwv} vom Berg
                  \label{T_L02022-1v}\edtext{(Semmering\oindex{Semmering@\textbf{Semmering}, \emph{Verwaltungsgebiet}|pw})}{\lemma{\textnormal{\emph{(Semmering)}}}\Cendnote{\textnormal{Schnitzler verwendet eckige
               Klammern.}}}\label{T_L02022-1},{ }ſondern \label{K_L02022-1v}\edtext{den vom Berger\pwindex{Berger, Alfred von 30.\,4.\,1853 Wien – 24.\,8.\,1912 ebd.@\textsc{Berger, Alfred von} (30.\,4.\,1853 Wien – 24.\,8.\,1912 ebd.), \emph{Schriftsteller, Journalist, Theaterleiter}|pw}}{\lemma{\textnormal{\emph{den vom Berger}}}\Cendnote{\textnormal{Richard Rosenbaum\pwindex{Rosenbaum, Richard 4.\,11.\,1867 Žikov – 25.\,6.\,1942 Konzentrationslager Theresienstadt@\textsc{Rosenbaum, Richard} (4.\,11.\,1867 Žikov – 25.\,6.\,1942 Konzentrationslager Theresienstadt), \emph{Dramaturg, Verleger}|pwk} war beim \emph{Burgtheater}\orgindex{Burgtheater@Burgtheater|pwk} angestellt, dessen Direktor Alfred von Berger\pwindex{Berger, Alfred von 30.\,4.\,1853 Wien – 24.\,8.\,1912 ebd.@\textsc{Berger, Alfred von} (30.\,4.\,1853 Wien – 24.\,8.\,1912 ebd.), \emph{Schriftsteller, Journalist, Theaterleiter}|pwk} war.}}}\label{K_L02022-1}){ }ſamt \textsc{Towska}\pwindex{Rosenbaum, Kory Elisabeth 26.\,6.\,1868 Berlin – 28.\,1.\,1930 Wien@\textsc{Rosenbaum, Kory Elisabeth} (26.\,6.\,1868 Berlin – 28.\,1.\,1930 Wien), \emph{Schriftstellerin}|pw} vorfinden; der erſtere{ }ſehr nett, die zweitere {\pb}mir noch wenig bekannt.\pend
           
\pstart
           Und wann reiſen Sie? Wir \label{K_L02022-2v}\edtext{gegen 26.}{\lemma{\textnormal{\emph{gegen 26.}}}\Cendnote{\textnormal{Zu der Reise kam es
                  nicht.}}}\label{K_L02022-2} – \textsc{Seis}\oindex{Seis am Schlern@\textbf{Seis am Schlern}|pw}.\pend
           
\pstart
           Herzlichſt{\\[\baselineskip]}Ihr{\\[\baselineskip]}\spacefill\mbox{A.}\pend
           \leftskip=0em{}\selectlanguage{ngerman}\endnumbering\briefempfaengerindex{Beer-Hofmann, Richard@\textsc{Beer-Hofmann, Richard}!zzzSchnitzler, Arthur@\emph{von Arthur Schnitzler}!1911-06-151@{15. 6. 1911}|)be}\mylabel{L02022h}  \newcommand{\dateiname}{L02022}\newcommand{\titel}{Arthur Schnitzler an Richard Beer-Hofmann, 15. 6. 1911}\newcommand{\editorInnen}{Martin Anton Müller und Gerd-Hermann Susen}%% latex-leseansicht-abspann.tex
%% Abspann für die Leseansicht.
%% Der Schalter \ifkorrekturansicht ist bereits durch den Vorspann gesetzt.

%% latex-abspann.tex
%% Gemeinsamer Abspann für Korrekturansicht und Leseansicht.
%% Setzt den Schalter \ifkorrekturansicht voraus (gesetzt in den
%% einbindenden Dateien latex-korrekturansicht-abspann.tex bzw.
%% latex-leseansicht-abspann.tex).
%% ---------------------------------------------------------------

\normalsize

% Das esempio-Environment wird nur in der Leseansicht benötigt
\ifkorrekturansicht\else
\newenvironment{esempio}[3]%
{
    \vspace{1.5ex}
    \rlap{\underline{#1}}
    \par
    \setlength{\parindent}{0cm}
    \nopagebreak
    \leftskip=#2cm
    \rightskip=#3cm
}
{
    \par
}
\fi

\doendnotes{C}
\bigskip
\vfill

\clearpage

\footnotesize

\ifkorrekturansicht
  \lohead{\textsc{register}}
\fi

% theindex-Environment neu definieren ohne reledmac
\makeatletter
\renewenvironment{theindex}{%
  \ifkorrekturansicht
    \section*{\indexname}%
  \else
    \subsubsection*{Index der erwähnten Entitäten}%
  \fi
  \setlength{\parindent}{0pt}%
  \setlength{\parskip}{0pt plus 0.3pt}%
  \let\item\@idxitem
}{%
  \ifkorrekturansicht\clearpage\fi
}
\makeatother

\IfFileExists{\jobname-pw.ind}{\input{\jobname-pw.ind}}{}

% Quellenangabe nur in der Leseansicht
\ifkorrekturansicht\else
% Fallback-Definitionen, falls die .tex-Datei \titel etc. nicht gesetzt hat
\providecommand{\titel}{}
\providecommand{\editorInnen}{}
\providecommand{\dateiname}{\jobname}

\vspace{3cm}

\vfill

\footnotesize
\textsc{Quelle}: \titel. Herausgegeben von {\editorInnen}. In: \emph{Arthur Schnitzler: Briefwechsel mit Autorinnen und Autoren}.
 Digitale Edition, https://schnitzler-briefe.acdh.oeaw.ac.at/{\dateiname}.html (Stand \today)
\fi

\end{document}


