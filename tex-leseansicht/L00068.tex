%% latex-leseansicht-vorspann.tex
%% Vorspann für die Leseansicht.
%% Lädt die gemeinsame Datei latex-vorspann.tex mit nicht gesetztem Schalter.

\newif\ifkorrekturansicht
\korrekturansichtfalse

\input{../tex-inputs/latex-vorspann}


         
         \renewcommand{\erwaehntePersonen}{Personen: Adalbert von Goldschmidt, Gerhart Hauptmann, Hugo von Hofmannsthal, Emanuel Reicher}
         \renewcommand{\erwaehnteOrte}{Orte: Burgtheater, Wien}
         \renewcommand{\erwaehnteWerke}{Werke: College Crampton. Komödie in fünf Akten, Der Garten der Bérenice, Die Blinden, Die sieben Prinzessinnen}
               \section[Arthur Schnitzler an Hugo von Hofmannsthal, {[}1.? 2. 1892{]}]{ Arthur Schnitzler an Hugo von Hofmannsthal, {[}1.? 2. 1892{]}}\nopagebreak\mylabel{v}\rehead{ }\begin{ledgroupsized}[t]{13cm}\normalsize\beginnumbering \toendnotes[C]{\smallbreak\pagebreak[2]} \Standort{FDH, Hs-30885,17.}
\physDesc{Briefkarte
\newline{}Handschrift: Bleistift, deutsche Kurrent\newline{}Ordnung: 1) von Schnitzler mutmaßlich bei der Durchsicht der Korrespondenz 1929 mit Bleistift datiert: »9/\substVorne{}\textsuperscript{3}\substDazwischen{}4\substHinten{}?{ }90?«  2) mit Bleistift von unbekannter Hand nummeriert: »17«}\buchAbdrucke{\weitereDrucke{Hugo von Hofmannsthal, Arthur Schnitzler: \emph{Briefwechsel}. Hg. Therese Nickl und Heinrich Schnitzler. Frankfurt am Main: \emph{S. Fischer} 1964, S. 15.} }\toendnotes[C]{\smallbreak}\pstart
           \noindent{}{\pb}Lieber Freund, hier ſind die Bücher\pwindex{\textcolor{red}{\textsuperscript{XXXX1 indx}}!Blinden1891@\strich\emph{Die Blinden} {[}1891{]}|pwv}\pwindex{\textcolor{red}{\textsuperscript{XXXX1 indx}}!Garten der Berenice1891@\strich\emph{Der Garten der Bérenice} {[}1891{]}|pwv}\pwindex{\textcolor{red}{\textsuperscript{XXXX1 indx}}!sieben Prinzessinnen1891@\strich\emph{Die sieben Prinzessinnen} {[}1891{]}|pwv}. So{\geminationn}tag ist \label{K_L00068_1v}\edtext{\textsc{Goldschmidt}\pwindex{Goldschmidt, Adalbert von 1848-05-05 – 1906-12-21@\textsc{Goldschmidt, Adalbert von} (1848-05-05 – 1906-12-21), \emph{Schriftsteller, Komponist}|pw}}{\lemma{\textnormal{\emph{Goldschmidt}}}\Cendnote{\textnormal{Am 7. 2. 1892 fand eine Matinée mit Emanuel Reicher\pwindex{Reicher, Emanuel 18.06.1849 – 15.05.1924@\textsc{Reicher, Emanuel} (18.06.1849 – 15.05.1924), \emph{Schauspieler}|pwk} im Haus von Adalbert von Goldschmidt\pwindex{Goldschmidt, Adalbert von 1848-05-05 – 1906-12-21@\textsc{Goldschmidt, Adalbert von} (1848-05-05 – 1906-12-21), \emph{Schriftsteller, Komponist}|pwk} statt, an der
                            Schnitzler\pwindex{Schnitzler, Arthur 15.05.1862 – 21.10.1931@\textsc{Schnitzler, Arthur} (15.05.1862 – 21.10.1931), \emph{Schriftsteller, Mediziner}|pwk} teilnahm.}}}\label{K_L00068_1h} von
                        3 an, alſo wohl bis 6. Und am Abend bin ich
                    eingeladen. Ich fände es hübſch, we{\geminationn} wir an irgend
                    einem Wochentagsabend die Zuſa{\geminationm}enkunft arrangirten.
                    Z. B. Samſtag{ }{\pb}Abend um 7 Uhr bei mir? Oder Anfangs nächſter Woche?
                        \label{K_L00068_2v}\edtext{Montag z. B. – Doch da ist \textsc{Crampton}\pwindex{Hauptmann, Gerhart 15.11.1862 – 06.06.1946@\textsc{Hauptmann, Gerhart} (15.11.1862 – 06.06.1946), \emph{Schriftsteller}!College Crampton. Komoedie in fuenf Akten21. 1. 1892@\strich\emph{College Crampton. Komödie in fünf Akten} {[}21. 1. 1892{]}|pw}}{\lemma{\textnormal{\emph{Montag … Crampton}}}\Cendnote{\textnormal{Schnitzler\pwindex{Schnitzler, Arthur 15.05.1862 – 21.10.1931@\textsc{Schnitzler, Arthur} (15.05.1862 – 21.10.1931), \emph{Schriftsteller, Mediziner}|pwk} besuchte die Premiere von Gerhart Hauptmann\pwindex{Hauptmann, Gerhart 15.11.1862 – 06.06.1946@\textsc{Hauptmann, Gerhart} (15.11.1862 – 06.06.1946), \emph{Schriftsteller}|pwk}s \emph{College Crampton}\pwindex{Hauptmann, Gerhart 15.11.1862 – 06.06.1946@\textsc{Hauptmann, Gerhart} (15.11.1862 – 06.06.1946), \emph{Schriftsteller}!College Crampton. Komoedie in fuenf Akten21. 1. 1892@\strich\emph{College Crampton. Komödie in fünf Akten} {[}21. 1. 1892{]}|pwk} im Burgtheater\oindex{Burgtheater@\textbf{Burgtheater}|pwk} am 8. 2. 1892 (\emph{Cambridge University Library}, A 179a).}}}\label{K_L00068_2h}.
                        Mittwoch? –\pend
           \pstart
           Herzlichſt Ihr{\\[\baselineskip]}\spacefill\mbox{Arthur}\pend
           \leftskip=0em{}
         
         \endnumbering\mylabel{h}\end{ledgroupsized}  \newcommand{\dateiname}{L00068}\newcommand{\titel}{Arthur Schnitzler an Hugo von Hofmannsthal, [1.? 2. 1892]}\newcommand{\editorInnen}{Martin Anton Müller und Gerd-Hermann Susen}%% latex-leseansicht-abspann.tex
%% Abspann für die Leseansicht.
%% Der Schalter \ifkorrekturansicht ist bereits durch den Vorspann gesetzt.

%% latex-abspann.tex
%% Gemeinsamer Abspann für Korrekturansicht und Leseansicht.
%% Setzt den Schalter \ifkorrekturansicht voraus (gesetzt in den
%% einbindenden Dateien latex-korrekturansicht-abspann.tex bzw.
%% latex-leseansicht-abspann.tex).
%% ---------------------------------------------------------------

\normalsize

% Das esempio-Environment wird nur in der Leseansicht benötigt
\ifkorrekturansicht\else
\newenvironment{esempio}[3]%
{
    \vspace{1.5ex}
    \rlap{\underline{#1}}
    \par
    \setlength{\parindent}{0cm}
    \nopagebreak
    \leftskip=#2cm
    \rightskip=#3cm
}
{
    \par
}
\fi

\doendnotes{C}
\bigskip
\vfill

\clearpage

\footnotesize

\ifkorrekturansicht
  \lohead{\textsc{register}}
\fi

% theindex-Environment neu definieren ohne reledmac
\makeatletter
\renewenvironment{theindex}{%
  \ifkorrekturansicht
    \section*{\indexname}%
  \else
    \subsubsection*{Index der erwähnten Entitäten}%
  \fi
  \setlength{\parindent}{0pt}%
  \setlength{\parskip}{0pt plus 0.3pt}%
  \let\item\@idxitem
}{%
  \ifkorrekturansicht\clearpage\fi
}
\makeatother

\IfFileExists{\jobname-pw.ind}{\input{\jobname-pw.ind}}{}

% Quellenangabe nur in der Leseansicht
\ifkorrekturansicht\else
% Fallback-Definitionen, falls die .tex-Datei \titel etc. nicht gesetzt hat
\providecommand{\titel}{}
\providecommand{\editorInnen}{}
\providecommand{\dateiname}{\jobname}

\vspace{3cm}

\vfill

\footnotesize
\textsc{Quelle}: \titel. Herausgegeben von {\editorInnen}. In: \emph{Arthur Schnitzler: Briefwechsel mit Autorinnen und Autoren}.
 Digitale Edition, https://schnitzler-briefe.acdh.oeaw.ac.at/{\dateiname}.html (Stand \today)
\fi

\end{document}


      