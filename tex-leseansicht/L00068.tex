%% latex-korrekturansicht-vorspann.tex
%% Vorspann für die Korrekturansicht.
%% Lädt die gemeinsame Datei latex-vorspann.tex mit gesetztem Schalter.

\newif\ifkorrekturansicht
\korrekturansichttrue

\input{../tex-inputs/latex-vorspann}


\section[Arthur Schnitzler an Hugo von Hofmannsthal, {[}1.? 2. 1892{]}]{L00068 Arthur Schnitzler an Hugo von Hofmannsthal, {[}1.? 2. 1892{]}}
\nopagebreak\mylabel{L00068v}
\rehead{ }\normalsize\beginnumbering\briefempfaengerindex{Hofmannsthal, Hugo von@\textsc{Hofmannsthal, Hugo von}!zzzSchnitzler, Arthur@\emph{von Arthur Schnitzler}!1892-02-013@{{[}1.? 2. 1892{]}}|(be}
\toendnotes[C]{\smallbreak\pagebreak[2]}\Standort{FDH, Hs-30885,17.}
\physDesc{Briefkarte, 328 Zeichen
\newline{}Handschrift: Bleistift, deutsche Kurrent
\newline{}Hofmannsthal: mit Bleistift von Schnitzler mutmaßlich bei der Durchsicht der Korrespondenz
                                    1929 datiert: »9/\substVorne{}\textsuperscript{3}\substDazwischen{}4\substHinten{}?{ }90?« 
\newline{}Ordnung: mit Bleistift von unbekannter Hand nummeriert:
                                    »17« }
\buchAbdrucke{\weitereDrucke{Hugo von Hofmannsthal, Arthur Schnitzler: \emph{Briefwechsel}. Frankfurt am Main: \emph{S. Fischer} 1964, S. 15.} }\toendnotes[C]{\smallbreak}
\pstart
           \noindent{}{\pb}Lieber Freund, hier ſind die Bücher\pwindex{Blinden@\emph{Die Blinden}|pwv}\pwindex{Garten der Berenice@\emph{Der Garten der Bérenice}|pwv}\pwindex{sieben Prinzessinnen@\emph{Die sieben Prinzessinnen}|pwv}. So{\geminationn}tag ist \label{K_L00068-1v}\edtext{\textsc{Goldschmidt}\pwindex{Goldschmidt, Adalbert von 1848-05-05 – 1906-12-21@\textsc{Goldschmidt, Adalbert von} (1848-05-05 – 1906-12-21), \emph{Schriftsteller/Schriftstellerin, Komponist/Komponistin}|pw}}{\lemma{\textnormal{\emph{Goldschmidt}}}\Cendnote{\textnormal{Am 7. 2. 1892 fand eine Matinée mit Emanuel Reicher\pwindex{Reicher, Emanuel 18.06.1849 – 15.05.1924@\textsc{Reicher, Emanuel} (18.06.1849 – 15.05.1924), \emph{Schauspieler/Schauspielerin}|pwk} im Haus von Adalbert von Goldschmidt\pwindex{Goldschmidt, Adalbert von 1848-05-05 – 1906-12-21@\textsc{Goldschmidt, Adalbert von} (1848-05-05 – 1906-12-21), \emph{Schriftsteller/Schriftstellerin, Komponist/Komponistin}|pwk} statt, an der Schnitzler teilnahm.}}}\label{K_L00068-1} von
                  3 an, alſo wohl bis 6. Und am Abend bin ich eingeladen.
               Ich fände es hübſch, we{\geminationn} wir an irgend einem
               Wochentagsabend die Zuſa{\geminationm}enkunft arrangirten. Z. B.
                  Samſtag{ }{\pb}Abend um 7 Uhr bei mir? Oder Anfangs nächſter Woche?
                  \label{K_L00068-2v}\edtext{Montag z. B. – Doch da ist \textsc{Crampton}\pwindex{College Crampton. Komoedie in fuenf Akten@\emph{College Crampton. Komödie in fünf Akten}|pw}}{\lemma{\textnormal{\emph{Montag … Crampton}}}\Cendnote{\textnormal{Schnitzler besuchte die Premiere von Gerhart Hauptmanns\pwindex{Hauptmann, Gerhart 15.11.1862 – 06.06.1946@\textsc{Hauptmann, Gerhart} (15.11.1862 – 06.06.1946), \emph{Schriftsteller/Schriftstellerin}|pwk}{ }\emph{College Crampton}\pwindex{College Crampton. Komoedie in fuenf Akten@\emph{College Crampton. Komödie in fünf Akten}|pwk} im Burgtheater\oindex{Burgtheater@\textbf{Burgtheater}, \emph{S.THTR}|pwk} am 8. 2. 1892 (\emph{Cambridge University Library}, A 179a).}}}\label{K_L00068-2}.
                  Mittwoch? –\pend
           
\pstart
           Herzlichſt Ihr{\\[\baselineskip]}\spacefill\mbox{Arthur}\pend
           \leftskip=0em{}\selectlanguage{ngerman}\endnumbering\briefempfaengerindex{Hofmannsthal, Hugo von@\textsc{Hofmannsthal, Hugo von}!zzzSchnitzler, Arthur@\emph{von Arthur Schnitzler}!1892-02-013@{{[}1.? 2. 1892{]}}|)be}\mylabel{L00068h}  \normalsize

\doendnotes{C}
\bigskip
\vfill

\clearpage

\footnotesize

\lohead{\textsc{register}}

% Definiere theindex-Environment komplett neu ohne reledmac
\makeatletter
\renewenvironment{theindex}{%
  \section*{\indexname}%
  \setlength{\parindent}{0pt}%
  \setlength{\parskip}{0pt plus 0.3pt}%
  \let\item\@idxitem
}{%
  \clearpage
}
\makeatother

\IfFileExists{\jobname-pw.ind}{\input{\jobname-pw.ind}}{}

\end{document}

      