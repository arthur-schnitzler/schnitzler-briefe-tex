%% latex-leseansicht-vorspann.tex
%% Vorspann für die Leseansicht.
%% Lädt die gemeinsame Datei latex-vorspann.tex mit nicht gesetztem Schalter.

\newif\ifkorrekturansicht
\korrekturansichtfalse

\input{../tex-inputs/latex-vorspann}


\section[Berta Zuckerkandl an Arthur Schnitzler, {[}9. 10. 1911?{]}]{L04005 Berta Zuckerkandl an Arthur Schnitzler, {[}9. 10. 1911?{]}}
\nopagebreak\mylabel{L04005v}
\rehead{ }\normalsize\beginnumbering\briefempfaengerindex{Schnitzler, Arthur@\textsc{Schnitzler, Arthur}!zzzZuckerkandl, Berta@\emph{von Berta Zuckerkandl}!1911-10-092@{{[}9. 10. 1911?{]}}|(be}
\toendnotes[C]{\smallbreak\pagebreak[2]}
\correspDesc{Versand  durch Berta Zuckerkandl am [9. 10. 1911?] in Wien
\newline{}Erhalt  durch Arthur Schnitzler im Zeitraum [9. 10. 1911
                  – 12. 10. 1911?] in Wien}\toendnotes[C]{\smallbreak}
\Standort{CUL, Schnitzler, B 200.}
\physDesc{Brief, 1 Blatt, 4 Seiten, 739 Zeichen
\newline{}Handschrift: schwarze Tinte, lateinische Kurrent
\newline{}Schnitzler: 1) mit Bleistift beschriftet: »Zucker«  2) mit rotem Buntstift eine Unterstreichung}\toendnotes[C]{\smallbreak}
\pstart
           \raggedleft{}{\pb}\label{K_L04005-1v}\edtext{Montag}{\lemma{\textnormal{\emph{Montag}}}\Cendnote{\textnormal{Das Korrespondenzstück ist nicht datiert. Der Montag in
                     der Woche der Generalprobe\eventindex{Burgtheater@\textbf{Burgtheater}!Generalprobe von Das weite Land, 13.10.1911@Generalprobe von Das weite Land, 13.10.1911|pwkv} und Premiere von \emph{Das weite Land}\pwindex{Schnitzler, Arthur 15. 5. 1862 Wien – 21. 10. 1931 ebd.@\textsc{Schnitzler, Arthur} (15. 5. 1862 Wien – 21. 10. 1931 ebd.), \emph{Schriftsteller, Mediziner}!weite Land. Tragikomödie in fünf Akten@\strich\emph{Das weite Land. Tragikomödie in fünf Akten}|pwk}\eventindex{Burgtheater@\textbf{Burgtheater}!Premiere von Das weite Land, 14.10.1911 [I.]@Premiere von Das weite Land, 14.10.1911 [I.]|pwk} war der 9. 10. 1911, was eine Abfassung an diesem Tag nahelegt.}}}\label{K_L04005-1}.\pend
           
\pstart{}Hochverehrter Herr Doktor!\pend\vspace{0.5em}
\pstart
           Entschuldigen Sie wenn ich in diesen letzten Probe-Tagen\pwindex{Schnitzler, Arthur 15. 5. 1862 Wien – 21. 10. 1931 ebd.@\textsc{Schnitzler, Arthur} (15. 5. 1862 Wien – 21. 10. 1931 ebd.), \emph{Schriftsteller, Mediziner}!weite Land. Tragikomödie in fünf Akten@\strich\emph{Das weite Land. Tragikomödie in fünf Akten}|pwv} – störe. Ich bin nur
               Überbringerin einer Bitte Alma Mahler’s\pwindex{Mahler-Werfel, Alma Maria 31.\,8.\,1879 Wien – 11.\,12.\,1964 New York City@\textsc{Mahler-Werfel, Alma Maria} (31.\,8.\,1879 Wien – 11.\,12.\,1964 New York City)|pw}. Sie
               ist \label{K_L04005-2v}\edtext{durch ihre Trauer}{\lemma{\textnormal{\emph{durch ihre Trauer}}}\Cendnote{\textnormal{Gustav Mahler\pwindex{Mahler, Gustav 7.\,7.\,1860 Kaliště – 18.\,5.\,1911 Wien@\textsc{Mahler, Gustav} (7.\,7.\,1860 Kaliště – 18.\,5.\,1911 Wien), \emph{Theaterleiter, Komponist, Dirigent}|pwk} war am
                     18. 5. 1911 gestorben.}}}\label{K_L04005-2} gehindert das Theater\oindex{Wien@\textbf{Wien}!I., Innere Stadt@\textbf{I., Innere Stadt}!Burgtheater@\textbf{Burgtheater}, \emph{Theater}|pw} zu besuchen, und hätte den grossen {\pb}Wunsch – das weite Land\pwindex{Schnitzler, Arthur 15. 5. 1862 Wien – 21. 10. 1931 ebd.@\textsc{Schnitzler, Arthur} (15. 5. 1862 Wien – 21. 10. 1931 ebd.), \emph{Schriftsteller, Mediziner}!weite Land. Tragikomödie in fünf Akten@\strich\emph{Das weite Land. Tragikomödie in fünf Akten}|pw} zu sehen. Gustav
                  Mahler\pwindex{Mahler, Gustav 7.\,7.\,1860 Kaliště – 18.\,5.\,1911 Wien@\textsc{Mahler, Gustav} (7.\,7.\,1860 Kaliště – 18.\,5.\,1911 Wien), \emph{Theaterleiter, Komponist, Dirigent}|pw} hegte für Ihre Produktion so grosses Interesse – und so auch die Frau\pwindex{Mahler-Werfel, Alma Maria 31.\,8.\,1879 Wien – 11.\,12.\,1964 New York City@\textsc{Mahler-Werfel, Alma Maria} (31.\,8.\,1879 Wien – 11.\,12.\,1964 New York City)|pwv} welche ihm am nächsten
               stand. Desshalb frägt Alma\pwindex{Mahler-Werfel, Alma Maria 31.\,8.\,1879 Wien – 11.\,12.\,1964 New York City@\textsc{Mahler-Werfel, Alma Maria} (31.\,8.\,1879 Wien – 11.\,12.\,1964 New York City)|pw} an ob Ihre
               Vermittlung ihr den Eintritt {\pb}in die
                  \label{K_L04005-3v}\edtext{Generalprobe\eventindex{Burgtheater@\textbf{Burgtheater}!Generalprobe von Das weite Land, 13.10.1911@Generalprobe von Das weite Land, 13.10.1911|pwv}}{\lemma{\textnormal{\emph{Generalprobe}}}\Cendnote{\textnormal{Möglicherweise zielte Berta Zuckerkandl\pwindex{Zuckerkandl, Berta 13.\,4.\,1864 Wien – 16.\,10.\,1945 Paris@\textsc{Zuckerkandl, Berta} (13.\,4.\,1864 Wien – 16.\,10.\,1945 Paris), \emph{Schriftstellerin, Journalistin, Übersetzerin}|pwk} mit ihrer Anfrage auf die Kostümprobe\eventindex{Burgtheater@\textbf{Burgtheater}!Kostümprobe von Das weite Land, 11.10.1911@Kostümprobe von Das weite Land, 11.10.1911|pwk}, die am Mittwoch, den 11. 10. 1911 stattfand,
                  während die Generalprobe\eventindex{Burgtheater@\textbf{Burgtheater}!Generalprobe von Das weite Land, 13.10.1911@Generalprobe von Das weite Land, 13.10.1911|pwkv}
                  erst auf Freitag, den 13. 10. 1911
                  vonstatten ging. Im \emph{Tagebuch}\pwindex{Schnitzler, Arthur 15. 5. 1862 Wien – 21. 10. 1931 ebd.@\textsc{Schnitzler, Arthur} (15. 5. 1862 Wien – 21. 10. 1931 ebd.), \emph{Schriftsteller, Mediziner}!Tagebuch@\strich\emph{Tagebuch}|pwk}, wo Schnitzler den
                  Besuch seiner Geschwister\pwindex{Hajek, Gisela 20.\,12.\,1867 Wien – 3.\,2.\,1953 Cambridge@\textsc{Hajek, Gisela} (20.\,12.\,1867 Wien – 3.\,2.\,1953 Cambridge)|pwkv}\pwindex{Schnitzler, Julius 13.\,7.\,1865 Wien – 29.\,6.\,1939 ebd.@\textsc{Schnitzler, Julius} (13.\,7.\,1865 Wien – 29.\,6.\,1939 ebd.), \emph{Chirurg}|pwkv}\pwindex{Schnitzler, Helene 16.\,7.\,1871 Budapest – September 1941 Atlantischer Ozean@\textsc{Schnitzler, Helene} (16.\,7.\,1871 Budapest – September 1941 Atlantischer Ozean)|pwkv} bei der Generalprobe\eventindex{Burgtheater@\textbf{Burgtheater}!Generalprobe von Das weite Land, 13.10.1911@Generalprobe von Das weite Land, 13.10.1911|pwkv} notiert, erwähnt er die beiden Frauen\pwindex{Zuckerkandl, Berta 13.\,4.\,1864 Wien – 16.\,10.\,1945 Paris@\textsc{Zuckerkandl, Berta} (13.\,4.\,1864 Wien – 16.\,10.\,1945 Paris), \emph{Schriftstellerin, Journalistin, Übersetzerin}|pwkv}\pwindex{Mahler-Werfel, Alma Maria 31.\,8.\,1879 Wien – 11.\,12.\,1964 New York City@\textsc{Mahler-Werfel, Alma Maria} (31.\,8.\,1879 Wien – 11.\,12.\,1964 New York City)|pwkv} nicht, vgl. A. S.: \emph{Tagebuch}, 13. 10. 1911.}}}\label{K_L04005-3}
               ermöglichen würde. Vielleicht – dass von Ihnen befürwortet – diese Ausnahme gemacht
               werden würde.\pend
           
\pstart
           Um Ihnen hochverehrter Herr Doktor eine Antwort-Mühe zu ersparen – werde
                  ich Mittwoch gegen 2 Uhr mein {\pb}Fräulein\pwindex{?? [Hausanstellte bei Berta Zuckerkandl] @\textsc{?? [Hausanstellte bei Berta Zuckerkandl]}|pwv} um einen
               mündlichen Bescheid zu Ihnen senden.\pend
           
\pstart
           Alle herzlichste Empfehlungen. {\\[\baselineskip]}\spacefill\mbox{Berta Zuckerkandl}\pend
           \leftskip=0em{}\selectlanguage{ngerman}\endnumbering\briefempfaengerindex{Schnitzler, Arthur@\textsc{Schnitzler, Arthur}!zzzZuckerkandl, Berta@\emph{von Berta Zuckerkandl}!1911-10-092@{{[}9. 10. 1911?{]}}|)be}\mylabel{L04005h}
\begin{anhang}
\end{anhang}\newcommand{\dateiname}{L04005}\newcommand{\titel}{Berta Zuckerkandl an Arthur Schnitzler, [9. 10. 1911?]}\newcommand{\editorInnen}{Herausgegeben von Jahnke, SelmaMüller, Martin Anton}%% latex-leseansicht-abspann.tex
%% Abspann für die Leseansicht.
%% Der Schalter \ifkorrekturansicht ist bereits durch den Vorspann gesetzt.

%% latex-abspann.tex
%% Gemeinsamer Abspann für Korrekturansicht und Leseansicht.
%% Setzt den Schalter \ifkorrekturansicht voraus (gesetzt in den
%% einbindenden Dateien latex-korrekturansicht-abspann.tex bzw.
%% latex-leseansicht-abspann.tex).
%% ---------------------------------------------------------------

\normalsize

% Das esempio-Environment wird nur in der Leseansicht benötigt
\ifkorrekturansicht\else
\newenvironment{esempio}[3]%
{
    \vspace{1.5ex}
    \rlap{\underline{#1}}
    \par
    \setlength{\parindent}{0cm}
    \nopagebreak
    \leftskip=#2cm
    \rightskip=#3cm
}
{
    \par
}
\fi

\doendnotes{C}
\bigskip
\vfill

\clearpage

\footnotesize

\ifkorrekturansicht
  \lohead{\textsc{register}}
\fi

% theindex-Environment neu definieren ohne reledmac
\makeatletter
\renewenvironment{theindex}{%
  \ifkorrekturansicht
    \section*{\indexname}%
  \else
    \subsubsection*{Index der erwähnten Entitäten}%
  \fi
  \setlength{\parindent}{0pt}%
  \setlength{\parskip}{0pt plus 0.3pt}%
  \let\item\@idxitem
}{%
  \ifkorrekturansicht\clearpage\fi
}
\makeatother

\IfFileExists{\jobname-pw.ind}{\input{\jobname-pw.ind}}{}

% Quellenangabe nur in der Leseansicht
\ifkorrekturansicht\else
% Fallback-Definitionen, falls die .tex-Datei \titel etc. nicht gesetzt hat
\providecommand{\titel}{}
\providecommand{\editorInnen}{}
\providecommand{\dateiname}{\jobname}

\vspace{3cm}

\vfill

\footnotesize
\textsc{Quelle}: \titel. Herausgegeben von {\editorInnen}. In: \emph{Arthur Schnitzler: Briefwechsel mit Autorinnen und Autoren}.
 Digitale Edition, https://schnitzler-briefe.acdh.oeaw.ac.at/{\dateiname}.html (Stand \today)
\fi

\end{document}


