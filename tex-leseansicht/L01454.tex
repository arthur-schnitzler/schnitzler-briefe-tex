%% latex-leseansicht-vorspann.tex
%% Vorspann für die Leseansicht.
%% Lädt die gemeinsame Datei latex-vorspann.tex mit nicht gesetztem Schalter.

\newif\ifkorrekturansicht
\korrekturansichtfalse

\input{../tex-inputs/latex-vorspann}


               \section[Hugo von Hofmannsthal an Arthur Schnitzler, 14. 10. 1904]{ Hugo von Hofmannsthal an Arthur Schnitzler, 14. 10. 1904}\nopagebreak\mylabel{v}\rehead{ }\begin{ledgroupsized}[t]{13cm}\normalsize\beginnumbering\briefempfaengerindex{Schnitzler, Arthur@\textsc{Schnitzler, Arthur}!zzzHofmannsthal, Hugo von@\emph{von Hugo von Hofmannsthal}!1904-10-141@{14. 10. 1904}|(be} \toendnotes[C]{\smallbreak\pagebreak[2]} \Standort{CUL, Schnitzler, B 43.}
\physDesc{Brief, 1 Blatt, 3 Seiten
\newline{}Handschrift Gertrude von Hofmannsthal: schwarze Tinte, lateinische Kurrent
\newline{}Schnitzler: mit Bleistift beschriftet: »\textsc{Hugo}« \newline{}Ordnung: 1) mit Bleistift von unbekannter Hand nummeriert: »\strikeout{229}« 2) mit Bleistift von unbekannter Hand nummeriert:
                                    »238«}\buchAbdrucke{\weitereDrucke{Hugo von Hofmannsthal, Arthur Schnitzler: \emph{Briefwechsel}. Hg. Therese Nickl und Heinrich Schnitzler. Frankfurt am Main: \emph{S. Fischer} 1964, S. 204.} }\toendnotes[C]{\smallbreak}\pstart
           {\pb}(dictiert)\hfill Rodaun\oindex{Rodaun@\textbf{Rodaun}|pw}, d. 14. X. 1904.\pend
           \pstart
           Mein lieber Arthur, ich muss Sie bitten den inliegenden leider sehr
               unleserlichen \label{K_L01454_1v}\edtext{Brief}{\lemma{\textnormal{\emph{Brief}}}\Cendnote{\textnormal{Dieser befindet sich heute im Nachlass Hofmannsthal\pwindex{Hofmannsthal, Hugo von 01.02.1874 – 15.07.1929@\textsc{Hofmannsthal, Hugo von} (01.02.1874 – 15.07.1929), \emph{Schriftsteller}|pwk}s (Hs-30605,8). Der
                  Brief abgedruckt in: \emph{Louise Dumont. Eine Kulturgeschichte in Briefen und
                        Dokumenten.} Bd. 1: 1879–1904. Hg. Gertrude Cepl-Kaufmann, Michael
                     Matzigkeit Winrich Meiszies. Bearbeitet von Jasmin Grande, Nina Heidrich,
                     Karoline Riener. Essen: \emph{Klartext}{ }2013, S. 354–356.}}}\label{K_L01454_1h} der Dumont\pwindex{Dumont, Louise 22.02.1862 – 16.05.1932@\textsc{Dumont, Louise} (22.02.1862 – 16.05.1932), \emph{Theaterleiterin, Schauspielerin}|pw} zu lesen und mir über diese Sache umgehend Ihren Rat zu
               geben. Es ist gewissermassen eine gemeinsame Angelegenheit. Die Unternehmung Dumont\pwindex{Dumont, Louise 22.02.1862 – 16.05.1932@\textsc{Dumont, Louise} (22.02.1862 – 16.05.1932), \emph{Theaterleiterin, Schauspielerin}|pw}–Lindemann\pwindex{Lindemann, Gustav 24.08.1872 – 06.05.1960@\textsc{Lindemann, Gustav} (24.08.1872 – 06.05.1960), \emph{Theaterleiter, Schauspieler}|pw} bewirbt sich um fast sämmtliche meiner dram. Arbeiten, was für
               mich immerhin nicht unwichtig. Nun war ich {\pb}durch S. Fischer\pwindex{Fischer, Samuel 24.12.1859 – 15.10.1934@\textsc{Fischer, Samuel} (24.12.1859 – 15.10.1934), \emph{Verleger}|pw} davon unterrichtet, dass sich die gleiche
               Unternehmung gegen Sie (Einsamer Weg\pwindex{Schnitzler, Arthur 15.05.1862 – 21.10.1931@\textsc{Schnitzler, Arthur} (15.05.1862 – 21.10.1931), \emph{Schriftsteller, Mediziner}!einsame Weg. Schauspiel in fuenf Akten1904@\strich\emph{Der einsame Weg. Schauspiel in fünf Akten} {[}1904{]}|pw}) uncorrect
               oder direct unanständig benommen habe. Ich that daher das Selbstverständliche d. h.
               ich verweigerte meine sämmtlichen Stücke »bis ich erfahren hätte, dass diese
               Angelegenheit zu Ihrer Befriedigung beigelegt sei«. Nun stellt der inliegende {\pb}Brief der Dumont\pwindex{Dumont, Louise 22.02.1862 – 16.05.1932@\textsc{Dumont, Louise} (22.02.1862 – 16.05.1932), \emph{Theaterleiterin, Schauspielerin}|pw} die Sache ganz anders dar und ich bitte daher Sie mir
               mit zwei Worten zu sagen wo die Wahrheit liegt und ob vielleicht wirklich eine
               Ungeschicklichkeit Fischers\pwindex{Fischer, Samuel 24.12.1859 – 15.10.1934@\textsc{Fischer, Samuel} (24.12.1859 – 15.10.1934), \emph{Verleger}|pw} die Sache auf diesen
               bösen Punkt getrieben hat, in welchem Falle ich mich natürlich nicht verpflichtet
               hielte die Stücke zu verweigern.\pend
           \pstart
           Herzlich{\\[\baselineskip]}Ihr{\\[\baselineskip]}\spacefill\mbox{Hugo.}\pend
           \leftskip=0em{}\endnumbering\briefempfaengerindex{Schnitzler, Arthur@\textsc{Schnitzler, Arthur}!zzzHofmannsthal, Hugo von@\emph{von Hugo von Hofmannsthal}!1904-10-141@{14. 10. 1904}|)be}\mylabel{h}\end{ledgroupsized}  \newcommand{\dateiname}{L01454}\newcommand{\titel}{Hugo von Hofmannsthal an Arthur Schnitzler, 14. 10. 1904}\newcommand{\editorInnen}{Martin Anton Müller und Gerd-Hermann Susen}%% latex-leseansicht-abspann.tex
%% Abspann für die Leseansicht.
%% Der Schalter \ifkorrekturansicht ist bereits durch den Vorspann gesetzt.

%% latex-abspann.tex
%% Gemeinsamer Abspann für Korrekturansicht und Leseansicht.
%% Setzt den Schalter \ifkorrekturansicht voraus (gesetzt in den
%% einbindenden Dateien latex-korrekturansicht-abspann.tex bzw.
%% latex-leseansicht-abspann.tex).
%% ---------------------------------------------------------------

\normalsize

% Das esempio-Environment wird nur in der Leseansicht benötigt
\ifkorrekturansicht\else
\newenvironment{esempio}[3]%
{
    \vspace{1.5ex}
    \rlap{\underline{#1}}
    \par
    \setlength{\parindent}{0cm}
    \nopagebreak
    \leftskip=#2cm
    \rightskip=#3cm
}
{
    \par
}
\fi

\doendnotes{C}
\bigskip
\vfill

\clearpage

\footnotesize

\ifkorrekturansicht
  \lohead{\textsc{register}}
\fi

% theindex-Environment neu definieren ohne reledmac
\makeatletter
\renewenvironment{theindex}{%
  \ifkorrekturansicht
    \section*{\indexname}%
  \else
    \subsubsection*{Index der erwähnten Entitäten}%
  \fi
  \setlength{\parindent}{0pt}%
  \setlength{\parskip}{0pt plus 0.3pt}%
  \let\item\@idxitem
}{%
  \ifkorrekturansicht\clearpage\fi
}
\makeatother

\IfFileExists{\jobname-pw.ind}{\input{\jobname-pw.ind}}{}

% Quellenangabe nur in der Leseansicht
\ifkorrekturansicht\else
% Fallback-Definitionen, falls die .tex-Datei \titel etc. nicht gesetzt hat
\providecommand{\titel}{}
\providecommand{\editorInnen}{}
\providecommand{\dateiname}{\jobname}

\vspace{3cm}

\vfill

\footnotesize
\textsc{Quelle}: \titel. Herausgegeben von {\editorInnen}. In: \emph{Arthur Schnitzler: Briefwechsel mit Autorinnen und Autoren}.
 Digitale Edition, https://schnitzler-briefe.acdh.oeaw.ac.at/{\dateiname}.html (Stand \today)
\fi

\end{document}


      