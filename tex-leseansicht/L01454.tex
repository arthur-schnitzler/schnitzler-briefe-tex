%% latex-korrekturansicht-vorspann.tex
%% Vorspann für die Korrekturansicht.
%% Lädt die gemeinsame Datei latex-vorspann.tex mit gesetztem Schalter.

\newif\ifkorrekturansicht
\korrekturansichttrue

\input{../tex-inputs/latex-vorspann}


\section[Hugo von Hofmannsthal an Arthur Schnitzler, 14. 10. 1904]{L01454 Hugo von Hofmannsthal an Arthur Schnitzler, 14. 10. 1904}
\nopagebreak\mylabel{L01454v}
\rehead{ }\normalsize\beginnumbering\briefempfaengerindex{Schnitzler, Arthur@\textsc{Schnitzler, Arthur}!zzzHofmannsthal, Hugo von@\emph{von Hugo von Hofmannsthal}!1904-10-141@{14. 10. 1904}|(be}
\toendnotes[C]{\smallbreak\pagebreak[2]}\Standort{CUL, Schnitzler, B 43.}
\physDesc{Brief, 1 Blatt, 3 Seiten, 1047 Zeichen
\newline{}Handschrift Gertrude von Hofmannsthal: schwarze Tinte, lateinische Kurrent
\newline{}Schnitzler: mit Bleistift beschriftet: »\textsc{Hugo}« 
\newline{}Ordnung: 1) mit Bleistift von unbekannter Hand nummeriert: »\strikeout{229}«  2) mit Bleistift von unbekannter Hand nummeriert:
                                    »238«}
\buchAbdrucke{\weitereDrucke{Hugo von Hofmannsthal, Arthur Schnitzler: \emph{Briefwechsel}. Frankfurt am Main: \emph{S. Fischer} 1964, S. 204.} }\toendnotes[C]{\smallbreak}
\pstart
           
\pstart
           {\pb}(dictiert)\pend
           
\pstart
           \raggedleft{}Rodaun\oindex{Rodaun@\textbf{Rodaun}, \emph{A.ADM4}|pw}, d. 14. X. 1904.\pend
           \pend
           \vspace{0.5em}
\pstart
           Mein lieber Arthur, ich muss Sie bitten den inliegenden leider sehr
               unleserlichen \label{K_L01454-1v}\edtext{Brief}{\lemma{\textnormal{\emph{Brief}}}\Cendnote{\textnormal{Dieser befindet sich heute im Nachlass Hofmannsthals\pwindex{Hofmannsthal, Hugo von 1874-02-01 – 1929-07-15@\textsc{Hofmannsthal, Hugo von} (1874-02-01 – 1929-07-15), \emph{Schriftsteller/Schriftstellerin}|pwk} (Hs-30605,8). Der
                  Brief wurde abgedruckt in: \emph{Louise Dumont. Eine Kulturgeschichte in Briefen und
                        Dokumenten. Bd. 1: 1879–1904}. Herausgegeben von  Gertrude Cepl-Kaufmann, Michael
                     Matzigkeit, Winrich Meiszies. Bearbeitet von Jasmin Grande, Nina Heidrich,
                     Karoline Riener. Essen: \emph{Klartext}{ }2013, S. 354–356.}}}\label{K_L01454-1} der Dumont\pwindex{Dumont, Louise 22.02.1862 – 16.05.1932@\textsc{Dumont, Louise} (22.02.1862 – 16.05.1932), \emph{Theaterleiter/Theaterleiterin, Schauspieler/Schauspielerin}|pw} zu lesen und mir über diese Sache umgehend Ihren Rat zu
               geben. Es ist gewissermassen eine gemeinsame Angelegenheit. Die Unternehmung Dumont\pwindex{Dumont, Louise 22.02.1862 – 16.05.1932@\textsc{Dumont, Louise} (22.02.1862 – 16.05.1932), \emph{Theaterleiter/Theaterleiterin, Schauspieler/Schauspielerin}|pw}–Lindemann\pwindex{Lindemann, Gustav 24.08.1872 – 06.05.1960@\textsc{Lindemann, Gustav} (24.08.1872 – 06.05.1960), \emph{Theaterleiter/Theaterleiterin, Schauspieler/Schauspielerin}|pw} bewirbt sich um fast sämmtliche meiner dram. Arbeiten, was für
               mich immerhin nicht unwichtig. Nun war ich {\pb}durch S. Fischer\pwindex{Fischer, Samuel 24.12.1859 – 15.10.1934@\textsc{Fischer, Samuel} (24.12.1859 – 15.10.1934), \emph{Verleger/Verlegerin}|pw} davon unterrichtet, dass sich die gleiche
               Unternehmung gegen Sie (Einsamer Weg\pwindex{einsame Weg. Schauspiel in fuenf Akten@\emph{Der einsame Weg. Schauspiel in fünf Akten}|pw}) uncorrect
               oder direct unanständig benommen habe. Ich that daher das Selbstverständliche d. h.
               ich verweigerte meine sämmtlichen Stücke »bis ich erfahren hätte, dass diese
               Angelegenheit zu Ihrer Befriedigung beigelegt sei«. Nun stellt der inliegende {\pb}Brief der Dumont\pwindex{Dumont, Louise 22.02.1862 – 16.05.1932@\textsc{Dumont, Louise} (22.02.1862 – 16.05.1932), \emph{Theaterleiter/Theaterleiterin, Schauspieler/Schauspielerin}|pw} die Sache ganz anders dar und ich bitte daher Sie mir
               mit zwei Worten zu sagen wo die Wahrheit liegt und ob vielleicht wirklich eine
               Ungeschicklichkeit Fischers\pwindex{Fischer, Samuel 24.12.1859 – 15.10.1934@\textsc{Fischer, Samuel} (24.12.1859 – 15.10.1934), \emph{Verleger/Verlegerin}|pw} die Sache auf
               diesen bösen Punkt getrieben hat, in welchem Falle ich mich natürlich nicht
               verpflichtet hielte die Stücke zu verweigern.\pend
           
\pstart
           Herzlich{\\[\baselineskip]}Ihr{\\[\baselineskip]}\spacefill\mbox{Hugo.}\pend
           \leftskip=0em{}\selectlanguage{ngerman}\endnumbering\briefempfaengerindex{Schnitzler, Arthur@\textsc{Schnitzler, Arthur}!zzzHofmannsthal, Hugo von@\emph{von Hugo von Hofmannsthal}!1904-10-141@{14. 10. 1904}|)be}\mylabel{L01454h}  \normalsize

\doendnotes{C}
\bigskip
\vfill

\clearpage

\footnotesize

\lohead{\textsc{register}}

% Definiere theindex-Environment komplett neu ohne reledmac
\makeatletter
\renewenvironment{theindex}{%
  \section*{\indexname}%
  \setlength{\parindent}{0pt}%
  \setlength{\parskip}{0pt plus 0.3pt}%
  \let\item\@idxitem
}{%
  \clearpage
}
\makeatother

\IfFileExists{\jobname-pw.ind}{\input{\jobname-pw.ind}}{}

\end{document}

      