%% latex-leseansicht-vorspann.tex
%% Vorspann für die Leseansicht.
%% Lädt die gemeinsame Datei latex-vorspann.tex mit nicht gesetztem Schalter.

\newif\ifkorrekturansicht
\korrekturansichtfalse

\input{../tex-inputs/latex-vorspann}


\section[Hugo von Hofmannsthal an Arthur Schnitzler, 14. 10. 1904]{L01454 Hugo von Hofmannsthal an Arthur Schnitzler, 14. 10. 1904}
\nopagebreak\mylabel{L01454v}
\rehead{ }\normalsize\beginnumbering\briefempfaengerindex{Schnitzler, Arthur@\textsc{Schnitzler, Arthur}!zzzHofmannsthal, Hugo von@\emph{von Hugo von Hofmannsthal}!1904-10-141@{14. 10. 1904}|(be}
\toendnotes[C]{\smallbreak\pagebreak[2]}
\correspDesc{Versand  durch Hugo von Hofmannsthal am 14. 10. 1904 in Rodaun
\newline{}Erhalt  durch Arthur Schnitzler im Zeitraum [15. 10. 1904 – 19. 10. 1904?] in Wien}\toendnotes[C]{\smallbreak}
\Standort{CUL, Schnitzler, B 43.}
\physDesc{Brief, 1 Blatt, 3 Seiten, 1047 Zeichen
\newline{}HandschriftX2  : schwarze Tinte, lateinische Kurrent
\newline{}Schnitzler: mit Bleistift beschriftet: »\textsc{Hugo}« 
\newline{}Ordnung: 1) mit Bleistift von unbekannter Hand nummeriert: »\strikeout{229}«  2) mit Bleistift von unbekannter Hand nummeriert:
                                    »238«}
\buchAbdrucke{\weitereDrucke{Hugo von Hofmannsthal, Arthur Schnitzler: \emph{Briefwechsel}. Herausgegeben von Therese Nickl und Heinrich Schnitzler. Frankfurt am Main: \emph{S. Fischer} 1964, S. 204.} }\toendnotes[C]{\smallbreak}
\pstart
           
\pstart
           {\pb}(dictiert)\pend
           
\pstart
           \raggedleft{}Rodaun\oindex{Wien@\textbf{Wien}!XXIII., Liesing@\textbf{XXIII., Liesing}!Rodaun@\textbf{Rodaun}, \emph{Region}|pw}, d. 14. X. 1904.\pend
           \pend
           \vspace{0.5em}
\pstart
           Mein lieber Arthur, ich muss Sie bitten den inliegenden leider sehr
               unleserlichen \label{K_L01454-1v}\edtext{Brief}{\lemma{\textnormal{\emph{Brief}}}\Cendnote{\textnormal{Dieser befindet sich heute im Nachlass Hofmannsthals\pwindex{Hofmannsthal, Hugo von 1.\,2.\,1874 Wien – 15.\,7.\,1929 Rodaun@\textsc{Hofmannsthal, Hugo von} (1.\,2.\,1874 Wien – 15.\,7.\,1929 Rodaun), \emph{Schriftsteller}|pwk} (Hs-30605,8). Der
                  Brief wurde abgedruckt in: \emph{Louise Dumont. Eine Kulturgeschichte in Briefen und
                        Dokumenten. Bd. 1: 1879–1904}. Herausgegeben von  Gertrude Cepl-Kaufmann, Michael
                     Matzigkeit, Winrich Meiszies. Bearbeitet von Jasmin Grande, Nina Heidrich,
                     Karoline Riener. Essen: \emph{Klartext}{ }2013, S. 354–356.}}}\label{K_L01454-1} der Dumont\pwindex{Dumont, Louise 22.\,2.\,1862 Köln – 16.\,5.\,1932 Düsseldorf@\textsc{Dumont, Louise} (22.\,2.\,1862 Köln – 16.\,5.\,1932 Düsseldorf), \emph{Theaterleiterin, Schauspielerin}|pw} zu lesen und mir über diese Sache umgehend Ihren Rat zu
               geben. Es ist gewissermassen eine gemeinsame Angelegenheit. Die Unternehmung Dumont\pwindex{Dumont, Louise 22.\,2.\,1862 Köln – 16.\,5.\,1932 Düsseldorf@\textsc{Dumont, Louise} (22.\,2.\,1862 Köln – 16.\,5.\,1932 Düsseldorf), \emph{Theaterleiterin, Schauspielerin}|pw}–Lindemann\pwindex{Lindemann, Gustav 24.\,8.\,1872 Danzig – 6.\,5.\,1960 Sonnenholz@\textsc{Lindemann, Gustav} (24.\,8.\,1872 Danzig – 6.\,5.\,1960 Sonnenholz), \emph{Theaterleiter, Schauspieler}|pw} bewirbt sich um fast sämmtliche meiner dram. Arbeiten, was für
               mich immerhin nicht unwichtig. Nun war ich {\pb}durch S. Fischer\pwindex{Fischer, Samuel 24.\,12.\,1859 Liptovský Mikuláš – 15.\,10.\,1934 Berlin@\textsc{Fischer, Samuel} (24.\,12.\,1859 Liptovský Mikuláš – 15.\,10.\,1934 Berlin), \emph{Verleger}|pw} davon unterrichtet, dass sich die gleiche
               Unternehmung gegen Sie (Einsamer Weg\pwindex{Schnitzler, Arthur 15.\,5.\,1862 Wien – 21.\,10.\,1931 ebd.@\textsc{Schnitzler, Arthur} (15.\,5.\,1862 Wien – 21.\,10.\,1931 ebd.), \emph{Schriftsteller, Mediziner}!einsame Weg. Schauspiel in fünf Akten@\strich\emph{Der einsame Weg. Schauspiel in fünf Akten}|pw}) uncorrect
               oder direct unanständig benommen habe. Ich that daher das Selbstverständliche d. h.
               ich verweigerte meine sämmtlichen Stücke »bis ich erfahren hätte, dass diese
               Angelegenheit zu Ihrer Befriedigung beigelegt sei«. Nun stellt der inliegende {\pb}Brief der Dumont\pwindex{Dumont, Louise 22.\,2.\,1862 Köln – 16.\,5.\,1932 Düsseldorf@\textsc{Dumont, Louise} (22.\,2.\,1862 Köln – 16.\,5.\,1932 Düsseldorf), \emph{Theaterleiterin, Schauspielerin}|pw} die Sache ganz anders dar und ich bitte daher Sie mir
               mit zwei Worten zu sagen wo die Wahrheit liegt und ob vielleicht wirklich eine
               Ungeschicklichkeit Fischers\pwindex{Fischer, Samuel 24.\,12.\,1859 Liptovský Mikuláš – 15.\,10.\,1934 Berlin@\textsc{Fischer, Samuel} (24.\,12.\,1859 Liptovský Mikuláš – 15.\,10.\,1934 Berlin), \emph{Verleger}|pw} die Sache auf
               diesen bösen Punkt getrieben hat, in welchem Falle ich mich natürlich nicht
               verpflichtet hielte die Stücke zu verweigern.\pend
           
\pstart
           Herzlich{\\[\baselineskip]}Ihr{\\[\baselineskip]}\spacefill\mbox{Hugo.}\pend
           \leftskip=0em{}\selectlanguage{ngerman}\endnumbering\briefempfaengerindex{Schnitzler, Arthur@\textsc{Schnitzler, Arthur}!zzzHofmannsthal, Hugo von@\emph{von Hugo von Hofmannsthal}!1904-10-141@{14. 10. 1904}|)be}\mylabel{L01454h}  \newcommand{\dateiname}{L01454}\newcommand{\titel}{Hugo von Hofmannsthal an Arthur Schnitzler, 14. 10. 1904}\newcommand{\editorInnen}{Martin Anton Müller und Gerd-Hermann Susen}%% latex-leseansicht-abspann.tex
%% Abspann für die Leseansicht.
%% Der Schalter \ifkorrekturansicht ist bereits durch den Vorspann gesetzt.

%% latex-abspann.tex
%% Gemeinsamer Abspann für Korrekturansicht und Leseansicht.
%% Setzt den Schalter \ifkorrekturansicht voraus (gesetzt in den
%% einbindenden Dateien latex-korrekturansicht-abspann.tex bzw.
%% latex-leseansicht-abspann.tex).
%% ---------------------------------------------------------------

\normalsize

% Das esempio-Environment wird nur in der Leseansicht benötigt
\ifkorrekturansicht\else
\newenvironment{esempio}[3]%
{
    \vspace{1.5ex}
    \rlap{\underline{#1}}
    \par
    \setlength{\parindent}{0cm}
    \nopagebreak
    \leftskip=#2cm
    \rightskip=#3cm
}
{
    \par
}
\fi

\doendnotes{C}
\bigskip
\vfill

\clearpage

\footnotesize

\ifkorrekturansicht
  \lohead{\textsc{register}}
\fi

% theindex-Environment neu definieren ohne reledmac
\makeatletter
\renewenvironment{theindex}{%
  \ifkorrekturansicht
    \section*{\indexname}%
  \else
    \subsubsection*{Index der erwähnten Entitäten}%
  \fi
  \setlength{\parindent}{0pt}%
  \setlength{\parskip}{0pt plus 0.3pt}%
  \let\item\@idxitem
}{%
  \ifkorrekturansicht\clearpage\fi
}
\makeatother

\IfFileExists{\jobname-pw.ind}{\input{\jobname-pw.ind}}{}

% Quellenangabe nur in der Leseansicht
\ifkorrekturansicht\else
% Fallback-Definitionen, falls die .tex-Datei \titel etc. nicht gesetzt hat
\providecommand{\titel}{}
\providecommand{\editorInnen}{}
\providecommand{\dateiname}{\jobname}

\vspace{3cm}

\vfill

\footnotesize
\textsc{Quelle}: \titel. Herausgegeben von {\editorInnen}. In: \emph{Arthur Schnitzler: Briefwechsel mit Autorinnen und Autoren}.
 Digitale Edition, https://schnitzler-briefe.acdh.oeaw.ac.at/{\dateiname}.html (Stand \today)
\fi

\end{document}


