%% latex-leseansicht-vorspann.tex
%% Vorspann für die Leseansicht.
%% Lädt die gemeinsame Datei latex-vorspann.tex mit nicht gesetztem Schalter.

\newif\ifkorrekturansicht
\korrekturansichtfalse

\input{../tex-inputs/latex-vorspann}


         \renewcommand{\erwaehnteInstitutionen}{Institutionen: Propyläen Verlag}
         \renewcommand{\erwaehnteOrte}{Orte: Wien}
         \renewcommand{\erwaehnteWerke}{Werke: Die Frau des Richters. Novelle}
               \section[Felix Braun an Arthur Schnitzler, 14. 12. 1925]{ Felix Braun an Arthur Schnitzler, 14. 12. 1925}\nopagebreak\mylabel{v}\rehead{ }\begin{ledgroupsized}[t]{13cm}\normalsize\beginnumbering \toendnotes[C]{\smallbreak\pagebreak[2]} \Standort{DLA, A:Schnitzler, HS.NZ85.1.2604,6.}
\physDesc{Briefkarte
\newline{}Handschrift: schwarze Tinte, deutsche Kurrent
\newline{}Schnitzler: 1) mit Bleistift beschriftet: »\textsc{Braun}«  2) mit rotem Buntstift zwei Unterstreichungen}\toendnotes[C]{\smallbreak}\pstart
           \centering{}{\pb}Wien\oindex{Wien@\textbf{Wien}|pw}, den 14. XII. 25\pend
           \pstart{}Verehrter Herr Doktor!\pend\pstart
           Haben Sie den herzlichſten Dank für die Überſendung Ihres Buchs »Die Frau des Richters\pwindex{Schnitzler, Arthur 15.05.1862 – 21.10.1931@\textsc{Schnitzler, Arthur} (15.05.1862 – 21.10.1931), \emph{Schriftsteller, Mediziner}!Frau des Richters. Novelle7.8.1925 – 15.8.1925@\strich\emph{Die Frau des Richters. Novelle} {[}7.8.1925 – 15.8.1925{]}|pw}« durch den Propyläen-Verlag\orgindex{Propylaeen Verlag@Propyläen Verlag|pw}. War ſchon der Empfang durch das
                    Bewußtſein, daß Sie ſelbſt, verehrter Herr Doktor, der Auftraggeber geweſen
                    ſind, eine große Freude, ſo auch die Lektüre. Denn ein meiſterliches Werk iſt
                    Ihnen da wieder und makellos geglückt. Sowohl die herrliche Proſa als auch die
                    Geſtaltung der Charaktere kann nur mit dem Prädikat der Meiſter{\pb}ſchaft gerühmt werden. Solange ſolche Bewältigungen möglich ſind, kann von
                    einem Abſtieg unſerer Zeit und Kunſt die Rede nicht ſein.\pend
           \pstart
           Immer war das Menſchliche – in einem weiteren als nur dem ethiſchen Sinn genommen
                    – Ihnen zu dichten gegeben: auch hier, am ſchönſten in der Geſtalt der Frau\pwindex{Schnitzler, Arthur 15.05.1862 – 21.10.1931@\textsc{Schnitzler, Arthur} (15.05.1862 – 21.10.1931), \emph{Schriftsteller, Mediziner}!Frau des Richters. Novelle7.8.1925 – 15.8.1925@\strich\emph{Die Frau des Richters. Novelle} {[}7.8.1925 – 15.8.1925{]}|pwv}, und frei und leicht in
                    der des Rebellen\pwindex{Schnitzler, Arthur 15.05.1862 – 21.10.1931@\textsc{Schnitzler, Arthur} (15.05.1862 – 21.10.1931), \emph{Schriftsteller, Mediziner}!Frau des Richters. Novelle7.8.1925 – 15.8.1925@\strich\emph{Die Frau des Richters. Novelle} {[}7.8.1925 – 15.8.1925{]}|pwv}, iſt es
                    Ihnen geglückt.\hspace*{1.5em}In Verehrung grüße ich Sie,
                    werter Herr Doktor, und ſage nochmals wärmſten Dank.\pend
           \pstart
           Ihr ergebener{\\[\baselineskip]}\spacefill\mbox{Felix Braun.}\pend
           \leftskip=0em{}
         
         \endnumbering\mylabel{h}\end{ledgroupsized}  \newcommand{\dateiname}{L02458}\newcommand{\titel}{Felix Braun an Arthur Schnitzler, 14. 12. 1925}\newcommand{\editorInnen}{Martin Anton Müller und Gerd-Hermann Susen}%% latex-leseansicht-abspann.tex
%% Abspann für die Leseansicht.
%% Der Schalter \ifkorrekturansicht ist bereits durch den Vorspann gesetzt.

%% latex-abspann.tex
%% Gemeinsamer Abspann für Korrekturansicht und Leseansicht.
%% Setzt den Schalter \ifkorrekturansicht voraus (gesetzt in den
%% einbindenden Dateien latex-korrekturansicht-abspann.tex bzw.
%% latex-leseansicht-abspann.tex).
%% ---------------------------------------------------------------

\normalsize

% Das esempio-Environment wird nur in der Leseansicht benötigt
\ifkorrekturansicht\else
\newenvironment{esempio}[3]%
{
    \vspace{1.5ex}
    \rlap{\underline{#1}}
    \par
    \setlength{\parindent}{0cm}
    \nopagebreak
    \leftskip=#2cm
    \rightskip=#3cm
}
{
    \par
}
\fi

\doendnotes{C}
\bigskip
\vfill

\clearpage

\footnotesize

\ifkorrekturansicht
  \lohead{\textsc{register}}
\fi

% theindex-Environment neu definieren ohne reledmac
\makeatletter
\renewenvironment{theindex}{%
  \ifkorrekturansicht
    \section*{\indexname}%
  \else
    \subsubsection*{Index der erwähnten Entitäten}%
  \fi
  \setlength{\parindent}{0pt}%
  \setlength{\parskip}{0pt plus 0.3pt}%
  \let\item\@idxitem
}{%
  \ifkorrekturansicht\clearpage\fi
}
\makeatother

\IfFileExists{\jobname-pw.ind}{\input{\jobname-pw.ind}}{}

% Quellenangabe nur in der Leseansicht
\ifkorrekturansicht\else
% Fallback-Definitionen, falls die .tex-Datei \titel etc. nicht gesetzt hat
\providecommand{\titel}{}
\providecommand{\editorInnen}{}
\providecommand{\dateiname}{\jobname}

\vspace{3cm}

\vfill

\footnotesize
\textsc{Quelle}: \titel. Herausgegeben von {\editorInnen}. In: \emph{Arthur Schnitzler: Briefwechsel mit Autorinnen und Autoren}.
 Digitale Edition, https://schnitzler-briefe.acdh.oeaw.ac.at/{\dateiname}.html (Stand \today)
\fi

\end{document}


      