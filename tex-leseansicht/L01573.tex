%% latex-leseansicht-vorspann.tex
%% Vorspann für die Leseansicht.
%% Lädt die gemeinsame Datei latex-vorspann.tex mit nicht gesetztem Schalter.

\newif\ifkorrekturansicht
\korrekturansichtfalse

\input{../tex-inputs/latex-vorspann}

\begin{center}
            \textcolor{red}{ENTWURF. ENTZIFFERUNG NOCH NICHT KORREKTURGELESEN}
                      \end{center}
            
               \section[Arthur Schnitzler an Gerhart Hauptmann, {[}17. 1. 1906{]}]{ Arthur Schnitzler an Gerhart Hauptmann,
                    {[}17. 1. 1906{]}}\nopagebreak\mylabel{v}\rehead{ }\begin{ledgroupsized}[t]{13cm}\normalsize\beginnumbering\briefempfaengerindex{Hauptmann, Gerhart@\textsc{Hauptmann, Gerhart}!zzzSchnitzler, Arthur@\emph{von Arthur Schnitzler}!1906-01-171@{{[}17. 1. 1906{]}}|(be} \toendnotes[C]{\smallbreak\pagebreak[2]} \Standort{Staatsbibliothek Berlin – Preußischer Kulturbesitz, GHBrBl A:Schnitzler (1).}
\physDesc{Briefkarte
\newline{}Handschrift: schwarze Tinte, deutsche Kurrent\newline{}Ordnung: mit Bleistift von unbekannter Hand nummeriert: »1« }\toendnotes[C]{\smallbreak}\pstart
           \noindent{}{\pb}\textcolor{gray}{\textbf{Dr. Arthur
                            Schnitzler}}{\\}\textcolor{gray}{\textbf{Wien XVIII. Spoettelgasse 7\oindex{Edmund-Weiss-Gasse@\textbf{Edmund-Weiß-Gasse}|pw}.}}\pend
           \pstart
           lieber Herr Hauptmann, der Überbringer\pwindex{Knapitsch, Siegfried 20.10.1883 – 16.05.1962@\textsc{Knapitsch, Siegfried} (20.10.1883 – 16.05.1962), \emph{Schriftsteller/Schriftstellerin, Journalist/Journalistin, Rechtsanwalt/Rechtsanwältin}|pw} dieſer \label{K_L01573_1v}\edtext{Karte}{\lemma{\textnormal{\emph{Karte}}}\Cendnote{\textnormal{Vgl. A. S.: \emph{Tagebuch}, 17. 1. 1906: Knapitsch\pwindex{Knapitsch, Siegfried 20.10.1883 – 16.05.1962@\textsc{Knapitsch, Siegfried} (20.10.1883 – 16.05.1962), \emph{Schriftsteller/Schriftstellerin, Journalist/Journalistin, Rechtsanwalt/Rechtsanwältin}|pwk} »behufs Unterzeichnung eines
                        Aufrufs zur Gründung einer deutschen ›société‹. – Ich gab ihm eine Karte an
                            Hauptmann\pwindex{Hauptmann, Gerhart 15.11.1862 – 06.06.1946@\textsc{Hauptmann, Gerhart} (15.11.1862 – 06.06.1946), \emph{Schriftsteller}|pwk}.«}}}\label{K_L01573_1h}, Herr \textsc{cand. jur. Siegfried
                            Knapitsch\pwindex{Knapitsch, Siegfried 20.10.1883 – 16.05.1962@\textsc{Knapitsch, Siegfried} (20.10.1883 – 16.05.1962), \emph{Schriftsteller/Schriftstellerin, Journalist/Journalistin, Rechtsanwalt/Rechtsanwältin}|pw}}, beabſichtigt, in Verbindung mit vielen andren
                    Bühnenautoren eine deutſche \strikeout{Bühnen} Geſellschaft
                    in der Art der »\textsc{societé des
                        auteurs dram\orgindex{Societe des Auteurs et Compositeurs Dramatiques@Société des Auteurs et Compositeurs Dramatiques|pw}. etc}{[}«{]} ins Leben zu rufen, eine {\pb}Idee, die gewiſs alle Förderung verdient.\pend
           \pstart
           Wollen Sie die große Güte haben, Herrn \textsc{Knapitsch\pwindex{Knapitsch, Siegfried 20.10.1883 – 16.05.1962@\textsc{Knapitsch, Siegfried} (20.10.1883 – 16.05.1962), \emph{Schriftsteller/Schriftstellerin, Journalist/Journalistin, Rechtsanwalt/Rechtsanwältin}|pw}} anzuhören?\pend
           \pstart
           Herzlich grüßend{\\[\baselineskip]}Ihr{\\[\baselineskip]}\spacefill\mbox{Arth Schnitzler}\pend
           \leftskip=0em{}\endnumbering\briefempfaengerindex{Hauptmann, Gerhart@\textsc{Hauptmann, Gerhart}!zzzSchnitzler, Arthur@\emph{von Arthur Schnitzler}!1906-01-171@{{[}17. 1. 1906{]}}|)be}\mylabel{h}\end{ledgroupsized}  \newcommand{\dateiname}{L01573}\newcommand{\titel}{Arthur Schnitzler an Gerhart Hauptmann, [17. 1. 1906]}\newcommand{\editorInnen}{ Martin Anton Müller und Gerd-Hermann Susen}%% latex-leseansicht-abspann.tex
%% Abspann für die Leseansicht.
%% Der Schalter \ifkorrekturansicht ist bereits durch den Vorspann gesetzt.

%% latex-abspann.tex
%% Gemeinsamer Abspann für Korrekturansicht und Leseansicht.
%% Setzt den Schalter \ifkorrekturansicht voraus (gesetzt in den
%% einbindenden Dateien latex-korrekturansicht-abspann.tex bzw.
%% latex-leseansicht-abspann.tex).
%% ---------------------------------------------------------------

\normalsize

% Das esempio-Environment wird nur in der Leseansicht benötigt
\ifkorrekturansicht\else
\newenvironment{esempio}[3]%
{
    \vspace{1.5ex}
    \rlap{\underline{#1}}
    \par
    \setlength{\parindent}{0cm}
    \nopagebreak
    \leftskip=#2cm
    \rightskip=#3cm
}
{
    \par
}
\fi

\doendnotes{C}
\bigskip
\vfill

\clearpage

\footnotesize

\ifkorrekturansicht
  \lohead{\textsc{register}}
\fi

% theindex-Environment neu definieren ohne reledmac
\makeatletter
\renewenvironment{theindex}{%
  \ifkorrekturansicht
    \section*{\indexname}%
  \else
    \subsubsection*{Index der erwähnten Entitäten}%
  \fi
  \setlength{\parindent}{0pt}%
  \setlength{\parskip}{0pt plus 0.3pt}%
  \let\item\@idxitem
}{%
  \ifkorrekturansicht\clearpage\fi
}
\makeatother

\IfFileExists{\jobname-pw.ind}{\input{\jobname-pw.ind}}{}

% Quellenangabe nur in der Leseansicht
\ifkorrekturansicht\else
% Fallback-Definitionen, falls die .tex-Datei \titel etc. nicht gesetzt hat
\providecommand{\titel}{}
\providecommand{\editorInnen}{}
\providecommand{\dateiname}{\jobname}

\vspace{3cm}

\vfill

\footnotesize
\textsc{Quelle}: \titel. Herausgegeben von {\editorInnen}. In: \emph{Arthur Schnitzler: Briefwechsel mit Autorinnen und Autoren}.
 Digitale Edition, https://schnitzler-briefe.acdh.oeaw.ac.at/{\dateiname}.html (Stand \today)
\fi

\end{document}


      