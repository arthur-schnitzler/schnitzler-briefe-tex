%% latex-leseansicht-vorspann.tex
%% Vorspann für die Leseansicht.
%% Lädt die gemeinsame Datei latex-vorspann.tex mit nicht gesetztem Schalter.

\newif\ifkorrekturansicht
\korrekturansichtfalse

\input{../tex-inputs/latex-vorspann}


\section[Arthur Schnitzler an Gustav Schwarzkopf, 19. 7. 1904]{L04026 Arthur Schnitzler an Gustav Schwarzkopf, 19. 7. 1904}
\nopagebreak\mylabel{L04026v}
\rehead{ }\normalsize\beginnumbering\briefempfaengerindex{Schwarzkopf, Gustav@\textsc{Schwarzkopf, Gustav}!zzzSchnitzler, Arthur@\emph{von Arthur Schnitzler}!1904-07-291@{29. 7. 1904}|(be}
\toendnotes[C]{\smallbreak\pagebreak[2]}
\correspDesc{Versand  durch Arthur Schnitzler am 29. 7. 1904 in Wien
\newline{}Übermittlung  am 30. 7. 1904 in Wien
\newline{}Erhalt  durch Gustav Schwarzkopf im Zeitraum [30. 7. 1904 – 3. 8. 1904?] in Sekirn}\toendnotes[C]{\smallbreak}
\Standort{CUL, Schnitzler, B 96.}
\physDesc{Kartenbrief, 804 Zeichen
\newline{}Handschrift: schwarze Tinte, deutsche Kurrent
\newline{}Versand: Stempel: »\nobreak{}\oindex{XVIII., Währing@\textbf{XVIII., Währing}, \emph{Verwaltungsgebiet}|pwk}\textcolor{gray}{18}/\textcolor{gray}{1}
                                       Wien, 30. 7. 04\nobreak{}«.  }\toendnotes[C]{\smallbreak}\pstart{}{\pb}Herrn Guſtav
                  Schwarzkopf\pend{}\pstart{}pr. Adr. \textsc{Robert
                        Hirschfeld\pwindex{Hirschfeld, Robert 17.\,9.\,1857 Žďár nad Sázavou – 2.\,4.\,1914 Salzburg@\textsc{Hirschfeld, Robert} (17.\,9.\,1857 Žďár nad Sázavou – 2.\,4.\,1914 Salzburg), \emph{Journalist, Musikkritiker}|pw}}\pend{}\pstart{}\textsc{Sekirn am
                        Wörtersee}\oindex{Sekirn@\textbf{Sekirn}|pw}\pend{}{\bigskip}\vspace{1em}
\pstart
           \raggedleft{}{\pb}Wien XVIII.\oindex{XVIII., Währing@\textbf{XVIII., Währing}, \emph{Verwaltungsgebiet}|pw}{ }29. 7. 904\pend
           \vspace{0.5em}
\pstart
           lieber Guſtav, ich nehme an Sie ſind noch in Sekirn\oindex{Sekirn@\textbf{Sekirn}|pw} u ſo \substVorne{}\textsuperscript{erreichen}\substDazwischen{}werden\substHinten{} ſowohl Sie als die freundlichen Mitunterzeichner\pwindex{Hirschfeld, Emilie 12.\,12.\,1868 Berlin – 3.\,4.\,1942 Wien@\textsc{Hirschfeld, Emilie} (12.\,12.\,1868 Berlin – 3.\,4.\,1942 Wien), \emph{Sängerin}|pwv}\pwindex{Hirschfeld, Robert 17.\,9.\,1857 Žďár nad Sázavou – 2.\,4.\,1914 Salzburg@\textsc{Hirschfeld, Robert} (17.\,9.\,1857 Žďár nad Sázavou – 2.\,4.\,1914 Salzburg), \emph{Journalist, Musikkritiker}|pwv}\pwindex{Chiavacci, Vincenz 15.\,6.\,1847 Wien – 2.\,2.\,1916 ebd.@\textsc{Chiavacci, Vincenz} (15.\,6.\,1847 Wien – 2.\,2.\,1916 ebd.), \emph{Schriftsteller, Journalist, Beamter}|pwv}\pwindex{Schönherr, Malvine 19.\,11.\,1867 Wien – 7.\,2.\,1956 ebd.@\textsc{Schönherr, Malvine} (19.\,11.\,1867 Wien – 7.\,2.\,1956 ebd.)|pwv} der \label{K_L04026-1v}\edtext{Anſichtskarte}{\lemma{\textnormal{\emph{Ansichtskarte}}}\Cendnote{\textnormal{{XXXX ref}XXXX Postkarte vom 22. 7. 1904}}}\label{K_L04026-1}{ }Hirschfelds\pwindex{Hirschfeld, Robert 17.\,9.\,1857 Žďár nad Sázavou – 2.\,4.\,1914 Salzburg@\textsc{Hirschfeld, Robert} (17.\,9.\,1857 Žďár nad Sázavou – 2.\,4.\,1914 Salzburg), \emph{Journalist, Musikkritiker}|pw}\pwindex{Hirschfeld, Emilie 12.\,12.\,1868 Berlin – 3.\,4.\,1942 Wien@\textsc{Hirschfeld, Emilie} (12.\,12.\,1868 Berlin – 3.\,4.\,1942 Wien), \emph{Sängerin}|pw} u.
               Chiavaccis\pwindex{Chiavacci, Vincenz 15.\,6.\,1847 Wien – 2.\,2.\,1916 ebd.@\textsc{Chiavacci, Vincenz} (15.\,6.\,1847 Wien – 2.\,2.\,1916 ebd.), \emph{Schriftsteller, Journalist, Beamter}|pw}\pwindex{Schönherr, Malvine 19.\,11.\,1867 Wien – 7.\,2.\,1956 ebd.@\textsc{Schönherr, Malvine} (19.\,11.\,1867 Wien – 7.\,2.\,1956 ebd.)|pw} von meinen, \textsc{resp.} unſern herzlichen Gegengrüßen dort erreicht. In Reichenau\oindex{Reichenau an der Rax@\textbf{Reichenau an der Rax}, \emph{Verwaltungsgebiet}|pw} war es ſehr ſchön, u es iſt ſchade, dſs Sie
                  \textcolor{gray}{und} dort nicht ein paar Stunden geſchenkt haben. Am \label{K_L04026-2v}\edtext{Montag u.
                  Dinſtag}{\lemma{\textnormal{\emph{Montag u.
                  Dinstag}}}\Cendnote{\textnormal{Siehe A. S.: \emph{Wiener Schnitzler}, 18. 7. 1904 und 19. 7. 1904.}}}\label{K_L04026-2} war
               ich auf der Rax\oindex{Rax@\textbf{Rax}, \emph{Berg}|pw} (mit Paul Marx\pwindex{Marx, Paul 21.\,7.\,1879 Wien – 30.\,10.\,1956 ebd.@\textsc{Marx, Paul} (21.\,7.\,1879 Wien – 30.\,10.\,1956 ebd.), \emph{Regisseur, Schauspieler}|pw}); ſo iſt mein Mittwoch \label{K_L04026-3v}\edtext{Telegra{\geminationm}}{\lemma{\textnormal{\emph{Telegramm}}}\Cendnote{\textnormal{XXXX Auszeichnungsfehler: Dokument L04212 nicht gefunden.}}}\label{K_L04026-3} leider zu ſpät für Sie angelangt. Nun iſt es
               hier in Wien\oindex{Wien@\textbf{Wien}, \emph{Verwaltungsgebiet}|pw} auch gar nicht übel, ich arbeite
               nicht wenig, und vonSo{\geminationn}tag an haben wir den
               Wagen von Julius\pwindex{Schnitzler, Julius 13.\,7.\,1865 Wien – 29.\,6.\,1939 ebd.@\textsc{Schnitzler, Julius} (13.\,7.\,1865 Wien – 29.\,6.\,1939 ebd.), \emph{Chirurg}|pw} auf 3 Wochen zur
               Verfügung, was alle eventuellen Großſtadtſo{\geminationm}erſchrecken
               mildert. Laſſen Sie ſichs wohlergehen und bald von ſich hören. Von Olga\pwindex{Schnitzler, Olga 17.\,1.\,1882 Wien – 13.\,1.\,1970 Lugano@\textsc{Schnitzler, Olga} (17.\,1.\,1882 Wien – 13.\,1.\,1970 Lugano), \emph{Schauspielerin, Sängerin}|pw} die ſchönsten Grüße.\pend
           \pstart Ihr \spacefill\mbox{A.}\pend{}\selectlanguage{ngerman}\endnumbering\briefempfaengerindex{Schwarzkopf, Gustav@\textsc{Schwarzkopf, Gustav}!zzzSchnitzler, Arthur@\emph{von Arthur Schnitzler}!1904-07-291@{29. 7. 1904}|)be}\mylabel{L04026h}
\begin{anhang}
\end{anhang}\newcommand{\dateiname}{L04026}\newcommand{\titel}{Arthur Schnitzler an Gustav Schwarzkopf, 19. 7. 1904}\newcommand{\editorInnen}{Herausgegeben von Jahnke, SelmaMüller, Martin Anton}%% latex-leseansicht-abspann.tex
%% Abspann für die Leseansicht.
%% Der Schalter \ifkorrekturansicht ist bereits durch den Vorspann gesetzt.

%% latex-abspann.tex
%% Gemeinsamer Abspann für Korrekturansicht und Leseansicht.
%% Setzt den Schalter \ifkorrekturansicht voraus (gesetzt in den
%% einbindenden Dateien latex-korrekturansicht-abspann.tex bzw.
%% latex-leseansicht-abspann.tex).
%% ---------------------------------------------------------------

\normalsize

% Das esempio-Environment wird nur in der Leseansicht benötigt
\ifkorrekturansicht\else
\newenvironment{esempio}[3]%
{
    \vspace{1.5ex}
    \rlap{\underline{#1}}
    \par
    \setlength{\parindent}{0cm}
    \nopagebreak
    \leftskip=#2cm
    \rightskip=#3cm
}
{
    \par
}
\fi

\doendnotes{C}
\bigskip
\vfill

\clearpage

\footnotesize

\ifkorrekturansicht
  \lohead{\textsc{register}}
\fi

% theindex-Environment neu definieren ohne reledmac
\makeatletter
\renewenvironment{theindex}{%
  \ifkorrekturansicht
    \section*{\indexname}%
  \else
    \subsubsection*{Index der erwähnten Entitäten}%
  \fi
  \setlength{\parindent}{0pt}%
  \setlength{\parskip}{0pt plus 0.3pt}%
  \let\item\@idxitem
}{%
  \ifkorrekturansicht\clearpage\fi
}
\makeatother

\IfFileExists{\jobname-pw.ind}{\input{\jobname-pw.ind}}{}

% Quellenangabe nur in der Leseansicht
\ifkorrekturansicht\else
% Fallback-Definitionen, falls die .tex-Datei \titel etc. nicht gesetzt hat
\providecommand{\titel}{}
\providecommand{\editorInnen}{}
\providecommand{\dateiname}{\jobname}

\vspace{3cm}

\vfill

\footnotesize
\textsc{Quelle}: \titel. Herausgegeben von {\editorInnen}. In: \emph{Arthur Schnitzler: Briefwechsel mit Autorinnen und Autoren}.
 Digitale Edition, https://schnitzler-briefe.acdh.oeaw.ac.at/{\dateiname}.html (Stand \today)
\fi

\end{document}


