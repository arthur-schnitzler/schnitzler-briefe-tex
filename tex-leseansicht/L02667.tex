%% latex-korrekturansicht-vorspann.tex
%% Vorspann für die Korrekturansicht.
%% Lädt die gemeinsame Datei latex-vorspann.tex mit gesetztem Schalter.

\newif\ifkorrekturansicht
\korrekturansichttrue

\input{../tex-inputs/latex-vorspann}


\section[Paul Goldmann an Arthur Schnitzler, 25. 7. 1891]{L02667 Paul Goldmann an Arthur Schnitzler, 25. 7. 1891}
\nopagebreak\mylabel{L02667v}
\rehead{ }\normalsize\beginnumbering\briefempfaengerindex{Schnitzler, Arthur@\textsc{Schnitzler, Arthur}!zzzGoldmann, Paul@\emph{von Paul Goldmann}!1891-07-251@{25. 7. 1891}|(be}
\toendnotes[C]{\smallbreak\pagebreak[2]}\Standort{DLA, A:Schnitzler, HS.NZ85.1.3162.}
\physDesc{Postkarte, 472 Zeichen
\newline{}Handschrift: 1) Bleistift, deutsche Kurrent\hspace{1em}2) Bleistift, lateinische Kurrent (\noindent{}Adresse)\hspace{1em}
\newline{}Versand: 1) Stempel: »\nobreak{}\oindex{Koeln@\textbf{Köln}, \emph{P.PPLA2}|pwk}Cöln (Rhein{[}land){]}, 25 7 91, Zug 13\nobreak{}«.   2) Stempel: »\nobreak{}\oindex{I., Innere Stadt@\textbf{I., Innere Stadt}, \emph{A.ADM3}|pwk}Wien 1/1, 27/7 91, 9½–11V, Bestellt\nobreak{}«. 
\newline{}Schnitzler: mit Bleistift das Datum »15/ 7 91« vermerkt }\toendnotes[C]{\smallbreak}\pstart{}{\pb}Österreich\oindex{Oesterreich@\textbf{Österreich}, \emph{A.PCLI}|pw}!\pend{}\pstart{}Herrn\pend{}\pstart{}Dr. Arthur Schnitzler\pend{}\pstart{}Wien\oindex{Wien@\textbf{Wien}, \emph{A.ADM2}|pw}\pend{}\pstart{}I, Giselastraſse 11\oindex{Ordination Arthur Schnitzler [Boesendorferstrasse 11]@\textbf{Ordination Arthur Schnitzler [Bösendorferstraße 11]}, \emph{Ordination}|pw}.\pend{}{\bigskip}\vspace{1em}
\pstart
           \noindent{}{\pb}\textsc{Köln\oindex{Koeln@\textbf{Köln}, \emph{P.PPLA2}|pw}}, 25. 7. – 1 Uhr Nachts. Mein lieber Arthur! Ich kehre nach Brüſſel\oindex{Bruessel@\textbf{Brüssel}, \emph{P.PPLC}|pw} zurück von einem 7 tägigen Aufenthalt, den ich
                  \textcolor{gray}{in}{ }\textsc{Frankfurt\oindex{Frankfurt am Main@\textbf{Frankfurt am Main}, \emph{P.PPLA3}|pw}} in Familien u. Redactionsangelegenheiten geno{\geminationm}en.
               Ärgerniß u. Kümmerniß ringsum. Ich denke Dein in Treue und Schmerzen. Oh, mein lieber
               Arthur und immer liebes Wien\oindex{Wien@\textbf{Wien}, \emph{A.ADM2}|pw}! So fahre ich in die
               Nacht hinein wie ein Verdammter und Verfluchter! {\dots}\pend
           
\pstart
           Gott behüte Dich!{\\[\baselineskip]}Dein {\\[\baselineskip]}\spacefill\mbox{Paul}\pend
           \leftskip=0em{}
\pstart
           \noindent{}\label{T_L02667-1v}\edtext{Auf den Knien geſchrieben.}{\lemma{\textnormal{\emph{Auf … geſchrieben.}}}\Cendnote{\textnormal{am oberen Rand}}}\label{T_L02667-1}\pend
           \selectlanguage{ngerman}\endnumbering\briefempfaengerindex{Schnitzler, Arthur@\textsc{Schnitzler, Arthur}!zzzGoldmann, Paul@\emph{von Paul Goldmann}!1891-07-251@{25. 7. 1891}|)be}\mylabel{L02667h}  \normalsize

\doendnotes{C}
\bigskip
\vfill

\clearpage

\footnotesize

\lohead{\textsc{register}}

% Definiere theindex-Environment komplett neu ohne reledmac
\makeatletter
\renewenvironment{theindex}{%
  \section*{\indexname}%
  \setlength{\parindent}{0pt}%
  \setlength{\parskip}{0pt plus 0.3pt}%
  \let\item\@idxitem
}{%
  \clearpage
}
\makeatother

\IfFileExists{\jobname-pw.ind}{\input{\jobname-pw.ind}}{}

\end{document}

      