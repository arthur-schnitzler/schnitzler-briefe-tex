%% latex-leseansicht-vorspann.tex
%% Vorspann für die Leseansicht.
%% Lädt die gemeinsame Datei latex-vorspann.tex mit nicht gesetztem Schalter.

\newif\ifkorrekturansicht
\korrekturansichtfalse

\input{../tex-inputs/latex-vorspann}


         \newcommand{\erwaehnteOrte}{Orte: Brüssel, Bösendorferstraße, Frankfurt am Main, I., Innere Stadt, Köln, Wien, Österreich}
         \newcommand{\erwaehnteWerke}{
               \section[Paul Goldmann an Arthur Schnitzler, 25. 7. 1891]{ Paul Goldmann an Arthur Schnitzler, 25. 7. 1891}\nopagebreak\mylabel{v}\rehead{ }\begin{ledgroupsized}[t]{13cm}\normalsize\beginnumbering \toendnotes[C]{\smallbreak\pagebreak[2]} \Standort{DLA, A:Schnitzler, HS.NZ85.1.3162.}
\physDesc{Postkarte
\newline{}Handschrift: 1) Bleistift, deutsche Kurrent\hspace{1em}2) Bleistift, lateinische Kurrent (\noindent{}Adresse)\hspace{1em}\newline{}Versand: 1) Stempel: »\nobreak{}\oindex{Koeln@\textbf{Köln}|pwk}Cöln (Rhein{[}land){]}, 25 7 91, Zug 13\nobreak{}«.   2) Stempel: »\nobreak{}\oindex{I., Innere Stadt@\textbf{I., Innere Stadt}|pwk}Wien 1/1, 27/7 91, 9½–11V, Bestellt\nobreak{}«. 
\newline{}Schnitzler: mit Bleistift das Datum »15/ 7 91« vermerkt }\toendnotes[C]{\smallbreak}\pstart{}{\pb}Österreich\oindex{Oesterreich@\textbf{Österreich}|pw}!\pend{}\pstart{}Herrn\pend{}\pstart{}Dr. Arthur Schnitzler\pend{}\pstart{}Wien\oindex{Wien@\textbf{Wien}|pw}\pend{}\pstart{}I, Giselastraße 11\oindex{Boesendorferstrasse@\textbf{Bösendorferstraße}|pw}.\pend{}{\bigskip}\pstart
           \noindent{}{\pb}\textsc{Köln\oindex{Koeln@\textbf{Köln}|pw}}, 25. 7. – 1 Uhr Nachts. Mein lieber Arthur! Ich kehre nach Brüſſel\oindex{Bruessel@\textbf{Brüssel}|pw} zurück von einem 7 tägigen Aufenthalt, den ich
                  \textcolor{gray}{in}{ }\textsc{Frankfurt\oindex{Frankfurt am Main@\textbf{Frankfurt am Main}|pw}} in Familien u. Redactionsangelegenheiten geno{\geminationm}en.
               Ärgerniß u. Kümmerniß ringsum. Ich denke Dein in Treue und Schmerzen. Oh, mein lieber
               Arthur und immer liebes Wien\oindex{Wien@\textbf{Wien}|pw}! So fahre ich in die
               Nacht hinein wie ein Verdammter und Verfluchter! {\dots}\pend
           \pstart
           Gott behüte Dich!{\\[\baselineskip]}Dein {\\[\baselineskip]}\spacefill\mbox{Paul}\pend
           \leftskip=0em{}\pstart
           \noindent{}\label{T_L02667-1v}\edtext{Auf den Knien geſchrieben.}{\lemma{\textnormal{\emph{Auf … geſchrieben.}}}\Cendnote{\textnormal{am oberen Rand}}}\label{T_L02667-1h}\pend
           
         
         \endnumbering\mylabel{h}\end{ledgroupsized}  \newcommand{\dateiname}{L02667}\newcommand{\titel}{Paul Goldmann an Arthur Schnitzler, 25. 7. 1891}\newcommand{\editorInnen}{Martin Anton Müller und Laura Untner}%% latex-leseansicht-abspann.tex
%% Abspann für die Leseansicht.
%% Der Schalter \ifkorrekturansicht ist bereits durch den Vorspann gesetzt.

%% latex-abspann.tex
%% Gemeinsamer Abspann für Korrekturansicht und Leseansicht.
%% Setzt den Schalter \ifkorrekturansicht voraus (gesetzt in den
%% einbindenden Dateien latex-korrekturansicht-abspann.tex bzw.
%% latex-leseansicht-abspann.tex).
%% ---------------------------------------------------------------

\normalsize

% Das esempio-Environment wird nur in der Leseansicht benötigt
\ifkorrekturansicht\else
\newenvironment{esempio}[3]%
{
    \vspace{1.5ex}
    \rlap{\underline{#1}}
    \par
    \setlength{\parindent}{0cm}
    \nopagebreak
    \leftskip=#2cm
    \rightskip=#3cm
}
{
    \par
}
\fi

\doendnotes{C}
\bigskip
\vfill

\clearpage

\footnotesize

\ifkorrekturansicht
  \lohead{\textsc{register}}
\fi

% theindex-Environment neu definieren ohne reledmac
\makeatletter
\renewenvironment{theindex}{%
  \ifkorrekturansicht
    \section*{\indexname}%
  \else
    \subsubsection*{Index der erwähnten Entitäten}%
  \fi
  \setlength{\parindent}{0pt}%
  \setlength{\parskip}{0pt plus 0.3pt}%
  \let\item\@idxitem
}{%
  \ifkorrekturansicht\clearpage\fi
}
\makeatother

\IfFileExists{\jobname-pw.ind}{\input{\jobname-pw.ind}}{}

% Quellenangabe nur in der Leseansicht
\ifkorrekturansicht\else
% Fallback-Definitionen, falls die .tex-Datei \titel etc. nicht gesetzt hat
\providecommand{\titel}{}
\providecommand{\editorInnen}{}
\providecommand{\dateiname}{\jobname}

\vspace{3cm}

\vfill

\footnotesize
\textsc{Quelle}: \titel. Herausgegeben von {\editorInnen}. In: \emph{Arthur Schnitzler: Briefwechsel mit Autorinnen und Autoren}.
 Digitale Edition, https://schnitzler-briefe.acdh.oeaw.ac.at/{\dateiname}.html (Stand \today)
\fi

\end{document}


      