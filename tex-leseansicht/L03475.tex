%% latex-korrekturansicht-vorspann.tex
%% Vorspann für die Korrekturansicht.
%% Lädt die gemeinsame Datei latex-vorspann.tex mit gesetztem Schalter.

\newif\ifkorrekturansicht
\korrekturansichttrue

\input{../tex-inputs/latex-vorspann}


\section[ Paul Goldmann an Arthur Schnitzler, 13. 1. 1911]{L03475 Paul Goldmann an Arthur Schnitzler, 13. 1. 1911}
\nopagebreak\mylabel{L03475v}
\rehead{ }\normalsize\beginnumbering\briefempfaengerindex{Schnitzler, Arthur@\textsc{Schnitzler, Arthur}!zzzGoldmann, Paul@\emph{von Paul Goldmann}!1911-01-131@{13. 1. 1911}|(be}
\toendnotes[C]{\smallbreak\pagebreak[2]}\Standort{DLA, A:Schnitzler, HS.NZ85.1.3176.}
\physDesc{Brief, 7 Blätter, 26 Seiten, 12023 Zeichen
\newline{}Handschrift: blaue Tinte, deutsche Kurrent}\toendnotes[C]{\smallbreak}
\pstart
           \raggedleft{}{\pb}\textcolor{gray}{\textbf{W. SCHÖNEBERGER-UFER 34\oindex{Schoeneberger Ufer@\textbf{Schöneberger Ufer}, \emph{Straße (K.STR)}|pw}.}}\pend
           
\pstart
           1\substVorne{}\textsuperscript{\textcolor{gray}{1}}\substDazwischen{}3\substHinten{}. 1. 1\substVorne{}\textsuperscript{0}\substDazwischen{}1\substHinten{}.\pend
           
\pstart\center{}Lieber Freund,\pend\vspace{0.5em}
\pstart
           Die \label{K_L03475-1v}\edtext{Überſendung der Kopien meiner
                  Briefe}{\lemma{\textnormal{\emph{Überſendung … Briefe}}}\Cendnote{\textnormal{Siehe Paul Goldmann an Arthur Schnitzler, 30. 12. 1910.
               }}}\label{K_L03475-1} habe ich mit einiger Sorge erwartet. Denn in jener Zeit, in der dieſe
               Angelegenheit ſpielt, war mir die Freundſchaft mit Dir \strikeout{ſehr} viel, bildete ſie eines der großen Beſitztümer meines Lebens. Und ich
               fragte mich, ehe ich die Kopien erhielt\substVorne{}\textsuperscript{,}\substDazwischen{}:\substHinten{} ſollte ich nicht vielleicht, in der Sorge, dieſes Freundſchafts-Beſitztum
               vor jeder Gehahr zu behüten, mich ſchwach gezeigt haben?\pend
           
\pstart
           Als ich die Copien las, war ich ſtarr vor Staunen. Das alſo waren {\pb}die \label{K_L03475-2v}\edtext{»Beweisſtücke«}{\lemma{\textnormal{\emph{»Beweisſtücke«}}}\Cendnote{\textnormal{Bezug auf die Auseinandersetzungen am 26. 12. 1910 und vor allem am 28. 12. 1910, siehe Paul Goldmann an Arthur Schnitzler, 26. 12. 1910.
               }}}\label{K_L03475-2} gegen mich! Dies die Dokumente gegen meine Ehre! Denn es iſt Dir ſicherlich
               nicht klar geworden, daß es ſich in alledem um meine Ehre handelt, – daß Du meine
               Ehre angreifſt, indem Du mich als einen Menſchen hinſtellſt, der heimlich lobt u.
               öffentlich tadelt, der in ſeinen Briefen dem Freunde ſchmeichelt u. ihn dann
               öffentlich – noch dazu, wie Du weißt, mit einem beſonderen Vergnügen –
               herunterreißt.\pend
           
\pstart
           Das alſo waren die Dokumente! Ich las die Briefe u. fand, daß ich darin mit aller
               Deutlichkeit ſtarke Bedenken gegen Dein Werk\pwindex{Schleier der Beatrice. Schauspiel in fuenf Akten@\emph{Der Schleier der Beatrice. Schauspiel in fünf Akten}|pwv} formulirt hatte, – mit aller Deutlichkeit für Jedermann
               außer für {\pb}den durch Größengefühl und
               Selbſtgefälligkeit jeden Urteils beraubten Autor. Jeder ruhig u. objektiv Urteilende
               wird auch finden, daß meine ſpätere öffentliche \label{K_L03475-3v}\edtext{Kritik\pwindex{Berliner Theater. (»Der Schleier der Beatrice« von Arthur Schnitzler.)@\emph{Berliner Theater. (»Der Schleier der Beatrice« von Arthur Schnitzler.)}|pwv}}{\lemma{\textnormal{\emph{Kritik}}}\Cendnote{\textnormal{Paul Goldmann\pwindex{Goldmann, Paul 31.01.1865 – 25.09.1935@\textsc{Goldmann, Paul} (31.01.1865 – 25.09.1935), \emph{Schriftsteller/Schriftstellerin, Journalist/Journalistin}|pwk}: \emph{Berliner Theater. (»Der Schleier der Beatrice« von Arthur
                        Schnitzler)}\pwindex{Berliner Theater. (»Der Schleier der Beatrice« von Arthur Schnitzler.)@\emph{Berliner Theater. (»Der Schleier der Beatrice« von Arthur Schnitzler.)}|pwk}. In: \emph{Neue Freie
                        Presse}\pwindex{Neue Freie Presse@\emph{Neue Freie Presse}|pwk}, Nr. 13.851, 19. 3. 1903,
                     Morgenblatt, S. 1–5.}}}\label{K_L03475-3} nichts iſt als die Ausführung der in den
               Briefen bereits kurz formulirten Bedenken. Jeder ruhig u. objektiv Urteilende wird
               weiter finden, daß in dieſen Briefen ein Freund dem Freunde die Wahrheit ſagt, \strikeout{\textcolor{gray}{×}\-\textcolor{gray}{×}} daß der Freund aber gleichzeitig beſtrebt iſt, dem Freunde nicht wehzutun, u.
               daß er darum, damit der Tadel, den er auszuſprechen ſich genötigt ſieht, nur ja nicht
               verletze \strikeout{verletz\textcolor{gray}{e}} verletze, \strikeout{\textcolor{gray}{d}\textcolor{gray}{×}\-\textcolor{gray}{×}} das Lob, das er ſpenden kann, in möglichſt ſtarken Ausdrücken formulirt. {\pb}Di\substVorne{}\textsuperscript{r}\substDazwischen{}e\substHinten{} großen Fehler, unter den\substVorne{}\textsuperscript{,}\substDazwischen{}en,\substHinten{} meiner Anſicht nach, Dein Stück\pwindex{Schleier der Beatrice. Schauspiel in fuenf Akten@\emph{Der Schleier der Beatrice. Schauspiel in fünf Akten}|pwv} leidet, \substVorne{}\textsuperscript{\textcolor{gray}{i}}\substDazwischen{}ſ\substHinten{}ind in meinen Briefen klar gekennzeichnet. Du haſt darüber hinweggeleſen u.
               von meinen Briefen nur behalten, daß ich Dich mit \label{K_L03475-4v}\edtext{\textsc{Grillparzer\pwindex{Grillparzer, Franz 15.01.1791 – 21.01.1872@\textsc{Grillparzer, Franz} (15.01.1791 – 21.01.1872), \emph{Schriftsteller/Schriftstellerin, Beamter/Beamte}|pw}} verglichen}{\lemma{\textnormal{\emph{Grillparzer verglichen}}}\Cendnote{\textnormal{Vgl. Paul Goldmann an Arthur Schnitzler, 20. 2. 1900.
               }}}\label{K_L03475-4} habe. Das iſt bezeichnend – aber nicht für mich, ſondern für Dich.\pend
           
\pstart
           In meinen Briefen habe ich Dich gelobt. Und in meiner Kritik\pwindex{Berliner Theater. (»Der Schleier der Beatrice« von Arthur Schnitzler.)@\emph{Berliner Theater. (»Der Schleier der Beatrice« von Arthur Schnitzler.)}|pwv}? In meinen Briefen ſteht: »Seit \textsc{Grillparzer\pwindex{Grillparzer, Franz 15.01.1791 – 21.01.1872@\textsc{Grillparzer, Franz} (15.01.1791 – 21.01.1872), \emph{Schriftsteller/Schriftstellerin, Beamter/Beamte}|pw}} hat man auf dem Wien\oindex{Wien@\textbf{Wien}, \emph{A.ADM2}|pw}er Theater ſolche Verſe\pwindex{Schleier der Beatrice. Schauspiel in fuenf Akten@\emph{Der Schleier der Beatrice. Schauspiel in fünf Akten}|pwv} nicht gehört.« In meiner
                  Kritik\pwindex{Berliner Theater. (»Der Schleier der Beatrice« von Arthur Schnitzler.)@\emph{Berliner Theater. (»Der Schleier der Beatrice« von Arthur Schnitzler.)}|pwv}: »In der Form wenigſtens zeigt \textsc{Schnitzler} ſich als ein \uline{würdiger Schüler der Meiſter (der Klaſſiker)}, denen er {\pb}nacheifert. Daß \textsc{Schnitzler} dieſe Form ſich anzueignen vermochte, deutet auf eine
                  künſtleriſche Selbſterziehung hin \strikeout{hin}, die man bei
                  den deutſchen Autoren der Gegenwart ſelten findet; es iſt ein weiter, mühevoller,
                  ehrenvoller Weg vom ›\textsc{Anatol}\pwindex{Anatol@\emph{Anatol}|pw}‹ bis zum ›Schleier der
                     Beatrice\pwindex{Schleier der Beatrice. Schauspiel in fuenf Akten@\emph{Der Schleier der Beatrice. Schauspiel in fünf Akten}|pw}‹{[}.{]} Das Drama\pwindex{Schleier der Beatrice. Schauspiel in fuenf Akten@\emph{Der Schleier der Beatrice. Schauspiel in fünf Akten}|pwv} ſpricht namentlich in ſeinen Verſen – wohllautenden
                  Verſen von wien\oindex{Wien@\textbf{Wien}, \emph{A.ADM2}|pw}eriſcher Weichheit – eine
                  vornehme Sprache.\pwindex{Berliner Theater. (»Der Schleier der Beatrice« von Arthur Schnitzler.)@\emph{Berliner Theater. (»Der Schleier der Beatrice« von Arthur Schnitzler.)}|pwv}« An einer anderen Stelle\pwindex{Berliner Theater. (»Der Schleier der Beatrice« von Arthur Schnitzler.)@\emph{Berliner Theater. (»Der Schleier der Beatrice« von Arthur Schnitzler.)}|pwv} wird von »prächtigen Verſen\pwindex{Berliner Theater. (»Der Schleier der Beatrice« von Arthur Schnitzler.)@\emph{Berliner Theater. (»Der Schleier der Beatrice« von Arthur Schnitzler.)}|pwv}« geſprochen, die dann citirt werden. Von
                  \textsc{Beatrice\pwindex{Schleier der Beatrice. Schauspiel in fuenf Akten@\emph{Der Schleier der Beatrice. Schauspiel in fünf Akten}|pwv}} wird geſagt, daß ſie »ein
                  liebliches {\pb}Geſchöpf iſt, eine echt \textsc{Schnitzlerische} Mädchengeſtalt, von poetiſchem Schimmer
                  umfloſſen\pwindex{Berliner Theater. (»Der Schleier der Beatrice« von Arthur Schnitzler.)@\emph{Berliner Theater. (»Der Schleier der Beatrice« von Arthur Schnitzler.)}|pwv}«. Von einer Scene\pwindex{Schleier der Beatrice. Schauspiel in fuenf Akten@\emph{Der Schleier der Beatrice. Schauspiel in fünf Akten}|pwv} wird geſagt, daß ſie »die bedeutendſte des Stück\pwindex{Schleier der Beatrice. Schauspiel in fuenf Akten@\emph{Der Schleier der Beatrice. Schauspiel in fünf Akten}|pwv}es iſt u. \textsc{Schnitzlers}
                  dramatiſche Begabung im hellſten Lichte zeigt\pwindex{Berliner Theater. (»Der Schleier der Beatrice« von Arthur Schnitzler.)@\emph{Berliner Theater. (»Der Schleier der Beatrice« von Arthur Schnitzler.)}|pwv}« \textsc{etc}.\pend
           
\pstart
           Und von dieſer Kritik\pwindex{Berliner Theater. (»Der Schleier der Beatrice« von Arthur Schnitzler.)@\emph{Berliner Theater. (»Der Schleier der Beatrice« von Arthur Schnitzler.)}|pwv} wagſt
               Du zu behaupten, daß ſie \strikeout{doch} Dein Werk\pwindex{Schleier der Beatrice. Schauspiel in fuenf Akten@\emph{Der Schleier der Beatrice. Schauspiel in fünf Akten}|pwv} verreißt, während meine Briefe es
               gelobt haben? Ich muß noch die Einſchränkung machen, daß die lobenden Ausdrücke in
               meinen Briefen \strikeout{\textcolor{gray}{×}\-\textcolor{gray}{×}\-\textcolor{gray}{×}\-\textcolor{gray}{×}} ſtärker klingen, als in der Kritik\pwindex{Berliner Theater. (»Der Schleier der Beatrice« von Arthur Schnitzler.)@\emph{Berliner Theater. (»Der Schleier der Beatrice« von Arthur Schnitzler.)}|pwv}. \strikeout{\textcolor{gray}{×}} Einen Grund dafür – das Beſtreben des Freundes, mit möglichſt viel {\pb}Lob den Tadel, den er ausſpricht, weniger
               empfindlich zu machen – habe ich ſchon angeführt. Ein anderer Grund iſt der, daß man
               in einem Privatbrief ſeine Ausdrücke nicht ſo vorſichtig abwägt, wie man dies tut,
               wenn man in der Ausübung ſeines kritiſchen Berufes, \strikeout{\textcolor{gray}{×}\-\textcolor{gray}{×}\-\textcolor{gray}{×}} in dem Bewußtſein, daß man für jedes Wort die volle Verantwortung zu
               übernehmen hat, \strikeout{\textcolor{gray}{vor}} öffentlich ſich äußert. Entſteht aus dieſem Grunde ein Widerſpruch zwiſchen
               Privatbriefen des Kritikers u. der von ihm {\pb}veröffentlichten Kritik, ſo trifft die Verantwortung nicht den Kritiker, ſondern
               den, der es verſucht, \strikeout{P} deſſen Privatbriefe gegen ihn
               auszuſpielen.\pend
           
\pstart
           Im Übrigen aber habe ich angeſichts der Briefkopien u. der Kritik\pwindex{Berliner Theater. (»Der Schleier der Beatrice« von Arthur Schnitzler.)@\emph{Berliner Theater. (»Der Schleier der Beatrice« von Arthur Schnitzler.)}|pwv}, die beide hier vor mir liegen, mit
               aller Entſchiedenheit zu erklären: Die Briefe loben nicht nur das Stück\pwindex{Schleier der Beatrice. Schauspiel in fuenf Akten@\emph{Der Schleier der Beatrice. Schauspiel in fünf Akten}|pwv}, ſondern ſie ſprechen auch bereits
               die \label{K_L03475-5v}\edtext{Einwendungen}{\lemma{\textnormal{\emph{Einwendungen}}}\Cendnote{\textnormal{Siehe insbesondere die Briefe Goldmanns\pwindex{Goldmann, Paul 31.01.1865 – 25.09.1935@\textsc{Goldmann, Paul} (31.01.1865 – 25.09.1935), \emph{Schriftsteller/Schriftstellerin, Journalist/Journalistin}|pwk} an Schnitzler vom 11. 2. 1900, 25. 1. [1902] und 17. 3. [1903].}}}\label{K_L03475-5} aus, die, meiner Anſicht nach,
               dagegen zu erheben ſind. Die Kritik\pwindex{Berliner Theater. (»Der Schleier der Beatrice« von Arthur Schnitzler.)@\emph{Berliner Theater. (»Der Schleier der Beatrice« von Arthur Schnitzler.)}|pwv} tadelt nicht nur das \strikeout{S\textcolor{gray}{t}}{ }Stück\pwindex{Schleier der Beatrice. Schauspiel in fuenf Akten@\emph{Der Schleier der Beatrice. Schauspiel in fünf Akten}|pwv}, ſondern {\pb}läßt ihm auch alle jene Anerkennung zuteil werden,
               die es\pwindex{Schleier der Beatrice. Schauspiel in fuenf Akten@\emph{Der Schleier der Beatrice. Schauspiel in fünf Akten}|pwv}, meiner Anſicht nach,
               verdient. Es beſteht höchſtens in der \textsc{Nuance} einiger
               Ausdrücke, aber im Weſen kein Widerſpruch zwiſchen den Briefen u. der Kritik\pwindex{Berliner Theater. (»Der Schleier der Beatrice« von Arthur Schnitzler.)@\emph{Berliner Theater. (»Der Schleier der Beatrice« von Arthur Schnitzler.)}|pwv}. Und den Vorwurf, den \strikeout{g} Du gegen mich erhoben haſt, daß ich als Freund wie
               als Kritiker meine Pflicht gegen Dich vergeſſen habe, weiſe ich mit Entrüſtung
                  zurück{\dotsfive}\pend
           
\pstart
           Ich komme jetzt zum {\pb}zweiten Fall, dem Fall der
                  »Lebendigen Stunden\pwindex{Lebendige Stunden. Vier Einakter@\emph{Lebendige Stunden. Vier Einakter}|pw}«. Hier liegen leider
               keine Dokumente vor, keine \label{K_L03475-6v}\edtext{Briefe}{\lemma{\textnormal{\emph{Briefe}}}\Cendnote{\textnormal{Das ist ein weiteres Indiz dafür, dass Schnitzler nur Ausschnitte aus der
                  Korrespondenz von 1900 als Briefkopien vorgelegt hatte.
                  Tatsächlich hatte Goldmann\pwindex{Goldmann, Paul 31.01.1865 – 25.09.1935@\textsc{Goldmann, Paul} (31.01.1865 – 25.09.1935), \emph{Schriftsteller/Schriftstellerin, Journalist/Journalistin}|pwk} selten brieflich
                  Kritik an \emph{Lebendige Stunden}\pwindex{Lebendige Stunden. Vier Einakter@\emph{Lebendige Stunden. Vier Einakter}|pwk} geübt, siehe Paul Goldmann an Arthur Schnitzler und Olga
               Gussmann, 23. 12. [1901] und 25. 1. [1902].}}}\label{K_L03475-6}, von
               denen Du Kopien hätteſt machen können. Hier handelt es ſich um mündliche Äußerungen,
               die ich getan haben ſoll. Würden ſie im genauen, beglaubigten Wortlaut vorliegen, ſo
               würden ſich die »Widersprüche« zwiſchen dieſen Äußerungen u. meiner ſpäter
               veröffentlichten \label{K_L03475-7v}\edtext{Kritik\pwindex{Berliner Theater. (»Lebendige Stunden« von Arthur Schnitzler.)@\emph{Berliner Theater. (»Lebendige Stunden« von Arthur Schnitzler.)}|pwv}}{\lemma{\textnormal{\emph{Kritik}}}\Cendnote{\textnormal{Paul Goldmann\pwindex{Goldmann, Paul 31.01.1865 – 25.09.1935@\textsc{Goldmann, Paul} (31.01.1865 – 25.09.1935), \emph{Schriftsteller/Schriftstellerin, Journalist/Journalistin}|pwk}: \emph{Berliner Theater. (»Lebendige Stunden« von Arthur
                        Schnitzler)}\pwindex{Berliner Theater. (»Lebendige Stunden« von Arthur Schnitzler.)@\emph{Berliner Theater. (»Lebendige Stunden« von Arthur Schnitzler.)}|pwk}. In: \emph{Neue Freie
                        Presse}\pwindex{Neue Freie Presse@\emph{Neue Freie Presse}|pwk}, Nr. 13.438, 22. 1. 1902,
                     Morgenblatt, S. 1–4.}}}\label{K_L03475-7} wahrſcheinlich ebenſo aufklären, wie im
               Falle der »\textsc{Beatrice\pwindex{Schleier der Beatrice. Schauspiel in fuenf Akten@\emph{Der Schleier der Beatrice. Schauspiel in fünf Akten}|pw}}«. {\pb}Möglicherweiſe habe ich auch hier
               Einwendungen formulirt, über die Du hinweggehört haſt, wie Du über die gegen die »\textsc{Beatrice\pwindex{Schleier der Beatrice. Schauspiel in fuenf Akten@\emph{Der Schleier der Beatrice. Schauspiel in fünf Akten}|pw}}« in meinen Briefen hinweggeleſen haſt. Ich habe nicht einmal meine Kritik\pwindex{Berliner Theater. (»Lebendige Stunden« von Arthur Schnitzler.)@\emph{Berliner Theater. (»Lebendige Stunden« von Arthur Schnitzler.)}|pwv} über die »Lebendigen Stunden\pwindex{Lebendige Stunden. Vier Einakter@\emph{Lebendige Stunden. Vier Einakter}|pw}« zur Hand u. kann daher nicht
               konſtatiren, ob ſie wirklich ſo ohne jede Einſchränkung tadelnd war, wie Du
               behaupteſt. Denn ich habe dieſe \strikeout{Kr\textcolor{gray}{i}}{ }Beſprechung\pwindex{Berliner Theater. (»Lebendige Stunden« von Arthur Schnitzler.)@\emph{Berliner Theater. (»Lebendige Stunden« von Arthur Schnitzler.)}|pwv} in die \label{K_L03475-8v}\edtext{Sammlungen meiner Kritiken\pwindex{»neue Richtung«. Polemische Aufsaetze ueber Berliner Theater-Auffuehrungen@\emph{Die »neue Richtung«. Polemische Aufsätze über Berliner Theater-Aufführungen}|pwv}\pwindex{Aus dem dramatischen Irrgarten. Polemische Aufsaetze ueber Berliner Theaterauffuehrungen@\emph{Aus dem dramatischen Irrgarten. Polemische Aufsätze über Berliner Theateraufführungen}|pwv}\pwindex{Vom Rueckgang der deutschen Buehne. Polemische Aufsaetze ueber Berliner Theater-Auffuehrungen@\emph{Vom Rückgang der deutschen Bühne. Polemische Aufsätze über Berliner Theater-Aufführungen}|pwv}\pwindex{Literatenstuecke und Ausstattungsregie. Polemische Aufsaetze ueber Berliner Theater-Auffuehrungen@\emph{Literatenstücke und Ausstattungsregie. Polemische Aufsätze über Berliner Theater-Aufführungen}|pwv}}{\lemma{\textnormal{\emph{Sammlungen … Kritiken}}}\Cendnote{\textnormal{Goldmann\pwindex{Goldmann, Paul 31.01.1865 – 25.09.1935@\textsc{Goldmann, Paul} (31.01.1865 – 25.09.1935), \emph{Schriftsteller/Schriftstellerin, Journalist/Journalistin}|pwk} hatte bereits mehrere Kritiksammlungen\pwindex{»neue Richtung«. Polemische Aufsaetze ueber Berliner Theater-Auffuehrungen@\emph{Die »neue Richtung«. Polemische Aufsätze über Berliner Theater-Aufführungen}|pwkv}\pwindex{Aus dem dramatischen Irrgarten. Polemische Aufsaetze ueber Berliner Theaterauffuehrungen@\emph{Aus dem dramatischen Irrgarten. Polemische Aufsätze über Berliner Theateraufführungen}|pwkv}\pwindex{Vom Rueckgang der deutschen Buehne. Polemische Aufsaetze ueber Berliner Theater-Auffuehrungen@\emph{Vom Rückgang der deutschen Bühne. Polemische Aufsätze über Berliner Theater-Aufführungen}|pwkv}\pwindex{Literatenstuecke und Ausstattungsregie. Polemische Aufsaetze ueber Berliner Theater-Auffuehrungen@\emph{Literatenstücke und Ausstattungsregie. Polemische Aufsätze über Berliner Theater-Aufführungen}|pwkv} veröffentlicht (\emph{Die
                     »neue Richtung«}\pwindex{»neue Richtung«. Polemische Aufsaetze ueber Berliner Theater-Auffuehrungen@\emph{Die »neue Richtung«. Polemische Aufsätze über Berliner Theater-Aufführungen}|pwk}, 1903, \emph{Aus dem dramatischen Irrgarten}\pwindex{Aus dem dramatischen Irrgarten. Polemische Aufsaetze ueber Berliner Theaterauffuehrungen@\emph{Aus dem dramatischen Irrgarten. Polemische Aufsätze über Berliner Theateraufführungen}|pwk}, 1905, \emph{Vom Rückgang der deutschen
                     Bühne}\pwindex{Vom Rueckgang der deutschen Buehne. Polemische Aufsaetze ueber Berliner Theater-Auffuehrungen@\emph{Vom Rückgang der deutschen Bühne. Polemische Aufsätze über Berliner Theater-Aufführungen}|pwk}, 1908, und \emph{Literatenstücke und Ausstattungsregie}\pwindex{Literatenstuecke und Ausstattungsregie. Polemische Aufsaetze ueber Berliner Theater-Auffuehrungen@\emph{Literatenstücke und Ausstattungsregie. Polemische Aufsätze über Berliner Theater-Aufführungen}|pwk},
                     1910). In dem Band\pwindex{Vom Rueckgang der deutschen Buehne. Polemische Aufsaetze ueber Berliner Theater-Auffuehrungen@\emph{Vom Rückgang der deutschen Bühne. Polemische Aufsätze über Berliner Theater-Aufführungen}|pwkv} von 1905 sind Goldmanns\pwindex{Goldmann, Paul 31.01.1865 – 25.09.1935@\textsc{Goldmann, Paul} (31.01.1865 – 25.09.1935), \emph{Schriftsteller/Schriftstellerin, Journalist/Journalistin}|pwk}{ }Kritiken\pwindex{Berliner Theater. (»Der Schleier der Beatrice« von Arthur Schnitzler.)@\emph{Berliner Theater. (»Der Schleier der Beatrice« von Arthur Schnitzler.)}|pwkv}\pwindex{Berliner Theater. »Der einsame Weg«. Von Arthur Schnitzler@\emph{Berliner Theater. »Der einsame Weg«. Von Arthur Schnitzler}|pwkv} zu \emph{Der
                     Schleier der Beatrice}\pwindex{einsame Weg. Schauspiel in fuenf Akten@\emph{Der einsame Weg. Schauspiel in fünf Akten}|pwk} und zu \emph{Der einsame
                     Weg}\pwindex{Schleier der Beatrice. Schauspiel in fuenf Akten@\emph{Der Schleier der Beatrice. Schauspiel in fünf Akten}|pwk} enthalten. Der Band\pwindex{Vom Rueckgang der deutschen Buehne. Polemische Aufsaetze ueber Berliner Theater-Auffuehrungen@\emph{Vom Rückgang der deutschen Bühne. Polemische Aufsätze über Berliner Theater-Aufführungen}|pwkv} von 1908 enthält Goldmanns\pwindex{Goldmann, Paul 31.01.1865 – 25.09.1935@\textsc{Goldmann, Paul} (31.01.1865 – 25.09.1935), \emph{Schriftsteller/Schriftstellerin, Journalist/Journalistin}|pwk}{ }Kritik\pwindex{Berliner Theater. »Der Ruf des Lebens« von Arthur Schnitzler@\emph{Berliner Theater. »Der Ruf des Lebens« von Arthur Schnitzler}|pwkv} zu \emph{Der Ruf des
                     Lebens}\pwindex{Ruf des Lebens. Schauspiel in drei Akten@\emph{Der Ruf des Lebens. Schauspiel in drei Akten}|pwk}.}}}\label{K_L03475-8} nicht aufgenommen. Warum nicht? Weil ich mir damals ſagte:
               die Kritik\pwindex{Berliner Theater. (»Lebendige Stunden« von Arthur Schnitzler.)@\emph{Berliner Theater. (»Lebendige Stunden« von Arthur Schnitzler.)}|pwv} zu ſchreiben, war
               meine Pflicht; {\pb}ſie in mein Buch aufzunehmen, bin
               ich nicht verpflichtet; u. ich habe ſie nicht aufgenommen, aus Rückſicht auf den
               Freund, über deſſen Werk\pwindex{Lebendige Stunden. Vier Einakter@\emph{Lebendige Stunden. Vier Einakter}|pwv} ſie
               ungünſtig urteilte. In einem eigentümlichen Lichte erſcheint mir heut dieſe Rückſicht
               auf den Freund, der Briefe von mir, in denen ich redlich beſtrebt war, ein herzliches
               freundſchaftliches Empfinden mit der Wahrheit in Einklang zu bringen, heranzieht, um
               damit meine Charakterloſigkleit zu beweiſen!\pend
           
\pstart
           Es fehlen mir alſo für den Fall der »Lebendigen
                  Stunden\pwindex{Lebendige Stunden. Vier Einakter@\emph{Lebendige Stunden. Vier Einakter}|pw}« {\pb}alle Dokumente\strikeout{,} u. ich bin auf mein Gedächtnis angewieſen. Dieſes
               Gedächtnis ſagt mir, daß ich mich, nach der \label{K_L03475-9v}\edtext{Vorleſung\pwindex{Lebendige Stunden. Vier Einakter@\emph{Lebendige Stunden. Vier Einakter}|pwv} im Walde zu \textsc{Welsberg}\oindex{Welsberg-Taisten@\textbf{Welsberg-Taisten}, \emph{A.ADM3}|pw}}{\lemma{\textnormal{\emph{Vorleſung … Welsberg}}}\Cendnote{\textnormal{Am 24. 8. 1901, siehe auch A. S.: \emph{Tagebuch}, 5. 12. 1921.
               }}}\label{K_L03475-9}, über die Stücke\pwindex{Lebendige Stunden. Vier Einakter@\emph{Lebendige Stunden. Vier Einakter}|pwv}
               lobend geäußert habe. Als ich ſie dann auf der Bühne\oindex{Deutsches Theater Berlin@\textbf{Deutsches Theater Berlin}, \emph{Theater (K.THE)}|pwv} ſah u. ihre Schwächen klar erkannte, habe ich dem Ausdruck\pwindex{Berliner Theater. (»Lebendige Stunden« von Arthur Schnitzler.)@\emph{Berliner Theater. (»Lebendige Stunden« von Arthur Schnitzler.)}|pwv} gegeben. Mein
               kritiſches Gewiſſen fühlt ſich durch dieſen »Widerſpruch« nicht im mindeſten
               belaſtet. Denn Stücke ſind nicht dazu da, im Walde vorgeleſen, ſondern aufgeführt zu
               werden; u. \strikeout{\textcolor{gray}{ein}} jedes vor der Aufführung abgegebene {\pb}Urteil über ein Stück kann immer nur ein Urteil mit Vorbehalt ſein. Wenn ich nach
               der Aufführung über die »Lebendigen Stunden\pwindex{Lebendige Stunden. Vier Einakter@\emph{Lebendige Stunden. Vier Einakter}|pw}«
               ungünſtig geurteilt haben würde u. die Stücke\pwindex{Lebendige Stunden. Vier Einakter@\emph{Lebendige Stunden. Vier Einakter}|pwv} wären doch gut, hätte ich als Kritiker gefehlt. Da ich
               die Stücke\pwindex{Lebendige Stunden. Vier Einakter@\emph{Lebendige Stunden. Vier Einakter}|pwv} aber nach wie vor
               nicht für gut halte (von manchen Qualitäten abgeſehen, welche die erſten haben, u.
               abgeſehen auch von dem ſehr hübſchen Einakter »Literatur\pwindex{Literatur@\emph{Literatur}|pw}«), da überdies ihr geringer Erfolg auf der Bühne \strikeout{mein} das in meiner Beſprechung\pwindex{Berliner Theater. (»Lebendige Stunden« von Arthur Schnitzler.)@\emph{Berliner Theater. (»Lebendige Stunden« von Arthur Schnitzler.)}|pwv} ausgeſprochene Urteil beſtätigt, {\pb}bin ich als Kritiker ſicher nicht im Unrecht; u.
               ich finde, daß es eine Lächerlichkeit iſt, gegen das öffentlich abgegebene Urteil\pwindex{Berliner Theater. (»Lebendige Stunden« von Arthur Schnitzler.)@\emph{Berliner Theater. (»Lebendige Stunden« von Arthur Schnitzler.)}|pwv} eines Kritikers, das er
               genau u. ſachlich begründet hat, Äußerungen ausſpielen zu wollen, die er nach einer
                  Vorleſung\pwindex{Lebendige Stunden. Vier Einakter@\emph{Lebendige Stunden. Vier Einakter}|pwv} im Walde\oindex{Welsberg-Taisten@\textbf{Welsberg-Taisten}, \emph{A.ADM3}|pwv} getan hat.\pend
           
\pstart
           Ich habe mein Gedächtnis weiter angeſtrengt u. kann mich an die Äußerung, die ich \substVorne{}\textsuperscript{\textcolor{gray}{we}iter}\substDazwischen{}außerdem\substHinten{} getan haben ſoll, daß ich nämlich bedaure, nicht ſelbſt ſolche Stücke\pwindex{Schleier der Beatrice. Schauspiel in fuenf Akten@\emph{Der Schleier der Beatrice. Schauspiel in fünf Akten}|pwv}\pwindex{Lebendige Stunden. Vier Einakter@\emph{Lebendige Stunden. Vier Einakter}|pwv} ſchreiben zu
               können, nicht mehr erinnern. Aber ich will nicht in Abrede ſtellen, ſie getan zu
               haben. {\pb}Warum ſollte ich auch nicht von Stücken\pwindex{Schleier der Beatrice. Schauspiel in fuenf Akten@\emph{Der Schleier der Beatrice. Schauspiel in fünf Akten}|pwv}\pwindex{Lebendige Stunden. Vier Einakter@\emph{Lebendige Stunden. Vier Einakter}|pwv}, die mir
               gefielen, geſagt haben, daß ich bedaure, ſie nicht auch ſchreiben zu können? Wenn
               aber weiter behauptet wird, ich hätte geſagt, ich möchte mich »erſchießen«, weil ich
               Solches nicht leiſten kann, ſo erkläre ich dies für eine \uline{Unwahrheit}. \strikeout{\textcolor{gray}{J}\textcolor{gray}{×}\-\textcolor{gray}{×}\-\textcolor{gray}{×}\-\textcolor{gray}{×}{ }\textcolor{gray}{Feſtſtellung dieſ}\textcolor{gray}{×}\-\textcolor{gray}{×}{ }\textcolor{gray}{×}\-\textcolor{gray}{×}\-\textcolor{gray}{×}\-\textcolor{gray}{×}\-\textcolor{gray}{×}\-\textcolor{gray}{×}\-\textcolor{gray}{×}{ }\textcolor{gray}{[2 Zeilen unleserlich{]} }.} Ich \uline{weiß}, daß ich das nicht geſagt haben kann u. auch
               nicht geſagt habe, weil ich weiß, daß ich mich nicht mit weibiſchem Schwulſt {\pb}auszudrücken pflege, ſondern die Gewohnheit habe,
               zu reden, wie ein Mann{\dotsseven}\pend
           
\pstart
           Lieber Freund, Du haſt mir auch bei unſerem letzten \label{K_L03475-10v}\edtext{Beiſammenſein}{\lemma{\textnormal{\emph{Beiſammenſein}}}\Cendnote{\textnormal{Am 28. 12. 1910,
                  siehe oben.}}}\label{K_L03475-10} wieder jede Fähigkeit zum Kritiker abgeſprochen. Dieſe Deine
               Anſicht über mich iſt mir ſeit Langem bekannt. Sie iſt für mich gewiß nicht
               belanglos. Denn ich habe nicht die Selbſtſicherheit, die Du beſitzeſt u. die Dich zu
               dem Ausſpuch veranlaßt, daß es {\pb}Dir gleichgiltig
               iſt, was \strikeout{die} »wir Andern« über Dich ſchreiben. Mir
               iſt es gar nicht gleichgiltig, was die Andern über mich ſchreiben oder ſagen. Wohl
               habe ich künſtleriſche \substVorne{}\textsuperscript{\textcolor{gray}{Weltanſchauungen}}\substDazwischen{}Anſchauungen\substHinten{}, von deren Richtigkeit ich unerſchütterlich überzeugt bin. Aber ich prüfe
               jedes noch ſo ungünſtige Urteil über mich, ob es nicht vielleicht doch etwas Wahres
               enthält, u. ſuche von jedem Andern, auch vom heftigſten Gegner, etwas zu lernen. Man
               muß ſchon ein mit Erfolg aufgeführter dramatiſcher Autor ſein, {\pb}um das Bewußtſein mit ſich herumzutragen, daß
               man von Anderen nichts mehr zu lernen habe. Bei ernſt ſtrebenden Menſchen in anderen
               Berufsarten wird man dieſes Bewußtſein kaum wiederfinden.\pend
           
\pstart
           Mir iſt es nicht gleichgiltig, was die Andern von mir ſagen, – u. ganz gewiß nicht
               gleichgiltig, was ein alter Freund von mir denkt. Aber mit Deiner Mißbilligung meiner
               Wirkſamkeit als Kritiker habe ich mich \introOben{}längſt\introOben{} abgefunden.
               Ich habe mir geſagt, daß Dein\strikeout{e} u. mein Lebensweg ſo
               weit auseinandergegangen ſind, {\pb}daß Deine u. meine
               Entwickelung eine ſo gänzlich verſchiedene Richtung eingeſchlagen haben, daß Du mich
               eben nicht mehr verſtehſt u. verſtehen kannſt. Du ſiehſt ja auch all’ das, worüber
               ich als Kritiker zu urteilen habe, von einem ganz anderen Standpunkt an, als ich. Du
               biſt ſelbſt beteiligt, biſt ſelbſt Partei. Meine künſtleriſchen Überzeugungen haben
               mich dazu geführt, Stellung gegen \strikeout{d\textcolor{gray}{ie}} die meiſten der dramatiſchen Autoren unſerer Generation, Stellung ſogar gegen
               manches Deiner Werke zu nehmen. {\pb}Wie darf ich da
               von Dir erwarten oder gar beanſpruchen, daß Du meine kritiſche Tätigkeit
               billigſt!\pend
           
\pstart
           Ich habe es Dir alſo niemals verargt, daß Du mich für einen ſchlechten Kritiker
               hältſt. Ich habe allerdings, wenn ich mit Dir ſprach u. von Dir ſo manche Anſchauung
               hörte, die ich für falſch halten muß, im Stillen Gott gedankt, daß ich nicht ein
               Kritiker geworden bin, den Du für gut halten würdeſt.\pend
           
\pstart
           {\pb}Deine Urteile über meine kritiſche Tätigkeit
               haben mich alſo nie von Dir abgeſtoßen; u. ich war feſt entſchloſſen, trotz alledem
                  \strikeout{\textcolor{gray}{Dir}} eine Freundſchaft zu erhalten, die nun ſchon mehr als zwanzig Jahre alt
                  iſt\strikeout{,} u. von der, ſo ſehr wir auch innerlich
               entfremdet ſind, doch ein enormes u. herzliches Gefühl für Dich bei mir
               zurückgeblieben iſt. \strikeout{\textcolor{gray}{×}}\pend
           
\pstart
           Nun aber haſt Du in unſerer letzten Unterredung im Hauſe\oindex{Frankgasse 1@\textbf{Frankgasse 1}, \emph{Wohngebäude (K.WHS)}|pwv} Deiner Mutter\pwindex{Schnitzler, Louise 1840-07-08 – 1911-09-09@\textsc{Schnitzler, Louise} (1840-07-08 – 1911-09-09)|pwv} in Deinen Angriffen gegen mich eine Grenze
               überſchritten, die Du {\pb}nicht überſchreiten
               durfteſt. Von meinen Fähigkeiten als Kritiker darfſt Du ſagen, was Du willſt. In
               dieſer Unterredung aber haſt Du es verſucht, meine Ehre anzutaſten. Und dieſen
               Verſuch muß ich mit der äußerſten Schärfe zurückweiſen. \strikeout{Die
                  Sprache \textcolor{gray}{n}u} Selbſt eine zwanzigjährige Freundſchaft gibt
               Dir nicht das Recht zu einer Sprache, \substVorne{}\textsuperscript{\textcolor{gray}{D}i\textcolor{gray}{e}}\substDazwischen{}die\substHinten{} Du in jener Unterredung Dir herausgenommen haſt, gegen mich zu führen. Das
               kann u. werde ich nicht {\pb}dulden! Und es iſt \strikeout{unehr} unerhört, es iſt eine der bitterſten Erfahrungen
               meines Lebens, daß ich, nachdem ich in einem ſchweren Lebenskampfe meine Ehre rein u.
               flankenlos erhalten habe, mich nun gegen den älteſten u. mir einſt nächſten Freund
               zur Wehr ſetzen \strikeout{will} muß, der meine Ehre \strikeout{\textcolor{gray}{bef}} beflecken will. An jener Unterredung, in der \strikeout{Du \textcolor{gray}{×}\-\textcolor{gray}{×}\-\textcolor{gray}{×}\-\textcolor{gray}{×} ich \textcolor{gray}{×}\-\textcolor{gray}{×}\-\textcolor{gray}{×}{ }\textcolor{gray}{×}\-\textcolor{gray}{×}\-\textcolor{gray}{×}} Du über mich, der ich als Gaſt im Hauſe\oindex{Frankgasse 1@\textbf{Frankgasse 1}, \emph{Wohngebäude (K.WHS)}|pwv} Deiner Mutter\pwindex{Schnitzler, Louise 1840-07-08 – 1911-09-09@\textsc{Schnitzler, Louise} (1840-07-08 – 1911-09-09)|pwv} weilte, {\pb}\strikeout{\textcolor{gray}{×}\-\textcolor{gray}{×}\-\textcolor{gray}{×}\-\textcolor{gray}{×}\-\textcolor{gray}{×}\-\textcolor{gray}{×}} hergefallen biſt, wie über einen charakterloſen Lumpen, denke ich zurück mit
               einer Miſchung von Scham, Widerwillen u. Empörung; u. ich konnte nicht Ruhe finden,
               ehe ich Dir dieſen Brief geſchrieben, um Deine Anwürfe von mir abzuſchütteln, –
               ſelbſt auf die Gefahr hin, daß dieſer Brief den Bruch unſerer zwanzigjährigen
               Freundſchaft herbeiführen ſollte.\pend
           
\pstart
           {\pb}Mit herzlichem Gruß {\\[\baselineskip]}Dein {\\[\baselineskip]}\spacefill\mbox{Paul Goldmann.}\pend
           \leftskip=0em{}\selectlanguage{ngerman}\endnumbering\briefempfaengerindex{Schnitzler, Arthur@\textsc{Schnitzler, Arthur}!zzzGoldmann, Paul@\emph{von Paul Goldmann}!1911-01-131@{13. 1. 1911}|)be}\mylabel{L03475h}  \normalsize

\doendnotes{C}
\bigskip
\vfill

\clearpage

\footnotesize

\lohead{\textsc{register}}

% Definiere theindex-Environment komplett neu ohne reledmac
\makeatletter
\renewenvironment{theindex}{%
  \section*{\indexname}%
  \setlength{\parindent}{0pt}%
  \setlength{\parskip}{0pt plus 0.3pt}%
  \let\item\@idxitem
}{%
  \clearpage
}
\makeatother

\IfFileExists{\jobname-pw.ind}{\input{\jobname-pw.ind}}{}

\end{document}

      