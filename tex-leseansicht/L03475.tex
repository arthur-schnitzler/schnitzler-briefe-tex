%% latex-leseansicht-vorspann.tex
%% Vorspann für die Leseansicht.
%% Lädt die gemeinsame Datei latex-vorspann.tex mit nicht gesetztem Schalter.

\newif\ifkorrekturansicht
\korrekturansichtfalse

\input{../tex-inputs/latex-vorspann}

\begin{center}
            \textcolor{red}{ENTWURF, NICHT FERTIG KORRIGIERT}
                      \end{center}
            
         
         \renewcommand{\erwaehntePersonen}{Personen: Franz Grillparzer, Louise Schnitzler}
         \renewcommand{\erwaehnteOrte}{Orte: Berlin, Deutsches Theater Berlin, Frankgasse, Schöneberger Ufer, Welsberg-Taisten, Wien}
         \renewcommand{\erwaehnteWerke}{Werke: Anatol, Aus dem dramatischen Irrgarten. Polemische Aufsätze über Berliner Theateraufführungen, Berliner Theater. (»Der Schleier der Beatrice« von Arthur Schnitzler.), Berliner Theater. (»Lebendige Stunden« von Arthur Schnitzler.), Berliner Theater. »Der Ruf des Lebens« von Arthur Schnitzler, Berliner Theater. »Der einsame Weg«. Von Arthur Schnitzler, Der Ruf des Lebens. Schauspiel in drei Akten, Der Schleier der Beatrice. Schauspiel in fünf Akten, Der einsame Weg. Schauspiel in fünf Akten, Die »neue Richtung«. Polemische Aufsätze über Berliner Theater-Aufführungen, Lebendige Stunden. Vier Einakter, Literatenstücke und Ausstattungsregie. Polemische Aufsätze über Berliner Theater-Aufführungen, Literatur, Neue Freie Presse, Vom Rückgang der deutschen Bühne. Polemische Aufsätze über Berliner Theater-Aufführungen}
               \section[ Paul Goldmann an Arthur Schnitzler, 13. 1. 1911]{ Paul Goldmann an Arthur Schnitzler, 13. 1. 1911}\nopagebreak\mylabel{v}\rehead{ }\begin{ledgroupsized}[t]{13cm}\normalsize\beginnumbering \toendnotes[C]{\smallbreak\pagebreak[2]} \Standort{DLA, A:Schnitzler, HS.NZ85.1.3176.}
\physDesc{Brief, 7 Blätter, 26 Seiten, 12013 Zeichen
\newline{}Handschrift: blaue Tinte, deutsche Kurrent}\toendnotes[C]{\smallbreak}\pstart
           \noindent{}\raggedleft{}{\pb}\textcolor{gray}{\textbf{W. SCHÖNEBERGER-UFER 34\oindex{Schoeneberger Ufer@\textbf{Schöneberger Ufer}|pw}.}}\pend
           \pstart
           13. 1. 1\substVorne{}\textsuperscript{0}\substDazwischen{}1\substHinten{}.\pend
           \pstart\center{}Lieber Freund,\pend\pstart
           Die \label{K_L03475-1v}\edtext{Überſendung der Kopien meiner
                  Briefe}{\lemma{\textnormal{\emph{Überſendung … Briefe}}}\Cendnote{\textnormal{siehe Paul Goldmann an Arthur Schnitzler, 30. 12. 1910}}}\label{K_L03475-1h} habe ich mit einiger Sorge erwartet. Denn in jener Zeit, in der dieſe
               Angelegenheit ſpielt, war mir die Freundſchaft mit Dir ſehr viel, bildete ſie eines
               der großen Beſitztümer meines Lebens. Und ich fragte mich, ehe ich die Kopien erhielt\substVorne{}\textsuperscript{,}\substDazwischen{}:\substHinten{} ſollte ich nicht vielleicht, in der Sorge, dieſes Freundſchafts-Beſitztum
               vor jeder Gehahr zu behüten, mich ſchwach gezeigt haben?\pend
           \pstart
           Als ich die Copien las, war ich ſtarr vor Staunen. Das alſo waren {\pb}die \label{K_L03475-2v}\edtext{»Beweisſtücke«}{\lemma{\textnormal{\emph{»Beweisſtücke«}}}\Cendnote{\textnormal{Bezug auf die Auseinandersetzungen am 26. 12. 1910 und vor allem am 28. 12. 1910, siehe Paul Goldmann an Arthur Schnitzler, 26. 12. 1910}}}\label{K_L03475-2h} gegen mich! Dies die Dokumente gegen meine Ehre! Denn es iſt Dir ſicherlich
               nicht klar geworden, daß es ſich in alledem um meine Ehre handelt, – daß Du meine
               Ehre angreifſt, indem Du mich als einen Menſchen hinſtellſt, der heimlich lobt u. öffentlich\pwindex{Goldmann, Paul 31.01.1865 – 25.09.1935@\textsc{Goldmann, Paul} (31.01.1865 – 25.09.1935), \emph{Schriftsteller, Journalist}!Berliner Theater. (»Der Schleier der Beatrice« von Arthur Schnitzler.)1903-03-19@\strich\emph{Berliner Theater. (»Der Schleier der Beatrice« von Arthur Schnitzler.)} {[}1903-03-19{]}|pwv} tadelt, der in
               ſeinen Briefen dem Freunde ſchmeichelt, u. ihn dann öffentlich – noch dazu, wie Du
               weißt, mit einem beſonderen Vergnügen – herunterreißt.\pend
           \pstart
           Das alſo waren die Dokumente! Ich las die Briefe u. fand, daß ich darin mit aller
               Deutlichkeit ſtarke Bedenken gegen Dein Werk\pwindex{Schnitzler, Arthur 15.05.1862 – 21.10.1931@\textsc{Schnitzler, Arthur} (15.05.1862 – 21.10.1931), \emph{Schriftsteller, Mediziner}!Schleier der Beatrice. Schauspiel in fuenf Akten1900-12-01@\strich\emph{Der Schleier der Beatrice. Schauspiel in fünf Akten} {[}1900-12-01{]}|pwv} formulirt hatte, – mit aller Deutlichkeit für Jedermann
               außer für {\pb}den durch Größengefühl und
               Selbſtgefälligkeit jeden Urteils beraubten Autor. Jeder ruhig u. objektiv Urteilende
               wird auch finden, daß meine ſpätere öffentliche \label{K_L03475-4v}\edtext{Kritik\pwindex{Goldmann, Paul 31.01.1865 – 25.09.1935@\textsc{Goldmann, Paul} (31.01.1865 – 25.09.1935), \emph{Schriftsteller, Journalist}!Berliner Theater. (»Der Schleier der Beatrice« von Arthur Schnitzler.)1903-03-19@\strich\emph{Berliner Theater. (»Der Schleier der Beatrice« von Arthur Schnitzler.)} {[}1903-03-19{]}|pwv}}{\lemma{\textnormal{\emph{Kritik}}}\Cendnote{\textnormal{Paul Goldmann\pwindex{Goldmann, Paul 31.01.1865 – 25.09.1935@\textsc{Goldmann, Paul} (31.01.1865 – 25.09.1935), \emph{Schriftsteller, Journalist}|pwk}: \emph{Berliner Theater. (»Der Schleier der Beatrice« von Arthur
                        Schnitzler.)}\pwindex{Goldmann, Paul 31.01.1865 – 25.09.1935@\textsc{Goldmann, Paul} (31.01.1865 – 25.09.1935), \emph{Schriftsteller, Journalist}!Berliner Theater. (»Der Schleier der Beatrice« von Arthur Schnitzler.)1903-03-19@\strich\emph{Berliner Theater. (»Der Schleier der Beatrice« von Arthur Schnitzler.)} {[}1903-03-19{]}|pwk}. In: \emph{Neue Freie
                        Presse}\pwindex{Neue Freie Presse1864 – 1939@\emph{Neue Freie Presse} {[}1864 – 1939{]}|pwk}, Nr. 13.851, 19. 3. 1903,
                     Morgenblatt, S. 1–5.}}}\label{K_L03475-4h} nichts iſt als die Ausführung der in den
               Briefen bereits kurz formulirten Bedenken. Jeder ruhig u. objektiv Urteilende wird
               weiter finden, daß in Dieſen Briefen ein Freund dem Freund die Wahrheit ſagt, \strikeout{\textcolor{gray}{×}\-\textcolor{gray}{×}\-\textcolor{gray}{×}} daß der Freund aber gleichzeitig beſtrebt iſt, dem Freunde nicht wehzutun, u.
               daß er darum, damit der Tadel, den er auſzuſprechen ſich genötigt ſieht, nur ja nicht
               verletze \strikeout{verletz\textcolor{gray}{e}} verletze, \strikeout{\textcolor{gray}{d}\textcolor{gray}{×}\-\textcolor{gray}{×}} das Lob, das er ſpenden kann, in möglichſt ſtarken Ausdrücken formulirt. {\pb}Di\substVorne{}\textsuperscript{r}\substDazwischen{}e\substHinten{} großen Fehler, unter den\substVorne{}\textsuperscript{,}\substDazwischen{}e\substHinten{}n, meiner Anſicht nach, Dein Stück\pwindex{Schnitzler, Arthur 15.05.1862 – 21.10.1931@\textsc{Schnitzler, Arthur} (15.05.1862 – 21.10.1931), \emph{Schriftsteller, Mediziner}!Schleier der Beatrice. Schauspiel in fuenf Akten1900-12-01@\strich\emph{Der Schleier der Beatrice. Schauspiel in fünf Akten} {[}1900-12-01{]}|pwv} leidet, ſind in meinen Briefen klar gekennzeichnet. Du
               haſt darüber hinweggeleſen u. von meinen Briefen nur behalten, daß ich Dich mit \textsc{Grillparzer\pwindex{Grillparzer, Franz 15.01.1791 – 21.01.1872@\textsc{Grillparzer, Franz} (15.01.1791 – 21.01.1872), \emph{Schriftsteller, Beamter}|pw}} verglichen habe. Das iſt bezeichnend – aber nicht für mich, ſondern für
               Dich.\pend
           \pstart
           In meinen Briefen habe ich Dich gelobt. Und in meiner Kritik\pwindex{Goldmann, Paul 31.01.1865 – 25.09.1935@\textsc{Goldmann, Paul} (31.01.1865 – 25.09.1935), \emph{Schriftsteller, Journalist}!Berliner Theater. (»Der Schleier der Beatrice« von Arthur Schnitzler.)1903-03-19@\strich\emph{Berliner Theater. (»Der Schleier der Beatrice« von Arthur Schnitzler.)} {[}1903-03-19{]}|pwv}? In meinen Briefen ſteht: \label{K_L03475-3v}\edtext{»Seit \textsc{Grillparzer\pwindex{Grillparzer, Franz 15.01.1791 – 21.01.1872@\textsc{Grillparzer, Franz} (15.01.1791 – 21.01.1872), \emph{Schriftsteller, Beamter}|pw}} hat man auf dem Wien\oindex{Wien@\textbf{Wien}|pw}er Theater ſolche Verſe\pwindex{Schnitzler, Arthur 15.05.1862 – 21.10.1931@\textsc{Schnitzler, Arthur} (15.05.1862 – 21.10.1931), \emph{Schriftsteller, Mediziner}!Schleier der Beatrice. Schauspiel in fuenf Akten1900-12-01@\strich\emph{Der Schleier der Beatrice. Schauspiel in fünf Akten} {[}1900-12-01{]}|pwv} nicht gehört.«}{\lemma{\textnormal{\emph{»Seit … gehört.«}}}\Cendnote{\textnormal{vgl. Paul Goldmann an Arthur Schnitzler, 20. 2. 1900}}}\label{K_L03475-3h} In meiner Kritik\pwindex{Goldmann, Paul 31.01.1865 – 25.09.1935@\textsc{Goldmann, Paul} (31.01.1865 – 25.09.1935), \emph{Schriftsteller, Journalist}!Berliner Theater. (»Der Schleier der Beatrice« von Arthur Schnitzler.)1903-03-19@\strich\emph{Berliner Theater. (»Der Schleier der Beatrice« von Arthur Schnitzler.)} {[}1903-03-19{]}|pwv}:
                  »In der Form wenigſtens zeigt
                     \textsc{Schnitzler} ſich als ein \uline{würdiger Schüler der Meiſter (der Klaſſiker)}, denen er {\pb}nacheifert. Daß \textsc{Schnitzler} dieſe Form ſich anzueignen vermochte, deutet auf eine
                  künſtleriſche Selbſterziehung hin \strikeout{hi\textcolor{gray}{n}}, die man bei den deutſchen Autoren der Gegenwart ſelten findet; es iſt ein
                  weiter, mühevoller, ehrenvoller Weg vom »\textsc{Anatol}\pwindex{Schnitzler, Arthur 15.05.1862 – 21.10.1931@\textsc{Schnitzler, Arthur} (15.05.1862 – 21.10.1931), \emph{Schriftsteller, Mediziner}!Anatol1892-10-29@\strich\emph{Anatol} {[}1892-10-29{]}|pw}« bis zum »Schleier der Beatrice\pwindex{Schnitzler, Arthur 15.05.1862 – 21.10.1931@\textsc{Schnitzler, Arthur} (15.05.1862 – 21.10.1931), \emph{Schriftsteller, Mediziner}!Schleier der Beatrice. Schauspiel in fuenf Akten1900-12-01@\strich\emph{Der Schleier der Beatrice. Schauspiel in fünf Akten} {[}1900-12-01{]}|pw}«{[}.{]} Das Drama\pwindex{Schnitzler, Arthur 15.05.1862 – 21.10.1931@\textsc{Schnitzler, Arthur} (15.05.1862 – 21.10.1931), \emph{Schriftsteller, Mediziner}!Schleier der Beatrice. Schauspiel in fuenf Akten1900-12-01@\strich\emph{Der Schleier der Beatrice. Schauspiel in fünf Akten} {[}1900-12-01{]}|pwv} ſpricht namentlich in ſeinen
                  Verſen – wohllautenden Verſen von Wien\oindex{Wien@\textbf{Wien}|pw}eriſcher
                  Weichheit – eine vornehme Sprache.\pwindex{Goldmann, Paul 31.01.1865 – 25.09.1935@\textsc{Goldmann, Paul} (31.01.1865 – 25.09.1935), \emph{Schriftsteller, Journalist}!Berliner Theater. (»Der Schleier der Beatrice« von Arthur Schnitzler.)1903-03-19@\strich\emph{Berliner Theater. (»Der Schleier der Beatrice« von Arthur Schnitzler.)} {[}1903-03-19{]}|pwv}« An einer anderen Stelle\pwindex{Goldmann, Paul 31.01.1865 – 25.09.1935@\textsc{Goldmann, Paul} (31.01.1865 – 25.09.1935), \emph{Schriftsteller, Journalist}!Berliner Theater. (»Der Schleier der Beatrice« von Arthur Schnitzler.)1903-03-19@\strich\emph{Berliner Theater. (»Der Schleier der Beatrice« von Arthur Schnitzler.)} {[}1903-03-19{]}|pwv} wird von »prächtigen Verſen\pwindex{Goldmann, Paul 31.01.1865 – 25.09.1935@\textsc{Goldmann, Paul} (31.01.1865 – 25.09.1935), \emph{Schriftsteller, Journalist}!Berliner Theater. (»Der Schleier der Beatrice« von Arthur Schnitzler.)1903-03-19@\strich\emph{Berliner Theater. (»Der Schleier der Beatrice« von Arthur Schnitzler.)} {[}1903-03-19{]}|pwv}« geſprochen, die dann
               citirt werden. Von \textsc{Beatrice\pwindex{Schnitzler, Arthur 15.05.1862 – 21.10.1931@\textsc{Schnitzler, Arthur} (15.05.1862 – 21.10.1931), \emph{Schriftsteller, Mediziner}!Schleier der Beatrice. Schauspiel in fuenf Akten1900-12-01@\strich\emph{Der Schleier der Beatrice. Schauspiel in fünf Akten} {[}1900-12-01{]}|pwv}} wird geſagt, daß ſie »ein
                  liebliches {\pb}Geſchöpf iſt, eine echt \textsc{Schnitzlerische} Mädchengeſtalt, von poetiſchem Schimmer
                  umfloſſen\pwindex{Goldmann, Paul 31.01.1865 – 25.09.1935@\textsc{Goldmann, Paul} (31.01.1865 – 25.09.1935), \emph{Schriftsteller, Journalist}!Berliner Theater. (»Der Schleier der Beatrice« von Arthur Schnitzler.)1903-03-19@\strich\emph{Berliner Theater. (»Der Schleier der Beatrice« von Arthur Schnitzler.)} {[}1903-03-19{]}|pwv}«. Von einer Scene\pwindex{Schnitzler, Arthur 15.05.1862 – 21.10.1931@\textsc{Schnitzler, Arthur} (15.05.1862 – 21.10.1931), \emph{Schriftsteller, Mediziner}!Schleier der Beatrice. Schauspiel in fuenf Akten1900-12-01@\strich\emph{Der Schleier der Beatrice. Schauspiel in fünf Akten} {[}1900-12-01{]}|pwv} wird geſagt, daß ſie »die bedeutendſte des Stück\pwindex{Schnitzler, Arthur 15.05.1862 – 21.10.1931@\textsc{Schnitzler, Arthur} (15.05.1862 – 21.10.1931), \emph{Schriftsteller, Mediziner}!Schleier der Beatrice. Schauspiel in fuenf Akten1900-12-01@\strich\emph{Der Schleier der Beatrice. Schauspiel in fünf Akten} {[}1900-12-01{]}|pwv}es iſt u. \textsc{Schnitzlers}
                  dramatiſche Begabung im hellſten Lichte zeigt\pwindex{Goldmann, Paul 31.01.1865 – 25.09.1935@\textsc{Goldmann, Paul} (31.01.1865 – 25.09.1935), \emph{Schriftsteller, Journalist}!Berliner Theater. (»Der Schleier der Beatrice« von Arthur Schnitzler.)1903-03-19@\strich\emph{Berliner Theater. (»Der Schleier der Beatrice« von Arthur Schnitzler.)} {[}1903-03-19{]}|pwv}« \textsc{etc}.\pend
           \pstart
           Und von dieſer Kritik\pwindex{Goldmann, Paul 31.01.1865 – 25.09.1935@\textsc{Goldmann, Paul} (31.01.1865 – 25.09.1935), \emph{Schriftsteller, Journalist}!Berliner Theater. (»Der Schleier der Beatrice« von Arthur Schnitzler.)1903-03-19@\strich\emph{Berliner Theater. (»Der Schleier der Beatrice« von Arthur Schnitzler.)} {[}1903-03-19{]}|pwv} wagſt
               Du zu behaupten, daß ſie \strikeout{doch} Dein Werk\pwindex{Schnitzler, Arthur 15.05.1862 – 21.10.1931@\textsc{Schnitzler, Arthur} (15.05.1862 – 21.10.1931), \emph{Schriftsteller, Mediziner}!Schleier der Beatrice. Schauspiel in fuenf Akten1900-12-01@\strich\emph{Der Schleier der Beatrice. Schauspiel in fünf Akten} {[}1900-12-01{]}|pwv} verreißt, während meine Briefe es
               gelobt haben? Ich muß noch die Einſchränkung machen, daß die lobenden Ausdrücke in
               meinen Briefen \strikeout{\textcolor{gray}{×}\-\textcolor{gray}{×}\-\textcolor{gray}{×}\-\textcolor{gray}{×}\-\textcolor{gray}{×}\-\textcolor{gray}{×}} ſtärker klingen, als in der Kritik\pwindex{Goldmann, Paul 31.01.1865 – 25.09.1935@\textsc{Goldmann, Paul} (31.01.1865 – 25.09.1935), \emph{Schriftsteller, Journalist}!Berliner Theater. (»Der Schleier der Beatrice« von Arthur Schnitzler.)1903-03-19@\strich\emph{Berliner Theater. (»Der Schleier der Beatrice« von Arthur Schnitzler.)} {[}1903-03-19{]}|pwv}. \strikeout{\textcolor{gray}{×}} Einen Grund dafür – das Beſtreben des Freundes, mit möglichſt viel {\pb}Lob den Tadel, den er ausſpricht, weniger
               empfindlich zu machen – habe ich ſchon angeführt. Ein anderer Grund iſt der, daß man
               in einem Privatbrief ſeine Ausdrücke nicht ſo vorſichtig abwägt, wie man dies tut,
               wenn man in der Ausübung ſeines kritiſchen Berufes, \strikeout{\textcolor{gray}{×}\-\textcolor{gray}{×}\-\textcolor{gray}{×}} in dem Bewußtſein, daß man für jedes Wort die volle Verantwortung zu
               übernehmen hat, \strikeout{\textcolor{gray}{vor}} öffentlich ſich äußert\pwindex{Goldmann, Paul 31.01.1865 – 25.09.1935@\textsc{Goldmann, Paul} (31.01.1865 – 25.09.1935), \emph{Schriftsteller, Journalist}!Berliner Theater. (»Der Schleier der Beatrice« von Arthur Schnitzler.)1903-03-19@\strich\emph{Berliner Theater. (»Der Schleier der Beatrice« von Arthur Schnitzler.)} {[}1903-03-19{]}|pwv}. Entſteht aus dieſem Grunde ein Widerſpruch zwiſchen Privatbriefen des
               Kritikers u. der von ihm {\pb}veröffentlichten Kritik\pwindex{Goldmann, Paul 31.01.1865 – 25.09.1935@\textsc{Goldmann, Paul} (31.01.1865 – 25.09.1935), \emph{Schriftsteller, Journalist}!Berliner Theater. (»Der Schleier der Beatrice« von Arthur Schnitzler.)1903-03-19@\strich\emph{Berliner Theater. (»Der Schleier der Beatrice« von Arthur Schnitzler.)} {[}1903-03-19{]}|pwv}, ſo trifft die
               Verantwortung nicht den Kritiker, ſondern den, der es verſucht, \strikeout{P} deſſen Privatbriefe gegen ihn auszuſpielen.\pend
           \pstart
           Im Übrigen aber habe ich angeſichts der Briefkopien u. der Kritik\pwindex{Goldmann, Paul 31.01.1865 – 25.09.1935@\textsc{Goldmann, Paul} (31.01.1865 – 25.09.1935), \emph{Schriftsteller, Journalist}!Berliner Theater. (»Der Schleier der Beatrice« von Arthur Schnitzler.)1903-03-19@\strich\emph{Berliner Theater. (»Der Schleier der Beatrice« von Arthur Schnitzler.)} {[}1903-03-19{]}|pwv}, die beide hier vor mir liegen, mit
               aller Entſchiedenheit zu erklären: Die Briefe loben nicht nur das Stück\pwindex{Schnitzler, Arthur 15.05.1862 – 21.10.1931@\textsc{Schnitzler, Arthur} (15.05.1862 – 21.10.1931), \emph{Schriftsteller, Mediziner}!Schleier der Beatrice. Schauspiel in fuenf Akten1900-12-01@\strich\emph{Der Schleier der Beatrice. Schauspiel in fünf Akten} {[}1900-12-01{]}|pwv}, ſondern ſie ſprechen auch bereits
               die \label{K_L03475-5v}\edtext{Einwendungen}{\lemma{\textnormal{\emph{Einwendungen}}}\Cendnote{\textnormal{Siehe insbesondere die Briefe Goldmann\pwindex{Goldmann, Paul 31.01.1865 – 25.09.1935@\textsc{Goldmann, Paul} (31.01.1865 – 25.09.1935), \emph{Schriftsteller, Journalist}|pwk}s an Schnitzler\pwindex{Schnitzler, Arthur 15.05.1862 – 21.10.1931@\textsc{Schnitzler, Arthur} (15.05.1862 – 21.10.1931), \emph{Schriftsteller, Mediziner}|pwk} vom 11. 2. 1900, 25. 1. [1902] und 17. 3. [1903].}}}\label{K_L03475-5h} aus, die, meiner Anſicht nach,
               dagegen zu erheben ſind. Die Kritik\pwindex{Goldmann, Paul 31.01.1865 – 25.09.1935@\textsc{Goldmann, Paul} (31.01.1865 – 25.09.1935), \emph{Schriftsteller, Journalist}!Berliner Theater. (»Der Schleier der Beatrice« von Arthur Schnitzler.)1903-03-19@\strich\emph{Berliner Theater. (»Der Schleier der Beatrice« von Arthur Schnitzler.)} {[}1903-03-19{]}|pwv} tadelt nicht nur das \strikeout{S\textcolor{gray}{t}}{ }Stück\pwindex{Schnitzler, Arthur 15.05.1862 – 21.10.1931@\textsc{Schnitzler, Arthur} (15.05.1862 – 21.10.1931), \emph{Schriftsteller, Mediziner}!Schleier der Beatrice. Schauspiel in fuenf Akten1900-12-01@\strich\emph{Der Schleier der Beatrice. Schauspiel in fünf Akten} {[}1900-12-01{]}|pwv}, ſondern {\pb}läßt ihm auch alle jene Anerkennung zuteil werden,
               die es\pwindex{Schnitzler, Arthur 15.05.1862 – 21.10.1931@\textsc{Schnitzler, Arthur} (15.05.1862 – 21.10.1931), \emph{Schriftsteller, Mediziner}!Schleier der Beatrice. Schauspiel in fuenf Akten1900-12-01@\strich\emph{Der Schleier der Beatrice. Schauspiel in fünf Akten} {[}1900-12-01{]}|pwv}, meiner Anſicht nach,
               verdient. Es beſteht höchſtens in der \textsc{Nuance} einiger
               Ausdrücke, aber im Weſen kein Widerſpruch zwiſchen den Briefen u. der Kritik\pwindex{Goldmann, Paul 31.01.1865 – 25.09.1935@\textsc{Goldmann, Paul} (31.01.1865 – 25.09.1935), \emph{Schriftsteller, Journalist}!Berliner Theater. (»Der Schleier der Beatrice« von Arthur Schnitzler.)1903-03-19@\strich\emph{Berliner Theater. (»Der Schleier der Beatrice« von Arthur Schnitzler.)} {[}1903-03-19{]}|pwv}. Und den Vorwurf, den \strikeout{g} Du gegen mich erhoben haſt, daß ich als Freund wie
               als Kritiker meine Pflicht gegen Dich vergeſſen habe, weiſe ich mit Entrüſtung
                  zurück{\dotsfive}\pend
           \pstart
           Ich komme jetzt zum {\pb}zweiten Fall, dem Fall der
                  »Lebendigen Stunden\pwindex{Schnitzler, Arthur 15.05.1862 – 21.10.1931@\textsc{Schnitzler, Arthur} (15.05.1862 – 21.10.1931), \emph{Schriftsteller, Mediziner}!Lebendige Stunden. Vier Einakter1901-12-23@\strich\emph{Lebendige Stunden. Vier Einakter} {[}1901-12-23{]}|pw}«. Hier liegen leider
               keine Dokumente vor, kein \label{K_L03475-12v}\edtext{Brief}{\lemma{\textnormal{\emph{Brief}}}\Cendnote{\textnormal{Goldmann\pwindex{Goldmann, Paul 31.01.1865 – 25.09.1935@\textsc{Goldmann, Paul} (31.01.1865 – 25.09.1935), \emph{Schriftsteller, Journalist}|pwk} hatte tatsächlich nur selten
                  brieflich Kritik an \emph{Lebendige Stunden}\pwindex{Schnitzler, Arthur 15.05.1862 – 21.10.1931@\textsc{Schnitzler, Arthur} (15.05.1862 – 21.10.1931), \emph{Schriftsteller, Mediziner}!Lebendige Stunden. Vier Einakter1901-12-23@\strich\emph{Lebendige Stunden. Vier Einakter} {[}1901-12-23{]}|pwk} geübt,
                     siehe Paul Goldmann an Arthur Schnitzler und Olga
               Gussmann, 23. 12. [1901] und 25. 1. [1902].}}}\label{K_L03475-12h}, von dem
               Du Kopien hätteſt machen können. Hier handelt es ſich um mündliche Äußerungen, die
               ich getan haben ſoll. Würden ſie im genauen, beglaubigten Wortlaut vorliegen, ſo
               würden ſich die »Widersprüche« zwiſchen dieſen Äußerungen u. meiner ſpäter
               veröffentlichten \label{K_L03475-8v}\edtext{Kritik\pwindex{Goldmann, Paul 31.01.1865 – 25.09.1935@\textsc{Goldmann, Paul} (31.01.1865 – 25.09.1935), \emph{Schriftsteller, Journalist}!Berliner Theater. (»Lebendige Stunden« von Arthur Schnitzler.)1902-01-22@\strich\emph{Berliner Theater. (»Lebendige Stunden« von Arthur Schnitzler.)} {[}1902-01-22{]}|pwv}}{\lemma{\textnormal{\emph{Kritik}}}\Cendnote{\textnormal{Paul Goldmann\pwindex{Goldmann, Paul 31.01.1865 – 25.09.1935@\textsc{Goldmann, Paul} (31.01.1865 – 25.09.1935), \emph{Schriftsteller, Journalist}|pwk}: \emph{Berliner Theater. (»Lebendige Stunden« von Arthur
                        Schnitzler.)}\pwindex{Goldmann, Paul 31.01.1865 – 25.09.1935@\textsc{Goldmann, Paul} (31.01.1865 – 25.09.1935), \emph{Schriftsteller, Journalist}!Berliner Theater. (»Lebendige Stunden« von Arthur Schnitzler.)1902-01-22@\strich\emph{Berliner Theater. (»Lebendige Stunden« von Arthur Schnitzler.)} {[}1902-01-22{]}|pwk}. In: \emph{Neue Freie
                        Presse}\pwindex{Neue Freie Presse1864 – 1939@\emph{Neue Freie Presse} {[}1864 – 1939{]}|pwk}, Nr. 13.438, 22. 1. 1902,
                     Morgenblatt, S. 1–4.}}}\label{K_L03475-8h} wahrſcheinlich ebenſo aufklären, wie im
               Falle der »\textsc{Beatrice\pwindex{Schnitzler, Arthur 15.05.1862 – 21.10.1931@\textsc{Schnitzler, Arthur} (15.05.1862 – 21.10.1931), \emph{Schriftsteller, Mediziner}!Schleier der Beatrice. Schauspiel in fuenf Akten1900-12-01@\strich\emph{Der Schleier der Beatrice. Schauspiel in fünf Akten} {[}1900-12-01{]}|pw}}«. {\pb}Möglicherweiſe habe ich auch hier
               Einwendungen formulirt, über die Du hinweggehört haſt, wie Du über die gegen die »\textsc{Beatrice\pwindex{Schnitzler, Arthur 15.05.1862 – 21.10.1931@\textsc{Schnitzler, Arthur} (15.05.1862 – 21.10.1931), \emph{Schriftsteller, Mediziner}!Schleier der Beatrice. Schauspiel in fuenf Akten1900-12-01@\strich\emph{Der Schleier der Beatrice. Schauspiel in fünf Akten} {[}1900-12-01{]}|pw}}« in meinen Briefen hinweggeleſen haſt. Ich habe nicht einmal meine Kritik\pwindex{Goldmann, Paul 31.01.1865 – 25.09.1935@\textsc{Goldmann, Paul} (31.01.1865 – 25.09.1935), \emph{Schriftsteller, Journalist}!Berliner Theater. (»Lebendige Stunden« von Arthur Schnitzler.)1902-01-22@\strich\emph{Berliner Theater. (»Lebendige Stunden« von Arthur Schnitzler.)} {[}1902-01-22{]}|pwv} über die »Lebendigen Stunden\pwindex{Schnitzler, Arthur 15.05.1862 – 21.10.1931@\textsc{Schnitzler, Arthur} (15.05.1862 – 21.10.1931), \emph{Schriftsteller, Mediziner}!Lebendige Stunden. Vier Einakter1901-12-23@\strich\emph{Lebendige Stunden. Vier Einakter} {[}1901-12-23{]}|pw}« zur Hand u. kann daher nicht
               konſtatiren, ob ſie wirklich ſo ohne jede Einſchränkung tadelnd war, wie Du
               behaupteſt. Denn ich habe dieſe \strikeout{Kr\textcolor{gray}{i}}{ }Beſprechung\pwindex{Goldmann, Paul 31.01.1865 – 25.09.1935@\textsc{Goldmann, Paul} (31.01.1865 – 25.09.1935), \emph{Schriftsteller, Journalist}!Berliner Theater. (»Lebendige Stunden« von Arthur Schnitzler.)1902-01-22@\strich\emph{Berliner Theater. (»Lebendige Stunden« von Arthur Schnitzler.)} {[}1902-01-22{]}|pwv} in die \label{K_L03475-11v}\edtext{Sammlungen meiner Kritiken\pwindex{Goldmann, Paul 31.01.1865 – 25.09.1935@\textsc{Goldmann, Paul} (31.01.1865 – 25.09.1935), \emph{Schriftsteller, Journalist}!»neue Richtung«. Polemische Aufsaetze ueber Berliner
                  Theater-Auffuehrungen1902-10-17@\strich\emph{Die »neue Richtung«. Polemische Aufsätze über Berliner Theater-Aufführungen} {[}1902-10-17{]}|pwv}\pwindex{Goldmann, Paul 31.01.1865 – 25.09.1935@\textsc{Goldmann, Paul} (31.01.1865 – 25.09.1935), \emph{Schriftsteller, Journalist}!Aus dem dramatischen Irrgarten. Polemische Aufsaetze ueber Berliner
                  Theaterauffuehrungen1905@\strich\emph{Aus dem dramatischen Irrgarten. Polemische Aufsätze über Berliner Theateraufführungen} {[}1905{]}|pwv}\pwindex{Goldmann, Paul 31.01.1865 – 25.09.1935@\textsc{Goldmann, Paul} (31.01.1865 – 25.09.1935), \emph{Schriftsteller, Journalist}!Vom Rueckgang der deutschen Buehne. Polemische Aufsaetze ueber Berliner
                  Theater-Auffuehrungen1908@\strich\emph{Vom Rückgang der deutschen Bühne. Polemische Aufsätze über Berliner Theater-Aufführungen} {[}1908{]}|pwv}\pwindex{Goldmann, Paul 31.01.1865 – 25.09.1935@\textsc{Goldmann, Paul} (31.01.1865 – 25.09.1935), \emph{Schriftsteller, Journalist}!Literatenstuecke und Ausstattungsregie. Polemische Aufsaetze ueber Berliner
                  Theater-Auffuehrungen1910-11-12@\strich\emph{Literatenstücke und Ausstattungsregie. Polemische Aufsätze über Berliner Theater-Aufführungen} {[}1910-11-12{]}|pwv}}{\lemma{\textnormal{\emph{Sammlungen … Kritiken}}}\Cendnote{\textnormal{Goldmann\pwindex{Goldmann, Paul 31.01.1865 – 25.09.1935@\textsc{Goldmann, Paul} (31.01.1865 – 25.09.1935), \emph{Schriftsteller, Journalist}|pwk} hatte bereits mehrere Kritiksammlungen\pwindex{Goldmann, Paul 31.01.1865 – 25.09.1935@\textsc{Goldmann, Paul} (31.01.1865 – 25.09.1935), \emph{Schriftsteller, Journalist}!»neue Richtung«. Polemische Aufsaetze ueber Berliner
                  Theater-Auffuehrungen1902-10-17@\strich\emph{Die »neue Richtung«. Polemische Aufsätze über Berliner Theater-Aufführungen} {[}1902-10-17{]}|pwkv}\pwindex{Goldmann, Paul 31.01.1865 – 25.09.1935@\textsc{Goldmann, Paul} (31.01.1865 – 25.09.1935), \emph{Schriftsteller, Journalist}!Aus dem dramatischen Irrgarten. Polemische Aufsaetze ueber Berliner
                  Theaterauffuehrungen1905@\strich\emph{Aus dem dramatischen Irrgarten. Polemische Aufsätze über Berliner Theateraufführungen} {[}1905{]}|pwkv}\pwindex{Goldmann, Paul 31.01.1865 – 25.09.1935@\textsc{Goldmann, Paul} (31.01.1865 – 25.09.1935), \emph{Schriftsteller, Journalist}!Vom Rueckgang der deutschen Buehne. Polemische Aufsaetze ueber Berliner
                  Theater-Auffuehrungen1908@\strich\emph{Vom Rückgang der deutschen Bühne. Polemische Aufsätze über Berliner Theater-Aufführungen} {[}1908{]}|pwkv}\pwindex{Goldmann, Paul 31.01.1865 – 25.09.1935@\textsc{Goldmann, Paul} (31.01.1865 – 25.09.1935), \emph{Schriftsteller, Journalist}!Literatenstuecke und Ausstattungsregie. Polemische Aufsaetze ueber Berliner
                  Theater-Auffuehrungen1910-11-12@\strich\emph{Literatenstücke und Ausstattungsregie. Polemische Aufsätze über Berliner Theater-Aufführungen} {[}1910-11-12{]}|pwkv} veröffentlicht (\emph{Die
                     »neue Richtung«}\pwindex{Goldmann, Paul 31.01.1865 – 25.09.1935@\textsc{Goldmann, Paul} (31.01.1865 – 25.09.1935), \emph{Schriftsteller, Journalist}!»neue Richtung«. Polemische Aufsaetze ueber Berliner
                  Theater-Auffuehrungen1902-10-17@\strich\emph{Die »neue Richtung«. Polemische Aufsätze über Berliner Theater-Aufführungen} {[}1902-10-17{]}|pwk}, 1903, \emph{Aus dem dramatischen Irrgarten}\pwindex{Goldmann, Paul 31.01.1865 – 25.09.1935@\textsc{Goldmann, Paul} (31.01.1865 – 25.09.1935), \emph{Schriftsteller, Journalist}!Aus dem dramatischen Irrgarten. Polemische Aufsaetze ueber Berliner
                  Theaterauffuehrungen1905@\strich\emph{Aus dem dramatischen Irrgarten. Polemische Aufsätze über Berliner Theateraufführungen} {[}1905{]}|pwk}, 1905, \emph{Vom Rückgang der deutschen
                     Bühne}\pwindex{Goldmann, Paul 31.01.1865 – 25.09.1935@\textsc{Goldmann, Paul} (31.01.1865 – 25.09.1935), \emph{Schriftsteller, Journalist}!Vom Rueckgang der deutschen Buehne. Polemische Aufsaetze ueber Berliner
                  Theater-Auffuehrungen1908@\strich\emph{Vom Rückgang der deutschen Bühne. Polemische Aufsätze über Berliner Theater-Aufführungen} {[}1908{]}|pwk}, 1908, und Literatenstücke und Ausstattungsregie\pwindex{Goldmann, Paul 31.01.1865 – 25.09.1935@\textsc{Goldmann, Paul} (31.01.1865 – 25.09.1935), \emph{Schriftsteller, Journalist}!Literatenstuecke und Ausstattungsregie. Polemische Aufsaetze ueber Berliner
                  Theater-Auffuehrungen1910-11-12@\strich\emph{Literatenstücke und Ausstattungsregie. Polemische Aufsätze über Berliner Theater-Aufführungen} {[}1910-11-12{]}|pwkv},
                     1910). In dem Band\pwindex{Goldmann, Paul 31.01.1865 – 25.09.1935@\textsc{Goldmann, Paul} (31.01.1865 – 25.09.1935), \emph{Schriftsteller, Journalist}!Vom Rueckgang der deutschen Buehne. Polemische Aufsaetze ueber Berliner
                  Theater-Auffuehrungen1908@\strich\emph{Vom Rückgang der deutschen Bühne. Polemische Aufsätze über Berliner Theater-Aufführungen} {[}1908{]}|pwkv} von 1905 sind Goldmann\pwindex{Goldmann, Paul 31.01.1865 – 25.09.1935@\textsc{Goldmann, Paul} (31.01.1865 – 25.09.1935), \emph{Schriftsteller, Journalist}|pwk}s Kritiken\pwindex{Goldmann, Paul 31.01.1865 – 25.09.1935@\textsc{Goldmann, Paul} (31.01.1865 – 25.09.1935), \emph{Schriftsteller, Journalist}!Berliner Theater. (»Der Schleier der Beatrice« von Arthur Schnitzler.)1903-03-19@\strich\emph{Berliner Theater. (»Der Schleier der Beatrice« von Arthur Schnitzler.)} {[}1903-03-19{]}|pwkv}\pwindex{Goldmann, Paul 31.01.1865 – 25.09.1935@\textsc{Goldmann, Paul} (31.01.1865 – 25.09.1935), \emph{Schriftsteller, Journalist}!Berliner Theater. »Der einsame Weg«. Von Arthur Schnitzler1904-02-23@\strich\emph{Berliner Theater. »Der einsame Weg«. Von Arthur Schnitzler} {[}1904-02-23{]}|pwkv} zu \emph{Der
                     Schleier der Beatrice}\pwindex{Schnitzler, Arthur 15.05.1862 – 21.10.1931@\textsc{Schnitzler, Arthur} (15.05.1862 – 21.10.1931), \emph{Schriftsteller, Mediziner}!einsame Weg. Schauspiel in fuenf Akten1904@\strich\emph{Der einsame Weg. Schauspiel in fünf Akten} {[}1904{]}|pwk} und zu \emph{Der einsame
                     Weg}\pwindex{Schnitzler, Arthur 15.05.1862 – 21.10.1931@\textsc{Schnitzler, Arthur} (15.05.1862 – 21.10.1931), \emph{Schriftsteller, Mediziner}!Schleier der Beatrice. Schauspiel in fuenf Akten1900-12-01@\strich\emph{Der Schleier der Beatrice. Schauspiel in fünf Akten} {[}1900-12-01{]}|pwk} enthalten. Der Band\pwindex{Goldmann, Paul 31.01.1865 – 25.09.1935@\textsc{Goldmann, Paul} (31.01.1865 – 25.09.1935), \emph{Schriftsteller, Journalist}!Vom Rueckgang der deutschen Buehne. Polemische Aufsaetze ueber Berliner
                  Theater-Auffuehrungen1908@\strich\emph{Vom Rückgang der deutschen Bühne. Polemische Aufsätze über Berliner Theater-Aufführungen} {[}1908{]}|pwkv} von 1908 enthält Goldmann\pwindex{Goldmann, Paul 31.01.1865 – 25.09.1935@\textsc{Goldmann, Paul} (31.01.1865 – 25.09.1935), \emph{Schriftsteller, Journalist}|pwk}s Kritik\pwindex{Goldmann, Paul 31.01.1865 – 25.09.1935@\textsc{Goldmann, Paul} (31.01.1865 – 25.09.1935), \emph{Schriftsteller, Journalist}!Berliner Theater. »Der Ruf des Lebens« von Arthur Schnitzler1906-03-14@\strich\emph{Berliner Theater. »Der Ruf des Lebens« von Arthur Schnitzler} {[}1906-03-14{]}|pwkv} zu \emph{Der Ruf des
                     Lebens}\pwindex{Schnitzler, Arthur 15.05.1862 – 21.10.1931@\textsc{Schnitzler, Arthur} (15.05.1862 – 21.10.1931), \emph{Schriftsteller, Mediziner}!Ruf des Lebens. Schauspiel in drei Akten1906-02-20@\strich\emph{Der Ruf des Lebens. Schauspiel in drei Akten} {[}1906-02-20{]}|pwk}.}}}\label{K_L03475-11h} nicht aufgenommen. Warum nicht? Weil ich mir damals ſagte:
               die Kritik\pwindex{Goldmann, Paul 31.01.1865 – 25.09.1935@\textsc{Goldmann, Paul} (31.01.1865 – 25.09.1935), \emph{Schriftsteller, Journalist}!Berliner Theater. (»Lebendige Stunden« von Arthur Schnitzler.)1902-01-22@\strich\emph{Berliner Theater. (»Lebendige Stunden« von Arthur Schnitzler.)} {[}1902-01-22{]}|pwv} zu ſchreiben, war
               meine Pflicht; {\pb}ſie in mein Buch aufzunehmen, bin
               ich nicht vepflichtet; u. ich habe ſie nicht aufgenommen, aus Rückſicht auf den
               Freund, über deſſen Werk\pwindex{Schnitzler, Arthur 15.05.1862 – 21.10.1931@\textsc{Schnitzler, Arthur} (15.05.1862 – 21.10.1931), \emph{Schriftsteller, Mediziner}!Lebendige Stunden. Vier Einakter1901-12-23@\strich\emph{Lebendige Stunden. Vier Einakter} {[}1901-12-23{]}|pwv} ſie
               ungünſtig urteilt. In einem eigentümlichen Lichte erſcheint mir heut dieſe Rückſicht
               auf den Freund, der Briefe von mir, in denen ich redlich beſtrebt war, ein herzliches
               freundſchaftliches Empfinden mit der Wahrheit in Einklang zu bringen, heranzieht, um
               damit meine Charakterloſigkleit zu beiweiſen!\pend
           \pstart
           Es fehlen mir alſo für den Fall der »Lebendigen
                  Stunden\pwindex{Schnitzler, Arthur 15.05.1862 – 21.10.1931@\textsc{Schnitzler, Arthur} (15.05.1862 – 21.10.1931), \emph{Schriftsteller, Mediziner}!Lebendige Stunden. Vier Einakter1901-12-23@\strich\emph{Lebendige Stunden. Vier Einakter} {[}1901-12-23{]}|pw}« {\pb}alle Dokumente\strikeout{,} u. ich bin auf mein Gedächtnis angewieſen. Dieſes
               Gedächtnis ſagt mir, daß ich mich nach der \label{K_L03475-14v}\edtext{Vorleſung\pwindex{Schnitzler, Arthur 15.05.1862 – 21.10.1931@\textsc{Schnitzler, Arthur} (15.05.1862 – 21.10.1931), \emph{Schriftsteller, Mediziner}!Lebendige Stunden. Vier Einakter1901-12-23@\strich\emph{Lebendige Stunden. Vier Einakter} {[}1901-12-23{]}|pwv} im Walde zu Welsberg\oindex{Welsberg-Taisten@\textbf{Welsberg-Taisten}|pw}}{\lemma{\textnormal{\emph{Vorleſung … Welsberg}}}\Cendnote{\textnormal{am 24. 8. 1901; siehe auch A. S.: \emph{Tagebuch}, 5. 12. 1921}}}\label{K_L03475-14h}, über die Stücke\pwindex{Schnitzler, Arthur 15.05.1862 – 21.10.1931@\textsc{Schnitzler, Arthur} (15.05.1862 – 21.10.1931), \emph{Schriftsteller, Mediziner}!Lebendige Stunden. Vier Einakter1901-12-23@\strich\emph{Lebendige Stunden. Vier Einakter} {[}1901-12-23{]}|pwv}
               lobend geäußert habe. Als ich ſie dann auf der Bühne\oindex{Deutsches Theater Berlin@\textbf{Deutsches Theater Berlin}|pwv} ſah u. ihre Schwächen klar erkannte, habe ich dem Ausdruck\pwindex{Goldmann, Paul 31.01.1865 – 25.09.1935@\textsc{Goldmann, Paul} (31.01.1865 – 25.09.1935), \emph{Schriftsteller, Journalist}!Berliner Theater. (»Lebendige Stunden« von Arthur Schnitzler.)1902-01-22@\strich\emph{Berliner Theater. (»Lebendige Stunden« von Arthur Schnitzler.)} {[}1902-01-22{]}|pwv} gegeben. Mein
               kritiſches Gewiſſen fühlt ſich durch dieſen »Widerſpruch« nicht im mindeſten
               belaſtet. Denn Stücke ſind nicht dazu da, im Walde vorgeleſen, ſondern aufgeführt zu
               werden; u. \strikeout{\textcolor{gray}{ein}} jedes vor der Aufführung abgegebene {\pb}Urteil über ein Stück kann immer nur ein Urteil mit Vorbehalt ſein. Wenn ich nach
               der Aufführung über die »Lebendigen Stunden\pwindex{Schnitzler, Arthur 15.05.1862 – 21.10.1931@\textsc{Schnitzler, Arthur} (15.05.1862 – 21.10.1931), \emph{Schriftsteller, Mediziner}!Lebendige Stunden. Vier Einakter1901-12-23@\strich\emph{Lebendige Stunden. Vier Einakter} {[}1901-12-23{]}|pw}«
               ungünſtig geurteilt haben würde u. die Stücke\pwindex{Schnitzler, Arthur 15.05.1862 – 21.10.1931@\textsc{Schnitzler, Arthur} (15.05.1862 – 21.10.1931), \emph{Schriftsteller, Mediziner}!Lebendige Stunden. Vier Einakter1901-12-23@\strich\emph{Lebendige Stunden. Vier Einakter} {[}1901-12-23{]}|pwv} wären doch gut, hätte ich als Kritiker gefehlt. Da ich
               die Stücke\pwindex{Schnitzler, Arthur 15.05.1862 – 21.10.1931@\textsc{Schnitzler, Arthur} (15.05.1862 – 21.10.1931), \emph{Schriftsteller, Mediziner}!Lebendige Stunden. Vier Einakter1901-12-23@\strich\emph{Lebendige Stunden. Vier Einakter} {[}1901-12-23{]}|pwv} aber nach wie vor
               nicht für gut halte (von manchen Au{[}r{]}alitäten abgeſehen, welche
               die erſten haben, u. abgeſehen auch von dem ſehr hübſchen Einakter »Literatur\pwindex{Schnitzler, Arthur 15.05.1862 – 21.10.1931@\textsc{Schnitzler, Arthur} (15.05.1862 – 21.10.1931), \emph{Schriftsteller, Mediziner}!Literatur1901@\strich\emph{Literatur} {[}1901{]}|pw}«), da überdies ihr geringer Erfolg auf der Bühne
                  \strikeout{mein} das in meiner Beſprechung\pwindex{Goldmann, Paul 31.01.1865 – 25.09.1935@\textsc{Goldmann, Paul} (31.01.1865 – 25.09.1935), \emph{Schriftsteller, Journalist}!Berliner Theater. (»Lebendige Stunden« von Arthur Schnitzler.)1902-01-22@\strich\emph{Berliner Theater. (»Lebendige Stunden« von Arthur Schnitzler.)} {[}1902-01-22{]}|pwv} ausgeſprochene Urteil
               beſtätigt, {\pb}bin ich als Kritiker ſicher nicht im
               Unrecht; u. ich finde, daß es eine Lächerlichkeit iſt, gegen das öffentlich
               abgegebene Urteil\pwindex{Goldmann, Paul 31.01.1865 – 25.09.1935@\textsc{Goldmann, Paul} (31.01.1865 – 25.09.1935), \emph{Schriftsteller, Journalist}!Berliner Theater. (»Lebendige Stunden« von Arthur Schnitzler.)1902-01-22@\strich\emph{Berliner Theater. (»Lebendige Stunden« von Arthur Schnitzler.)} {[}1902-01-22{]}|pwv} eines
               Kritikers, das er genau u. ſachlich begründet hat, Äußerungen ausſpielen zu wollen,
               die er nach einer Vorleſung\pwindex{Schnitzler, Arthur 15.05.1862 – 21.10.1931@\textsc{Schnitzler, Arthur} (15.05.1862 – 21.10.1931), \emph{Schriftsteller, Mediziner}!Lebendige Stunden. Vier Einakter1901-12-23@\strich\emph{Lebendige Stunden. Vier Einakter} {[}1901-12-23{]}|pwv} im
                  Walde\oindex{Welsberg-Taisten@\textbf{Welsberg-Taisten}|pwv} getan hat.\pend
           \pstart
           Ich habe mein Gedächtnis weiter angeſtrengt u. kann mich an die Äußerung, die ich \substVorne{}\textsuperscript{\textcolor{gray}{we}iter}{\allowbreak}\substDazwischen{}außerdem\substHinten{} getan haben ſoll, daß ich nämlich bedaure, nicht ſelbſt ſolche Stücke\pwindex{Schnitzler, Arthur 15.05.1862 – 21.10.1931@\textsc{Schnitzler, Arthur} (15.05.1862 – 21.10.1931), \emph{Schriftsteller, Mediziner}!Schleier der Beatrice. Schauspiel in fuenf Akten1900-12-01@\strich\emph{Der Schleier der Beatrice. Schauspiel in fünf Akten} {[}1900-12-01{]}|pwv}\pwindex{Schnitzler, Arthur 15.05.1862 – 21.10.1931@\textsc{Schnitzler, Arthur} (15.05.1862 – 21.10.1931), \emph{Schriftsteller, Mediziner}!Lebendige Stunden. Vier Einakter1901-12-23@\strich\emph{Lebendige Stunden. Vier Einakter} {[}1901-12-23{]}|pwv} ſchreiben zu
               können, nicht mehr erinnern. Aber ich will nicht in Abrede ſtellen, ſie getan zu
               haben. {\pb}Warum ſollte ich auch nicht von Stücke\pwindex{Schnitzler, Arthur 15.05.1862 – 21.10.1931@\textsc{Schnitzler, Arthur} (15.05.1862 – 21.10.1931), \emph{Schriftsteller, Mediziner}!Schleier der Beatrice. Schauspiel in fuenf Akten1900-12-01@\strich\emph{Der Schleier der Beatrice. Schauspiel in fünf Akten} {[}1900-12-01{]}|pwv}\pwindex{Schnitzler, Arthur 15.05.1862 – 21.10.1931@\textsc{Schnitzler, Arthur} (15.05.1862 – 21.10.1931), \emph{Schriftsteller, Mediziner}!Lebendige Stunden. Vier Einakter1901-12-23@\strich\emph{Lebendige Stunden. Vier Einakter} {[}1901-12-23{]}|pwv}n, die mir
               gefielen, geſagt haben, daß ich bedaure, ſie nicht auch ſchreiben zu können? Wenn
               aber weiter behauptet wird, ich hätte geſagt, ich möchte mich »erſchießen«, weil ich
               Solches nicht leiſten kann, ſo erkläre ich dies für eine \uline{Unwahrheit}. \strikeout{\textcolor{gray}{×}\-\textcolor{gray}{×}\-\textcolor{gray}{×}\-\textcolor{gray}{×}{ }\textcolor{gray}{Feſtſtellung dieſ}\textcolor{gray}{×}\-\textcolor{gray}{×}{ }\textcolor{gray}{×}\-\textcolor{gray}{×}\-\textcolor{gray}{×}\-\textcolor{gray}{×}\-\textcolor{gray}{×}\-\textcolor{gray}{×}\-\textcolor{gray}{×}}\strikeout{\textcolor{gray}{[unleserliche Zeile{]} }}\strikeout{\textcolor{gray}{[unleserliche Zeile{]} }} Ich \uline{weiß}, daß ich das nicht geſagt haben kann
               u. auch nicht geſagt habe, weil ich weiß, daß ich mich nicht mit weibiſchem Schwulſt
                  {\pb}auszudrücken pflege, ſondern die Gewohnheit
               habe, zu reden, wie ein Mann{\dotsseven}\pend
           \pstart
           Lieber Freund, Du haſt mir auch bei unſerem letzten \label{K_L03475-17v}\edtext{Beiſammenſein}{\lemma{\textnormal{\emph{Beiſammenſein}}}\Cendnote{\textnormal{am 28. 12. 1910,
                  siehe oben}}}\label{K_L03475-17h} wieder jede Fähigkeit zum Kritiker abgeſprochen. Dieſe Deine
               Anſicht über mich iſt mir ſeit Langem bekannt. Sie iſt für mich gewiß nicht
               belanglos. Denn ich habe nicht die Selbſtſicherheit, die Du beſitzeſt u. die Dich zu
               dem Ausſpuch veranlaßt, daß es {\pb}Dir gleichgiltig
               iſt, was \strikeout{die} »wir Andern« über Dich ſchreiben. Mir
               iſt es gar nicht gleichgiltig, was die Andern über mich ſchreiben oder ſagen. Wohl
               habe ich künſtleriſche \substVorne{}\textsuperscript{\textcolor{gray}{We}lt\textcolor{gray}{anſchauunge}n}{\allowbreak}\substDazwischen{}Anſchauungen\substHinten{}, von deren Richtigkeit ich unerſchütterlich überzeugt bin. Aber ich prüfe
               jedes noch ſo ungünſtige Urteil über mich, ob es nicht vielleicht doch etwas Wahres
               enthält, u. ſuche von jedem Andern, auch von heftigſten Gegnern, etwas zu lernen. Man
               muß ſchon ein mit Erfolg aufgeführter dramatiſcher Autor ſein, {\pb}um das Bewußtſein mit ſich herumzutragen, daß
               man von Anderen nichts mehr zu lernen habe. Bei ernſt ſtrebenden Menſchen in anderen
               Berufsarten wird man dieſes Bewußtſein kaum wiederfinden.\pend
           \pstart
           Mir iſt es nicht gleichgiltig, was die Andern von mir ſagen, – u. ganz gewiß nicht
               gleichgiltig, was ein alter Freund von mir denkt, Aber mit Deiner Mißbilligung meiner
               Wirkſamkeit als Kritiker habe ich mich \introOben{}längſt\introOben{} abgefunden.
               Ich habe mir geſagt, daß Dein\strikeout{e} u. mein Lebensweg ſo
               weit auseinandergegangen ſind, {\pb}daß Deine u. meine
               Entwicklung eine ſo gänzlich verſchiedene Richtung eingeſchlagen haben, daß Du mich
               eben nicht mehr verſtehſt u. verſtehen kannſt. Du ſiehſt ja auch all’ das, worüber
               ich als Kritiker zu urteilen habe, von einem ganz anderen Standpunkt an, als ich. Du
               biſt ſelbſt beteiligt, biſt ſelbſt Partei. Meine künſtleriſchen Überzeugungen haben
               mich dazu geführt, Stellung gegen \strikeout{d\textcolor{gray}{ie}} die meiſten der dramatiſchen Autoren unſerer Generation, Stellung ſogar gegen
               manches Deiner Werke zu nehmen. {\pb}Wie darf ich da
               von Dir erwarten oder gar beanſpruchen, daß Du meine kritiſche Tätigkeit
               billigſt!\pend
           \pstart
           Ich habe es Dir alſo niemals \textcolor{gray}{abverlan}gt, daß Du mich für einen
               ſchlechten Kritiker hältſt. Ich habe allerdings, wenn ich mit Dir ſprach u. von Dir
               ſo manche Anſchauung hörte, die ich für falſch halten muß, im Stillen Gott gedankt,
               daß ich nicht ein Kritiker geworden bin, den Du für gut halten würdeſt.\pend
           \pstart
           {\pb}Deine Urteile über meine kritiſche Tätigkeit
               haben mich alſo nie von Dir abgeſtoßen, u. ich war feſt entſchloſſen, trotz alledem
                  \strikeout{\textcolor{gray}{Dir}} eine Freundſchaft zu erhalten, die nun ſchon mehr als zwanzig Jahre alt
                  iſt\strikeout{,} u. von der, ſo ſehr wir auch innerlich
               entfremdet ſind, doch ein enormes u. herzliches Gefühl für Dich bei mir
               zurückgeblieben iſt.\strikeout{\textcolor{gray}{×}}\pend
           \pstart
           Nun aber haſt Du in unſerer letzten Unterredung im Hauſe\oindex{Frankgasse@\textbf{Frankgasse}|pwv} Deiner Mutter\pwindex{Schnitzler, Louise 1840-07-08 – 1911-09-09@\textsc{Schnitzler, Louise} (1840-07-08 – 1911-09-09)|pwv} in Deinen Angriffen gegen mich eine Grenze
               überſchritten, die Du {\pb}nicht überſchreiten
               durfteſt. Von meinen Fähigkeiten als Kritiker darfſt Du ſagen, was Du willſt. In
               dieſer Unterredung aber haſt Du es verſucht, meine Ehre anzutaſten. Und dieſen
               Verſuch muß ich mit der äußerſten Schärfe zurückweiſen.\strikeout{Die
                  Sprache \textcolor{gray}{×}u} Selbſt eine
               zwanzigjährige Freundſchaft gibt Dir nicht das Recht zu einer Sprache, \substVorne{}\textsuperscript{\textcolor{gray}{D}i\textcolor{gray}{e}}\substDazwischen{}die\substHinten{} Du in jener Unterredung Dir herausgenommen haſt, gegen mich zu führen. Das
               kann u. werde ich nicht {\pb}dulden! Und es iſt \strikeout{unehr} unerhört, es iſt eine der bitterſten Erfahrungen
               meines Lebens, daß ich, nachdem ich in einem ſchweren Lebenskampfe meine Ehre rein u.
               flankenlos erhalten habe, mich nun gegen den älteſten u. mir einſt nächſten Freund
               zur Wehr ſetzen \strikeout{will} muß, der meine Ehre \strikeout{\textcolor{gray}{bef}} beflecken will. An jener Unterredung, in der \strikeout{Du \textcolor{gray}{×}\-\textcolor{gray}{×}\-\textcolor{gray}{×}\-\textcolor{gray}{×} ich \textcolor{gray}{×}\-\textcolor{gray}{×}\-\textcolor{gray}{×}{ }\textcolor{gray}{×}\-\textcolor{gray}{×}\-\textcolor{gray}{×}} Du über mich, der ich als Gaſt im Hauſe\oindex{Frankgasse@\textbf{Frankgasse}|pwv} Deiner Mutter\pwindex{Schnitzler, Louise 1840-07-08 – 1911-09-09@\textsc{Schnitzler, Louise} (1840-07-08 – 1911-09-09)|pwv} weilte, {\pb}\strikeout{\textcolor{gray}{×}\-\textcolor{gray}{×}\-\textcolor{gray}{×}\-\textcolor{gray}{×}\-\textcolor{gray}{×}\-\textcolor{gray}{×}\-\textcolor{gray}{×}} hergefalllen biſt, wie über einen characterloſen Lumpen, denke ich zurück mit
               einer Miſchung von Scham, Widerwillen u. Empörung; u. ich konnte nicht Ruhe finden,
               ehe ich Dir dieſen Brief geſchrieben, um Deine Anwürfe von mir abzuſchütteln, –
               ſelbſt auf die Gefahr hin, daß dieſer Brief den Bruch unſerer zwanzigjährigen
               Freundſchaft herbeiführen ſollte.\pend
           \pstart
           {\pb}Mit herzlichem Gruß {\\[\baselineskip]}Dein {\\[\baselineskip]}\spacefill\mbox{Paul Goldmann.}\pend
           \leftskip=0em{}
         
         \endnumbering\mylabel{h}\end{ledgroupsized}\begin{anhang}\end{anhang}\newcommand{\dateiname}{L03475}\newcommand{\titel}{Paul Goldmann an Arthur Schnitzler, 13. 1. 1911}\newcommand{\editorInnen}{Martin Anton Müller und Laura Untner}%% latex-leseansicht-abspann.tex
%% Abspann für die Leseansicht.
%% Der Schalter \ifkorrekturansicht ist bereits durch den Vorspann gesetzt.

%% latex-abspann.tex
%% Gemeinsamer Abspann für Korrekturansicht und Leseansicht.
%% Setzt den Schalter \ifkorrekturansicht voraus (gesetzt in den
%% einbindenden Dateien latex-korrekturansicht-abspann.tex bzw.
%% latex-leseansicht-abspann.tex).
%% ---------------------------------------------------------------

\normalsize

% Das esempio-Environment wird nur in der Leseansicht benötigt
\ifkorrekturansicht\else
\newenvironment{esempio}[3]%
{
    \vspace{1.5ex}
    \rlap{\underline{#1}}
    \par
    \setlength{\parindent}{0cm}
    \nopagebreak
    \leftskip=#2cm
    \rightskip=#3cm
}
{
    \par
}
\fi

\doendnotes{C}
\bigskip
\vfill

\clearpage

\footnotesize

\ifkorrekturansicht
  \lohead{\textsc{register}}
\fi

% theindex-Environment neu definieren ohne reledmac
\makeatletter
\renewenvironment{theindex}{%
  \ifkorrekturansicht
    \section*{\indexname}%
  \else
    \subsubsection*{Index der erwähnten Entitäten}%
  \fi
  \setlength{\parindent}{0pt}%
  \setlength{\parskip}{0pt plus 0.3pt}%
  \let\item\@idxitem
}{%
  \ifkorrekturansicht\clearpage\fi
}
\makeatother

\IfFileExists{\jobname-pw.ind}{\input{\jobname-pw.ind}}{}

% Quellenangabe nur in der Leseansicht
\ifkorrekturansicht\else
% Fallback-Definitionen, falls die .tex-Datei \titel etc. nicht gesetzt hat
\providecommand{\titel}{}
\providecommand{\editorInnen}{}
\providecommand{\dateiname}{\jobname}

\vspace{3cm}

\vfill

\footnotesize
\textsc{Quelle}: \titel. Herausgegeben von {\editorInnen}. In: \emph{Arthur Schnitzler: Briefwechsel mit Autorinnen und Autoren}.
 Digitale Edition, https://schnitzler-briefe.acdh.oeaw.ac.at/{\dateiname}.html (Stand \today)
\fi

\end{document}


      