\input{../tex-inputs/latex-pdf-vorspann}
\begin{center}
            \textcolor{red}{ENTWURF. ENTZIFFERUNG NOCH NICHT KORREKTURGELESEN}
                      \end{center}
            
               \section[Hugo von Hofmannsthal an Arthur Schnitzler, 2. 6. 1896]{ Hugo von Hofmannsthal an Arthur Schnitzler, 2. 6. 1896}\nopagebreak\mylabel{v}\rehead{ }\begin{ledgroupsized}[t]{13cm}\normalsize\beginnumbering\briefempfaengerindex{Schnitzler, Arthur@\textsc{Schnitzler, Arthur}!zzzHofmannsthal, Hugo von@\emph{von Hugo von Hofmannsthal}!1896-06-021@{2. 6. 1896}|(be} \toendnotes[C]{\smallbreak\pagebreak[2]} \Standort{CUL, Schnitzler, B 43.}
\physDesc{Postkarte
\newline{}Handschrift: Bleistift, deutsche Kurrent\newline{}Versand: 1) Rohrpost 2) Stempel: »\nobreak{}\oindex{III., Landstrasse@\textbf{III., Landstraße}|pwk}Wien 3/3, 2 VI 96, 10–V\nobreak{}«. 3) Stempel: »\nobreak{}\oindex{IX., Alsergrund@\textbf{IX., Alsergrund}|pwk}Wien 9/3, 2 VI 96, 11 10V\nobreak{}«. \newline{}Ordnung: 1) mit Bleistift von unbekannter Hand auf der
                                    Anschriftenseite nummeriert: »77a« 2) mit Bleistift von unbekannter Hand auf der
                                 Textseite nummeriert: »178«}\buchAbdrucke{\weitereDrucke{Hugo von Hofmannsthal, Arthur Schnitzler: \emph{Briefwechsel}. Hg. Therese Nickl und Heinrich Schnitzler. Frankfurt am Main: \emph{S. Fischer} 1964, S. 67.} }\pstart{}{\pb}\textsc{Herrn D\textsuperscript{r} Arthur Schnitzler}\pend{}\pstart{}\textsc{IX
                            Franckgasse} 1\oindex{Frankgasse@\textbf{Frankgasse}|pw}\pend{}\pstart{}\textsc{Wien}\oindex{Wien@\textbf{Wien}|pw}\pend{}{\bigskip}\pstart
           \noindent{}{\pb}Ich werde
                        8–½ 9 im Grienſteidl\oindex{Cafe Griensteidl@\textbf{Café Griensteidl}|pw}{ }ſein, eventuell vorher bei Richard\pwindex{Beer-Hofmann, Richard 11.07.1866 – 26.09.1945@\textsc{Beer-Hofmann, Richard} (11.07.1866 – 26.09.1945), \emph{Schriftsteller}|pw} anläuten.\pend
           \pstart
           Bitte verſtändigen Sie ihn, ich habe keine Zeit dazu\pend
           \pstart \spacefill\mbox{Hugo}\pend{}\pstart
           \noindent{}eine eventuelle Änderung kann dann Richard\pwindex{Beer-Hofmann, Richard 11.07.1866 – 26.09.1945@\textsc{Beer-Hofmann, Richard} (11.07.1866 – 26.09.1945), \emph{Schriftsteller}|pw} mir bis 5{ }ſagen laſſen.\pend
           \endnumbering\briefempfaengerindex{Schnitzler, Arthur@\textsc{Schnitzler, Arthur}!zzzHofmannsthal, Hugo von@\emph{von Hugo von Hofmannsthal}!1896-06-021@{2. 6. 1896}|)be}\mylabel{h}\end{ledgroupsized}  \newcommand{\dateiname}{L00547}\newcommand{\titel}{Hugo von Hofmannsthal an Arthur Schnitzler, 2. 6. 1896}\newcommand{\editorInnen}{Martin Anton Müller und Gerd-Hermann Susen}\input{../tex-inputs/latex-pdf-abspann}
      