%% latex-leseansicht-vorspann.tex
%% Vorspann für die Leseansicht.
%% Lädt die gemeinsame Datei latex-vorspann.tex mit nicht gesetztem Schalter.

\newif\ifkorrekturansicht
\korrekturansichtfalse

\input{../tex-inputs/latex-vorspann}

\begin{center}
            \textcolor{red}{ENTWURF, NICHT FERTIG KORRIGIERT}
                      \end{center}
            
         
         \renewcommand{\erwaehntePersonen}{Personen: Lili Cappellini, Olga Schnitzler, Heinrich Schnitzler, Johanna Sophie Wollf}
         \renewcommand{\erwaehnteOrte}{Orte: Brennerriesensteig, Brijuni, Salzkammergut, Schafberg (St. Gilgen), Steinbach am Attersee, Unterach am Attersee}
         \renewcommand{\erwaehnteWerke}{}
               \section[Felix Salten u. a. an Arthur und Olga Schnitzler, {[}Ende Juli/August 1912?{]}]{ Felix Salten u. a. an Arthur und Olga Schnitzler, {[}Ende
               Juli/August 1912?{]}}\nopagebreak\mylabel{v}\rehead{ }\begin{ledgroupsized}[t]{13cm}\normalsize\beginnumbering \toendnotes[C]{\smallbreak\pagebreak[2]} \Standort{CUL, Schnitzler, B 89, B 2.}
\physDesc{Bildpostkarte, 403 Zeichen
\newline{}Handschrift Felix Salten: schwarze Tinte, lateinische Kurrent\newline{}Handschrift Ottilie Salten: schwarze Tinte, lateinische Kurrent\newline{}Handschrift Julius Ferdinand Wollf: schwarze Tinte, lateinische Kurrent\newline{}Handschrift Helene Jarosy: schwarze Tinte, lateinische Kurrent\newline{}Handschrift Richard Metzl: schwarze Tinte, deutsche Kurrent
\newline{}Versand: Stempel: »\nobreak{}\oindex{Unterach am Attersee@\textbf{Unterach am Attersee}|pwk}Unter\textcolor{gray}{ach am
                                       Attersee} a\nobreak{}«.  
\newline{}Ordnung: mit Bleistift von unbekannter Hand nummeriert: »288« }\toendnotes[C]{\smallbreak}\pstart{}{\pb}Herrn u. Frau\pend{}\pstart{}D\textsuperscript{r} Arthur Schnitzler\pend{}\pstart{}Brioni\oindex{Brijuni@\textbf{Brijuni}|pw}\pend{}{\bigskip}\pstart
           \noindent{}\centering{}{\pb}\textcolor{gray}{\textbf{Salzkammergut\oindex{Salzkammergut@\textbf{Salzkammergut}|pw}. Blick vom Brennerriesensteig\oindex{Brennerriesensteig@\textbf{Brennerriesensteig}|pw} bei Steinbach auf den Attersee\oindex{Steinbach am Attersee@\textbf{Steinbach am Attersee}|pw} u. Schafberg\oindex{Schafberg (St. Gilgen)@\textbf{Schafberg (St. Gilgen)}|pw}.}}\pend
           \pstart
           {\pb}Lieber Arthur und liebe Olga, wir haben \label{K_L03575-1v}\edtext{heute}{\lemma{\textnormal{\emph{heute}}}\Cendnote{\textnormal{Die Bildpostkarte ist undatiert und der
                  Stempel nur teilweise gedruckt, sodass auf andere Kriterien zur Einordnung
                  zurückgegriffen werden muss. Es kommen zwei längere Aufenthalte Schnitzler\pwindex{Schnitzler, Arthur 15.05.1862 – 21.10.1931@\textsc{Schnitzler, Arthur} (15.05.1862 – 21.10.1931), \emph{Schriftsteller, Mediziner}|pwk}s in Brijuni\oindex{Brijuni@\textbf{Brijuni}|pwk} in Betracht: im Juli und August 1912 und im Juli und
                     August 1913. Da nur aus dem ersten Jahr
                  Korrespondenzstücke (Felix Salten an Arthur Schnitzler, 2. 7. 1912 und 22. 7. 1912) überliefert
                  sind, die belegen, dass ein Austausch mit Salten\pwindex{Salten, Felix 06.09.1869 – 08.10.1945@\textsc{Salten, Felix} (06.09.1869 – 08.10.1945), \emph{Schriftsteller, Journalist}|pwk} stattfand, dürfte diese Karte ebenso in diesem Jahr zu verorten sein. Zudem ist davon auszugehen, dass sie nach Salten\pwindex{Salten, Felix 06.09.1869 – 08.10.1945@\textsc{Salten, Felix} (06.09.1869 – 08.10.1945), \emph{Schriftsteller, Journalist}|pwk}s Brief vom 22. 7. 1912 einzuordnen ist,
                  da die Anwesenheit von Julius Ferdinand
                     Wollf\pwindex{Wollf, Julius Ferdinand 22.05.1871 – 01.03.1942@\textsc{Wollf, Julius Ferdinand} (22.05.1871 – 01.03.1942), \emph{Journalist, Herausgeber, Verleger}|pwk}, Helene Jarosy\pwindex{Jarosy, Helene @\textsc{Jarosy, Helene}, \emph{Opernsängerin}|pwk} und Richard Metzl\pwindex{Metzl, Richard 20.04.1870 – 31.10.1941@\textsc{Metzl, Richard} (20.04.1870 – 31.10.1941), \emph{Regisseur, Schauspieler, Theatersekretär}|pwk} in diesem Brief noch nicht
                  erwähnt wurde und Salten\pwindex{Salten, Felix 06.09.1869 – 08.10.1945@\textsc{Salten, Felix} (06.09.1869 – 08.10.1945), \emph{Schriftsteller, Journalist}|pwk} erst am 2. 9. 1912 schrieb, das
                  Ehepaar Wollf\pwindex{Wollf, Julius Ferdinand 22.05.1871 – 01.03.1942@\textsc{Wollf, Julius Ferdinand} (22.05.1871 – 01.03.1942), \emph{Journalist, Herausgeber, Verleger}|pwk}\pwindex{Wollf, Johanna Sophie 1877-10-18 – 1942-02-27@\textsc{Wollf, Johanna Sophie} (1877-10-18 – 1942-02-27)|pwk} sei »drei
                     Wochen lang bei uns gewesen«, was ebenso für eine spätere Einordnung
                  spricht. Nach vorne ist die Datierung durch Schnitzler\pwindex{Schnitzler, Arthur 15.05.1862 – 21.10.1931@\textsc{Schnitzler, Arthur} (15.05.1862 – 21.10.1931), \emph{Schriftsteller, Mediziner}|pwk}s Abreise am 24. 8. 1912 eingrenzbar.}}}\label{K_L03575-1h} in Herzlichkeit
               Ihrer gedacht und senden Ihnen viele Grüße! Hoffentlich haben Sie mit den Kinder\pwindex{Schnitzler, Heinrich 09.08.1902 – 12.07.1982@\textsc{Schnitzler, Heinrich} (09.08.1902 – 12.07.1982), \emph{Regisseur, Schauspieler}|pwv}\pwindex{Cappellini, Lili 13.09.1909 – 26.07.1928@\textsc{Cappellini, Lili} (13.09.1909 – 26.07.1928)|pwv}n schöne Tage.
               Herzlichst Ihr {\\}\spacefill\mbox{Salten}\pend
           \pstart
           \noindent{}{[}hs. Ottilie Salten:{]} Viele her\textcolor{gray}{r}liche Grüße
                  \spacefill\mbox{Ottilie}\pend
           \pstart
           \noindent{}{[}hs. Wollf:{]} Viele Grüsse von Ihrem ergebenen\pend
           \pstart \spacefill\mbox{Julius Ferdinand Wollf} und seiner Frau\pwindex{Wollf, Johanna Sophie 1877-10-18 – 1942-02-27@\textsc{Wollf, Johanna Sophie} (1877-10-18 – 1942-02-27)|pwv}\pend{}\pstart
           \noindent{}{[}hs. Jarosy:{]} Die schönsten Grüße Ihnen und der gnädigen Frau
                  \spacefill\mbox{Helene Jaroſy}\pend
           \pstart
           \noindent{}{[}hs. Metzl:{]} Beſte Grüße {\\}Ihr ergebener {\\}\spacefill\mbox{RichardMetzl}\pend
           
         
         \endnumbering\mylabel{h}\end{ledgroupsized}  \newcommand{\dateiname}{L03575}\newcommand{\titel}{Felix Salten u. a. an Arthur und Olga Schnitzler, [Ende Juli/August 1912?]}\newcommand{\editorInnen}{Martin Anton Müller und Laura Untner}%% latex-leseansicht-abspann.tex
%% Abspann für die Leseansicht.
%% Der Schalter \ifkorrekturansicht ist bereits durch den Vorspann gesetzt.

%% latex-abspann.tex
%% Gemeinsamer Abspann für Korrekturansicht und Leseansicht.
%% Setzt den Schalter \ifkorrekturansicht voraus (gesetzt in den
%% einbindenden Dateien latex-korrekturansicht-abspann.tex bzw.
%% latex-leseansicht-abspann.tex).
%% ---------------------------------------------------------------

\normalsize

% Das esempio-Environment wird nur in der Leseansicht benötigt
\ifkorrekturansicht\else
\newenvironment{esempio}[3]%
{
    \vspace{1.5ex}
    \rlap{\underline{#1}}
    \par
    \setlength{\parindent}{0cm}
    \nopagebreak
    \leftskip=#2cm
    \rightskip=#3cm
}
{
    \par
}
\fi

\doendnotes{C}
\bigskip
\vfill

\clearpage

\footnotesize

\ifkorrekturansicht
  \lohead{\textsc{register}}
\fi

% theindex-Environment neu definieren ohne reledmac
\makeatletter
\renewenvironment{theindex}{%
  \ifkorrekturansicht
    \section*{\indexname}%
  \else
    \subsubsection*{Index der erwähnten Entitäten}%
  \fi
  \setlength{\parindent}{0pt}%
  \setlength{\parskip}{0pt plus 0.3pt}%
  \let\item\@idxitem
}{%
  \ifkorrekturansicht\clearpage\fi
}
\makeatother

\IfFileExists{\jobname-pw.ind}{\input{\jobname-pw.ind}}{}

% Quellenangabe nur in der Leseansicht
\ifkorrekturansicht\else
% Fallback-Definitionen, falls die .tex-Datei \titel etc. nicht gesetzt hat
\providecommand{\titel}{}
\providecommand{\editorInnen}{}
\providecommand{\dateiname}{\jobname}

\vspace{3cm}

\vfill

\footnotesize
\textsc{Quelle}: \titel. Herausgegeben von {\editorInnen}. In: \emph{Arthur Schnitzler: Briefwechsel mit Autorinnen und Autoren}.
 Digitale Edition, https://schnitzler-briefe.acdh.oeaw.ac.at/{\dateiname}.html (Stand \today)
\fi

\end{document}


      