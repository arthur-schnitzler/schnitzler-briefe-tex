%% latex-leseansicht-vorspann.tex
%% Vorspann für die Leseansicht.
%% Lädt die gemeinsame Datei latex-vorspann.tex mit nicht gesetztem Schalter.

\newif\ifkorrekturansicht
\korrekturansichtfalse

\input{../tex-inputs/latex-vorspann}

\begin{center}
            \textcolor{red}{ENTWURF, NICHT FERTIG KORRIGIERT}
                      \end{center}
            
         
         \renewcommand{\erwaehntePersonen}{Personen: Olga Schnitzler, Heinrich Schnitzler, Lili Schnitzler, Johanna Sophie Wollf}
         \renewcommand{\erwaehnteOrte}{Orte: Brennerriesensteig, Brijuni, Salzkammergut, Schafberg (St. Gilgen), Steinbach am Attersee, Unterach am Attersee}
         \renewcommand{\erwaehnteWerke}{}
               \section[Felix Salten u. a. an Arthur und Olga Schnitzler, {[}Ende Juli/August 1912?{]}]{ Felix Salten u. a. an Arthur und Olga Schnitzler, {[}Ende Juli/August 1912?{]}}\nopagebreak\mylabel{v}\rehead{ }\begin{ledgroupsized}[t]{13cm}\normalsize\beginnumbering \toendnotes[C]{\smallbreak\pagebreak[2]} \Standort{CUL, Schnitzler, B 89, B 2.}
\physDesc{Bildpostkarte, 402 Zeichen
\newline{}Handschrift Felix Salten: schwarze Tinte, lateinische Kurrent\newline{}Handschrift Ottilie Salten: schwarze Tinte, lateinische Kurrent\newline{}Handschrift Julius Ferdinand Wollf: schwarze Tinte, lateinische Kurrent\newline{}Handschrift Helene Jarosy: schwarze Tinte, lateinische Kurrent\newline{}Handschrift Richard Metzl: schwarze Tinte, deutsche Kurrent
\newline{}Versand: Stempel: »\nobreak{}\oindex{Unterach am Attersee@\textbf{Unterach am Attersee}|pwk}Unter\textcolor{gray}{ach am
                                       Attersee}\nobreak{}«.  
\newline{}Ordnung: mit Bleistift von unbekannter Hand nummeriert:
                                 »288« }\toendnotes[C]{\smallbreak}\pstart{}{\pb}Herrn u.
                  Frau\pend{}\pstart{}D\textsuperscript{r} Arthur Schnitzler\pend{}\pstart{}Brioni\oindex{Brijuni@\textbf{Brijuni}|pw}\pend{}{\bigskip}\pstart
           \noindent{}{\pb}\textcolor{gray}{\textbf{Salzkammergut\oindex{Salzkammergut@\textbf{Salzkammergut}|pw}. Blick vom Brennerriesensteig\oindex{Brennerriesensteig@\textbf{Brennerriesensteig}|pw} bei Steinbach auf den Attersee\oindex{Steinbach am Attersee@\textbf{Steinbach am Attersee}|pw} u. Schafberg\oindex{Schafberg (St. Gilgen)@\textbf{Schafberg (St. Gilgen)}|pw}.}}\pend
           \pstart
           {\pb}Lieber Arthur und liebe Olga, wir haben \label{K_L03575-1v}\edtext{heute}{\lemma{\textnormal{\emph{heute}}}\Cendnote{\textnormal{Die Bildpostkarte
               ist undatiert und der Stempel nur teilweise gedruckt, so dass auf andere Kriterien zur Einordnung
               zurückgegriffen werden muss. Es kommen zwei Aufenthalte Schnitzler\pwindex{Schnitzler, Arthur 15.05.1862 – 21.10.1931@\textsc{Schnitzler, Arthur} (15.05.1862 – 21.10.1931), \emph{Schriftsteller, Mediziner}|pwk}s in
                  Brijuni\oindex{Brijuni@\textbf{Brijuni}|pwk} in Betracht, im Sommer 1912
                  und im Sommer 1913. Da aber nur aus dem ersten Jahr Korrespondenzstücke
                  (Felix Salten an Arthur Schnitzler, 2. 7. 1912 und 22. 7. 1912)
                  überliefert sind, die belegen, dass ein Austausch stattfand, dürfte diese Karte ebenso
                  in diesem Jahr zu verorten sein.}}}\label{K_L03575-1h} in Herzlichkeit Ihrer
               gedacht und senden Ihnen viele Grüße! Hoffentlich haben Sie mit den Kindern\pwindex{Schnitzler, Heinrich 09.08.1902 – 12.07.1982@\textsc{Schnitzler, Heinrich} (09.08.1902 – 12.07.1982), \emph{Regisseur, Schauspieler}|pwv}\pwindex{Schnitzler, Lili 13.09.1909 – 26.07.1928@\textsc{Schnitzler, Lili} (13.09.1909 – 26.07.1928)|pwv} schöne Tage! Herzlichst Ihr \pend
           \pstart \spacefill\mbox{Salten}\pend{}\pstart
           \noindent{}{[}hs. Ottilie Salten:{]} Viele herrliche Grüße \spacefill\mbox{Ottilie}\pend
           \pstart
           \noindent{}{[}hs. Wollf:{]} Viele Grüsse von Ihrem ergebenen\pend
           \pstart \spacefill\mbox{Julius Ferdinand Wollf} und seiner Frau\pwindex{Wollf, Johanna Sophie 1877-10-18 – 1942-02-27@\textsc{Wollf, Johanna Sophie} (1877-10-18 – 1942-02-27)|pw}\pend{}\pstart
           \noindent{}{[}hs. Jarosy:{]} Die schönsten Grüße Ihnen und der gnädigen Frau
                  \spacefill\mbox{Helene Jaroſy}\pend
           \pstart
           \noindent{}{[}hs. Metzl:{]} Beſte Grüße\pend
           \pstart
           Ihr ergebener {\\[\baselineskip]}\spacefill\mbox{RichardMetzl}\pend
           \leftskip=0em{}
         
         \endnumbering\mylabel{h}\end{ledgroupsized}\begin{anhang}\end{anhang}\newcommand{\dateiname}{L03575}\newcommand{\titel}{Felix Salten u. a. an Arthur und Olga Schnitzler, [Ende Juli/August 1912?]}\newcommand{\editorInnen}{Martin Anton Müller und Laura Untner}%% latex-leseansicht-abspann.tex
%% Abspann für die Leseansicht.
%% Der Schalter \ifkorrekturansicht ist bereits durch den Vorspann gesetzt.

%% latex-abspann.tex
%% Gemeinsamer Abspann für Korrekturansicht und Leseansicht.
%% Setzt den Schalter \ifkorrekturansicht voraus (gesetzt in den
%% einbindenden Dateien latex-korrekturansicht-abspann.tex bzw.
%% latex-leseansicht-abspann.tex).
%% ---------------------------------------------------------------

\normalsize

% Das esempio-Environment wird nur in der Leseansicht benötigt
\ifkorrekturansicht\else
\newenvironment{esempio}[3]%
{
    \vspace{1.5ex}
    \rlap{\underline{#1}}
    \par
    \setlength{\parindent}{0cm}
    \nopagebreak
    \leftskip=#2cm
    \rightskip=#3cm
}
{
    \par
}
\fi

\doendnotes{C}
\bigskip
\vfill

\clearpage

\footnotesize

\ifkorrekturansicht
  \lohead{\textsc{register}}
\fi

% theindex-Environment neu definieren ohne reledmac
\makeatletter
\renewenvironment{theindex}{%
  \ifkorrekturansicht
    \section*{\indexname}%
  \else
    \subsubsection*{Index der erwähnten Entitäten}%
  \fi
  \setlength{\parindent}{0pt}%
  \setlength{\parskip}{0pt plus 0.3pt}%
  \let\item\@idxitem
}{%
  \ifkorrekturansicht\clearpage\fi
}
\makeatother

\IfFileExists{\jobname-pw.ind}{\input{\jobname-pw.ind}}{}

% Quellenangabe nur in der Leseansicht
\ifkorrekturansicht\else
% Fallback-Definitionen, falls die .tex-Datei \titel etc. nicht gesetzt hat
\providecommand{\titel}{}
\providecommand{\editorInnen}{}
\providecommand{\dateiname}{\jobname}

\vspace{3cm}

\vfill

\footnotesize
\textsc{Quelle}: \titel. Herausgegeben von {\editorInnen}. In: \emph{Arthur Schnitzler: Briefwechsel mit Autorinnen und Autoren}.
 Digitale Edition, https://schnitzler-briefe.acdh.oeaw.ac.at/{\dateiname}.html (Stand \today)
\fi

\end{document}


      