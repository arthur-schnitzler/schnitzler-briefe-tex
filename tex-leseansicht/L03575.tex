%% latex-korrekturansicht-vorspann.tex
%% Vorspann für die Korrekturansicht.
%% Lädt die gemeinsame Datei latex-vorspann.tex mit gesetztem Schalter.

\newif\ifkorrekturansicht
\korrekturansichttrue

\input{../tex-inputs/latex-vorspann}


\section[Felix Salten u. a. an Arthur und Olga Schnitzler, {[}Ende Juli – 24. 8. 1912?{]}]{L03575 Felix Salten u. a. an Arthur und Olga Schnitzler, {[}Ende Juli –
               24. 8. 1912?{]}}
\nopagebreak\mylabel{L03575v}
\rehead{ }\normalsize\beginnumbering\briefempfaengerindex{Schnitzler, Olga@\textsc{Schnitzler, Olga}!zzzMetzl, Richard@\emph{von Richard Metzl}!1912-08-242@{{[}Ende Juli –
                  24. 8. 1912?{]}}|(be}\briefempfaengerindex{Schnitzler, Olga@\textsc{Schnitzler, Olga}!zzzJarosy, Helene@\emph{von Helene Jarosy}!1912-08-242@{{[}Ende Juli –
                  24. 8. 1912?{]}}|(be}\briefempfaengerindex{Schnitzler, Olga@\textsc{Schnitzler, Olga}!zzzWollf, Julius Ferdinand@\emph{von Julius Ferdinand Wollf}!1912-08-242@{{[}Ende Juli –
                  24. 8. 1912?{]}}|(be}\briefempfaengerindex{Schnitzler, Olga@\textsc{Schnitzler, Olga}!zzzSalten, Ottilie@\emph{von Ottilie Salten}!1912-08-242@{{[}Ende Juli –
                  24. 8. 1912?{]}}|(be}\briefempfaengerindex{Schnitzler, Olga@\textsc{Schnitzler, Olga}!zzzSalten, Felix@\emph{von Felix Salten}!1912-08-242@{{[}Ende Juli –
                  24. 8. 1912?{]}}|(be}\briefempfaengerindex{Schnitzler, Arthur@\textsc{Schnitzler, Arthur}!zzzMetzl, Richard@\emph{von Richard Metzl}!1912-08-242@{{[}Ende Juli –
                  24. 8. 1912?{]}}|(be}\briefempfaengerindex{Schnitzler, Arthur@\textsc{Schnitzler, Arthur}!zzzJarosy, Helene@\emph{von Helene Jarosy}!1912-08-242@{{[}Ende Juli –
                  24. 8. 1912?{]}}|(be}\briefempfaengerindex{Schnitzler, Arthur@\textsc{Schnitzler, Arthur}!zzzWollf, Julius Ferdinand@\emph{von Julius Ferdinand Wollf}!1912-08-242@{{[}Ende Juli –
                  24. 8. 1912?{]}}|(be}\briefempfaengerindex{Schnitzler, Arthur@\textsc{Schnitzler, Arthur}!zzzSalten, Ottilie@\emph{von Ottilie Salten}!1912-08-242@{{[}Ende Juli –
                  24. 8. 1912?{]}}|(be}\briefempfaengerindex{Schnitzler, Arthur@\textsc{Schnitzler, Arthur}!zzzSalten, Felix@\emph{von Felix Salten}!1912-08-242@{{[}Ende Juli –
                  24. 8. 1912?{]}}|(be}
\toendnotes[C]{\smallbreak\pagebreak[2]}\Standort{CUL, Schnitzler, B 89, B 2.}
\physDesc{Bildpostkarte, 403 Zeichen
\newline{}Handschrift Felix Salten: schwarze Tinte, lateinische Kurrent
\newline{}Handschrift Ottilie Salten: schwarze Tinte, lateinische Kurrent
\newline{}Handschrift Julius Ferdinand Wollf: schwarze Tinte, lateinische Kurrent
\newline{}Handschrift Helene Jarosy: schwarze Tinte, lateinische Kurrent
\newline{}Handschrift Richard Metzl: schwarze Tinte, deutsche Kurrent
\newline{}Versand: Stempel: »\nobreak{}\oindex{Unterach am Attersee@\textbf{Unterach am Attersee}, \emph{P.PPL}|pwk}Unter\textcolor{gray}{ach am Attersee}\nobreak{}«.  
\newline{}Ordnung: mit Bleistift von unbekannter Hand nummeriert: »288« }\toendnotes[C]{\smallbreak}\pstart{}{\pb}Herrn u. Frau\pend{}\pstart{}D\textsuperscript{r} Arthur Schnitzler\pend{}\pstart{}Brioni\oindex{Brijuni@\textbf{Brijuni}, \emph{P.PPL}|pw}\pend{}{\bigskip}
\pstart
           \noindent{}\centering{}{\pb}\textcolor{gray}{\textbf{Salzkammergut\oindex{Salzkammergut@\textbf{Salzkammergut}, \emph{L.RGN}|pw}. Blick vom Brennerriesensteig\oindex{Brennerriesensteig@\textbf{Brennerriesensteig}, \emph{Wanderweg (K.WND)}|pw} bei Steinbach auf den Attersee\oindex{Steinbach am Attersee@\textbf{Steinbach am Attersee}, \emph{A.ADM3}|pw} u. Schafberg\oindex{Schafberg [St. Gilgen]@\textbf{Schafberg [St. Gilgen]}, \emph{T.MT}|pw}.}}\pend
           \vspace{1em}
\pstart
           \noindent{}{\pb}Lieber Arthur und liebe Olga, wir haben \label{K_L03575-1v}\edtext{heute}{\lemma{\textnormal{\emph{heute}}}\Cendnote{\textnormal{Die Bildpostkarte ist undatiert und der
                  Stempel nur teilweise gedruckt. In Frage kommen zwei längere Aufenthalte Schnitzlers in Brijuni\oindex{Brijuni@\textbf{Brijuni}, \emph{P.PPL}|pwk}: vom 21. 7. 1912 bis zum 24. 8. 1912 und, im
                  Folgejahr, vom 24. 7. 1913 bis zum 22. 8. 1913. Für beide Jahre ist im \emph{Tagebuch}\pwindex{Tagebuch@\emph{Tagebuch}|pwk} keine persönliche Interaktion
                  zwischen Schnitzler und Salten\pwindex{Salten, Felix 06.09.1869 – 08.10.1945@\textsc{Salten, Felix} (06.09.1869 – 08.10.1945), \emph{Schriftsteller/Schriftstellerin, Journalist/Journalistin, Chefredakteur/Chefredakteurin}|pwk} rund um diese Zeiträume festgehalten. Nur für
                  das Jahr 1912 liegen Korrespondenzstücke vor (Felix Salten an Arthur Schnitzler, 2. 7. 1912, 22. 7. 1912; Felix Salten an Olga Schnitzler, 2. 9. 1912), die belegen, dass ein Austausch
                  stattfand. Das wird als entscheidendes Indiz gewertet, dass diese Karte im Jahr zu verorten ist. Auch lässt sich für 1912
                  ein dreiwöchiger Besuch des Ehepaars
                     Wollf\pwindex{Wollf, Julius Ferdinand 22.05.1871 – 01.03.1942@\textsc{Wollf, Julius Ferdinand} (22.05.1871 – 01.03.1942), \emph{Journalist/Journalistin, Herausgeber/Herausgeberin, Verleger/Verlegerin}|pwk}\pwindex{Wollf, Johanna Sophie 1877-10-18 – 1942-02-27@\textsc{Wollf, Johanna Sophie} (1877-10-18 – 1942-02-27)|pwk} belegen (siehe Felix Salten an Olga Schnitzler, 2. 9. 1912). Damit ist die Karte aber nach Saltens\pwindex{Salten, Felix 06.09.1869 – 08.10.1945@\textsc{Salten, Felix} (06.09.1869 – 08.10.1945), \emph{Schriftsteller/Schriftstellerin, Journalist/Journalistin, Chefredakteur/Chefredakteurin}|pwk}
                  Brief vom 22. 7. 1912
                  einzuordnen, da dieser mit Schnitzlers Urlaubsbeginn zusammenfällt und darin keine Anwesenheit weiterer Freunde thematisiert
                  wird. Nach hinten ist die Datierung durch Schnitzlers Abreise am 24. 8. 1912 eingrenzbar.}}}\label{K_L03575-1} in Herzlichkeit
               Ihrer gedacht und senden Ihnen viele Grüße! Hoffentlich haben Sie mit den Kinder\pwindex{Schnitzler, Heinrich 09.08.1902 – 12.07.1982@\textsc{Schnitzler, Heinrich} (09.08.1902 – 12.07.1982), \emph{Regisseur/Regisseurin, Schauspieler/Schauspielerin}|pwv}\pwindex{Cappellini, Lili 13.09.1909 – 26.07.1928@\textsc{Cappellini, Lili} (13.09.1909 – 26.07.1928)|pwv}n schöne Tage.
               Herzlichst Ihr {\\}\spacefill\mbox{Salten}\pend
           \selectlanguage{ngerman}\vspace{1em}
\pstart
           \noindent{}{[}hs. :{]} Viele herzliche Grüße \spacefill\mbox{Ottilie}\pend
           \selectlanguage{ngerman}\vspace{1em}
\pstart
           \noindent{}{[}hs. :{]} Viele Grüsse von Ihrem ergebenen\pend
           \pstart \spacefill\mbox{Julius Ferdinand Wollf} und seiner Frau\pwindex{Wollf, Johanna Sophie 1877-10-18 – 1942-02-27@\textsc{Wollf, Johanna Sophie} (1877-10-18 – 1942-02-27)|pwv}\pend{}\selectlanguage{ngerman}\vspace{1em}
\pstart
           \noindent{}{[}hs. :{]} Die schönsten Grüße Ihnen und der gnädigen Frau
                  \spacefill\mbox{Helene Jaroſy}\pend
           \selectlanguage{ngerman}\vspace{1em}
\pstart
           \noindent{}{[}hs. :{]} Beſte Grüße {\\}Ihr ergebener {\\}\spacefill\mbox{RichardMetzl}\pend
           \selectlanguage{ngerman}\endnumbering\briefempfaengerindex{Schnitzler, Olga@\textsc{Schnitzler, Olga}!zzzMetzl, Richard@\emph{von Richard Metzl}!1912-07-222@{{[}Ende Juli –
                  24. 8. 1912?{]}}|)be}\briefempfaengerindex{Schnitzler, Olga@\textsc{Schnitzler, Olga}!zzzJarosy, Helene@\emph{von Helene Jarosy}!1912-07-222@{{[}Ende Juli –
                  24. 8. 1912?{]}}|)be}\briefempfaengerindex{Schnitzler, Olga@\textsc{Schnitzler, Olga}!zzzWollf, Julius Ferdinand@\emph{von Julius Ferdinand Wollf}!1912-07-222@{{[}Ende Juli –
                  24. 8. 1912?{]}}|)be}\briefempfaengerindex{Schnitzler, Olga@\textsc{Schnitzler, Olga}!zzzSalten, Ottilie@\emph{von Ottilie Salten}!1912-07-222@{{[}Ende Juli –
                  24. 8. 1912?{]}}|)be}\briefempfaengerindex{Schnitzler, Olga@\textsc{Schnitzler, Olga}!zzzSalten, Felix@\emph{von Felix Salten}!1912-07-222@{{[}Ende Juli –
                  24. 8. 1912?{]}}|)be}\briefempfaengerindex{Schnitzler, Arthur@\textsc{Schnitzler, Arthur}!zzzMetzl, Richard@\emph{von Richard Metzl}!1912-07-222@{{[}Ende Juli –
                  24. 8. 1912?{]}}|)be}\briefempfaengerindex{Schnitzler, Arthur@\textsc{Schnitzler, Arthur}!zzzJarosy, Helene@\emph{von Helene Jarosy}!1912-07-222@{{[}Ende Juli –
                  24. 8. 1912?{]}}|)be}\briefempfaengerindex{Schnitzler, Arthur@\textsc{Schnitzler, Arthur}!zzzWollf, Julius Ferdinand@\emph{von Julius Ferdinand Wollf}!1912-07-222@{{[}Ende Juli –
                  24. 8. 1912?{]}}|)be}\briefempfaengerindex{Schnitzler, Arthur@\textsc{Schnitzler, Arthur}!zzzSalten, Ottilie@\emph{von Ottilie Salten}!1912-07-222@{{[}Ende Juli –
                  24. 8. 1912?{]}}|)be}\briefempfaengerindex{Schnitzler, Arthur@\textsc{Schnitzler, Arthur}!zzzSalten, Felix@\emph{von Felix Salten}!1912-07-222@{{[}Ende Juli –
                  24. 8. 1912?{]}}|)be}\mylabel{L03575h}  \normalsize

\doendnotes{C}
\bigskip
\vfill

\clearpage

\footnotesize

\lohead{\textsc{register}}

% Definiere theindex-Environment komplett neu ohne reledmac
\makeatletter
\renewenvironment{theindex}{%
  \section*{\indexname}%
  \setlength{\parindent}{0pt}%
  \setlength{\parskip}{0pt plus 0.3pt}%
  \let\item\@idxitem
}{%
  \clearpage
}
\makeatother

\IfFileExists{\jobname-pw.ind}{\input{\jobname-pw.ind}}{}

\end{document}

      