%% latex-leseansicht-vorspann.tex
%% Vorspann für die Leseansicht.
%% Lädt die gemeinsame Datei latex-vorspann.tex mit nicht gesetztem Schalter.

\newif\ifkorrekturansicht
\korrekturansichtfalse

\input{../tex-inputs/latex-vorspann}


         
         \renewcommand{\erwaehntePersonen}{Personen: Lili Cappellini, Helene Jarosy, Richard Metzl, Felix Salten, Ottilie Salten, Olga Schnitzler, Heinrich Schnitzler, Julius Ferdinand Wollf, Johanna Sophie Wollf}
         \renewcommand{\erwaehnteOrte}{Orte: Brennerriesensteig, Brijuni, Salzkammergut, Schafberg (St. Gilgen), Steinbach am Attersee, Unterach am Attersee}
         \renewcommand{\erwaehnteWerke}{Werke: Tagebuch}
               \section[Felix Salten u. a. an Arthur und Olga Schnitzler, {[}Ende Juli – 24. 8. 1912?{]}]{ Felix Salten u. a. an Arthur und Olga Schnitzler, {[}Ende Juli –
               24. 8. 1912?{]}}\nopagebreak\mylabel{v}\rehead{ }\begin{ledgroupsized}[t]{13cm}\normalsize\beginnumbering\briefempfaengerindex{Schnitzler, Olga@\textsc{Schnitzler, Olga}!zzzMetzl, Richard@\emph{von Richard Metzl}!1912-08-242@{{[}Ende Juli –
                  24. 8. 1912?{]}}|(be}\briefempfaengerindex{Schnitzler, Olga@\textsc{Schnitzler, Olga}!zzzJarosy, Helene@\emph{von Helene Jarosy}!1912-08-242@{{[}Ende Juli –
                  24. 8. 1912?{]}}|(be}\briefempfaengerindex{Schnitzler, Olga@\textsc{Schnitzler, Olga}!zzzWollf, Julius Ferdinand@\emph{von Julius Ferdinand Wollf}!1912-08-242@{{[}Ende Juli –
                  24. 8. 1912?{]}}|(be}\briefempfaengerindex{Schnitzler, Olga@\textsc{Schnitzler, Olga}!zzzSalten, Ottilie@\emph{von Ottilie Salten}!1912-08-242@{{[}Ende Juli –
                  24. 8. 1912?{]}}|(be}\briefempfaengerindex{Schnitzler, Olga@\textsc{Schnitzler, Olga}!zzzSalten, Felix@\emph{von Felix Salten}!1912-08-242@{{[}Ende Juli –
                  24. 8. 1912?{]}}|(be}\briefempfaengerindex{Schnitzler, Arthur@\textsc{Schnitzler, Arthur}!zzzMetzl, Richard@\emph{von Richard Metzl}!1912-08-242@{{[}Ende Juli –
                  24. 8. 1912?{]}}|(be}\briefempfaengerindex{Schnitzler, Arthur@\textsc{Schnitzler, Arthur}!zzzJarosy, Helene@\emph{von Helene Jarosy}!1912-08-242@{{[}Ende Juli –
                  24. 8. 1912?{]}}|(be}\briefempfaengerindex{Schnitzler, Arthur@\textsc{Schnitzler, Arthur}!zzzWollf, Julius Ferdinand@\emph{von Julius Ferdinand Wollf}!1912-08-242@{{[}Ende Juli –
                  24. 8. 1912?{]}}|(be}\briefempfaengerindex{Schnitzler, Arthur@\textsc{Schnitzler, Arthur}!zzzSalten, Ottilie@\emph{von Ottilie Salten}!1912-08-242@{{[}Ende Juli –
                  24. 8. 1912?{]}}|(be}\briefempfaengerindex{Schnitzler, Arthur@\textsc{Schnitzler, Arthur}!zzzSalten, Felix@\emph{von Felix Salten}!1912-08-242@{{[}Ende Juli –
                  24. 8. 1912?{]}}|(be} \toendnotes[C]{\smallbreak\pagebreak[2]} \Standort{CUL, Schnitzler, B 89, B 2.}
\physDesc{Bildpostkarte, 403 Zeichen
\newline{}Handschrift Felix Salten: schwarze Tinte, lateinische Kurrent\newline{}Handschrift Ottilie Salten: schwarze Tinte, lateinische Kurrent\newline{}Handschrift Julius Ferdinand Wollf: schwarze Tinte, lateinische Kurrent\newline{}Handschrift Helene Jarosy: schwarze Tinte, lateinische Kurrent\newline{}Handschrift Richard Metzl: schwarze Tinte, deutsche Kurrent
\newline{}Versand: Stempel: »\nobreak{}\oindex{Unterach am Attersee@\textbf{Unterach am Attersee}|pwk}Unter\textcolor{gray}{ach am Attersee}\nobreak{}«.  
\newline{}Ordnung: mit Bleistift von unbekannter Hand nummeriert: »288« }\toendnotes[C]{\smallbreak}\pstart{}{\pb}Herrn u. Frau\pend{}\pstart{}D\textsuperscript{r} Arthur Schnitzler\pend{}\pstart{}Brioni\oindex{Brijuni@\textbf{Brijuni}|pw}\pend{}{\bigskip}\pstart
           \noindent{}\centering{}{\pb}\textcolor{gray}{\textbf{Salzkammergut\oindex{Salzkammergut@\textbf{Salzkammergut}|pw}. Blick vom Brennerriesensteig\oindex{Brennerriesensteig@\textbf{Brennerriesensteig}|pw} bei Steinbach auf den Attersee\oindex{Steinbach am Attersee@\textbf{Steinbach am Attersee}|pw} u. Schafberg\oindex{Schafberg (St. Gilgen)@\textbf{Schafberg (St. Gilgen)}|pw}.}}\pend
           \pstart
           {\pb}Lieber Arthur und liebe Olga, wir haben \label{K_L03575-1v}\edtext{heute}{\lemma{\textnormal{\emph{heute}}}\Cendnote{\textnormal{Die Bildpostkarte ist undatiert und der
                  Stempel nur teilweise gedruckt. In Frage kommen zwei längere Aufenthalte Schnitzlers\pwindex{Schnitzler, Arthur 15.05.1862 – 21.10.1931@\textsc{Schnitzler, Arthur} (15.05.1862 – 21.10.1931), \emph{Schriftsteller, Mediziner}|pwk} in Brijuni\oindex{Brijuni@\textbf{Brijuni}|pwk}: vom 21. 7. 1912 bis zum 24. 8. 1912 und, im
                  Folgejahr, vom 24. 7. 1913 bis zum 22. 8. 1913. Für beide Jahre ist im \emph{Tagebuch}\pwindex{\textcolor{red}{\textsuperscript{XXXX1 indx}}!Tagebuch1981 – 2000@\strich\emph{Tagebuch} {[}Hrsg., 1981 – 2000{]}|pwk} keine persönliche Interaktion
                  zwischen Schnitzler\pwindex{Schnitzler, Arthur 15.05.1862 – 21.10.1931@\textsc{Schnitzler, Arthur} (15.05.1862 – 21.10.1931), \emph{Schriftsteller, Mediziner}|pwk} und Salten\pwindex{Salten, Felix 06.09.1869 – 08.10.1945@\textsc{Salten, Felix} (06.09.1869 – 08.10.1945), \emph{Schriftsteller, Journalist, Chefredakteur}|pwk} rund um diese Zeiträume festgehalten. Nur für
                  das Jahr 1912 liegen Korrespondenzstücke vor (Felix Salten an Arthur Schnitzler, 2. 7. 1912, 22. 7. 1912; Felix Salten an Olga Schnitzler, 2. 9. 1912), die belegen, dass ein Austausch
                  stattfand. Das wird als entscheidendes Indiz gewertet, dass diese Karte im Jahr zu verorten ist. Auch lässt sich für 1912
                  ein dreiwöchiger Besuch des Ehepaars
                     Wollf\pwindex{Wollf, Julius Ferdinand 22.05.1871 – 01.03.1942@\textsc{Wollf, Julius Ferdinand} (22.05.1871 – 01.03.1942), \emph{Journalist, Herausgeber, Verleger}|pwk}\pwindex{Wollf, Johanna Sophie 1877-10-18 – 1942-02-27@\textsc{Wollf, Johanna Sophie} (1877-10-18 – 1942-02-27)|pwk} belegen (siehe Felix Salten an Olga Schnitzler, 2. 9. 1912). Damit ist die Karte aber nach Saltens\pwindex{Salten, Felix 06.09.1869 – 08.10.1945@\textsc{Salten, Felix} (06.09.1869 – 08.10.1945), \emph{Schriftsteller, Journalist, Chefredakteur}|pwk}
                  Brief vom 22. 7. 1912
                  einzuordnen, da dieser mit Schnitzlers\pwindex{Schnitzler, Arthur 15.05.1862 – 21.10.1931@\textsc{Schnitzler, Arthur} (15.05.1862 – 21.10.1931), \emph{Schriftsteller, Mediziner}|pwk} Urlaubsbeginn zusammenfällt und darin keine Anwesenheit weiterer Freunde thematisiert
                  wird. Nach hinten ist die Datierung durch Schnitzlers\pwindex{Schnitzler, Arthur 15.05.1862 – 21.10.1931@\textsc{Schnitzler, Arthur} (15.05.1862 – 21.10.1931), \emph{Schriftsteller, Mediziner}|pwk} Abreise am 24. 8. 1912 eingrenzbar.}}}\label{K_L03575-1h} in Herzlichkeit
               Ihrer gedacht und senden Ihnen viele Grüße! Hoffentlich haben Sie mit den Kinder\pwindex{Schnitzler, Heinrich 09.08.1902 – 12.07.1982@\textsc{Schnitzler, Heinrich} (09.08.1902 – 12.07.1982), \emph{Regisseur, Schauspieler}|pwv}\pwindex{Cappellini, Lili 13.09.1909 – 26.07.1928@\textsc{Cappellini, Lili} (13.09.1909 – 26.07.1928)|pwv}n schöne Tage.
               Herzlichst Ihr {\\}\spacefill\mbox{Salten}\pend
           \pstart
           \noindent{}{[}hs. Ottilie Salten:{]} Viele herzliche Grüße \spacefill\mbox{Ottilie}\pend
           \pstart
           \noindent{}{[}hs. Wollf:{]} Viele Grüsse von Ihrem ergebenen\pend
           \pstart \spacefill\mbox{Julius Ferdinand Wollf} und seiner Frau\pwindex{Wollf, Johanna Sophie 1877-10-18 – 1942-02-27@\textsc{Wollf, Johanna Sophie} (1877-10-18 – 1942-02-27)|pwv}\pend{}\pstart
           \noindent{}{[}hs. Jarosy:{]} Die schönsten Grüße Ihnen und der gnädigen Frau
                  \spacefill\mbox{Helene Jaroſy}\pend
           \pstart
           \noindent{}{[}hs. Metzl:{]} Beſte Grüße {\\}Ihr ergebener {\\}\spacefill\mbox{RichardMetzl}\pend
           
         
         \endnumbering\mylabel{h}\end{ledgroupsized}  \newcommand{\dateiname}{L03575}\newcommand{\titel}{Felix Salten u. a. an Arthur und Olga Schnitzler, [Ende Juli – 24. 8. 1912?]}\newcommand{\editorInnen}{Martin Anton Müller und Laura Untner}%% latex-leseansicht-abspann.tex
%% Abspann für die Leseansicht.
%% Der Schalter \ifkorrekturansicht ist bereits durch den Vorspann gesetzt.

%% latex-abspann.tex
%% Gemeinsamer Abspann für Korrekturansicht und Leseansicht.
%% Setzt den Schalter \ifkorrekturansicht voraus (gesetzt in den
%% einbindenden Dateien latex-korrekturansicht-abspann.tex bzw.
%% latex-leseansicht-abspann.tex).
%% ---------------------------------------------------------------

\normalsize

% Das esempio-Environment wird nur in der Leseansicht benötigt
\ifkorrekturansicht\else
\newenvironment{esempio}[3]%
{
    \vspace{1.5ex}
    \rlap{\underline{#1}}
    \par
    \setlength{\parindent}{0cm}
    \nopagebreak
    \leftskip=#2cm
    \rightskip=#3cm
}
{
    \par
}
\fi

\doendnotes{C}
\bigskip
\vfill

\clearpage

\footnotesize

\ifkorrekturansicht
  \lohead{\textsc{register}}
\fi

% theindex-Environment neu definieren ohne reledmac
\makeatletter
\renewenvironment{theindex}{%
  \ifkorrekturansicht
    \section*{\indexname}%
  \else
    \subsubsection*{Index der erwähnten Entitäten}%
  \fi
  \setlength{\parindent}{0pt}%
  \setlength{\parskip}{0pt plus 0.3pt}%
  \let\item\@idxitem
}{%
  \ifkorrekturansicht\clearpage\fi
}
\makeatother

\IfFileExists{\jobname-pw.ind}{\input{\jobname-pw.ind}}{}

% Quellenangabe nur in der Leseansicht
\ifkorrekturansicht\else
% Fallback-Definitionen, falls die .tex-Datei \titel etc. nicht gesetzt hat
\providecommand{\titel}{}
\providecommand{\editorInnen}{}
\providecommand{\dateiname}{\jobname}

\vspace{3cm}

\vfill

\footnotesize
\textsc{Quelle}: \titel. Herausgegeben von {\editorInnen}. In: \emph{Arthur Schnitzler: Briefwechsel mit Autorinnen und Autoren}.
 Digitale Edition, https://schnitzler-briefe.acdh.oeaw.ac.at/{\dateiname}.html (Stand \today)
\fi

\end{document}


      