%% latex-leseansicht-vorspann.tex
%% Vorspann für die Leseansicht.
%% Lädt die gemeinsame Datei latex-vorspann.tex mit nicht gesetztem Schalter.

\newif\ifkorrekturansicht
\korrekturansichtfalse

\input{../tex-inputs/latex-vorspann}


\section[Arthur Schnitzler u. a. an Gustav Schwarzkopf, {[}zwischen 19. 11. 1899 und 4. 12. 1899?{]}]{L04175 Arthur Schnitzler u. a. an Gustav Schwarzkopf, {[}zwischen 19. 11. 1899 und 4. 12. 1899?{]}}
\nopagebreak\mylabel{L04175v}
\rehead{ }\normalsize\beginnumbering\briefempfaengerindex{Schwarzkopf, Gustav@\textsc{Schwarzkopf, Gustav}!zzzWolff, Ludwig@\emph{von Ludwig Wolff}!1899-12-041@{{[}zwischen 19. 11. 1899 und 4. 12. 1899?{]}}|(be}\briefempfaengerindex{Schwarzkopf, Gustav@\textsc{Schwarzkopf, Gustav}!zzzWassermann, Jakob@\emph{von Jakob Wassermann}!1899-12-041@{{[}zwischen 19. 11. 1899 und 4. 12. 1899?{]}}|(be}\briefempfaengerindex{Schwarzkopf, Gustav@\textsc{Schwarzkopf, Gustav}!zzzBeer-Hofmann, Richard@\emph{von Richard Beer-Hofmann}!1899-12-041@{{[}zwischen 19. 11. 1899 und 4. 12. 1899?{]}}|(be}\briefempfaengerindex{Schwarzkopf, Gustav@\textsc{Schwarzkopf, Gustav}!zzzSchnitzler, Arthur@\emph{von Arthur Schnitzler}!1899-12-041@{{[}zwischen 19. 11. 1899 und 4. 12. 1899?{]}}|(be}
\toendnotes[C]{\smallbreak\pagebreak[2]}
\correspDesc{Versand  durch Arthur Schnitzler, Richard Beer-Hofmann, Jakob Wassermann, Ludwig Wolff im Zeitraum [zwischen 19. 11. 1899 und 4. 12. 1899?] in Wien
\newline{}Erhalt  durch Gustav Schwarzkopf in Wien}\toendnotes[C]{\smallbreak}
\Standort{CUL, Schnitzler, B 96.}
\physDesc{Brief, 1 Blatt, 2 Seiten, 293 Zeichen
\newline{}Handschrift Arthur Schnitzler: Bleistift, deutsche Kurrent
\newline{}Handschrift Richard Beer-Hofmann: Bleistift
\newline{}Handschrift Jakob Wassermann: Bleistift
\newline{}Handschrift Ludwig Wolff: Bleistift}\toendnotes[C]{\smallbreak}
\pstart{}{\pb}lieber Guſtav!\pend\vspace{0.5em}
\pstart
           Wir beſchwören Sie! \label{K_L04175-1v}\edtext{treten Sie in den Club\orgindex{Wiener Schachclub@Wiener Schachclub|pwv}}{\lemma{\textnormal{\emph{treten Sie in den Club}}}\Cendnote{\textnormal{Der vorliegende Brief über den bevorstehenden Eintritt
                     in den \emph{Wiener Schachclub}\orgindex{Wiener Schachclub@Wiener Schachclub|pwk} ist undatiert. Er ist aller Wahrscheinlichkeit nach vor dem 5. 12. 1899
                     verfasst sein, da Schnitzler an diesem Tag erstmals im \emph{Schachclub}\orgindex{Wiener Schachclub@Wiener Schachclub|pwk} war (vgl. XXXX Auszeichnungsfehler: Dokument L04173 nicht gefunden).
                     Nachdem Felix Salten\pwindex{Salten, Felix 6.\,9.\,1869 Budapest – 8.\,10.\,1945 Zürich@\textsc{Salten, Felix} (6.\,9.\,1869 Budapest – 8.\,10.\,1945 Zürich), \emph{Schriftsteller, Journalist, Chefredakteur}|pwk} am XXXX Auszeichnungsfehler: Dokument L03302 nicht gefunden noch Bedingungen für seinen
                     Eintritt stellt, ist das auch der Zeitpunkt, der diesen Brief nach vorne beschränkt.}}}\label{K_L04175-1} ein! Wir alle ſind entſchloſſen\pend
           
\pstart
           Liſte: Hugo\pwindex{Hofmannsthal, Hugo von 1.\,2.\,1874 Wien – 15.\,7.\,1929 Rodaun@\textsc{Hofmannsthal, Hugo von} (1.\,2.\,1874 Wien – 15.\,7.\,1929 Rodaun), \emph{Schriftsteller}|pw}, Richard, Salten\pwindex{Salten, Felix 6.\,9.\,1869 Budapest – 8.\,10.\,1945 Zürich@\textsc{Salten, Felix} (6.\,9.\,1869 Budapest – 8.\,10.\,1945 Zürich), \emph{Schriftsteller, Journalist, Chefredakteur}|pw}, Robert Hirschf\pwindex{Hirschfeld, Robert 17.\,9.\,1857 Žďár nad Sázavou – 2.\,4.\,1914 Salzburg@\textsc{Hirschfeld, Robert} (17.\,9.\,1857 Žďár nad Sázavou – 2.\,4.\,1914 Salzburg), \emph{Journalist, Musikkritiker}|pw}
               (?= wegen \textsc{Leo Vanjung\pwindex{Van-Jung, Leo 15.\,10.\,1866 Odessa – 2.\,7.\,1939 Riga@\textsc{Van-Jung, Leo} (15.\,10.\,1866 Odessa – 2.\,7.\,1939 Riga), \emph{Gesangspädagoge, Mathematiker}|pw}}), Waſſerma{\geminationn},  Wolf. ich.\pend
           \pstart {\pb}Die Beſchwörenden:
                  \spacefill\mbox{Arthur}\spacefill\mbox{{[}hs. Beer-Hofmann:{]} Richard}\spacefill\mbox{{[}hs. Wassermann:{]} Jakob Wassermann}\spacefill\mbox{{[}hs. Wolff:{]} LudwigWolff}\pend{}
\pstart
           \noindent{}{[}hs. Schnitzler:{]} (die anderen\pwindex{Hofmannsthal, Hugo von 1.\,2.\,1874 Wien – 15.\,7.\,1929 Rodaun@\textsc{Hofmannsthal, Hugo von} (1.\,2.\,1874 Wien – 15.\,7.\,1929 Rodaun), \emph{Schriftsteller}|pwv}\pwindex{Salten, Felix 6.\,9.\,1869 Budapest – 8.\,10.\,1945 Zürich@\textsc{Salten, Felix} (6.\,9.\,1869 Budapest – 8.\,10.\,1945 Zürich), \emph{Schriftsteller, Journalist, Chefredakteur}|pwv}\pwindex{Hirschfeld, Robert 17.\,9.\,1857 Žďár nad Sázavou – 2.\,4.\,1914 Salzburg@\textsc{Hirschfeld, Robert} (17.\,9.\,1857 Žďár nad Sázavou – 2.\,4.\,1914 Salzburg), \emph{Journalist, Musikkritiker}|pwv} ſind nicht im \textsc{Café}haus, würden aber auch beſchwören.)\pend
           \selectlanguage{ngerman}\endnumbering\briefempfaengerindex{Schwarzkopf, Gustav@\textsc{Schwarzkopf, Gustav}!zzzWolff, Ludwig@\emph{von Ludwig Wolff}!1899-11-191@{{[}zwischen 19. 11. 1899 und 4. 12. 1899?{]}}|)be}\briefempfaengerindex{Schwarzkopf, Gustav@\textsc{Schwarzkopf, Gustav}!zzzWassermann, Jakob@\emph{von Jakob Wassermann}!1899-11-191@{{[}zwischen 19. 11. 1899 und 4. 12. 1899?{]}}|)be}\briefempfaengerindex{Schwarzkopf, Gustav@\textsc{Schwarzkopf, Gustav}!zzzBeer-Hofmann, Richard@\emph{von Richard Beer-Hofmann}!1899-11-191@{{[}zwischen 19. 11. 1899 und 4. 12. 1899?{]}}|)be}\briefempfaengerindex{Schwarzkopf, Gustav@\textsc{Schwarzkopf, Gustav}!zzzSchnitzler, Arthur@\emph{von Arthur Schnitzler}!1899-11-191@{{[}zwischen 19. 11. 1899 und 4. 12. 1899?{]}}|)be}\mylabel{L04175h}
\begin{anhang}
\end{anhang}\newcommand{\dateiname}{L04175}\newcommand{\titel}{Arthur Schnitzler u. a. an Gustav Schwarzkopf, [zwischen 19. 11. 1899 und 4. 12. 1899?]}\newcommand{\editorInnen}{Herausgegeben von Jahnke, SelmaMüller, Martin Anton}%% latex-leseansicht-abspann.tex
%% Abspann für die Leseansicht.
%% Der Schalter \ifkorrekturansicht ist bereits durch den Vorspann gesetzt.

%% latex-abspann.tex
%% Gemeinsamer Abspann für Korrekturansicht und Leseansicht.
%% Setzt den Schalter \ifkorrekturansicht voraus (gesetzt in den
%% einbindenden Dateien latex-korrekturansicht-abspann.tex bzw.
%% latex-leseansicht-abspann.tex).
%% ---------------------------------------------------------------

\normalsize

% Das esempio-Environment wird nur in der Leseansicht benötigt
\ifkorrekturansicht\else
\newenvironment{esempio}[3]%
{
    \vspace{1.5ex}
    \rlap{\underline{#1}}
    \par
    \setlength{\parindent}{0cm}
    \nopagebreak
    \leftskip=#2cm
    \rightskip=#3cm
}
{
    \par
}
\fi

\doendnotes{C}
\bigskip
\vfill

\clearpage

\footnotesize

\ifkorrekturansicht
  \lohead{\textsc{register}}
\fi

% theindex-Environment neu definieren ohne reledmac
\makeatletter
\renewenvironment{theindex}{%
  \ifkorrekturansicht
    \section*{\indexname}%
  \else
    \subsubsection*{Index der erwähnten Entitäten}%
  \fi
  \setlength{\parindent}{0pt}%
  \setlength{\parskip}{0pt plus 0.3pt}%
  \let\item\@idxitem
}{%
  \ifkorrekturansicht\clearpage\fi
}
\makeatother

\IfFileExists{\jobname-pw.ind}{\input{\jobname-pw.ind}}{}

% Quellenangabe nur in der Leseansicht
\ifkorrekturansicht\else
% Fallback-Definitionen, falls die .tex-Datei \titel etc. nicht gesetzt hat
\providecommand{\titel}{}
\providecommand{\editorInnen}{}
\providecommand{\dateiname}{\jobname}

\vspace{3cm}

\vfill

\footnotesize
\textsc{Quelle}: \titel. Herausgegeben von {\editorInnen}. In: \emph{Arthur Schnitzler: Briefwechsel mit Autorinnen und Autoren}.
 Digitale Edition, https://schnitzler-briefe.acdh.oeaw.ac.at/{\dateiname}.html (Stand \today)
\fi

\end{document}


