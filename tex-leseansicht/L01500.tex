%% latex-korrekturansicht-vorspann.tex
%% Vorspann für die Korrekturansicht.
%% Lädt die gemeinsame Datei latex-vorspann.tex mit gesetztem Schalter.

\newif\ifkorrekturansicht
\korrekturansichttrue

\input{../tex-inputs/latex-vorspann}


\section[Richard Beer-Hofmann an Arthur Schnitzler, 10. 2. 1905]{L01500 Richard Beer-Hofmann an Arthur Schnitzler, 10. 2. 1905}
\nopagebreak\mylabel{L01500v}
\rehead{ }\normalsize\beginnumbering\briefempfaengerindex{Schnitzler, Arthur@\textsc{Schnitzler, Arthur}!zzzBeer-Hofmann, Richard@\emph{von Richard Beer-Hofmann}!1905-02-101@{10. 2. 1905}|(be}
\toendnotes[C]{\smallbreak\pagebreak[2]}\Standort{CUL, Schnitzler, B 8.}
\physDesc{Bildpostkarte, 152 Zeichen
\newline{}Handschrift: Bleistift, lateinische Kurrent
\newline{}Versand: 1) Stempel: »\nobreak{}\oindex{Muenchen@\textbf{München}, \emph{P.PPLA}|pwk}München 16.P, 10 Feb 05, 3–4\nobreak{}«.   2) Stempel: »\nobreak{}\oindex{XVIII., Waehring@\textbf{XVIII., Währing}, \emph{A.ADM3}|pwk}18/1 Wien \textcolor{gray}{110}, 11. 2. 05, 8.V, Bestellt\nobreak{}«. 
\newline{}Ordnung: mit Bleistift von unbekannter Hand nummeriert:
                                    »198« }
\buchAbdrucke{\weitereDrucke{Arthur Schnitzler, Richard Beer-Hofmann: \emph{Briefwechsel 1891–1931}. Wien, Zürich: \emph{Europaverlag} 1992, S. 171.} }\toendnotes[C]{\smallbreak}\pstart{}{\pb}Herrn\pend{}\pstart{}D\textsuperscript{r} Arthur Schnitzler\pend{}\pstart{}Wien\oindex{Wien@\textbf{Wien}, \emph{A.ADM2}|pw}\pend{}\pstart{}XVIII Spöttelgasse 7\oindex{Edmund-Weiss-Gasse 7@\textbf{Edmund-Weiß-Gasse 7}, \emph{Wohngebäude (K.WHS)}|pw}\pend{}{\bigskip}
\pstart
           \noindent{}\centering{}{\pb}\textcolor{gray}{\textbf{Albrecht Dürer\pwindex{Duerer, Albrecht 21.05.1471 – 06.04.1528@\textsc{Dürer, Albrecht} (21.05.1471 – 06.04.1528), \emph{Maler/Malerin}|pw}-Serie}}\pend
           
\pstart
           \centering{}\textcolor{gray}{\textbf{Das kleine Pferd\pwindex{kleine Pferd@\emph{Das kleine Pferd}|pw}}}\pend
           \vspace{1em}
\pstart
           \noindent{}{\pb}Zur \label{K_L01500-1v}\edtext{Première\pwindex{Graf von Charolais. Ein Trauerspiel@\emph{Der Graf von Charolais. Ein Trauerspiel}|pwv}}{\lemma{\textnormal{\emph{Première}}}\Cendnote{\textnormal{\emph{Der Graf von Charolais}\pwindex{Graf von Charolais. Ein Trauerspiel@\emph{Der Graf von Charolais. Ein Trauerspiel}|pwk} hatte am 10. 2. 1905 am \emph{Münchner Residenztheater}\orgindex{Residenztheater Muenchen@Residenztheater München|pwk}
                   Premiere.}}}\label{K_L01500-1} gehen
               weder Paula\pwindex{Beer-Hofmann, Paula 25.02.1879 – 30.10.1939@\textsc{Beer-Hofmann, Paula} (25.02.1879 – 30.10.1939)|pw} noch ich. Die Generalprobe war zu
               schön. \label{T_L01500-1v}\edtext{↑ ist kein
                  Hoftheatermitglied}{\lemma{\textnormal{\emph{↑ … Hoftheatermitglied}}}\Cendnote{\textnormal{Der Pfeil zeigt auf
                  das Pferdemotiv der Karte.}}}\label{T_L01500-1}. \spacefill\mbox{Richard}\pend
           \selectlanguage{ngerman}\endnumbering\briefempfaengerindex{Schnitzler, Arthur@\textsc{Schnitzler, Arthur}!zzzBeer-Hofmann, Richard@\emph{von Richard Beer-Hofmann}!1905-02-101@{10. 2. 1905}|)be}\mylabel{L01500h}  \normalsize

\doendnotes{C}
\bigskip
\vfill

\clearpage

\footnotesize

\lohead{\textsc{register}}

% Definiere theindex-Environment komplett neu ohne reledmac
\makeatletter
\renewenvironment{theindex}{%
  \section*{\indexname}%
  \setlength{\parindent}{0pt}%
  \setlength{\parskip}{0pt plus 0.3pt}%
  \let\item\@idxitem
}{%
  \clearpage
}
\makeatother

\IfFileExists{\jobname-pw.ind}{\input{\jobname-pw.ind}}{}

\end{document}

      