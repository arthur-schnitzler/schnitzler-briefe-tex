%% latex-korrekturansicht-vorspann.tex
%% Vorspann für die Korrekturansicht.
%% Lädt die gemeinsame Datei latex-vorspann.tex mit gesetztem Schalter.

\newif\ifkorrekturansicht
\korrekturansichttrue

\input{../tex-inputs/latex-vorspann}


\section[Fedor Mamroth an Arthur Schnitzler, 2. 8. 1889]{L00001 Fedor Mamroth an Arthur Schnitzler,2. 8. 1889}
\nopagebreak\mylabel{L00001v}
\rehead{ }\normalsize\beginnumbering\briefempfaengerindex{Schnitzler, Arthur@\textsc{Schnitzler, Arthur}!zzzMamroth, Fedor@\emph{von Fedor Mamroth}!1889-08-021@{2. 8. 1889}|(be}
\toendnotes[C]{\smallbreak\pagebreak[2]}\Standort{CUL, Schnitzler, B 68.}
\physDesc{Brief, 1 Blatt, 1 Seite, 308 Zeichen
\newline{}Handschrift: blaue Tinte, deutsche Kurrent
\newline{}Schnitzler: 1) mit Bleistift nummeriert: »1.«  2) mit rotem Buntstift eine Unterstreichung}\toendnotes[C]{\smallbreak}
\pstart
           {\pb}\textcolor{gray}{\textbf{\textsc{Frankfurter Zeitung}}}{\\}\textcolor{gray}{\textbf{\textsc{und}}}{\\}\textcolor{gray}{\textbf{\textsc{Handelsblatt.}}}\orgindex{Frankfurter Zeitung@Frankfurter Zeitung|pw}\pend
           
\pstart
           \textcolor{gray}{\textbf{\textsc{Redaction.}}}\hfill \textcolor{gray}{\textbf{\textsc{Frankfurt a. M.\oindex{Frankfurt am Main@\textbf{Frankfurt am Main}|pw},}}}{ }2. Aug. \textcolor{gray}{\textbf{\textsc{188}}}9\pend
           
\pstart
           \textcolor{gray}{\textbf{\textsc{Telegramm-Adresse:}}}\pend
           
\pstart
           \textcolor{gray}{\textbf{\textsc{Zeitung Frankfurt Main}}}\pend
           
\pstart{}Hochgeehrter Herr Doctor!\pend\vspace{0.5em}
\pstart
           »\label{K_L00001-1v}\edtext{Der SohnSEXref\pwindex{Schnitzler, Arthur 15.\,5.\,1862 Wien – 21.\,10.\,1931 ebd.@\textsc{Schnitzler, Arthur} (15.\,5.\,1862 Wien – 21.\,10.\,1931 ebd.), \emph{Schriftsteller*in, Mediziner*in}!Sohn. Aus den Papieren eines Arztes@\strich\emph{Der Sohn. Aus den Papieren eines Arztes}|pw}}{\lemma{\textnormal{\emph{Der Sohn}}}\Cendnote{\textnormal{Die Erzählung entstand im Sommer 1889 (A. S.: \emph{Tagebuch}, 8. 9. 1889).}}}\label{K_L00001-1}«
               iſt leider auch mir zu düſter,{ }ſo kunſtvoll das pſychologiſche Motiv immer entwickelt
               iſt.\pend
           
\pstart
           Seien Sie mir nicht böſe, wenn ich Ihnen das MsSEXref\pwindex{Schnitzler, Arthur 15.\,5.\,1862 Wien – 21.\,10.\,1931 ebd.@\textsc{Schnitzler, Arthur} (15.\,5.\,1862 Wien – 21.\,10.\,1931 ebd.), \emph{Schriftsteller*in, Mediziner*in}!Sohn. Aus den Papieren eines Arztes@\strich\emph{Der Sohn. Aus den Papieren eines Arztes}|pwv} zurückſende, erfreuen Sie mich bald durch \label{K_L00001-2v}\edtext{einen anderen Beitrag}{\lemma{\textnormal{\emph{einen anderen Beitrag}}}\Cendnote{\textnormal{Erst am 24. 12. 1891 erschien
                  mit \emph{Weihnachts-Einkäufe}SEXref\pwindex{Schnitzler, Arthur 15.\,5.\,1862 Wien – 21.\,10.\,1931 ebd.@\textsc{Schnitzler, Arthur} (15.\,5.\,1862 Wien – 21.\,10.\,1931 ebd.), \emph{Schriftsteller*in, Mediziner*in}!Weihnachts-Einkaeufe@\strich\emph{Weihnachts-Einkäufe}|pwk} ein erster Beitrag
                     Schnitzlers in der \emph{Frankfurter Zeitung}\pwindex{Frankfurter Zeitung@\emph{Frankfurter Zeitung}|pwk} (Nr. 358, S. 1–2).}}}\label{K_L00001-2} u.
               empfangen Sie meine höflichſten Grüße.\pend
           
\pstart
           Ihr{\\[\baselineskip]}ergebener{\\[\baselineskip]}\spacefill\mbox{D\textsuperscript{r} FMamroth}\pend
           \leftskip=0em{}\selectlanguage{ngerman}\endnumbering\briefempfaengerindex{Schnitzler, Arthur@\textsc{Schnitzler, Arthur}!zzzMamroth, Fedor@\emph{von Fedor Mamroth}!1889-08-021@{2. 8. 1889}|)be}\mylabel{L00001h}  \normalsize

\doendnotes{C}
\bigskip
\vfill

\clearpage

\footnotesize

\lohead{\textsc{register}}

% Definiere theindex-Environment komplett neu ohne reledmac
\makeatletter
\renewenvironment{theindex}{%
  \section*{\indexname}%
  \setlength{\parindent}{0pt}%
  \setlength{\parskip}{0pt plus 0.3pt}%
  \let\item\@idxitem
}{%
  \clearpage
}
\makeatother

\IfFileExists{\jobname-pw.ind}{\input{\jobname-pw.ind}}{}

\end{document}

      