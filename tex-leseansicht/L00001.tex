%% latex-leseansicht-vorspann.tex
%% Vorspann für die Leseansicht.
%% Lädt die gemeinsame Datei latex-vorspann.tex mit nicht gesetztem Schalter.

\newif\ifkorrekturansicht
\korrekturansichtfalse

\input{../tex-inputs/latex-vorspann}


         
         \renewcommand{\erwaehntePersonen}{Personen: Paul Goldmann}
         \renewcommand{\erwaehnteInstitutionen}{Institutionen: Frankfurter Zeitung}
         \renewcommand{\erwaehnteOrte}{Orte: Frankfurt am Main, Wien}
         \renewcommand{\erwaehnteWerke}{Werke: Der Sohn. Aus den Papieren eines Arztes, Frankfurter Zeitung, Weihnachts-Einkäufe}
               \section[Fedor Mamroth an Arthur Schnitzler, 2. 8. 1889]{ Fedor Mamroth an Arthur Schnitzler, 2. 8. 1889}\nopagebreak\mylabel{v}\rehead{ }\begin{ledgroupsized}[t]{13cm}\normalsize\beginnumbering \toendnotes[C]{\smallbreak\pagebreak[2]} \Standort{CUL, Schnitzler, B 68.}
\physDesc{Brief, 1 Blatt, 1 Seite
\newline{}Handschrift Paul Goldmann: blaue Tinte, deutsche Kurrent
\newline{}Schnitzler: 1) mit Bleistift nummeriert: »1.«  2) mit rotem Buntstift eine Unterstreichung}\toendnotes[C]{\smallbreak}\pstart
           \noindent{}{\pb}\textcolor{gray}{\textbf{\textsc{Frankfurter Zeitung}}}{\\}\textcolor{gray}{\textbf{\textsc{und}}}{\\}\textcolor{gray}{\textbf{\textsc{Handelsblatt.}}}\orgindex{Frankfurter Zeitung@Frankfurter Zeitung|pw}\pend
           \pstart
           \textcolor{gray}{\textbf{\textsc{Redaction.}}}\hfill \textcolor{gray}{\textbf{\textsc{Frankfurt a. M.\oindex{Frankfurt am Main@\textbf{Frankfurt am Main}|pw},}}}{ }2. Aug. \textcolor{gray}{\textbf{\textsc{188}}}9\pend
           \pstart
           \textcolor{gray}{\textbf{\textsc{Telegramm-Adresse:}}}\pend
           \pstart
           \textcolor{gray}{\textbf{\textsc{Zeitung Frankfurt Main}}}\pend
           \pstart{}Hochgeehrter Herr Doctor!\pend\pstart
           \label{K_L00001_1v}\edtext{»Der
                  Sohn\pwindex{Schnitzler, Arthur 15.05.1862 – 21.10.1931@\textsc{Schnitzler, Arthur} (15.05.1862 – 21.10.1931), \emph{Schriftsteller, Mediziner}!Sohn. Aus den Papieren eines Arztes1. 1. 1892@\strich\emph{Der Sohn. Aus den Papieren eines Arztes} {[}1. 1. 1892{]}|pw}«}{\lemma{\textnormal{\emph{»Der
                  Sohn«}}}\Cendnote{\textnormal{Die Erzählung entstand im
                     Sommer 1889 (A. S.: \emph{Tagebuch}, 8. 9. 1889).}}}\label{K_L00001_1h} iſt leider auch mir zu düſter, ſo kunſtvoll das pſychologiſche Motiv
               immer entwickelt iſt.\pend
           \pstart
           Seien Sie mir nicht böſe, wenn ich Ihnen das Ms\pwindex{Schnitzler, Arthur 15.05.1862 – 21.10.1931@\textsc{Schnitzler, Arthur} (15.05.1862 – 21.10.1931), \emph{Schriftsteller, Mediziner}!Sohn. Aus den Papieren eines Arztes1. 1. 1892@\strich\emph{Der Sohn. Aus den Papieren eines Arztes} {[}1. 1. 1892{]}|pwv} zurückſende, erfreuen Sie mich bald durch \label{K_L00001_2v}\edtext{einen anderen Beitrag}{\lemma{\textnormal{\emph{einen anderen Beitrag}}}\Cendnote{\textnormal{Erst am 24. 12. 1891 erschien mit  \emph{Weihnachts-Einkäufe}\pwindex{Schnitzler, Arthur 15.05.1862 – 21.10.1931@\textsc{Schnitzler, Arthur} (15.05.1862 – 21.10.1931), \emph{Schriftsteller, Mediziner}!Weihnachts-Einkaeufe24. 12. 1891@\strich\emph{Weihnachts-Einkäufe} {[}24. 12. 1891{]}|pwk}
                  ein erster Beitrag Schnitzler\pwindex{Schnitzler, Arthur 15.05.1862 – 21.10.1931@\textsc{Schnitzler, Arthur} (15.05.1862 – 21.10.1931), \emph{Schriftsteller, Mediziner}|pwk}s in der \emph{Frankfurter Zeitung}\pwindex{?? Werk@Nicht ermittelte Verfasserinnen und Verfasser!Frankfurter Zeitung1856 – 1943@\emph{Frankfurter Zeitung} {[}1856 – 1943{]}|pwk} (Nr. 358, S. 1–2).}}}\label{K_L00001_2h} u. empfangen Sie
               meine höflichſten Grüße.\pend
           \pstart
           Ihr{\\[\baselineskip]}ergebener{\\[\baselineskip]}\spacefill\mbox{D\textsuperscript{r} FMamroth}\pend
           \leftskip=0em{}
         
         \endnumbering\mylabel{h}\end{ledgroupsized}  \newcommand{\dateiname}{L00001}\newcommand{\titel}{Fedor Mamroth an Arthur Schnitzler, 2. 8. 1889}\newcommand{\editorInnen}{Martin Anton Müller und Gerd-Hermann Susen}%% latex-leseansicht-abspann.tex
%% Abspann für die Leseansicht.
%% Der Schalter \ifkorrekturansicht ist bereits durch den Vorspann gesetzt.

%% latex-abspann.tex
%% Gemeinsamer Abspann für Korrekturansicht und Leseansicht.
%% Setzt den Schalter \ifkorrekturansicht voraus (gesetzt in den
%% einbindenden Dateien latex-korrekturansicht-abspann.tex bzw.
%% latex-leseansicht-abspann.tex).
%% ---------------------------------------------------------------

\normalsize

% Das esempio-Environment wird nur in der Leseansicht benötigt
\ifkorrekturansicht\else
\newenvironment{esempio}[3]%
{
    \vspace{1.5ex}
    \rlap{\underline{#1}}
    \par
    \setlength{\parindent}{0cm}
    \nopagebreak
    \leftskip=#2cm
    \rightskip=#3cm
}
{
    \par
}
\fi

\doendnotes{C}
\bigskip
\vfill

\clearpage

\footnotesize

\ifkorrekturansicht
  \lohead{\textsc{register}}
\fi

% theindex-Environment neu definieren ohne reledmac
\makeatletter
\renewenvironment{theindex}{%
  \ifkorrekturansicht
    \section*{\indexname}%
  \else
    \subsubsection*{Index der erwähnten Entitäten}%
  \fi
  \setlength{\parindent}{0pt}%
  \setlength{\parskip}{0pt plus 0.3pt}%
  \let\item\@idxitem
}{%
  \ifkorrekturansicht\clearpage\fi
}
\makeatother

\IfFileExists{\jobname-pw.ind}{\input{\jobname-pw.ind}}{}

% Quellenangabe nur in der Leseansicht
\ifkorrekturansicht\else
% Fallback-Definitionen, falls die .tex-Datei \titel etc. nicht gesetzt hat
\providecommand{\titel}{}
\providecommand{\editorInnen}{}
\providecommand{\dateiname}{\jobname}

\vspace{3cm}

\vfill

\footnotesize
\textsc{Quelle}: \titel. Herausgegeben von {\editorInnen}. In: \emph{Arthur Schnitzler: Briefwechsel mit Autorinnen und Autoren}.
 Digitale Edition, https://schnitzler-briefe.acdh.oeaw.ac.at/{\dateiname}.html (Stand \today)
\fi

\end{document}


      