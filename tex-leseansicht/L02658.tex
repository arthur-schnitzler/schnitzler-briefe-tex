%% latex-leseansicht-vorspann.tex
%% Vorspann für die Leseansicht.
%% Lädt die gemeinsame Datei latex-vorspann.tex mit nicht gesetztem Schalter.

\newif\ifkorrekturansicht
\korrekturansichtfalse

\input{../tex-inputs/latex-vorspann}


         
         \newcommand{\erwaehntePersonen}{Personen: Anna Hochenburger, Fedor Mamroth}
         \newcommand{\erwaehnteInstitutionen}{Institutionen: An der schönen blauen Donau, Burgtheater, Frankfurter Zeitung}
         \newcommand{\erwaehnteOrte}{Orte: Berlin, Wien}
         \newcommand{\erwaehnteWerke}{Werke: ?? [Rezension des Gastspiels von Anna Hochenburger, 7.1.1891], Romeo und Julia}
               \section[Paul Goldmann an Arthur Schnitzler, 7. 1. 1891]{ Paul Goldmann an Arthur Schnitzler, 7. 1. 1891}\nopagebreak\mylabel{v}\rehead{ }\begin{ledgroupsized}[t]{13cm}\normalsize\beginnumbering \toendnotes[C]{\smallbreak\pagebreak[2]} \Standort{DLA, A:Schnitzler, HS.NZ85.1.3162.}
\physDesc{Brief, 1 Blatt, 2 Seiten
\newline{}Handschrift: Bleistift, deutsche Kurrent
\newline{}Schnitzler: mit Bleistift das Datum »Jän 91« vermerkt }\toendnotes[C]{\smallbreak}\pstart\center{}{\pb}Lieber Arthur!\pend\pstart
           Eine große Gefälligkeit, bitte! Geh’ heut{ }Abend in’s Burgtheater\orgindex{Burgtheater@Burgtheater|pw} u »ſchreib«
               mir ein \label{K_L02658-11v}\edtext{Referat\pwindex{?? [Rezension des Gastspiels von Anna Hochenburger, 7.1.1891]None@\emph{?? [Rezension des Gastspiels von Anna Hochenburger, 7.1.1891]} {[}None{]}|pwv}}{\lemma{\textnormal{\emph{Referat}}}\Cendnote{\textnormal{Im letzten Heft des Jahres
                     1890 stand letztmalig Goldmann\pwindex{Goldmann, Paul 31.01.1865 – 25.09.1935@\textsc{Goldmann, Paul} (31.01.1865 – 25.09.1935), \emph{Schriftsteller, Journalist}|pwk}s Name als »Mit-Redakteur\pwindex{Goldmann, Paul 31.01.1865 – 25.09.1935@\textsc{Goldmann, Paul} (31.01.1865 – 25.09.1935), \emph{Schriftsteller, Journalist}|pwkv}« im Impressum von \emph{An der
                     schönen blauen Donau}\orgindex{der schoenen blauen Donau@An der schönen blauen Donau|pwk}. Anzunehmen ist, dass er danach gemeinsam mit dem Herausgeber\pwindex{Mamroth, Fedor 21.02.1851 – 25.06.1907@\textsc{Mamroth, Fedor} (21.02.1851 – 25.06.1907), \emph{Journalist, Kritiker}|pwkv} und Onkel\pwindex{Mamroth, Fedor 21.02.1851 – 25.06.1907@\textsc{Mamroth, Fedor} (21.02.1851 – 25.06.1907), \emph{Journalist, Kritiker}|pwkv}{ }Fedor Mamroth\pwindex{Mamroth, Fedor 21.02.1851 – 25.06.1907@\textsc{Mamroth, Fedor} (21.02.1851 – 25.06.1907), \emph{Journalist, Kritiker}|pwk} die Mitarbeit an der Zeitschrift\orgindex{der schoenen blauen Donau@An der schönen blauen Donau|pwkv} beendet hatte.
                  Nachdem er die Stelle bei der \emph{Frankfurter
                     Zeitung}\orgindex{Frankfurter Zeitung@Frankfurter Zeitung|pwk} erst mit April 1891 antrat und erst
                  kurz vorher davon erfahren haben dürfte (vgl. A. S.: \emph{Tagebuch}, 29. 3. 1891), bleibt offen, für welche Publikation er in den
                  ersten drei Monaten des Jahres 1891 tätig war. Weil er die Rezension\pwindex{?? [Rezension des Gastspiels von Anna Hochenburger, 7.1.1891]None@\emph{?? [Rezension des Gastspiels von Anna Hochenburger, 7.1.1891]} {[}None{]}|pwkv} erst für den übernächsten Tag erbittet, dürfte es sich um ein
                  Wochen- oder Monatsblatt handeln. Oder er benötigte das Referat\pwindex{?? [Rezension des Gastspiels von Anna Hochenburger, 7.1.1891]None@\emph{?? [Rezension des Gastspiels von Anna Hochenburger, 7.1.1891]} {[}None{]}|pwkv} als Stilprobe für eine
                  Stellenbewerbung, wogegen aber zu sprechen scheint, dass er über ein Büro
                  verfügte.}}}\label{K_L02658-11h} über die \label{K_L02658-1v}\edtext{\textsc{Hochenburger\pwindex{Hochenburger, Anna 1860 – 1911@\textsc{Hochenburger, Anna} (1860 – 1911), \emph{Schauspielerin}|pw}}}{\lemma{\textnormal{\emph{Hochenburger}}}\Cendnote{\textnormal{Die Berlin\oindex{Berlin@\textbf{Berlin}|pwk}er Schauspielerin\pwindex{Hochenburger, Anna 1860 – 1911@\textsc{Hochenburger, Anna} (1860 – 1911), \emph{Schauspielerin}|pwkv}{ }Anna Hochenburger\pwindex{Hochenburger, Anna 1860 – 1911@\textsc{Hochenburger, Anna} (1860 – 1911), \emph{Schauspielerin}|pwk} hatte im Januar 1891 ein Gastspiel am \emph{Burgtheater}\orgindex{Burgtheater@Burgtheater|pwk}. Es begann am 7. 1. 1891, sie gab Julia\pwindex{\textcolor{red}{\textsuperscript{XXXX1 indx}}!Romeo und Julia1594@\strich\emph{Romeo und Julia} {[}1594{]}|pwkv} in \emph{Romeo und Julia}\pwindex{\textcolor{red}{\textsuperscript{XXXX1 indx}}!Romeo und Julia1594@\strich\emph{Romeo und Julia} {[}1594{]}|pwk}. Schnitzler\pwindex{Schnitzler, Arthur 15.05.1862 – 21.10.1931@\textsc{Schnitzler, Arthur} (15.05.1862 – 21.10.1931), \emph{Schriftsteller, Mediziner}|pwk} nahm an
                  der Premiere am 7. 1. 1891 teil. Das und der Folgebrief (Paul Goldmann an Arthur Schnitzler, 7. 1. 1891) ermöglichen die verlässliche Datierung des undatierten
                  Korrespondenzstücks.}}}\label{K_L02658-1h}! Aus Gründen, die ich Dir für mich
                  ent\textcolor{gray}{wickeln} kann, bin ich verhindert ſelbſt zu gehen. Es darf
               aber Niemand wiſſen, daß du \uline{für mich} gehſt! Sollteſt
               Du aus irgend einem Grunde {\pb}verhindert ſein, \strikeout{m\textcolor{gray}{e}i} meine Bitte zu erfüllen, ſo
               ſchicke mir, bitte, \uuline{umgehend} die Karte in’s Bureau
               zurück. Das Referat\pwindex{?? [Rezension des Gastspiels von Anna Hochenburger, 7.1.1891]None@\emph{?? [Rezension des Gastspiels von Anna Hochenburger, 7.1.1891]} {[}None{]}|pw} müßte ich bis übermorgen{ }früh in Händen haben.\pend
           \pstart
           Herzl. Gruß! {\\[\baselineskip]}Dein {\\[\baselineskip]}\spacefill\mbox{Paul Goldm}\pend
           \leftskip=0em{}
         
         \endnumbering\mylabel{h}\end{ledgroupsized}  \newcommand{\dateiname}{L02658}\newcommand{\titel}{Paul Goldmann an Arthur Schnitzler, 7. 1. 1891}\newcommand{\editorInnen}{Martin Anton Müller und Laura Untner}%% latex-leseansicht-abspann.tex
%% Abspann für die Leseansicht.
%% Der Schalter \ifkorrekturansicht ist bereits durch den Vorspann gesetzt.

%% latex-abspann.tex
%% Gemeinsamer Abspann für Korrekturansicht und Leseansicht.
%% Setzt den Schalter \ifkorrekturansicht voraus (gesetzt in den
%% einbindenden Dateien latex-korrekturansicht-abspann.tex bzw.
%% latex-leseansicht-abspann.tex).
%% ---------------------------------------------------------------

\normalsize

% Das esempio-Environment wird nur in der Leseansicht benötigt
\ifkorrekturansicht\else
\newenvironment{esempio}[3]%
{
    \vspace{1.5ex}
    \rlap{\underline{#1}}
    \par
    \setlength{\parindent}{0cm}
    \nopagebreak
    \leftskip=#2cm
    \rightskip=#3cm
}
{
    \par
}
\fi

\doendnotes{C}
\bigskip
\vfill

\clearpage

\footnotesize

\ifkorrekturansicht
  \lohead{\textsc{register}}
\fi

% theindex-Environment neu definieren ohne reledmac
\makeatletter
\renewenvironment{theindex}{%
  \ifkorrekturansicht
    \section*{\indexname}%
  \else
    \subsubsection*{Index der erwähnten Entitäten}%
  \fi
  \setlength{\parindent}{0pt}%
  \setlength{\parskip}{0pt plus 0.3pt}%
  \let\item\@idxitem
}{%
  \ifkorrekturansicht\clearpage\fi
}
\makeatother

\IfFileExists{\jobname-pw.ind}{\input{\jobname-pw.ind}}{}

% Quellenangabe nur in der Leseansicht
\ifkorrekturansicht\else
% Fallback-Definitionen, falls die .tex-Datei \titel etc. nicht gesetzt hat
\providecommand{\titel}{}
\providecommand{\editorInnen}{}
\providecommand{\dateiname}{\jobname}

\vspace{3cm}

\vfill

\footnotesize
\textsc{Quelle}: \titel. Herausgegeben von {\editorInnen}. In: \emph{Arthur Schnitzler: Briefwechsel mit Autorinnen und Autoren}.
 Digitale Edition, https://schnitzler-briefe.acdh.oeaw.ac.at/{\dateiname}.html (Stand \today)
\fi

\end{document}


      