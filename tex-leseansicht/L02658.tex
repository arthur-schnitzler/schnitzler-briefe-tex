%% latex-korrekturansicht-vorspann.tex
%% Vorspann für die Korrekturansicht.
%% Lädt die gemeinsame Datei latex-vorspann.tex mit gesetztem Schalter.

\newif\ifkorrekturansicht
\korrekturansichttrue

\input{../tex-inputs/latex-vorspann}


\section[Paul Goldmann an Arthur Schnitzler, 7. 1. 1891]{L02658 Paul Goldmann an Arthur Schnitzler, 7. 1. 1891}
\nopagebreak\mylabel{L02658v}
\rehead{ }\normalsize\beginnumbering\briefempfaengerindex{Schnitzler, Arthur@\textsc{Schnitzler, Arthur}!zzzGoldmann, Paul@\emph{von Paul Goldmann}!1891-01-071@{7. 1. 1891}|(be}
\toendnotes[C]{\smallbreak\pagebreak[2]}\Standort{DLA, A:Schnitzler, HS.NZ85.1.3162.}
\physDesc{Brief, 1 Blatt, 2 Seiten, 478 Zeichen
\newline{}Handschrift: Bleistift, deutsche Kurrent
\newline{}Schnitzler: mit Bleistift das Datum »Jän 91« vermerkt }\toendnotes[C]{\smallbreak}
\pstart\center{}{\pb}Lieber Arthur!\pend\vspace{0.5em}
\pstart
           Eine große Gefälligkeit, bitte! Geh’ heut{ }Abend in’s Burgtheater\orgindex{Burgtheater@Burgtheater|pw} u »ſchreib«
               mir ein \label{K_L02658-1v}\edtext{Referat\pwindex{(Burgtheater.) [Rezension des Gastspiels von Anna Hochenburger]@\emph{(Burgtheater.) [Rezension des Gastspiels von Anna Hochenburger]}|pwv}}{\lemma{\textnormal{\emph{Referat}}}\Cendnote{\textnormal{[Arthur Schnitzler]: \emph{(Burgtheater)}\pwindex{(Burgtheater.) [Rezension des Gastspiels von Anna Hochenburger]@\emph{(Burgtheater.) [Rezension des Gastspiels von Anna Hochenburger]}|pwk}. In: \emph{Wiener Sonn- und Montags-Zeitung}\pwindex{Wiener Sonn- und Montagszeitung@\emph{Wiener Sonn- und Montagszeitung}|pwk}, Jg. 29, Nr. 2, 12. 1. 1891, S. 3. Goldmann\pwindex{Goldmann, Paul 31.01.1865 – 25.09.1935@\textsc{Goldmann, Paul} (31.01.1865 – 25.09.1935), \emph{Schriftsteller/Schriftstellerin, Journalist/Journalistin}|pwk} und Mamroth\pwindex{Mamroth, Fedor 21.02.1851 – 25.06.1907@\textsc{Mamroth, Fedor} (21.02.1851 – 25.06.1907), \emph{Journalist/Journalistin, Kritiker/Kritikerin}|pwk}
                  hatten Ende 1890 ihre Redaktionsarbeit für die \emph{Schöne Blaue Donau}\orgindex{der schoenen blauen Donau@An der schönen blauen Donau|pwk} niedergelegt. Danach
                  übernahm Goldmann\pwindex{Goldmann, Paul 31.01.1865 – 25.09.1935@\textsc{Goldmann, Paul} (31.01.1865 – 25.09.1935), \emph{Schriftsteller/Schriftstellerin, Journalist/Journalistin}|pwk} für kurze Zeit das \emph{Burgtheater}\orgindex{Burgtheater@Burgtheater|pwk}referat der \emph{Wiener Sonn- und Montags-Zeitung}\orgindex{Wiener Sonn- und Montagszeitung@Wiener Sonn- und Montagszeitung|pwk}.}}}\label{K_L02658-1} über die \label{K_L02658-2v}\edtext{\textsc{Hochenburger\pwindex{Hochenburger, Anna 1860 – 1911@\textsc{Hochenburger, Anna} (1860 – 1911), \emph{Schauspieler/Schauspielerin}|pw}}}{\lemma{\textnormal{\emph{Hochenburger}}}\Cendnote{\textnormal{Die Berlin\oindex{Berlin@\textbf{Berlin}, \emph{P.PPLC}|pwk}er Schauspielerin\pwindex{Hochenburger, Anna 1860 – 1911@\textsc{Hochenburger, Anna} (1860 – 1911), \emph{Schauspieler/Schauspielerin}|pwkv}{ }Anna Hochenburger\pwindex{Hochenburger, Anna 1860 – 1911@\textsc{Hochenburger, Anna} (1860 – 1911), \emph{Schauspieler/Schauspielerin}|pwk} hatte im Januar 1891 ein Gastspiel am \emph{Burgtheater}\orgindex{Burgtheater@Burgtheater|pwk}. Es begann am 7. 1. 1891, sie gab Julia\pwindex{Romeo and Juliet@\emph{Romeo and Juliet}|pwkv} in \emph{Romeo und Julia}\pwindex{Romeo and Juliet@\emph{Romeo and Juliet}|pwk}. Schnitzler nahm an
                  der Premiere teil. Das und der Folgebrief (Paul Goldmann an Arthur Schnitzler, 7. 1. 1891) ermöglichen die verlässliche Datierung des
                  undatierten Korrespondenzstücks.}}}\label{K_L02658-2}! Aus Gründen, die ich Dir für mich
                  ent\textcolor{gray}{wickeln} kann, bin ich verhindert ſelbſt zu gehen. Es darf
               aber Niemand wiſſen, daß du \uline{für mich} gehſt! Sollteſt
               Du aus irgend einem Grunde {\pb}verhindert ſein, \strikeout{m\textcolor{gray}{e}i} meine Bitte zu erfüllen, ſo
               ſchicke mir, bitte, \uuline{umgehend} die Karte in’s Bureau
               zurück. Das Referat\pwindex{(Burgtheater.) [Rezension des Gastspiels von Anna Hochenburger]@\emph{(Burgtheater.) [Rezension des Gastspiels von Anna Hochenburger]}|pw} müßte ich bis übermorgen{ }früh in Händen haben.\pend
           
\pstart
           Herzl. Gruß! {\\[\baselineskip]}Dein {\\[\baselineskip]}\spacefill\mbox{Paul Goldm}\pend
           \leftskip=0em{}\selectlanguage{ngerman}\endnumbering\briefempfaengerindex{Schnitzler, Arthur@\textsc{Schnitzler, Arthur}!zzzGoldmann, Paul@\emph{von Paul Goldmann}!1891-01-071@{7. 1. 1891}|)be}\mylabel{L02658h}  \normalsize

\doendnotes{C}
\bigskip
\vfill

\clearpage

\footnotesize

\lohead{\textsc{register}}

% Definiere theindex-Environment komplett neu ohne reledmac
\makeatletter
\renewenvironment{theindex}{%
  \section*{\indexname}%
  \setlength{\parindent}{0pt}%
  \setlength{\parskip}{0pt plus 0.3pt}%
  \let\item\@idxitem
}{%
  \clearpage
}
\makeatother

\IfFileExists{\jobname-pw.ind}{\input{\jobname-pw.ind}}{}

\end{document}

      