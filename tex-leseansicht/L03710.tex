%% latex-korrekturansicht-vorspann.tex
%% Vorspann für die Korrekturansicht.
%% Lädt die gemeinsame Datei latex-vorspann.tex mit gesetztem Schalter.

\newif\ifkorrekturansicht
\korrekturansichttrue

\input{../tex-inputs/latex-vorspann}


\section[Elsa Plessner an Arthur Schnitzler, 29. 12. 1896]{L03710 Elsa Plessner an Arthur Schnitzler, 29. 12. 1896}
\nopagebreak\mylabel{L03710v}
\rehead{ }\normalsize\beginnumbering\briefempfaengerindex{Schnitzler, Arthur@\textsc{Schnitzler, Arthur}!zzzPlessner, Elsa@\emph{von Elsa Plessner}!1896-12-291@{29. 12. 1896}|(be}
\toendnotes[C]{\smallbreak\pagebreak[2]}\Standort{DLA, A:Schnitzler, HS.1985.1.419.}
\physDesc{Brief,  Blätter, 3 Seiten, 2201 Zeichen
\newline{}Handschrift: , lateinische Kurrent
\newline{}Schnitzler: eine Unterstreichung }\toendnotes[C]{\smallbreak}
\pstart
           {\pb}Meran, Pension Wolf\oindex{Hotel Meranerhof@\textbf{Hotel Meranerhof}, \emph{Hotel (K.HTL)}|pw}, den
                     29. 12. 96.\pend
           
\pstart{}Hochverehrter Herr Doctor!\pend\vspace{0.5em}
\pstart
           \label{K_L03710-1v}\edtext{Anbei}{\lemma{\textnormal{\emph{Anbei}}}\Cendnote{\textnormal{Die Beilage ist nicht überliefert.}}}\label{K_L03710-1} »Orchideen\pwindex{Orchideen [Schauspiel in drei Akten]@\emph{Orchideen [Schauspiel in drei Akten]}|pw}«. Erschrecken Sie bitte nicht über die
               Dampfgeschwindigkeit, mit der ich Sie überfalle. Nämlich ich dachte so: »Ist das Stück\pwindex{Orchideen [Schauspiel in drei Akten]@\emph{Orchideen [Schauspiel in drei Akten]}|pwv} in der Anlage verhauen,
               so nützt keine Feile was, ist es aber gut, so können Sie sich die Feile hinzudenken«.
               Also nehme ich keinen Anstand, es Ihnen noch in einem noch wenig verfeinerten, ersten
               Justzustand zu übersenden mit der Bitte um \uline{strenges
                  Gericht}. das Sie vielleicht durch Roth oder Blaustift in den Text hinein
               bemerkbar machen {\pb}zu wollen, so gut sind!! – Erschrecken Sie, bitte
               nicht, wenn Sie den Lieutenant sehen — kein Bösewicht x-ter Auflage – . Die mit
               Bleistift notirte Rollenbesetzung  ist natürlich nur dazu da, Sie ein bisschen im
               vorhinein über die Figuren zu orientiren – – ! – Die Grundidee meines Stückes\pwindex{Orchideen [Schauspiel in drei Akten]@\emph{Orchideen [Schauspiel in drei Akten]}|pw} ist \introOben{}mir\introOben{} eigentlich gekommen
               durch die Töchter\pwindex{Tullia Major @\textsc{Tullia Major}, \emph{Prinz/Prinzessin}|pwv}\pwindex{Tullia Minor @\textsc{Tullia Minor}, \emph{Prinz/Prinzessin}|pwv} des Servius Tullus\pwindex{Servius Tullius @\textsc{Servius Tullius}, \emph{König/Königin}|pw} – und das
               sage ich Ihnen nur, weil ich nicht will, daß Sie an etwas \uline{anderes} denken, was Sie auch im Beginn gewiss thun werden. – Aber Sie werden
               ja sehen, wie verschieden es nachher wird!! – Über dem ganzen Stück\pwindex{Orchideen [Schauspiel in drei Akten]@\emph{Orchideen [Schauspiel in drei Akten]}|pwv} schwebt – als unausgesprochenes
               »Sesam« \uline{ein Wort}, das ich jedoch \uline{nirgends} gebraucht habe! – Ich glaube, es wird auch {\pb}Ihnen
               auf die Lippen treten. – Zum Schluß bitte ich Sie noch um Entschuldigung, wegen der
               mangelhaften äußeren Form des Manuscriptes\pwindex{Orchideen [Schauspiel in drei Akten]@\emph{Orchideen [Schauspiel in drei Akten]}|pwv} – war in der Schnelligkeit nicht anders möglich – und Geduld
               habe ich keine mehr! – So, jetzt wissen Sie alles, was ich auf dem Herzen habe – (d.
               h. diesbezüglich) und somit empfehle ich die »Orchideen\pwindex{Orchideen [Schauspiel in drei Akten]@\emph{Orchideen [Schauspiel in drei Akten]}|pw}« allen neun Musen und Ihrer Huld – – bitte! – bitte ! – bitte!!!!
               – lassen Sie mich nicht zu lange zappeln – aus Gesundheitsrücksichten für mich und
               meine »Nerven«, die sich in einem pitoyablen Zustand befinden!! – wirklich! – Ich
               gebe Ihnen die notariell beglaubigte Versicherung, daß ich bis zum Eintreffen Ihrer
               Meinungsabgabe keine geruhsame Nacht mehr erleben werde – und ob das recht viele sein
               werden, hängt von Ihrer Güte ab!! – – Die Sonne scheint jetzt wieder 25 Celsiusgrädig
               auf meinen Schreibtisch – d. h. spazieren gehen – also – – schließt mit
               hochachtungsvoller Ergebenheit und herzlichen Grüßen von der Frau Sonne und besten
               von mir\pend
           \pstart \spacefill\mbox{Elsa Plessner}\pend{}\selectlanguage{ngerman}\endnumbering\briefempfaengerindex{Schnitzler, Arthur@\textsc{Schnitzler, Arthur}!zzzPlessner, Elsa@\emph{von Elsa Plessner}!1896-12-291@{29. 12. 1896}|)be}\mylabel{L03710h}
\begin{anhang}
\end{anhang}\normalsize

\doendnotes{C}
\bigskip
\vfill

\clearpage

\footnotesize

\lohead{\textsc{register}}

% Definiere theindex-Environment komplett neu ohne reledmac
\makeatletter
\renewenvironment{theindex}{%
  \section*{\indexname}%
  \setlength{\parindent}{0pt}%
  \setlength{\parskip}{0pt plus 0.3pt}%
  \let\item\@idxitem
}{%
  \clearpage
}
\makeatother

\IfFileExists{\jobname-pw.ind}{\input{\jobname-pw.ind}}{}

\end{document}

      