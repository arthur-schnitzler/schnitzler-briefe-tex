%% latex-leseansicht-vorspann.tex
%% Vorspann für die Leseansicht.
%% Lädt die gemeinsame Datei latex-vorspann.tex mit nicht gesetztem Schalter.

\newif\ifkorrekturansicht
\korrekturansichtfalse

\input{../tex-inputs/latex-vorspann}


\section[Elsa Plessner an Arthur Schnitzler, 29. 12. 1896]{L03710 Elsa Plessner an Arthur Schnitzler, 29. 12. 1896}
\nopagebreak\mylabel{L03710v}
\rehead{ }\normalsize\beginnumbering\briefempfaengerindex{Schnitzler, Arthur@\textsc{Schnitzler, Arthur}!zzzPlessner, Elsa@\emph{von Elsa Plessner}!1896-12-291@{29. 12. 1896}|(be}
\toendnotes[C]{\smallbreak\pagebreak[2]}
\correspDesc{Versand  durch Elsa Plessner am 29. 12. 1896 in Meran
\newline{}Erhalt  durch Arthur Schnitzler im Zeitraum [30. 12. 1896 – 3. 1. 1897?] in Wien}\toendnotes[C]{\smallbreak}
\Standort{DLA, A:Schnitzler, HS.1985.1.419.}
\physDesc{Brief, 1 Blatt, 3 Seiten, 2206 Zeichen
\newline{}Handschrift: schwarze Tinte, lateinische Kurrent
\newline{}Schnitzler: mit rotem Buntstift eine Unterstreichung }\toendnotes[C]{\smallbreak}
\pstart
           {\pb}Meran, Pension Wolf\oindex{Hotel Meranerhof@\textbf{Hotel Meranerhof}, \emph{Hotel}|pw}, den
                     29. 12. 96.\pend
           
\pstart\center{}Hochverehrter Herr Doctor!\pend\vspace{0.5em}
\pstart
           Anbei »\label{K_L03710-1v}\edtext{Orchideen\pwindex{Plessner, Elsa 22.\,8.\,1875 Wien – 7.\,5.\,1932 Alicante@\textsc{Plessner, Elsa} (22.\,8.\,1875 Wien – 7.\,5.\,1932 Alicante), \emph{Schriftstellerin}!Orchideen [Schauspiel in drei Akten]@\strich\emph{Orchideen [Schauspiel in drei Akten]}|pw}}{\lemma{\textnormal{\emph{Orchideen}}}\Cendnote{\textnormal{Das Werk ist nicht überliefert.}}}\label{K_L03710-1}«.
               Erschrecken Sie, bitte, nicht über die Dampfgeschwindigkeit, mit der ich Sie
               überfalle. Nämlich ich dachte so: »Ist das Stück\pwindex{Plessner, Elsa 22.\,8.\,1875 Wien – 7.\,5.\,1932 Alicante@\textsc{Plessner, Elsa} (22.\,8.\,1875 Wien – 7.\,5.\,1932 Alicante), \emph{Schriftstellerin}!Orchideen [Schauspiel in drei Akten]@\strich\emph{Orchideen [Schauspiel in drei Akten]}|pwv} in der Anlage verhauen, so nützt keine »Feile« was,
               ist es aber gut, so können Sie sich die Feile {[}»{]}hinzudenken«.
               Also nehme ich keinen Anstand, es Ihnen noch in einem noch wenig verfeinerten, ersten
               Justzustand zu übersenden mit der Bitte um \uline{strenges
                  Gericht}, das Sie vielleicht durch Roth oder Blaustift in den Text hinein
               bemerkbar machen {\pb}zu wollen, so gut sind!! – Erschrecken
               Sie, bitte nicht, wenn Sie den Lieutenant sehen — kein \label{K_L03710-556v}\edtext{Brüsewicht\pwindex{Brüsewitz, Henning von 1862 – 24.\,1.\,1900@\textsc{Brüsewitz, Henning von} (1862 – 24.\,1.\,1900), \emph{Militär, Offizier}|pw}}{\lemma{\textnormal{\emph{Brüsewicht}}}\Cendnote{\textnormal{Henning von Brüsewicht\pwindex{Brüsewitz, Henning von 1862 – 24.\,1.\,1900@\textsc{Brüsewitz, Henning von} (1862 – 24.\,1.\,1900), \emph{Militär, Offizier}|pwk} hatte in der Nacht von
                     11. 10.  auf den 12. 10.  aus
                  gekränkter Soldatenehre einen Zivilisten\pwindex{Siepmann, Theodor †~11.\,10.\,1896 Karlsruhe@\textsc{Siepmann, Theodor} (†~11.\,10.\,1896 Karlsruhe), \emph{Handwerker, Mechaniker}|pwkv} ermordet.}}}\label{K_L03710-556} x-ter Auflage –. Die mit Bleistift notirte
               Rollenbesetzung  ist natürlich nur dazu da, Sie ein bisschen im vorhinein über die
               Figuren zu orientiren – – ! – Die Grundidee meines Stückes\pwindex{Plessner, Elsa 22.\,8.\,1875 Wien – 7.\,5.\,1932 Alicante@\textsc{Plessner, Elsa} (22.\,8.\,1875 Wien – 7.\,5.\,1932 Alicante), \emph{Schriftstellerin}!Orchideen [Schauspiel in drei Akten]@\strich\emph{Orchideen [Schauspiel in drei Akten]}|pw} ist \introOben{}mir\introOben{} eigentlich gekommen durch die Töchter\pwindex{Tullia Major @\textsc{Tullia Major}, \emph{Prinzessin}|pwv}\pwindex{Tullia Minor @\textsc{Tullia Minor}, \emph{Prinzessin}|pwv} des Servius Tullus\pwindex{Servius Tullius @\textsc{Servius Tullius}, \emph{König}|pw} – und das sage ich Ihnen nur,
               weil ich nicht will, dass Sie an etwas \uline{anderes} denken,
               was Sie auch im Beginn gewiss thun werden! – Aber Sie werden ja sehen, wie
               verschieden es nachher wird!! – Über dem ganzen Stück\pwindex{Plessner, Elsa 22.\,8.\,1875 Wien – 7.\,5.\,1932 Alicante@\textsc{Plessner, Elsa} (22.\,8.\,1875 Wien – 7.\,5.\,1932 Alicante), \emph{Schriftstellerin}!Orchideen [Schauspiel in drei Akten]@\strich\emph{Orchideen [Schauspiel in drei Akten]}|pwv} schwebt – als unausgesprochenes »Sesam« \uline{ein Wort}, das ich jedoch \uline{nirgends} gebraucht habe! – Ich glaube, es wird auch {\pb}Ihnen auf die Lippen treten. – Zum Schluss bitte ich Sie
               noch um Entschuldigung, wegen der mangelhaften äußeren Form des Manuscriptes\pwindex{Plessner, Elsa 22.\,8.\,1875 Wien – 7.\,5.\,1932 Alicante@\textsc{Plessner, Elsa} (22.\,8.\,1875 Wien – 7.\,5.\,1932 Alicante), \emph{Schriftstellerin}!Orchideen [Schauspiel in drei Akten]@\strich\emph{Orchideen [Schauspiel in drei Akten]}|pwv} – war in der Schnelligkeit
               nicht anders möglich – und Geduld habe ich keine mehr! – – So, jetzt wissen Sie alles,
               was ich auf dem Herzen habe – (d. h. diesbezüglich) und somit empfehle ich die »Orchideen\pwindex{Plessner, Elsa 22.\,8.\,1875 Wien – 7.\,5.\,1932 Alicante@\textsc{Plessner, Elsa} (22.\,8.\,1875 Wien – 7.\,5.\,1932 Alicante), \emph{Schriftstellerin}!Orchideen [Schauspiel in drei Akten]@\strich\emph{Orchideen [Schauspiel in drei Akten]}|pw}« allen neun Musen und Ihrer Huld – –
               bitte! – bitte ! – bitte!!!! – lassen Sie mich nicht zu lange zappeln – aus
               Gesundheitsrücksichten für mich und meine »Nerven« – die sich in einem pitoyablen
               Zustand befinden!! – wirklich! – Ich gebe Ihnen die notariell beglaubigte
               Versicherung, dass ich bis zum Eintreffen Ihrer Meinungsabgabe keine geruhsame Nacht
               mehr erleben werde – und ob das recht viele sein werden, hängt von Ihrer Güte ab!! –
               – Die Sonne scheint jetzt wieder 25 Celsiusgrädig auf meinen Schreibtisch – d. h.
               spazieren gehen – also – – schließt mit hochachtungsvoller Ergebenheit und herzlichen
               Grüßen von der Frau Sonne und besten von mir\pend
           \pstart \spacefill\mbox{Elsa Plessner.}\pend{}\selectlanguage{ngerman}\endnumbering\briefempfaengerindex{Schnitzler, Arthur@\textsc{Schnitzler, Arthur}!zzzPlessner, Elsa@\emph{von Elsa Plessner}!1896-12-291@{29. 12. 1896}|)be}\mylabel{L03710h}  \newcommand{\dateiname}{L03710}\newcommand{\titel}{Elsa Plessner an Arthur Schnitzler, 29. 12. 1896}\newcommand{\editorInnen}{Selma Jahnke und Martin Anton Müller}%% latex-leseansicht-abspann.tex
%% Abspann für die Leseansicht.
%% Der Schalter \ifkorrekturansicht ist bereits durch den Vorspann gesetzt.

%% latex-abspann.tex
%% Gemeinsamer Abspann für Korrekturansicht und Leseansicht.
%% Setzt den Schalter \ifkorrekturansicht voraus (gesetzt in den
%% einbindenden Dateien latex-korrekturansicht-abspann.tex bzw.
%% latex-leseansicht-abspann.tex).
%% ---------------------------------------------------------------

\normalsize

% Das esempio-Environment wird nur in der Leseansicht benötigt
\ifkorrekturansicht\else
\newenvironment{esempio}[3]%
{
    \vspace{1.5ex}
    \rlap{\underline{#1}}
    \par
    \setlength{\parindent}{0cm}
    \nopagebreak
    \leftskip=#2cm
    \rightskip=#3cm
}
{
    \par
}
\fi

\doendnotes{C}
\bigskip
\vfill

\clearpage

\footnotesize

\ifkorrekturansicht
  \lohead{\textsc{register}}
\fi

% theindex-Environment neu definieren ohne reledmac
\makeatletter
\renewenvironment{theindex}{%
  \ifkorrekturansicht
    \section*{\indexname}%
  \else
    \subsubsection*{Index der erwähnten Entitäten}%
  \fi
  \setlength{\parindent}{0pt}%
  \setlength{\parskip}{0pt plus 0.3pt}%
  \let\item\@idxitem
}{%
  \ifkorrekturansicht\clearpage\fi
}
\makeatother

\IfFileExists{\jobname-pw.ind}{\input{\jobname-pw.ind}}{}

% Quellenangabe nur in der Leseansicht
\ifkorrekturansicht\else
% Fallback-Definitionen, falls die .tex-Datei \titel etc. nicht gesetzt hat
\providecommand{\titel}{}
\providecommand{\editorInnen}{}
\providecommand{\dateiname}{\jobname}

\vspace{3cm}

\vfill

\footnotesize
\textsc{Quelle}: \titel. Herausgegeben von {\editorInnen}. In: \emph{Arthur Schnitzler: Briefwechsel mit Autorinnen und Autoren}.
 Digitale Edition, https://schnitzler-briefe.acdh.oeaw.ac.at/{\dateiname}.html (Stand \today)
\fi

\end{document}


