%% latex-leseansicht-vorspann.tex
%% Vorspann für die Leseansicht.
%% Lädt die gemeinsame Datei latex-vorspann.tex mit nicht gesetztem Schalter.

\newif\ifkorrekturansicht
\korrekturansichtfalse

\input{../tex-inputs/latex-vorspann}


\section[Theodor Herzl an Arthur Schnitzler, 6. 12. 1900]{L03875 Theodor Herzl an Arthur Schnitzler, 6. 12. 1900}
\nopagebreak\mylabel{L03875v}
\rehead{ }\normalsize\beginnumbering\briefempfaengerindex{Schnitzler, Arthur@\textsc{Schnitzler, Arthur}!zzzHerzl, Theodor@\emph{von Theodor Herzl}!1900-12-062@{6. 12. 1900}|(be}
\toendnotes[C]{\smallbreak\pagebreak[2]}
\correspDesc{Versand  durch Theodor Herzl am 6. 12. 1900 in Wien
\newline{}Erhalt  durch Arthur Schnitzler im Zeitraum [6. 12. 1900
                  – 9. 12. 1900?] in Wien}\toendnotes[C]{\smallbreak}
\Standort{CUL, Schnitzler, B 39.}
\physDesc{Brief, 1 Blatt, 1 Seite, 192 Zeichen
\newline{}Handschrift: schwarze Tinte, lateinische Kurrent
\newline{}Schnitzler: mit Bleistift datiert: »Vor Weihnacht 900« und erster Versuch der Datierung gestrichen: »We 16?/12« 
\newline{}Ordnung: mit Bleistift von unbekannter Hand nummeriert: »54« }
\buchAbdrucke{\weitereDrucke{Theodor Herzl: \emph{Briefe Ende August 1900 – Ende Dezember 1902}. Bearbeitet von Barbara Schäfer in Zusammenarbeit mit Sofia Gelmann, Chaya Harel und Ines Rubin. Berlin, Frankfurt am Main, Wien: \emph{Propyläen} 1993, S. 112 (Briefe und Tagebücher. Herausgegeben von Alex Bein, Hermann Greive, Moshe Schaerf, Julius H. Schoeps und Johannes Wachten, 6).} }\toendnotes[C]{\smallbreak}
\pstart
           {\pb}\textcolor{gray}{\textbf{NEUE FREIE PRESSE\orgindex{Neue Freie Presse@Neue Freie Presse|pw}. }}\pend
           
\pstart
           \textcolor{gray}{\textbf{\textsc{Redaction}:}}\pend
           
\pstart
           \textcolor{gray}{\textbf{WIEN\oindex{Wien@\textbf{Wien}, \emph{Verwaltungsgebiet}|pw}}}\pend
           
\pstart
           \textcolor{gray}{\textbf{Kolowratring, Fichtegasse Nr. 11\oindex{Wien@\textbf{Wien}!I., Innere Stadt@\textbf{I., Innere Stadt}!Fichtegasse 11@\textbf{Fichtegasse 11}, \emph{Gebäude}|pw}.}}\pend
           
\pstart{}Lieber Doctor,\pend\vspace{0.5em}
\pstart
           die Fahnen\pwindex{Schnitzler, Arthur 15.\,5.\,1862 Wien – 21.\,10.\,1931 ebd.@\textsc{Schnitzler, Arthur} (15.\,5.\,1862 Wien – 21.\,10.\,1931 ebd.), \emph{Schriftsteller, Mediziner}!Lieutenant Gustl. Novelle@\strich\emph{Lieutenant Gustl. Novelle}|pwv} werden Sie
               erhalten. Aendern Sie aber die \label{K_L03875-1v}\edtext{Nummer
               des Regiments}{\lemma{\textnormal{\emph{Nummer
               des Regiments}}}\Cendnote{\textnormal{Die Nummer des Regiments,
                  bei dem der Protagonist der Novelle\pwindex{Schnitzler, Arthur 15.\,5.\,1862 Wien – 21.\,10.\,1931 ebd.@\textsc{Schnitzler, Arthur} (15.\,5.\,1862 Wien – 21.\,10.\,1931 ebd.), \emph{Schriftsteller, Mediziner}!Lieutenant Gustl. Novelle@\strich\emph{Lieutenant Gustl. Novelle}|pwkv} dient, ist in der Druckfassung nicht erwähnt. Das Manuskript, das
                     Schnitzler bei der \emph{Neuen Freien Presse}\orgindex{Neue Freie Presse@Neue Freie Presse|pwk} zum Druck einreichte, ist nicht
                  erhalten. Aus einer erhaltenen früheren Texthandschrift geht hervor, dass Schnitzler Lieutnant Gustl zunächst als
                  Mitglied des 87. Regiments konzipiert hatte. In dieser Fassung überlegt der
                  Protagonist, was vor dem Suizid noch zu erledigen sei: »ich mach eine
                     Meldung ans Reg-Comand {\dotstwo} ganz wie eine dienstl
                     Meldung . [{\dots}] K.k Inf
                        Reg {\dotstwo} Nr {\dotstwo} 87 . – K.k
                     Lieu – seit {\dotsfive} meldet, dass er sich heut um 7 Uhr
                     umbring. wird« (Arthur Schnitzler: \emph{Lieutenant Gustl}\pwindex{Schnitzler, Arthur 15.\,5.\,1862 Wien – 21.\,10.\,1931 ebd.@\textsc{Schnitzler, Arthur} (15.\,5.\,1862 Wien – 21.\,10.\,1931 ebd.), \emph{Schriftsteller, Mediziner}!Toten schweigen@\strich\emph{Die Toten schweigen}|pwk}. Historisch-kritische Ausgabe.
                     Herausgegeben von Konstanze Fliedl. Berlin, Boston:
                        \emph{de Gruyter}{ }2011 (Werke in historisch-kritischen Ausgaben, herausgegeben von
                     Konstanze Fliedl), H 175). In der Druckfassung lautet die Passage
                  verkürzt: »ich mach’ eine Meldung ans Regiments=Commando … ganz wie eine
                     dienstliche Meldung {\dots} ja, auf einen umbrochenen
                     Bogen{\dots}« (Arthur Schnitzler: \emph{Lieutenant Gustl}\pwindex{Schnitzler, Arthur 15.\,5.\,1862 Wien – 21.\,10.\,1931 ebd.@\textsc{Schnitzler, Arthur} (15.\,5.\,1862 Wien – 21.\,10.\,1931 ebd.), \emph{Schriftsteller, Mediziner}!Lieutenant Gustl. Novelle@\strich\emph{Lieutenant Gustl. Novelle}|pwk}. In: \emph{Neue Freie Presse}\pwindex{Neue Freie Presse@\emph{Neue Freie Presse}|pwk}, Nr. 13.053, 25.\,12.\,1900, Morgenblatt,
                     S. 34–41, hier S. 40).}}}\label{K_L03875-1} phantastisch um, sonst werden zwei Herren
               vom Offizierscorps bei Ihnen erscheinen.\pend
           
\pstart
           Besten Gruss{\\[\baselineskip]}\spacefill\mbox{Herzl}\pend
           \leftskip=0em{}
\pstart
           6 XII 900\pend
           \selectlanguage{ngerman}\endnumbering\briefempfaengerindex{Schnitzler, Arthur@\textsc{Schnitzler, Arthur}!zzzHerzl, Theodor@\emph{von Theodor Herzl}!1900-12-062@{6. 12. 1900}|)be}\mylabel{L03875h}
\begin{anhang}
\end{anhang}\newcommand{\dateiname}{L03875}\newcommand{\titel}{Theodor Herzl an Arthur Schnitzler, 6. 12. 1900}\newcommand{\editorInnen}{Selma Jahnke und Martin Anton Müller}%% latex-leseansicht-abspann.tex
%% Abspann für die Leseansicht.
%% Der Schalter \ifkorrekturansicht ist bereits durch den Vorspann gesetzt.

%% latex-abspann.tex
%% Gemeinsamer Abspann für Korrekturansicht und Leseansicht.
%% Setzt den Schalter \ifkorrekturansicht voraus (gesetzt in den
%% einbindenden Dateien latex-korrekturansicht-abspann.tex bzw.
%% latex-leseansicht-abspann.tex).
%% ---------------------------------------------------------------

\normalsize

% Das esempio-Environment wird nur in der Leseansicht benötigt
\ifkorrekturansicht\else
\newenvironment{esempio}[3]%
{
    \vspace{1.5ex}
    \rlap{\underline{#1}}
    \par
    \setlength{\parindent}{0cm}
    \nopagebreak
    \leftskip=#2cm
    \rightskip=#3cm
}
{
    \par
}
\fi

\doendnotes{C}
\bigskip
\vfill

\clearpage

\footnotesize

\ifkorrekturansicht
  \lohead{\textsc{register}}
\fi

% theindex-Environment neu definieren ohne reledmac
\makeatletter
\renewenvironment{theindex}{%
  \ifkorrekturansicht
    \section*{\indexname}%
  \else
    \subsubsection*{Index der erwähnten Entitäten}%
  \fi
  \setlength{\parindent}{0pt}%
  \setlength{\parskip}{0pt plus 0.3pt}%
  \let\item\@idxitem
}{%
  \ifkorrekturansicht\clearpage\fi
}
\makeatother

\IfFileExists{\jobname-pw.ind}{\input{\jobname-pw.ind}}{}

% Quellenangabe nur in der Leseansicht
\ifkorrekturansicht\else
% Fallback-Definitionen, falls die .tex-Datei \titel etc. nicht gesetzt hat
\providecommand{\titel}{}
\providecommand{\editorInnen}{}
\providecommand{\dateiname}{\jobname}

\vspace{3cm}

\vfill

\footnotesize
\textsc{Quelle}: \titel. Herausgegeben von {\editorInnen}. In: \emph{Arthur Schnitzler: Briefwechsel mit Autorinnen und Autoren}.
 Digitale Edition, https://schnitzler-briefe.acdh.oeaw.ac.at/{\dateiname}.html (Stand \today)
\fi

\end{document}


