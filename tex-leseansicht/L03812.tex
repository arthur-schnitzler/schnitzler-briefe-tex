%% latex-leseansicht-vorspann.tex
%% Vorspann für die Leseansicht.
%% Lädt die gemeinsame Datei latex-vorspann.tex mit nicht gesetztem Schalter.

\newif\ifkorrekturansicht
\korrekturansichtfalse

\input{../tex-inputs/latex-vorspann}


\section[Olga Schnitzler an Stefan Zweig, 3. 5. 1916]{L03812 Olga Schnitzler an Stefan Zweig, 3. 5. 1916}
\nopagebreak\mylabel{L03812v}
\rehead{ }\normalsize\beginnumbering\briefempfaengerindex{Zweig, Stefan@\textsc{Zweig, Stefan}!zzzSchnitzler, Olga@\emph{von Olga Schnitzler}!1916-05-031@{3. 5. 1916}|(be}
\toendnotes[C]{\smallbreak\pagebreak[2]}
\correspDesc{Versand  durch Olga Schnitzler am 3. 5. 1916 in Wien
\newline{}Erhalt  durch Stefan Zweig im Zeitraum [3. 5. 1916
                  – 6. 5. 1916?] in Wien}\toendnotes[C]{\smallbreak}
\Standort{Jerusalem, National Library of Israel, ARC. Ms. Var. 305 1 58 Stefan Zweig Collection.}
\physDesc{Brief, 1 Blatt, 2 Seiten, 893 Zeichen
\newline{}Handschrift: schwarze Tinte, lateinische Kurrent}\toendnotes[C]{\smallbreak}
\pstart
           {\pb}\textcolor{gray}{\textbf{O. S.}}\pend
           \vspace{0.5em}
\pstart
           Lieber Herr Doctor, für Ihre so \label{K_L03812-1v}\edtext{freundlichen Worte}{\lemma{\textnormal{\emph{freundlichen Worte}}}\Cendnote{\textnormal{nicht erhalten}}}\label{K_L03812-1} haben Sie herzlichen Dank! ich freue
               mich, dass meine Stimme \label{K_L03812-2v}\edtext{neulich\eventindex{Allgemeine Poliklinik [neues Gebäude]@\textbf{Allgemeine Poliklinik [neues Gebäude]}!Gesangskonzert von Olga Schnitzler, 29.4.1916@Gesangskonzert von Olga Schnitzler, 29.4.1916|pwv}}{\lemma{\textnormal{\emph{neulich}}}\Cendnote{\textnormal{Am 29. 4. 1916{ }sang Olga Schnitzler\pwindex{Schnitzler, Olga 17.\,1.\,1882 Wien – 13.\,1.\,1970 Lugano@\textsc{Schnitzler, Olga} (17.\,1.\,1882 Wien – 13.\,1.\,1970 Lugano), \emph{Schauspielerin, Sängerin}|pwk}\eventindex{Allgemeine Poliklinik [neues Gebäude]@\textbf{Allgemeine Poliklinik [neues Gebäude]}!Gesangskonzert von Olga Schnitzler, 29.4.1916@Gesangskonzert von Olga Schnitzler, 29.4.1916|pwkv} im neuen Hörsaal der Allgemeinen
                     Poliklinik\oindex{Wien@\textbf{Wien}!IX., Alsergrund@\textbf{IX., Alsergrund}!Allgemeine Poliklinik [neues Gebäude]@\textbf{Allgemeine Poliklinik [neues Gebäude]}, \emph{Krankenhaus}|pwk}. }}}\label{K_L03812-2}, trotz des unbehaglichen und ungünstigen Raumes, doch
               halbwegs gut geklungen hat.\pend
           
\pstart
           An \label{K_L03812-3v}\edtext{jenem Abend im Volksheim\oindex{Wien@\textbf{Wien}!XVI., Ottakring@\textbf{XVI., Ottakring}!Volkshochschule Ottakring@\textbf{Volkshochschule Ottakring}, \emph{Gebäude}|pw}\eventindex{Volkshochschule Ottakring@\textbf{Volkshochschule Ottakring}!Gesangskonzert Olga Schnitzler, 5.2.1911@Gesangskonzert Olga Schnitzler, 5.2.1911|pwv}}{\lemma{\textnormal{\emph{jenem Abend im Volksheim}}}\Cendnote{\textnormal{Der erste öffentliche Auftritt von Olga Schnitzler\pwindex{Schnitzler, Olga 17.\,1.\,1882 Wien – 13.\,1.\,1970 Lugano@\textsc{Schnitzler, Olga} (17.\,1.\,1882 Wien – 13.\,1.\,1970 Lugano), \emph{Schauspielerin, Sängerin}|pwk}\eventindex{Volkshochschule Ottakring@\textbf{Volkshochschule Ottakring}!Gesangskonzert Olga Schnitzler, 5.2.1911@Gesangskonzert Olga Schnitzler, 5.2.1911|pwkv} als Sängering fand am 5. 2. 1911 im
                  Vereinsgebäude Volksheim\oindex{Wien@\textbf{Wien}!XVI., Ottakring@\textbf{XVI., Ottakring}!Volkshochschule Ottakring@\textbf{Volkshochschule Ottakring}, \emph{Gebäude}|pwk} statt. }}}\label{K_L03812-3} habe
               ich zum überhaupt ersten Mal öffentlich gesungen, und habe damals weder meine Stimme
               noch meine Nerven beherrschen können. Hätt ich nur damals schon bei Herrn
               Kammersänger Steiner\pwindex{Steiner, Franz 15.\,9.\,1873 Sopron – 4.\,11.\,1954 Mexico City@\textsc{Steiner, Franz} (15.\,9.\,1873 Sopron – 4.\,11.\,1954 Mexico City), \emph{Sänger}|pw} studiert! mir wäre
               mancher Umweg erspart geblieben.\pend
           
\pstart
           {\pb}Es wird Sie wahrscheinlich interessieren, denke ich, dass
                  Arthur wieder glücklicherweise in’s Arbeiten
               gekommen ist, – er hat mir am \label{K_L03812-4v}\edtext{Ostermontag eine neue Novelle\pwindex{Schnitzler, Arthur 15.\,5.\,1862 Wien – 21.\,10.\,1931 ebd.@\textsc{Schnitzler, Arthur} (15.\,5.\,1862 Wien – 21.\,10.\,1931 ebd.), \emph{Schriftsteller, Mediziner}!Flucht in die Finsternis@\strich\emph{Flucht in die Finsternis}|pwv} im Umfang von »Frau Beate\pwindex{Schnitzler, Arthur 15.\,5.\,1862 Wien – 21.\,10.\,1931 ebd.@\textsc{Schnitzler, Arthur} (15.\,5.\,1862 Wien – 21.\,10.\,1931 ebd.), \emph{Schriftsteller, Mediziner}!Frau Beate und ihr Sohn. Novelle@\strich\emph{Frau Beate und ihr Sohn. Novelle}|pw}« vorgelesen\eventindex{Sternwartestraße 71@\textbf{Sternwartestraße 71}!Private Lesung von Wahnsinnsnovelle [Flucht in die Finsternis], 23.4.1916@Private Lesung von Wahnsinnsnovelle [Flucht in die Finsternis], 23.4.1916|pwv}}{\lemma{\textnormal{\emph{Ostermontag … vorgelesen}}}\Cendnote{\textnormal{Tatsächlich dürfte es der Ostersonntag
                  gewesen sein, vgl. A. S.: \emph{Kulturveranstaltungen}, 23. 4. 1916.}}}\label{K_L03812-4}, – eine ebenso \label{K_L03812-5v}\edtext{grosse\pwindex{Schnitzler, Arthur 15.\,5.\,1862 Wien – 21.\,10.\,1931 ebd.@\textsc{Schnitzler, Arthur} (15.\,5.\,1862 Wien – 21.\,10.\,1931 ebd.), \emph{Schriftsteller, Mediziner}!Doktor Gräsler, Badearzt@\strich\emph{Doktor Gräsler, Badearzt}|pwv} ist, seit Monaten
                  fertig}{\lemma{\textnormal{\emph{grosse … fertig}}}\Cendnote{\textnormal{Vgl. A. S.: \emph{Tagebuch}, 8. 11. 1914.}}}\label{K_L03812-5}, – und
               nehmen Sie mir’s nicht übel, wenn ich beide\pwindex{Schnitzler, Arthur 15.\,5.\,1862 Wien – 21.\,10.\,1931 ebd.@\textsc{Schnitzler, Arthur} (15.\,5.\,1862 Wien – 21.\,10.\,1931 ebd.), \emph{Schriftsteller, Mediziner}!Flucht in die Finsternis@\strich\emph{Flucht in die Finsternis}|pwv}\pwindex{Schnitzler, Arthur 15.\,5.\,1862 Wien – 21.\,10.\,1931 ebd.@\textsc{Schnitzler, Arthur} (15.\,5.\,1862 Wien – 21.\,10.\,1931 ebd.), \emph{Schriftsteller, Mediziner}!Doktor Gräsler, Badearzt@\strich\emph{Doktor Gräsler, Badearzt}|pwv} – so verschieden sie sind, – sehr schön finde.\pend
           
\pstart
           Seien Sie herzlich gegrüsst – hoffentlich hören wir bald wieder von Ihnen!{\\[\baselineskip]}Ihre{\\[\baselineskip]}\spacefill\mbox{OlgaSchnitzler}\pend
           \leftskip=0em{}
\pstart
           3. Mai 1916.\pend
           \selectlanguage{ngerman}\endnumbering\briefempfaengerindex{Zweig, Stefan@\textsc{Zweig, Stefan}!zzzSchnitzler, Olga@\emph{von Olga Schnitzler}!1916-05-031@{3. 5. 1916}|)be}\mylabel{L03812h}  \newcommand{\dateiname}{L03812}\newcommand{\titel}{Olga Schnitzler an Stefan Zweig, 3. 5. 1916}\newcommand{\editorInnen}{Selma Jahnke und Martin Anton Müller}%% latex-leseansicht-abspann.tex
%% Abspann für die Leseansicht.
%% Der Schalter \ifkorrekturansicht ist bereits durch den Vorspann gesetzt.

%% latex-abspann.tex
%% Gemeinsamer Abspann für Korrekturansicht und Leseansicht.
%% Setzt den Schalter \ifkorrekturansicht voraus (gesetzt in den
%% einbindenden Dateien latex-korrekturansicht-abspann.tex bzw.
%% latex-leseansicht-abspann.tex).
%% ---------------------------------------------------------------

\normalsize

% Das esempio-Environment wird nur in der Leseansicht benötigt
\ifkorrekturansicht\else
\newenvironment{esempio}[3]%
{
    \vspace{1.5ex}
    \rlap{\underline{#1}}
    \par
    \setlength{\parindent}{0cm}
    \nopagebreak
    \leftskip=#2cm
    \rightskip=#3cm
}
{
    \par
}
\fi

\doendnotes{C}
\bigskip
\vfill

\clearpage

\footnotesize

\ifkorrekturansicht
  \lohead{\textsc{register}}
\fi

% theindex-Environment neu definieren ohne reledmac
\makeatletter
\renewenvironment{theindex}{%
  \ifkorrekturansicht
    \section*{\indexname}%
  \else
    \subsubsection*{Index der erwähnten Entitäten}%
  \fi
  \setlength{\parindent}{0pt}%
  \setlength{\parskip}{0pt plus 0.3pt}%
  \let\item\@idxitem
}{%
  \ifkorrekturansicht\clearpage\fi
}
\makeatother

\IfFileExists{\jobname-pw.ind}{\input{\jobname-pw.ind}}{}

% Quellenangabe nur in der Leseansicht
\ifkorrekturansicht\else
% Fallback-Definitionen, falls die .tex-Datei \titel etc. nicht gesetzt hat
\providecommand{\titel}{}
\providecommand{\editorInnen}{}
\providecommand{\dateiname}{\jobname}

\vspace{3cm}

\vfill

\footnotesize
\textsc{Quelle}: \titel. Herausgegeben von {\editorInnen}. In: \emph{Arthur Schnitzler: Briefwechsel mit Autorinnen und Autoren}.
 Digitale Edition, https://schnitzler-briefe.acdh.oeaw.ac.at/{\dateiname}.html (Stand \today)
\fi

\end{document}


