%% latex-korrekturansicht-vorspann.tex
%% Vorspann für die Korrekturansicht.
%% Lädt die gemeinsame Datei latex-vorspann.tex mit gesetztem Schalter.

\newif\ifkorrekturansicht
\korrekturansichttrue

\input{../tex-inputs/latex-vorspann}


\section[Arthur Schnitzler an Stefan Zweig, 3. 5. 1916]{L03812 Arthur Schnitzler an Stefan Zweig, 3. 5. 1916}
\nopagebreak\mylabel{L03812v}
\rehead{ }\normalsize\beginnumbering\briefempfaengerindex{Zweig, Stefan@\textsc{Zweig, Stefan}!zzzSchnitzler, Arthur@\emph{von Arthur Schnitzler}!1916-05-031@{3. 5. 1916}|(be}
\toendnotes[C]{\smallbreak\pagebreak[2]}\Standort{Jerusalem, National Library of Israel, ARC. Ms. Var. 305 1 58 Stefan Zweig Collection.}
\physDesc{Brief, 1 Blatt, 2 Seiten, 896 Zeichen
\newline{}Handschrift: schwarze Tinte, lateinische Kurrent}\toendnotes[C]{\smallbreak}
\pstart
           {\pb}\textcolor{gray}{\textbf{O. S.}}\pend
           \vspace{0.5em}
\pstart
           Lieber Herr Doctor, für Ihre so \label{K_L03813-1v}\edtext{freundlichen Worte}{\lemma{\textnormal{\emph{freundlichen Worte}}}\Cendnote{\textnormal{nicht erhalten}}}\label{K_L03813-1} haben Sie herzlichen Dank! ich freue
               mich, dass meine Stimme \label{K_L03813-2v}\edtext{neulich\eventindex{Allgemeine Poliklinik [neues Gebaeude]@\textbf{Allgemeine Poliklinik [neues Gebäude]}!Gesangskonzert von Olga Schnitzler, 29.4.1916@Gesangskonzert von Olga Schnitzler, 29.4.1916|pwv}}{\lemma{\textnormal{\emph{neulich}}}\Cendnote{\textnormal{Am 29. 4. 1916{ }sang Olga Schnitzler\pwindex{Schnitzler, Olga 17.01.1882 – 13.01.1970@\textsc{Schnitzler, Olga} (17.01.1882 – 13.01.1970), \emph{Schauspieler/Schauspielerin, Sänger/Sängerin}|pwk}\eventindex{Allgemeine Poliklinik [neues Gebaeude]@\textbf{Allgemeine Poliklinik [neues Gebäude]}!Gesangskonzert von Olga Schnitzler, 29.4.1916@Gesangskonzert von Olga Schnitzler, 29.4.1916|pwkv} im neuen Hörsaal der Allgemeinen
                     Poliklinik\oindex{Allgemeine Poliklinik [neues Gebaeude]@\textbf{Allgemeine Poliklinik [neues Gebäude]}, \emph{Krankenhaus (K.KKH)}|pwk}. }}}\label{K_L03813-2}, trotz des unbehaglichen und ungünstigen Raumes, doch
               halbwegs gut geklungen hat. An \label{K_L03813-3v}\edtext{jenem
                  Abend im Volksheim\oindex{Volkshochschule Ottakring@\textbf{Volkshochschule Ottakring}, \emph{Gebäude (K.GBD)}|pw}\eventindex{Volkshochschule Ottakring@\textbf{Volkshochschule Ottakring}!Gesangskonzert Olga Schnitzler, 5.2.1911@Gesangskonzert Olga Schnitzler, 5.2.1911|pwv}}{\lemma{\textnormal{\emph{jenem
                  Abend im Volksheim}}}\Cendnote{\textnormal{Der erste öffentliche Auftritt von Olga Schnitzler\pwindex{Schnitzler, Olga 17.01.1882 – 13.01.1970@\textsc{Schnitzler, Olga} (17.01.1882 – 13.01.1970), \emph{Schauspieler/Schauspielerin, Sänger/Sängerin}|pwk} als Sängering fand am 5. 2. 1911 am \emph{Verein Volksheim}\orgindex{Verein Volksheim@Verein Volksheim|pwk}\eventindex{Volkshochschule Ottakring@\textbf{Volkshochschule Ottakring}!Gesangskonzert Olga Schnitzler, 5.2.1911@Gesangskonzert Olga Schnitzler, 5.2.1911|pwkv} statt. }}}\label{K_L03813-3} habe ich zum überhaupt ersten Mal öffentlich gesungen, und
               habe damals weder meine Stimme noch meine Nerven beherrschen können. Hätt ich nur
               damals schon bei Herrn Kammersänger Steiner\pwindex{Steiner, Franz 15.09.1873 – 04.11.1954@\textsc{Steiner, Franz} (15.09.1873 – 04.11.1954), \emph{Sänger/Sängerin}|pw}
               studiert! mir wäre mancher Umweg erspart geblieben.\pend
           
\pstart
           {\pb}Es wird Sie wahrscheinlich interessieren, denke ich, dass
                  Arthur wieder glücklicherweise in’s Arbeiten
               gekommen ist, – er hat mir am \label{K_L03813-4v}\edtext{Ostermontag eine seine Novelle\pwindex{Flucht in die Finsternis@\emph{Flucht in die Finsternis}|pwv} im Umfang von »Frau Beate\pwindex{Frau Beate und ihr Sohn. Novelle@\emph{Frau Beate und ihr Sohn. Novelle}|pw}«
                  vorgelesen}{\lemma{\textnormal{\emph{Ostermontag … vorgelesen}}}\Cendnote{\textnormal{Tatsächlich dürfte es der
                  Ostersonntag gewesen sein, vgl. A. S.: \emph{Kulturveranstaltungen}, 23. 4. 1916.}}}\label{K_L03813-4}, – eine ebenso \label{K_L03813-5v}\edtext{grosse\pwindex{Doktor Graesler, Badearzt@\emph{Doktor Gräsler, Badearzt}|pwv} ist, seit Monaten
                  fertig}{\lemma{\textnormal{\emph{grosse … fertig}}}\Cendnote{\textnormal{Vgl. A. S.: \emph{Tagebuch}, 8. 11. 1914.}}}\label{K_L03813-5}, – und
               nehmen Sie mir’s nicht übel, wenn ich beide\pwindex{Flucht in die Finsternis@\emph{Flucht in die Finsternis}|pwv}\pwindex{Doktor Graesler, Badearzt@\emph{Doktor Gräsler, Badearzt}|pwv} – so verschieden sie sind, – sehr schön finde.\pend
           
\pstart
           Seien Sie herzlich gegrüsst – hoffentlich hören wir bald wieder von Ihnen.{\\[\baselineskip]}Ihre{\\[\baselineskip]}\spacefill\mbox{OlgaSchnitzler}\pend
           \leftskip=0em{}
\pstart
           3. Mai 1916.\pend
           \selectlanguage{ngerman}\endnumbering\briefempfaengerindex{Zweig, Stefan@\textsc{Zweig, Stefan}!zzzSchnitzler, Arthur@\emph{von Arthur Schnitzler}!1916-05-031@{3. 5. 1916}|)be}\mylabel{L03812h}
\begin{anhang}
\end{anhang}\normalsize

\doendnotes{C}
\bigskip
\vfill

\clearpage

\footnotesize

\lohead{\textsc{register}}

% Definiere theindex-Environment komplett neu ohne reledmac
\makeatletter
\renewenvironment{theindex}{%
  \section*{\indexname}%
  \setlength{\parindent}{0pt}%
  \setlength{\parskip}{0pt plus 0.3pt}%
  \let\item\@idxitem
}{%
  \clearpage
}
\makeatother

\IfFileExists{\jobname-pw.ind}{\input{\jobname-pw.ind}}{}

\end{document}

      