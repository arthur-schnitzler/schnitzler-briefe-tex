%% latex-korrekturansicht-vorspann.tex
%% Vorspann für die Korrekturansicht.
%% Lädt die gemeinsame Datei latex-vorspann.tex mit gesetztem Schalter.

\newif\ifkorrekturansicht
\korrekturansichttrue

\input{../tex-inputs/latex-vorspann}


\section[Elsa Plessner an Arthur Schnitzler, 15. 9. 1896]{L03701 Elsa Plessner an Arthur Schnitzler, 15. 9. 1896}
\nopagebreak\mylabel{L03701v}
\rehead{ }\normalsize\beginnumbering\briefempfaengerindex{Schnitzler, Arthur@\textsc{Schnitzler, Arthur}!zzzPlessner, Elsa@\emph{von Elsa Plessner}!1896-09-152@{15. 9. 1896}|(be}
\toendnotes[C]{\smallbreak\pagebreak[2]}\Standort{DLA, A:Schnitzler, HS.1985.1.419.}
\physDesc{Brief,  Blätter, 3 Seiten, 770 Zeichen
\newline{}Handschrift: , lateinische Kurrent
\newline{}Schnitzler: eine Unterstreichung }\toendnotes[C]{\smallbreak}
\pstart
           {\pb} I. Bäckerstraße N\textsuperscript{o}
                     1\oindex{Baeckerstrasse 1@\textbf{Bäckerstraße 1}, \emph{Wohngebäude (K.WHS)}|pw}, den 14. 9. 96. \pend
           
\pstart{}Hochverehrter Herr!\pend\vspace{0.5em}
\pstart
           Der Plagegeist, der Sie im vergangenen Winter mit \label{K_L03701-1v}\edtext{Manuscripten}{\lemma{\textnormal{\emph{Manuscripten}}}\Cendnote{\textnormal{Im Frühjahr des Jahres 1896 hatte Elsa Plessner\pwindex{Plessner, Elsa 22.08.1875 – 01.05.1932@\textsc{Plessner, Elsa} (22.08.1875 – 01.05.1932), \emph{Schriftsteller/Schriftstellerin}|pwk}{ }Schnitzler ihre
                  Schauspiel \emph{Heimkehr}\pwindex{Heimweh [dreiaktige Tragikomoedie]@\emph{Heimweh [dreiaktige Tragikomödie]}|pwk} (14. 3. 1896), den
                  Entwurf zur Novelle \emph{Warten}\pwindex{Warten@\emph{Warten}|pwk} (14. 4. 1896), neunzehn Gedichte
                  unter dem Titel \emph{Pierettes Tagebuch}\pwindex{Pierettes Tagebuch [19 unveroeffentlichte Gedichte]@\emph{Pierettes Tagebuch [19 unveröffentlichte Gedichte]}|pwk} und zwei weitere
                  kurze Texte gesendet (18. 3. 1896).}}}\label{K_L03701-1} bombardirt hat und dem Sie in himmlischer Geduld
               mehrmals schriftlich \label{K_L03701-2v}\edtext{Rede und
                  Antwort}{\lemma{\textnormal{\emph{Rede und
                  Antwort}}}\Cendnote{\textnormal{Schnitzlers Briefe an Elsa Plessner\pwindex{Plessner, Elsa 22.08.1875 – 01.05.1932@\textsc{Plessner, Elsa} (22.08.1875 – 01.05.1932), \emph{Schriftsteller/Schriftstellerin}|pwk} sind nicht überliefert.}}}\label{K_L03701-2} – will sagen
               Urtheil – standen, erlaubt sich hiemit die höfl. Anfrage, ob und wann Sie ihm in
               einer für ihn außerordentlich wichtigen Angelegenheit eine Audienz bewilligen. Es
               handelt sich um das Ihnen bekannte drei-actige Drama\pwindex{Heimweh [dreiaktige Tragikomoedie]@\emph{Heimweh [dreiaktige Tragikomödie]}|pwv}. –\pend
           
\pstart
           Wenn sie die große Liebenswürdigkeit haben wollten, mir mitzutheilen, wann Sie die
               noch größere besitzen {\pb}werden, für mich zu sprechen zu sein so bringen
               Sie das Maß Ihrer engelhaften Güte mir gegenüber zum Überfließen. –\pend
           
\pstart
           Und harrend der freudigen Botschaft zeichnet mit neuem Dank im Voraus – und
               alter, hochachtungsvoller Verehrung{\\[\baselineskip]}\spacefill\mbox{Elsa Plessner}\pend
           \leftskip=0em{}\selectlanguage{ngerman}\endnumbering\briefempfaengerindex{Schnitzler, Arthur@\textsc{Schnitzler, Arthur}!zzzPlessner, Elsa@\emph{von Elsa Plessner}!1896-09-152@{15. 9. 1896}|)be}\mylabel{L03701h}
\begin{anhang}
\end{anhang}\normalsize

\doendnotes{C}
\bigskip
\vfill

\clearpage

\footnotesize

\lohead{\textsc{register}}

% Definiere theindex-Environment komplett neu ohne reledmac
\makeatletter
\renewenvironment{theindex}{%
  \section*{\indexname}%
  \setlength{\parindent}{0pt}%
  \setlength{\parskip}{0pt plus 0.3pt}%
  \let\item\@idxitem
}{%
  \clearpage
}
\makeatother

\IfFileExists{\jobname-pw.ind}{\input{\jobname-pw.ind}}{}

\end{document}

      