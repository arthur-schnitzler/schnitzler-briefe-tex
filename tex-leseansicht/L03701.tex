%% latex-leseansicht-vorspann.tex
%% Vorspann für die Leseansicht.
%% Lädt die gemeinsame Datei latex-vorspann.tex mit nicht gesetztem Schalter.

\newif\ifkorrekturansicht
\korrekturansichtfalse

\input{../tex-inputs/latex-vorspann}


\section[Elsa Plessner an Arthur Schnitzler, 15. 9. 1896]{L03701 Elsa Plessner an Arthur Schnitzler, 15. 9. 1896}
\nopagebreak\mylabel{L03701v}
\rehead{ }\normalsize\beginnumbering\briefempfaengerindex{Schnitzler, Arthur@\textsc{Schnitzler, Arthur}!zzzPlessner, Elsa@\emph{von Elsa Plessner}!1896-09-152@{15. 9. 1896}|(be}
\toendnotes[C]{\smallbreak\pagebreak[2]}
\correspDesc{Versand  durch Elsa Plessner am 15. 9. 1896 in Wien
\newline{}Erhalt  durch Arthur Schnitzler im Zeitraum [16. 9. 1896
                  – 20. 9. 1896?] in Wien}\toendnotes[C]{\smallbreak}
\Standort{DLA, A:Schnitzler, HS.1985.1.419.}
\physDesc{Brief, 1 Blatt, 3 Seiten, 770 Zeichen
\newline{}Handschrift: schwarze Tinte, lateinische Kurrent
\newline{}Schnitzler: mit rotem Buntstift eine Unterstreichung }\toendnotes[C]{\smallbreak}
\pstart
           {\pb}I. Bäckerstraße N\textsuperscript{o}
                     1\oindex{Wien@\textbf{Wien}!I., Innere Stadt@\textbf{I., Innere Stadt}!Bäckerstraße 1@\textbf{Bäckerstraße 1}, \emph{Wohngebäude}|pw}, den 14. 9. 96.\pend
           
\pstart\center{}Hochverehrter Herr!\pend\vspace{0.5em}
\pstart
           Der Plagegeist, der Sie im vergangenen Winter mit \label{K_L03701-1v}\edtext{Manuscripten\pwindex{Plessner, Elsa 22.\,8.\,1875 Wien – 7.\,5.\,1932 Alicante@\textsc{Plessner, Elsa} (22.\,8.\,1875 Wien – 7.\,5.\,1932 Alicante), \emph{Schriftstellerin}!Heimweh [dreiaktige Tragikomödie]@\strich\emph{Heimweh [dreiaktige Tragikomödie]}|pwv}\pwindex{Plessner, Elsa 22.\,8.\,1875 Wien – 7.\,5.\,1932 Alicante@\textsc{Plessner, Elsa} (22.\,8.\,1875 Wien – 7.\,5.\,1932 Alicante), \emph{Schriftstellerin}!Begräbnißtag@\strich\emph{Der Begräbnißtag}|pwv}\pwindex{Plessner, Elsa 22.\,8.\,1875 Wien – 7.\,5.\,1932 Alicante@\textsc{Plessner, Elsa} (22.\,8.\,1875 Wien – 7.\,5.\,1932 Alicante), \emph{Schriftstellerin}!Baby@\strich\emph{Baby}|pwv}\pwindex{Plessner, Elsa 22.\,8.\,1875 Wien – 7.\,5.\,1932 Alicante@\textsc{Plessner, Elsa} (22.\,8.\,1875 Wien – 7.\,5.\,1932 Alicante), \emph{Schriftstellerin}!Pierettes Tagebuch [19 unveröffentlichte Gedichte]@\strich\emph{Pierettes Tagebuch [19 unveröffentlichte Gedichte]}|pwv}\pwindex{Plessner, Elsa 22.\,8.\,1875 Wien – 7.\,5.\,1932 Alicante@\textsc{Plessner, Elsa} (22.\,8.\,1875 Wien – 7.\,5.\,1932 Alicante), \emph{Schriftstellerin}!Warten. Novelle@\strich\emph{Warten. Novelle}|pwv}}{\lemma{\textnormal{\emph{Manuscripten}}}\Cendnote{\textnormal{Im Frühjahr des Jahres 1896 hatte Elsa Plessner\pwindex{Plessner, Elsa 22.\,8.\,1875 Wien – 7.\,5.\,1932 Alicante@\textsc{Plessner, Elsa} (22.\,8.\,1875 Wien – 7.\,5.\,1932 Alicante), \emph{Schriftstellerin}|pwk}{ }Schnitzler ihre Schauspiel \emph{Heimkehr}\pwindex{Plessner, Elsa 22.\,8.\,1875 Wien – 7.\,5.\,1932 Alicante@\textsc{Plessner, Elsa} (22.\,8.\,1875 Wien – 7.\,5.\,1932 Alicante), \emph{Schriftstellerin}!Heimweh [dreiaktige Tragikomödie]@\strich\emph{Heimweh [dreiaktige Tragikomödie]}|pwk} (XXXX Auszeichnungsfehler: Dokument L03698 nicht gefunden), neunzehn Gedichte unter dem Titel \emph{Pierettes
                     Tagebuch}\pwindex{Plessner, Elsa 22.\,8.\,1875 Wien – 7.\,5.\,1932 Alicante@\textsc{Plessner, Elsa} (22.\,8.\,1875 Wien – 7.\,5.\,1932 Alicante), \emph{Schriftstellerin}!Pierettes Tagebuch [19 unveröffentlichte Gedichte]@\strich\emph{Pierettes Tagebuch [19 unveröffentlichte Gedichte]}|pwk}, \emph{Baby}\pwindex{Plessner, Elsa 22.\,8.\,1875 Wien – 7.\,5.\,1932 Alicante@\textsc{Plessner, Elsa} (22.\,8.\,1875 Wien – 7.\,5.\,1932 Alicante), \emph{Schriftstellerin}!Baby@\strich\emph{Baby}|pwk} und \emph{Der Begräbnißtag}\pwindex{Plessner, Elsa 22.\,8.\,1875 Wien – 7.\,5.\,1932 Alicante@\textsc{Plessner, Elsa} (22.\,8.\,1875 Wien – 7.\,5.\,1932 Alicante), \emph{Schriftstellerin}!Begräbnißtag@\strich\emph{Der Begräbnißtag}|pwk} (XXXX Auszeichnungsfehler: Dokument L03699 nicht gefunden) sowie den Entwurf zur Novelle \emph{Warten}\pwindex{Plessner, Elsa 22.\,8.\,1875 Wien – 7.\,5.\,1932 Alicante@\textsc{Plessner, Elsa} (22.\,8.\,1875 Wien – 7.\,5.\,1932 Alicante), \emph{Schriftstellerin}!Warten. Novelle@\strich\emph{Warten. Novelle}|pwk} (XXXX Auszeichnungsfehler: Dokument L03700 nicht gefunden) gesandt.}}}\label{K_L03701-1}
               bombardirt hat und dem Sie in himmlischer Geduld mehrmals schriftlich \label{K_L03701-2v}\edtext{Rede und Antwort}{\lemma{\textnormal{\emph{Rede und Antwort}}}\Cendnote{\textnormal{Schnitzlers Briefe an Elsa Plessner\pwindex{Plessner, Elsa 22.\,8.\,1875 Wien – 7.\,5.\,1932 Alicante@\textsc{Plessner, Elsa} (22.\,8.\,1875 Wien – 7.\,5.\,1932 Alicante), \emph{Schriftstellerin}|pwk} sind nicht überliefert.}}}\label{K_L03701-2} – will sagen
               Urtheil – standen, erlaubt sich hiemit die höfl. Anfrage, ob und wann Sie ihm in
               einer für ihn außerordentlich wichtigen Angelegenheit eine Audienz bewilligen. Es
               handelt sich um das Ihnen bekannte drei-actige Drama\pwindex{Plessner, Elsa 22.\,8.\,1875 Wien – 7.\,5.\,1932 Alicante@\textsc{Plessner, Elsa} (22.\,8.\,1875 Wien – 7.\,5.\,1932 Alicante), \emph{Schriftstellerin}!Heimweh [dreiaktige Tragikomödie]@\strich\emph{Heimweh [dreiaktige Tragikomödie]}|pwv}. –\pend
           
\pstart
           Wenn sie die große Liebenswürdigkeit haben wollten, mir mitzutheilen, \label{K_L03701-11v}\edtext{wann}{\lemma{\textnormal{\emph{wann}}}\Cendnote{\textnormal{Das erste
                  persönliche Zusammentreffen fand am 15. 9. 1896 statt.}}}\label{K_L03701-11} Sie die
               noch größere besitzen {\pb}werden, für mich zu sprechen zu
               sein so bringen Sie das Maß Ihrer engelhaften Güte mir gegenüber zum Überfließen.
               –\pend
           
\pstart
           Und harrend der freudigen Botschaft zeichnet mit neuem Dank im Voraus – und
               alter, hochachtungsvoller Verehrung{\\[\baselineskip]}\spacefill\mbox{Elsa Plessner.}\pend
           \leftskip=0em{}\selectlanguage{ngerman}\endnumbering\briefempfaengerindex{Schnitzler, Arthur@\textsc{Schnitzler, Arthur}!zzzPlessner, Elsa@\emph{von Elsa Plessner}!1896-09-152@{15. 9. 1896}|)be}\mylabel{L03701h}  \newcommand{\dateiname}{L03701}\newcommand{\titel}{Elsa Plessner an Arthur Schnitzler, 15. 9. 1896}\newcommand{\editorInnen}{Selma Jahnke und Martin Anton Müller}%% latex-leseansicht-abspann.tex
%% Abspann für die Leseansicht.
%% Der Schalter \ifkorrekturansicht ist bereits durch den Vorspann gesetzt.

%% latex-abspann.tex
%% Gemeinsamer Abspann für Korrekturansicht und Leseansicht.
%% Setzt den Schalter \ifkorrekturansicht voraus (gesetzt in den
%% einbindenden Dateien latex-korrekturansicht-abspann.tex bzw.
%% latex-leseansicht-abspann.tex).
%% ---------------------------------------------------------------

\normalsize

% Das esempio-Environment wird nur in der Leseansicht benötigt
\ifkorrekturansicht\else
\newenvironment{esempio}[3]%
{
    \vspace{1.5ex}
    \rlap{\underline{#1}}
    \par
    \setlength{\parindent}{0cm}
    \nopagebreak
    \leftskip=#2cm
    \rightskip=#3cm
}
{
    \par
}
\fi

\doendnotes{C}
\bigskip
\vfill

\clearpage

\footnotesize

\ifkorrekturansicht
  \lohead{\textsc{register}}
\fi

% theindex-Environment neu definieren ohne reledmac
\makeatletter
\renewenvironment{theindex}{%
  \ifkorrekturansicht
    \section*{\indexname}%
  \else
    \subsubsection*{Index der erwähnten Entitäten}%
  \fi
  \setlength{\parindent}{0pt}%
  \setlength{\parskip}{0pt plus 0.3pt}%
  \let\item\@idxitem
}{%
  \ifkorrekturansicht\clearpage\fi
}
\makeatother

\IfFileExists{\jobname-pw.ind}{\input{\jobname-pw.ind}}{}

% Quellenangabe nur in der Leseansicht
\ifkorrekturansicht\else
% Fallback-Definitionen, falls die .tex-Datei \titel etc. nicht gesetzt hat
\providecommand{\titel}{}
\providecommand{\editorInnen}{}
\providecommand{\dateiname}{\jobname}

\vspace{3cm}

\vfill

\footnotesize
\textsc{Quelle}: \titel. Herausgegeben von {\editorInnen}. In: \emph{Arthur Schnitzler: Briefwechsel mit Autorinnen und Autoren}.
 Digitale Edition, https://schnitzler-briefe.acdh.oeaw.ac.at/{\dateiname}.html (Stand \today)
\fi

\end{document}


