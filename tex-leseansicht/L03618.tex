%% latex-leseansicht-vorspann.tex
%% Vorspann für die Leseansicht.
%% Lädt die gemeinsame Datei latex-vorspann.tex mit nicht gesetztem Schalter.

\newif\ifkorrekturansicht
\korrekturansichtfalse

\input{../tex-inputs/latex-vorspann}


\section[Arthur Schnitzler an Karl Emil Franzos, 29. 4. 1888]{L03618 Arthur Schnitzler an Karl Emil Franzos, 29. 4. 1888}
\nopagebreak\mylabel{L03618v}
\rehead{ }\normalsize\beginnumbering\briefempfaengerindex{Franzos, Karl Emil@\textsc{Franzos, Karl Emil}!zzzSchnitzler, Arthur@\emph{von Arthur Schnitzler}!1888-04-291@{29. 4. 1888}|(be}
\toendnotes[C]{\smallbreak\pagebreak[2]}
\correspDesc{Versand  durch Arthur Schnitzler am 29. 4. 1888 in Berlin
\newline{}Erhalt  durch Karl Emil Franzos im Zeitraum [29. 4. 1888
                  – 2. 5. 1888?] in Berlin}\toendnotes[C]{\smallbreak}
\Standort{Wienbibliothek im Rathaus, H.I.N.-60194.}
\physDesc{Brief, 1 Blatt, 2 Seiten, 639 Zeichen
\newline{}Handschrift: schwarze Tinte, deutsche Kurrent}
\buchAbdrucke{\weitereDrucke{Arthur Schnitzler: \emph{Briefe 1875–1912}. Herausgegeben von Therese Nickl und Heinrich Schnitzler. Frankfurt am Main: \emph{S. Fischer} 1981, S. 28.} }\toendnotes[C]{\smallbreak}
\pstart
           \raggedleft{}{\pb}\textsc{Berlin\oindex{Berlin@\textbf{Berlin}, \emph{Hauptstadt}|pw}}{ }29. 4. 88.\pend
           
\pstart{}Hochgeehrter Herr!\pend\vspace{0.5em}
\pstart
           Ich nehme mir die Freiheit, Ihnen zwei \label{K_L03618-1v}\edtext{Erzählungen}{\lemma{\textnormal{\emph{Erzählungen}}}\Cendnote{\textnormal{Von den erhaltenen Prosaarbeiten, die in diesem Zeitraum
                     entstanden, kommen \emph{Erbschaft}\pwindex{Schnitzler, Arthur 15.\,5.\,1862 Wien – 21.\,10.\,1931 ebd.@\textsc{Schnitzler, Arthur} (15.\,5.\,1862 Wien – 21.\,10.\,1931 ebd.), \emph{Schriftsteller, Mediziner}!Erbschaft@\strich\emph{Erbschaft}|pwk}, \emph{Mein Freund Ypsilon. Aus den Papieren eines
                        Arztes}\pwindex{Schnitzler, Arthur 15.\,5.\,1862 Wien – 21.\,10.\,1931 ebd.@\textsc{Schnitzler, Arthur} (15.\,5.\,1862 Wien – 21.\,10.\,1931 ebd.), \emph{Schriftsteller, Mediziner}!Mein Freund Ypsilon. Aus den Papieren eines Arztes@\strich\emph{Mein Freund Ypsilon. Aus den Papieren eines Arztes}|pwk} und \emph{Amerika}\pwindex{Schnitzler, Arthur 15.\,5.\,1862 Wien – 21.\,10.\,1931 ebd.@\textsc{Schnitzler, Arthur} (15.\,5.\,1862 Wien – 21.\,10.\,1931 ebd.), \emph{Schriftsteller, Mediziner}!Amerika@\strich\emph{Amerika}|pwk} in Frage, vgl. A. S.: \emph{Tagebuch}, 19. 10. 1887 und \emph{Jugend in Wien}\pwindex{Schnitzler, Arthur 15.\,5.\,1862 Wien – 21.\,10.\,1931 ebd.@\textsc{Schnitzler, Arthur} (15.\,5.\,1862 Wien – 21.\,10.\,1931 ebd.), \emph{Schriftsteller, Mediziner}!Jugend in Wien@\strich\emph{Jugend in Wien}|pwk} (Arthur Schnitzler: \emph{Jugend in Wien. Eine Autobiographie}\pwindex{Schnitzler, Arthur 15.\,5.\,1862 Wien – 21.\,10.\,1931 ebd.@\textsc{Schnitzler, Arthur} (15.\,5.\,1862 Wien – 21.\,10.\,1931 ebd.), \emph{Schriftsteller, Mediziner}!Jugend in Wien@\strich\emph{Jugend in Wien}|pwk}. Mit einem
                        Nachwort von Friedrich Torberg. Wien,
                        München, Zürich,
                        New York: \emph{S. Fischer}{ }1968, S. 320).}}}\label{K_L03618-1}\pwindex{Schnitzler, Arthur 15.\,5.\,1862 Wien – 21.\,10.\,1931 ebd.@\textsc{Schnitzler, Arthur} (15.\,5.\,1862 Wien – 21.\,10.\,1931 ebd.), \emph{Schriftsteller, Mediziner}!Amerika@\strich\emph{Amerika}|pwv}\pwindex{Schnitzler, Arthur 15.\,5.\,1862 Wien – 21.\,10.\,1931 ebd.@\textsc{Schnitzler, Arthur} (15.\,5.\,1862 Wien – 21.\,10.\,1931 ebd.), \emph{Schriftsteller, Mediziner}!Mein Freund Ypsilon. Aus den Papieren eines Arztes@\strich\emph{Mein Freund Ypsilon. Aus den Papieren eines Arztes}|pwv}\pwindex{Schnitzler, Arthur 15.\,5.\,1862 Wien – 21.\,10.\,1931 ebd.@\textsc{Schnitzler, Arthur} (15.\,5.\,1862 Wien – 21.\,10.\,1931 ebd.), \emph{Schriftsteller, Mediziner}!Erbschaft@\strich\emph{Erbschaft}|pwv} zu überſenden, von denen ich mir{ }ſelbst kaum einbilden will, daſs{ }ſie für Ihre
                  »\textsc{Dtsch. Dichtung\pwindex{Deutsche Dichtung@\emph{Deutsche Dichtung}|pw}}« der Vorzüge genug beſitzen. Jedenfalls aber wäre mir ein Urtheil von Ihnen
               höchſt erwünſcht, um das Sie hiemit zwar unbeſcheiden aber herzlichſt gebeten{ }ſind.
               Ich unterlieſs es, \label{K_L03618-2v}\edtext{perſönlich}{\lemma{\textnormal{\emph{persönlich}}}\Cendnote{\textnormal{Am 15. 4. 1888 und am 28. 4. 1888 war Schnitzler bei Franzos\pwindex{Franzos, Karl Emil 25.\,10.\,1848 Tschortkiw – 28.\,1.\,1904 Berlin@\textsc{Franzos, Karl Emil} (25.\,10.\,1848 Tschortkiw – 28.\,1.\,1904 Berlin), \emph{Schriftsteller, Journalist}|pwk} auf Besuch in der Kaiserin-Augusta-Straße 71\oindex{Kaiserin-Augusta-Straße 71@\textbf{Kaiserin-Augusta-Straße 71}, \emph{Wohngebäude}|pwk}. Die Einladung zum Souper am Vortag dieses
                  Briefes dürfte eine Folge des Empfehlungsschreibens von Johann Schnitzler\pwindex{Schnitzler, Johann 10.\,4.\,1835 Nagykanizsa – 2.\,5.\,1893 Wien@\textsc{Schnitzler, Johann} (10.\,4.\,1835 Nagykanizsa – 2.\,5.\,1893 Wien), \emph{Laryngologe}|pwk} (XXXX Auszeichnungsfehler: Dokument L03617 nicht gefunden) gewesen sein.}}}\label{K_L03618-2} mit Ihnen über dieſe Sache zu reden, da
               ich in dem Augenblicke dieser {\pb}Bitte am liebſten ein
               ganz und gar unbeka{\geminationn}ter, gewiſs aber nicht der gut
               empfohlene und{ }ſo liebenswürdig aufgeno{\geminationm}ene »Sohn meines
                  Vaters\pwindex{Schnitzler, Johann 10.\,4.\,1835 Nagykanizsa – 2.\,5.\,1893 Wien@\textsc{Schnitzler, Johann} (10.\,4.\,1835 Nagykanizsa – 2.\,5.\,1893 Wien), \emph{Laryngologe}|pwv}«{ }ſein möchte.\pend
           
\pstart
           Mit beſondrer Hochachtung Ihr{\\[\baselineskip]}ergebener{\\[\baselineskip]}\spacefill\mbox{Dr Arthur Schnitzler}\pend
           \leftskip=0em{}\selectlanguage{ngerman}\endnumbering\briefempfaengerindex{Franzos, Karl Emil@\textsc{Franzos, Karl Emil}!zzzSchnitzler, Arthur@\emph{von Arthur Schnitzler}!1888-04-291@{29. 4. 1888}|)be}\mylabel{L03618h}  \newcommand{\dateiname}{L03618}\newcommand{\titel}{Arthur Schnitzler an Karl Emil Franzos, 29. 4. 1888}\newcommand{\editorInnen}{Selma Jahnke und Martin Anton Müller}%% latex-leseansicht-abspann.tex
%% Abspann für die Leseansicht.
%% Der Schalter \ifkorrekturansicht ist bereits durch den Vorspann gesetzt.

%% latex-abspann.tex
%% Gemeinsamer Abspann für Korrekturansicht und Leseansicht.
%% Setzt den Schalter \ifkorrekturansicht voraus (gesetzt in den
%% einbindenden Dateien latex-korrekturansicht-abspann.tex bzw.
%% latex-leseansicht-abspann.tex).
%% ---------------------------------------------------------------

\normalsize

% Das esempio-Environment wird nur in der Leseansicht benötigt
\ifkorrekturansicht\else
\newenvironment{esempio}[3]%
{
    \vspace{1.5ex}
    \rlap{\underline{#1}}
    \par
    \setlength{\parindent}{0cm}
    \nopagebreak
    \leftskip=#2cm
    \rightskip=#3cm
}
{
    \par
}
\fi

\doendnotes{C}
\bigskip
\vfill

\clearpage

\footnotesize

\ifkorrekturansicht
  \lohead{\textsc{register}}
\fi

% theindex-Environment neu definieren ohne reledmac
\makeatletter
\renewenvironment{theindex}{%
  \ifkorrekturansicht
    \section*{\indexname}%
  \else
    \subsubsection*{Index der erwähnten Entitäten}%
  \fi
  \setlength{\parindent}{0pt}%
  \setlength{\parskip}{0pt plus 0.3pt}%
  \let\item\@idxitem
}{%
  \ifkorrekturansicht\clearpage\fi
}
\makeatother

\IfFileExists{\jobname-pw.ind}{\input{\jobname-pw.ind}}{}

% Quellenangabe nur in der Leseansicht
\ifkorrekturansicht\else
% Fallback-Definitionen, falls die .tex-Datei \titel etc. nicht gesetzt hat
\providecommand{\titel}{}
\providecommand{\editorInnen}{}
\providecommand{\dateiname}{\jobname}

\vspace{3cm}

\vfill

\footnotesize
\textsc{Quelle}: \titel. Herausgegeben von {\editorInnen}. In: \emph{Arthur Schnitzler: Briefwechsel mit Autorinnen und Autoren}.
 Digitale Edition, https://schnitzler-briefe.acdh.oeaw.ac.at/{\dateiname}.html (Stand \today)
\fi

\end{document}


