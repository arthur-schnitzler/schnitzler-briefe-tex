%% latex-korrekturansicht-vorspann.tex
%% Vorspann für die Korrekturansicht.
%% Lädt die gemeinsame Datei latex-vorspann.tex mit gesetztem Schalter.

\newif\ifkorrekturansicht
\korrekturansichttrue

\input{../tex-inputs/latex-vorspann}


\section[Arthur Schnitzler an Karl Emil Franzos, 29. 4. 1888]{L03618 Arthur Schnitzler an Karl Emil Franzos, 29. 4. 1888}
\nopagebreak\mylabel{L03618v}
\rehead{ }\normalsize\beginnumbering\briefempfaengerindex{Franzos, Karl Emil@\textsc{Franzos, Karl Emil}!zzzSchnitzler, Arthur@\emph{von Arthur Schnitzler}!1888-04-291@{29. 4. 1888}|(be}
\toendnotes[C]{\smallbreak\pagebreak[2]}\Standort{Wienbibliothek im Rathaus, H.I.N.-60194.}
\physDesc{Brief, 1 Blatt, 2 Seiten, 639 Zeichen
\newline{}Handschrift: schwarze Tinte, deutsche Kurrent}
\buchAbdrucke{\weitereDrucke{Arthur Schnitzler: \emph{Briefe 1875–1912}. Frankfurt am Main: \emph{S. Fischer} 1981, S. 28.} }\toendnotes[C]{\smallbreak}
\pstart
           \raggedleft{}{\pb}\textsc{Berlin\oindex{Berlin@\textbf{Berlin}, \emph{P.PPLC}|pw}}{ }29. 4. 88.\pend
           
\pstart{}Hochgeehrter Herr!\pend\vspace{0.5em}
\pstart
           Ich nehme mir die Freiheit, Ihnen zwei \label{K_L03618-1v}\edtext{Erzählungen}{\lemma{\textnormal{\emph{Erzählungen}}}\Cendnote{\textnormal{Von den erhaltenen Prosaarbeiten, die in diesem Zeitraum
                     entstanden, kommen \emph{Erbschaft}\pwindex{Erbschaft@\emph{Erbschaft}|pwk}, \emph{Mein Freund Ypsilon. Aus den Papieren eines
                        Arztes}\pwindex{Mein Freund Ypsilon. Aus den Papieren eines Arztes@\emph{Mein Freund Ypsilon. Aus den Papieren eines Arztes}|pwk} und \emph{Amerika}\pwindex{Amerika@\emph{Amerika}|pwk} in Frage, vgl. A. S.: \emph{Tagebuch}, 19. 10. 1887 und \emph{Jugend in Wien}\pwindex{Jugend in Wien@\emph{Jugend in Wien}|pwk} (Arthur Schnitzler: \emph{Jugend in Wien. Eine Autobiographie}\pwindex{Jugend in Wien@\emph{Jugend in Wien}|pwk}. Mit einem
                        Nachwort von Friedrich Torberg. Wien,
                        München, Zürich,
                        New York: \emph{S. Fischer}{ }1968, S. 320).}}}\label{K_L03618-1}\pwindex{Amerika@\emph{Amerika}|pwv}\pwindex{Mein Freund Ypsilon. Aus den Papieren eines Arztes@\emph{Mein Freund Ypsilon. Aus den Papieren eines Arztes}|pwv}\pwindex{Erbschaft@\emph{Erbschaft}|pwv} zu überſenden, von denen ich mir ſelbst kaum einbilden will, daſs ſie für Ihre
                  »\textsc{Dtsch. Dichtung\pwindex{Deutsche Dichtung@\emph{Deutsche Dichtung}|pw}}« der Vorzüge genug beſitzen. Jedenfalls aber wäre mir ein Urtheil von Ihnen
               höchſt erwünſcht, um das Sie hiemit zwar unbeſcheiden aber herzlichſt gebeten ſind.
               Ich unterlieſs es, \label{K_L03618-2v}\edtext{perſönlich}{\lemma{\textnormal{\emph{perſönlich}}}\Cendnote{\textnormal{Am 15. 4. 1888 und am 28. 4. 1888 war Schnitzler bei Franzos\pwindex{Franzos, Karl Emil 25.10.1848 – 28.01.1904@\textsc{Franzos, Karl Emil} (25.10.1848 – 28.01.1904), \emph{Schriftsteller/Schriftstellerin, Journalist/Journalistin}|pwk} auf Besuch in der Kaiserin-Augusta-Straße 71\oindex{Kaiserin-Augusta-Strasse 71@\textbf{Kaiserin-Augusta-Straße 71}, \emph{Wohngebäude (K.WHS)}|pwk}. Die Einladung zum Souper am Vortag dieses
                  Briefes dürfte eine Folge des Empfehlungsschreibens von Johann Schnitzler\pwindex{Schnitzler, Johann 10.04.1835 – 02.05.1893@\textsc{Schnitzler, Johann} (10.04.1835 – 02.05.1893), \emph{Laryngologe/Laryngologin}|pwk} (Johann Schnitzler an Karl Emil Franzos, 4. 4. 1888) gewesen sein.}}}\label{K_L03618-2} mit Ihnen über dieſe Sache zu reden, da
               ich in dem Augenblicke dieser {\pb}Bitte am liebſten ein
               ganz und gar unbeka{\geminationn}ter, gewiſs aber nicht der gut
               empfohlene und ſo liebenswürdig aufgeno{\geminationm}ene »Sohn meines
                  Vaters\pwindex{Schnitzler, Johann 10.04.1835 – 02.05.1893@\textsc{Schnitzler, Johann} (10.04.1835 – 02.05.1893), \emph{Laryngologe/Laryngologin}|pwv}« ſein möchte.\pend
           
\pstart
           Mit beſondrer Hochachtung Ihr{\\[\baselineskip]}ergebener{\\[\baselineskip]}\spacefill\mbox{Dr Arthur Schnitzler}\pend
           \leftskip=0em{}\selectlanguage{ngerman}\endnumbering\briefempfaengerindex{Franzos, Karl Emil@\textsc{Franzos, Karl Emil}!zzzSchnitzler, Arthur@\emph{von Arthur Schnitzler}!1888-04-291@{29. 4. 1888}|)be}\mylabel{L03618h}  \normalsize

\doendnotes{C}
\bigskip
\vfill

\clearpage

\footnotesize

\lohead{\textsc{register}}

% Definiere theindex-Environment komplett neu ohne reledmac
\makeatletter
\renewenvironment{theindex}{%
  \section*{\indexname}%
  \setlength{\parindent}{0pt}%
  \setlength{\parskip}{0pt plus 0.3pt}%
  \let\item\@idxitem
}{%
  \clearpage
}
\makeatother

\IfFileExists{\jobname-pw.ind}{\input{\jobname-pw.ind}}{}

\end{document}

      