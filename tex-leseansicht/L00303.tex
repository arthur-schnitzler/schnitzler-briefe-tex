%% latex-leseansicht-vorspann.tex
%% Vorspann für die Leseansicht.
%% Lädt die gemeinsame Datei latex-vorspann.tex mit nicht gesetztem Schalter.

\newif\ifkorrekturansicht
\korrekturansichtfalse

\input{../tex-inputs/latex-vorspann}


         
         \renewcommand{\erwaehntePersonen}{Personen: Richard Beer-Hofmann}
         \renewcommand{\erwaehnteOrte}{Orte: Berlin, Hotel Westminster, IX., Alsergrund, Wien}
         \renewcommand{\erwaehnteWerke}{Werke: Anatol}
               \section[Arthur Schnitzler an Richard Beer-Hofmann, 2. 3. 1894]{ Arthur Schnitzler an Richard Beer-Hofmann, 2. 3. 1894}\nopagebreak\mylabel{v}\rehead{ }\begin{ledgroupsized}[t]{13cm}\normalsize\beginnumbering\briefempfaengerindex{Beer-Hofmann, Richard@\textsc{Beer-Hofmann, Richard}!zzzSchnitzler, Arthur@\emph{von Arthur Schnitzler}!1894-03-022@{2. 3. 1894}|(be} \toendnotes[C]{\smallbreak\pagebreak[2]} \Standort{YCGL, MSS 31.}
\physDesc{Postkarte, 319 Zeichen
\newline{}Handschrift: 1) Bleistift, deutsche Kurrent\hspace{1em}2) Bleistift, lateinische Kurrent (\noindent{}Adresse)\hspace{1em}
\newline{}Versand: 1) Stempel: »\nobreak{}\oindex{IX., Alsergrund@\textbf{IX., Alsergrund}|pwk}Wien 9/3, 2. 3. 94, 3–4 N\nobreak{}«.   2) Stempel: »\nobreak{}\oindex{Berlin@\textbf{Berlin}|pwk}Berlin, 3|8. 94, 3–3½N, Bestellt vom Postamte 64\nobreak{}«. }\pstart{}{\pb}Herrn Dr. Richard Beer Hofmann\pend{}\pstart{}Berlin\oindex{Berlin@\textbf{Berlin}|pw}\pend{}\pstart{}Hotel Westminster\oindex{Hotel Westminster@\textbf{Hotel Westminster}|pw}\pend{}{\bigskip}\pstart
           \noindent{}{\pb}Lieber Richard, ſollten Sie Anatol\pwindex{Schnitzler, Arthur 15.05.1862 – 21.10.1931@\textsc{Schnitzler, Arthur} (15.05.1862 – 21.10.1931), \emph{Schriftsteller, Mediziner}!Anatol1892-10-29@\strich\emph{Anatol} {[}1892-10-29{]}|pw} brauchen, ſo kaufen Sie gef. auf meine
               Koſten ein Exemplar; ich müßte das gebundene, das ich habe, als Paket aufgeben, was
               Umſtände macht. Auch ka{\geminationn} ich das ungebundene ſehr gut
               brauchen. Schade, daſs Sie nicht ſchreiben.\pend
           \pstart Herzl Ihr\spacefill\mbox{Arthur}\pend{}
         
         \endnumbering\mylabel{h}\end{ledgroupsized}  \newcommand{\dateiname}{L00303}\newcommand{\titel}{Arthur Schnitzler an Richard Beer-Hofmann, 2. 3. 1894}\newcommand{\editorInnen}{Martin Anton Müller und Gerd-Hermann Susen}%% latex-leseansicht-abspann.tex
%% Abspann für die Leseansicht.
%% Der Schalter \ifkorrekturansicht ist bereits durch den Vorspann gesetzt.

%% latex-abspann.tex
%% Gemeinsamer Abspann für Korrekturansicht und Leseansicht.
%% Setzt den Schalter \ifkorrekturansicht voraus (gesetzt in den
%% einbindenden Dateien latex-korrekturansicht-abspann.tex bzw.
%% latex-leseansicht-abspann.tex).
%% ---------------------------------------------------------------

\normalsize

% Das esempio-Environment wird nur in der Leseansicht benötigt
\ifkorrekturansicht\else
\newenvironment{esempio}[3]%
{
    \vspace{1.5ex}
    \rlap{\underline{#1}}
    \par
    \setlength{\parindent}{0cm}
    \nopagebreak
    \leftskip=#2cm
    \rightskip=#3cm
}
{
    \par
}
\fi

\doendnotes{C}
\bigskip
\vfill

\clearpage

\footnotesize

\ifkorrekturansicht
  \lohead{\textsc{register}}
\fi

% theindex-Environment neu definieren ohne reledmac
\makeatletter
\renewenvironment{theindex}{%
  \ifkorrekturansicht
    \section*{\indexname}%
  \else
    \subsubsection*{Index der erwähnten Entitäten}%
  \fi
  \setlength{\parindent}{0pt}%
  \setlength{\parskip}{0pt plus 0.3pt}%
  \let\item\@idxitem
}{%
  \ifkorrekturansicht\clearpage\fi
}
\makeatother

\IfFileExists{\jobname-pw.ind}{\input{\jobname-pw.ind}}{}

% Quellenangabe nur in der Leseansicht
\ifkorrekturansicht\else
% Fallback-Definitionen, falls die .tex-Datei \titel etc. nicht gesetzt hat
\providecommand{\titel}{}
\providecommand{\editorInnen}{}
\providecommand{\dateiname}{\jobname}

\vspace{3cm}

\vfill

\footnotesize
\textsc{Quelle}: \titel. Herausgegeben von {\editorInnen}. In: \emph{Arthur Schnitzler: Briefwechsel mit Autorinnen und Autoren}.
 Digitale Edition, https://schnitzler-briefe.acdh.oeaw.ac.at/{\dateiname}.html (Stand \today)
\fi

\end{document}


      