%% latex-leseansicht-vorspann.tex
%% Vorspann für die Leseansicht.
%% Lädt die gemeinsame Datei latex-vorspann.tex mit nicht gesetztem Schalter.

\newif\ifkorrekturansicht
\korrekturansichtfalse

\input{../tex-inputs/latex-vorspann}


\section[Arthur Schnitzler an Stefan Zweig, 25. 6. 1923]{L03749 Arthur Schnitzler an Stefan Zweig, 25. 6. 1923}
\nopagebreak\mylabel{L03749v}
\rehead{ }\normalsize\beginnumbering\briefempfaengerindex{Zweig, Stefan@\textsc{Zweig, Stefan}!zzzSchnitzler, Arthur@\emph{von Arthur Schnitzler}!1923-06-251@{25. 6. 1923}|(be}
\toendnotes[C]{\smallbreak\pagebreak[2]}
\correspDesc{Versand  durch Arthur Schnitzler am 25. 6. 1923 in Wien
\newline{}Übermittlung  durch Arthur Schnitzler am 26. 6. 1923 in Wien
\newline{}Erhalt  durch Stefan Zweig im Zeitraum [27. 6. 1923 – 1. 7. 1923?] in Salzburg}\toendnotes[C]{\smallbreak}
\Standort{Jerusalem, National Library of Israel, ARC. Ms. Var. 305 1 58 Stefan Zweig Collection.}
\physDesc{Postkarte, 721 Zeichen
\newline{}Handschrift: Bleistift, lateinische Kurrent
\newline{}Versand: Stempel: »\nobreak{}\oindex{IX., Alsergrund@\textbf{IX., Alsergrund}, \emph{Verwaltungsgebiet}|pwk}9 Wien 72, 26. VI. 23, 16\nobreak{}«.  
\newline{}Zweig: mit schwarzer Tinte Vermerk: »\textsc{beantw.}« }\toendnotes[C]{\smallbreak}\pstart{}{\pb}\label{T_L03749-1v}\edtext{\textcolor{gray}{\textbf{A. S.}}}{\lemma{\textnormal{\emph{A. S.}}}\Cendnote{\textnormal{ovaler Absenderkleber}}}\label{T_L03749-1}\pend{}\pstart{}\textcolor{gray}{\textbf{WIEN, XVIII.}}\oindex{XVIII., Währing@\textbf{XVIII., Währing}, \emph{Verwaltungsgebiet}|pw}\pend{}\pstart{}\textcolor{gray}{\textbf{STERNWARTESTR. 71}}\oindex{Wien@\textbf{Wien}!XVIII., Währing@\textbf{XVIII., Währing}!Sternwartestraße 71@\textbf{Sternwartestraße 71}, \emph{Wohngebäude}|pw}\pend{}{\bigskip}\pstart{}Hrn\pend{}\pstart{}Dr Stefan Zweig\pend{}\pstart{}Salzburg\oindex{Salzburg@\textbf{Salzburg}, \emph{Verwaltungsgebiet}|pw}\pend{}\pstart{}Kapuzinerberg 5\oindex{Paschinger Schlössl@\textbf{Paschinger Schlössl}, \emph{Wohngebäude}|pw}\pend{}{\bigskip}\vspace{1em}
\pstart
           \raggedleft{}{\pb}Wien\oindex{Wien@\textbf{Wien}, \emph{Verwaltungsgebiet}|pw}, 25. 6. 23\pend
           \vspace{0.5em}
\pstart
           lieber Herr Doctor Zweig, das »Gänsemännchen\pwindex{\textcolor{red}{\textsuperscript{XXXX indx1}}!Gänsemännchen. Roman@\strich\emph{Das Gänsemännchen. Roman}|pw}« auf dessen Erscheinen im Antiqu. Catalog Hirsch\orgindex{Antiquariat Emil Hirsch@Antiquariat Emil Hirsch|pw} Sie mich liebenswürdiger Weise aufmerksam gemacht
               haben, beko{\geminationm} ich zurück. Ein Bekannter\pwindex{Krell, Max 24.\,9.\,1887 Hubertusburg – 11.\,6.\,1962 Florenz@\textsc{Krell, Max} (24.\,9.\,1887 Hubertusburg – 11.\,6.\,1962 Florenz), \emph{Schriftsteller, Verlagslektor}|pwv} meiner Schwägerin Steinrück\pwindex{Steinrück, Elisabeth 19.\,11.\,1885 – 7.\,4.\,1920 Partenkirchen@\textsc{Steinrück, Elisabeth} (19.\,11.\,1885 – 7.\,4.\,1920 Partenkirchen)|pw},
               dem sie das Exempl. angeblich vermacht hatte, hat es zur Versteigerung dem Hirsch\pwindex{Hirsch, Emil 14.\,3.\,1866 Bad Mergentheim – 27.\,7.\,1954 New York City@\textsc{Hirsch, Emil} (14.\,3.\,1866 Bad Mergentheim – 27.\,7.\,1954 New York City), \emph{Verleger, Antiquar}|pw} überlassen.\pend
           
\pstart
           – Eigentlich aber
               schreib ich Ihnen um Ihnen zu sagen, wie sehr mich Ihr wunderschöner \label{K_L03749-1v}\edtext{Artikel\pwindex{Zweig, Stefan 28.\,11.\,1881 Wien – 23.\,2.\,1942 Petrópolis@\textsc{Zweig, Stefan} (28.\,11.\,1881 Wien – 23.\,2.\,1942 Petrópolis), \emph{Schriftsteller}!Zum Andenken Walter Rathenaus. Am Jahrestage seiner Ermordung, 24. Juni 1922@\strich\emph{Zum Andenken Walter Rathenaus. Am Jahrestage seiner Ermordung, 24. Juni 1922}|pwv} über Rathenau\pwindex{Rathenau, Walther 29.\,9.\,1867 Berlin – 24.\,6.\,1922 ebd.@\textsc{Rathenau, Walther} (29.\,9.\,1867 Berlin – 24.\,6.\,1922 ebd.), \emph{Politiker, Industrieller}|pw}}{\lemma{\textnormal{\emph{Artikel über Rathenau}}}\Cendnote{\textnormal{Stefan Zweig\pwindex{Zweig, Stefan 28.\,11.\,1881 Wien – 23.\,2.\,1942 Petrópolis@\textsc{Zweig, Stefan} (28.\,11.\,1881 Wien – 23.\,2.\,1942 Petrópolis), \emph{Schriftsteller}|pwk}: \emph{Zum
                        Andenken Walter Rathenaus. Am Jahrestage seiner Ermordung, 24. Juni
                        1922}\pwindex{Zweig, Stefan 28.\,11.\,1881 Wien – 23.\,2.\,1942 Petrópolis@\textsc{Zweig, Stefan} (28.\,11.\,1881 Wien – 23.\,2.\,1942 Petrópolis), \emph{Schriftsteller}!Zum Andenken Walter Rathenaus. Am Jahrestage seiner Ermordung, 24. Juni 1922@\strich\emph{Zum Andenken Walter Rathenaus. Am Jahrestage seiner Ermordung, 24. Juni 1922}|pwk}. In: \emph{Neue Freie Presse}\pwindex{Neue Freie Presse@\emph{Neue Freie Presse}|pwk},
                     Nr. 21.116, 24. 6. 1923, Morgenblatt,
                     S. 1–3.}}}\label{K_L03749-1} ergriffen hat; als essayistisches Meisterstück und
               als menschliches Document. Ich habe R\pwindex{Rathenau, Walther 29.\,9.\,1867 Berlin – 24.\,6.\,1922 ebd.@\textsc{Rathenau, Walther} (29.\,9.\,1867 Berlin – 24.\,6.\,1922 ebd.), \emph{Politiker, Industrieller}|pw} nicht
               gekannt, aber nie ist mir seine Persönlichkeit so einleuchtend geworden, als aus
               Ihrer  Gestaltung.\pend
           \pstart Seien Sie bedankt und gegrüßt! Herzlichst Ihr sehr ergebner \spacefill\mbox{ArthSchnitzler}\pend{}\selectlanguage{ngerman}\endnumbering\briefempfaengerindex{Zweig, Stefan@\textsc{Zweig, Stefan}!zzzSchnitzler, Arthur@\emph{von Arthur Schnitzler}!1923-06-251@{25. 6. 1923}|)be}\mylabel{L03749h}  \newcommand{\dateiname}{L03749}\newcommand{\titel}{Arthur Schnitzler an Stefan Zweig, 25. 6. 1923}\newcommand{\editorInnen}{Selma Jahnke und Martin Anton Müller}%% latex-leseansicht-abspann.tex
%% Abspann für die Leseansicht.
%% Der Schalter \ifkorrekturansicht ist bereits durch den Vorspann gesetzt.

%% latex-abspann.tex
%% Gemeinsamer Abspann für Korrekturansicht und Leseansicht.
%% Setzt den Schalter \ifkorrekturansicht voraus (gesetzt in den
%% einbindenden Dateien latex-korrekturansicht-abspann.tex bzw.
%% latex-leseansicht-abspann.tex).
%% ---------------------------------------------------------------

\normalsize

% Das esempio-Environment wird nur in der Leseansicht benötigt
\ifkorrekturansicht\else
\newenvironment{esempio}[3]%
{
    \vspace{1.5ex}
    \rlap{\underline{#1}}
    \par
    \setlength{\parindent}{0cm}
    \nopagebreak
    \leftskip=#2cm
    \rightskip=#3cm
}
{
    \par
}
\fi

\doendnotes{C}
\bigskip
\vfill

\clearpage

\footnotesize

\ifkorrekturansicht
  \lohead{\textsc{register}}
\fi

% theindex-Environment neu definieren ohne reledmac
\makeatletter
\renewenvironment{theindex}{%
  \ifkorrekturansicht
    \section*{\indexname}%
  \else
    \subsubsection*{Index der erwähnten Entitäten}%
  \fi
  \setlength{\parindent}{0pt}%
  \setlength{\parskip}{0pt plus 0.3pt}%
  \let\item\@idxitem
}{%
  \ifkorrekturansicht\clearpage\fi
}
\makeatother

\IfFileExists{\jobname-pw.ind}{\input{\jobname-pw.ind}}{}

% Quellenangabe nur in der Leseansicht
\ifkorrekturansicht\else
% Fallback-Definitionen, falls die .tex-Datei \titel etc. nicht gesetzt hat
\providecommand{\titel}{}
\providecommand{\editorInnen}{}
\providecommand{\dateiname}{\jobname}

\vspace{3cm}

\vfill

\footnotesize
\textsc{Quelle}: \titel. Herausgegeben von {\editorInnen}. In: \emph{Arthur Schnitzler: Briefwechsel mit Autorinnen und Autoren}.
 Digitale Edition, https://schnitzler-briefe.acdh.oeaw.ac.at/{\dateiname}.html (Stand \today)
\fi

\end{document}


