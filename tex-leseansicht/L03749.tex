%% latex-korrekturansicht-vorspann.tex
%% Vorspann für die Korrekturansicht.
%% Lädt die gemeinsame Datei latex-vorspann.tex mit gesetztem Schalter.

\newif\ifkorrekturansicht
\korrekturansichttrue

\input{../tex-inputs/latex-vorspann}


\section[Arthur Schnitzler an Stefan Zweig, 25. 6. 1923]{L03749 Arthur Schnitzler an Stefan Zweig, 25. 6. 1923}
\nopagebreak\mylabel{L03749v}
\rehead{ }\normalsize\beginnumbering\briefempfaengerindex{Zweig, Stefan@\textsc{Zweig, Stefan}!zzzSchnitzler, Arthur@\emph{von Arthur Schnitzler}!1923-06-251@{25. 6. 1923}|(be}
\toendnotes[C]{\smallbreak\pagebreak[2]}\Standort{Jerusalem, National Library of Israel, ARC. Ms. Var. 305 1 58 Stefan Zweig Collection.}
\physDesc{Postkarte, 1 Blatt, 2 Seiten, 722 Zeichen
\newline{}Handschrift: Bleistift, lateinische Kurrent
\newline{}Versand: Stempel: »\nobreak{}\oindex{IX., Alsergrund@\textbf{IX., Alsergrund}, \emph{A.ADM3}|pwk}9 Wien 72, 26. VI. 23, 16\nobreak{}«.  
\newline{}Zweig: mit schwarzer Tinte Vermerk: »\textsc{beantw.}« }\toendnotes[C]{\smallbreak}\pstart{}{\pb}\label{T_L03749-1v}\edtext{\textcolor{gray}{\textbf{A. S.}}}{\lemma{\textnormal{\emph{A. S.}}}\Cendnote{\textnormal{ovaler Absenderkleber}}}\label{T_L03749-1}\pend{}\pstart{}\textcolor{gray}{\textbf{WIEN, XVIII.}}\oindex{XVIII., Waehring@\textbf{XVIII., Währing}, \emph{A.ADM3}|pw}\pend{}\pstart{}\textcolor{gray}{\textbf{STERNWARTESTR. 71}}\oindex{Sternwartestrasse 71@\textbf{Sternwartestraße 71}, \emph{Wohngebäude (K.WHS)}|pw}\pend{}{\bigskip}\pstart{}Hrn\pend{}\pstart{}Dr Stefan Zweig\pend{}\pstart{}Salzburg\oindex{Salzburg@\textbf{Salzburg}, \emph{A.ADM2}|pw}\pend{}\pstart{}Kapuzinerberg 5\oindex{Paschinger Schloessl@\textbf{Paschinger Schlössl}, \emph{Wohngebäude (K.WHS)}|pw}\pend{}{\bigskip}\vspace{1em}
\pstart
           \raggedleft{}{\pb}Wien\oindex{Wien@\textbf{Wien}, \emph{A.ADM2}|pw}, 25. 6. 23\pend
           \vspace{0.5em}
\pstart
           lieber Herr Doctor Zweig, das »Gänsemännchen\pwindex{Gaensemaennchen. Roman@\emph{Das Gänsemännchen. Roman}|pw}« auf dessen Erscheinen im Antiqu. Catalog Hirsch\orgindex{Antiquariat Emil Hirsch@Antiquariat Emil Hirsch|pw} Sie mich liebenswürdiger Weise aufmerksam gemacht
               haben, beko{\geminationm} ich zurück. Ein Bekannter\pwindex{Krell, Max 24.09.1887 – 11.06.1962@\textsc{Krell, Max} (24.09.1887 – 11.06.1962), \emph{Schriftsteller/Schriftstellerin, Verlagslektor/Verlagslektorin}|pwv} meiner Schwägerin Steinrück\pwindex{Steinrueck, Elisabeth 19.11.1885 – 07.04.1920@\textsc{Steinrück, Elisabeth} (19.11.1885 – 07.04.1920)|pw},
               dem sie das Exempl. angeblich vermacht hatte, hat es zur Versteigerung dem Hirsch\pwindex{Hirsch, Emil 1866-03-14 – 1954-07-27@\textsc{Hirsch, Emil} (1866-03-14 – 1954-07-27), \emph{Verleger/Verlegerin, Antiquar/Antiquarin}|pw} überlassen. – Eigentlich aber
               schreib ich Ihnen um Ihnen zu sagen, wie sehr mich Ihr wunderschöner \label{K_L03749-1v}\edtext{Artikel\pwindex{Zum Andenken Walter Rathenaus. Am Jahrestage seiner Ermordung, 24. Juni 1922@\emph{Zum Andenken Walter Rathenaus. Am Jahrestage seiner Ermordung, 24. Juni 1922}|pwv} über Rathenau\pwindex{Rathenau, Walther 29.09.1867 – 24.06.1922@\textsc{Rathenau, Walther} (29.09.1867 – 24.06.1922), \emph{Politiker/Politikerin, Industrieller/Industrielle}|pw}}{\lemma{\textnormal{\emph{Artikel über Rathenau}}}\Cendnote{\textnormal{Stefan Zweig\pwindex{Zweig, Stefan 28.11.1881 – 23.02.1942@\textsc{Zweig, Stefan} (28.11.1881 – 23.02.1942), \emph{Schriftsteller/Schriftstellerin}|pwk}: \emph{Zum
                        Andenken Walter Rathenaus. Am Jahrestage seiner Ermordung, 24. Juni
                        1922}\pwindex{Zum Andenken Walter Rathenaus. Am Jahrestage seiner Ermordung, 24. Juni 1922@\emph{Zum Andenken Walter Rathenaus. Am Jahrestage seiner Ermordung, 24. Juni 1922}|pwk}. In: \emph{Neue Freie Presse}\pwindex{Neue Freie Presse@\emph{Neue Freie Presse}|pwk},
                     Nr. 21.116, 24. 6. 1923, Morgenblatt,
                     S. 1–3.}}}\label{K_L03749-1} ergriffen hat; als essayistisches Meisterstück und
               als menschliches Document. Ich habe R\pwindex{Rathenau, Walther 29.09.1867 – 24.06.1922@\textsc{Rathenau, Walther} (29.09.1867 – 24.06.1922), \emph{Politiker/Politikerin, Industrieller/Industrielle}|pw} nicht
               gekannt, aber nie ist mir seine Persönlichkeit so einleuchtend geworden, als aus
               Ihrer  Gestaltung. Seien Sie bedankt und gegrüßt!\pend
           \pstart Herzlichst Ihr sehr ergebner \spacefill\mbox{ArthSchnitzler}\pend{}\selectlanguage{ngerman}\endnumbering\briefempfaengerindex{Zweig, Stefan@\textsc{Zweig, Stefan}!zzzSchnitzler, Arthur@\emph{von Arthur Schnitzler}!1923-06-251@{25. 6. 1923}|)be}\mylabel{L03749h}
\begin{anhang}
\end{anhang}\normalsize

\doendnotes{C}
\bigskip
\vfill

\clearpage

\footnotesize

\lohead{\textsc{register}}

% Definiere theindex-Environment komplett neu ohne reledmac
\makeatletter
\renewenvironment{theindex}{%
  \section*{\indexname}%
  \setlength{\parindent}{0pt}%
  \setlength{\parskip}{0pt plus 0.3pt}%
  \let\item\@idxitem
}{%
  \clearpage
}
\makeatother

\IfFileExists{\jobname-pw.ind}{\input{\jobname-pw.ind}}{}

\end{document}

      