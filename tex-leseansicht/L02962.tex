%% latex-leseansicht-vorspann.tex
%% Vorspann für die Leseansicht.
%% Lädt die gemeinsame Datei latex-vorspann.tex mit nicht gesetztem Schalter.

\newif\ifkorrekturansicht
\korrekturansichtfalse

\input{../tex-inputs/latex-vorspann}


\section[Arthur Schnitzler an Felix Salten, {{[}}25. 9. 1893?{{]}}]{L02962 Arthur Schnitzler an Felix Salten, {[}25. 9. 1893?{]}}
\nopagebreak\mylabel{L02962v}
\rehead{ }\normalsize\beginnumbering\briefempfaengerindex{Salten, Felix@\textsc{Salten, Felix}!zzzSchnitzler, Arthur@\emph{von Arthur Schnitzler}!1893-09-251@{{[}25. 9. 1893?{]}}|(be}
\toendnotes[C]{\smallbreak\pagebreak[2]}
\correspDesc{Versand  durch Arthur Schnitzler am [25. 9. 1893?] in Wien
\newline{}Erhalt  durch Felix Salten im Zeitraum [25. 9. 1893 –
                  26. 9. 1893?] in Wien}\toendnotes[C]{\smallbreak}
\Standort{Wienbibliothek im Rathaus, ZPH 1681, 2.1.516.}
\physDesc{Brief, 1 Blatt, 3 Seiten, 422 Zeichen (Briefpapier mit Trauerrand)
\newline{}Handschrift: Bleistift, deutsche Kurrent
\newline{}Ordnung: mit Bleistift von unbekannter Hand die erste und dritte Seite paginiert:
                                    »13«–»14« }\toendnotes[C]{\smallbreak}
\pstart{}{\pb}Hochverehrter Herr von Salten!\pend\vspace{0.5em}
\pstart
           \label{K_L02962-1v}\edtext{Morgen Dinſtag}{\lemma{\textnormal{\emph{Morgen Dinstag}}}\Cendnote{\textnormal{Siehe A. S.: \emph{Tagebuch}, 26. 9. 1893.
               }}}\label{K_L02962-1}{ }Nachmittag 4 Uhr ko{\geminationm}en \textsc{Loris\pwindex{Hofmannsthal, Hugo von 1.\,2.\,1874 Wien – 15.\,7.\,1929 Rodaun@\textsc{Hofmannsthal, Hugo von} (1.\,2.\,1874 Wien – 15.\,7.\,1929 Rodaun), \emph{Schriftsteller}|pw}} u. Richard\pwindex{Beer-Hofmann, Richard 11.\,7.\,1866 Wien – 26.\,9.\,1945 New York City@\textsc{Beer-Hofmann, Richard} (11.\,7.\,1866 Wien – 26.\,9.\,1945 New York City), \emph{Schriftsteller}|pw} zu mir, und außerdem Herr \textsc{Richard Mandl\pwindex{Mandl, Richard 9.\,5.\,1859 Prostějov – 31.\,3.\,1918 Wien@\textsc{Mandl, Richard} (9.\,5.\,1859 Prostějov – 31.\,3.\,1918 Wien), \emph{Komponist}|pw}}, (Componiſt, Paris\oindex{Paris@\textbf{Paris}, \emph{Hauptstadt}|pw}) {\pb}der uns auf dem Piano artige Dinge zu{ }ſpielen
               gedenkt, welches ich Ihnen mittheile, um Sie zu bewegen, mir gleichfalls die Ehre
               Ihres Beſuches zu{ }ſchenken, der mir denn {\pb}ſicherlich höflich willkom\textcolor{gray}{m}en{ }ſein wird.\pend
           
\pstart
           Leben Sie wohl und{ }ſagen mir bald gute Nachricht von Ihrem \label{K_L02962-2v}\edtext{Roman\pwindex{Salten, Felix 6.\,9.\,1869 Budapest – 8.\,10.\,1945 Zürich@\textsc{Salten, Felix} (6.\,9.\,1869 Budapest – 8.\,10.\,1945 Zürich), \emph{Schriftsteller, Journalist, Chefredakteur}!?? [Romanprojekt]@\strich\emph{?? [Romanprojekt]}|pwv}}{\lemma{\textnormal{\emph{Roman}}}\Cendnote{\textnormal{Von Salten\pwindex{Salten, Felix 6.\,9.\,1869 Budapest – 8.\,10.\,1945 Zürich@\textsc{Salten, Felix} (6.\,9.\,1869 Budapest – 8.\,10.\,1945 Zürich), \emph{Schriftsteller, Journalist, Chefredakteur}|pwk} erschien
               in diesen Jahren keine Romanveröffentlichung.}}}\label{K_L02962-2}.\pend
           \pstart Ihr \spacefill\mbox{ArthS}\pend{}
\pstart
           Montag.\pend
           \selectlanguage{ngerman}\endnumbering\briefempfaengerindex{Salten, Felix@\textsc{Salten, Felix}!zzzSchnitzler, Arthur@\emph{von Arthur Schnitzler}!1893-09-251@{{[}25. 9. 1893?{]}}|)be}\mylabel{L02962h}  \newcommand{\dateiname}{L02962}\newcommand{\titel}{Arthur Schnitzler an Felix Salten, [25. 9. 1893?]}\newcommand{\editorInnen}{Martin Anton Müller und Laura Untner}%% latex-leseansicht-abspann.tex
%% Abspann für die Leseansicht.
%% Der Schalter \ifkorrekturansicht ist bereits durch den Vorspann gesetzt.

%% latex-abspann.tex
%% Gemeinsamer Abspann für Korrekturansicht und Leseansicht.
%% Setzt den Schalter \ifkorrekturansicht voraus (gesetzt in den
%% einbindenden Dateien latex-korrekturansicht-abspann.tex bzw.
%% latex-leseansicht-abspann.tex).
%% ---------------------------------------------------------------

\normalsize

% Das esempio-Environment wird nur in der Leseansicht benötigt
\ifkorrekturansicht\else
\newenvironment{esempio}[3]%
{
    \vspace{1.5ex}
    \rlap{\underline{#1}}
    \par
    \setlength{\parindent}{0cm}
    \nopagebreak
    \leftskip=#2cm
    \rightskip=#3cm
}
{
    \par
}
\fi

\doendnotes{C}
\bigskip
\vfill

\clearpage

\footnotesize

\ifkorrekturansicht
  \lohead{\textsc{register}}
\fi

% theindex-Environment neu definieren ohne reledmac
\makeatletter
\renewenvironment{theindex}{%
  \ifkorrekturansicht
    \section*{\indexname}%
  \else
    \subsubsection*{Index der erwähnten Entitäten}%
  \fi
  \setlength{\parindent}{0pt}%
  \setlength{\parskip}{0pt plus 0.3pt}%
  \let\item\@idxitem
}{%
  \ifkorrekturansicht\clearpage\fi
}
\makeatother

\IfFileExists{\jobname-pw.ind}{\input{\jobname-pw.ind}}{}

% Quellenangabe nur in der Leseansicht
\ifkorrekturansicht\else
% Fallback-Definitionen, falls die .tex-Datei \titel etc. nicht gesetzt hat
\providecommand{\titel}{}
\providecommand{\editorInnen}{}
\providecommand{\dateiname}{\jobname}

\vspace{3cm}

\vfill

\footnotesize
\textsc{Quelle}: \titel. Herausgegeben von {\editorInnen}. In: \emph{Arthur Schnitzler: Briefwechsel mit Autorinnen und Autoren}.
 Digitale Edition, https://schnitzler-briefe.acdh.oeaw.ac.at/{\dateiname}.html (Stand \today)
\fi

\end{document}


