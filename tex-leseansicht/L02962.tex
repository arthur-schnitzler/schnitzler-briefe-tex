%% latex-leseansicht-vorspann.tex
%% Vorspann für die Leseansicht.
%% Lädt die gemeinsame Datei latex-vorspann.tex mit nicht gesetztem Schalter.

\newif\ifkorrekturansicht
\korrekturansichtfalse

\input{../tex-inputs/latex-vorspann}

\begin{center}
            \textcolor{red}{ENTWURF, NICHT FERTIG KORRIGIERT}
                      \end{center}
            
         
         \newcommand{\erwaehntePersonen}{Personen: Richard Beer-Hofmann, Hugo von Hofmannsthal, Richard Mandl, Felix Salten}
         \newcommand{\erwaehnteOrte}{Orte: Paris, Wien}
         \newcommand{\erwaehnteWerke}{
               \section[Arthur Schnitzler an Felix Salten, {[}26. 9. 1893?{]}]{ Arthur Schnitzler an Felix Salten, {[}26. 9. 1893?{]}}\nopagebreak\mylabel{v}\rehead{ }\begin{ledgroupsized}[t]{13cm}\normalsize\beginnumbering \toendnotes[C]{\smallbreak\pagebreak[2]} \Standort{Wienbibliothek im Rathaus, ZPH 1681, 2.1.516.}
\physDesc{
\newline{}Handschrift: , deutsche Kurrent}\toendnotes[C]{\smallbreak}\pstart
           \noindent{}{\pb}Hochverehrter Herr von
                  Salten!  Morgen Dinſtag{ }Nachmittag 4 Uhr ko{\geminationm}en \textsc{Loris\pwindex{Hofmannsthal, Hugo von 1874-02-01 – 1929-07-15@\textsc{Hofmannsthal, Hugo von} (1874-02-01 – 1929-07-15), \emph{Schriftsteller}|pw}} u. Richard\pwindex{Beer-Hofmann, Richard 1866-07-11 – 1945-09-26@\textsc{Beer-Hofmann, Richard} (1866-07-11 – 1945-09-26), \emph{Schriftsteller}|pw} zu mir, und außerdem Herr \textsc{Richard Mandl\pwindex{Mandl, Richard 1859-05-09 – 1918-03-31@\textsc{Mandl, Richard} (1859-05-09 – 1918-03-31), \emph{Komponist}|pw}}, (Componiſt, \textsc{Paris\oindex{Paris@\textbf{Paris}|pw}}) {\pb}der uns auch dem Piano
               artige Dinge zu ſpielen gedenkt, welches ich Ihnen mittheile, um Sie zu bewegen, mir
               gleichfalls die Ehre Ihres Beſuches zu ſchenken, der mir denn {\pb}ſicherlich höflich willko{\geminationm}en ſein wird. \pend
           \pstart
           Leben Sie wohl und ſagen mir bald gute Nachricht von Ihrem Roman\textcolor{red}{\textsuperscript{\textbf{KEY}}}. \pend
           \pstart Ihr \spacefill\mbox{ArthS}\pend{}\pstart
           \raggedleft{}Montag\pend
           
         
         \endnumbering\mylabel{h}\end{ledgroupsized}\begin{anhang}\end{anhang}\newcommand{\dateiname}{L02962}\newcommand{\titel}{Arthur Schnitzler an Felix Salten, [26. 9. 1893?]}\newcommand{\editorInnen}{Martin Anton Müller und Laura Untner}%% latex-leseansicht-abspann.tex
%% Abspann für die Leseansicht.
%% Der Schalter \ifkorrekturansicht ist bereits durch den Vorspann gesetzt.

%% latex-abspann.tex
%% Gemeinsamer Abspann für Korrekturansicht und Leseansicht.
%% Setzt den Schalter \ifkorrekturansicht voraus (gesetzt in den
%% einbindenden Dateien latex-korrekturansicht-abspann.tex bzw.
%% latex-leseansicht-abspann.tex).
%% ---------------------------------------------------------------

\normalsize

% Das esempio-Environment wird nur in der Leseansicht benötigt
\ifkorrekturansicht\else
\newenvironment{esempio}[3]%
{
    \vspace{1.5ex}
    \rlap{\underline{#1}}
    \par
    \setlength{\parindent}{0cm}
    \nopagebreak
    \leftskip=#2cm
    \rightskip=#3cm
}
{
    \par
}
\fi

\doendnotes{C}
\bigskip
\vfill

\clearpage

\footnotesize

\ifkorrekturansicht
  \lohead{\textsc{register}}
\fi

% theindex-Environment neu definieren ohne reledmac
\makeatletter
\renewenvironment{theindex}{%
  \ifkorrekturansicht
    \section*{\indexname}%
  \else
    \subsubsection*{Index der erwähnten Entitäten}%
  \fi
  \setlength{\parindent}{0pt}%
  \setlength{\parskip}{0pt plus 0.3pt}%
  \let\item\@idxitem
}{%
  \ifkorrekturansicht\clearpage\fi
}
\makeatother

\IfFileExists{\jobname-pw.ind}{\input{\jobname-pw.ind}}{}

% Quellenangabe nur in der Leseansicht
\ifkorrekturansicht\else
% Fallback-Definitionen, falls die .tex-Datei \titel etc. nicht gesetzt hat
\providecommand{\titel}{}
\providecommand{\editorInnen}{}
\providecommand{\dateiname}{\jobname}

\vspace{3cm}

\vfill

\footnotesize
\textsc{Quelle}: \titel. Herausgegeben von {\editorInnen}. In: \emph{Arthur Schnitzler: Briefwechsel mit Autorinnen und Autoren}.
 Digitale Edition, https://schnitzler-briefe.acdh.oeaw.ac.at/{\dateiname}.html (Stand \today)
\fi

\end{document}


      