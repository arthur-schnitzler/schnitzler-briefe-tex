%% latex-korrekturansicht-vorspann.tex
%% Vorspann für die Korrekturansicht.
%% Lädt die gemeinsame Datei latex-vorspann.tex mit gesetztem Schalter.

\newif\ifkorrekturansicht
\korrekturansichttrue

\input{../tex-inputs/latex-vorspann}


\section[Felix Salten an Arthur Schnitzler, {[}29. 6.? 1894{]}]{L03139 Felix Salten an Arthur Schnitzler, {[}29. 6.? 1894{]}}
\nopagebreak\mylabel{L03139v}
\rehead{ }\normalsize\beginnumbering\briefempfaengerindex{Schnitzler, Arthur@\textsc{Schnitzler, Arthur}!zzzSalten, Felix@\emph{von Felix Salten}!1894-06-291@{{[}29. 6.? 1894{]}}|(be}
\toendnotes[C]{\smallbreak\pagebreak[2]}\Standort{CUL, Schnitzler, B 89, A 1.}
\physDesc{Brief, 1 Blatt, 3 Seiten, 391 Zeichen
\newline{}Handschrift: Bleistift, lateinische Kurrent
\newline{}Schnitzler: mit Bleistift datiert: »2\substVorne{}\textsuperscript{8}\substDazwischen{}9\substHinten{}/\textcolor{gray}{6} 94« 
\newline{}Ordnung: mit Bleistift von unbekannter Hand nummeriert: »40« }\toendnotes[C]{\smallbreak}
\pstart
           \noindent{}{\pb}Lieber Freund! Um \label{K_L03139-1v}\edtext{¼ ½}{\lemma{\textnormal{\emph{¼ ½}}}\Cendnote{\textnormal{15 Minuten,
                  30 Minuten nach der vollen Stunde}}}\label{K_L03139-1} kann ich leider nicht \label{K_L03139-2v}\edtext{wegfahren}{\lemma{\textnormal{\emph{wegfahren}}}\Cendnote{\textnormal{vermutlich Bezug auf die gemeinsame Radtour am 1. 7. 1894}}}\label{K_L03139-2}, und um \substVorne{}\textsuperscript{½}\substDazwischen{}2\substHinten{} U.? Sie wissen ja, ich habe \label{K_L03139-3v}\edtext{keine N\textsuperscript{o},}{\lemma{\textnormal{\emph{keine N\textsuperscript{o},}}}\Cendnote{\textnormal{Siehe Felix Salten an Arthur Schnitzler, [7.? 5. 1894].
               }}}\label{K_L03139-3} wie soll ich da nach \label{T_L03139-1v}\edtext{Rodaun\oindex{Rodaun@\textbf{Rodaun}, \emph{A.ADM4}|pw}}{\lemma{\textnormal{\emph{Rodaun}}}\Cendnote{\textnormal{Er 
               schreibt »Rodaum«}}}\label{T_L03139-1} kommen.
               Ausserdem {\pb}ist es \substVorne{}\textsuperscript{kein}\substDazwischen{}nic\substHinten{}ht so schön, wenn wir nicht allein sein können.\pend
           
\pstart
           Nach Rodaun\oindex{Rodaun@\textbf{Rodaun}, \emph{A.ADM4}|pw} kann ich also wol nicht
                  fahren\textcolor{gray}{.} Ich habe mir vorgestellt, dass Sie frei sein werden u.
               dass wir um 4 Uhr abfahren, Tulln\oindex{Tulln an der Donau@\textbf{Tulln an der Donau}, \emph{A.ADM3}|pw},
               oder \textcolor{gray}{ir.} etwas. Sind Sie {\pb}Abends eventuell im \label{K_L03139-4v}\edtext{Café}{\lemma{\textnormal{\emph{Café}}}\Cendnote{\textnormal{Schnitzler hielt sich am Nachmittag
                  des 29. 6. 1894 in
                     Rodaun\oindex{Rodaun@\textbf{Rodaun}, \emph{A.ADM4}|pwk} auf. Den Abend verbrachte er mit
                  Adele Sandrock\pwindex{Sandrock, Adele 1863-08-19 – 1937-08-30@\textsc{Sandrock, Adele} (1863-08-19 – 1937-08-30), \emph{Schauspieler/Schauspielerin}|pwk}.}}}\label{K_L03139-4}?\pend
           
\pstart
           Herzlichst {\\[\baselineskip]}Ihr {\\[\baselineskip]}\spacefill\mbox{Salten}\pend
           \leftskip=0em{}\selectlanguage{ngerman}\endnumbering\briefempfaengerindex{Schnitzler, Arthur@\textsc{Schnitzler, Arthur}!zzzSalten, Felix@\emph{von Felix Salten}!1894-06-291@{{[}29. 6.? 1894{]}}|)be}\mylabel{L03139h}  \normalsize

\doendnotes{C}
\bigskip
\vfill

\clearpage

\footnotesize

\lohead{\textsc{register}}

% Definiere theindex-Environment komplett neu ohne reledmac
\makeatletter
\renewenvironment{theindex}{%
  \section*{\indexname}%
  \setlength{\parindent}{0pt}%
  \setlength{\parskip}{0pt plus 0.3pt}%
  \let\item\@idxitem
}{%
  \clearpage
}
\makeatother

\IfFileExists{\jobname-pw.ind}{\input{\jobname-pw.ind}}{}

\end{document}

      