%% latex-leseansicht-vorspann.tex
%% Vorspann für die Leseansicht.
%% Lädt die gemeinsame Datei latex-vorspann.tex mit nicht gesetztem Schalter.

\newif\ifkorrekturansicht
\korrekturansichtfalse

\input{../tex-inputs/latex-vorspann}


\section[Felix Salten an Arthur Schnitzler, {{[}}29. 6.? 1894{{]}}]{L03139 Felix Salten an Arthur Schnitzler, {[}29. 6.? 1894{]}}
\nopagebreak\mylabel{L03139v}
\rehead{ }\normalsize\beginnumbering\briefempfaengerindex{Schnitzler, Arthur@\textsc{Schnitzler, Arthur}!zzzSalten, Felix@\emph{von Felix Salten}!1894-06-291@{{[}29. 6.? 1894{]}}|(be}
\toendnotes[C]{\smallbreak\pagebreak[2]}
\correspDesc{Versand  durch Felix Salten am [29. 6.? 1894] in Wien
\newline{}Erhalt  durch Arthur Schnitzler am [29. 6.? 1894] in Wien}\toendnotes[C]{\smallbreak}
\Standort{CUL, Schnitzler, B 89, A 1.}
\physDesc{Brief, 1 Blatt, 3 Seiten, 391 Zeichen
\newline{}Handschrift: Bleistift, lateinische Kurrent
\newline{}Schnitzler: mit Bleistift datiert: »2\substVorne{}\textsuperscript{8}\substDazwischen{}9\substHinten{}/\textcolor{gray}{6} 94« 
\newline{}Ordnung: mit Bleistift von unbekannter Hand nummeriert: »40« }\toendnotes[C]{\smallbreak}
\pstart
           \noindent{}{\pb}Lieber Freund! Um \label{K_L03139-1v}\edtext{¼ ½}{\lemma{\textnormal{\emph{¼ ½}}}\Cendnote{\textnormal{15 Minuten,
                  30 Minuten nach der vollen Stunde}}}\label{K_L03139-1} kann ich leider nicht \label{K_L03139-2v}\edtext{wegfahren}{\lemma{\textnormal{\emph{wegfahren}}}\Cendnote{\textnormal{vermutlich Bezug auf die gemeinsame Radtour am 1. 7. 1894}}}\label{K_L03139-2}, und um \substVorne{}\textsuperscript{½}\substDazwischen{}2\substHinten{} U.? Sie wissen ja, ich habe \label{K_L03139-3v}\edtext{keine N\textsuperscript{o},}{\lemma{\textnormal{\emph{keine N\textsuperscript{o},}}}\Cendnote{\textnormal{Siehe XXXX Auszeichnungsfehler: Dokument L03133 nicht gefunden.
               }}}\label{K_L03139-3} wie soll ich da nach \label{T_L03139-1v}\edtext{Rodaun\oindex{Wien@\textbf{Wien}!XXIII., Liesing@\textbf{XXIII., Liesing}!Rodaun@\textbf{Rodaun}, \emph{Region}|pw}}{\lemma{\textnormal{\emph{Rodaun}}}\Cendnote{\textnormal{Er 
               schreibt »Rodaum«}}}\label{T_L03139-1} kommen.
               Ausserdem {\pb}ist es \substVorne{}\textsuperscript{kein}\substDazwischen{}nic\substHinten{}ht so schön, wenn wir nicht allein sein können.\pend
           
\pstart
           Nach Rodaun\oindex{Wien@\textbf{Wien}!XXIII., Liesing@\textbf{XXIII., Liesing}!Rodaun@\textbf{Rodaun}, \emph{Region}|pw} kann ich also wol nicht
                  fahren\textcolor{gray}{.} Ich habe mir vorgestellt, dass Sie frei sein werden u.
               dass wir um 4 Uhr abfahren, Tulln\oindex{Tulln an der Donau@\textbf{Tulln an der Donau}, \emph{Verwaltungsgebiet}|pw},
               oder \textcolor{gray}{ir.} etwas. Sind Sie {\pb}Abends eventuell im \label{K_L03139-4v}\edtext{Café}{\lemma{\textnormal{\emph{Café}}}\Cendnote{\textnormal{Schnitzler hielt sich am Nachmittag
                  des 29. 6. 1894 in
                     Rodaun\oindex{Wien@\textbf{Wien}!XXIII., Liesing@\textbf{XXIII., Liesing}!Rodaun@\textbf{Rodaun}, \emph{Region}|pwk} auf. Den Abend verbrachte er mit
                  Adele Sandrock\pwindex{Sandrock, Adele 19.\,8.\,1863 Rotterdam – 30.\,8.\,1937 Berlin@\textsc{Sandrock, Adele} (19.\,8.\,1863 Rotterdam – 30.\,8.\,1937 Berlin), \emph{Schauspielerin}|pwk}.}}}\label{K_L03139-4}?\pend
           
\pstart
           Herzlichst {\\[\baselineskip]}Ihr {\\[\baselineskip]}\spacefill\mbox{Salten}\pend
           \leftskip=0em{}\selectlanguage{ngerman}\endnumbering\briefempfaengerindex{Schnitzler, Arthur@\textsc{Schnitzler, Arthur}!zzzSalten, Felix@\emph{von Felix Salten}!1894-06-291@{{[}29. 6.? 1894{]}}|)be}\mylabel{L03139h}  \newcommand{\dateiname}{L03139}\newcommand{\titel}{Felix Salten an Arthur Schnitzler, [29. 6.? 1894]}\newcommand{\editorInnen}{Martin Anton Müller und Laura Untner}%% latex-leseansicht-abspann.tex
%% Abspann für die Leseansicht.
%% Der Schalter \ifkorrekturansicht ist bereits durch den Vorspann gesetzt.

%% latex-abspann.tex
%% Gemeinsamer Abspann für Korrekturansicht und Leseansicht.
%% Setzt den Schalter \ifkorrekturansicht voraus (gesetzt in den
%% einbindenden Dateien latex-korrekturansicht-abspann.tex bzw.
%% latex-leseansicht-abspann.tex).
%% ---------------------------------------------------------------

\normalsize

% Das esempio-Environment wird nur in der Leseansicht benötigt
\ifkorrekturansicht\else
\newenvironment{esempio}[3]%
{
    \vspace{1.5ex}
    \rlap{\underline{#1}}
    \par
    \setlength{\parindent}{0cm}
    \nopagebreak
    \leftskip=#2cm
    \rightskip=#3cm
}
{
    \par
}
\fi

\doendnotes{C}
\bigskip
\vfill

\clearpage

\footnotesize

\ifkorrekturansicht
  \lohead{\textsc{register}}
\fi

% theindex-Environment neu definieren ohne reledmac
\makeatletter
\renewenvironment{theindex}{%
  \ifkorrekturansicht
    \section*{\indexname}%
  \else
    \subsubsection*{Index der erwähnten Entitäten}%
  \fi
  \setlength{\parindent}{0pt}%
  \setlength{\parskip}{0pt plus 0.3pt}%
  \let\item\@idxitem
}{%
  \ifkorrekturansicht\clearpage\fi
}
\makeatother

\IfFileExists{\jobname-pw.ind}{\input{\jobname-pw.ind}}{}

% Quellenangabe nur in der Leseansicht
\ifkorrekturansicht\else
% Fallback-Definitionen, falls die .tex-Datei \titel etc. nicht gesetzt hat
\providecommand{\titel}{}
\providecommand{\editorInnen}{}
\providecommand{\dateiname}{\jobname}

\vspace{3cm}

\vfill

\footnotesize
\textsc{Quelle}: \titel. Herausgegeben von {\editorInnen}. In: \emph{Arthur Schnitzler: Briefwechsel mit Autorinnen und Autoren}.
 Digitale Edition, https://schnitzler-briefe.acdh.oeaw.ac.at/{\dateiname}.html (Stand \today)
\fi

\end{document}


