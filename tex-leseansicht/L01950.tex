%% latex-korrekturansicht-vorspann.tex
%% Vorspann für die Korrekturansicht.
%% Lädt die gemeinsame Datei latex-vorspann.tex mit gesetztem Schalter.

\newif\ifkorrekturansicht
\korrekturansichttrue

\input{../tex-inputs/latex-vorspann}


\section[Arthur Schnitzler an Richard Beer-Hofmann, 23. 7. 1910]{L01950 Arthur Schnitzler an Richard Beer-Hofmann, 23. 7. 1910}
\nopagebreak\mylabel{L01950v}
\rehead{ }\normalsize\beginnumbering\briefempfaengerindex{Beer-Hofmann, Richard@\textsc{Beer-Hofmann, Richard}!zzzSchnitzler, Arthur@\emph{von Arthur Schnitzler}!1910-07-231@{23. 7. 1910}|(be}
\toendnotes[C]{\smallbreak\pagebreak[2]}\Standort{YCGL, MSS 31.}
\physDesc{Brief, 1 Blatt, 4 Seiten, Umschlag, 647 Zeichen
\newline{}\noindent{}Adresse mit Schreibmaschine\noindent{}Adresse mit Schreibmaschine
\newline{}Handschrift: Bleistift, deutsche Kurrent
\newline{}Versand: Stempel: »\nobreak{}\oindex{XVIII., Waehring@\textbf{XVIII., Währing}, \emph{A.ADM3}|pwk}18/\textcolor{gray}{3} Wien
                                          1\textcolor{gray}{14}, 23. VII. 10, 3\nobreak{}«.  
\newline{}Ordnung: mit Bleistift von unbekannter Hand am Umschlag datiert: »23. 7.« }
\buchAbdrucke{\weitereDrucke{Arthur Schnitzler, Richard Beer-Hofmann: \emph{Briefwechsel 1891–1931}. Wien, Zürich: \emph{Europaverlag} 1992, S. 211–212.} }\toendnotes[C]{\smallbreak}\pstart{}{\pb}\textcolor{gray}{\textbf{Dr. Arthur Schnitzler}}\pend{}\pstart{}\textcolor{gray}{\textbf{Wien XVIII. Spoettelgasse 7\oindex{Edmund-Weiss-Gasse 7@\textbf{Edmund-Weiß-Gasse 7}, \emph{Wohngebäude (K.WHS)}|pw}.}}\pend{}{\bigskip}\pstart{}{\pb}Herrn\pend{}\pstart{}Dr. Richard Beer-Hofmann\pend{}\pstart{}\so{Ischl}\oindex{Bad Ischl@\textbf{Bad Ischl}, \emph{P.PPL}|pw}\pend{}\pstart{}Steinfeld 6\oindex{Steinfeld@\textbf{Steinfeld}, \emph{P.PPL}|pw}\pend{}{\bigskip}\vspace{1em}
\pstart
           {\pb}\textcolor{gray}{\textbf{Dr. Arthur Schnitzler}}\hfill XVIII Sternwartestr 71\oindex{Sternwartestrasse 71@\textbf{Sternwartestraße 71}, \emph{Wohngebäude (K.WHS)}|pw}\pend
           
\pstart
           \textcolor{gray}{\textbf{Wien XVIII. Spoettelgasse 7\oindex{Edmund-Weiss-Gasse 7@\textbf{Edmund-Weiß-Gasse 7}, \emph{Wohngebäude (K.WHS)}|pw}.}}\pend
           
\pstart{}mein lieber Richard,\pend\vspace{0.5em}
\pstart
           hier ſende ich Ihnen Ihr Gedicht\pwindex{Schlaflied fuer Mirjam@\emph{Schlaflied für Mirjam}|pwv}{ }ſammt Abſchrift, von der So{\geminationm}erremplacantin\pwindex{Hoffmann, Grethe @\textsc{Hoffmann, Grethe}, \emph{Schauspieler/Schauspielerin, Schreiber/Schreiberin}|pwv} der braven Frieda\pwindex{Pollak, Frieda 08.12.1881 – 13.07.1937@\textsc{Pollak, Frieda} (08.12.1881 – 13.07.1937), \emph{Sekretär/Sekretärin}|pw}. –\pend
           
\pstart
           Wir ſind leidlich in Ordnung und {\pb}freuen uns des neuen
               Heims. Ich fahre Dinſtag wieder auf ein paar Tage auf den Semmering\oindex{Semmering@\textbf{Semmering}, \emph{A.ADM3}|pw}, zu Brahm\pwindex{Brahm, Otto 05.02.1856 – 28.11.1912@\textsc{Brahm, Otto} (05.02.1856 – 28.11.1912), \emph{Theaterleiter/Theaterleiterin, Regisseur/Regisseurin}|pw} u Kainz\pwindex{Kainz, Josef 02.01.1858 – 20.09.1910@\textsc{Kainz, Josef} (02.01.1858 – 20.09.1910), \emph{Schauspieler/Schauspielerin}|pw}, der vom Hofreiter\pwindex{weite Land. Tragikomoedie in fuenf Akten@\emph{Das weite Land. Tragikomödie in fünf Akten}|pwv} ſehr angethan iſt und
               ihn {\pb}gleich ſpielen will.\pend
           
\pstart
           Erſter Beſuch in dieſem Hauſe: Baron Berger\pwindex{Berger, Alfred von 30.04.1853 – 24.08.1912@\textsc{Berger, Alfred von} (30.04.1853 – 24.08.1912), \emph{Schriftsteller/Schriftstellerin, Journalist/Journalistin, Theaterleiter/Theaterleiterin}|pw},
               aus ſolchem Grund. Aber die Sache iſt, aus mannigfachen Gründen noch nicht ganz
               ſicher. – Ins Salzka{\geminationm}er{\pb}gut\oindex{Salzkammergut@\textbf{Salzkammergut}, \emph{L.RGN}|pw}, we{\geminationn}
               alles in Ordnung hoffen wir nach 20. Auguſt zu reiſen.\pend
           
\pstart
           Ich hoffe es geht Ihnen allen ſo wie wirs wünſchen.\pend
           
\pstart
           Von Herzen Ihr{\\[\baselineskip]}\spacefill\mbox{A.}\pend
           \leftskip=0em{}\selectlanguage{ngerman}\endnumbering\briefempfaengerindex{Beer-Hofmann, Richard@\textsc{Beer-Hofmann, Richard}!zzzSchnitzler, Arthur@\emph{von Arthur Schnitzler}!1910-07-231@{23. 7. 1910}|)be}\mylabel{L01950h}  \normalsize

\doendnotes{C}
\bigskip
\vfill

\clearpage

\footnotesize

\lohead{\textsc{register}}

% Definiere theindex-Environment komplett neu ohne reledmac
\makeatletter
\renewenvironment{theindex}{%
  \section*{\indexname}%
  \setlength{\parindent}{0pt}%
  \setlength{\parskip}{0pt plus 0.3pt}%
  \let\item\@idxitem
}{%
  \clearpage
}
\makeatother

\IfFileExists{\jobname-pw.ind}{\input{\jobname-pw.ind}}{}

\end{document}

      