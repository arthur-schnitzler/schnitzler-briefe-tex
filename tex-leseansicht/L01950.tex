%% latex-leseansicht-vorspann.tex
%% Vorspann für die Leseansicht.
%% Lädt die gemeinsame Datei latex-vorspann.tex mit nicht gesetztem Schalter.

\newif\ifkorrekturansicht
\korrekturansichtfalse

\input{../tex-inputs/latex-vorspann}


         
         \renewcommand{\erwaehntePersonen}{Personen: Richard Beer-Hofmann, Alfred von Berger, Otto Brahm, Grethe Hoffmann, Josef Kainz, Frieda Pollak}
         \renewcommand{\erwaehnteOrte}{Orte: Bad Ischl, Edmund-Weiß-Gasse, Salzkammergut, Semmering, Steinfeld, Sternwartestraße, Wien, XVIII., Währing}
         \renewcommand{\erwaehnteWerke}{Werke: Das weite Land. Tragikomödie in fünf Akten, Schlaflied für Mirjam}
               \section[Arthur Schnitzler an Richard Beer-Hofmann, 23. 7. 1910]{ Arthur Schnitzler an Richard Beer-Hofmann, 23. 7. 1910}\nopagebreak\mylabel{v}\rehead{ }\begin{ledgroupsized}[t]{13cm}\normalsize\beginnumbering \toendnotes[C]{\smallbreak\pagebreak[2]} \Standort{YCGL, MSS 31.}
\physDesc{Brief, 1 Blatt, 4 Seiten, , , , Umschlag
\newline{}Adresse mit Schreibmaschine
\newline{}Handschrift: Bleistift, deutsche Kurrent\newline{}Versand: Stempel: »\nobreak{}\oindex{XVIII., Waehring@\textbf{XVIII., Währing}|pwk}18/\textcolor{gray}{3} Wien
                                          1\textcolor{gray}{14}, 23. VII. 10, 3\nobreak{}«.  \newline{}Ordnung: mit Bleistift von unbekannter Hand am Umschlag datiert: »23. 7.« }\buchAbdrucke{\weitereDrucke{Arthur Schnitzler, Richard Beer-Hofmann: \emph{Briefwechsel 1891–1931}. Hg. Konstanze Fliedl. Wien, Zürich: \emph{Europaverlag} 1992, S. 211–212.} }\toendnotes[C]{\smallbreak}\pstart{}{\pb}\textcolor{gray}{\textbf{Dr. Arthur Schnitzler}}\pend{}\pstart{}\textcolor{gray}{\textbf{Wien XVIII. Spoettelgasse 7\oindex{Edmund-Weiss-Gasse@\textbf{Edmund-Weiß-Gasse}|pw}.}}\pend{}{\bigskip}\pstart{}{\pb}Herrn\pend{}\pstart{}Dr. Richard Beer-Hofmann\pend{}\pstart{}\so{Ischl}\oindex{Bad Ischl@\textbf{Bad Ischl}|pw}\pend{}\pstart{}Steinfeld 6\oindex{Steinfeld@\textbf{Steinfeld}|pw}\pend{}{\bigskip}\pstart
           \noindent{}{\pb}\textcolor{gray}{\textbf{Dr. Arthur Schnitzler}}\hfill XVIII Sternwartestr 71\oindex{Sternwartestrasse@\textbf{Sternwartestraße}|pw}\pend
           \pstart
           \textcolor{gray}{\textbf{Wien XVIII. Spoettelgasse 7\oindex{Edmund-Weiss-Gasse@\textbf{Edmund-Weiß-Gasse}|pw}.}}\pend
           \pstart{}mein lieber Richard,\pend\pstart
           hier ſende ich Ihnen Ihr Gedicht\pwindex{Beer-Hofmann, Richard 1866-07-11 – 1945-09-26@\textsc{Beer-Hofmann, Richard} (1866-07-11 – 1945-09-26), \emph{Schriftsteller}!Schlaflied fuer Mirjam15. 11. 1898@\strich\emph{Schlaflied für Mirjam} {[}15. 11. 1898{]}|pwv}{ }ſammt Abſchrift, von der So{\geminationm}erremplacantin\pwindex{Hoffmann, Grethe @\textsc{Hoffmann, Grethe}, \emph{Schauspielerin, Schreiberin}|pwv}
               der braven Frieda\pwindex{Pollak, Frieda 08.12.1881 – 13.07.1937@\textsc{Pollak, Frieda} (08.12.1881 – 13.07.1937), \emph{Sekretärin}|pw}. –\pend
           \pstart
           Wir ſind leidlich in Ordnung und {\pb}freuen uns des neuen
               Heims. Ich fahre Dinſtag wieder auf ein paar Tage auf den Semmering\oindex{Semmering@\textbf{Semmering}|pw}, zu Brahm\pwindex{Brahm, Otto 05.02.1856 – 28.11.1912@\textsc{Brahm, Otto} (05.02.1856 – 28.11.1912), \emph{Theaterleiter, Regisseur}|pw} u Kainz\pwindex{Kainz, Josef 02.01.1858 – 20.09.1910@\textsc{Kainz, Josef} (02.01.1858 – 20.09.1910), \emph{Schauspieler}|pw}, der vom Hofreiter\pwindex{Schnitzler, Arthur 15.05.1862 – 21.10.1931@\textsc{Schnitzler, Arthur} (15.05.1862 – 21.10.1931), \emph{Schriftsteller, Mediziner}!weite Land. Tragikomoedie in fuenf Akten1910-10-20@\strich\emph{Das weite Land. Tragikomödie in fünf Akten} {[}1910-10-20{]}|pwv} ſehr angethan iſt und ihn {\pb}gleich ſpielen will.\pend
           \pstart
           Erſter Beſuch in dieſem Hauſe: Baron Berger\pwindex{Berger, Alfred von 30.04.1853 – 24.08.1912@\textsc{Berger, Alfred von} (30.04.1853 – 24.08.1912), \emph{Schriftsteller, Journalist, Theaterleiter}|pw}, aus
               ſolchem Grund. Aber die Sache iſt, aus mannigfachen Gründen noch nicht ganz ſicher. –
               Ins Salzka{\geminationm}er{\pb}gut\oindex{Salzkammergut@\textbf{Salzkammergut}|pw}, we{\geminationn} alles in
               Ordnung hoffen wir nach 20. Auguſt zu reiſen.\pend
           \pstart
           Ich hoffe es geht Ihnen allen ſo wie wirs wünſchen.\pend
           \pstart
           Von Herzen Ihr{\\[\baselineskip]}\spacefill\mbox{A.}\pend
           \leftskip=0em{}
         
         \endnumbering\mylabel{h}\end{ledgroupsized}  \newcommand{\dateiname}{L01950}\newcommand{\titel}{Arthur Schnitzler an Richard Beer-Hofmann, 23. 7. 1910}\newcommand{\editorInnen}{Martin Anton Müller und Gerd-Hermann Susen}%% latex-leseansicht-abspann.tex
%% Abspann für die Leseansicht.
%% Der Schalter \ifkorrekturansicht ist bereits durch den Vorspann gesetzt.

%% latex-abspann.tex
%% Gemeinsamer Abspann für Korrekturansicht und Leseansicht.
%% Setzt den Schalter \ifkorrekturansicht voraus (gesetzt in den
%% einbindenden Dateien latex-korrekturansicht-abspann.tex bzw.
%% latex-leseansicht-abspann.tex).
%% ---------------------------------------------------------------

\normalsize

% Das esempio-Environment wird nur in der Leseansicht benötigt
\ifkorrekturansicht\else
\newenvironment{esempio}[3]%
{
    \vspace{1.5ex}
    \rlap{\underline{#1}}
    \par
    \setlength{\parindent}{0cm}
    \nopagebreak
    \leftskip=#2cm
    \rightskip=#3cm
}
{
    \par
}
\fi

\doendnotes{C}
\bigskip
\vfill

\clearpage

\footnotesize

\ifkorrekturansicht
  \lohead{\textsc{register}}
\fi

% theindex-Environment neu definieren ohne reledmac
\makeatletter
\renewenvironment{theindex}{%
  \ifkorrekturansicht
    \section*{\indexname}%
  \else
    \subsubsection*{Index der erwähnten Entitäten}%
  \fi
  \setlength{\parindent}{0pt}%
  \setlength{\parskip}{0pt plus 0.3pt}%
  \let\item\@idxitem
}{%
  \ifkorrekturansicht\clearpage\fi
}
\makeatother

\IfFileExists{\jobname-pw.ind}{\input{\jobname-pw.ind}}{}

% Quellenangabe nur in der Leseansicht
\ifkorrekturansicht\else
% Fallback-Definitionen, falls die .tex-Datei \titel etc. nicht gesetzt hat
\providecommand{\titel}{}
\providecommand{\editorInnen}{}
\providecommand{\dateiname}{\jobname}

\vspace{3cm}

\vfill

\footnotesize
\textsc{Quelle}: \titel. Herausgegeben von {\editorInnen}. In: \emph{Arthur Schnitzler: Briefwechsel mit Autorinnen und Autoren}.
 Digitale Edition, https://schnitzler-briefe.acdh.oeaw.ac.at/{\dateiname}.html (Stand \today)
\fi

\end{document}


      