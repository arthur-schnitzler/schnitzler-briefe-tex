%% latex-korrekturansicht-vorspann.tex
%% Vorspann für die Korrekturansicht.
%% Lädt die gemeinsame Datei latex-vorspann.tex mit gesetztem Schalter.

\newif\ifkorrekturansicht
\korrekturansichttrue

\input{../tex-inputs/latex-vorspann}


\section[Richard Beer-Hofmann an Arthur Schnitzler, 12. 7. 1909]{L01855 Richard Beer-Hofmann an Arthur Schnitzler, 12. 7. 1909}
\nopagebreak\mylabel{L01855v}
\rehead{ }\normalsize\beginnumbering\briefempfaengerindex{Schnitzler, Arthur@\textsc{Schnitzler, Arthur}!zzzBeer-Hofmann, Richard@\emph{von Richard Beer-Hofmann}!1909-07-121@{12. 7. 1909}|(be}
\toendnotes[C]{\smallbreak\pagebreak[2]}\Standort{CUL, Schnitzler, B 8.}
\physDesc{Kartenbrief, 292 Zeichen
\newline{}Handschrift: schwarze Tinte, lateinische Kurrent
\newline{}Versand: 1) Stempel: »\nobreak{}\oindex{Pichl am See@\textbf{Pichl am See}, \emph{P.PPL}|pwk}Pichl am Mondsee, 13 7 09\nobreak{}«.   2) Stempel: »\nobreak{}\oindex{Edlach@\textbf{Edlach}, \emph{P.PPL}|pwk}Edlach b. Reichenau in N.OE, 14 7 09, 8–12V\nobreak{}«. 
\newline{}Schnitzler: mit Bleistift beschriftet: »\textsc{Beerhof}« 
\newline{}Ordnung: mit Bleistift von unbekannter Hand nummeriert:
                                    »220« }
\buchAbdrucke{\weitereDrucke{Arthur Schnitzler, Richard Beer-Hofmann: \emph{Briefwechsel 1891–1931}. Wien, Zürich: \emph{Europaverlag} 1992, S. 194.} }\toendnotes[C]{\smallbreak}\pstart{}{\pb}Herrn\pend{}\pstart{}D\textsuperscript{r} Arthur Schnitzler\pend{}\pstart{}Edlach\oindex{Edlach@\textbf{Edlach}, \emph{P.PPL}|pw}\pend{}\pstart{}bei Reichenau\oindex{Reichenau an der Rax@\textbf{Reichenau an der Rax}, \emph{A.ADM3}|pw}\pend{}{\bigskip}\vspace{1em}
\pstart
           \raggedleft{}{\pb}12./VII 09\pend
           \vspace{0.5em}
\pstart
           Lieber Arthur! In 12 Tagen, 9 Regentage. Der Regen hält an, 5° \strikeout{Kälte} Wärme (?) am Nachmittag. Wir wollen schon am
                  15. an den Lido\oindex{Lido@\textbf{Lido}, \emph{P.PPL}|pw}, u. sehnen uns
               nach Hitze. Vielleicht sind wir von 1–15 Aug. in Wien\oindex{Wien@\textbf{Wien}, \emph{A.ADM2}|pw}. Herzliche Grüsse Ihnen und Ihrer Frau\pwindex{Schnitzler, Olga 17.01.1882 – 13.01.1970@\textsc{Schnitzler, Olga} (17.01.1882 – 13.01.1970), \emph{Schauspieler/Schauspielerin, Sänger/Sängerin}|pwv}.\pend
           \pstart \label{T_L01855-1v}\edtext{Ihr \spacefill\mbox{Richard.}}{\lemma{\textnormal{\emph{Ihr Richard.}}}\Cendnote{\textnormal{quer am rechten Rand}}}\label{T_L01855-1}\pend{}\selectlanguage{ngerman}\endnumbering\briefempfaengerindex{Schnitzler, Arthur@\textsc{Schnitzler, Arthur}!zzzBeer-Hofmann, Richard@\emph{von Richard Beer-Hofmann}!1909-07-121@{12. 7. 1909}|)be}\mylabel{L01855h}  \normalsize

\doendnotes{C}
\bigskip
\vfill

\clearpage

\footnotesize

\lohead{\textsc{register}}

% Definiere theindex-Environment komplett neu ohne reledmac
\makeatletter
\renewenvironment{theindex}{%
  \section*{\indexname}%
  \setlength{\parindent}{0pt}%
  \setlength{\parskip}{0pt plus 0.3pt}%
  \let\item\@idxitem
}{%
  \clearpage
}
\makeatother

\IfFileExists{\jobname-pw.ind}{\input{\jobname-pw.ind}}{}

\end{document}

      