%% latex-leseansicht-vorspann.tex
%% Vorspann für die Leseansicht.
%% Lädt die gemeinsame Datei latex-vorspann.tex mit nicht gesetztem Schalter.

\newif\ifkorrekturansicht
\korrekturansichtfalse

\input{../tex-inputs/latex-vorspann}


\section[Friedrich M. Fels an Arthur Schnitzler, 18. 6. 1891]{L00019 Friedrich M. Fels an Arthur Schnitzler, 18. 6. 1891}
\nopagebreak\mylabel{L00019v}
\rehead{ }\normalsize\beginnumbering\briefempfaengerindex{Schnitzler, Arthur@\textsc{Schnitzler, Arthur}!zzzFels, Friedrich Michael@\emph{von Friedrich Michael Fels}!1891-06-181@{18. 6. 1891}|(be}
\toendnotes[C]{\smallbreak\pagebreak[2]}
\correspDesc{Versand  durch Friedrich M. Fels am 18. 6. 1891 in Wien
\newline{}Erhalt  durch Arthur Schnitzler im Zeitraum [18. 6. 1891
                  – 22. 6. 1891?] in Wien}\toendnotes[C]{\smallbreak}
\Standort{DLA, A:Schnitzler, HS.NZ85.1.2956.}
\physDesc{Brief, 1 Blatt, 1 Seite, 500 Zeichen
\newline{}Handschrift: schwarze Tinte, lateinische Kurrent
\newline{}Schnitzler: 1) mit Bleistift beschriftet mit »\textsc{Fels}«  2) mit rotem Buntstift beschriftet mit »\textsc{Fels}«}\toendnotes[C]{\smallbreak}
\pstart
           {\pb}\textcolor{gray}{\textbf{»Moderne
                        Rundſchau\orgindex{Moderne Dichtung/Moderne Rundschau@Moderne Dichtung/Moderne Rundschau|pw}«}}\hfill \textcolor{gray}{\textbf{Redaction:}}\pend
           
\pstart
           \textcolor{gray}{\textbf{Halbmonatsſchrift}}\hfill \textcolor{gray}{\textbf{VIII., Buchfeldgaſſe 8\oindex{Wien@\textbf{Wien}!VIII., Josefstadt@\textbf{VIII., Josefstadt}!Buchfeldgasse@\textbf{Buchfeldgasse}, \emph{Straße}|pw}}}\pend
           
\pstart
           \textcolor{gray}{\textbf{Herausgegeben von \textbf{Dr. J. Joachim\pwindex{Joachim, Jaques 24.\,11.\,1866 Wien – 7.\,11.\,1925 ebd.@\textsc{Joachim, Jaques} (24.\,11.\,1866 Wien – 7.\,11.\,1925 ebd.), \emph{Rechtswissenschaftler, Rechtsanwalt, Herausgeber}|pw}} und \textbf{E. M. Kafka\pwindex{Kafka, Eduard Michael 11.\,3.\,1869 Wien – 6.\,8.\,1893 Brünn@\textsc{Kafka, Eduard Michael} (11.\,3.\,1869 Wien – 6.\,8.\,1893 Brünn), \emph{Redakteur}|pw}}}}\pend
           
\pstart
           \textcolor{gray}{\textbf{Verlag von \textbf{Leopold Weiß\orgindex{Leopold Weiss@Leopold Weiss|pw}}}}\hfill \textcolor{gray}{\textbf{Adminiſtration:}}\pend
           
\pstart
           \raggedleft{}\textcolor{gray}{\textbf{I., Tuchlauben 7\oindex{Wien@\textbf{Wien}!I., Innere Stadt@\textbf{I., Innere Stadt}!Tuchlauben@\textbf{Tuchlauben}, \emph{Straße}|pw}}}\pend
           
\pstart
           \raggedleft{}\textcolor{gray}{\textbf{Wien\oindex{Wien@\textbf{Wien}, \emph{Verwaltungsgebiet}|pw}}}\pend
           
\pstart
           \raggedleft{}\textcolor{gray}{\textbf{am}}{ }18. Juni \textcolor{gray}{\textbf{189}}1\pend
           
\pstart\center{}Lieber Herr Doktor!\pend\vspace{0.5em}
\pstart
           Haben Sie keine Skizze von 2–3 Druckseiten fertig? Wir brauchen für das nächste Heft
               unumgänglich eine so kurze, da \label{K_L00019-1v}\edtext{Held\pwindex{Held, Franz 30.\,5.\,1862 Düsseldorf – 4.\,2.\,1908 Rankweil@\textsc{Held, Franz} (30.\,5.\,1862 Düsseldorf – 4.\,2.\,1908 Rankweil), \emph{Schriftsteller}|pw}\pwindex{Held, Franz 30.\,5.\,1862 Düsseldorf – 4.\,2.\,1908 Rankweil@\textsc{Held, Franz} (30.\,5.\,1862 Düsseldorf – 4.\,2.\,1908 Rankweil), \emph{Schriftsteller}!Pyrrhus-Sieg. Geschichte eines glücklichen Pechvogels@\strich\emph{Ein Pyrrhus-Sieg. Geschichte eines glücklichen Pechvogels}|pwv}}{\lemma{\textnormal{\emph{Held}}}\Cendnote{\textnormal{\emph{Ein Pyrrhus-Sieg. Geschichte eines glücklichen
                     Pechvogels}\pwindex{Held, Franz 30.\,5.\,1862 Düsseldorf – 4.\,2.\,1908 Rankweil@\textsc{Held, Franz} (30.\,5.\,1862 Düsseldorf – 4.\,2.\,1908 Rankweil), \emph{Schriftsteller}!Pyrrhus-Sieg. Geschichte eines glücklichen Pechvogels@\strich\emph{Ein Pyrrhus-Sieg. Geschichte eines glücklichen Pechvogels}|pwk} erschien in sechs Teilen zwischen 15. 5. und
                     1. 8. 1891.}}}\label{K_L00019-1} und besonders \label{K_L00019-2v}\edtext{David\pwindex{David, Jakob Julius 6.\,2.\,1859 Hranice – 20.\,11.\,1906 Wien@\textsc{David, Jakob Julius} (6.\,2.\,1859 Hranice – 20.\,11.\,1906 Wien), \emph{Schriftsteller, Journalist}!Hagars Sohn. Schauspiel in vier Acten@\strich\emph{Hagars Sohn. Schauspiel in vier Acten}|pwv}\pwindex{David, Jakob Julius 6.\,2.\,1859 Hranice – 20.\,11.\,1906 Wien@\textsc{David, Jakob Julius} (6.\,2.\,1859 Hranice – 20.\,11.\,1906 Wien), \emph{Schriftsteller, Journalist}|pw}}{\lemma{\textnormal{\emph{David}}}\Cendnote{\textnormal{\emph{Hagars Sohn. Schauspiel in vier Acten}\pwindex{David, Jakob Julius 6.\,2.\,1859 Hranice – 20.\,11.\,1906 Wien@\textsc{David, Jakob Julius} (6.\,2.\,1859 Hranice – 20.\,11.\,1906 Wien), \emph{Schriftsteller, Journalist}!Hagars Sohn. Schauspiel in vier Acten@\strich\emph{Hagars Sohn. Schauspiel in vier Acten}|pwk}
                  erschien in vier Teilen zwischen 1. 6. und
                  15. 7. 1891.}}}\label{K_L00019-2} zu viel Raum in Anspruch nehmen; vorrätig haben
               wir aber nur längere Novelletten. Sie würden uns ausserordentlich verpflichten, we{\geminationn} Sie uns etwas gäben; Kafka\pwindex{Kafka, Eduard Michael 11.\,3.\,1869 Wien – 6.\,8.\,1893 Brünn@\textsc{Kafka, Eduard Michael} (11.\,3.\,1869 Wien – 6.\,8.\,1893 Brünn), \emph{Redakteur}|pw}{ }ſprach von einem \label{K_L00019-3v}\edtext{Märchen\pwindex{Schnitzler, Arthur 15.\,5.\,1862 Wien – 21.\,10.\,1931 ebd.@\textsc{Schnitzler, Arthur} (15.\,5.\,1862 Wien – 21.\,10.\,1931 ebd.), \emph{Schriftsteller, Mediziner}!drei Elixire@\strich\emph{Die drei Elixire}|pwuv}}{\lemma{\textnormal{\emph{Märchen}}}\Cendnote{\textnormal{Unter dem Namen \emph{Jung Wien}\orgindex{Jung Wien@Jung Wien|pwk} agierte ein Verein, der sich zumindest zwischen
                     17. 3. 1891 und
                     5. 5. 1891
                  wöchentlich in der Weinhandlung Wieninger\oindex{Wien@\textbf{Wien}!I., Innere Stadt@\textbf{I., Innere Stadt}!Joseph G. Wieninger, Weinhandlung@\textbf{Joseph G. Wieninger, Weinhandlung}, \emph{Gastgewerbegebäude}|pwk}
                  traf. Am 14. 4. 1891
                  las Schnitzler dort \emph{Die drei Elixire}\pwindex{Schnitzler, Arthur 15.\,5.\,1862 Wien – 21.\,10.\,1931 ebd.@\textsc{Schnitzler, Arthur} (15.\,5.\,1862 Wien – 21.\,10.\,1931 ebd.), \emph{Schriftsteller, Mediziner}!drei Elixire@\strich\emph{Die drei Elixire}|pwk} vor. Eine Lesung aus dem Theaterstück \emph{Das Märchen}\pwindex{Schnitzler, Arthur 15.\,5.\,1862 Wien – 21.\,10.\,1931 ebd.@\textsc{Schnitzler, Arthur} (15.\,5.\,1862 Wien – 21.\,10.\,1931 ebd.), \emph{Schriftsteller, Mediziner}!Märchen. Schauspiel in drei Aufzügen@\strich\emph{Das Märchen. Schauspiel in drei Aufzügen}|pwk}, das er gerade schrieb, ist zu
                  diesem Zeitpunkt eher unwahrscheinlich. }}}\label{K_L00019-3}, das Sie bei Wieninger\oindex{Wien@\textbf{Wien}!I., Innere Stadt@\textbf{I., Innere Stadt}!Joseph G. Wieninger, Weinhandlung@\textbf{Joseph G. Wieninger, Weinhandlung}, \emph{Gastgewerbegebäude}|pw} vorgelesen haben sollen – wohl \label{K_L00019-4v}\edtext{ehe ich dem Kreise angehörte}{\lemma{\textnormal{\emph{ehe … angehörte}}}\Cendnote{\textnormal{Friedrich M. Fels\pwindex{Fels, Friedrich Michael *~1864 Bad Dürkheim@\textsc{Fels, Friedrich Michael} (*~1864 Bad Dürkheim), \emph{Journalist}|pwk} wird in Schnitzlers{ }\emph{Tagebuch}\pwindex{Schnitzler, Arthur 15.\,5.\,1862 Wien – 21.\,10.\,1931 ebd.@\textsc{Schnitzler, Arthur} (15.\,5.\,1862 Wien – 21.\,10.\,1931 ebd.), \emph{Schriftsteller, Mediziner}!Tagebuch@\strich\emph{Tagebuch}|pwk}
                  erstmals am 21. 4. 1891 erwähnt.}}}\label{K_L00019-4}.\pend
           \pstart Mit bestem Gruss\pend{}
\pstart
           \raggedleft{}\textcolor{gray}{\textbf{\textit{Redaction der »Modernen
                     Rundſchau\orgindex{Moderne Dichtung/Moderne Rundschau@Moderne Dichtung/Moderne Rundschau|pw}.«}}}\pend
           
\pstart
           \spacefill\mbox{I. V.\hspace*{1.5em}Friedr. M. Fels}\pend
           \selectlanguage{ngerman}\endnumbering\briefempfaengerindex{Schnitzler, Arthur@\textsc{Schnitzler, Arthur}!zzzFels, Friedrich Michael@\emph{von Friedrich Michael Fels}!1891-06-181@{18. 6. 1891}|)be}\mylabel{L00019h}  \newcommand{\dateiname}{L00019}\newcommand{\titel}{Friedrich M. Fels an Arthur Schnitzler, 18. 6. 1891}\newcommand{\editorInnen}{Martin Anton Müller und Gerd-Hermann Susen}%% latex-leseansicht-abspann.tex
%% Abspann für die Leseansicht.
%% Der Schalter \ifkorrekturansicht ist bereits durch den Vorspann gesetzt.

%% latex-abspann.tex
%% Gemeinsamer Abspann für Korrekturansicht und Leseansicht.
%% Setzt den Schalter \ifkorrekturansicht voraus (gesetzt in den
%% einbindenden Dateien latex-korrekturansicht-abspann.tex bzw.
%% latex-leseansicht-abspann.tex).
%% ---------------------------------------------------------------

\normalsize

% Das esempio-Environment wird nur in der Leseansicht benötigt
\ifkorrekturansicht\else
\newenvironment{esempio}[3]%
{
    \vspace{1.5ex}
    \rlap{\underline{#1}}
    \par
    \setlength{\parindent}{0cm}
    \nopagebreak
    \leftskip=#2cm
    \rightskip=#3cm
}
{
    \par
}
\fi

\doendnotes{C}
\bigskip
\vfill

\clearpage

\footnotesize

\ifkorrekturansicht
  \lohead{\textsc{register}}
\fi

% theindex-Environment neu definieren ohne reledmac
\makeatletter
\renewenvironment{theindex}{%
  \ifkorrekturansicht
    \section*{\indexname}%
  \else
    \subsubsection*{Index der erwähnten Entitäten}%
  \fi
  \setlength{\parindent}{0pt}%
  \setlength{\parskip}{0pt plus 0.3pt}%
  \let\item\@idxitem
}{%
  \ifkorrekturansicht\clearpage\fi
}
\makeatother

\IfFileExists{\jobname-pw.ind}{\input{\jobname-pw.ind}}{}

% Quellenangabe nur in der Leseansicht
\ifkorrekturansicht\else
% Fallback-Definitionen, falls die .tex-Datei \titel etc. nicht gesetzt hat
\providecommand{\titel}{}
\providecommand{\editorInnen}{}
\providecommand{\dateiname}{\jobname}

\vspace{3cm}

\vfill

\footnotesize
\textsc{Quelle}: \titel. Herausgegeben von {\editorInnen}. In: \emph{Arthur Schnitzler: Briefwechsel mit Autorinnen und Autoren}.
 Digitale Edition, https://schnitzler-briefe.acdh.oeaw.ac.at/{\dateiname}.html (Stand \today)
\fi

\end{document}


