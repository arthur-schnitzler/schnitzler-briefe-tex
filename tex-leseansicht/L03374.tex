%% latex-korrekturansicht-vorspann.tex
%% Vorspann für die Korrekturansicht.
%% Lädt die gemeinsame Datei latex-vorspann.tex mit gesetztem Schalter.

\newif\ifkorrekturansicht
\korrekturansichttrue

\input{../tex-inputs/latex-vorspann}


\section[ Paul Goldmann an Arthur Schnitzler, 15. 6. {[}1903{]}]{L03374 Paul Goldmann an Arthur Schnitzler, 15. 6. {[}1903{]}}
\nopagebreak\mylabel{L03374v}
\rehead{ }\normalsize\beginnumbering\briefempfaengerindex{Schnitzler, Arthur@\textsc{Schnitzler, Arthur}!zzzGoldmann, Paul@\emph{von Paul Goldmann}!1903-06-151@{15. 6. {[}1903{]}}|(be}
\toendnotes[C]{\smallbreak\pagebreak[2]}\Standort{DLA, A:Schnitzler, HS.NZ85.1.3173.}
\physDesc{Brief, 1 Blatt, 2 Seiten, 452 Zeichen
\newline{}Handschrift: blaue Tinte, deutsche Kurrent
\newline{}Schnitzler: 1) mit Bleistift das Jahr »903.« vermerkt  2) mit rotem Buntstift eine Unterstreichung}\toendnotes[C]{\smallbreak}
\pstart
           \raggedleft{}{\pb}\textcolor{gray}{\textbf{DESSAUERSTRASSE 19\oindex{Dessauer Strasse@\textbf{Dessauer Straße}, \emph{Straße (K.STR)}|pw}}}\pend
           
\pstart
           Berlin\oindex{Berlin@\textbf{Berlin}, \emph{P.PPLC}|pw}, 15. Juni.\pend
           
\pstart\center{}Mein lieber Freund,\pend\vspace{0.5em}
\pstart
           Ich danke Dir für Deine lieben \label{K_L03374-1v}\edtext{Karten}{\lemma{\textnormal{\emph{Karten}}}\Cendnote{\textnormal{Schnitzler und Olga Gussmann\pwindex{Schnitzler, Olga 17.01.1882 – 13.01.1970@\textsc{Schnitzler, Olga} (17.01.1882 – 13.01.1970), \emph{Schauspieler/Schauspielerin, Sänger/Sängerin}|pwk} reisten zwischen 28. 5. 1903 und 15. 6. 1903 nach Italien\oindex{Italien@\textbf{Italien}, \emph{A.PCLI}|pwk} und Südtirol\oindex{Suedtirol@\textbf{Südtirol}, \emph{A.ADM2}|pwk}.}}}\label{K_L03374-1} und bitte Dich, \textsc{Olga\pwindex{Schnitzler, Olga 17.01.1882 – 13.01.1970@\textsc{Schnitzler, Olga} (17.01.1882 – 13.01.1970), \emph{Schauspieler/Schauspielerin, Sänger/Sängerin}|pw}} für ihre Grüße zu danken.\pend
           
\pstart
           Ich habe wahnſinnig viel zu thun und kann daher \textsc{Olgas\pwindex{Schnitzler, Olga 17.01.1882 – 13.01.1970@\textsc{Schnitzler, Olga} (17.01.1882 – 13.01.1970), \emph{Schauspieler/Schauspielerin, Sänger/Sängerin}|pw}} Brief \label{K_L03374-2v}\edtext{noch immer nicht}{\lemma{\textnormal{\emph{noch immer nicht}}}\Cendnote{\textnormal{Siehe Paul Goldmann an Arthur Schnitzler, 2[2?]. 5. [1903].
               }}}\label{K_L03374-2} beantworten.\pend
           
\pstart
           \textsc{Fuldas\pwindex{Fulda, Ludwig 15.07.1862 – 30.03.1939@\textsc{Fulda, Ludwig} (15.07.1862 – 30.03.1939), \emph{Schriftsteller/Schriftstellerin, Übersetzer/Übersetzerin}|pw}\pwindex{DAlbert, Ida 05.12.1869 – 1926-10-06@\textsc{d’Albert, Ida} (05.12.1869 – 1926-10-06), \emph{Schauspieler/Schauspielerin}|pw}} laſſen ſich, wie ich höre, diesmal ernſtlich {\pb}\label{K_L03374-3v}\edtext{ſcheiden}{\lemma{\textnormal{\emph{ſcheiden}}}\Cendnote{\textnormal{Schnitzler hatte bereits am 28. 4. 1903 von der
                  Scheidung von Ludwig\pwindex{Fulda, Ludwig 15.07.1862 – 30.03.1939@\textsc{Fulda, Ludwig} (15.07.1862 – 30.03.1939), \emph{Schriftsteller/Schriftstellerin, Übersetzer/Übersetzerin}|pwk} und Ida Fulda\pwindex{DAlbert, Ida 05.12.1869 – 1926-10-06@\textsc{d’Albert, Ida} (05.12.1869 – 1926-10-06), \emph{Schauspieler/Schauspielerin}|pwk} erfahren. Diese waren seit 1893 verheiratet gewesen.}}}\label{K_L03374-3}; die Scheidungsklage ſoll bereits eingereicht
               ſein. Weißt Du etwas davon? Er\pwindex{Fulda, Ludwig 15.07.1862 – 30.03.1939@\textsc{Fulda, Ludwig} (15.07.1862 – 30.03.1939), \emph{Schriftsteller/Schriftstellerin, Übersetzer/Übersetzerin}|pwv} iſt in Baden Baden\oindex{Baden-Baden@\textbf{Baden-Baden}, \emph{P.PPL}|pw}, ſie\pwindex{DAlbert, Ida 05.12.1869 – 1926-10-06@\textsc{d’Albert, Ida} (05.12.1869 – 1926-10-06), \emph{Schauspieler/Schauspielerin}|pwv}, glaube ich, in Berlin\oindex{Berlin@\textbf{Berlin}, \emph{P.PPLC}|pw}.\pend
           
\pstart
           Herzlichſte Grüße Dir und \textsc{Olga\pwindex{Schnitzler, Olga 17.01.1882 – 13.01.1970@\textsc{Schnitzler, Olga} (17.01.1882 – 13.01.1970), \emph{Schauspieler/Schauspielerin, Sänger/Sängerin}|pw}}! Und weiter: \label{K_L03374-4v}\edtext{glückliche
                  Fahrt}{\lemma{\textnormal{\emph{glückliche
                  Fahrt}}}\Cendnote{\textnormal{Schnitzlers Reise endete am Morgen des
                  Folgetags.}}}\label{K_L03374-4}! {\\[\baselineskip]}Dein {\\[\baselineskip]}\spacefill\mbox{Paul Goldmn}\pend
           \leftskip=0em{}\selectlanguage{ngerman}\endnumbering\briefempfaengerindex{Schnitzler, Arthur@\textsc{Schnitzler, Arthur}!zzzGoldmann, Paul@\emph{von Paul Goldmann}!1903-06-151@{15. 6. {[}1903{]}}|)be}\mylabel{L03374h}  \normalsize

\doendnotes{C}
\bigskip
\vfill

\clearpage

\footnotesize

\lohead{\textsc{register}}

% Definiere theindex-Environment komplett neu ohne reledmac
\makeatletter
\renewenvironment{theindex}{%
  \section*{\indexname}%
  \setlength{\parindent}{0pt}%
  \setlength{\parskip}{0pt plus 0.3pt}%
  \let\item\@idxitem
}{%
  \clearpage
}
\makeatother

\IfFileExists{\jobname-pw.ind}{\input{\jobname-pw.ind}}{}

\end{document}

      