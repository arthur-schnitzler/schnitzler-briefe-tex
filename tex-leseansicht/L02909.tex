%% latex-leseansicht-vorspann.tex
%% Vorspann für die Leseansicht.
%% Lädt die gemeinsame Datei latex-vorspann.tex mit nicht gesetztem Schalter.

\newif\ifkorrekturansicht
\korrekturansichtfalse

\input{../tex-inputs/latex-vorspann}


\section[ Paul Goldmann an Arthur Schnitzler, 29. 3. {[}1900{]}]{L02909 Paul Goldmann an Arthur Schnitzler,  29. 3. [1900]}
\nopagebreak\mylabel{L02909v}
\rehead{ }\normalsize\beginnumbering\briefempfaengerindex{Schnitzler, Arthur@\textsc{Schnitzler, Arthur}!zzzGoldmann, Paul@\emph{von Paul Goldmann}!1900-03-291@{29. 3. [1900]}|(be}
\toendnotes[C]{\smallbreak\pagebreak[2]}
\correspDesc{Versand  durch Paul Goldmann am 29. 3. [1900] in Berlin
\newline{}Erhalt  durch Arthur Schnitzler am [8. 4. 1900?] in Wien}\toendnotes[C]{\smallbreak}
\Standort{DLA, A:Schnitzler, HS.NZ85.1.3170.}
\physDesc{Brief, 1 Blatt, 3 Seiten, 1508 Zeichen
\newline{}Handschrift: blaue Tinte, deutsche Kurrent
\newline{}Beilage: Zeitungsausschnitt, der Text in zwei Spalten, diese beschnitten
                                 und aneinandergeklebt 
\newline{}Schnitzler: 1) mit Bleistift das Jahr »900.« vermerkt  2) mit rotem Buntstift drei Unterstreichungen}\toendnotes[C]{\smallbreak}
\pstart
           {\pb}\textcolor{gray}{\textbf{DESSAUERSTRASSE 19}}\oindex{Dessauer Straße@\textbf{Dessauer Straße}, \emph{Straße}|pw}\pend
           
\pstart
           \raggedleft{}Berlin\oindex{Berlin@\textbf{Berlin}, \emph{Hauptstadt}|pw}, 29. März.\pend
           
\pstart{}Mein lieber Freund,\pend\vspace{0.5em}
\pstart
           Dieſer Brief trifft Dich hoffentlich{ }ſchon irgendwo \strikeout{i\textcolor{gray}{m}} an einem blauen \label{K_L02909-1v}\edtext{Meer}{\lemma{\textnormal{\emph{Meer}}}\Cendnote{\textnormal{Schnitzler war am 27. 3. 1900 über Triest\oindex{Triest@\textbf{Triest}, \emph{Verwaltungsgebiet}|pwk} nach Kroatien\oindex{Kroatien@\textbf{Kroatien}|pwk} verreist, wo er sich bis 7. 4. 1900 aufhielt.}}}\label{K_L02909-1}. Meine treueſten
               Wünſche begleiten Dich auf der Fahrt nach dem Süden{\dotsfour}\pend
           
\pstart
           Anbei der im »Berl. Tageblatt\pwindex{Berliner Tageblatt@\emph{Berliner Tageblatt}|pw}« erſchienene
                  \label{K_L02909-2v}\edtext{Bericht\pwindex{A. P. @\textsc{A. P.}, \emph{Journalist/Journalistin}!In der Gesellschaft für Kunst und Wissenschaft sprach am Mittwoch Abend Adele Schreiber über Arthur Schnitzler]@\strich\emph{[In der Gesellschaft für Kunst und Wissenschaft sprach am Mittwoch Abend Adele Schreiber über Arthur Schnitzler]}|pwv}}{\lemma{\textnormal{\emph{Bericht}}}\Cendnote{\textnormal{A. P.\pwindex{A. P. @\textsc{A. P.}, \emph{Journalist/Journalistin}|pwk}: \emph{[In der Gesellschaft für Kunst und Wissenschaft sprach am Mittwoch Abend
                        Adele Schreiber über Arthur Schnitzler]}\pwindex{A. P. @\textsc{A. P.}, \emph{Journalist/Journalistin}!In der Gesellschaft für Kunst und Wissenschaft sprach am Mittwoch Abend Adele Schreiber über Arthur Schnitzler]@\strich\emph{[In der Gesellschaft für Kunst und Wissenschaft sprach am Mittwoch Abend Adele Schreiber über Arthur Schnitzler]}|pwk}. In: \emph{Berliner Tageblatt}\pwindex{Berliner Tageblatt@\emph{Berliner Tageblatt}|pwk}, Jg. 29, Nr. 162, 29. 3. 1900, Abend-Ausgabe, S. 2–3.}}}\label{K_L02909-2}
               über den Vortrag\pwindex{Schreiber, Adele 29.\,4.\,1872 Wien – 20.\,2.\,1957 Herrliberg@\textsc{Schreiber, Adele} (29.\,4.\,1872 Wien – 20.\,2.\,1957 Herrliberg), \emph{Schriftstellerin, Politikerin, Pädagogin}!Vortrag über Arthur Schnitzler]@\strich\emph{[Vortrag über Arthur Schnitzler]}|pwv}\eventindex{Berlin@\textbf{Berlin}!Vortrag von Adele Schreiber über Arthur Schnitzler, 28.3.1900@Vortrag von Adele Schreiber über Arthur Schnitzler, 28.3.1900|pwv}, den geſtern dieſe \textsc{Adele Schreiber\pwindex{Schreiber, Adele 29.\,4.\,1872 Wien – 20.\,2.\,1957 Herrliberg@\textsc{Schreiber, Adele} (29.\,4.\,1872 Wien – 20.\,2.\,1957 Herrliberg), \emph{Schriftstellerin, Politikerin, Pädagogin}|pw}} über Dich gehalten hat. Er war platt und albern. Nur eine Literatur-Jüdin hat
               die Frechheit, auf die Tribüne zu \strikeout{ſteig}{ }ſteigen, wenn{ }ſie{ }ſo gar nichts zu{ }ſagen hat. Das Schönſte war die Verleſung\eventindex{Berlin@\textbf{Berlin}!Vortrag von Adele Schreiber über Arthur Schnitzler, 28.3.1900@Vortrag von Adele Schreiber über Arthur Schnitzler, 28.3.1900|pwv} der »Weihnachtseinkäufe\pwindex{Schnitzler, Arthur 15.\,5.\,1862 Wien – 21.\,10.\,1931 ebd.@\textsc{Schnitzler, Arthur} (15.\,5.\,1862 Wien – 21.\,10.\,1931 ebd.), \emph{Schriftsteller, Mediziner}!Weihnachts-Einkäufe@\strich\emph{Weihnachts-Einkäufe}|pw}«. Sie wurden erbärmlich geleſen; aber {\pb}nach ihrem Schluß gab es Beifall mitten im Vortrag.
               Es iſt eben etwas darin, das{ }ſelbſt eine Literatur-Jüdin nicht umzubringen vermag.
               Auch die \label{K_L02909-3v}\edtext{Gedichte}{\lemma{\textnormal{\emph{Gedichte}}}\Cendnote{\textnormal{Welches Gedicht gemeint ist, konnte nicht ermittelt werden. Laut dem erwähnten Zeitungsbericht\pwindex{A. P. @\textsc{A. P.}, \emph{Journalist/Journalistin}!In der Gesellschaft für Kunst und Wissenschaft sprach am Mittwoch Abend Adele Schreiber über Arthur Schnitzler]@\strich\emph{[In der Gesellschaft für Kunst und Wissenschaft sprach am Mittwoch Abend Adele Schreiber über Arthur Schnitzler]}|pwkv} handelte es
                  sich um drei Gedichte.}}}\label{K_L02909-3} gefielen{ }ſehr{\dotsfour}\pend
           
\pstart
           \label{K_L02909-4v}\edtext{\textsc{Hoffmannsthal\pwindex{Hofmannsthal, Hugo von 1.\,2.\,1874 Wien – 15.\,7.\,1929 Rodaun@\textsc{Hofmannsthal, Hugo von} (1.\,2.\,1874 Wien – 15.\,7.\,1929 Rodaun), \emph{Schriftsteller}|pw}’s}{ }»\textsc{Antigone}«-Vorſpiel\pwindex{Hofmannsthal, Hugo von 1.\,2.\,1874 Wien – 15.\,7.\,1929 Rodaun@\textsc{Hofmannsthal, Hugo von} (1.\,2.\,1874 Wien – 15.\,7.\,1929 Rodaun), \emph{Schriftsteller}!Vorspiel zur Antigone des Sophokles@\strich\emph{Vorspiel zur Antigone des Sophokles}|pw}}{\lemma{\textnormal{\emph{Hoffmannsthal’s »Antigone«-Vorspiel}}}\Cendnote{\textnormal{Die Uraufführung\eventindex{Berlin@\textbf{Berlin}!Aufführung von Antigone, Uraufführung von Hofmannsthals Vorspiel@Aufführung von Antigone, Uraufführung von Hofmannsthals Vorspiel|pwkv} von Hugo von Hofmannsthals\pwindex{Hofmannsthal, Hugo von 1.\,2.\,1874 Wien – 15.\,7.\,1929 Rodaun@\textsc{Hofmannsthal, Hugo von} (1.\,2.\,1874 Wien – 15.\,7.\,1929 Rodaun), \emph{Schriftsteller}|pwk}{ }\emph{Vorspiel zur Antigone des Sophokles}\pwindex{Hofmannsthal, Hugo von 1.\,2.\,1874 Wien – 15.\,7.\,1929 Rodaun@\textsc{Hofmannsthal, Hugo von} (1.\,2.\,1874 Wien – 15.\,7.\,1929 Rodaun), \emph{Schriftsteller}!Vorspiel zur Antigone des Sophokles@\strich\emph{Vorspiel zur Antigone des Sophokles}|pwk} hatte wenige Tage zuvor, am 26. 3. 1900 im Berlin\oindex{Berlin@\textbf{Berlin}, \emph{Hauptstadt}|pwk}er \emph{Lessing-Theater}\orgindex{Lessing-Theater@Lessing-Theater|pwk}
                  stattgefunden.}}}\label{K_L02909-4} iſt glatt durchgefallen, – ganz nach Verdienſt. Die Kritik
               verwirft und verhöhnt es, und{ }ſie hat Recht. Es iſt ein Skandal, den klaren und edlen
                  Verſen\pwindex{Sophokles 497/496? v.\,u.\,Z. Kolonos – 406/405 v.\,u.\,Z. Athen@\textsc{Sophokles} (497/496? v.\,u.\,Z. Kolonos – 406/405 v.\,u.\,Z. Athen), \emph{Schriftsteller}!Antigone@\strich\emph{Antigone}|pwv} des \textsc{Sophocles\pwindex{Sophokles 497/496? v.\,u.\,Z. Kolonos – 406/405 v.\,u.\,Z. Athen@\textsc{Sophokles} (497/496? v.\,u.\,Z. Kolonos – 406/405 v.\,u.\,Z. Athen), \emph{Schriftsteller}|pw}} dieſes verworrene Gewäſch\pwindex{Hofmannsthal, Hugo von 1.\,2.\,1874 Wien – 15.\,7.\,1929 Rodaun@\textsc{Hofmannsthal, Hugo von} (1.\,2.\,1874 Wien – 15.\,7.\,1929 Rodaun), \emph{Schriftsteller}!Vorspiel zur Antigone des Sophokles@\strich\emph{Vorspiel zur Antigone des Sophokles}|pwv} voranzuſchicken!\pend
           
\pstart
           \textsc{Hoffmannsthal\pwindex{Hofmannsthal, Hugo von 1.\,2.\,1874 Wien – 15.\,7.\,1929 Rodaun@\textsc{Hofmannsthal, Hugo von} (1.\,2.\,1874 Wien – 15.\,7.\,1929 Rodaun), \emph{Schriftsteller}|pw}}, der mir in den fünfzehn Jahren,{ }ſeit ich von Wien\oindex{Wien@\textbf{Wien}, \emph{Verwaltungsgebiet}|pw} fort bin, nicht eine Zeile geſchrieben hat, hat es fertig gebracht, mir
                  {\pb}einen Brief zu{ }ſchreiben, damit ich für{ }ſein Stück\pwindex{Hofmannsthal, Hugo von 1.\,2.\,1874 Wien – 15.\,7.\,1929 Rodaun@\textsc{Hofmannsthal, Hugo von} (1.\,2.\,1874 Wien – 15.\,7.\,1929 Rodaun), \emph{Schriftsteller}!Vorspiel zur Antigone des Sophokles@\strich\emph{Vorspiel zur Antigone des Sophokles}|pwv} Reklame mache. Er\pwindex{Hofmannsthal, Hugo von 1.\,2.\,1874 Wien – 15.\,7.\,1929 Rodaun@\textsc{Hofmannsthal, Hugo von} (1.\,2.\,1874 Wien – 15.\,7.\,1929 Rodaun), \emph{Schriftsteller}|pwv}{ }ſpricht es zwar nicht
               direkt aus, aber die Aufforderung liegt indirekt in dem Briefe. Ein lieber Herr!\pend
           
\pstart
           Ein lieber Herr auch der \textsc{Dr. Brahm\pwindex{Brahm, Otto 5.\,2.\,1856 Hamburg – 28.\,11.\,1912 Berlin@\textsc{Brahm, Otto} (5.\,2.\,1856 Hamburg – 28.\,11.\,1912 Berlin), \emph{Theaterleiter, Regisseur}|pw}}, der, weil ich einige{ }ſeiner direktorialen Mißgriffe \label{K_L02909-5v}\edtext{in der N. Fr. Pr.\pwindex{Neue Freie Presse@\emph{Neue Freie Presse}|pw}{ }conſtatirt\pwindex{Goldmann, Paul 31.\,1.\,1865 Breslau – 25.\,9.\,1935 Wien@\textsc{Goldmann, Paul} (31.\,1.\,1865 Breslau – 25.\,9.\,1935 Wien), \emph{Schriftsteller, Journalist}!Berliner Theater. (Max Halbe’s »Das tausendjährige Reich«.)@\strich\emph{Berliner Theater. (Max Halbe’s »Das tausendjährige Reich«.)}|pwv}}{\lemma{\textnormal{\emph{in … constatirt}}}\Cendnote{\textnormal{Als Hinweis für den Auslöser des Unmuts
                  kann beispielsweise Goldmanns\pwindex{Goldmann, Paul 31.\,1.\,1865 Breslau – 25.\,9.\,1935 Wien@\textsc{Goldmann, Paul} (31.\,1.\,1865 Breslau – 25.\,9.\,1935 Wien), \emph{Schriftsteller, Journalist}|pwk} Feuilleton
                  vom 1. 3. 1900 herangezogen werden, das folgendermaßen begann:
                  »Bei der Aufführung\eventindex{Deutsches Theater Berlin@\textbf{Deutsches Theater Berlin}!Uraufführung von Das tausendjährige Reich, 25.2.1900@Uraufführung von Das tausendjährige Reich, 25.2.1900|pwv} von Max
                     Halbe\pwindex{Halbe, Max 4.\,10.\,1865 Gmina Suchy Dąb – 30.\,11.\,1944 Neuötting@\textsc{Halbe, Max} (4.\,10.\,1865 Gmina Suchy Dąb – 30.\,11.\,1944 Neuötting), \emph{Schriftsteller}|pw}’s neuem Schauſpiel ›Das
                        tauſendjährige Reich\pwindex{Halbe, Max 4.\,10.\,1865 Gmina Suchy Dąb – 30.\,11.\,1944 Neuötting@\textsc{Halbe, Max} (4.\,10.\,1865 Gmina Suchy Dąb – 30.\,11.\,1944 Neuötting), \emph{Schriftsteller}!tausendjährige Reich. Drama in vier Aufzügen@\strich\emph{Das tausendjährige Reich. Drama in vier Aufzügen}|pw}‹ wurde im Deutſchen
                        Theater\orgindex{Deutsches Theater Berlin@Deutsches Theater Berlin|pw} viel geziſcht. Sonſt iſt, namentlich in dieſem Hauſe, das
                     Ziſchen oft eine Gegendemonſtration, die hervorgerufen wird durch den
                     übereifrigen Applaus, welcher dem Autor unbedingt getreue Gefolgſchaft ohne
                     Rückſicht auf Werth oder Unwerth des Stückes{ }ſpendet. Hier aber war es eher
                     umgekehrt das Ziſchen, welches den Applaus hervorrief.« (Paul Goldmann\pwindex{Goldmann, Paul 31.\,1.\,1865 Breslau – 25.\,9.\,1935 Wien@\textsc{Goldmann, Paul} (31.\,1.\,1865 Breslau – 25.\,9.\,1935 Wien), \emph{Schriftsteller, Journalist}|pwk}: \emph{Berliner Theater. (Max Halbe’s »Das tausendjährige
                        Reich«)}\pwindex{Goldmann, Paul 31.\,1.\,1865 Breslau – 25.\,9.\,1935 Wien@\textsc{Goldmann, Paul} (31.\,1.\,1865 Breslau – 25.\,9.\,1935 Wien), \emph{Schriftsteller, Journalist}!Berliner Theater. (Max Halbe’s »Das tausendjährige Reich«.)@\strich\emph{Berliner Theater. (Max Halbe’s »Das tausendjährige Reich«.)}|pwk}. In: \emph{Neue Freie Presse}\pwindex{Neue Freie Presse@\emph{Neue Freie Presse}|pwk},
                     Nr. 12.758, 1. 3. 1900, Morgenblatt, S. 1–4, hier:
                  S. 1).}}}\label{K_L02909-5} habe, mir bei der Begegnung die Hand verweigert! {\dots}\pend
           
\pstart
           Grüß’ Dich Gott, mein lieber Freund, und{ }ſei froh da unten, wo die hellere Sonne{ }ſcheint!\pend
           
\pstart
           Dein {\\[\baselineskip]}\spacefill\mbox{Paul Goldmann.}\pend
           \leftskip=0em{}\selectlanguage{ngerman}\vspace{1em}{\vspace{1\baselineskip}}
\pstart
           {\pb}\textcolor{gray}{\textbf{\textbf{A. P.\pwindex{A. P. @\textsc{A. P.}, \emph{Journalist/Journalistin}|pw}}{ }\textbf{In der Geſellschaft für Kunſt und
                        Wiſſenschaft\orgindex{Lessing-Gesellschaft für Kunst und Wissenschaft@Lessing-Gesellschaft für Kunst und Wissenschaft|pw}}{ }ſprach am Mittwoch{ }Abend{ }\so{Adele Schreiber}\pwindex{Schreiber, Adele 29.\,4.\,1872 Wien – 20.\,2.\,1957 Herrliberg@\textsc{Schreiber, Adele} (29.\,4.\,1872 Wien – 20.\,2.\,1957 Herrliberg), \emph{Schriftstellerin, Politikerin, Pädagogin}|pw} über \so{Arthur Schnitzler}. Die junge Oeſterreich\oindex{Österreich@\textbf{Österreich}|pw}erin\pwindex{Schreiber, Adele 29.\,4.\,1872 Wien – 20.\,2.\,1957 Herrliberg@\textsc{Schreiber, Adele} (29.\,4.\,1872 Wien – 20.\,2.\,1957 Herrliberg), \emph{Schriftstellerin, Politikerin, Pädagogin}|pwv} entrollte in
                  knappen,{ }ſicheren Linien ein Bild von dem geiſtigen Schaffen ihres Landsmannes,
                  dem das norddeutſch\oindex{Deutschland@\textbf{Deutschland}|pwv}e
                  Publikum trotz einiger Bühnenerfolge ziemlich verſtändnißlos gegenüberſteht.
                  Freilich, »wer den Dichter will
                     verſtehen, muß in Dichters Lande gehen,\pwindex{\textcolor{red}{\textsuperscript{XXXX indx1}}!West-östlicher Divan@\strich\emph{West-östlicher Divan}|pwv}« er muß ihn mit dem Gemüth
                  erfaſſen. Dazu den Weg zu zeigen, gelang der Vortragenden\pwindex{Schreiber, Adele 29.\,4.\,1872 Wien – 20.\,2.\,1957 Herrliberg@\textsc{Schreiber, Adele} (29.\,4.\,1872 Wien – 20.\,2.\,1957 Herrliberg), \emph{Schriftstellerin, Politikerin, Pädagogin}|pwv} vortrefflich. Selber ein Wien\oindex{Wien@\textbf{Wien}, \emph{Verwaltungsgebiet}|pw}er Kind, hat{ }ſie in dem Milieu des »Jungen-Wien\oindex{Wien@\textbf{Wien}, \emph{Verwaltungsgebiet}|pw}« gelebt, und mit wenigen feinen Strichen
                  vermochte{ }ſie die Eigenart dieſes Kreiſes zu{ }ſkizziren: Hofmannsthal\pwindex{Hofmannsthal, Hugo von 1.\,2.\,1874 Wien – 15.\,7.\,1929 Rodaun@\textsc{Hofmannsthal, Hugo von} (1.\,2.\,1874 Wien – 15.\,7.\,1929 Rodaun), \emph{Schriftsteller}|pw}, der zartſinnige Symboliſt, Bahr\pwindex{Bahr, Hermann 19.\,7.\,1863 Linz – 15.\,1.\,1934 München@\textsc{Bahr, Hermann} (19.\,7.\,1863 Linz – 15.\,1.\,1934 München), \emph{Schriftsteller, Kritiker}|pw}, der Satiriker, Hirſchfeld\pwindex{Hirschfeld, Robert 17.\,9.\,1857 Žďár nad Sázavou – 2.\,4.\,1914 Salzburg@\textsc{Hirschfeld, Robert} (17.\,9.\,1857 Žďár nad Sázavou – 2.\,4.\,1914 Salzburg), \emph{Journalist, Musikkritiker}|pw}, der Humoriſt, Altenberg\pwindex{Altenberg, Peter 9.\,3.\,1859 Wien – 8.\,1.\,1919 ebd.@\textsc{Altenberg, Peter} (9.\,3.\,1859 Wien – 8.\,1.\,1919 ebd.), \emph{Schriftsteller}|pw}, der{ }ſenſitive Stimmungsmenſch, und endlich Schnitzler, der
                  potenzirte Oeſterreich\oindex{Österreich@\textbf{Österreich}|pw}er. Sie{ }ſind
                  Realiſten, aber keine von der derben Sorte, die Heimath ihrer Seele iſt Griechenland\oindex{Griechenland@\textbf{Griechenland}|pw},{ }ſie{ }ſind Schönheitsſucher. Ihre
                  Poeſie iſt eine Miſchung aus romaniſch-ſlawisch-orientaliſchen Einflüſſen, wie{ }ſie
                  das moderne Oeſterreich\oindex{Österreich@\textbf{Österreich}|pw} kennzeichnen. Sie
                  haben etwas den Fran\oindex{Frankreich@\textbf{Frankreich}|pwv}zoſen
                  Verwandtes. Wie dieſe{ }ſind{ }ſie Plauderer, vor allem hat Schnitzler die Grazie der
                  Form. Eine weiche Müdigkeit liegt über{ }ſeinen Schöpfungen, von denen jede ein
                  Stück Selbſtbiographie iſt. »Einen leichtſinnigen Melancholiker\pwindex{Schnitzler, Arthur 15.\,5.\,1862 Wien – 21.\,10.\,1931 ebd.@\textsc{Schnitzler, Arthur} (15.\,5.\,1862 Wien – 21.\,10.\,1931 ebd.), \emph{Schriftsteller, Mediziner}!Weihnachts-Einkäufe@\strich\emph{Weihnachts-Einkäufe}|pwv}« nennt er{ }ſich einmal darin. Er liebt
                  die matten, feinen,{ }ſubtilen Farben. Der nüchterne Verſtandesmenſch nennt ihn
                  leicht \so{weibiſch}, aber er iſt nur{ }ſenſitiv. Allerdings,
                  die großen, neuen Probleme gehen ihn nichts an,{ }ſeine Dichtungen haben nur einen
                  Inhalt: \so{die Frau}, aber nicht die ringende, kämpfende,
                  nur die liebende. Seine Heldinnen{ }ſind immer die kleinen,{ }ſüßen Mädel der Wien\oindex{Wien@\textbf{Wien}, \emph{Verwaltungsgebiet}|pw}er Vorſtadt oder verheirathete Weltdamen,
                  die Troſt für ihre Herzensleere im Bruch der ehelichen Treue{ }ſuchen.}}\pend
           
\pstart
           \textcolor{gray}{\textbf{Es iſt ein Inſtrument mit einer Saite, das Schnitzler{ }ſpielt,
                  aber er weiß ihm{ }ſympathische Klänge von wehmüthigem Reiz zu entlocken. Auch wenn
                  er das Intimſte erzählt, bleibt er immer graziös und wird nie unzüchtig. Mit{ }ſeinen erſten Arbeiten trat Schnitzler 1886 hervor. Es
                  war das Märchen »Alcantils
                     Lied\pwindex{Schnitzler, Arthur 15.\,5.\,1862 Wien – 21.\,10.\,1931 ebd.@\textsc{Schnitzler, Arthur} (15.\,5.\,1862 Wien – 21.\,10.\,1931 ebd.), \emph{Schriftsteller, Mediziner}!Alkandi’s Lied@\strich\emph{Alkandi’s Lied}|pwv}«, dann folgte das »Märchen von den Gefallenen\pwindex{Schnitzler, Arthur 15.\,5.\,1862 Wien – 21.\,10.\,1931 ebd.@\textsc{Schnitzler, Arthur} (15.\,5.\,1862 Wien – 21.\,10.\,1931 ebd.), \emph{Schriftsteller, Mediziner}!Märchen. Schauspiel in drei Aufzügen@\strich\emph{Das Märchen. Schauspiel in drei Aufzügen}|pwv}«, in dem der Held alle alten Vorurtheile
                  überwunden hat und ihnen doch beim erſten Verſuch in der Praxis unterliegt. Das
                  Drama »Freiwild\pwindex{Schnitzler, Arthur 15.\,5.\,1862 Wien – 21.\,10.\,1931 ebd.@\textsc{Schnitzler, Arthur} (15.\,5.\,1862 Wien – 21.\,10.\,1931 ebd.), \emph{Schriftsteller, Mediziner}!Freiwild. Schauspiel in 3 Akten@\strich\emph{Freiwild. Schauspiel in 3 Akten}|pw}« behandelt das Duellmotiv in
                  einem meiſterhaft geſchilderten Milieu. Nun folgte »Liebelei\pwindex{Schnitzler, Arthur 15.\,5.\,1862 Wien – 21.\,10.\,1931 ebd.@\textsc{Schnitzler, Arthur} (15.\,5.\,1862 Wien – 21.\,10.\,1931 ebd.), \emph{Schriftsteller, Mediziner}!Liebelei. Schauspiel in drei Akten@\strich\emph{Liebelei. Schauspiel in drei Akten}|pw}«, die Tragödie des Mädchens aus dem Volke,
                  vielleicht des Mädchens überhaupt. Es begründete Schnitzlers Ruf und wurde in die
                  verſchiedenſten Sprachen überſetzt. Das folgende »Vermächtniß\pwindex{Schnitzler, Arthur 15.\,5.\,1862 Wien – 21.\,10.\,1931 ebd.@\textsc{Schnitzler, Arthur} (15.\,5.\,1862 Wien – 21.\,10.\,1931 ebd.), \emph{Schriftsteller, Mediziner}!Vermächtnis. Schauspiel in drei Akten@\strich\emph{Das Vermächtnis. Schauspiel in drei Akten}|pw}« iſt ein{ }ſchwaches Stück, »Die Gefährtin\pwindex{Schnitzler, Arthur 15.\,5.\,1862 Wien – 21.\,10.\,1931 ebd.@\textsc{Schnitzler, Arthur} (15.\,5.\,1862 Wien – 21.\,10.\,1931 ebd.), \emph{Schriftsteller, Mediziner}!Gefährtin. Schauspiel in einem Akt@\strich\emph{Die Gefährtin. Schauspiel in einem Akt}|pw}« dagegen voll Feinheit und Eleganz. In »Paracelſus\pwindex{Schnitzler, Arthur 15.\,5.\,1862 Wien – 21.\,10.\,1931 ebd.@\textsc{Schnitzler, Arthur} (15.\,5.\,1862 Wien – 21.\,10.\,1931 ebd.), \emph{Schriftsteller, Mediziner}!Paracelsus. Versspiel in einem Akt@\strich\emph{Paracelsus. Versspiel in einem Akt}|pw}«{ }ſind die Farben etwas{ }ſtark aufgetragen,
                  großen Bühnenerfolg hatte die{ }ſozialpolitiſche Burleske »Der grüne Kakadu\pwindex{Schnitzler, Arthur 15.\,5.\,1862 Wien – 21.\,10.\,1931 ebd.@\textsc{Schnitzler, Arthur} (15.\,5.\,1862 Wien – 21.\,10.\,1931 ebd.), \emph{Schriftsteller, Mediziner}!grüne Kakadu. Groteske in einem Akt@\strich\emph{Der grüne Kakadu. Groteske in einem Akt}|pw}«, die trotz der hiſtoriſchen Maske völlig
                  modern wirkt. Schnitzlers neueſtes, noch nicht aufgeführtes Stück nennt{ }ſich »Beatrice\pwindex{Schnitzler, Arthur 15.\,5.\,1862 Wien – 21.\,10.\,1931 ebd.@\textsc{Schnitzler, Arthur} (15.\,5.\,1862 Wien – 21.\,10.\,1931 ebd.), \emph{Schriftsteller, Mediziner}!Schleier der Beatrice. Schauspiel in fünf Akten@\strich\emph{Der Schleier der Beatrice. Schauspiel in fünf Akten}|pw}« und iſt in Verſen geſchrieben. Ein
                  Mittelding zwiſchen Buch und Bühne iſt{ }ſein »Anatol\pwindex{Schnitzler, Arthur 15.\,5.\,1862 Wien – 21.\,10.\,1931 ebd.@\textsc{Schnitzler, Arthur} (15.\,5.\,1862 Wien – 21.\,10.\,1931 ebd.), \emph{Schriftsteller, Mediziner}!Anatol@\strich\emph{Anatol}|pw}«, ein Meiſterſtück genialer Plauderei, während{ }ſeine »Novellen\pwindex{Schnitzler, Arthur 15.\,5.\,1862 Wien – 21.\,10.\,1931 ebd.@\textsc{Schnitzler, Arthur} (15.\,5.\,1862 Wien – 21.\,10.\,1931 ebd.), \emph{Schriftsteller, Mediziner}!Frau des Weisen. Novelletten@\strich\emph{Die Frau des Weisen. Novelletten}|pwv}« das Problem des
                  Sterbens, des Loslöſens des Lebenden von dem dem Tode Verfallenen, ergreifend{ }ſchildern. Leichtſinn und Melancholie, beides weiß Schnitzler zu verklären, der
                  vielleicht kein Unſterblicher, aber ein echter Künſtler iſt. Zum Schluß las Adele Schreiber\pwindex{Schreiber, Adele 29.\,4.\,1872 Wien – 20.\,2.\,1957 Herrliberg@\textsc{Schreiber, Adele} (29.\,4.\,1872 Wien – 20.\,2.\,1957 Herrliberg), \emph{Schriftstellerin, Politikerin, Pädagogin}|pw} drei{ }ſeiner lyriſchen
                  Gedichte und die Szene »Weihnachtseinkäufe\pwindex{Schnitzler, Arthur 15.\,5.\,1862 Wien – 21.\,10.\,1931 ebd.@\textsc{Schnitzler, Arthur} (15.\,5.\,1862 Wien – 21.\,10.\,1931 ebd.), \emph{Schriftsteller, Mediziner}!Weihnachts-Einkäufe@\strich\emph{Weihnachts-Einkäufe}|pw}«
                  aus »Anatol\pwindex{Schnitzler, Arthur 15.\,5.\,1862 Wien – 21.\,10.\,1931 ebd.@\textsc{Schnitzler, Arthur} (15.\,5.\,1862 Wien – 21.\,10.\,1931 ebd.), \emph{Schriftsteller, Mediziner}!Anatol@\strich\emph{Anatol}|pw}« vor, und der Beifall, den{ }ſie
                  fand, bewies, daß ihre graziöſe, gleichgeſtimmte Art das Weſen ihres Land\oindex{Österreich@\textbf{Österreich}|pwv}smannes den Hörern
                  wirklich näher gebracht hatte, obgleich wir Norddeutſch\oindex{Deutschland@\textbf{Deutschland}|pwv}en mehr die friſche, klare Morgenluft lieben als den
                  düſteſchweren Hauch{ }ſchwüler Sommernächte voll banger Todesſehnſucht.}}\pend
           \selectlanguage{ngerman}\endnumbering\briefempfaengerindex{Schnitzler, Arthur@\textsc{Schnitzler, Arthur}!zzzGoldmann, Paul@\emph{von Paul Goldmann}!1900-03-291@{29. 3. [1900]}|)be}\mylabel{L02909h}  \newcommand{\dateiname}{L02909}\newcommand{\titel}{Paul Goldmann an Arthur Schnitzler, 29. 3. [1900]}\newcommand{\editorInnen}{Martin Anton Müller und Laura Untner}%% latex-leseansicht-abspann.tex
%% Abspann für die Leseansicht.
%% Der Schalter \ifkorrekturansicht ist bereits durch den Vorspann gesetzt.

%% latex-abspann.tex
%% Gemeinsamer Abspann für Korrekturansicht und Leseansicht.
%% Setzt den Schalter \ifkorrekturansicht voraus (gesetzt in den
%% einbindenden Dateien latex-korrekturansicht-abspann.tex bzw.
%% latex-leseansicht-abspann.tex).
%% ---------------------------------------------------------------

\normalsize

% Das esempio-Environment wird nur in der Leseansicht benötigt
\ifkorrekturansicht\else
\newenvironment{esempio}[3]%
{
    \vspace{1.5ex}
    \rlap{\underline{#1}}
    \par
    \setlength{\parindent}{0cm}
    \nopagebreak
    \leftskip=#2cm
    \rightskip=#3cm
}
{
    \par
}
\fi

\doendnotes{C}
\bigskip
\vfill

\clearpage

\footnotesize

\ifkorrekturansicht
  \lohead{\textsc{register}}
\fi

% theindex-Environment neu definieren ohne reledmac
\makeatletter
\renewenvironment{theindex}{%
  \ifkorrekturansicht
    \section*{\indexname}%
  \else
    \subsubsection*{Index der erwähnten Entitäten}%
  \fi
  \setlength{\parindent}{0pt}%
  \setlength{\parskip}{0pt plus 0.3pt}%
  \let\item\@idxitem
}{%
  \ifkorrekturansicht\clearpage\fi
}
\makeatother

\IfFileExists{\jobname-pw.ind}{\input{\jobname-pw.ind}}{}

% Quellenangabe nur in der Leseansicht
\ifkorrekturansicht\else
% Fallback-Definitionen, falls die .tex-Datei \titel etc. nicht gesetzt hat
\providecommand{\titel}{}
\providecommand{\editorInnen}{}
\providecommand{\dateiname}{\jobname}

\vspace{3cm}

\vfill

\footnotesize
\textsc{Quelle}: \titel. Herausgegeben von {\editorInnen}. In: \emph{Arthur Schnitzler: Briefwechsel mit Autorinnen und Autoren}.
 Digitale Edition, https://schnitzler-briefe.acdh.oeaw.ac.at/{\dateiname}.html (Stand \today)
\fi

\end{document}


