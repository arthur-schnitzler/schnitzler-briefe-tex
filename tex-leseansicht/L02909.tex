%% latex-korrekturansicht-vorspann.tex
%% Vorspann für die Korrekturansicht.
%% Lädt die gemeinsame Datei latex-vorspann.tex mit gesetztem Schalter.

\newif\ifkorrekturansicht
\korrekturansichttrue

\input{../tex-inputs/latex-vorspann}


\section[ Paul Goldmann an Arthur Schnitzler, 29. 3. {[}1900{]}]{L02909 Paul Goldmann an Arthur Schnitzler, 29. 3. {[}1900{]}}
\nopagebreak\mylabel{L02909v}
\rehead{ }\normalsize\beginnumbering\briefempfaengerindex{Schnitzler, Arthur@\textsc{Schnitzler, Arthur}!zzzGoldmann, Paul@\emph{von Paul Goldmann}!1900-03-291@{29. 3. {[}1900{]}}|(be}
\toendnotes[C]{\smallbreak\pagebreak[2]}\Standort{DLA, A:Schnitzler, HS.NZ85.1.3170.}
\physDesc{Brief, 1 Blatt, 3 Seiten, 1508 Zeichen
\newline{}Handschrift: blaue Tinte, deutsche Kurrent
\newline{}Beilage: Zeitungsausschnitt, der Text in zwei Spalten, diese beschnitten
                                 und aneinandergeklebt 
\newline{}Schnitzler: 1) mit Bleistift das Jahr »900.« vermerkt  2) mit rotem Buntstift drei Unterstreichungen}\toendnotes[C]{\smallbreak}
\pstart
           {\pb}\textcolor{gray}{\textbf{DESSAUERSTRASSE 19}}\oindex{Dessauer Strasse@\textbf{Dessauer Straße}, \emph{Straße (K.STR)}|pw}\pend
           
\pstart
           \raggedleft{}Berlin\oindex{Berlin@\textbf{Berlin}, \emph{P.PPLC}|pw}, 29. März.\pend
           
\pstart{}Mein lieber Freund,\pend\vspace{0.5em}
\pstart
           Dieſer Brief trifft Dich hoffentlich ſchon irgendwo \strikeout{i\textcolor{gray}{m}} an einem blauen \label{K_L02909-1v}\edtext{Meer}{\lemma{\textnormal{\emph{Meer}}}\Cendnote{\textnormal{Schnitzler war am 27. 3. 1900 über Triest\oindex{Triest@\textbf{Triest}, \emph{A.ADM3}|pwk} nach Kroatien\oindex{Kroatien@\textbf{Kroatien}, \emph{A.PCLI}|pwk} verreist, wo er sich bis 7. 4. 1900 aufhielt.}}}\label{K_L02909-1}. Meine treueſten
               Wünſche begleiten Dich auf der Fahrt nach dem Süden{\dotsfour}\pend
           
\pstart
           Anbei der im »Berl. Tageblatt\pwindex{Berliner Tageblatt@\emph{Berliner Tageblatt}|pw}« erſchienene
                  \label{K_L02909-2v}\edtext{Bericht\pwindex{In der Gesellschaft fuer Kunst und Wissenschaft sprach am Mittwoch Abend Adele Schreiber ueber Arthur Schnitzler]@\emph{[In der Gesellschaft für Kunst und Wissenschaft sprach am Mittwoch Abend Adele Schreiber über Arthur Schnitzler]}|pwv}}{\lemma{\textnormal{\emph{Bericht}}}\Cendnote{\textnormal{A. P.\pwindex{A. P. @\textsc{A. P.}, \emph{Journalist/Journalistin}|pwk}: \emph{[In der Gesellschaft für Kunst und Wissenschaft sprach am Mittwoch Abend
                        Adele Schreiber über Arthur Schnitzler]}\pwindex{In der Gesellschaft fuer Kunst und Wissenschaft sprach am Mittwoch Abend Adele Schreiber ueber Arthur Schnitzler]@\emph{[In der Gesellschaft für Kunst und Wissenschaft sprach am Mittwoch Abend Adele Schreiber über Arthur Schnitzler]}|pwk}. In: \emph{Berliner Tageblatt}\pwindex{Berliner Tageblatt@\emph{Berliner Tageblatt}|pwk}, Jg. 29, Nr. 162, 29. 3. 1900, Abend-Ausgabe, S. 2–3.}}}\label{K_L02909-2}
               über den Vortrag\pwindex{Vortrag ueber Arthur Schnitzler]@\emph{[Vortrag über Arthur Schnitzler]}|pwv}, den geſtern dieſe \textsc{Adele Schreiber\pwindex{Schreiber, Adele 1872-04-29 – 1957-02-20@\textsc{Schreiber, Adele} (1872-04-29 – 1957-02-20), \emph{Schriftsteller/Schriftstellerin, Politiker/Politikerin, Pädagoge/Pädagogin}|pw}} über Dich gehalten hat. Er war platt und albern. Nur eine Literatur-Jüdin hat
               die Frechheit, auf die Tribüne zu \strikeout{ſteig} ſteigen, wenn
               ſie ſo gar nichts zu ſagen hat. Das Schönſte war die Verleſung der »Weihnachtseinkäufe\pwindex{Weihnachts-Einkaeufe@\emph{Weihnachts-Einkäufe}|pw}«. Sie wurden erbärmlich geleſen; aber {\pb}nach ihrem Schluß gab es Beifall mitten im Vortrag.
               Es iſt eben etwas darin, das ſelbſt eine Literatur-Jüdin nicht umzubringen vermag.
               Auch die \label{K_L02909-3v}\edtext{Gedichte}{\lemma{\textnormal{\emph{Gedichte}}}\Cendnote{\textnormal{Welches Gedicht gemeint ist, konnte nicht ermittelt werden. Laut dem erwähnten Zeitungsbericht\pwindex{In der Gesellschaft fuer Kunst und Wissenschaft sprach am Mittwoch Abend Adele Schreiber ueber Arthur Schnitzler]@\emph{[In der Gesellschaft für Kunst und Wissenschaft sprach am Mittwoch Abend Adele Schreiber über Arthur Schnitzler]}|pwkv} handelte es
                  sich um drei Gedichte.}}}\label{K_L02909-3} gefielen ſehr{\dotsfour}\pend
           
\pstart
           \label{K_L02909-4v}\edtext{\textsc{Hoffmannsthal\pwindex{Hofmannsthal, Hugo von 1874-02-01 – 1929-07-15@\textsc{Hofmannsthal, Hugo von} (1874-02-01 – 1929-07-15), \emph{Schriftsteller/Schriftstellerin}|pw}’s}{ }»\textsc{Antigone}«-Vorſpiel\pwindex{Vorspiel zur Antigone des Sophokles@\emph{Vorspiel zur Antigone des Sophokles}|pw}}{\lemma{\textnormal{\emph{Hoffmannsthal’s »Antigone«-Vorſpiel}}}\Cendnote{\textnormal{Die Uraufführung von Hugo von Hofmannsthals\pwindex{Hofmannsthal, Hugo von 1874-02-01 – 1929-07-15@\textsc{Hofmannsthal, Hugo von} (1874-02-01 – 1929-07-15), \emph{Schriftsteller/Schriftstellerin}|pwk}{ }\emph{Vorspiel zur Antigone des Sophokles}\pwindex{Vorspiel zur Antigone des Sophokles@\emph{Vorspiel zur Antigone des Sophokles}|pwk} hatte wenige Tage zuvor, am 26. 3. 1900 im Berlin\oindex{Berlin@\textbf{Berlin}, \emph{P.PPLC}|pwk}er \emph{Lessing-Theater}\orgindex{Lessing-Theater@Lessing-Theater|pwk}
                  stattgefunden.}}}\label{K_L02909-4} iſt glatt durchgefallen, – ganz nach Verdienſt. Die Kritik
               verwirft und verhöhnt es, und ſie hat Recht. Es iſt ein Skandal, den klaren und edlen
                  Verſen\pwindex{Antigone@\emph{Antigone}|pwv} des \textsc{Sophocles\pwindex{Sophokles 497/496? v. u. Z. – 406/405 v. u. Z.@\textsc{Sophokles} (497/496? v. u. Z. – 406/405 v. u. Z.), \emph{Schriftsteller/Schriftstellerin}|pw}} dieſes verworrene Gewäſch\pwindex{Vorspiel zur Antigone des Sophokles@\emph{Vorspiel zur Antigone des Sophokles}|pwv} voranzuſchicken!\pend
           
\pstart
           \textsc{Hoffmannsthal\pwindex{Hofmannsthal, Hugo von 1874-02-01 – 1929-07-15@\textsc{Hofmannsthal, Hugo von} (1874-02-01 – 1929-07-15), \emph{Schriftsteller/Schriftstellerin}|pw}}, der mir in den fünfzehn Jahren, ſeit ich von Wien\oindex{Wien@\textbf{Wien}, \emph{A.ADM2}|pw} fort bin, nicht eine Zeile geſchrieben hat, hat es fertig gebracht, mir
                  {\pb}einen Brief zu ſchreiben, damit ich für ſein Stück\pwindex{Vorspiel zur Antigone des Sophokles@\emph{Vorspiel zur Antigone des Sophokles}|pwv} Reklame mache. Er\pwindex{Hofmannsthal, Hugo von 1874-02-01 – 1929-07-15@\textsc{Hofmannsthal, Hugo von} (1874-02-01 – 1929-07-15), \emph{Schriftsteller/Schriftstellerin}|pwv} ſpricht es zwar nicht
               direkt aus, aber die Aufforderung liegt indirekt in dem Briefe. Ein lieber Herr!\pend
           
\pstart
           Ein lieber Herr auch der \textsc{Dr. Brahm\pwindex{Brahm, Otto 05.02.1856 – 28.11.1912@\textsc{Brahm, Otto} (05.02.1856 – 28.11.1912), \emph{Theaterleiter/Theaterleiterin, Regisseur/Regisseurin}|pw}}, der, weil ich einige ſeiner direktorialen Mißgriffe \label{K_L02909-5v}\edtext{in der N. Fr. Pr.\pwindex{Neue Freie Presse@\emph{Neue Freie Presse}|pw}{ }conſtatirt\pwindex{Berliner Theater. (Max Halbe s »Das tausendjaehrige Reich«.)@\emph{Berliner Theater. (Max Halbe’s »Das tausendjährige Reich«.)}|pwv}}{\lemma{\textnormal{\emph{in … conſtatirt}}}\Cendnote{\textnormal{Als Hinweis für den Auslöser des Unmuts
                  kann beispielsweise Goldmanns\pwindex{Goldmann, Paul 31.01.1865 – 25.09.1935@\textsc{Goldmann, Paul} (31.01.1865 – 25.09.1935), \emph{Schriftsteller/Schriftstellerin, Journalist/Journalistin}|pwk} Feuilleton
                  vom 1. 3. 1900 herangezogen werden, das folgendermaßen begann:
                     »Bei der Aufführung von Max
                     Halbe\pwindex{Halbe, Max 04.10.1865 – 30.11.1944@\textsc{Halbe, Max} (04.10.1865 – 30.11.1944), \emph{Schriftsteller/Schriftstellerin}|pw}’s neuem Schauſpiel ›Das
                        tauſendjährige Reich\pwindex{tausendjaehrige Reich. Drama in vier Aufzuegen@\emph{Das tausendjährige Reich. Drama in vier Aufzügen}|pw}‹ wurde im Deutſchen
                        Theater\orgindex{Deutsches Theater Berlin@Deutsches Theater Berlin|pw} viel geziſcht. Sonſt iſt, namentlich in dieſem Hauſe, das
                     Ziſchen oft eine Gegendemonſtration, die hervorgerufen wird durch den
                     übereifrigen Applaus, welcher dem Autor unbedingt getreue Gefolgſchaft ohne
                     Rückſicht auf Werth oder Unwerth des Stückes ſpendet. Hier aber war es eher
                     umgekehrt das Ziſchen, welches den Applaus hervorrief.« (Paul Goldmann\pwindex{Goldmann, Paul 31.01.1865 – 25.09.1935@\textsc{Goldmann, Paul} (31.01.1865 – 25.09.1935), \emph{Schriftsteller/Schriftstellerin, Journalist/Journalistin}|pwk}: \emph{Berliner Theater. (Max Halbe’s »Das tausendjährige
                        Reich«)}\pwindex{Berliner Theater. (Max Halbe s »Das tausendjaehrige Reich«.)@\emph{Berliner Theater. (Max Halbe’s »Das tausendjährige Reich«.)}|pwk}. In: \emph{Neue Freie Presse}\pwindex{Neue Freie Presse@\emph{Neue Freie Presse}|pwk},
                     Nr. 12.758, 1. 3. 1900, Morgenblatt, S. 1–4, hier:
                  S. 1).}}}\label{K_L02909-5} habe, mir bei der Begegnung die Hand verweigert! {\dots}\pend
           
\pstart
           Grüß’ Dich Gott, mein lieber Freund, und ſei froh da unten, wo die hellere Sonne
               ſcheint!\pend
           
\pstart
           Dein {\\[\baselineskip]}\spacefill\mbox{Paul Goldmann.}\pend
           \leftskip=0em{}\selectlanguage{ngerman}\vspace{1em}{\vspace{1\baselineskip}}
\pstart
           {\pb}\textcolor{gray}{\textbf{\textbf{A. P.\pwindex{A. P. @\textsc{A. P.}, \emph{Journalist/Journalistin}|pw}}{ }\textbf{In der Geſellschaft für Kunſt und
                        Wiſſenschaft\orgindex{Lessing-Gesellschaft fuer Kunst und Wissenschaft@Lessing-Gesellschaft für Kunst und Wissenschaft|pw}} ſprach am Mittwoch{ }Abend{ }\so{Adele Schreiber}\pwindex{Schreiber, Adele 1872-04-29 – 1957-02-20@\textsc{Schreiber, Adele} (1872-04-29 – 1957-02-20), \emph{Schriftsteller/Schriftstellerin, Politiker/Politikerin, Pädagoge/Pädagogin}|pw} über \so{Arthur Schnitzler}. Die junge Oeſterreich\oindex{Oesterreich@\textbf{Österreich}, \emph{A.PCLI}|pw}erin\pwindex{Schreiber, Adele 1872-04-29 – 1957-02-20@\textsc{Schreiber, Adele} (1872-04-29 – 1957-02-20), \emph{Schriftsteller/Schriftstellerin, Politiker/Politikerin, Pädagoge/Pädagogin}|pwv} entrollte in
                  knappen, ſicheren Linien ein Bild von dem geiſtigen Schaffen ihres Landsmannes,
                  dem das norddeutſch\oindex{Deutschland@\textbf{Deutschland}, \emph{A.PCLI}|pwv}e
                  Publikum trotz einiger Bühnenerfolge ziemlich verſtändnißlos gegenüberſteht.
                  Freilich, »wer den Dichter will
                     verſtehen, muß in Dichters Lande gehen,\pwindex{West-oestlicher Divan@\emph{West-östlicher Divan}|pwv}« er muß ihn mit dem Gemüth
                  erfaſſen. Dazu den Weg zu zeigen, gelang der Vortragenden\pwindex{Schreiber, Adele 1872-04-29 – 1957-02-20@\textsc{Schreiber, Adele} (1872-04-29 – 1957-02-20), \emph{Schriftsteller/Schriftstellerin, Politiker/Politikerin, Pädagoge/Pädagogin}|pwv} vortrefflich. Selber ein Wien\oindex{Wien@\textbf{Wien}, \emph{A.ADM2}|pw}er Kind, hat ſie in dem Milieu des »Jungen-Wien\oindex{Wien@\textbf{Wien}, \emph{A.ADM2}|pw}« gelebt, und mit wenigen feinen Strichen
                  vermochte ſie die Eigenart dieſes Kreiſes zu ſkizziren: Hofmannsthal\pwindex{Hofmannsthal, Hugo von 1874-02-01 – 1929-07-15@\textsc{Hofmannsthal, Hugo von} (1874-02-01 – 1929-07-15), \emph{Schriftsteller/Schriftstellerin}|pw}, der zartſinnige Symboliſt, Bahr\pwindex{Bahr, Hermann 19.07.1863 – 15.01.1934@\textsc{Bahr, Hermann} (19.07.1863 – 15.01.1934), \emph{Schriftsteller/Schriftstellerin, Kritiker/Kritikerin}|pw}, der Satiriker, Hirſchfeld\pwindex{Hirschfeld, Robert 17.09.1857 – 02.04.1914@\textsc{Hirschfeld, Robert} (17.09.1857 – 02.04.1914), \emph{Journalist/Journalistin, Musikkritiker/Musikkritikerin}|pw}, der Humoriſt, Altenberg\pwindex{Altenberg, Peter 09.03.1859 – 08.01.1919@\textsc{Altenberg, Peter} (09.03.1859 – 08.01.1919), \emph{Schriftsteller/Schriftstellerin}|pw}, der ſenſitive Stimmungsmenſch, und endlich Schnitzler, der
                  potenzirte Oeſterreich\oindex{Oesterreich@\textbf{Österreich}, \emph{A.PCLI}|pw}er. Sie ſind
                  Realiſten, aber keine von der derben Sorte, die Heimath ihrer Seele iſt Griechenland\oindex{Griechenland@\textbf{Griechenland}, \emph{A.PCLI}|pw}, ſie ſind Schönheitsſucher. Ihre
                  Poeſie iſt eine Miſchung aus romaniſch-ſlawisch-orientaliſchen Einflüſſen, wie ſie
                  das moderne Oeſterreich\oindex{Oesterreich@\textbf{Österreich}, \emph{A.PCLI}|pw} kennzeichnen. Sie
                  haben etwas den Fran\oindex{Frankreich@\textbf{Frankreich}, \emph{A.PCLI}|pwv}zoſen
                  Verwandtes. Wie dieſe ſind ſie Plauderer, vor allem hat Schnitzler die Grazie der
                  Form. Eine weiche Müdigkeit liegt über ſeinen Schöpfungen, von denen jede ein
                  Stück Selbſtbiographie iſt. »Einen leichtſinnigen Melancholiker\pwindex{Weihnachts-Einkaeufe@\emph{Weihnachts-Einkäufe}|pwv}« nennt er ſich einmal darin. Er liebt
                  die matten, feinen, ſubtilen Farben. Der nüchterne Verſtandesmenſch nennt ihn
                  leicht \so{weibiſch}, aber er iſt nur ſenſitiv. Allerdings,
                  die großen, neuen Probleme gehen ihn nichts an, ſeine Dichtungen haben nur einen
                  Inhalt: \so{die Frau}, aber nicht die ringende, kämpfende,
                  nur die liebende. Seine Heldinnen ſind immer die kleinen, ſüßen Mädel der Wien\oindex{Wien@\textbf{Wien}, \emph{A.ADM2}|pw}er Vorſtadt oder verheirathete Weltdamen,
                  die Troſt für ihre Herzensleere im Bruch der ehelichen Treue ſuchen.}}\pend
           
\pstart
           \textcolor{gray}{\textbf{Es iſt ein Inſtrument mit einer Saite, das Schnitzler ſpielt,
                  aber er weiß ihm ſympathische Klänge von wehmüthigem Reiz zu entlocken. Auch wenn
                  er das Intimſte erzählt, bleibt er immer graziös und wird nie unzüchtig. Mit
                  ſeinen erſten Arbeiten trat Schnitzler 1886 hervor. Es
                  war das Märchen »Alcantils
                     Lied\pwindex{Alkandi s Lied@\emph{Alkandi’s Lied}|pwv}«, dann folgte das »Märchen von den Gefallenen\pwindex{Maerchen. Schauspiel in drei Aufzuegen@\emph{Das Märchen. Schauspiel in drei Aufzügen}|pwv}«, in dem der Held alle alten Vorurtheile
                  überwunden hat und ihnen doch beim erſten Verſuch in der Praxis unterliegt. Das
                  Drama »Freiwild\pwindex{Freiwild. Schauspiel in 3 Akten@\emph{Freiwild. Schauspiel in 3 Akten}|pw}« behandelt das Duellmotiv in
                  einem meiſterhaft geſchilderten Milieu. Nun folgte »Liebelei\pwindex{Liebelei. Schauspiel in drei Akten@\emph{Liebelei. Schauspiel in drei Akten}|pw}«, die Tragödie des Mädchens aus dem Volke,
                  vielleicht des Mädchens überhaupt. Es begründete Schnitzlers Ruf und wurde in die
                  verſchiedenſten Sprachen überſetzt. Das folgende »Vermächtniß\pwindex{Vermaechtnis. Schauspiel in drei Akten@\emph{Das Vermächtnis. Schauspiel in drei Akten}|pw}« iſt ein ſchwaches Stück, »Die Gefährtin\pwindex{Gefaehrtin. Schauspiel in einem Akt@\emph{Die Gefährtin. Schauspiel in einem Akt}|pw}« dagegen voll Feinheit und Eleganz. In »Paracelſus\pwindex{Paracelsus. Versspiel in einem Akt@\emph{Paracelsus. Versspiel in einem Akt}|pw}« ſind die Farben etwas ſtark aufgetragen,
                  großen Bühnenerfolg hatte die ſozialpolitiſche Burleske »Der grüne Kakadu\pwindex{gruene Kakadu. Groteske in einem Akt@\emph{Der grüne Kakadu. Groteske in einem Akt}|pw}«, die trotz der hiſtoriſchen Maske völlig
                  modern wirkt. Schnitzlers neueſtes, noch nicht aufgeführtes Stück nennt ſich »Beatrice\pwindex{Schleier der Beatrice. Schauspiel in fuenf Akten@\emph{Der Schleier der Beatrice. Schauspiel in fünf Akten}|pw}« und iſt in Verſen geſchrieben. Ein
                  Mittelding zwiſchen Buch und Bühne iſt ſein »Anatol\pwindex{Anatol@\emph{Anatol}|pw}«, ein Meiſterſtück genialer Plauderei, während ſeine »Novellen\pwindex{Frau des Weisen. Novelletten@\emph{Die Frau des Weisen. Novelletten}|pwv}« das Problem des
                  Sterbens, des Loslöſens des Lebenden von dem dem Tode Verfallenen, ergreifend
                  ſchildern. Leichtſinn und Melancholie, beides weiß Schnitzler zu verklären, der
                  vielleicht kein Unſterblicher, aber ein echter Künſtler iſt. Zum Schluß las Adele Schreiber\pwindex{Schreiber, Adele 1872-04-29 – 1957-02-20@\textsc{Schreiber, Adele} (1872-04-29 – 1957-02-20), \emph{Schriftsteller/Schriftstellerin, Politiker/Politikerin, Pädagoge/Pädagogin}|pw} drei ſeiner lyriſchen
                  Gedichte und die Szene »Weihnachtseinkäufe\pwindex{Weihnachts-Einkaeufe@\emph{Weihnachts-Einkäufe}|pw}«
                  aus »Anatol\pwindex{Anatol@\emph{Anatol}|pw}« vor, und der Beifall, den ſie
                  fand, bewies, daß ihre graziöſe, gleichgeſtimmte Art das Weſen ihres Land\oindex{Oesterreich@\textbf{Österreich}, \emph{A.PCLI}|pwv}smannes den Hörern
                  wirklich näher gebracht hatte, obgleich wir Norddeutſch\oindex{Deutschland@\textbf{Deutschland}, \emph{A.PCLI}|pwv}en mehr die friſche, klare Morgenluft lieben als den
                  düſteſchweren Hauch ſchwüler Sommernächte voll banger Todesſehnſucht.}}\pend
           \selectlanguage{ngerman}\endnumbering\briefempfaengerindex{Schnitzler, Arthur@\textsc{Schnitzler, Arthur}!zzzGoldmann, Paul@\emph{von Paul Goldmann}!1900-03-291@{29. 3. {[}1900{]}}|)be}\mylabel{L02909h}  \normalsize

\doendnotes{C}
\bigskip
\vfill

\clearpage

\footnotesize

\lohead{\textsc{register}}

% Definiere theindex-Environment komplett neu ohne reledmac
\makeatletter
\renewenvironment{theindex}{%
  \section*{\indexname}%
  \setlength{\parindent}{0pt}%
  \setlength{\parskip}{0pt plus 0.3pt}%
  \let\item\@idxitem
}{%
  \clearpage
}
\makeatother

\IfFileExists{\jobname-pw.ind}{\input{\jobname-pw.ind}}{}

\end{document}

      