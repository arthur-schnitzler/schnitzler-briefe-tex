%% latex-leseansicht-vorspann.tex
%% Vorspann für die Leseansicht.
%% Lädt die gemeinsame Datei latex-vorspann.tex mit nicht gesetztem Schalter.

\newif\ifkorrekturansicht
\korrekturansichtfalse

\input{../tex-inputs/latex-vorspann}


\section[Arthur Schnitzler an Hermann Bahr, 27. 9. 1910]{L01959 Arthur Schnitzler an Hermann Bahr, 27. 9. 1910}
\nopagebreak\mylabel{L01959v}
\rehead{ }\normalsize\beginnumbering\briefempfaengerindex{Bahr, Hermann@\textsc{Bahr, Hermann}!zzzSchnitzler, Arthur@\emph{von Arthur Schnitzler}!1910-09-271@{27. 9. 1910}|(be}
\toendnotes[C]{\smallbreak\pagebreak[2]}
\correspDesc{Versand  durch Arthur Schnitzler am 27. 9. 1910 in Wien
\newline{}Erhalt  durch Hermann Bahr im Zeitraum [27. 9. 1910
                  – 1. 10. 1910?] in Wien}\toendnotes[C]{\smallbreak}
\Standort{TMW, HS AM 23391 Ba.}
\physDesc{Brief, 1 Blatt, 3 Seiten, 797 Zeichen
\newline{}Handschrift: 1) schwarze Tinte, deutsche Kurrent\hspace{1em}2) roter Buntstift (\noindent{}Umrahmung des gedruckten Briefkopfs mit der handschriftlichen
                                 Adresskorrektur)\hspace{1em}
\newline{}Ordnung: Lochung }
\buchAbdrucke{\weitereDrucke{1) \emph{27. 9. 1910.} In: Arthur Schnitzler: \emph{The Letters of Arthur Schnitzler to Hermann Bahr}. Edited, annotated, and with an introduction, by Donald G. Daviau. Chapel Hill: \emph{The University of North Carolina Press} 1978, S. 106 (University of North Carolina studies in the Germanic languages
                        and literatures, 89).} \weitereDrucke{2) Hermann Bahr, Arthur Schnitzler: \emph{Briefwechsel, Aufzeichnungen, Dokumente (1891–1931)}. Herausgegeben von Kurt Ifkovits und Martin Anton Müller. Göttingen: \emph{Wallstein} 2018, S. 438.} }\toendnotes[C]{\smallbreak}
\pstart
           {\pb}\textcolor{gray}{\textbf{Dr. Arthur Schnitzler}}\hfill 27/9. 910\pend
           
\pstart
           \textcolor{gray}{\textbf{Wien XVIII.\oindex{XVIII., Währing@\textbf{XVIII., Währing}, \emph{Verwaltungsgebiet}|pw}}}{ }\substVorne{}\textsuperscript{\textcolor{gray}{\textbf{Spoettelgasse 7\oindex{Wien@\textbf{Wien}!XVIII., Währing@\textbf{XVIII., Währing}!Edmund-Weiß-Gasse 7@\textbf{Edmund-Weiß-Gasse 7}, \emph{Wohngebäude}|pw}.}}}\substDazwischen{}\textsc{Sternwartestr 71.}\substHinten{}\pend
           
\pstart{}mein lieber Hermann,\pend\vspace{0.5em}
\pstart
           wie die Dinge{ }ſtehn, dürfte der \textsc{\damage{M}edardus\pwindex{Schnitzler, Arthur 15.\,5.\,1862 Wien – 21.\,10.\,1931 ebd.@\textsc{Schnitzler, Arthur} (15.\,5.\,1862 Wien – 21.\,10.\,1931 ebd.), \emph{Schriftsteller, Mediziner}!junge Medardus. Dramatische Historie in einem Vorspiel und fünf Aufzügen@\strich\emph{Der junge Medardus. Dramatische Historie in einem Vorspiel und fünf Aufzügen}|pw}} gerade Anfang November, alſo zur Zeit, da du wieder für einige
               Tage oder Wochen in Wien\oindex{Wien@\textbf{Wien}, \emph{Verwaltungsgebiet}|pw} biſt, aufgeführt werden.
               Mir wird es \damage{ſe}hr lieb{ }ſein, wenn du das Stück auf der Bühne{ }ſiehſt, wo es hingehört, wie
               noch{ }ſelten was von mir hingehört hat. Aber da {\pb}ich bald \uline{fertige} Bühnenmanuſcripte kriege,{ }ſchicke ich \strikeout{dich} dir{ }ſehr gern ein Exemplar nach London\oindex{London@\textbf{London}, \emph{Hauptstadt}|pw}, und wünſche, daſs es dich bei guter Laune u\damage{nd} Geſundheit dort antrifft (nicht um des Stückes willen.)\pend
           
\pstart
           Geſtern traf dein neuer \label{K_L01959-1v}\edtext{Roman\pwindex{Bahr, Hermann 19.\,7.\,1863 Linz – 15.\,1.\,1934 München@\textsc{Bahr, Hermann} (19.\,7.\,1863 Linz – 15.\,1.\,1934 München), \emph{Schriftsteller, Kritiker}!O Mensch@\strich\emph{O Mensch{\rufezeichen}}|pwv}}{\lemma{\textnormal{\emph{Roman}}}\Cendnote{\textnormal{Hermann Bahr\pwindex{Bahr, Hermann 19.\,7.\,1863 Linz – 15.\,1.\,1934 München@\textsc{Bahr, Hermann} (19.\,7.\,1863 Linz – 15.\,1.\,1934 München), \emph{Schriftsteller, Kritiker}|pwk}: \emph{O Mensch. Roman}\pwindex{Bahr, Hermann 19.\,7.\,1863 Linz – 15.\,1.\,1934 München@\textsc{Bahr, Hermann} (19.\,7.\,1863 Linz – 15.\,1.\,1934 München), \emph{Schriftsteller, Kritiker}!O Mensch@\strich\emph{O Mensch{\rufezeichen}}|pwk}. Berlin: \emph{S. Fischer}\orgindex{S. Fischer Verlag@S. Fischer Verlag|pwk}{ }1910.}}}\label{K_L01959-1} von \textsc{S. Fischer\orgindex{S. Fischer Verlag@S. Fischer Verlag|pw}} bei mir ein. Ich freu mich{ }ſehr darauf. Hab mich diesmal \label{K_L01959-2v}\edtext{zurückgehalten, auch nur einen \uline{Blick} in die \textsc{N. Fr. Pr.\orgindex{Neue Freie Presse@Neue Freie Presse|pw}} zu thun}{\lemma{\textnormal{\emph{zurückgehalten, … thun}}}\Cendnote{\textnormal{Der Vorabdruck von \emph{O Mensch}\pwindex{Bahr, Hermann 19.\,7.\,1863 Linz – 15.\,1.\,1934 München@\textsc{Bahr, Hermann} (19.\,7.\,1863 Linz – 15.\,1.\,1934 München), \emph{Schriftsteller, Kritiker}!O Mensch@\strich\emph{O Mensch{\rufezeichen}}|pwk} erschien vom 31. 5. 1910
                  bis zum 4. 9. 1910 in der \emph{Neuen Freien
                     Presse}\orgindex{Neue Freie Presse@Neue Freie Presse|pwk}.}}}\label{K_L01959-2}.\pend
           
\pstart
           {\pb}Dich und deine Frau\pwindex{Bahr-Mildenburg, Anna 29.\,11.\,1872 Wien – 27.\,1.\,1947 ebd.@\textsc{Bahr-Mildenburg, Anna} (29.\,11.\,1872 Wien – 27.\,1.\,1947 ebd.), \emph{Sängerin}|pwv} endlich einmal bei uns
               zu begrüßen,{ }ſoll uns eine{ }ſchöne Winterhoffnung{ }ſein.\pend
           
\pstart
           Herzlichſt dein{\\[\baselineskip]}\spacefill\mbox{Arthur}\pend
           \leftskip=0em{}\selectlanguage{ngerman}\endnumbering\briefempfaengerindex{Bahr, Hermann@\textsc{Bahr, Hermann}!zzzSchnitzler, Arthur@\emph{von Arthur Schnitzler}!1910-09-271@{27. 9. 1910}|)be}\mylabel{L01959h}  \newcommand{\dateiname}{L01959}\newcommand{\titel}{Arthur Schnitzler an Hermann Bahr, 27. 9. 1910}\newcommand{\editorInnen}{Herausgegeben von Martin Anton Müller}%% latex-leseansicht-abspann.tex
%% Abspann für die Leseansicht.
%% Der Schalter \ifkorrekturansicht ist bereits durch den Vorspann gesetzt.

%% latex-abspann.tex
%% Gemeinsamer Abspann für Korrekturansicht und Leseansicht.
%% Setzt den Schalter \ifkorrekturansicht voraus (gesetzt in den
%% einbindenden Dateien latex-korrekturansicht-abspann.tex bzw.
%% latex-leseansicht-abspann.tex).
%% ---------------------------------------------------------------

\normalsize

% Das esempio-Environment wird nur in der Leseansicht benötigt
\ifkorrekturansicht\else
\newenvironment{esempio}[3]%
{
    \vspace{1.5ex}
    \rlap{\underline{#1}}
    \par
    \setlength{\parindent}{0cm}
    \nopagebreak
    \leftskip=#2cm
    \rightskip=#3cm
}
{
    \par
}
\fi

\doendnotes{C}
\bigskip
\vfill

\clearpage

\footnotesize

\ifkorrekturansicht
  \lohead{\textsc{register}}
\fi

% theindex-Environment neu definieren ohne reledmac
\makeatletter
\renewenvironment{theindex}{%
  \ifkorrekturansicht
    \section*{\indexname}%
  \else
    \subsubsection*{Index der erwähnten Entitäten}%
  \fi
  \setlength{\parindent}{0pt}%
  \setlength{\parskip}{0pt plus 0.3pt}%
  \let\item\@idxitem
}{%
  \ifkorrekturansicht\clearpage\fi
}
\makeatother

\IfFileExists{\jobname-pw.ind}{\input{\jobname-pw.ind}}{}

% Quellenangabe nur in der Leseansicht
\ifkorrekturansicht\else
% Fallback-Definitionen, falls die .tex-Datei \titel etc. nicht gesetzt hat
\providecommand{\titel}{}
\providecommand{\editorInnen}{}
\providecommand{\dateiname}{\jobname}

\vspace{3cm}

\vfill

\footnotesize
\textsc{Quelle}: \titel. Herausgegeben von {\editorInnen}. In: \emph{Arthur Schnitzler: Briefwechsel mit Autorinnen und Autoren}.
 Digitale Edition, https://schnitzler-briefe.acdh.oeaw.ac.at/{\dateiname}.html (Stand \today)
\fi

\end{document}


