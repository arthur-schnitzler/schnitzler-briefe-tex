%% latex-korrekturansicht-vorspann.tex
%% Vorspann für die Korrekturansicht.
%% Lädt die gemeinsame Datei latex-vorspann.tex mit gesetztem Schalter.

\newif\ifkorrekturansicht
\korrekturansichttrue

\input{../tex-inputs/latex-vorspann}


\section[Arthur Schnitzler an Hermann Bahr, 27. 9. 1910]{L01959 Arthur Schnitzler an Hermann Bahr, 27. 9. 1910}
\nopagebreak\mylabel{L01959v}
\rehead{ }\normalsize\beginnumbering\briefempfaengerindex{Bahr, Hermann@\textsc{Bahr, Hermann}!zzzSchnitzler, Arthur@\emph{von Arthur Schnitzler}!1910-09-271@{27. 9. 1910}|(be}
\toendnotes[C]{\smallbreak\pagebreak[2]}\Standort{TMW, HS AM 23391 Ba.}
\physDesc{Brief, 1 Blatt, 3 Seiten, 797 Zeichen
\newline{}Handschrift: 1) schwarze Tinte, deutsche Kurrent\hspace{1em}2) roter Buntstift (\noindent{}Umrahmung des gedruckten Briefkopfs mit der handschriftlichen
                                 Adresskorrektur)\hspace{1em}
\newline{}Ordnung: Lochung }
\buchAbdrucke{\weitereDrucke{1) Arthur Schnitzler: \emph{The Letters of Arthur Schnitzler to Hermann Bahr}. Chapel Hill: \emph{The University of North Carolina Press} 1978, S. 106.} \weitereDrucke{2) Hermann Bahr, Arthur Schnitzler: \emph{Briefwechsel, Aufzeichnungen, Dokumente (1891–1931)}. Göttingen: \emph{Wallstein} 2018, S. 438.} }\toendnotes[C]{\smallbreak}
\pstart
           {\pb}\textcolor{gray}{\textbf{Dr. Arthur Schnitzler}}\hfill 27/9. 910\pend
           
\pstart
           \textcolor{gray}{\textbf{Wien XVIII.\oindex{XVIII., Waehring@\textbf{XVIII., Währing}, \emph{A.ADM3}|pw}}}{ }\substVorne{}\textsuperscript{\textcolor{gray}{\textbf{Spoettelgasse 7\oindex{Edmund-Weiss-Gasse 7@\textbf{Edmund-Weiß-Gasse 7}, \emph{Wohngebäude (K.WHS)}|pw}.}}}\substDazwischen{}\textsc{Sternwartestr 71.}\substHinten{}\pend
           
\pstart{}mein lieber Hermann,\pend\vspace{0.5em}
\pstart
           wie die Dinge ſtehn, dürfte der \textsc{\damage{M}edardus\pwindex{junge Medardus. Dramatische Historie in einem Vorspiel und fuenf Aufzuegen@\emph{Der junge Medardus. Dramatische Historie in einem Vorspiel und fünf Aufzügen}|pw}} gerade Anfang November, alſo zur Zeit, da du wieder für einige
               Tage oder Wochen in Wien\oindex{Wien@\textbf{Wien}, \emph{A.ADM2}|pw} biſt, aufgeführt werden.
               Mir wird es \damage{ſe}hr lieb ſein, wenn du das Stück auf der Bühne ſiehſt, wo es hingehört, wie
               noch ſelten was von mir hingehört hat. Aber da {\pb}ich bald \uline{fertige} Bühnenmanuſcripte kriege, ſchicke ich \strikeout{dich} dir ſehr gern ein Exemplar nach London\oindex{London@\textbf{London}, \emph{P.PPLC}|pw}, und wünſche, daſs es dich bei guter Laune u\damage{nd} Geſundheit dort antrifft (nicht um des Stückes willen.)\pend
           
\pstart
           Geſtern traf dein neuer \label{K_L01959-1v}\edtext{Roman\pwindex{O Mensch@\emph{O Mensch{\rufezeichen}}|pwv}}{\lemma{\textnormal{\emph{Roman}}}\Cendnote{\textnormal{Hermann Bahr\pwindex{Bahr, Hermann 19.07.1863 – 15.01.1934@\textsc{Bahr, Hermann} (19.07.1863 – 15.01.1934), \emph{Schriftsteller/Schriftstellerin, Kritiker/Kritikerin}|pwk}: \emph{O Mensch. Roman}\pwindex{O Mensch@\emph{O Mensch{\rufezeichen}}|pwk}. Berlin: \emph{S. Fischer}\orgindex{S. Fischer Verlag@S. Fischer Verlag|pwk}{ }1910.}}}\label{K_L01959-1} von \textsc{S. Fischer\orgindex{S. Fischer Verlag@S. Fischer Verlag|pw}} bei mir ein. Ich freu mich ſehr darauf. Hab mich diesmal \label{K_L01959-2v}\edtext{zurückgehalten, auch nur einen \uline{Blick} in die \textsc{N. Fr. Pr.\orgindex{Neue Freie Presse@Neue Freie Presse|pw}} zu thun}{\lemma{\textnormal{\emph{zurückgehalten, … thun}}}\Cendnote{\textnormal{Der Vorabdruck von \emph{O Mensch}\pwindex{O Mensch@\emph{O Mensch{\rufezeichen}}|pwk} erschien vom 31. 5. 1910
                  bis zum 4. 9. 1910 in der \emph{Neuen Freien
                     Presse}\orgindex{Neue Freie Presse@Neue Freie Presse|pwk}.}}}\label{K_L01959-2}.\pend
           
\pstart
           {\pb}Dich und deine Frau\pwindex{Bahr-Mildenburg, Anna 29.11.1872 – 27.01.1947@\textsc{Bahr-Mildenburg, Anna} (29.11.1872 – 27.01.1947), \emph{Sänger/Sängerin}|pwv} endlich einmal bei uns
               zu begrüßen, ſoll uns eine ſchöne Winterhoffnung ſein.\pend
           
\pstart
           Herzlichſt dein{\\[\baselineskip]}\spacefill\mbox{Arthur}\pend
           \leftskip=0em{}\selectlanguage{ngerman}\endnumbering\briefempfaengerindex{Bahr, Hermann@\textsc{Bahr, Hermann}!zzzSchnitzler, Arthur@\emph{von Arthur Schnitzler}!1910-09-271@{27. 9. 1910}|)be}\mylabel{L01959h}  \normalsize

\doendnotes{C}
\bigskip
\vfill

\clearpage

\footnotesize

\lohead{\textsc{register}}

% Definiere theindex-Environment komplett neu ohne reledmac
\makeatletter
\renewenvironment{theindex}{%
  \section*{\indexname}%
  \setlength{\parindent}{0pt}%
  \setlength{\parskip}{0pt plus 0.3pt}%
  \let\item\@idxitem
}{%
  \clearpage
}
\makeatother

\IfFileExists{\jobname-pw.ind}{\input{\jobname-pw.ind}}{}

\end{document}

      