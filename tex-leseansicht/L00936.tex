%% latex-leseansicht-vorspann.tex
%% Vorspann für die Leseansicht.
%% Lädt die gemeinsame Datei latex-vorspann.tex mit nicht gesetztem Schalter.

\newif\ifkorrekturansicht
\korrekturansichtfalse

\input{../tex-inputs/latex-vorspann}


\section[Hugo von Hofmannsthal an Arthur Schnitzler, 7. 7. [1899]]{L00936 Hugo von Hofmannsthal an Arthur Schnitzler, 7. 7. [1899]}
\nopagebreak\mylabel{L00936v}
\rehead{ }\normalsize\beginnumbering\briefempfaengerindex{Schnitzler, Arthur@\textsc{Schnitzler, Arthur}!zzzHofmannsthal, Hugo von@\emph{von Hugo von Hofmannsthal}!1899-07-071@{7. 7. [1899]}|(be}
\toendnotes[C]{\smallbreak\pagebreak[2]}
\correspDesc{Versand  durch Hugo von Hofmannsthal am 7. 7. [1899] in Marienbad
\newline{}Erhalt  durch Arthur Schnitzler im Zeitraum [8. 7. 1899
                  – 12. 7. 1899?] in Wien}\toendnotes[C]{\smallbreak}
\Standort{CUL, Schnitzler, B 43.}
\physDesc{Brief, 1 Blatt, 4 Seiten, 975 Zeichen
\newline{}Handschrift: schwarze Tinte, deutsche Kurrent
\newline{}Schnitzler: mit Bleistift die Jahreszahl ergänzt: »99« 
\newline{}Ordnung: 1) mit Bleistift von unbekannter Hand nummeriert:
                                    »150«  2) mit Bleistift von unbekannter Hand nummeriert: »\strikeout{153}«}
\buchAbdrucke{\weitereDrucke{1) Hugo von Hofmannsthal: \emph{Briefe. 1890–1901}. Berlin: \emph{S. Fischer} 1935, S. 286–287.} \weitereDrucke{2) Hugo von Hofmannsthal, Arthur Schnitzler: \emph{Briefwechsel}. Herausgegeben von Therese Nickl und Heinrich Schnitzler. Frankfurt am Main: \emph{S. Fischer} 1964, S. 124–125.} }\toendnotes[C]{\smallbreak}
\pstart
           \raggedleft{}{\pb}7 VII.\pend
           \vspace{0.5em}
\pstart
           Bin{ }ſehr froh endlich zu wiſſen, wo Sie{ }ſind, denn{ }ſelbſt darüber in Ungewiſsheit zu{ }ſein, iſt peinlich. Von Richard\pwindex{Beer-Hofmann, Richard 11.\,7.\,1866 Wien – 26.\,9.\,1945 New York City@\textsc{Beer-Hofmann, Richard} (11.\,7.\,1866 Wien – 26.\,9.\,1945 New York City), \emph{Schriftsteller}|pw} hab ich nach
               wie vor keine Zeile.\pend
           
\pstart
           Der »Zeit\orgindex{Zeit. Wiener Wochenschrift@Die Zeit. Wiener Wochenschrift|pw}«{ }ſtelle ich meinen Namen in
               unverbindlicher Weiſe natürlich gern zur Verfügung. Habe an einem Stück\pwindex{Hofmannsthal, Hugo von 1.\,2.\,1874 Wien – 15.\,7.\,1929 Rodaun@\textsc{Hofmannsthal, Hugo von} (1.\,2.\,1874 Wien – 15.\,7.\,1929 Rodaun), \emph{Schriftsteller}!Bergwerk zu Falun@\strich\emph{Das Bergwerk zu Falun}|pwv} (5 Acte, in Verſen) {\pb}zu arbeiten begonnen, bin aber
               gleich in den Anfängen durch ganz unglaubliches deprimierendes Wetter gehemmt
               worden.\pend
           
\pstart
           Bleibe wohl bis gegen Ende July hier und werde dann, hoffentlich mitten
               in der Arbeit, wohl nach Salzburg\oindex{Salzburg@\textbf{Salzburg}, \emph{Verwaltungsgebiet}|pw} überſiedeln.
               Gegen Ende Auguſt hoffe ich die innere {\pb}und äußere Möglichkeit zu einer
               kleinen deutſchen\oindex{Deutschland@\textbf{Deutschland}|pw} Tour zu finden.\pend
           
\pstart
           Minnie\pwindex{Schaffgotsch, Hermine von 25.\,11.\,1871 Wien – 25.\,11.\,1928 Purgstall@\textsc{Schaffgotsch, Hermine von} (25.\,11.\,1871 Wien – 25.\,11.\,1928 Purgstall)|pw}{ }ſehe ich ungefähr täglich ¼ – ½ Stunde. Das
               Geſpräch entfernt{ }ſich nie vom peinlich-banalen. Sie thut mir recht leid. Es kommt
               etwas tief Freudloſes und Bitteres in ihr Weſen.\hspace*{1.5em}Sind Sie wenigſtens {\pb}einigermaßen im Stand{ }ſich mit Stück\pwindex{Schnitzler, Arthur 15.\,5.\,1862 Wien – 21.\,10.\,1931 ebd.@\textsc{Schnitzler, Arthur} (15.\,5.\,1862 Wien – 21.\,10.\,1931 ebd.), \emph{Schriftsteller, Mediziner}!Schleier der Beatrice. Schauspiel in fünf Akten@\strich\emph{Der Schleier der Beatrice. Schauspiel in fünf Akten}|pwv} oder Novelle\pwindex{Schnitzler, Arthur 15.\,5.\,1862 Wien – 21.\,10.\,1931 ebd.@\textsc{Schnitzler, Arthur} (15.\,5.\,1862 Wien – 21.\,10.\,1931 ebd.), \emph{Schriftsteller, Mediziner}!Nächste@\strich\emph{Die Nächste}|pwv} zu
               beſchäftigen?\pend
           
\pstart
           Herzlich Ihr{\\[\baselineskip]}\spacefill\mbox{Hugo.}\pend
           \leftskip=0em{}
\pstart
           \noindent{}\textsc{P. S.}{ }\uline{Giebt} es ein Leben zweiter oder dritter Ordnung?
                  Auf die Dauer doch wohl kaum.\pend
           \selectlanguage{ngerman}\endnumbering\briefempfaengerindex{Schnitzler, Arthur@\textsc{Schnitzler, Arthur}!zzzHofmannsthal, Hugo von@\emph{von Hugo von Hofmannsthal}!1899-07-071@{7. 7. [1899]}|)be}\mylabel{L00936h}  \newcommand{\dateiname}{L00936}\newcommand{\titel}{Hugo von Hofmannsthal an Arthur Schnitzler, 7. 7. [1899]}\newcommand{\editorInnen}{Martin Anton Müller und Gerd-Hermann Susen}%% latex-leseansicht-abspann.tex
%% Abspann für die Leseansicht.
%% Der Schalter \ifkorrekturansicht ist bereits durch den Vorspann gesetzt.

%% latex-abspann.tex
%% Gemeinsamer Abspann für Korrekturansicht und Leseansicht.
%% Setzt den Schalter \ifkorrekturansicht voraus (gesetzt in den
%% einbindenden Dateien latex-korrekturansicht-abspann.tex bzw.
%% latex-leseansicht-abspann.tex).
%% ---------------------------------------------------------------

\normalsize

% Das esempio-Environment wird nur in der Leseansicht benötigt
\ifkorrekturansicht\else
\newenvironment{esempio}[3]%
{
    \vspace{1.5ex}
    \rlap{\underline{#1}}
    \par
    \setlength{\parindent}{0cm}
    \nopagebreak
    \leftskip=#2cm
    \rightskip=#3cm
}
{
    \par
}
\fi

\doendnotes{C}
\bigskip
\vfill

\clearpage

\footnotesize

\ifkorrekturansicht
  \lohead{\textsc{register}}
\fi

% theindex-Environment neu definieren ohne reledmac
\makeatletter
\renewenvironment{theindex}{%
  \ifkorrekturansicht
    \section*{\indexname}%
  \else
    \subsubsection*{Index der erwähnten Entitäten}%
  \fi
  \setlength{\parindent}{0pt}%
  \setlength{\parskip}{0pt plus 0.3pt}%
  \let\item\@idxitem
}{%
  \ifkorrekturansicht\clearpage\fi
}
\makeatother

\IfFileExists{\jobname-pw.ind}{\input{\jobname-pw.ind}}{}

% Quellenangabe nur in der Leseansicht
\ifkorrekturansicht\else
% Fallback-Definitionen, falls die .tex-Datei \titel etc. nicht gesetzt hat
\providecommand{\titel}{}
\providecommand{\editorInnen}{}
\providecommand{\dateiname}{\jobname}

\vspace{3cm}

\vfill

\footnotesize
\textsc{Quelle}: \titel. Herausgegeben von {\editorInnen}. In: \emph{Arthur Schnitzler: Briefwechsel mit Autorinnen und Autoren}.
 Digitale Edition, https://schnitzler-briefe.acdh.oeaw.ac.at/{\dateiname}.html (Stand \today)
\fi

\end{document}


