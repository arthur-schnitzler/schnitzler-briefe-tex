%% latex-leseansicht-vorspann.tex
%% Vorspann für die Leseansicht.
%% Lädt die gemeinsame Datei latex-vorspann.tex mit nicht gesetztem Schalter.

\newif\ifkorrekturansicht
\korrekturansichtfalse

\input{../tex-inputs/latex-vorspann}


         
         \renewcommand{\erwaehntePersonen}{Personen: Richard Beer-Hofmann, Hermine von Schaffgotsch}
         \renewcommand{\erwaehnteInstitutionen}{Institutionen: Die Zeit. Wiener Wochenschrift}
         \renewcommand{\erwaehnteOrte}{Orte: Deutschland, Marienbad, Salzburg, Wien}
         \renewcommand{\erwaehnteWerke}{Werke: Das Bergwerk zu Falun, Der Schleier der Beatrice. Schauspiel in fünf Akten, Die Nächste}
               \section[Hugo von Hofmannsthal an Arthur Schnitzler, 7. 7. {[}1899{]}]{ Hugo von Hofmannsthal an Arthur Schnitzler, 7. 7. {[}1899{]}}\nopagebreak\mylabel{v}\rehead{ }\begin{ledgroupsized}[t]{13cm}\normalsize\beginnumbering \toendnotes[C]{\smallbreak\pagebreak[2]} \Standort{CUL, Schnitzler, B 43.}
\physDesc{Brief, 1 Blatt, 4 Seiten
\newline{}Handschrift: schwarze Tinte, deutsche Kurrent
\newline{}Schnitzler: mit Bleistift die Jahreszahl ergänzt: »99« \newline{}Ordnung: 1) mit Bleistift von unbekannter Hand nummeriert:
                                        »150«  2) mit Bleistift von unbekannter Hand nummeriert: »\strikeout{153}«}\buchAbdrucke{\weitereDrucke{1) Hugo von Hofmannsthal: \emph{Briefe. 1890–1901}. Berlin: \emph{S. Fischer} 1935, S. 286–287.} \weitereDrucke{2) Hugo von Hofmannsthal, Arthur Schnitzler: \emph{Briefwechsel}. Hg. Therese Nickl und Heinrich Schnitzler. Frankfurt am Main: \emph{S. Fischer} 1964, S. 124–125.} }\toendnotes[C]{\smallbreak}\pstart
           \raggedleft{}{\pb}7 VII.\pend
           \pstart
           Bin ſehr froh endlich zu wiſſen, wo Sie ſind, denn ſelbſt darüber in Ungewiſsheit
                    zu ſein, iſt peinlich. Von Richard\pwindex{Beer-Hofmann, Richard 1866-07-11 – 1945-09-26@\textsc{Beer-Hofmann, Richard} (1866-07-11 – 1945-09-26), \emph{Schriftsteller}|pw} hab ich
                    nach wie vor keine Zeile.\pend
           \pstart
           Der »Zeit\orgindex{Zeit. Wiener Wochenschrift@Die Zeit. Wiener Wochenschrift|pw}« ſtelle ich meinen Namen in
                    unverbindlicher Weiſe natürlich gern zur Verfügung. Habe an einem Stück\pwindex{Hofmannsthal, Hugo von 1874-02-01 – 1929-07-15@\textsc{Hofmannsthal, Hugo von} (1874-02-01 – 1929-07-15), \emph{Schriftsteller}!Bergwerk zu Falun1900 – 1933@\strich\emph{Das Bergwerk zu Falun} {[}1900 – 1933{]}|pwv} (5 Acte, in Verſen) {\pb}zu arbeiten begonnen, bin
                    aber gleich in den Anfängen durch ganz unglaubliches deprimierendes Wetter
                    gehemmt worden.\pend
           \pstart
           Bleibe wohl bis gegen Ende July hier und werde dann, hoffentlich
                    mitten in der Arbeit, wohl nach Salzburg\oindex{Salzburg@\textbf{Salzburg}|pw}
                    überſiedeln. Gegen Ende Auguſt hoffe ich die innere {\pb}und äußere Möglichkeit zu
                    einer kleinen deutſchen\oindex{Deutschland@\textbf{Deutschland}|pw} Tour zu finden.\pend
           \pstart
           Minnie\pwindex{Schaffgotsch, Hermine von 25.11.1871 – 25.11.1928@\textsc{Schaffgotsch, Hermine von} (25.11.1871 – 25.11.1928)|pw}{ }ſehe ich ungefähr täglich ¼ –
                    ½ Stunde. Das Geſpräch entfernt ſich nie vom peinlich-banalen. Sie thut mir
                    recht leid. Es kommt etwas tief Freudloſes und Bitteres in ihr Weſen.\hspace*{1.5em}Sind Sie wenigſtens {\pb}einigermaßen im Stand ſich
                    mit Stück\pwindex{Schnitzler, Arthur 15.05.1862 – 21.10.1931@\textsc{Schnitzler, Arthur} (15.05.1862 – 21.10.1931), \emph{Schriftsteller, Mediziner}!Schleier der Beatrice. Schauspiel in fuenf Akten1900-12-01@\strich\emph{Der Schleier der Beatrice. Schauspiel in fünf Akten} {[}1900-12-01{]}|pwv} oder Novelle\pwindex{Schnitzler, Arthur 15.05.1862 – 21.10.1931@\textsc{Schnitzler, Arthur} (15.05.1862 – 21.10.1931), \emph{Schriftsteller, Mediziner}!Naechste1899@\strich\emph{Die Nächste} {[}1899{]}|pwv}\pwindex{Schnitzler, Arthur 15.05.1862 – 21.10.1931@\textsc{Schnitzler, Arthur} (15.05.1862 – 21.10.1931), \emph{Schriftsteller, Mediziner}!Naechste1899@\strich\emph{Die Nächste} {[}1899{]}|pwv} zu beſchäftigen?\pend
           \pstart
           Herzlich Ihr{\\[\baselineskip]}\spacefill\mbox{Hugo.}\pend
           \leftskip=0em{}\pstart
           \noindent{}\textsc{P. S.}{ }\uline{Giebt} es ein Leben zweiter oder dritter
                        Ordnung? Auf die Dauer doch wohl kaum.\pend
           
         
         \endnumbering\mylabel{h}\end{ledgroupsized}  \newcommand{\dateiname}{L00936}\newcommand{\titel}{Hugo von Hofmannsthal an Arthur Schnitzler, 7. 7. [1899]}\newcommand{\editorInnen}{Martin Anton Müller und Gerd-Hermann Susen}%% latex-leseansicht-abspann.tex
%% Abspann für die Leseansicht.
%% Der Schalter \ifkorrekturansicht ist bereits durch den Vorspann gesetzt.

%% latex-abspann.tex
%% Gemeinsamer Abspann für Korrekturansicht und Leseansicht.
%% Setzt den Schalter \ifkorrekturansicht voraus (gesetzt in den
%% einbindenden Dateien latex-korrekturansicht-abspann.tex bzw.
%% latex-leseansicht-abspann.tex).
%% ---------------------------------------------------------------

\normalsize

% Das esempio-Environment wird nur in der Leseansicht benötigt
\ifkorrekturansicht\else
\newenvironment{esempio}[3]%
{
    \vspace{1.5ex}
    \rlap{\underline{#1}}
    \par
    \setlength{\parindent}{0cm}
    \nopagebreak
    \leftskip=#2cm
    \rightskip=#3cm
}
{
    \par
}
\fi

\doendnotes{C}
\bigskip
\vfill

\clearpage

\footnotesize

\ifkorrekturansicht
  \lohead{\textsc{register}}
\fi

% theindex-Environment neu definieren ohne reledmac
\makeatletter
\renewenvironment{theindex}{%
  \ifkorrekturansicht
    \section*{\indexname}%
  \else
    \subsubsection*{Index der erwähnten Entitäten}%
  \fi
  \setlength{\parindent}{0pt}%
  \setlength{\parskip}{0pt plus 0.3pt}%
  \let\item\@idxitem
}{%
  \ifkorrekturansicht\clearpage\fi
}
\makeatother

\IfFileExists{\jobname-pw.ind}{\input{\jobname-pw.ind}}{}

% Quellenangabe nur in der Leseansicht
\ifkorrekturansicht\else
% Fallback-Definitionen, falls die .tex-Datei \titel etc. nicht gesetzt hat
\providecommand{\titel}{}
\providecommand{\editorInnen}{}
\providecommand{\dateiname}{\jobname}

\vspace{3cm}

\vfill

\footnotesize
\textsc{Quelle}: \titel. Herausgegeben von {\editorInnen}. In: \emph{Arthur Schnitzler: Briefwechsel mit Autorinnen und Autoren}.
 Digitale Edition, https://schnitzler-briefe.acdh.oeaw.ac.at/{\dateiname}.html (Stand \today)
\fi

\end{document}


      