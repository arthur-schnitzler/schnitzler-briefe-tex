%% latex-leseansicht-vorspann.tex
%% Vorspann für die Leseansicht.
%% Lädt die gemeinsame Datei latex-vorspann.tex mit nicht gesetztem Schalter.

\newif\ifkorrekturansicht
\korrekturansichtfalse

\input{../tex-inputs/latex-vorspann}


         
         \newcommand{\erwaehntePersonen}{Personen: Hugo von Hofmannsthal, Ferdinand Miliczek}
         \newcommand{\erwaehnteInstitutionen}{}
         \newcommand{\erwaehnteOrte}{Orte: Altaussee, Bad Ischl, Gasthaus Brunnthaler}
         \newcommand{\erwaehnteWerke}{
               \section[Arthur Schnitzler an Hugo von Hofmannsthal, 8. 9. 1899]{ Arthur Schnitzler an Hugo von Hofmannsthal, 8. 9. 1899}\nopagebreak\mylabel{v}\rehead{ }\begin{ledgroupsized}[t]{13cm}\normalsize\beginnumbering \toendnotes[C]{\smallbreak\pagebreak[2]} \Standort{FDH, Hs-30885,86.}
\physDesc{Postkarte
\newline{}Handschrift: Bleistift, deutsche Kurrent\newline{}Versand: 1) Stempel: »\nobreak{}\oindex{Bad Ischl@\textbf{Bad Ischl}|pwk}Ischl, 8. 9. 99, 8–9N\nobreak{}«.   2) Stempel: »\nobreak{}\oindex{Gasthaus Brunnthaler@\textbf{Gasthaus Brunnthaler}|pwk}Alt-Aussee, 9 9 99\nobreak{}«. \newline{}Ordnung: von Schnitzler mit Bleistift mutmaßlich bei der Durchsicht der
                                 Briefe 1929 datiert: »9/9 99« }\buchAbdrucke{\weitereDrucke{Hugo von Hofmannsthal, Arthur Schnitzler: \emph{Briefwechsel}. Hg. Therese Nickl und Heinrich Schnitzler. Frankfurt am Main: \emph{S. Fischer} 1964, S. 130.} }\toendnotes[C]{\smallbreak}\pstart{}{\pb}Herrn \textsc{Hugo v Hofmannsthal}\pend{}\pstart{}\textsc{Altaussee\oindex{Gasthaus Brunnthaler@\textbf{Gasthaus Brunnthaler}|pw}}\pend{}\pstart{}\textsc{Brunthaler}s Gaſthaus\oindex{Gasthaus Brunnthaler@\textbf{Gasthaus Brunnthaler}|pw}\pend{}{\bigskip}\pstart
           \noindent{}{\pb}lieber, bin eben auf der Bahn, habe Stationschef\pwindex{Miliczek, Ferdinand @\textsc{Miliczek, Ferdinand}, \emph{Stationsvorsteher}|pwv} geſprochen, der ſofort Träger 1 rufen lieſs,
               welch letzterer sich \uline{abſolut}{ }\uline{nicht} an Ihr Futteral erinnern will. Auch \uline{gefunden} wurde es nicht. – Wohl in ein fremdes \textsc{Coupé} gerathen? –\pend
           \pstart
           Ich werde wahrſcheinlich So{\geminationn}tag{ }Mittag bei \textsc{Brunthaler}\oindex{Gasthaus Brunnthaler@\textbf{Gasthaus Brunnthaler}|pw}{ }ſein. Herzlich Ihr \spacefill\mbox{A. S.}\pend
           
         
         \endnumbering\mylabel{h}\end{ledgroupsized}  \newcommand{\dateiname}{L00970}\newcommand{\titel}{Arthur Schnitzler an Hugo von Hofmannsthal, 8. 9. 1899}\newcommand{\editorInnen}{Martin Anton Müller und Gerd-Hermann Susen}%% latex-leseansicht-abspann.tex
%% Abspann für die Leseansicht.
%% Der Schalter \ifkorrekturansicht ist bereits durch den Vorspann gesetzt.

%% latex-abspann.tex
%% Gemeinsamer Abspann für Korrekturansicht und Leseansicht.
%% Setzt den Schalter \ifkorrekturansicht voraus (gesetzt in den
%% einbindenden Dateien latex-korrekturansicht-abspann.tex bzw.
%% latex-leseansicht-abspann.tex).
%% ---------------------------------------------------------------

\normalsize

% Das esempio-Environment wird nur in der Leseansicht benötigt
\ifkorrekturansicht\else
\newenvironment{esempio}[3]%
{
    \vspace{1.5ex}
    \rlap{\underline{#1}}
    \par
    \setlength{\parindent}{0cm}
    \nopagebreak
    \leftskip=#2cm
    \rightskip=#3cm
}
{
    \par
}
\fi

\doendnotes{C}
\bigskip
\vfill

\clearpage

\footnotesize

\ifkorrekturansicht
  \lohead{\textsc{register}}
\fi

% theindex-Environment neu definieren ohne reledmac
\makeatletter
\renewenvironment{theindex}{%
  \ifkorrekturansicht
    \section*{\indexname}%
  \else
    \subsubsection*{Index der erwähnten Entitäten}%
  \fi
  \setlength{\parindent}{0pt}%
  \setlength{\parskip}{0pt plus 0.3pt}%
  \let\item\@idxitem
}{%
  \ifkorrekturansicht\clearpage\fi
}
\makeatother

\IfFileExists{\jobname-pw.ind}{\input{\jobname-pw.ind}}{}

% Quellenangabe nur in der Leseansicht
\ifkorrekturansicht\else
% Fallback-Definitionen, falls die .tex-Datei \titel etc. nicht gesetzt hat
\providecommand{\titel}{}
\providecommand{\editorInnen}{}
\providecommand{\dateiname}{\jobname}

\vspace{3cm}

\vfill

\footnotesize
\textsc{Quelle}: \titel. Herausgegeben von {\editorInnen}. In: \emph{Arthur Schnitzler: Briefwechsel mit Autorinnen und Autoren}.
 Digitale Edition, https://schnitzler-briefe.acdh.oeaw.ac.at/{\dateiname}.html (Stand \today)
\fi

\end{document}


      