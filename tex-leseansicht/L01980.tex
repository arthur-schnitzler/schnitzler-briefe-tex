%% latex-leseansicht-vorspann.tex
%% Vorspann für die Leseansicht.
%% Lädt die gemeinsame Datei latex-vorspann.tex mit nicht gesetztem Schalter.

\newif\ifkorrekturansicht
\korrekturansichtfalse

\input{../tex-inputs/latex-vorspann}


         
         \renewcommand{\erwaehntePersonen}{Personen: Hermann Bahr, Anna Bahr-Mildenburg, Lisa Clarus, Alfred Gerasch, Olga Schnitzler}
         \renewcommand{\erwaehnteInstitutionen}{Institutionen: Neues Wiener Journal}
         \renewcommand{\erwaehnteOrte}{Orte: Burgtheater, London, Ober Sankt Veit, Wien}
         \renewcommand{\erwaehnteWerke}{Werke: Der junge Medardus. Dramatische Historie in einem Vorspiel und fünf Aufzügen, Fremden-Blatt}
               \section[Hermann Bahr an Arthur Schnitzler, 15. 11. 1910]{ Hermann Bahr an Arthur Schnitzler, 15. 11. 1910}\nopagebreak\mylabel{v}\rehead{ }\begin{ledgroupsized}[t]{13cm}\normalsize\beginnumbering\briefempfaengerindex{Schnitzler, Arthur@\textsc{Schnitzler, Arthur}!zzzBahr, Hermann@\emph{von Hermann Bahr}!1910-11-151@{15. 11. 1910}|(be} \toendnotes[C]{\smallbreak\pagebreak[2]} \Standort{CUL, Schnitzler, B 5b.}
\physDesc{Brief, 2 Blätter, 5 Seiten, 2661 Zeichen
\newline{}Handschrift Lisa Clarus: schwarze Tinte, lateinische Kurrent\newline{}Handschrift Hermann Bahr: schwarze Tinte, deutsche Kurrent (\noindent{}Unterschrift)
\newline{}Schnitzler: mit Bleistift ergänzt »Bahr« 
\newline{}Ordnung: mit Bleistift von unbekannter Hand nummeriert:
                                    »169« }\buchAbdrucke{\weitereDrucke{Hermann Bahr, Arthur Schnitzler: \emph{Briefwechsel, Aufzeichnungen, Dokumente (1891–1931)}. Hg. Kurt Ifkovits und Martin Anton Müller. Göttingen: \emph{Wallstein} 2018, S. 442–443.} }\toendnotes[C]{\smallbreak}\pstart
           \raggedleft{}{\pb}Wien XIII/\textsubscript{7}\oindex{Ober Sankt Veit@\textbf{Ober Sankt Veit}|pw}\hspace*{2.5em}15. 11. 10.\pend
           \pstart\center{}Lieber Arthur!\pend\pstart
           Ich habe jeden Tag zu Dir kommen wollen, nie wars möglich, nun muss ich morgen wieder
               auf Vorlesungen fort, bis zum 5. Dezember.\pend
           \pstart
           Lass mich Dir also kurz schreiben, was ich Dir lieber ausführlich gesagt hätte: warum
               ich nämlich den Gedanken aufgegeben habe, über Dein Stück\pwindex{Schnitzler, Arthur 15.05.1862 – 21.10.1931@\textsc{Schnitzler, Arthur} (15.05.1862 – 21.10.1931), \emph{Schriftsteller, Mediziner}!junge Medardus. Dramatische Historie in einem Vorspiel und fuenf Aufzuegen1910-10-26@\strich\emph{Der junge Medardus. Dramatische Historie in einem Vorspiel und fünf Aufzügen} {[}1910-10-26{]}|pwv} im Neuen W\textsuperscript{r} Journal\orgindex{Neues Wiener Journal@Neues Wiener Journal|pw}, unabhängig von der ersten Aufführung,
               bei der ich ja leider nicht sein kann, zu sprechen.\pend
           \pstart
           Ich habs in London\oindex{London@\textbf{London}|pw} gleich gelesen, und {\pb}dann hier noch einmal. Beide Male war der Eindruck
               der selbe. Ich habe mich sehr stark für den Medardus\pwindex{Schnitzler, Arthur 15.05.1862 – 21.10.1931@\textsc{Schnitzler, Arthur} (15.05.1862 – 21.10.1931), \emph{Schriftsteller, Mediziner}!junge Medardus. Dramatische Historie in einem Vorspiel und fuenf Aufzuegen1910-10-26@\strich\emph{Der junge Medardus. Dramatische Historie in einem Vorspiel und fünf Aufzügen} {[}1910-10-26{]}|pwv} selbst interessiert, der mir, kein halber, sondern
               ein sechzehntelheld, eben darin ein vollkommenes Exempel des Wieners\oindex{Wien@\textbf{Wien}|pw} zu sein scheint. Wie es aussieht, wenn ein
                  Wiener\oindex{Wien@\textbf{Wien}|pw} zur tragischen Figur wird, das finde ich
               an diesem Fall wunderbar dargestellt. Allerdings ist das Missverständnis möglich, der
               Autor habe selbst einen tragischen Helden zeichnen wollen. Ich glaube das nicht und
               werde darin durch die Schilderung der anderen Wiener\oindex{Wien@\textbf{Wien}|pw} im Stück bekräftigt. Diese Schilderung hat freilich erst dann auf
               mich gewirkt, als ich mir die Mühe nahm, {\pb}das Stück\pwindex{Schnitzler, Arthur 15.05.1862 – 21.10.1931@\textsc{Schnitzler, Arthur} (15.05.1862 – 21.10.1931), \emph{Schriftsteller, Mediziner}!junge Medardus. Dramatische Historie in einem Vorspiel und fuenf Aufzuegen1910-10-26@\strich\emph{Der junge Medardus. Dramatische Historie in einem Vorspiel und fünf Aufzügen} {[}1910-10-26{]}|pwv} im Geiste sozusagen zu
               inszenieren und es mir Szene für Szene auf der Bühne vorzustellen. Ich rechne ihm das
               als einen Vorzug an, es ist ein durchaus bühnenmässiges Stück\pwindex{Schnitzler, Arthur 15.05.1862 – 21.10.1931@\textsc{Schnitzler, Arthur} (15.05.1862 – 21.10.1931), \emph{Schriftsteller, Mediziner}!junge Medardus. Dramatische Historie in einem Vorspiel und fuenf Aufzuegen1910-10-26@\strich\emph{Der junge Medardus. Dramatische Historie in einem Vorspiel und fünf Aufzügen} {[}1910-10-26{]}|pwv}, das dargestellt noch ganz anders
               wirken muss als aus dem Buch. Wenn es nämlich wirklich dargestellt wird, wenn es
               bühnenmässig gelöst wird! Und da kam nun, als ich die \label{K_L01980-1v}\edtext{Besetzung las}{\lemma{\textnormal{\emph{Besetzung las}}}\Cendnote{\textnormal{Das
                     \emph{Fremden-Blatt}\pwindex{?? Werk@Nicht ermittelte Verfasserinnen und Verfasser!Fremden-Blatt1.7.1847 – 22.3.1919@\emph{Fremden-Blatt} {[}1.7.1847 – 22.3.1919{]}|pwk} (Jg. 64, Nr. 305,
                     Morgen-Blatt, S. 18) führt am 6. 11. 1910 nur an, welche
                     Burgschauspieler\oindex{Burgtheater@\textbf{Burgtheater}|pwk}{ }\emph{nicht} im Schauspiel auftreten werden.}}}\label{K_L01980-1h}, meine
               Hauptsorge. Ich würde herzlich wünschen, dass ich mich völlig irre. Wie ich aber
               diese Herrschaften, die jetzt im Burgtheater\oindex{Burgtheater@\textbf{Burgtheater}|pw}
               herumdillettieren, und die dortigen hilflosen Inszenierungen kenne, muss ich
               fürchten, dass sie aus Deinem Stück\pwindex{Schnitzler, Arthur 15.05.1862 – 21.10.1931@\textsc{Schnitzler, Arthur} (15.05.1862 – 21.10.1931), \emph{Schriftsteller, Mediziner}!junge Medardus. Dramatische Historie in einem Vorspiel und fuenf Aufzuegen1910-10-26@\strich\emph{Der junge Medardus. Dramatische Historie in einem Vorspiel und fünf Aufzügen} {[}1910-10-26{]}|pwv} eine Karikatur machen {\pb}werden. Wäre
               ich nun selbst bei der Première, so könnte ich schreiben: Das was ihr gestern gesehen
               habt, war gar nicht Schnitzlers
                  Stück\pwindex{Schnitzler, Arthur 15.05.1862 – 21.10.1931@\textsc{Schnitzler, Arthur} (15.05.1862 – 21.10.1931), \emph{Schriftsteller, Mediziner}!junge Medardus. Dramatische Historie in einem Vorspiel und fuenf Aufzuegen1910-10-26@\strich\emph{Der junge Medardus. Dramatische Historie in einem Vorspiel und fünf Aufzügen} {[}1910-10-26{]}|pwv}, sondern sein Stück ist vielmehr so und so! Da ich selbst nicht dabei
               bin, könnte ich nur schreiben: Das Stück\pwindex{Schnitzler, Arthur 15.05.1862 – 21.10.1931@\textsc{Schnitzler, Arthur} (15.05.1862 – 21.10.1931), \emph{Schriftsteller, Mediziner}!junge Medardus. Dramatische Historie in einem Vorspiel und fuenf Aufzuegen1910-10-26@\strich\emph{Der junge Medardus. Dramatische Historie in einem Vorspiel und fünf Aufzügen} {[}1910-10-26{]}|pwv} ist so und so! Aber die Leute, die das lesen würden, werden, fürchte
               ich, ein ganz anderes Stück gesehen haben. Ich würde über den Medardus\pwindex{Schnitzler, Arthur 15.05.1862 – 21.10.1931@\textsc{Schnitzler, Arthur} (15.05.1862 – 21.10.1931), \emph{Schriftsteller, Mediziner}!junge Medardus. Dramatische Historie in einem Vorspiel und fuenf Aufzuegen1910-10-26@\strich\emph{Der junge Medardus. Dramatische Historie in einem Vorspiel und fünf Aufzügen} {[}1910-10-26{]}|pwv} schreiben, sie aber werden den
               Herrn Gerasch\pwindex{Gerasch, Alfred 17.08.1877 – 12.08.1955@\textsc{Gerasch, Alfred} (17.08.1877 – 12.08.1955), \emph{Schauspieler}|pw} sehen und ich fürchte, dass
               zwischen diesen beiden Personen jede Ähnlichkeit ausgeschlossen ist. Zum Schluss wäre
               wahrscheinlich der arme Herr Kollege, der die Notiz über die Darstellung und die
               Aufnahme schreiben müsste, völlig ratlos und würde noch gegen mich polemisieren {\pb}müssen. Ich sehe nur Unannehmlichkeiten für Dich
               und für mich und für alle Beteiligten. Ich hoffe, Du nimmst das so, wie es gemeint
               ist, und verstehst es.\pend
           \pstart
           Grüsse Deine liebe Frau\pwindex{Schnitzler, Olga 17.01.1882 – 13.01.1970@\textsc{Schnitzler, Olga} (17.01.1882 – 13.01.1970), \emph{Schauspielerin, Sängerin}|pwv}
               herzlichst und sei selbst herzlichst von uns\pwindex{Bahr-Mildenburg, Anna 29.11.1872 – 27.01.1947@\textsc{Bahr-Mildenburg, Anna} (29.11.1872 – 27.01.1947), \emph{Sängerin}|pwv} beiden gegrüsst!{\\[\baselineskip]}Dein{\\[\baselineskip]}alter{\\[\baselineskip]}\spacefill\mbox{{[}hs. Bahr:{]} Hermann}\pend
           \leftskip=0em{}
         
         \endnumbering\mylabel{h}\end{ledgroupsized}  \newcommand{\dateiname}{L01980}\newcommand{\titel}{Hermann Bahr an Arthur Schnitzler, 15. 11. 1910}\newcommand{\editorInnen}{ Kurt Ifkovits,  Martin Anton Müller}%% latex-leseansicht-abspann.tex
%% Abspann für die Leseansicht.
%% Der Schalter \ifkorrekturansicht ist bereits durch den Vorspann gesetzt.

%% latex-abspann.tex
%% Gemeinsamer Abspann für Korrekturansicht und Leseansicht.
%% Setzt den Schalter \ifkorrekturansicht voraus (gesetzt in den
%% einbindenden Dateien latex-korrekturansicht-abspann.tex bzw.
%% latex-leseansicht-abspann.tex).
%% ---------------------------------------------------------------

\normalsize

% Das esempio-Environment wird nur in der Leseansicht benötigt
\ifkorrekturansicht\else
\newenvironment{esempio}[3]%
{
    \vspace{1.5ex}
    \rlap{\underline{#1}}
    \par
    \setlength{\parindent}{0cm}
    \nopagebreak
    \leftskip=#2cm
    \rightskip=#3cm
}
{
    \par
}
\fi

\doendnotes{C}
\bigskip
\vfill

\clearpage

\footnotesize

\ifkorrekturansicht
  \lohead{\textsc{register}}
\fi

% theindex-Environment neu definieren ohne reledmac
\makeatletter
\renewenvironment{theindex}{%
  \ifkorrekturansicht
    \section*{\indexname}%
  \else
    \subsubsection*{Index der erwähnten Entitäten}%
  \fi
  \setlength{\parindent}{0pt}%
  \setlength{\parskip}{0pt plus 0.3pt}%
  \let\item\@idxitem
}{%
  \ifkorrekturansicht\clearpage\fi
}
\makeatother

\IfFileExists{\jobname-pw.ind}{\input{\jobname-pw.ind}}{}

% Quellenangabe nur in der Leseansicht
\ifkorrekturansicht\else
% Fallback-Definitionen, falls die .tex-Datei \titel etc. nicht gesetzt hat
\providecommand{\titel}{}
\providecommand{\editorInnen}{}
\providecommand{\dateiname}{\jobname}

\vspace{3cm}

\vfill

\footnotesize
\textsc{Quelle}: \titel. Herausgegeben von {\editorInnen}. In: \emph{Arthur Schnitzler: Briefwechsel mit Autorinnen und Autoren}.
 Digitale Edition, https://schnitzler-briefe.acdh.oeaw.ac.at/{\dateiname}.html (Stand \today)
\fi

\end{document}


      