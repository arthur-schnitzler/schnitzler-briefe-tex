%% latex-leseansicht-vorspann.tex
%% Vorspann für die Leseansicht.
%% Lädt die gemeinsame Datei latex-vorspann.tex mit nicht gesetztem Schalter.

\newif\ifkorrekturansicht
\korrekturansichtfalse

\input{../tex-inputs/latex-vorspann}

\begin{center}
            \textcolor{red}{ENTWURF, NICHT FERTIG KORRIGIERT}
                      \end{center}
            
         
         \renewcommand{\erwaehntePersonen}{Personen: Felix Salten}
         \renewcommand{\erwaehnteOrte}{Orte: Wien}
         \renewcommand{\erwaehnteWerke}{Werke: Jahrbuch Paul Zsolnay Verlag}
               \section[Arthur Schnitzler an Felix Salten, 29. 7. 1929]{ Arthur Schnitzler an Felix Salten, 29. 7. 1929}\nopagebreak\mylabel{v}\rehead{ }\begin{ledgroupsized}[t]{13cm}\normalsize\beginnumbering \toendnotes[C]{\smallbreak\pagebreak[2]} \Standort{DLA, A:Schnitzler, XXXX.}
\physDesc{Brief, 3 Blätter, 3 Seiten
\newline{}maschinell
\newline{}Handschrift: Bleistift (\noindent{}zwei marginale Korrekturen)}\Standort{DLA, A:Schnitzler, HS.NZ85.1.1751.}
\physDesc{Brief, Maschinenschriftliche Abschrift, 2 Blätter, 2 Seiten
\newline{}maschinell\newline{}Ordnung: 1) mit schwarzer Tinte Vermerk »Salten«  2) mit Bleistift Vermerk »6. 9. 1929«}\buchAbdrucke{\weitereDrucke{1) \pwindex{?? Werk@Nicht ermittelte Verfasserinnen und Verfasser!Jahrbuch Paul Zsolnay VerlagNone@\emph{Jahrbuch Paul Zsolnay Verlag} {[}None{]}|pwk}\emph{Jahrbuch Paul Zsolnay Verlag – 1930}. Berlin, Wien, Leipzig: \emph{Zsolnay} [November] 1929, S. 12–14.} \weitereDrucke{2) Arthur Schnitzler: \emph{Briefe 1913–1931}. Hg. Peter Michael Braunwarth, Richard Miklin, Susanne Pertlik und Heinrich Schnitzler. Frankfurt am Main: \emph{S. Fischer} 1984, S. 619–620.} }\toendnotes[C]{\smallbreak}\pstart{}{\pb}Mein lieber Felix Salten.\pend\pstart
           Am liebsten hätte ich Ihnen zu Ihrem \label{K_L02950-1v}\edtext{60. Geburtstag}{\lemma{\textnormal{\emph{60. Geburtstag}}}\Cendnote{\textnormal{Salten\pwindex{Salten, Felix 06.09.1869 – 08.10.1945@\textsc{Salten, Felix} (06.09.1869 – 08.10.1945), \emph{Schriftsteller, Journalist}|pwk} feierte am 6. 9. 1929
                  seinen 60. Geburtstag. Schnitzler\pwindex{Schnitzler, Arthur 15.05.1862 – 21.10.1931@\textsc{Schnitzler, Arthur} (15.05.1862 – 21.10.1931), \emph{Schriftsteller, Mediziner}|pwk}
                  finalisierte den Text am 29. 7. 1929. Der »Brief« erschien im \emph{Jahrbuch Paul Zsolnay Verlag}\pwindex{?? Werk@Nicht ermittelte Verfasserinnen und Verfasser!Jahrbuch Paul Zsolnay VerlagNone@\emph{Jahrbuch Paul Zsolnay Verlag} {[}None{]}|pwk} für das Jahr 1930, das ab
                     8. 11. 1929 lieferbar war. Die Druckfassung weicht an mehreren
                  Stellen von der hier präsentierten Fassung ab, die eindeutig die frühere Form
                  darstellt. Ob Salten\pwindex{Salten, Felix 06.09.1869 – 08.10.1945@\textsc{Salten, Felix} (06.09.1869 – 08.10.1945), \emph{Schriftsteller, Journalist}|pwk} bereits diese oder erst
                  die gedruckte Fassung zu sehen bekam, muss offen bleiben.}}}\label{K_L02950-1h} ganz privat und
               sehr herzlich die Hand gedrückt, Sie hätten dann ohneweiteres gewusst und empfunden,
               was ich hier niederzuschreiben \label{K_L02950-11v}\edtext{versuche}{\lemma{\textnormal{\emph{versuche}}}\Cendnote{\textnormal{In der Druckfassung
                     steht: »vergeblich versuchen werde«}}}\label{K_L02950-11h} – und etwas mehr.
               Denn bei einem solchen Anlass und gar vor \label{K_L02950-111v}\edtext{der Oeffentlichkeit}{\lemma{\textnormal{\emph{der Oeffentlichkeit}}}\Cendnote{\textnormal{In der Druckfassung steht: »mehr oder minder fremden
                     Leuten«}}}\label{K_L02950-111h} die rechten Worte zu finden ist nicht ganz leicht, zumal
               für Einen, der weder zum Essayisten noch zum Festredner geboren ist. \pend
           \pstart
           Ueber das, was man gemeiniglich Leistungen zu nennen pflegt, werden Ihnen \label{K_L02950-1111v}\edtext{die Berufenen allerlei}{\lemma{\textnormal{\emph{die Berufenen allerlei}}}\Cendnote{\textnormal{In der Druckfassung steht: »in
                     diesen Tagen Berufene nach Verdienst viel«}}}\label{K_L02950-1111h} Ehrenvolles zu sagen
               wissen; \label{K_L02950-2v}\edtext{mir ist jenseits }{\lemma{\textnormal{\emph{mir ist jenseits }}}\Cendnote{\textnormal{In der Druckfassung steht: »mir
                     persönlich ist jenseits all«}}}\label{K_L02950-2h} des Ausserordentlichen, was Sie als
               Dichter, Journalist und Schriftsteller \label{K_L02950-3v}\edtext{geleistet}{\lemma{\textnormal{\emph{geleistet}}}\Cendnote{\textnormal{In der Druckfassung steht:
                     »gewirkt«}}}\label{K_L02950-3h} haben, (dies ist eine alphabetische Reihenfolge
               und keine Klassifikation) \label{K_L02950-v}\edtext{Ihre
                  Persönlichkeit}{\lemma{\textnormal{\emph{Ihre
                  Persönlichkeit}}}\Cendnote{\textnormal{In der Druckfassung
                  steht: »vor allem des Gesamtbild Ihres Wesens«}}}\label{K_L02950-h} wert und
               bedeutungsvoll, \label{K_L02950-88v}\edtext{deren}{\lemma{\textnormal{\emph{deren}}}\Cendnote{\textnormal{In der Druckfassung steht:
                     »dessen«}}}\label{K_L02950-88h} Entwicklung seit frühesten Anfängen ich mit
               Spannung, Sympathie und Teilnahme nachbarlich mitangesehen und bis zum heutigen Tage
               als Freund begleitet habe. Einem \label{K_L02950-7v}\edtext{Mann
               wie Ihnen, angeregt von allen Seiten und anregend nach überallhin, erfüllt von der
               fruchtbarsten Neugier und zugleich von Interessen bewegt, die ins Umfassende und
               Tiefe gingen, Einfühler und Eindenker im besten Sinn und dabei eigensinnig und
               selbständig an Geist und Seele, der sich nach reichem äusseren und inneren Verdienst
               mit der Zeit so viele Bewunderer erwarb, konnte es natürlich auch an Gegnern nicht
                  fehlen}{\lemma{\textnormal{\emph{Mann … fehlen}}}\Cendnote{\textnormal{In der Druckfassung steht:
                     »Manne wie Sie, der, erfüllt von der fruchtbarsten Neugier und von der
                     dankbarsten Empfänglichkeit, angeregt von überallher, anregend in die Nähe und
                     in die Ferne, Einfühler und Eindenker im besten Sinn, und dabei eigenwillig und
                     selbständig wie Wenige, sich so viele Schätzer und Bewunderer erwarb, konnte es
                     natürlich auch nicht an Widersachern fehlen«}}}\label{K_L02950-7h};– welche Genugtuung
               muss es \label{K_L02950-12v}\edtext{Ihnen}{\lemma{\textnormal{\emph{Ihnen}}}\Cendnote{\textnormal{In der Druckfassung steht: »für Sie«}}}\label{K_L02950-12h}
               sein, wenn Sie heute an der Schwelle Ihrer dritten Jugend, in diesem Land der
               Missgunst und der Vorbehalte \label{K_L02950-888v}\edtext{Ihre
               vielseitige und immer lebendige Begabung}{\lemma{\textnormal{\emph{Ihre … Begabung}}}\Cendnote{\textnormal{In der Druckfassung steht: »sich sagen dürfen, dass
                     Ihre reiche, vielfältige und in jedem Augenblick lebendige
                  Begabung«}}}\label{K_L02950-888h} gegen manches nicht immer unabsichtliche \label{K_L02950-76v}\edtext{Missverstehen}{\lemma{\textnormal{\emph{Missverstehen}}}\Cendnote{\textnormal{In der Druckfassung steht: »Missverstehen
                     sich«}}}\label{K_L02950-76h} von Jahr zu Jahr in stets höherem Masse durchzusetzen
               vermochten. Sie stehen am Ziele – würde ich sagen, wenn ich nicht, \label{K_L02950-14v}\edtext{verwöhnt durch Ihre eigene Schuld}{\lemma{\textnormal{\emph{verwöhnt … Schuld}}}\Cendnote{\textnormal{In der Druckfassung steht: »durch
                     Ihre eigene Schuld verwöhnt«}}}\label{K_L02950-14h}, gerade nach {\pb}\label{K_L02950-90v}\edtext{Ihren Arbeits- und Lebensleistungen
                  der}{\lemma{\textnormal{\emph{Ihren … der}}}\Cendnote{\textnormal{In der Druckfassung steht:
                     »den Arbeits- und Lebensleistungen Ihrer«}}}\label{K_L02950-90h}
               letztvergangenen Jahre \label{K_L02950-66v}\edtext{ein}{\lemma{\textnormal{\emph{ein}}}\Cendnote{\textnormal{In der Druckfassung steht: »ein
                     immer«}}}\label{K_L02950-66h} Weiter- und Höherschreiten mit froher Gewissheit von
               Ihnen erwartete. \label{K_L02950-15v}\edtext{Aber ich}{\lemma{\textnormal{\emph{Aber ich}}}\Cendnote{\textnormal{In der Druckfassung steht:
                     »Ich«}}}\label{K_L02950-15h} will nichts prophezeien, so wenig diese \label{K_L02950-57v}\edtext{paar}{\lemma{\textnormal{\emph{paar}}}\Cendnote{\textnormal{In der Druckfassung steht:
                  »bescheidenen«}}}\label{K_L02950-57h} Worte als Rückblick gelten dürfen,– \label{K_L02950-64v}\edtext{ich will mich nur freuen}{\lemma{\textnormal{\emph{ich will mich nur freuen}}}\Cendnote{\textnormal{In der Druckfassung steht: »aber
                     freuen will ich mich«}}}\label{K_L02950-64h}, dass man Ihnen, mein lieber Freund, an
                  \label{K_L02950-112v}\edtext{einem solchen Festtag}{\lemma{\textnormal{\emph{einem solchen Festtag}}}\Cendnote{\textnormal{In der Druckfassung steht: »diesem
                     festlichen Tage«}}}\label{K_L02950-112h} in doppelter Hinsicht, den Blick sowohl in die
               Vergangenheit als \label{K_L02950-123v}\edtext{in die Zukunft
                  gewendet}{\lemma{\textnormal{\emph{in die Zukunft
                  gewendet}}}\Cendnote{\textnormal{In der Druckfassung steht:
                     »der Zukunft zugewendet«}}}\label{K_L02950-123h}, so vertrauensvoll und so von
               ganzem Herzen Glück wünschen kann. \pend
           
         
         \endnumbering\mylabel{h}\end{ledgroupsized}  \newcommand{\dateiname}{L02950}\newcommand{\titel}{Arthur Schnitzler an Felix Salten, 29. 7. 1929}\newcommand{\editorInnen}{Martin Anton Müller und Laura Untner}%% latex-leseansicht-abspann.tex
%% Abspann für die Leseansicht.
%% Der Schalter \ifkorrekturansicht ist bereits durch den Vorspann gesetzt.

%% latex-abspann.tex
%% Gemeinsamer Abspann für Korrekturansicht und Leseansicht.
%% Setzt den Schalter \ifkorrekturansicht voraus (gesetzt in den
%% einbindenden Dateien latex-korrekturansicht-abspann.tex bzw.
%% latex-leseansicht-abspann.tex).
%% ---------------------------------------------------------------

\normalsize

% Das esempio-Environment wird nur in der Leseansicht benötigt
\ifkorrekturansicht\else
\newenvironment{esempio}[3]%
{
    \vspace{1.5ex}
    \rlap{\underline{#1}}
    \par
    \setlength{\parindent}{0cm}
    \nopagebreak
    \leftskip=#2cm
    \rightskip=#3cm
}
{
    \par
}
\fi

\doendnotes{C}
\bigskip
\vfill

\clearpage

\footnotesize

\ifkorrekturansicht
  \lohead{\textsc{register}}
\fi

% theindex-Environment neu definieren ohne reledmac
\makeatletter
\renewenvironment{theindex}{%
  \ifkorrekturansicht
    \section*{\indexname}%
  \else
    \subsubsection*{Index der erwähnten Entitäten}%
  \fi
  \setlength{\parindent}{0pt}%
  \setlength{\parskip}{0pt plus 0.3pt}%
  \let\item\@idxitem
}{%
  \ifkorrekturansicht\clearpage\fi
}
\makeatother

\IfFileExists{\jobname-pw.ind}{\input{\jobname-pw.ind}}{}

% Quellenangabe nur in der Leseansicht
\ifkorrekturansicht\else
% Fallback-Definitionen, falls die .tex-Datei \titel etc. nicht gesetzt hat
\providecommand{\titel}{}
\providecommand{\editorInnen}{}
\providecommand{\dateiname}{\jobname}

\vspace{3cm}

\vfill

\footnotesize
\textsc{Quelle}: \titel. Herausgegeben von {\editorInnen}. In: \emph{Arthur Schnitzler: Briefwechsel mit Autorinnen und Autoren}.
 Digitale Edition, https://schnitzler-briefe.acdh.oeaw.ac.at/{\dateiname}.html (Stand \today)
\fi

\end{document}


      