%% latex-leseansicht-vorspann.tex
%% Vorspann für die Leseansicht.
%% Lädt die gemeinsame Datei latex-vorspann.tex mit nicht gesetztem Schalter.

\newif\ifkorrekturansicht
\korrekturansichtfalse

\input{../tex-inputs/latex-vorspann}


\section[ Arthur Schnitzler an Felix Salten, 29. 7. 1929]{L02950 Arthur Schnitzler an Felix Salten,  29. 7. 1929}
\nopagebreak\mylabel{L02950v}
\rehead{ }\normalsize\beginnumbering\briefempfaengerindex{Salten, Felix@\textsc{Salten, Felix}!zzzSchnitzler, Arthur@\emph{von Arthur Schnitzler}!1929-07-291@{29. 7. 1929}|(be}
\toendnotes[C]{\smallbreak\pagebreak[2]}
\correspDesc{Versand  durch Arthur Schnitzler am 29. 7. 1929 in Wien
\newline{}Erhalt  durch Felix Salten am [6. 9.?] 1929 in Wien}\toendnotes[C]{\smallbreak}
\Standort{Wienbibliothek im Rathaus, ZPH 1681/19, 4.1.2.14.}
\physDesc{Brief, 3 Blätter, 3 Seiten, 2368 Zeichen
\newline{}Schreibmaschine
\newline{}Handschrift: 1) schwarze Tinte (\noindent{}Schlussformel und Unterschrift)\hspace{1em}2) Bleistift, lateinische Kurrent (\noindent{}Korrekturen mit Bleistift)\hspace{1em}
\newline{}Ordnung: 1) mit Bleistift von unbekannter Hand über dem Text Vermerk: »ARTHUR SCHNITZLER«  2) mit Bleistift von unbekannter Hand in lateinischer Kurrentschrift seitlich
                                 neben der Unterschrift Vermerk: »\noindent{}NB: bleibt!{ / }in normaler Schrift, nicht gesperrt.« 3) mit Bleistift von unbekannter Hand in deutscher Kurrentschrift unterhalb der
                                 Unterschrift Vermerk: »\noindent{}Arthur Schnitzler«}\Standort{DLA, A:Schnitzler, HS.NZ85.1.1751.}
\physDesc{Brief, maschinenschriftliche Abschrift, 2 Blätter, 2 Seiten, 2368 Zeichen
\newline{}maschinell}
\buchAbdrucke{\weitereDrucke{1) \emph{Arthur Schnitzler.} In: \emph{Felix Salten, dem Freund und verehrten Autor zu seinem
                        60. Geburtstag mit herzlichen Glückwünschen überreicht vom Paul Zsolnay
                        Verlag}. Berlin, Wien, Leipzig: \emph{Zsolnay} 6. September 1929, S. 12–13.} \weitereDrucke{2) \pwindex{Jahrbuch Paul Zsolnay Verlag@\emph{Jahrbuch Paul Zsolnay Verlag}|pwk}\emph{Jahrbuch Paul Zsolnay Verlag – 1930}. Berlin, Wien, Leipzig: \emph{Zsolnay} [November] 1929, S. 12–14.} \weitereDrucke{3) Arthur Schnitzler: \emph{Briefe 1913–1931}. Herausgegeben von Peter Michael Braunwarth, Richard Miklin, Susanne Pertlik und Heinrich Schnitzler. Frankfurt am Main: \emph{S. Fischer} 1984, S. 619–620.} \weitereDrucke{4) Arthur Schnitzler: \emph{»Das
                        Zeitlose ist von kürzester Dauer«. Interviews, Meinungen, Proteste}. Herausgegeben von Martin Anton Müller. Göttingen: \emph{Wallstein} 2023, S. 537–538.} }\toendnotes[C]{\smallbreak}
\pstart{}{\pb}Mein lieber Felix Salten.\pend\vspace{0.5em}
\pstart
           Am liebsten hätte ich Ihnen zu Ihrem \label{K_L02950-1v}\edtext{sechzigsten Geburtstag}{\lemma{\textnormal{\emph{sechzigsten Geburtstag}}}\Cendnote{\textnormal{Salten\pwindex{Salten, Felix 6.\,9.\,1869 Budapest – 8.\,10.\,1945 Zürich@\textsc{Salten, Felix} (6.\,9.\,1869 Budapest – 8.\,10.\,1945 Zürich), \emph{Schriftsteller, Journalist, Chefredakteur}|pwk} feierte am 6. 9. 1929 seinen 60. Geburtstag, Schnitzler finalisierte den Text jedoch bereits am 29. 7. 1929, da er
                  gedruckt werden sollte. Der ›Brief\pwindex{Schnitzler, Arthur 15.\,5.\,1862 Wien – 21.\,10.\,1931 ebd.@\textsc{Schnitzler, Arthur} (15.\,5.\,1862 Wien – 21.\,10.\,1931 ebd.), \emph{Schriftsteller, Mediziner}!Mein lieber Felix Salten]@\strich\emph{[Mein lieber Felix Salten]}|pwkv}‹ erschien zuerst in einem Sonderdruck für den Jubilar\pwindex{Salten, Felix 6.\,9.\,1869 Budapest – 8.\,10.\,1945 Zürich@\textsc{Salten, Felix} (6.\,9.\,1869 Budapest – 8.\,10.\,1945 Zürich), \emph{Schriftsteller, Journalist, Chefredakteur}|pwkv} in Großformat und auf
                  Büttenpapier, und wenige Wochen später noch einmal im \emph{Jahrbuch Paul Zsolnay Verlag}\pwindex{Jahrbuch Paul Zsolnay Verlag@\emph{Jahrbuch Paul Zsolnay Verlag}|pwk} für das Jahr 1930, das ab 8. 11. 1929
                  lieferbar war (siehe A. S.: \emph{»Das Zeitlose ist von kürzester Dauer«}, [Mein lieber Felix Salten!], [November 1929]).
                  Eine Entwurfsfassung mit teilweise unterschiedlichen Formulierungen, jedoch ohne
                  inhaltlich bedeutsam abzuweichen, ist im DLA überliefert. Da Salten\pwindex{Salten, Felix 6.\,9.\,1869 Budapest – 8.\,10.\,1945 Zürich@\textsc{Salten, Felix} (6.\,9.\,1869 Budapest – 8.\,10.\,1945 Zürich), \emph{Schriftsteller, Journalist, Chefredakteur}|pwk} sich am XXXX Auszeichnungsfehler: Dokument L03587 nicht gefunden für das
                     »sozusagen öffentlich geäußerte Wort« bedankt, dürfte er sich auf den Büttendruck beziehen, aber auch die handschriftlich von Schnitzler signierte Fassung ist in Saltens\pwindex{Salten, Felix 6.\,9.\,1869 Budapest – 8.\,10.\,1945 Zürich@\textsc{Salten, Felix} (6.\,9.\,1869 Budapest – 8.\,10.\,1945 Zürich), \emph{Schriftsteller, Journalist, Chefredakteur}|pwk} Nachlass überliefert und wird hier als
                  Textgrundlage verwendet.}}}\label{K_L02950-1} ganz privat und sehr herzlich die Hand gedrückt;
               Sie hätten dann ohneweiters gewusst und empfunden, was ich hier niederzuschreiben
               vergeblich versuchen werde – und etwas mehr. Denn bei einem solchen Anlass und gar
               vor mehr oder minder fremden Leuten die rechten Worte zu finden, ist nicht ganz
               leicht, zumal für Einen, der weder zum Essayisten noch zum Festredner geboren
               ist.\pend
           
\pstart
           Ueber das, was man gemeiniglich Leistungen zu nennen pflegt, werden Ihnen in diesen
               Tagen Berufene nach Verdienst viel Ehrenvolles zu sagen wissen; mir persönlich ist
                  \strikeout{noch} jenseits 
               des Ausserordentlichen, was Sie als Dichter, Journalist und Schriftsteller gewirkt
               haben (dies ist eine alphabetische Reihenfolge und keine Klassifikation) {\pb}\introOben{}vor allem\introOben{} das Gesamtbild Ihres Wesens wert und
               bedeutungsvoll, dessen Entwicklung seit frühesten Anfängen ich mit Spannung,
               Sympathie und Teilnahme nachbarlich mitangesehen und bis zum heutigen Tage als Freund
               begleitet habe. Einem Manne, wie Sie, der, erfüllt von der fruchtbarsten Neugier und
               von der dankbarsten Empfänglichkeit, angeregt von überallher, anregend in die Nähe
               und in die Ferne, Einfühler und Eindenker in bestem Sinn, und dabei eigenwillig und
               selbstständig wie Wenige, sich so viele Schätzer und Bewunderer erwarb, konnte es
               natürlich auch nicht an Widersachern fehlen; – welche Genugtuung muss es für Sie sein,
               wenn Sie heute an der Schwelle Ihrer dritten Jugend, in diesem Land der Missgunst und
               der Vorbehalte sich sagen dürfen, dass Ihre reiche, vielfältige und in jedem
               Augen\label{T_L02950-1v}\edtext{blick lebendige Begabung}{\lemma{\textnormal{\emph{blick lebendige Begabung}}}\Cendnote{\textnormal{Die dritte Seite beginnt mit der Wiederholung von »blick lebendige Begabung«, was darauf hindeutet, 
               dass die zweite Seite neu getippt und der Anschluss der Seiten korrumpiert wurde.}}}\label{T_L02950-1}{\pb}gegen
               manches nicht immer unabsichtliche Missverstehen sich von Jahr zu Jahr in stets
               höherem Maasse durchzusetzen vermocht\substVorne{}\textsuperscript{e}\substDazwischen{}{ }hat\substHinten{}. Sie stehen am Ziele – würde ich sagen, wenn ich nicht, durch Ihre eigene
               Schuld verwöhnt, gerade nach den Arbeits- und Lebensleistungen Ihrer letztvergangenen
               Jahre ein immer Weiter- und Höherschreiten mit froher Gewissheit von Ihnen erwartete.
               Ich will nichts prophezeien, so wenig diese bescheidenen Worte als Rückblick gelten
               dürfen,– aber freuen  will ich mich, dass man
               Ihnen, mein lieber Freund, an diesem festlichen Tage in doppelter Hinsicht, den Blick
               sowohl in die Vergangenheit als der Zukunft \substVorne{}\textsuperscript{zugewendet}\substDazwischen{}zugewandt\substHinten{}, so vertrauensvoll und so von ganzem Herzen Glück wünschen kann.\pend
           
\pstart
           {[}hs.:{]} Ihr getreuer{\\[\baselineskip]}\spacefill\mbox{ArthurSchnitzler}\pend
           \leftskip=0em{}\selectlanguage{ngerman}\endnumbering\briefempfaengerindex{Salten, Felix@\textsc{Salten, Felix}!zzzSchnitzler, Arthur@\emph{von Arthur Schnitzler}!1929-07-291@{29. 7. 1929}|)be}\mylabel{L02950h}  \newcommand{\dateiname}{L02950}\newcommand{\titel}{Arthur Schnitzler an Felix Salten, 29. 7. 1929}\newcommand{\editorInnen}{Martin Anton Müller und Laura Untner}%% latex-leseansicht-abspann.tex
%% Abspann für die Leseansicht.
%% Der Schalter \ifkorrekturansicht ist bereits durch den Vorspann gesetzt.

%% latex-abspann.tex
%% Gemeinsamer Abspann für Korrekturansicht und Leseansicht.
%% Setzt den Schalter \ifkorrekturansicht voraus (gesetzt in den
%% einbindenden Dateien latex-korrekturansicht-abspann.tex bzw.
%% latex-leseansicht-abspann.tex).
%% ---------------------------------------------------------------

\normalsize

% Das esempio-Environment wird nur in der Leseansicht benötigt
\ifkorrekturansicht\else
\newenvironment{esempio}[3]%
{
    \vspace{1.5ex}
    \rlap{\underline{#1}}
    \par
    \setlength{\parindent}{0cm}
    \nopagebreak
    \leftskip=#2cm
    \rightskip=#3cm
}
{
    \par
}
\fi

\doendnotes{C}
\bigskip
\vfill

\clearpage

\footnotesize

\ifkorrekturansicht
  \lohead{\textsc{register}}
\fi

% theindex-Environment neu definieren ohne reledmac
\makeatletter
\renewenvironment{theindex}{%
  \ifkorrekturansicht
    \section*{\indexname}%
  \else
    \subsubsection*{Index der erwähnten Entitäten}%
  \fi
  \setlength{\parindent}{0pt}%
  \setlength{\parskip}{0pt plus 0.3pt}%
  \let\item\@idxitem
}{%
  \ifkorrekturansicht\clearpage\fi
}
\makeatother

\IfFileExists{\jobname-pw.ind}{\input{\jobname-pw.ind}}{}

% Quellenangabe nur in der Leseansicht
\ifkorrekturansicht\else
% Fallback-Definitionen, falls die .tex-Datei \titel etc. nicht gesetzt hat
\providecommand{\titel}{}
\providecommand{\editorInnen}{}
\providecommand{\dateiname}{\jobname}

\vspace{3cm}

\vfill

\footnotesize
\textsc{Quelle}: \titel. Herausgegeben von {\editorInnen}. In: \emph{Arthur Schnitzler: Briefwechsel mit Autorinnen und Autoren}.
 Digitale Edition, https://schnitzler-briefe.acdh.oeaw.ac.at/{\dateiname}.html (Stand \today)
\fi

\end{document}


