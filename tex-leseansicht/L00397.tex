%% latex-leseansicht-vorspann.tex
%% Vorspann für die Leseansicht.
%% Lädt die gemeinsame Datei latex-vorspann.tex mit nicht gesetztem Schalter.

\newif\ifkorrekturansicht
\korrekturansichtfalse

\input{../tex-inputs/latex-vorspann}


\section[Friedrich M. Fels an Arthur Schnitzler, 6. 11. 1894]{L00397 Friedrich M. Fels an Arthur Schnitzler, 6. 11. 1894}
\nopagebreak\mylabel{L00397v}
\rehead{ }\normalsize\beginnumbering\briefempfaengerindex{Schnitzler, Arthur@\textsc{Schnitzler, Arthur}!zzzFels, Friedrich Michael@\emph{von Friedrich Michael Fels}!1894-11-061@{6. 11. 1894}|(be}
\toendnotes[C]{\smallbreak\pagebreak[2]}
\correspDesc{Versand  durch Friedrich M. Fels am 6. 11. 1894 in Wien
\newline{}Erhalt  durch Arthur Schnitzler im Zeitraum [6. 11. 1894
                  – 10. 11. 1894?] in Wien}\toendnotes[C]{\smallbreak}
\Standort{DLA, A:Schnitzler, HS.NZ85.1.2956.}
\physDesc{Brief, 1 Blatt, 2 Seiten, 1626 Zeichen
\newline{}Handschrift: schwarze Tinte, lateinische Kurrent
\newline{}Schnitzler: 1) mit Bleistift nummeriert: »18«  2) mit rotem Buntstift eine Unterstreichung}
\buchAbdrucke{\weitereDrucke{Hermann Bahr, Arthur Schnitzler: \emph{Briefwechsel, Aufzeichnungen, Dokumente (1891–1931)}. Herausgegeben von Kurt Ifkovits und Martin Anton Müller. Göttingen: \emph{Wallstein} 2018, S. 86.} }\toendnotes[C]{\smallbreak}
\pstart
           \raggedleft{}{\pb}Wien XVIII, Gürtelstr. 90\oindex{Wien@\textbf{Wien}!IX., Alsergrund@\textbf{IX., Alsergrund}!Währinger Gürtel@\textbf{Währinger Gürtel}, \emph{Straße}|pw}\oindex{Wien@\textbf{Wien}!XVIII., Währing@\textbf{XVIII., Währing}!Währinger Gürtel@\textbf{Währinger Gürtel}, \emph{Straße}|pw}{\\}6. Nov. 94\pend
           
\pstart{}Lieber Doktor Schnitzler!\pend\vspace{0.5em}
\pstart
           Herma{\geminationn} Bahr\pwindex{Bahr, Hermann 19.\,7.\,1863 Linz – 15.\,1.\,1934 München@\textsc{Bahr, Hermann} (19.\,7.\,1863 Linz – 15.\,1.\,1934 München), \emph{Schriftsteller, Kritiker}|pw} hat
               den Artikel »Skandinavien in Deutschland\pwindex{Fels, Friedrich Michael *~1864 Bad Dürkheim@\textsc{Fels, Friedrich Michael} (*~1864 Bad Dürkheim), \emph{Journalist}!Skandinavien in Deutschland@\strich\emph{Skandinavien in Deutschland}|pw}«
               abgelehnt, weil er nicht aktuell genug sei und deshalb vor 3–4 Monaten nicht
               erscheinen kö{\geminationn}e. Da er selbstredend! gar nicht annahm,
               daſs ich so lange warten werde, habe ich auch nichts gesagt, obgleich ich herzlich
               froh gewesen wäre, we{\geminationn} er da{\geminationn} erschienen wäre; ich werde froh sein müſsen, we{\geminationn} er
               anderswo so bald erscheint. Aber man muſs den Leuten \introOben{}die\introOben{}
               Ausreden nicht zu schwer machen. Von Artikeln war keine Rede mehr; dagegen sagte Bahr\pwindex{Bahr, Hermann 19.\,7.\,1863 Linz – 15.\,1.\,1934 München@\textsc{Bahr, Hermann} (19.\,7.\,1863 Linz – 15.\,1.\,1934 München), \emph{Schriftsteller, Kritiker}|pw}, er werde mir Buchbesprechungen und zwar
               von literarhistorischen Werken – von andern verstehe ich wohl zu wenig – übertragen;
               ich nahm mit Dank an und habe nun die Hoffnung, we{\geminationn}s
               sehr gut geht, in einem Jahr drei Rezensionen schreiben zu dürfen und damit {\pb}5 fl zu verdienen. Hingehen werde ich wohl kaum mehr,
               da er, als ich gemeldet wurde, obgleich ich auf heute 4 Uhr von ihm bestellt war,
               laut aufseufzte und \textcolor{gray}{vernehmlich}{ }ſagte »So lassen Sie ihn in Gottes Namen
               herein.« –\pend
           
\pstart
           Den Artikel\pwindex{Fels, Friedrich Michael *~1864 Bad Dürkheim@\textsc{Fels, Friedrich Michael} (*~1864 Bad Dürkheim), \emph{Journalist}!Skandinavien in Deutschland@\strich\emph{Skandinavien in Deutschland}|pwv} werde ich morgen
               nach Berlin\oindex{Berlin@\textbf{Berlin}, \emph{Hauptstadt}|pw}{ }ſchicken, den beka{\geminationn}ten
               Weg: zuerst Zukunft\orgindex{Zukunft@Die Zukunft|pw}, da{\geminationn}{ }Nation\orgindex{Nation@Die Nation|pw}, da{\geminationn}{ }Tante Voss\orgindex{Vossische Zeitung@Vossische Zeitung|pwv}, da{\geminationn}{ }Gegenwart\orgindex{Gegenwart@Die Gegenwart|pw}, da{\geminationn}{ }{\dots} wer weiss, wohin noch. Den von \label{K_L00397-1v}\edtext{David\pwindex{David, Jakob Julius 6.\,2.\,1859 Hranice – 20.\,11.\,1906 Wien@\textsc{David, Jakob Julius} (6.\,2.\,1859 Hranice – 20.\,11.\,1906 Wien), \emph{Schriftsteller, Journalist}|pw}}{\lemma{\textnormal{\emph{David}}}\Cendnote{\textnormal{von der \emph{Wiener Allgemeinen Zeitung}\orgindex{Wiener Allgemeine Zeitung@Wiener Allgemeine Zeitung|pwk}}}}\label{K_L00397-1} refusierten \label{K_L00397-2v}\edtext{Sealsfield\pwindex{Sealsfield, Charles 3.\,3.\,1793 Popice – 26.\,5.\,1864 Solothurn@\textsc{Sealsfield, Charles} (3.\,3.\,1793 Popice – 26.\,5.\,1864 Solothurn), \emph{Schriftsteller}|pw}artikel}{\lemma{\textnormal{\emph{Sealsfieldartikel}}}\Cendnote{\textnormal{Der Text dürfte der Einleitung von Charles Sealsfield\pwindex{Sealsfield, Charles 3.\,3.\,1793 Popice – 26.\,5.\,1864 Solothurn@\textsc{Sealsfield, Charles} (3.\,3.\,1793 Popice – 26.\,5.\,1864 Solothurn), \emph{Schriftsteller}|pwk}: \emph{Das Kajütenbuch oder nationale Charakteristiken}\pwindex{Sealsfield, Charles 3.\,3.\,1793 Popice – 26.\,5.\,1864 Solothurn@\textsc{Sealsfield, Charles} (3.\,3.\,1793 Popice – 26.\,5.\,1864 Solothurn), \emph{Schriftsteller}!Kajütenbuch oder nationale Charakteristiken@\strich\emph{Das Kajütenbuch oder nationale Charakteristiken}|pwk}. Herausgegeben und
                     eingeleitet von Friedrich M. Fels\pwindex{Fels, Friedrich Michael *~1864 Bad Dürkheim@\textsc{Fels, Friedrich Michael} (*~1864 Bad Dürkheim), \emph{Journalist}|pwk}. Stuttgart:
                        \emph{Philipp Reclam Jun.}\orgindex{Philipp Reclam jun.@Philipp Reclam jun.|pwk} [o. J.] entsprechen.
               }}}\label{K_L00397-2} bringe ich \label{K_L00397-3v}\edtext{Uhl\pwindex{Uhl, Friedrich 14.\,5.\,1825 Cieszyn – 20.\,1.\,1906 Mondsee@\textsc{Uhl, Friedrich} (14.\,5.\,1825 Cieszyn – 20.\,1.\,1906 Mondsee), \emph{Journalist}|pw}}{\lemma{\textnormal{\emph{Uhl}}}\Cendnote{\textnormal{der \emph{Wiener Zeitung}\orgindex{Wiener Zeitung@Wiener Zeitung|pwk}}}}\label{K_L00397-3}, da{\geminationn}{ }\label{K_L00397-4v}\edtext{Pötzl\pwindex{Pötzl, Eduard 17.\,3.\,1851 Wien – 20.\,8.\,1914 Mödling@\textsc{Pötzl, Eduard} (17.\,3.\,1851 Wien – 20.\,8.\,1914 Mödling), \emph{Schriftsteller, Journalist}|pw}}{\lemma{\textnormal{\emph{Pötzl}}}\Cendnote{\textnormal{dem \emph{Neuen
                     Wiener Tagblatt}\orgindex{Neues Wiener Tagblatt@Neues Wiener Tagblatt|pwk}}}}\label{K_L00397-4}, da{\geminationn}{ }\label{K_L00397-5v}\edtext{Schönthan\pwindex{Schönthan-Pernwald, Paul von 19.\,3.\,1853 Wien – 4.\,8.\,1905 ebd.@\textsc{Schönthan-Pernwald, Paul von} (19.\,3.\,1853 Wien – 4.\,8.\,1905 ebd.), \emph{Schriftsteller, Journalist}|pw}}{\lemma{\textnormal{\emph{Schönthan}}}\Cendnote{\textnormal{dem \emph{Wiener Tagblatt}\orgindex{Wiener Tagblatt@Wiener Tagblatt|pwk}}}}\label{K_L00397-5}, da{\geminationn}{ }\label{K_L00397-6v}\edtext{Granichstädten\pwindex{Granichstaedten, Emil 8.\,7.\,1847 Wien – 2.\,7.\,1904 Berlin@\textsc{Granichstaedten, Emil} (8.\,7.\,1847 Wien – 2.\,7.\,1904 Berlin), \emph{Journalist, Rechtswissenschaftler}|pw}}{\lemma{\textnormal{\emph{Granichstädten}}}\Cendnote{\textnormal{der \emph{Presse}\orgindex{Presse@Die Presse|pwk}}}}\label{K_L00397-6}{ }{\dots} da{\geminationn} gehe ich in die
               Provinz, nach Brü{\geminationn}\oindex{Brünn@\textbf{Brünn}|pw} und Olmütz\oindex{Olomouc@\textbf{Olomouc}|pw}; vielleicht, dass man ihn in
                  Sealsfields\pwindex{Sealsfield, Charles 3.\,3.\,1793 Popice – 26.\,5.\,1864 Solothurn@\textsc{Sealsfield, Charles} (3.\,3.\,1793 Popice – 26.\,5.\,1864 Solothurn), \emph{Schriftsteller}|pw} Heimat ni{\geminationm}t, und 3 fl sind besser als nichts.\pend
           
\pstart
           Besten Gruſs{\\[\baselineskip]}\spacefill\mbox{Fels}\pend
           \leftskip=0em{}
\pstart
           \noindent{}Ich merke eben, dass ich die ekelhafte Gewohnheit angeno{\geminationm}en habe, Ihnen mein Leid, wenn ich nicht ko{\geminationm}en ka{\geminationn}, weil ich an dem
                  Tag schon bei Ihnen war, – schriftlich zu klagen. Seien Sie mir nicht böse!\pend
           \selectlanguage{ngerman}\endnumbering\briefempfaengerindex{Schnitzler, Arthur@\textsc{Schnitzler, Arthur}!zzzFels, Friedrich Michael@\emph{von Friedrich Michael Fels}!1894-11-061@{6. 11. 1894}|)be}\mylabel{L00397h}  \newcommand{\dateiname}{L00397}\newcommand{\titel}{Friedrich M. Fels an Arthur Schnitzler, 6. 11. 1894}\newcommand{\editorInnen}{Kurt Ifkovits und Martin Anton Müller}%% latex-leseansicht-abspann.tex
%% Abspann für die Leseansicht.
%% Der Schalter \ifkorrekturansicht ist bereits durch den Vorspann gesetzt.

%% latex-abspann.tex
%% Gemeinsamer Abspann für Korrekturansicht und Leseansicht.
%% Setzt den Schalter \ifkorrekturansicht voraus (gesetzt in den
%% einbindenden Dateien latex-korrekturansicht-abspann.tex bzw.
%% latex-leseansicht-abspann.tex).
%% ---------------------------------------------------------------

\normalsize

% Das esempio-Environment wird nur in der Leseansicht benötigt
\ifkorrekturansicht\else
\newenvironment{esempio}[3]%
{
    \vspace{1.5ex}
    \rlap{\underline{#1}}
    \par
    \setlength{\parindent}{0cm}
    \nopagebreak
    \leftskip=#2cm
    \rightskip=#3cm
}
{
    \par
}
\fi

\doendnotes{C}
\bigskip
\vfill

\clearpage

\footnotesize

\ifkorrekturansicht
  \lohead{\textsc{register}}
\fi

% theindex-Environment neu definieren ohne reledmac
\makeatletter
\renewenvironment{theindex}{%
  \ifkorrekturansicht
    \section*{\indexname}%
  \else
    \subsubsection*{Index der erwähnten Entitäten}%
  \fi
  \setlength{\parindent}{0pt}%
  \setlength{\parskip}{0pt plus 0.3pt}%
  \let\item\@idxitem
}{%
  \ifkorrekturansicht\clearpage\fi
}
\makeatother

\IfFileExists{\jobname-pw.ind}{\input{\jobname-pw.ind}}{}

% Quellenangabe nur in der Leseansicht
\ifkorrekturansicht\else
% Fallback-Definitionen, falls die .tex-Datei \titel etc. nicht gesetzt hat
\providecommand{\titel}{}
\providecommand{\editorInnen}{}
\providecommand{\dateiname}{\jobname}

\vspace{3cm}

\vfill

\footnotesize
\textsc{Quelle}: \titel. Herausgegeben von {\editorInnen}. In: \emph{Arthur Schnitzler: Briefwechsel mit Autorinnen und Autoren}.
 Digitale Edition, https://schnitzler-briefe.acdh.oeaw.ac.at/{\dateiname}.html (Stand \today)
\fi

\end{document}


