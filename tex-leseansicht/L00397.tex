%% latex-korrekturansicht-vorspann.tex
%% Vorspann für die Korrekturansicht.
%% Lädt die gemeinsame Datei latex-vorspann.tex mit gesetztem Schalter.

\newif\ifkorrekturansicht
\korrekturansichttrue

\input{../tex-inputs/latex-vorspann}


\section[Friedrich M. Fels an Arthur Schnitzler, 6. 11. 1894]{L00397 Friedrich M. Fels an Arthur Schnitzler, 6. 11. 1894}
\nopagebreak\mylabel{L00397v}
\rehead{ }\normalsize\beginnumbering\briefempfaengerindex{Schnitzler, Arthur@\textsc{Schnitzler, Arthur}!zzzFels, Friedrich Michael@\emph{von Friedrich Michael Fels}!1894-11-061@{6. 11. 1894}|(be}
\toendnotes[C]{\smallbreak\pagebreak[2]}\Standort{DLA, A:Schnitzler, HS.NZ85.1.2956.}
\physDesc{Brief, 1 Blatt, 2 Seiten, 1626 Zeichen
\newline{}Handschrift: schwarze Tinte, lateinische Kurrent
\newline{}Schnitzler: 1) mit Bleistift nummeriert: »18«  2) mit rotem Buntstift eine Unterstreichung}
\buchAbdrucke{\weitereDrucke{Hermann Bahr, Arthur Schnitzler: \emph{Briefwechsel, Aufzeichnungen, Dokumente (1891–1931)}. Göttingen: \emph{Wallstein} 2018, S. 86.} }\toendnotes[C]{\smallbreak}
\pstart
           \raggedleft{}{\pb}Wien XVIII, Gürtelstr. 90\oindex{Waehringer Guertel@\textbf{Währinger Gürtel}, \emph{Straße (K.STR)}|pw}{\\}6. Nov. 94\pend
           
\pstart{}Lieber Doktor Schnitzler!\pend\vspace{0.5em}
\pstart
           Herma{\geminationn} Bahr\pwindex{Bahr, Hermann 19.07.1863 – 15.01.1934@\textsc{Bahr, Hermann} (19.07.1863 – 15.01.1934), \emph{Schriftsteller/Schriftstellerin, Kritiker/Kritikerin}|pw} hat
               den Artikel »Skandinavien in Deutschland\pwindex{Skandinavien in Deutschland@\emph{Skandinavien in Deutschland}|pw}«
               abgelehnt, weil er nicht aktuell genug sei und deshalb vor 3–4 Monaten nicht
               erscheinen kö{\geminationn}e. Da er selbstredend! gar nicht annahm,
               daſs ich so lange warten werde, habe ich auch nichts gesagt, obgleich ich herzlich
               froh gewesen wäre, we{\geminationn} er da{\geminationn} erschienen wäre; ich werde froh sein müſsen, we{\geminationn} er
               anderswo so bald erscheint. Aber man muſs den Leuten \introOben{}die\introOben{}
               Ausreden nicht zu schwer machen. Von Artikeln war keine Rede mehr; dagegen sagte Bahr\pwindex{Bahr, Hermann 19.07.1863 – 15.01.1934@\textsc{Bahr, Hermann} (19.07.1863 – 15.01.1934), \emph{Schriftsteller/Schriftstellerin, Kritiker/Kritikerin}|pw}, er werde mir Buchbesprechungen und zwar
               von literarhistorischen Werken – von andern verstehe ich wohl zu wenig – übertragen;
               ich nahm mit Dank an und habe nun die Hoffnung, we{\geminationn}s
               sehr gut geht, in einem Jahr drei Rezensionen schreiben zu dürfen und damit {\pb}5 fl zu verdienen. Hingehen werde ich wohl kaum mehr,
               da er, als ich gemeldet wurde, obgleich ich auf heute 4 Uhr von ihm bestellt war,
               laut aufseufzte und \textcolor{gray}{vernehmlich}{ }ſagte »So lassen Sie ihn in Gottes Namen
               herein.« –\pend
           
\pstart
           Den Artikel\pwindex{Skandinavien in Deutschland@\emph{Skandinavien in Deutschland}|pwv} werde ich morgen
               nach Berlin\oindex{Berlin@\textbf{Berlin}, \emph{P.PPLC}|pw}{ }ſchicken, den beka{\geminationn}ten
               Weg: zuerst Zukunft\orgindex{Zukunft@Die Zukunft|pw}, da{\geminationn}{ }Nation\orgindex{Nation@Die Nation|pw}, da{\geminationn}{ }Tante Voss\orgindex{Vossische Zeitung@Vossische Zeitung|pwv}, da{\geminationn}{ }Gegenwart\orgindex{Gegenwart@Die Gegenwart|pw}, da{\geminationn}{ }{\dots} wer weiss, wohin noch. Den von \label{K_L00397-1v}\edtext{David\pwindex{David, Jakob Julius 1859-02-06 – 1906-11-20@\textsc{David, Jakob Julius} (1859-02-06 – 1906-11-20), \emph{Schriftsteller/Schriftstellerin, Journalist/Journalistin}|pw}}{\lemma{\textnormal{\emph{David}}}\Cendnote{\textnormal{von der \emph{Wiener Allgemeinen Zeitung}\orgindex{Wiener Allgemeine Zeitung@Wiener Allgemeine Zeitung|pwk}}}}\label{K_L00397-1} refusierten \label{K_L00397-2v}\edtext{Sealsfield\pwindex{Sealsfield, Charles 1793-03-03 – 1864-05-26@\textsc{Sealsfield, Charles} (1793-03-03 – 1864-05-26), \emph{Schriftsteller/Schriftstellerin}|pw}artikel}{\lemma{\textnormal{\emph{Sealsfieldartikel}}}\Cendnote{\textnormal{Der Text dürfte der Einleitung von Charles Sealsfield\pwindex{Sealsfield, Charles 1793-03-03 – 1864-05-26@\textsc{Sealsfield, Charles} (1793-03-03 – 1864-05-26), \emph{Schriftsteller/Schriftstellerin}|pwk}: \emph{Das Kajütenbuch oder nationale Charakteristiken}\pwindex{Kajuetenbuch oder nationale Charakteristiken@\emph{Das Kajütenbuch oder nationale Charakteristiken}|pwk}. Herausgegeben und
                     eingeleitet von Friedrich M. Fels\pwindex{Fels, Friedrich Michael *~1864@\textsc{Fels, Friedrich Michael} (*~1864), \emph{Journalist/Journalistin}|pwk}. Stuttgart:
                        \emph{Philipp Reclam Jun.}\orgindex{Philipp Reclam jun.@Philipp Reclam jun.|pwk} [o. J.] entsprechen.
               }}}\label{K_L00397-2} bringe ich \label{K_L00397-3v}\edtext{Uhl\pwindex{Uhl, Friedrich 14.05.1825 – 20.01.1906@\textsc{Uhl, Friedrich} (14.05.1825 – 20.01.1906), \emph{Journalist/Journalistin}|pw}}{\lemma{\textnormal{\emph{Uhl}}}\Cendnote{\textnormal{der \emph{Wiener Zeitung}\orgindex{Wiener Zeitung@Wiener Zeitung|pwk}}}}\label{K_L00397-3}, da{\geminationn}{ }\label{K_L00397-4v}\edtext{Pötzl\pwindex{Poetzl, Eduard 17.03.1851 – 20.08.1914@\textsc{Pötzl, Eduard} (17.03.1851 – 20.08.1914), \emph{Schriftsteller/Schriftstellerin, Journalist/Journalistin}|pw}}{\lemma{\textnormal{\emph{Pötzl}}}\Cendnote{\textnormal{dem \emph{Neuen
                     Wiener Tagblatt}\orgindex{Neues Wiener Tagblatt@Neues Wiener Tagblatt|pwk}}}}\label{K_L00397-4}, da{\geminationn}{ }\label{K_L00397-5v}\edtext{Schönthan\pwindex{Schoenthan-Pernwald, Paul von 19.03.1853 – 04.08.1905@\textsc{Schönthan-Pernwald, Paul von} (19.03.1853 – 04.08.1905), \emph{Schriftsteller/Schriftstellerin, Journalist/Journalistin}|pw}}{\lemma{\textnormal{\emph{Schönthan}}}\Cendnote{\textnormal{dem \emph{Wiener Tagblatt}\orgindex{Wiener Tagblatt@Wiener Tagblatt|pwk}}}}\label{K_L00397-5}, da{\geminationn}{ }\label{K_L00397-6v}\edtext{Granichstädten\pwindex{Granichstaedten, Emil 1847-07-08 – 1904-07-02@\textsc{Granichstaedten, Emil} (1847-07-08 – 1904-07-02), \emph{Journalist/Journalistin, Rechtswissenschaftler/Rechtswissenschaftlerin}|pw}}{\lemma{\textnormal{\emph{Granichstädten}}}\Cendnote{\textnormal{der \emph{Presse}\orgindex{Presse@Die Presse|pwk}}}}\label{K_L00397-6}{ }{\dots} da{\geminationn} gehe ich in die
               Provinz, nach Brü{\geminationn}\oindex{Bruenn@\textbf{Brünn}, \emph{P.PPLA}|pw} und Olmütz\oindex{Olomouc@\textbf{Olomouc}, \emph{P.PPLA}|pw}; vielleicht, dass man ihn in
                  Sealsfields\pwindex{Sealsfield, Charles 1793-03-03 – 1864-05-26@\textsc{Sealsfield, Charles} (1793-03-03 – 1864-05-26), \emph{Schriftsteller/Schriftstellerin}|pw} Heimat ni{\geminationm}t, und 3 fl sind besser als nichts.\pend
           
\pstart
           Besten Gruſs{\\[\baselineskip]}\spacefill\mbox{Fels}\pend
           \leftskip=0em{}
\pstart
           \noindent{}Ich merke eben, dass ich die ekelhafte Gewohnheit angeno{\geminationm}en habe, Ihnen mein Leid, wenn ich nicht ko{\geminationm}en ka{\geminationn}, weil ich an dem
                  Tag schon bei Ihnen war, – schriftlich zu klagen. Seien Sie mir nicht böse!\pend
           \selectlanguage{ngerman}\endnumbering\briefempfaengerindex{Schnitzler, Arthur@\textsc{Schnitzler, Arthur}!zzzFels, Friedrich Michael@\emph{von Friedrich Michael Fels}!1894-11-061@{6. 11. 1894}|)be}\mylabel{L00397h}  \normalsize

\doendnotes{C}
\bigskip
\vfill

\clearpage

\footnotesize

\lohead{\textsc{register}}

% Definiere theindex-Environment komplett neu ohne reledmac
\makeatletter
\renewenvironment{theindex}{%
  \section*{\indexname}%
  \setlength{\parindent}{0pt}%
  \setlength{\parskip}{0pt plus 0.3pt}%
  \let\item\@idxitem
}{%
  \clearpage
}
\makeatother

\IfFileExists{\jobname-pw.ind}{\input{\jobname-pw.ind}}{}

\end{document}

      