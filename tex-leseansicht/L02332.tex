%% latex-leseansicht-vorspann.tex
%% Vorspann für die Leseansicht.
%% Lädt die gemeinsame Datei latex-vorspann.tex mit nicht gesetztem Schalter.

\newif\ifkorrekturansicht
\korrekturansichtfalse

\input{../tex-inputs/latex-vorspann}


         
         \newcommand{\erwaehntePersonen}{Personen:  ?? [Vermieter von Hugo von Hofmannsthal], Gertrude von Hofmannsthal, Christiane von Hofmannsthal, Raimund von Hofmannsthal, Franz von Hofmannsthal, Frieda Pollak, Olga Schnitzler}
         \newcommand{\erwaehnteInstitutionen}{}
         \newcommand{\erwaehnteOrte}{Orte: Bad Aussee, Rodaun, Wien}
         \newcommand{\erwaehnteWerke}{Werke: Die Schwestern oder Casanova in Spa. Lustspiel in Versen}
               \section[Hugo Hofmannsthal an Arthur Schnitzler, 8. 12. {[}1919{]}]{ Hugo Hofmannsthal an Arthur Schnitzler,
               8. 12. {[}1919{]}}\nopagebreak\mylabel{v}\rehead{ }\begin{ledgroupsized}[t]{13cm}\normalsize\beginnumbering \toendnotes[C]{\smallbreak\pagebreak[2]} \Standort{CUL, Schnitzler, B 43.}
\physDesc{Brief, 1 Blatt, 2 Seiten
\newline{}Handschrift: schwarze Tinte, deutsche Kurrent
\newline{}Schnitzler: mit Bleistift die Jahreszahl ein zweites Mal ergänzt:
               »19« \newline{}Ordnung: 1) mit Bleistift von Frieda Pollak\pwindex{Pollak, Frieda 08.12.1881 – 13.07.1937@\textsc{Pollak, Frieda} (08.12.1881 – 13.07.1937), \emph{Sekretärin}|pw} (?) mit dem Buchstaben »A« (Abgeschrieben/Abschrift) gekennzeichnet  2) mit Bleistift von unbekannter Hand nummeriert: »\strikeout{353}« 3) mit Bleistift von unbekannter Hand nummeriert: »384«}\buchAbdrucke{\weitereDrucke{Hugo von Hofmannsthal, Arthur Schnitzler: \emph{Briefwechsel}. Hg. Therese Nickl und Heinrich Schnitzler. Frankfurt am Main: \emph{S. Fischer} 1964, S. 289.} }\toendnotes[C]{\smallbreak}\pstart
           \raggedleft{}{\pb}R.\oindex{Rodaun@\textbf{Rodaun}|pw}{ }8 XII \textsubscript{19}.\pend
           \pstart{}mein lieber Arthur\pend\pstart
           ich dank Ihnen ſchön für den Brief den Sie mir nach Auſſee\oindex{Bad Aussee@\textbf{Bad Aussee}|pw} geſchrieben haben.\hspace*{1.5em}Ich bin nun
               zurück und wünſche mir, wie herzlich, Sie zu ſehen.\hspace*{1.5em}Aber ich bin ſelten in der Stadt – Gerty\pwindex{Hofmannsthal, Gertrude von 16.03.1880 – 09.11.1959@\textsc{Hofmannsthal, Gertrude von} (16.03.1880 – 09.11.1959)|pw} und die
                  Kinder\pwindex{Hofmannsthal, Christiane von 14.05.1902 – 05.01.1987@\textsc{Hofmannsthal, Christiane von} (14.05.1902 – 05.01.1987)|pwv}\pwindex{Hofmannsthal, Raimund von 26.5.1906 – 20.03.1974@\textsc{Hofmannsthal, Raimund von} (26.5.1906 – 20.03.1974)|pwv}\pwindex{Hofmannsthal, Franz von 20.10.1903 – 13.07.1929@\textsc{Hofmannsthal, Franz von} (20.10.1903 – 13.07.1929)|pwv} weit
               öfter, ich aber hab mir hier ein ganz kleines Zimmer bei Rodaun\oindex{Rodaun@\textbf{Rodaun}|pw}er Leuten\pwindex{?? [Vermieter von Hugo von Hofmannsthal] @\textsc{?? [Vermieter von Hugo von Hofmannsthal]}|pwv}
               gemiethet das ſich mit Holz erträglich heizen läſst und ſo bleib ich ſo viel als
               möglich heraußen, eine leidliche Productivität im Fluſs zu halten, denn ich kenne
               mich vor angefangenen Dingen, Plänen u. \textsc{Scenarien} wirklich
                  {\pb}nicht aus und muſs sehen, daſs
               alles weiter \label{T_L02332_1v}\edtext{k\textcolor{gray}{o{\geminationm}t}\strikeout{e}}{\lemma{\textnormal{\emph{kommte}}}\Cendnote{\textnormal{unsichere Lesart; von unbekannter Hand mit Bleistift unterstrichen und am Rand
                  mit einem Fragezeichen markiert.}}}\label{T_L02332_1h}. (Von Ihrem \textsc{Casanova}ſtück\pwindex{Schnitzler, Arthur 15.05.1862 – 21.10.1931@\textsc{Schnitzler, Arthur} (15.05.1862 – 21.10.1931), \emph{Schriftsteller, Mediziner}!Schwestern oder Casanova in Spa. Lustspiel in Versen01. 10. 1919@\strich\emph{Die Schwestern oder Casanova in Spa. Lustspiel in Versen} {[}01. 10. 1919{]}|pwv} höre ich übrigens daſs es beſonders
               reizend fröhlich u. erfreuend iſt, und daſs es bald geſpielt wird, melde mich alſo
               hiemit für die \label{K_L02332_1v}\edtext{Première}{\lemma{\textnormal{\emph{Première}}}\Cendnote{\textnormal{siehe A. S.: \emph{Tagebuch}, 26. 3. 1920}}}\label{K_L02332_1h}.)\pend
           \pstart
           Wie ſehe ich Sie aber mit alledem? Welche Stunde, mit Olga\pwindex{Schnitzler, Olga 17.01.1882 – 13.01.1970@\textsc{Schnitzler, Olga} (17.01.1882 – 13.01.1970), \emph{Schauspielerin, Sängerin}|pw} in die Stadt zu uns zu ko{\geminationm}en iſt denn
               Ihnen u. ihr halbwegs convenierend?\pend
           \pstart
           Sie ſind der Mann der ſtrengen Einteilung, ich bin, \uline{wenn} ich in der Stadt bin, alle Wochen 1 ½ – 2 Tage, dann ganz
               frei! Also ſchreiben Sie mir ein Wort, wie Sie’s beide wollen, ob Sie zu einem ſehr
               beſcheidenen Nachtmahl \label{T_L02332_2v}\edtext{ko{\geminationm}en
               wollen, das wäre das Gemütlichſte – oder wie immer! Ihr \spacefill\mbox{Hugo.}}{\lemma{\textnormal{\emph{kommen … Hugo.}}}\Cendnote{\textnormal{quer am linken
                  Rand}}}\label{T_L02332_2h}\pend
           
         
         \endnumbering\mylabel{h}\end{ledgroupsized}  \newcommand{\dateiname}{L02332}\newcommand{\titel}{Hugo Hofmannsthal an Arthur Schnitzler, 8. 12. [1919]}\newcommand{\editorInnen}{Martin Anton Müller und Gerd-Hermann Susen}%% latex-leseansicht-abspann.tex
%% Abspann für die Leseansicht.
%% Der Schalter \ifkorrekturansicht ist bereits durch den Vorspann gesetzt.

%% latex-abspann.tex
%% Gemeinsamer Abspann für Korrekturansicht und Leseansicht.
%% Setzt den Schalter \ifkorrekturansicht voraus (gesetzt in den
%% einbindenden Dateien latex-korrekturansicht-abspann.tex bzw.
%% latex-leseansicht-abspann.tex).
%% ---------------------------------------------------------------

\normalsize

% Das esempio-Environment wird nur in der Leseansicht benötigt
\ifkorrekturansicht\else
\newenvironment{esempio}[3]%
{
    \vspace{1.5ex}
    \rlap{\underline{#1}}
    \par
    \setlength{\parindent}{0cm}
    \nopagebreak
    \leftskip=#2cm
    \rightskip=#3cm
}
{
    \par
}
\fi

\doendnotes{C}
\bigskip
\vfill

\clearpage

\footnotesize

\ifkorrekturansicht
  \lohead{\textsc{register}}
\fi

% theindex-Environment neu definieren ohne reledmac
\makeatletter
\renewenvironment{theindex}{%
  \ifkorrekturansicht
    \section*{\indexname}%
  \else
    \subsubsection*{Index der erwähnten Entitäten}%
  \fi
  \setlength{\parindent}{0pt}%
  \setlength{\parskip}{0pt plus 0.3pt}%
  \let\item\@idxitem
}{%
  \ifkorrekturansicht\clearpage\fi
}
\makeatother

\IfFileExists{\jobname-pw.ind}{\input{\jobname-pw.ind}}{}

% Quellenangabe nur in der Leseansicht
\ifkorrekturansicht\else
% Fallback-Definitionen, falls die .tex-Datei \titel etc. nicht gesetzt hat
\providecommand{\titel}{}
\providecommand{\editorInnen}{}
\providecommand{\dateiname}{\jobname}

\vspace{3cm}

\vfill

\footnotesize
\textsc{Quelle}: \titel. Herausgegeben von {\editorInnen}. In: \emph{Arthur Schnitzler: Briefwechsel mit Autorinnen und Autoren}.
 Digitale Edition, https://schnitzler-briefe.acdh.oeaw.ac.at/{\dateiname}.html (Stand \today)
\fi

\end{document}


      