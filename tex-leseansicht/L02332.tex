%% latex-korrekturansicht-vorspann.tex
%% Vorspann für die Korrekturansicht.
%% Lädt die gemeinsame Datei latex-vorspann.tex mit gesetztem Schalter.

\newif\ifkorrekturansicht
\korrekturansichttrue

\input{../tex-inputs/latex-vorspann}


\section[Hugo Hofmannsthal an Arthur Schnitzler, 8. 12. {[}1919{]}]{L02332 Hugo Hofmannsthal an Arthur Schnitzler, 8. 12. {[}1919{]}}
\nopagebreak\mylabel{L02332v}
\rehead{ }\normalsize\beginnumbering\briefempfaengerindex{Schnitzler, Arthur@\textsc{Schnitzler, Arthur}!zzzHofmannsthal, Hugo von@\emph{von Hugo von Hofmannsthal}!1919-12-082@{8. 12. {[}1919{]}}|(be}
\toendnotes[C]{\smallbreak\pagebreak[2]}\Standort{CUL, Schnitzler, B 43.}
\physDesc{Brief, 1 Blatt, 2 Seiten, 1147 Zeichen
\newline{}Handschrift: schwarze Tinte, deutsche Kurrent
\newline{}Schnitzler: mit Bleistift die Jahreszahl ein zweites Mal ergänzt:
                                    »19« 
\newline{}Ordnung: 1) mit Bleistift von Frieda
                                    Pollak\pwindex{Pollak, Frieda 08.12.1881 – 13.07.1937@\textsc{Pollak, Frieda} (08.12.1881 – 13.07.1937), \emph{Sekretär/Sekretärin}|pw} (?) mit dem Buchstaben »A«
                                 (Abgeschrieben/Abschrift) gekennzeichnet  2) mit Bleistift von unbekannter Hand nummeriert: »\strikeout{353}« 3) mit Bleistift von unbekannter Hand nummeriert:
                                    »384«}
\buchAbdrucke{\weitereDrucke{Hugo von Hofmannsthal, Arthur Schnitzler: \emph{Briefwechsel}. Frankfurt am Main: \emph{S. Fischer} 1964, S. 289.} }\toendnotes[C]{\smallbreak}
\pstart
           \raggedleft{}{\pb}R.\oindex{Rodaun@\textbf{Rodaun}, \emph{A.ADM4}|pw}{ }8 XII \textsubscript{19}.\pend
           
\pstart{}mein lieber Arthur\pend\vspace{0.5em}
\pstart
           ich dank Ihnen ſchön für den Brief den Sie mir nach Auſſee\oindex{Bad Aussee@\textbf{Bad Aussee}, \emph{P.PPLA3}|pw} geſchrieben haben.\hspace*{1.5em}Ich bin nun
               zurück und wünſche mir, wie herzlich, Sie zu ſehen.\hspace*{1.5em}Aber ich bin ſelten in der Stadt – Gerty\pwindex{Hofmannsthal, Gertrude von 16.03.1880 – 09.11.1959@\textsc{Hofmannsthal, Gertrude von} (16.03.1880 – 09.11.1959)|pw} und
               die Kinder\pwindex{Zimmer, Christiane 14.05.1902 – 05.01.1987@\textsc{Zimmer, Christiane} (14.05.1902 – 05.01.1987)|pwv}\pwindex{Hofmannsthal, Raimund von 26.5.1906 – 20.03.1974@\textsc{Hofmannsthal, Raimund von} (26.5.1906 – 20.03.1974)|pwv}\pwindex{Hofmannsthal, Franz von 20.10.1903 – 13.07.1929@\textsc{Hofmannsthal, Franz von} (20.10.1903 – 13.07.1929)|pwv}
               weit öfter, ich aber hab mir hier ein ganz kleines Zimmer bei Rodaun\oindex{Rodaun@\textbf{Rodaun}, \emph{A.ADM4}|pw}er Leuten\pwindex{?? [Vermieter von Hugo von Hofmannsthal] @\textsc{?? [Vermieter von Hugo von Hofmannsthal]}|pwv} gemiethet das ſich mit Holz erträglich heizen läſst und ſo bleib ich
               ſo viel als möglich heraußen, eine leidliche Productivität im Fluſs zu halten, denn
               ich kenne mich vor angefangenen Dingen, Plänen u. \textsc{Scenarien}
               wirklich {\pb}nicht aus und muſs
               sehen, daſs alles weiter \label{T_L02332-1v}\edtext{k\textcolor{gray}{o{\geminationm}t}\strikeout{e}}{\lemma{\textnormal{\emph{kommt}}}\Cendnote{\textnormal{unsichere Lesart; von unbekannter Hand
                  mit Bleistift unterstrichen und am Rand mit einem Fragezeichen markiert.}}}\label{T_L02332-1}.
               (Von Ihrem \textsc{Casanova}ſtück\pwindex{Schwestern oder Casanova in Spa. Lustspiel in Versen@\emph{Die Schwestern oder Casanova in Spa. Lustspiel in Versen}|pwv} höre ich übrigens daſs es beſonders
               reizend fröhlich u. erfreuend iſt, und daſs es bald geſpielt wird, melde mich alſo
               hiemit für die \label{K_L02332-1v}\edtext{Première}{\lemma{\textnormal{\emph{Première}}}\Cendnote{\textnormal{Siehe A. S.: \emph{Tagebuch}, 26. 3. 1920.
               }}}\label{K_L02332-1}.)\pend
           
\pstart
           Wie ſehe ich Sie aber mit alledem? Welche Stunde, mit Olga\pwindex{Schnitzler, Olga 17.01.1882 – 13.01.1970@\textsc{Schnitzler, Olga} (17.01.1882 – 13.01.1970), \emph{Schauspieler/Schauspielerin, Sänger/Sängerin}|pw} in die Stadt zu uns zu ko{\geminationm}en
               iſt denn Ihnen u. ihr halbwegs convenierend?\pend
           
\pstart
           Sie ſind der Mann der ſtrengen Einteilung, ich bin, \uline{wenn} ich in der Stadt bin, alle Wochen 1 ½ – 2 Tage, dann ganz frei! Also
               ſchreiben Sie mir ein Wort, wie Sie’s beide wollen, ob Sie zu einem ſehr beſcheidenen
               Nachtmahl \label{T_L02332-2v}\edtext{ko{\geminationm}en wollen, das wäre das Gemütlichſte – oder wie immer!
               Ihr \spacefill\mbox{Hugo.}}{\lemma{\textnormal{\emph{kommen … Hugo.}}}\Cendnote{\textnormal{quer am linken Rand}}}\label{T_L02332-2}\pend
           \selectlanguage{ngerman}\endnumbering\briefempfaengerindex{Schnitzler, Arthur@\textsc{Schnitzler, Arthur}!zzzHofmannsthal, Hugo von@\emph{von Hugo von Hofmannsthal}!1919-12-082@{8. 12. {[}1919{]}}|)be}\mylabel{L02332h}  \normalsize

\doendnotes{C}
\bigskip
\vfill

\clearpage

\footnotesize

\lohead{\textsc{register}}

% Definiere theindex-Environment komplett neu ohne reledmac
\makeatletter
\renewenvironment{theindex}{%
  \section*{\indexname}%
  \setlength{\parindent}{0pt}%
  \setlength{\parskip}{0pt plus 0.3pt}%
  \let\item\@idxitem
}{%
  \clearpage
}
\makeatother

\IfFileExists{\jobname-pw.ind}{\input{\jobname-pw.ind}}{}

\end{document}

      