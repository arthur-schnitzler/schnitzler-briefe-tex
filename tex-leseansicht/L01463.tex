%% latex-korrekturansicht-vorspann.tex
%% Vorspann für die Korrekturansicht.
%% Lädt die gemeinsame Datei latex-vorspann.tex mit gesetztem Schalter.

\newif\ifkorrekturansicht
\korrekturansichttrue

\input{../tex-inputs/latex-vorspann}


\section[Arthur Schnitzler an Richard Beer-Hofmann, 31. 10. 1904]{L01463 Arthur Schnitzler an Richard Beer-Hofmann, 31. 10. 1904}
\nopagebreak\mylabel{L01463v}
\rehead{ }\normalsize\beginnumbering\briefempfaengerindex{Beer-Hofmann, Richard@\textsc{Beer-Hofmann, Richard}!zzzSchnitzler, Arthur@\emph{von Arthur Schnitzler}!1904-10-311@{31. 10. 1904}|(be}
\toendnotes[C]{\smallbreak\pagebreak[2]}\Standort{YCGL, MSS 31.}
\physDesc{Telegramm, 131 Zeichen
\newline{}Handschrift einer Schreibkraft: Bleistift, deutsche Kurrent
\newline{}Versand: »\noindent{}\textcolor{gray}{\textbf{Gattung des Telegrammes.}} p{ / }\textcolor{gray}{\textbf{Aufg\damage{egeben am}}}{ }31/X \textcolor{gray}{\textbf{190}}{\dots}{ }\textcolor{gray}{\textbf{um}}{ }\textcolor{gray}{X}\textcolor{gray}{\textbf{Uhr {\dots} Min {\dots} Mittag}}{ / }\textcolor{gray}{\textbf{Eingelangt von}} W.40 Wien\oindex{Wien@\textbf{Wien}, \emph{A.ADM2}|pw}{ }\textcolor{gray}{\textbf{auf Leitung Nr. {\dots} am}}{ }31/X \textcolor{gray}{\textbf{190{\dots}}}{ }\textcolor{gray}{\textbf{um}}{ }XII \textcolor{gray}{\textbf{Uhr}} – \textcolor{gray}{\textbf{Min.}}{ }\textcolor{gray}{v}\textcolor{gray}{\textbf{Mittag}}{ / }\textcolor{gray}{\textbf{Aufgenommen durch}}{ }\textcolor{gray}{RM}{ / }\textcolor{gray}{\textbf{Von}}{ }Wien 111\oindex{Wien@\textbf{Wien}, \emph{A.ADM2}|pw}{ }\textcolor{gray}{\textbf{Aufgabe-Nr.}} 902{ }\textcolor{gray}{\textbf{mit}} 21 \textcolor{gray}{\textbf{Taxworten ({\dots} Worten {\dots} Chiffern)}}« 
\newline{}Ordnung: mit Bleistift von unbekannter Hand datiert: »31. 10. 1904« }\pstart{}{\pb}Richard Beer Hofma{\geminationn}\pend{}\pstart{}Lieſingſtraſſe\oindex{Liesingerstrasse@\textbf{Liesingerstraße}, \emph{Straße (K.STR)}|pw}\pend{}\pstart{}\textcolor{gray}{\textbf{\textit{Rodaun\oindex{Rodaun@\textbf{Rodaun}, \emph{A.ADM4}|pw}}}}\pend{}{\bigskip}\vspace{1em}
\pstart
           \noindent{}{\pb}Freue mich ſehr Sie heute Abends zu ſehn Olga\pwindex{Schnitzler, Olga 17.01.1882 – 13.01.1970@\textsc{Schnitzler, Olga} (17.01.1882 – 13.01.1970), \emph{Schauspieler/Schauspielerin, Sänger/Sängerin}|pw} ko{\geminationm}t mit Paula\pwindex{Beer-Hofmann, Paula 25.02.1879 – 30.10.1939@\textsc{Beer-Hofmann, Paula} (25.02.1879 – 30.10.1939)|pw} hoffentlich auch\pend
           
\pstart
           Herzlichſt{\\[\baselineskip]}\spacefill\mbox{Arthur}\pend
           \leftskip=0em{}\selectlanguage{ngerman}\endnumbering\briefempfaengerindex{Beer-Hofmann, Richard@\textsc{Beer-Hofmann, Richard}!zzzSchnitzler, Arthur@\emph{von Arthur Schnitzler}!1904-10-311@{31. 10. 1904}|)be}\mylabel{L01463h}  \normalsize

\doendnotes{C}
\bigskip
\vfill

\clearpage

\footnotesize

\lohead{\textsc{register}}

% Definiere theindex-Environment komplett neu ohne reledmac
\makeatletter
\renewenvironment{theindex}{%
  \section*{\indexname}%
  \setlength{\parindent}{0pt}%
  \setlength{\parskip}{0pt plus 0.3pt}%
  \let\item\@idxitem
}{%
  \clearpage
}
\makeatother

\IfFileExists{\jobname-pw.ind}{\input{\jobname-pw.ind}}{}

\end{document}

      