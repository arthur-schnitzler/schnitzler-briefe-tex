%% latex-leseansicht-vorspann.tex
%% Vorspann für die Leseansicht.
%% Lädt die gemeinsame Datei latex-vorspann.tex mit nicht gesetztem Schalter.

\newif\ifkorrekturansicht
\korrekturansichtfalse

\input{../tex-inputs/latex-vorspann}

\begin{center}
            \textcolor{red}{ENTWURF. ENTZIFFERUNG NOCH NICHT KORREKTURGELESEN}
                      \end{center}
            
               \section[Arthur Schnitzler an Richard Beer-Hofmann, 31. 10. 1904]{ Arthur Schnitzler an Richard Beer-Hofmann, 31. 10. 1904}\nopagebreak\mylabel{v}\rehead{ }\begin{ledgroupsized}[t]{13cm}\normalsize\beginnumbering\briefempfaengerindex{Beer-Hofmann, Richard@\textsc{Beer-Hofmann, Richard}!zzzSchnitzler, Arthur@\emph{von Arthur Schnitzler}!1904-10-311@{31. 10. 1904}|(be} \toendnotes[C]{\smallbreak\pagebreak[2]} \Standort{YCGL, MSS 31.}
\physDesc{Telegramm
\newline{}Handschrift einer Schreibkraft: Bleistift, deutsche Kurrent\newline{}Versand: »\noindent{}\textcolor{gray}{\textbf{Gattung des Telegrammes.}} p{ / }\textcolor{gray}{\textbf{Aufg\damage{egeben am}}}{ }31/X \textcolor{gray}{\textbf{190}}{\dots}{ }\textcolor{gray}{\textbf{um}}{ }\textcolor{gray}{X} \textcolor{gray}{\textbf{Uhr {\dots} Min {\dots} Mittag}}{ / }\textcolor{gray}{\textbf{Eingelangt von}} W.40 Wien\oindex{Wien@\textbf{Wien}|pw}{ }\textcolor{gray}{\textbf{auf Leitung Nr. {\dots} am}}{ }31/X \textcolor{gray}{\textbf{190{\dots}}}{ }\textcolor{gray}{\textbf{um}}{ }XII \textcolor{gray}{\textbf{Uhr}} – \textcolor{gray}{\textbf{Min.}}{ }\textcolor{gray}{v} \textcolor{gray}{\textbf{Mittag}}{ / }\textcolor{gray}{\textbf{Aufgenommen durch}}{ }\textcolor{gray}{RM}{ / }\textcolor{gray}{\textbf{Von}}{ }Wien 111\oindex{Wien@\textbf{Wien}|pw}{ }\textcolor{gray}{\textbf{Aufgabe-Nr.}} 902{ }\textcolor{gray}{\textbf{mit}} 21 \textcolor{gray}{\textbf{Taxworten ({\dots} Worten {\dots} Chiffern)}}« \newline{}Ordnung: mit Bleistift von unbekannter Hand datiert: »31. 10. 1904« }\pstart{}{\pb}Richard Beer Hofma{\geminationn}\pend{}\pstart{}Lieſingſtraſſe\oindex{Liesingerstrasse@\textbf{Liesingerstraße}|pw}\pend{}\pstart{}\textcolor{gray}{\textbf{\textit{Rodaun\oindex{Rodaun@\textbf{Rodaun}|pw}}}}\pend{}{\bigskip}\pstart
           \noindent{}{\pb}Freue mich ſehr Sie heute Abends zu ſehn Olga\pwindex{Schnitzler, Olga 17.01.1882 – 13.01.1970@\textsc{Schnitzler, Olga} (17.01.1882 – 13.01.1970), \emph{Schauspielerin, Sängerin}|pw} ko{\geminationm}t mit Paula\pwindex{Beer-Hofmann, Paula 25.02.1879 – 30.10.1939@\textsc{Beer-Hofmann, Paula} (25.02.1879 – 30.10.1939)|pw} hoffentlich auch\pend
           \pstart
           Herzlichſt{\\[\baselineskip]}\spacefill\mbox{Arthur}\pend
           \leftskip=0em{}\endnumbering\briefempfaengerindex{Beer-Hofmann, Richard@\textsc{Beer-Hofmann, Richard}!zzzSchnitzler, Arthur@\emph{von Arthur Schnitzler}!1904-10-311@{31. 10. 1904}|)be}\mylabel{h}\end{ledgroupsized}  \newcommand{\dateiname}{L01463}\newcommand{\titel}{Arthur Schnitzler an Richard Beer-Hofmann, 31. 10. 1904}\newcommand{\editorInnen}{Martin Anton Müller und Gerd-Hermann Susen}%% latex-leseansicht-abspann.tex
%% Abspann für die Leseansicht.
%% Der Schalter \ifkorrekturansicht ist bereits durch den Vorspann gesetzt.

%% latex-abspann.tex
%% Gemeinsamer Abspann für Korrekturansicht und Leseansicht.
%% Setzt den Schalter \ifkorrekturansicht voraus (gesetzt in den
%% einbindenden Dateien latex-korrekturansicht-abspann.tex bzw.
%% latex-leseansicht-abspann.tex).
%% ---------------------------------------------------------------

\normalsize

% Das esempio-Environment wird nur in der Leseansicht benötigt
\ifkorrekturansicht\else
\newenvironment{esempio}[3]%
{
    \vspace{1.5ex}
    \rlap{\underline{#1}}
    \par
    \setlength{\parindent}{0cm}
    \nopagebreak
    \leftskip=#2cm
    \rightskip=#3cm
}
{
    \par
}
\fi

\doendnotes{C}
\bigskip
\vfill

\clearpage

\footnotesize

\ifkorrekturansicht
  \lohead{\textsc{register}}
\fi

% theindex-Environment neu definieren ohne reledmac
\makeatletter
\renewenvironment{theindex}{%
  \ifkorrekturansicht
    \section*{\indexname}%
  \else
    \subsubsection*{Index der erwähnten Entitäten}%
  \fi
  \setlength{\parindent}{0pt}%
  \setlength{\parskip}{0pt plus 0.3pt}%
  \let\item\@idxitem
}{%
  \ifkorrekturansicht\clearpage\fi
}
\makeatother

\IfFileExists{\jobname-pw.ind}{\input{\jobname-pw.ind}}{}

% Quellenangabe nur in der Leseansicht
\ifkorrekturansicht\else
% Fallback-Definitionen, falls die .tex-Datei \titel etc. nicht gesetzt hat
\providecommand{\titel}{}
\providecommand{\editorInnen}{}
\providecommand{\dateiname}{\jobname}

\vspace{3cm}

\vfill

\footnotesize
\textsc{Quelle}: \titel. Herausgegeben von {\editorInnen}. In: \emph{Arthur Schnitzler: Briefwechsel mit Autorinnen und Autoren}.
 Digitale Edition, https://schnitzler-briefe.acdh.oeaw.ac.at/{\dateiname}.html (Stand \today)
\fi

\end{document}


      