%% latex-leseansicht-vorspann.tex
%% Vorspann für die Leseansicht.
%% Lädt die gemeinsame Datei latex-vorspann.tex mit nicht gesetztem Schalter.

\newif\ifkorrekturansicht
\korrekturansichtfalse

\input{../tex-inputs/latex-vorspann}


\section[Josef Rosengart an Arthur Schnitzler, 5. 12. 1893]{L02798 Josef Rosengart an Arthur Schnitzler, 5. 12. 1893}
\nopagebreak\mylabel{L02798v}
\rehead{ }\normalsize\beginnumbering\briefempfaengerindex{Schnitzler, Arthur@\textsc{Schnitzler, Arthur}!zzzRosengart, Josef@\emph{von Josef Rosengart}!1893-12-052@{5. 12. 1893}|(be}
\toendnotes[C]{\smallbreak\pagebreak[2]}
\correspDesc{Versand  durch Josef Rosengart am 5. 12. 1893 in Frankfurt am Main
\newline{}Erhalt  durch Arthur Schnitzler im Zeitraum [6. 12. 1893
                  – 10. 12. 1893?] in Wien}\toendnotes[C]{\smallbreak}
\Standort{DLA, A:Schnitzler, HS.NZ85.1.4334.}
\physDesc{Briefkarte, 786 Zeichen
\newline{}Handschrift: schwarze Tinte, deutsche Kurrent
\newline{}Schnitzler: 1) mit Bleistift Vermerk »\textsc{Rosengart\pwindex{Rosengart, Josef 8.\,2.\,1860 Laupheim – 4.\,8.\,1927 Frankfurt am Main@\textsc{Rosengart, Josef} (8.\,2.\,1860 Laupheim – 4.\,8.\,1927 Frankfurt am Main), \emph{Arzt}|pw}}«  2) mit rotem Buntstift eine Unterstreichung}\toendnotes[C]{\smallbreak}
\pstart
           \raggedleft{}{\pb}Frankfurtm\oindex{Frankfurt am Main@\textbf{Frankfurt am Main}, \emph{Hauptstadt}|pw}, 5. Dezbr 1893.\pend
           
\pstart{}Sehr geehrter Herr Doctor!\pend\vspace{0.5em}
\pstart
           Durch meinen Schwager \textsc{Paul Goldmann\pwindex{Goldmann, Paul 31.\,1.\,1865 Breslau – 25.\,9.\,1935 Wien@\textsc{Goldmann, Paul} (31.\,1.\,1865 Breslau – 25.\,9.\,1935 Wien), \emph{Schriftsteller, Journalist}|pw}} in \textsc{Paris\oindex{Paris@\textbf{Paris}, \emph{Hauptstadt}|pw}} erfahre ich, daß ich Ihrer beſonderen Liebenswürdigkeit die \label{K_L02798-1v}\edtext{Zuſendung der{ }ſo{ }ſehr intereſſanten und
               wiſſenſchaftlich bedeutenden »Internationalen
                  kliniſchen Rundſchau\pwindex{Internationale klinische Rundschau@\emph{Internationale klinische Rundschau}|pw}}{\lemma{\textnormal{\emph{Zusendung … Rundschau}}}\Cendnote{\textnormal{Siehe XXXX Auszeichnungsfehler: Dokument L02719 nicht gefunden.
               }}}\label{K_L02798-1}« verdanke. Ich danke Ihnen hierfür ganz beſonders, übertragen Sie hierdurch
               doch ein Stückchen Ihrer Freundſchaft für {\pb}meinen Schwager\pwindex{Goldmann, Paul 31.\,1.\,1865 Breslau – 25.\,9.\,1935 Wien@\textsc{Goldmann, Paul} (31.\,1.\,1865 Breslau – 25.\,9.\,1935 Wien), \emph{Schriftsteller, Journalist}|pwv} auf mich!\pend
           
\pstart
           Ich erlaube mir, Ihnen bei dieſer Gelegenheit – und als nunmehr bei Ihnen eingeführt
               zu dem Erfolge Ihres in \label{K_L02798-2v}\edtext{\textsc{Wien}\oindex{Wien@\textbf{Wien}, \emph{Verwaltungsgebiet}|pw} aufgeführten Stück\pwindex{Schnitzler, Arthur 15.\,5.\,1862 Wien – 21.\,10.\,1931 ebd.@\textsc{Schnitzler, Arthur} (15.\,5.\,1862 Wien – 21.\,10.\,1931 ebd.), \emph{Schriftsteller, Mediziner}!Märchen. Schauspiel in drei Aufzügen@\strich\emph{Das Märchen. Schauspiel in drei Aufzügen}|pwv}es}{\lemma{\textnormal{\emph{Wien … Stückes}}}\Cendnote{\textnormal{Die Uraufführung von \emph{Das Märchen}\pwindex{Schnitzler, Arthur 15.\,5.\,1862 Wien – 21.\,10.\,1931 ebd.@\textsc{Schnitzler, Arthur} (15.\,5.\,1862 Wien – 21.\,10.\,1931 ebd.), \emph{Schriftsteller, Mediziner}!Märchen. Schauspiel in drei Aufzügen@\strich\emph{Das Märchen. Schauspiel in drei Aufzügen}|pwk}\eventindex{Volkstheater@\textbf{Volkstheater}!Uraufführung von Das Märchen, 1.12.1893@Uraufführung von Das Märchen, 1.12.1893|pwk} hatte am 1. 12. 1893 am \emph{Deutschen Volkstheater}\orgindex{Volkstheater@Volkstheater|pwk} stattgefunden.}}}\label{K_L02798-2} Glück zu
               wünſchen. \textsc{Paul\pwindex{Goldmann, Paul 31.\,1.\,1865 Breslau – 25.\,9.\,1935 Wien@\textsc{Goldmann, Paul} (31.\,1.\,1865 Breslau – 25.\,9.\,1935 Wien), \emph{Schriftsteller, Journalist}|pw}} hat uns{ }ſchon i{\geminationm}er von Ihnen und von dem Großen,
               was er von Ihnen erwartet, erzählt, daß wir von Ihren Erfolgen nicht überraſcht
               waren. Genehmigen Sie,{ }ſehr geehrter Herr Doctor, den Ausdruck der Hochachtung Ihres
               ergebenen\pend
           \pstart \spacefill\mbox{DrRosengart.}\pend{}\selectlanguage{ngerman}\endnumbering\briefempfaengerindex{Schnitzler, Arthur@\textsc{Schnitzler, Arthur}!zzzRosengart, Josef@\emph{von Josef Rosengart}!1893-12-052@{5. 12. 1893}|)be}\mylabel{L02798h}  \newcommand{\dateiname}{L02798}\newcommand{\titel}{Josef Rosengart an Arthur Schnitzler, 5. 12. 1893}\newcommand{\editorInnen}{Martin Anton Müller und Laura Untner}%% latex-leseansicht-abspann.tex
%% Abspann für die Leseansicht.
%% Der Schalter \ifkorrekturansicht ist bereits durch den Vorspann gesetzt.

%% latex-abspann.tex
%% Gemeinsamer Abspann für Korrekturansicht und Leseansicht.
%% Setzt den Schalter \ifkorrekturansicht voraus (gesetzt in den
%% einbindenden Dateien latex-korrekturansicht-abspann.tex bzw.
%% latex-leseansicht-abspann.tex).
%% ---------------------------------------------------------------

\normalsize

% Das esempio-Environment wird nur in der Leseansicht benötigt
\ifkorrekturansicht\else
\newenvironment{esempio}[3]%
{
    \vspace{1.5ex}
    \rlap{\underline{#1}}
    \par
    \setlength{\parindent}{0cm}
    \nopagebreak
    \leftskip=#2cm
    \rightskip=#3cm
}
{
    \par
}
\fi

\doendnotes{C}
\bigskip
\vfill

\clearpage

\footnotesize

\ifkorrekturansicht
  \lohead{\textsc{register}}
\fi

% theindex-Environment neu definieren ohne reledmac
\makeatletter
\renewenvironment{theindex}{%
  \ifkorrekturansicht
    \section*{\indexname}%
  \else
    \subsubsection*{Index der erwähnten Entitäten}%
  \fi
  \setlength{\parindent}{0pt}%
  \setlength{\parskip}{0pt plus 0.3pt}%
  \let\item\@idxitem
}{%
  \ifkorrekturansicht\clearpage\fi
}
\makeatother

\IfFileExists{\jobname-pw.ind}{\input{\jobname-pw.ind}}{}

% Quellenangabe nur in der Leseansicht
\ifkorrekturansicht\else
% Fallback-Definitionen, falls die .tex-Datei \titel etc. nicht gesetzt hat
\providecommand{\titel}{}
\providecommand{\editorInnen}{}
\providecommand{\dateiname}{\jobname}

\vspace{3cm}

\vfill

\footnotesize
\textsc{Quelle}: \titel. Herausgegeben von {\editorInnen}. In: \emph{Arthur Schnitzler: Briefwechsel mit Autorinnen und Autoren}.
 Digitale Edition, https://schnitzler-briefe.acdh.oeaw.ac.at/{\dateiname}.html (Stand \today)
\fi

\end{document}


