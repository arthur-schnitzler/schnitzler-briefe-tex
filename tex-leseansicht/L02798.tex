%% latex-leseansicht-vorspann.tex
%% Vorspann für die Leseansicht.
%% Lädt die gemeinsame Datei latex-vorspann.tex mit nicht gesetztem Schalter.

\newif\ifkorrekturansicht
\korrekturansichtfalse

\input{../tex-inputs/latex-vorspann}

\begin{center}
            \textcolor{red}{ENTWURF, NICHT FERTIG KORRIGIERT}
                      \end{center}
            
         
         \renewcommand{\erwaehntePersonen}{Personen: Paul Goldmann, Josef Rosengart}
         \renewcommand{\erwaehnteOrte}{Orte: Frankfurt am Main, Paris, Volkstheater, Wien}
         \renewcommand{\erwaehnteWerke}{Werke: Das Märchen. Schauspiel in drei Aufzügen, Internationale klinische Rundschau}
               \section[Josef Rosengart an Arthur Schnitzler, 5. 12. 1893]{ Josef Rosengart an Arthur Schnitzler, 5. 12. 1893}\nopagebreak\mylabel{v}\rehead{ }\begin{ledgroupsized}[t]{13cm}\normalsize\beginnumbering \toendnotes[C]{\smallbreak\pagebreak[2]} \Standort{DLA, A:Schnitzler, HS.NZ85.1.4334.}
\physDesc{Briefkarte, 786 Zeichen
\newline{}Handschrift: schwarze Tinte, deutsche Kurrent
\newline{}Schnitzler: 1) mit Bleistift »\textsc{Rosengart\pwindex{Rosengart, Josef 1860-02-08 – 1927-08-04@\textsc{Rosengart, Josef} (1860-02-08 – 1927-08-04), \emph{Arzt}|pw}}« vermerkt  2) mit rotem Buntstift eine Unterstreichung}\toendnotes[C]{\smallbreak}\pstart
           \raggedleft{}{\pb}Frankfurtm\oindex{Frankfurt am Main@\textbf{Frankfurt am Main}|pw}, 5. Dezbr
                  1893.\pend
           \pstart{}Sehr geehrter Herr Doctor!\pend\pstart
           Durch meinen Schwager \textsc{Paul Goldmann\pwindex{Goldmann, Paul 31.01.1865 – 25.09.1935@\textsc{Goldmann, Paul} (31.01.1865 – 25.09.1935), \emph{Schriftsteller, Journalist}|pw}} in \textsc{Paris\oindex{Paris@\textbf{Paris}|pw}} erfahre ich, daß ich Ihrer beſonderen Liebenswürdigkeit die \label{K_L02798-1v}\edtext{Zuſendung der ſo ſehr intereſſanten und
               wiſſenſchaftlich bedeutenden »Internationalen
                  kliniſchen Rundſchau\pwindex{Internationale klinische Rundschau1887-01-01 – 1922@\emph{Internationale klinische Rundschau} {[}1887-01-01 – 1922{]}|pw}}{\lemma{\textnormal{\emph{Zuſendung … Rundſchau}}}\Cendnote{\textnormal{siehe Paul Goldmann an Arthur Schnitzler, 4. 11. [1893]}}}\label{K_L02798-1h}« verdanke. Ich danke Ihnen hierfür ganz beſonders, übertragen Sie hierdurch
               doch ein Stückchen Ihrer Freundſchaft für {\pb}meinen Schwager\pwindex{Goldmann, Paul 31.01.1865 – 25.09.1935@\textsc{Goldmann, Paul} (31.01.1865 – 25.09.1935), \emph{Schriftsteller, Journalist}|pwv} auf mich!\pend
           \pstart
           Ich erlaube mir, Ihnen bei dieſer Gelegenheit – und als nunmehr bei Ihnen eingeführt
               zu dem Erfolge Ihres in \label{K_L02798-2v}\edtext{\textsc{Wien}\oindex{Wien@\textbf{Wien}|pw} aufgeführten Stück\pwindex{Schnitzler, Arthur 15.05.1862 – 21.10.1931@\textsc{Schnitzler, Arthur} (15.05.1862 – 21.10.1931), \emph{Schriftsteller, Mediziner}!Maerchen. Schauspiel in drei Aufzuegen1893-12-01@\strich\emph{Das Märchen. Schauspiel in drei Aufzügen} {[}1893-12-01{]}|pwv}es}{\lemma{\textnormal{\emph{Wien … Stückes}}}\Cendnote{\textnormal{Die Uraufführung von \emph{Das Märchen}\pwindex{Schnitzler, Arthur 15.05.1862 – 21.10.1931@\textsc{Schnitzler, Arthur} (15.05.1862 – 21.10.1931), \emph{Schriftsteller, Mediziner}!Maerchen. Schauspiel in drei Aufzuegen1893-12-01@\strich\emph{Das Märchen. Schauspiel in drei Aufzügen} {[}1893-12-01{]}|pwk} hatte am 1. 12. 1893 am Deutschen Volkstheater\oindex{Volkstheater@\textbf{Volkstheater}|pwk} stattgefunden.}}}\label{K_L02798-2h} Glück zu
               wünſchen. \textsc{Paul\pwindex{Goldmann, Paul 31.01.1865 – 25.09.1935@\textsc{Goldmann, Paul} (31.01.1865 – 25.09.1935), \emph{Schriftsteller, Journalist}|pw}} hat uns ſchon i{\geminationm}er von Ihnen und von dem Großen,
               was er von Ihnen erwartet, erzählt, daß wir von Ihren Erfolgen nicht überraſcht
               waren. Genehmigen Sie, ſehr geehrter Herr Doctor, den Ausdruck der Hochachtung Ihres
               ergebenen\pend
           \pstart \spacefill\mbox{DrRosengart.}\pend{}
         
         \endnumbering\mylabel{h}\end{ledgroupsized}  \newcommand{\dateiname}{L02798}\newcommand{\titel}{Josef Rosengart an Arthur Schnitzler, 5. 12. 1893}\newcommand{\editorInnen}{Martin Anton Müller und Laura Untner}%% latex-leseansicht-abspann.tex
%% Abspann für die Leseansicht.
%% Der Schalter \ifkorrekturansicht ist bereits durch den Vorspann gesetzt.

%% latex-abspann.tex
%% Gemeinsamer Abspann für Korrekturansicht und Leseansicht.
%% Setzt den Schalter \ifkorrekturansicht voraus (gesetzt in den
%% einbindenden Dateien latex-korrekturansicht-abspann.tex bzw.
%% latex-leseansicht-abspann.tex).
%% ---------------------------------------------------------------

\normalsize

% Das esempio-Environment wird nur in der Leseansicht benötigt
\ifkorrekturansicht\else
\newenvironment{esempio}[3]%
{
    \vspace{1.5ex}
    \rlap{\underline{#1}}
    \par
    \setlength{\parindent}{0cm}
    \nopagebreak
    \leftskip=#2cm
    \rightskip=#3cm
}
{
    \par
}
\fi

\doendnotes{C}
\bigskip
\vfill

\clearpage

\footnotesize

\ifkorrekturansicht
  \lohead{\textsc{register}}
\fi

% theindex-Environment neu definieren ohne reledmac
\makeatletter
\renewenvironment{theindex}{%
  \ifkorrekturansicht
    \section*{\indexname}%
  \else
    \subsubsection*{Index der erwähnten Entitäten}%
  \fi
  \setlength{\parindent}{0pt}%
  \setlength{\parskip}{0pt plus 0.3pt}%
  \let\item\@idxitem
}{%
  \ifkorrekturansicht\clearpage\fi
}
\makeatother

\IfFileExists{\jobname-pw.ind}{\input{\jobname-pw.ind}}{}

% Quellenangabe nur in der Leseansicht
\ifkorrekturansicht\else
% Fallback-Definitionen, falls die .tex-Datei \titel etc. nicht gesetzt hat
\providecommand{\titel}{}
\providecommand{\editorInnen}{}
\providecommand{\dateiname}{\jobname}

\vspace{3cm}

\vfill

\footnotesize
\textsc{Quelle}: \titel. Herausgegeben von {\editorInnen}. In: \emph{Arthur Schnitzler: Briefwechsel mit Autorinnen und Autoren}.
 Digitale Edition, https://schnitzler-briefe.acdh.oeaw.ac.at/{\dateiname}.html (Stand \today)
\fi

\end{document}


      