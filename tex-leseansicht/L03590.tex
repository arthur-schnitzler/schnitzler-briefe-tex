%% latex-leseansicht-vorspann.tex
%% Vorspann für die Leseansicht.
%% Lädt die gemeinsame Datei latex-vorspann.tex mit nicht gesetztem Schalter.

\newif\ifkorrekturansicht
\korrekturansichtfalse

\input{../tex-inputs/latex-vorspann}

\begin{center}
            \textcolor{red}{ENTWURF, NICHT FERTIG KORRIGIERT}
                      \end{center}
            
         
         \renewcommand{\erwaehntePersonen}{Personen: Felix Salten}
         \renewcommand{\erwaehnteOrte}{Orte: Berlin, Sternwartestraße 71, Wien}
         \renewcommand{\erwaehnteWerke}{}
               \section[Felix Salten an Arthur Schnitzler, 24. 12. {[}1929?{]}]{ Felix Salten an Arthur Schnitzler, 24. 12. {[}1929?{]}}\nopagebreak\mylabel{v}\rehead{ }\begin{ledgroupsized}[t]{13cm}\normalsize\beginnumbering \toendnotes[C]{\smallbreak\pagebreak[2]} \Standort{CUL, Schnitzler, B 89, B 2.}
\physDesc{Telegramm, 1 Blatt, 1 Seite, 211 Zeichen
\newline{}maschinell
\newline{}Versand: 1) gestempelt am Vordruck: »\noindent{}\textcolor{gray}{\textbf{Aufgenommen von}}{ }\textcolor{gray}{\textbf{\textit{B 25}}}{ / }\textcolor{gray}{\textbf{auf Ltg. Nr.}}{ }\textcolor{gray}{\textbf{\textit{24/12}}}{ / }\textcolor{gray}{\textbf{\textit{K\textcolor{gray}{l G}}}}«  2) Stempel: »\nobreak{}24 Dec, 13, ausgefertigt\nobreak{}«. 
\newline{}Schnitzler: mit rotem Buntstift datiert: »24/12 \textcolor{gray}{×}\textcolor{gray}{9}« und zwei
                                 Unterstreichungen }\pstart{}{\pb}schnitzler\pend{}\pstart{}sternwartestrasse 71\oindex{Sternwartestrasse 71@\textbf{Sternwartestraße 71}|pw}\pend{}\pstart{}wien\oindex{Wien@\textbf{Wien}|pw} =\pend{}{\bigskip}\pstart
           54 berlin\oindex{Berlin@\textbf{Berlin}|pw} /50 757 27/26 24 1215 /
               \pend
           \pstart
           wir alle denken an diesen tagen vol freundschaft liebe und verehrung an sie und
               senden tausend herzliche wuensche aufrichtig = ihr \pend
           \pstart \spacefill\mbox{felix salten +}\pend{}
         
         \endnumbering\mylabel{h}\end{ledgroupsized}\begin{anhang}\end{anhang}\newcommand{\dateiname}{L03590}\newcommand{\titel}{Felix Salten an Arthur Schnitzler, 24. 12. [1929?]}\newcommand{\editorInnen}{Martin Anton Müller und Laura Untner}%% latex-leseansicht-abspann.tex
%% Abspann für die Leseansicht.
%% Der Schalter \ifkorrekturansicht ist bereits durch den Vorspann gesetzt.

%% latex-abspann.tex
%% Gemeinsamer Abspann für Korrekturansicht und Leseansicht.
%% Setzt den Schalter \ifkorrekturansicht voraus (gesetzt in den
%% einbindenden Dateien latex-korrekturansicht-abspann.tex bzw.
%% latex-leseansicht-abspann.tex).
%% ---------------------------------------------------------------

\normalsize

% Das esempio-Environment wird nur in der Leseansicht benötigt
\ifkorrekturansicht\else
\newenvironment{esempio}[3]%
{
    \vspace{1.5ex}
    \rlap{\underline{#1}}
    \par
    \setlength{\parindent}{0cm}
    \nopagebreak
    \leftskip=#2cm
    \rightskip=#3cm
}
{
    \par
}
\fi

\doendnotes{C}
\bigskip
\vfill

\clearpage

\footnotesize

\ifkorrekturansicht
  \lohead{\textsc{register}}
\fi

% theindex-Environment neu definieren ohne reledmac
\makeatletter
\renewenvironment{theindex}{%
  \ifkorrekturansicht
    \section*{\indexname}%
  \else
    \subsubsection*{Index der erwähnten Entitäten}%
  \fi
  \setlength{\parindent}{0pt}%
  \setlength{\parskip}{0pt plus 0.3pt}%
  \let\item\@idxitem
}{%
  \ifkorrekturansicht\clearpage\fi
}
\makeatother

\IfFileExists{\jobname-pw.ind}{\input{\jobname-pw.ind}}{}

% Quellenangabe nur in der Leseansicht
\ifkorrekturansicht\else
% Fallback-Definitionen, falls die .tex-Datei \titel etc. nicht gesetzt hat
\providecommand{\titel}{}
\providecommand{\editorInnen}{}
\providecommand{\dateiname}{\jobname}

\vspace{3cm}

\vfill

\footnotesize
\textsc{Quelle}: \titel. Herausgegeben von {\editorInnen}. In: \emph{Arthur Schnitzler: Briefwechsel mit Autorinnen und Autoren}.
 Digitale Edition, https://schnitzler-briefe.acdh.oeaw.ac.at/{\dateiname}.html (Stand \today)
\fi

\end{document}


      