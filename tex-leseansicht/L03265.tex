%% latex-leseansicht-vorspann.tex
%% Vorspann für die Leseansicht.
%% Lädt die gemeinsame Datei latex-vorspann.tex mit nicht gesetztem Schalter.

\newif\ifkorrekturansicht
\korrekturansichtfalse

\input{../tex-inputs/latex-vorspann}


         
         \renewcommand{\erwaehntePersonen}{Personen: Paul Goldmann, Felix Salten}
         \renewcommand{\erwaehnteOrte}{Orte: Paris, Riva del Garda, Wien, rue de Maubeuge, rue de la Bourse}
         \renewcommand{\erwaehnteWerke}{}
               \section[ Felix Salten an Arthur Schnitzler, 16. 5. 1897]{ Felix Salten an Arthur Schnitzler, 16. 5. 1897}\nopagebreak\mylabel{v}\rehead{ }\begin{ledgroupsized}[t]{13cm}\normalsize\beginnumbering\briefempfaengerindex{Schnitzler, Arthur@\textsc{Schnitzler, Arthur}!zzzSalten, Felix@\emph{von Felix Salten}!1897-05-161@{16. 5. 1897}|(be} \toendnotes[C]{\smallbreak\pagebreak[2]} \Standort{CUL, Schnitzler, B 89, A 2.}
\physDesc{Brief, 1 Blatt, 1 Seite, 739 Zeichen
\newline{}Handschrift: schwarze Tinte, lateinische Kurrent
\newline{}Ordnung: mit Bleistift von unbekannter Hand nummeriert: »88« }\toendnotes[C]{\smallbreak}\pstart
           \raggedleft{}{\pb}Wien\oindex{Wien@\textbf{Wien}|pw}, 16. Mai 97\pend
           \pstart
           Lieber Arthur,{ }gestern{ }Abend erfuhr ich durch Zufall Ihre jetzige \label{K_L03265-1v}\edtext{Adresse}{\lemma{\textnormal{\emph{Adresse}}}\Cendnote{\textnormal{Salten\pwindex{Salten, Felix 06.09.1869 – 08.10.1945@\textsc{Salten, Felix} (06.09.1869 – 08.10.1945), \emph{Schriftsteller, Journalist}|pwk} bringt hier mehrere Dinge
                  durcheinander. Schnitzler\pwindex{Schnitzler, Arthur 15.05.1862 – 21.10.1931@\textsc{Schnitzler, Arthur} (15.05.1862 – 21.10.1931), \emph{Schriftsteller, Mediziner}|pwk}s Adresse in Paris\oindex{Paris@\textbf{Paris}|pwk} lag in der rue de Maubeuge\oindex{rue de Maubeuge@\textbf{rue de Maubeuge}|pwk}, wohin Salten\pwindex{Salten, Felix 06.09.1869 – 08.10.1945@\textsc{Salten, Felix} (06.09.1869 – 08.10.1945), \emph{Schriftsteller, Journalist}|pwk} am
                     5. 5. 1897 geschrieben
                  haben dürfte. Schnitzler\pwindex{Schnitzler, Arthur 15.05.1862 – 21.10.1931@\textsc{Schnitzler, Arthur} (15.05.1862 – 21.10.1931), \emph{Schriftsteller, Mediziner}|pwk} hat das
                  Schreiben auch erhalten, es ist in seinem Nachlass überliefert.
                  Die rue de la Bourse\oindex{rue de la Bourse@\textbf{rue de la Bourse}|pwk} (was Salten\pwindex{Salten, Felix 06.09.1869 – 08.10.1945@\textsc{Salten, Felix} (06.09.1869 – 08.10.1945), \emph{Schriftsteller, Journalist}|pwk} im vorliegenden Brief als Adresse nennt) war die
                  Postadresse von Paul Goldmann\pwindex{Goldmann, Paul 31.01.1865 – 25.09.1935@\textsc{Goldmann, Paul} (31.01.1865 – 25.09.1935), \emph{Schriftsteller, Journalist}|pwk}, die Schnitzler\pwindex{Schnitzler, Arthur 15.05.1862 – 21.10.1931@\textsc{Schnitzler, Arthur} (15.05.1862 – 21.10.1931), \emph{Schriftsteller, Mediziner}|pwk} zu verwenden bat, vgl. Arthur Schnitzler an Richard Beer-Hofmann, 19. 4. 1897.}}}\label{K_L03265-1h}, und erklärte
               mir daraus, weshalb Sie mir wol bis heute nicht
               geantwortet haben. Offenbar haben Sie meinen \label{K_L03265-2v}\edtext{Brief nicht erhalten, den ich Ihnen vor mehr als vierzehn
                  Tagen}{\lemma{\textnormal{\emph{Brief … Tagen}}}\Cendnote{\textnormal{Nachdem der Brief vom 5. 5. 1897 erhalten ist,
                  dürfte Schnitzler\pwindex{Schnitzler, Arthur 15.05.1862 – 21.10.1931@\textsc{Schnitzler, Arthur} (15.05.1862 – 21.10.1931), \emph{Schriftsteller, Mediziner}|pwk} ihn regulär erhalten
                  haben, aber auf eine unmittelbare Antwort verzichtet haben – oder diese ging
                  verlustig. Jedenfalls irrt Salten\pwindex{Salten, Felix 06.09.1869 – 08.10.1945@\textsc{Salten, Felix} (06.09.1869 – 08.10.1945), \emph{Schriftsteller, Journalist}|pwk}, sein
                  Schreiben lag noch keine zwei Wochen zurück.}}}\label{K_L03265-2h} schrieb. Ich kam Ende April aus Riva\oindex{Riva del Garda@\textbf{Riva del Garda}|pw} zurück
               und fand Ihre Karte und Ihren Brief. Darauf habe ich ziemlich ausführlich erwiedert
               und, da Sie es zu wünschen schienen, über mein Leben und meine Arbeiten ec.
               berichtet. Auf die Adresse schrieb ich nach Ihrer Angabe rue de la Bourse\oindex{rue de la Bourse@\textbf{rue de la Bourse}|pw}. Offenbar haben Sie dieses Schreiben nicht
               erhalten, und da ich hier mit Niemandem verkehre, habe ich erst gestern{ }Abend Ihre neue Wohnung erfahren und glaube, Ihnen das zur Aufklärung
               sagen zu müßen.\pend
           \pstart
           Herzlich {\\[\baselineskip]}\spacefill\mbox{Salten}\pend
           \leftskip=0em{}
         
         \endnumbering\mylabel{h}\end{ledgroupsized}  \newcommand{\dateiname}{L03265}\newcommand{\titel}{Felix Salten an Arthur Schnitzler, 16. 5. 1897}\newcommand{\editorInnen}{Martin Anton Müller und Laura Untner}%% latex-leseansicht-abspann.tex
%% Abspann für die Leseansicht.
%% Der Schalter \ifkorrekturansicht ist bereits durch den Vorspann gesetzt.

%% latex-abspann.tex
%% Gemeinsamer Abspann für Korrekturansicht und Leseansicht.
%% Setzt den Schalter \ifkorrekturansicht voraus (gesetzt in den
%% einbindenden Dateien latex-korrekturansicht-abspann.tex bzw.
%% latex-leseansicht-abspann.tex).
%% ---------------------------------------------------------------

\normalsize

% Das esempio-Environment wird nur in der Leseansicht benötigt
\ifkorrekturansicht\else
\newenvironment{esempio}[3]%
{
    \vspace{1.5ex}
    \rlap{\underline{#1}}
    \par
    \setlength{\parindent}{0cm}
    \nopagebreak
    \leftskip=#2cm
    \rightskip=#3cm
}
{
    \par
}
\fi

\doendnotes{C}
\bigskip
\vfill

\clearpage

\footnotesize

\ifkorrekturansicht
  \lohead{\textsc{register}}
\fi

% theindex-Environment neu definieren ohne reledmac
\makeatletter
\renewenvironment{theindex}{%
  \ifkorrekturansicht
    \section*{\indexname}%
  \else
    \subsubsection*{Index der erwähnten Entitäten}%
  \fi
  \setlength{\parindent}{0pt}%
  \setlength{\parskip}{0pt plus 0.3pt}%
  \let\item\@idxitem
}{%
  \ifkorrekturansicht\clearpage\fi
}
\makeatother

\IfFileExists{\jobname-pw.ind}{\input{\jobname-pw.ind}}{}

% Quellenangabe nur in der Leseansicht
\ifkorrekturansicht\else
% Fallback-Definitionen, falls die .tex-Datei \titel etc. nicht gesetzt hat
\providecommand{\titel}{}
\providecommand{\editorInnen}{}
\providecommand{\dateiname}{\jobname}

\vspace{3cm}

\vfill

\footnotesize
\textsc{Quelle}: \titel. Herausgegeben von {\editorInnen}. In: \emph{Arthur Schnitzler: Briefwechsel mit Autorinnen und Autoren}.
 Digitale Edition, https://schnitzler-briefe.acdh.oeaw.ac.at/{\dateiname}.html (Stand \today)
\fi

\end{document}


      