%% latex-korrekturansicht-vorspann.tex
%% Vorspann für die Korrekturansicht.
%% Lädt die gemeinsame Datei latex-vorspann.tex mit gesetztem Schalter.

\newif\ifkorrekturansicht
\korrekturansichttrue

\input{../tex-inputs/latex-vorspann}


\section[ Felix Salten an Arthur Schnitzler, 16. 5. 1897]{L03265 Felix Salten an Arthur Schnitzler, 16. 5. 1897}
\nopagebreak\mylabel{L03265v}
\rehead{ }\normalsize\beginnumbering\briefempfaengerindex{Schnitzler, Arthur@\textsc{Schnitzler, Arthur}!zzzSalten, Felix@\emph{von Felix Salten}!1897-05-161@{16. 5. 1897}|(be}
\toendnotes[C]{\smallbreak\pagebreak[2]}\Standort{CUL, Schnitzler, B 89, A 2.}
\physDesc{Brief, 1 Blatt, 1 Seite, 739 Zeichen
\newline{}Handschrift: schwarze Tinte, lateinische Kurrent
\newline{}Ordnung: mit Bleistift von unbekannter Hand nummeriert: »88« }\toendnotes[C]{\smallbreak}
\pstart
           \raggedleft{}{\pb}Wien\oindex{Wien@\textbf{Wien}, \emph{A.ADM2}|pw}, 16. Mai 97\pend
           \vspace{0.5em}
\pstart
           Lieber Arthur,{ }gestern{ }Abend erfuhr ich durch Zufall Ihre jetzige \label{K_L03265-1v}\edtext{Adresse}{\lemma{\textnormal{\emph{Adresse}}}\Cendnote{\textnormal{Salten\pwindex{Salten, Felix 06.09.1869 – 08.10.1945@\textsc{Salten, Felix} (06.09.1869 – 08.10.1945), \emph{Schriftsteller/Schriftstellerin, Journalist/Journalistin, Chefredakteur/Chefredakteurin}|pwk} bringt hier mehrere Dinge
                  durcheinander. Schnitzlers Adresse in Paris\oindex{Paris@\textbf{Paris}, \emph{P.PPLC}|pwk} war in der rue de Maubeuge\oindex{rue de Maubeuge@\textbf{rue de Maubeuge}, \emph{Straße (K.STR)}|pwk}, wohin Salten\pwindex{Salten, Felix 06.09.1869 – 08.10.1945@\textsc{Salten, Felix} (06.09.1869 – 08.10.1945), \emph{Schriftsteller/Schriftstellerin, Journalist/Journalistin, Chefredakteur/Chefredakteurin}|pwk} am
                     5. 5. 1897 geschrieben
                  haben dürfte. Schnitzler hat das
                  Schreiben auch erhalten, es ist in seinem Nachlass überliefert.
                  Die rue de la Bourse\oindex{rue de la Bourse@\textbf{rue de la Bourse}, \emph{Straße (K.STR)}|pwk} (was Salten\pwindex{Salten, Felix 06.09.1869 – 08.10.1945@\textsc{Salten, Felix} (06.09.1869 – 08.10.1945), \emph{Schriftsteller/Schriftstellerin, Journalist/Journalistin, Chefredakteur/Chefredakteurin}|pwk} im vorliegenden Brief als Adresse nennt) war die
                  Postadresse von Paul Goldmann\pwindex{Goldmann, Paul 31.01.1865 – 25.09.1935@\textsc{Goldmann, Paul} (31.01.1865 – 25.09.1935), \emph{Schriftsteller/Schriftstellerin, Journalist/Journalistin}|pwk}, die Schnitzler zu verwenden gebeten hatte, vgl. Arthur Schnitzler an Richard Beer-Hofmann, 19. 4. 1897.}}}\label{K_L03265-1}, und erklärte
               mir daraus, weshalb Sie mir wol bis heute nicht
               geantwortet haben. Offenbar haben Sie meinen \label{K_L03265-2v}\edtext{Brief nicht erhalten, den ich Ihnen vor mehr als vierzehn
                  Tagen}{\lemma{\textnormal{\emph{Brief … Tagen}}}\Cendnote{\textnormal{Da der Brief vom 5. 5. 1897 erhalten ist,
                  dürfte Schnitzler ihn regulär erhalten, aber auf eine unmittelbare Antwort verzichtet haben – oder diese ging
                  verlustig. Jedenfalls irrt Salten\pwindex{Salten, Felix 06.09.1869 – 08.10.1945@\textsc{Salten, Felix} (06.09.1869 – 08.10.1945), \emph{Schriftsteller/Schriftstellerin, Journalist/Journalistin, Chefredakteur/Chefredakteurin}|pwk}, sein
                  Schreiben lag noch keine zwei Wochen zurück.}}}\label{K_L03265-2} schrieb. Ich kam Ende April aus Riva\oindex{Riva del Garda@\textbf{Riva del Garda}, \emph{P.PPLA3}|pw} zurück
               und fand Ihre Karte und Ihren Brief. Darauf habe ich ziemlich ausführlich erwiedert
               und, da Sie es zu wünschen schienen, über mein Leben und meine Arbeiten ec.
               berichtet. Auf die Adresse schrieb ich nach Ihrer Angabe rue de la Bourse\oindex{rue de la Bourse@\textbf{rue de la Bourse}, \emph{Straße (K.STR)}|pw}. Offenbar haben Sie dieses Schreiben nicht
               erhalten, und da ich hier mit Niemandem verkehre, habe ich erst gestern{ }Abend Ihre neue Wohnung erfahren und glaube, Ihnen das zur Aufklärung
               sagen zu müßen.\pend
           
\pstart
           Herzlich {\\[\baselineskip]}\spacefill\mbox{Salten}\pend
           \leftskip=0em{}\selectlanguage{ngerman}\endnumbering\briefempfaengerindex{Schnitzler, Arthur@\textsc{Schnitzler, Arthur}!zzzSalten, Felix@\emph{von Felix Salten}!1897-05-161@{16. 5. 1897}|)be}\mylabel{L03265h}  \normalsize

\doendnotes{C}
\bigskip
\vfill

\clearpage

\footnotesize

\lohead{\textsc{register}}

% Definiere theindex-Environment komplett neu ohne reledmac
\makeatletter
\renewenvironment{theindex}{%
  \section*{\indexname}%
  \setlength{\parindent}{0pt}%
  \setlength{\parskip}{0pt plus 0.3pt}%
  \let\item\@idxitem
}{%
  \clearpage
}
\makeatother

\IfFileExists{\jobname-pw.ind}{\input{\jobname-pw.ind}}{}

\end{document}

      