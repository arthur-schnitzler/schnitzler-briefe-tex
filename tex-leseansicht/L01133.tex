%% latex-leseansicht-vorspann.tex
%% Vorspann für die Leseansicht.
%% Lädt die gemeinsame Datei latex-vorspann.tex mit nicht gesetztem Schalter.

\newif\ifkorrekturansicht
\korrekturansichtfalse

\input{../tex-inputs/latex-vorspann}


\section[Arthur Schnitzler an Richard Beer-Hofmann, 25. 6. 1901]{L01133 Arthur Schnitzler an Richard Beer-Hofmann, 25. 6. 1901}
\nopagebreak\mylabel{L01133v}
\rehead{ }\normalsize\beginnumbering\briefempfaengerindex{Beer-Hofmann, Richard@\textsc{Beer-Hofmann, Richard}!zzzSchnitzler, Arthur@\emph{von Arthur Schnitzler}!1901-06-251@{25. 6. 1901}|(be}
\toendnotes[C]{\smallbreak\pagebreak[2]}
\correspDesc{Versand  durch Arthur Schnitzler am 25. 6. 1901 in Salzburg
\newline{}Erhalt  durch Richard Beer-Hofmann im Zeitraum [26. 6. 1901
                  – 30. 6. 1901?] in Pörtschach}\toendnotes[C]{\smallbreak}
\Standort{YCGL, MSS 31.}
\physDesc{Bildpostkarte, 185 Zeichen
\newline{}Handschrift: Bleistift, deutsche Kurrent
\newline{}Versand: 1) Stempel: »\nobreak{}\oindex{Hauptbahnhof Salzburg@\textbf{Hauptbahnhof Salzburg}, \emph{Bahnhofsgebäude}|pwk}Salzburg Bahnhof, 25/6 \textcolor{gray}{01}, 7–N\nobreak{}«.   2) Stempel: »\nobreak{}\oindex{Pörtschach am Wörthersee@\textbf{Pörtschach am Wörthersee}|pwk}{[}Pörtscha{]}ch \textcolor{gray}{am}
                                       See\nobreak{}«. 
\newline{}Ordnung: mit Bleistift von unbekannter Hand datiert: »25. 6.« }\pstart{}{\pb}Hrn Dr. \textsc{Richard
                     Beer-Hofmann}\pend{}\pstart{}\textsc{Villa Arnstein}\oindex{Villa Arnstein@\textbf{Villa Arnstein}, \emph{Wohngebäude}|pw}\pend{}\pstart{}\textsc{Pörtschach} am Wörtherſee\oindex{Pörtschach am Wörthersee@\textbf{Pörtschach am Wörthersee}|pw}\pend{}{\bigskip}
\pstart
           \noindent{}\centering{}{\pb}\textcolor{gray}{\textbf{Mirabellgarten in Salzburg\oindex{Schloss Mirabell@\textbf{Schloss Mirabell}, \emph{Schloss}|pw}.\hspace*{1.5em}(J.
                     Forster\pwindex{Forster, Jacob 1.\,5.\,1857 Laufen – 5.\,1.\,1920 Salzburg@\textsc{Forster, Jacob} (1.\,5.\,1857 Laufen – 5.\,1.\,1920 Salzburg), \emph{Maler}|pw}).}}\pend
           \vspace{1em}
\pstart
           {\pb}25. 6. 901\pend
           \vspace{0.5em}
\pstart
           Herzlichen Gruſs. Ich fahre heut, ich will heute nach \textsc{Innsbruck\oindex{Innsbruck@\textbf{Innsbruck}, \emph{Verwaltungsgebiet}|pw}} fahren. Schreiben Sie doch eine Zeile (Wien\oindex{Wien@\textbf{Wien}, \emph{Verwaltungsgebiet}|pw})\pend
           \pstart Ihr\spacefill\mbox{A.}\pend{}\selectlanguage{ngerman}\endnumbering\briefempfaengerindex{Beer-Hofmann, Richard@\textsc{Beer-Hofmann, Richard}!zzzSchnitzler, Arthur@\emph{von Arthur Schnitzler}!1901-06-251@{25. 6. 1901}|)be}\mylabel{L01133h}  \newcommand{\dateiname}{L01133}\newcommand{\titel}{Arthur Schnitzler an Richard Beer-Hofmann, 25. 6. 1901}\newcommand{\editorInnen}{Martin Anton Müller und Gerd-Hermann Susen}%% latex-leseansicht-abspann.tex
%% Abspann für die Leseansicht.
%% Der Schalter \ifkorrekturansicht ist bereits durch den Vorspann gesetzt.

%% latex-abspann.tex
%% Gemeinsamer Abspann für Korrekturansicht und Leseansicht.
%% Setzt den Schalter \ifkorrekturansicht voraus (gesetzt in den
%% einbindenden Dateien latex-korrekturansicht-abspann.tex bzw.
%% latex-leseansicht-abspann.tex).
%% ---------------------------------------------------------------

\normalsize

% Das esempio-Environment wird nur in der Leseansicht benötigt
\ifkorrekturansicht\else
\newenvironment{esempio}[3]%
{
    \vspace{1.5ex}
    \rlap{\underline{#1}}
    \par
    \setlength{\parindent}{0cm}
    \nopagebreak
    \leftskip=#2cm
    \rightskip=#3cm
}
{
    \par
}
\fi

\doendnotes{C}
\bigskip
\vfill

\clearpage

\footnotesize

\ifkorrekturansicht
  \lohead{\textsc{register}}
\fi

% theindex-Environment neu definieren ohne reledmac
\makeatletter
\renewenvironment{theindex}{%
  \ifkorrekturansicht
    \section*{\indexname}%
  \else
    \subsubsection*{Index der erwähnten Entitäten}%
  \fi
  \setlength{\parindent}{0pt}%
  \setlength{\parskip}{0pt plus 0.3pt}%
  \let\item\@idxitem
}{%
  \ifkorrekturansicht\clearpage\fi
}
\makeatother

\IfFileExists{\jobname-pw.ind}{\input{\jobname-pw.ind}}{}

% Quellenangabe nur in der Leseansicht
\ifkorrekturansicht\else
% Fallback-Definitionen, falls die .tex-Datei \titel etc. nicht gesetzt hat
\providecommand{\titel}{}
\providecommand{\editorInnen}{}
\providecommand{\dateiname}{\jobname}

\vspace{3cm}

\vfill

\footnotesize
\textsc{Quelle}: \titel. Herausgegeben von {\editorInnen}. In: \emph{Arthur Schnitzler: Briefwechsel mit Autorinnen und Autoren}.
 Digitale Edition, https://schnitzler-briefe.acdh.oeaw.ac.at/{\dateiname}.html (Stand \today)
\fi

\end{document}


