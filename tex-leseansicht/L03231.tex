%% latex-leseansicht-vorspann.tex
%% Vorspann für die Leseansicht.
%% Lädt die gemeinsame Datei latex-vorspann.tex mit nicht gesetztem Schalter.

\newif\ifkorrekturansicht
\korrekturansichtfalse

\input{../tex-inputs/latex-vorspann}

\begin{center}
            \textcolor{red}{ENTWURF, NICHT FERTIG KORRIGIERT}
                      \end{center}
            
         
         \renewcommand{\erwaehntePersonen}{Personen: Olga Schnitzler}
         \renewcommand{\erwaehnteOrte}{Orte: Frankfurt am Main, Wien}
         \renewcommand{\erwaehnteWerke}{}
               \section[ Paul Goldmann an Arthur Schnitzler, 28. 12. {[}1902{]}]{ Paul Goldmann an Arthur Schnitzler, 28. 12. {[}1902{]}}\nopagebreak\mylabel{v}\rehead{ }\begin{ledgroupsized}[t]{13cm}\normalsize\beginnumbering \toendnotes[C]{\smallbreak\pagebreak[2]} \Standort{DLA, A:Schnitzler, HS.NZ85.1.3172.}
\physDesc{Brief, 1 Blatt, 3 Seiten
\newline{}Handschrift: schwarze Tinte, deutsche Kurrent
\newline{}Schnitzler: mit Bleistift das Jahr »{[}1{]}902«
                                            vermerkt }\pstart
           \noindent{}{\pb}\textcolor{gray}{\textbf{TELEPHON{ }\textbf{4167.}}}\hfill \textcolor{gray}{\textbf{TELEGRAMM-ADRESSE:}}\pend
           \pstart
           \textcolor{gray}{\textbf{UND{ }\textbf{3940.}}}\hfill \textcolor{gray}{\textbf{\textbf{PALAST FÜRSTENHOF\textcolor{red}{\textsuperscript{\textbf{KEY}}}{ }FRANKFURTMAIN\oindex{Frankfurt am Main@\textbf{Frankfurt am Main}|pw}.}}}\pend
           \pstart
           \centering{}\textcolor{gray}{\textbf{\textbf{PALAST-HOTEL\textcolor{red}{\textsuperscript{\textbf{KEY}}}}}}\pend
           \pstart
           \noindent{}\centering{}\textcolor{gray}{\textbf{FÜRSTENHOF\textcolor{red}{\textsuperscript{\textbf{KEY}}}}}\pend
           \pstart
           \noindent{}\centering{}\textcolor{gray}{\textbf{LOUIS BOLLE-RITZ\textcolor{red}{\textsuperscript{\textbf{KEY}}}.}}\pend
           \pstart
           \noindent{}\centering{}\textcolor{gray}{\textbf{(KAISERSTRASSE\textcolor{red}{\textsuperscript{\textbf{KEY}}} – KRONPRINZENSTRASSE\textcolor{red}{\textsuperscript{\textbf{KEY}}})}}\pend
           \pstart
           \raggedleft{}\textcolor{gray}{\textbf{Frankfurt \textsuperscript{a} M.}}\oindex{Frankfurt am Main@\textbf{Frankfurt am Main}|pw}{ }28. Dezember.\pend
           \pstart\center{}Mein lieber Freund,\pend\pstart
           Ich habe Wochen verſtreichen laſſen müſſen, ehe ich für Deinen lieben Brief, der
                    mich ganz beſonders erfreut hat, weil er ſo viel Schönes über Dich ſelbſt
                    enthielt, auch nur danken konnte. Eine das gewöhnliche Maß noch weit
                    überſteigende Häufung von Arbeit (Du wirſt ſie ja ſelbſt in der N. Fr. Pr.\textcolor{red}{\textsuperscript{\textbf{KEY}}} beobachtet haben) war die Urſache. Hier in Frankfurt\textcolor{red}{\textsuperscript{\textbf{KEY}}}, wo ich, meiner Gewohnheit gemäß, die
                    Zeit von Weihnachten bis Neujahr verbringe, finde ich
                    endlich die {\pb} Muße, Dir zu ſchreiben. Freilich,
                    der ausführliche Brief, den ich plante, kommt wieder nicht zu Stande. Und das
                    geſchieht deshalb nicht, weil ich ſo Fürchterliches hier erlebe, daß ich nicht
                    fähig bin, zu ſchreiben. Meine Beziehungen zu der Frau\textcolor{red}{\textsuperscript{\textbf{KEY}}}, die Du kennſt, haben in dieſen Tagen ihr Ende gefunden. Durch
                    meine Schuld: Denn als ich vor drei Monaten allerlei Klatſch über ſie erfuhr,
                    ſtieß ich ſie fort. Sonst iſt ſie immer wiedergekommen. Diesmal aber habe ich
                    ihr offenbar Unrecht gethan. Und das Schlimmſte: es war ein Tröſter\textcolor{red}{\textsuperscript{\textbf{KEY}}} bei der Hand. Geſtern erhielt ich den
                    Abſchiedsbrief: »Lebe wohl! Du haſt ſchlecht an mir gehandelt! Ich kann Dir
                    nicht verzeihen. Ich habe einen Beſſeren\textcolor{red}{\textsuperscript{\textbf{KEY}}}
                    gefunden!« \pend
           \pstart
           Und das Entſetzliche iſt, daß ich ſie jetzt liebe, – liebe, wie ich ſie nie
                    geliebt habe. Und daß in meinem armen Leben nirgends ein Erſatz iſt und nie mehr
                    ſich finden wird. Ich erinnere mich nicht, jemals ſo gelitten zu haben. Am Tage
                    die Erinnerungen auf Schritt und Tritt – Nachts die Marter {\pb} der Gewiſſensvorwürfe! \pend
           \pstart
           Liebſter Freund! Verzeih’ mir, daß ich Dir nicht mehr, – daß ich Dir nicht
                    über Dich ſchreibe. Entſchuldige mich auch bei \textsc{Olga\pwindex{Schnitzler, Olga 17.01.1882 – 13.01.1970@\textsc{Schnitzler, Olga} (17.01.1882 – 13.01.1970), \emph{Schauspielerin, Sängerin}|pw}}, der ich von hier\textcolor{red}{\textsuperscript{\textbf{KEY}}} aus für ihren lieben Brief
                    danken wollte. Ich wünſche Euch Beiden\textcolor{red}{\textsuperscript{\textbf{KEY}}} ein
                    glückliches neues Jahr! {\\[\baselineskip]}Viele treue Grüße!\pend
           \leftskip=0em{}\pstart
           {\\[\baselineskip]}Dein\pend
           \leftskip=0em{}\pstart
           {\\[\baselineskip]}\spacefill\mbox{Paul Goldmann. }\pend
           \leftskip=0em{}
         
         \endnumbering\mylabel{h}\end{ledgroupsized}\begin{anhang}\end{anhang}\newcommand{\dateiname}{L03231}\newcommand{\titel}{Paul Goldmann an Arthur Schnitzler, 28. 12. [1902]}\newcommand{\editorInnen}{Martin Anton Müller und Laura Untner}%% latex-leseansicht-abspann.tex
%% Abspann für die Leseansicht.
%% Der Schalter \ifkorrekturansicht ist bereits durch den Vorspann gesetzt.

%% latex-abspann.tex
%% Gemeinsamer Abspann für Korrekturansicht und Leseansicht.
%% Setzt den Schalter \ifkorrekturansicht voraus (gesetzt in den
%% einbindenden Dateien latex-korrekturansicht-abspann.tex bzw.
%% latex-leseansicht-abspann.tex).
%% ---------------------------------------------------------------

\normalsize

% Das esempio-Environment wird nur in der Leseansicht benötigt
\ifkorrekturansicht\else
\newenvironment{esempio}[3]%
{
    \vspace{1.5ex}
    \rlap{\underline{#1}}
    \par
    \setlength{\parindent}{0cm}
    \nopagebreak
    \leftskip=#2cm
    \rightskip=#3cm
}
{
    \par
}
\fi

\doendnotes{C}
\bigskip
\vfill

\clearpage

\footnotesize

\ifkorrekturansicht
  \lohead{\textsc{register}}
\fi

% theindex-Environment neu definieren ohne reledmac
\makeatletter
\renewenvironment{theindex}{%
  \ifkorrekturansicht
    \section*{\indexname}%
  \else
    \subsubsection*{Index der erwähnten Entitäten}%
  \fi
  \setlength{\parindent}{0pt}%
  \setlength{\parskip}{0pt plus 0.3pt}%
  \let\item\@idxitem
}{%
  \ifkorrekturansicht\clearpage\fi
}
\makeatother

\IfFileExists{\jobname-pw.ind}{\input{\jobname-pw.ind}}{}

% Quellenangabe nur in der Leseansicht
\ifkorrekturansicht\else
% Fallback-Definitionen, falls die .tex-Datei \titel etc. nicht gesetzt hat
\providecommand{\titel}{}
\providecommand{\editorInnen}{}
\providecommand{\dateiname}{\jobname}

\vspace{3cm}

\vfill

\footnotesize
\textsc{Quelle}: \titel. Herausgegeben von {\editorInnen}. In: \emph{Arthur Schnitzler: Briefwechsel mit Autorinnen und Autoren}.
 Digitale Edition, https://schnitzler-briefe.acdh.oeaw.ac.at/{\dateiname}.html (Stand \today)
\fi

\end{document}


      