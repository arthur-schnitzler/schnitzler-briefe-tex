%% latex-korrekturansicht-vorspann.tex
%% Vorspann für die Korrekturansicht.
%% Lädt die gemeinsame Datei latex-vorspann.tex mit gesetztem Schalter.

\newif\ifkorrekturansicht
\korrekturansichttrue

\input{../tex-inputs/latex-vorspann}


\section[ Paul Goldmann an Arthur Schnitzler, 28. 12. {[}1902{]}]{L03231 Paul Goldmann an Arthur Schnitzler, 28. 12. {[}1902{]}}
\nopagebreak\mylabel{L03231v}
\rehead{ }\normalsize\beginnumbering\briefempfaengerindex{Schnitzler, Arthur@\textsc{Schnitzler, Arthur}!zzzGoldmann, Paul@\emph{von Paul Goldmann}!1902-12-281@{28. 12. {[}1902{]}}|(be}
\toendnotes[C]{\smallbreak\pagebreak[2]}\Standort{DLA, A:Schnitzler, HS.NZ85.1.3172.}
\physDesc{Brief, 1 Blatt, 3 Seiten, 1763 Zeichen
\newline{}Handschrift: schwarze Tinte, deutsche Kurrent
\newline{}Schnitzler: mit Bleistift das Jahr »902« vermerkt }\toendnotes[C]{\smallbreak}
\pstart
           {\pb}\textcolor{gray}{\textbf{\textsc{Telephon}{ }4167. }}\hfill \textcolor{gray}{\textbf{\textsc{Telegramm-adresse}:}}\pend
           
\pstart
           \textcolor{gray}{\textbf{\textsc{und}{ }3940. }}\hfill \textcolor{gray}{\textbf{\textsc{Palast Fürstenhof\oindex{Fuerstenhof [Frankfurt am Main]@\textbf{Fürstenhof [Frankfurt am Main]}, \emph{Gebäude (K.GBD)}|pw}{ }Frankfurtmain\oindex{Frankfurt am Main@\textbf{Frankfurt am Main}, \emph{P.PPLA3}|pw}. }}}\pend
           
\pstart
           \centering{}\textcolor{gray}{\textbf{\textsc{\textbf{Palast-Hotel\orgindex{Hotel Fuerstenhof@Hotel Fürstenhof|pw}}}}}\pend
           
\pstart
           \centering{}\textcolor{gray}{\textbf{\textsc{Fürstenhof\orgindex{Hotel Fuerstenhof@Hotel Fürstenhof|pw}}}}\pend
           
\pstart
           \centering{}\textcolor{gray}{\textbf{\textsc{Louis Bolle-Ritz\pwindex{Bolle-Ritz, Louis @\textsc{Bolle-Ritz, Louis}, \emph{Hotelbesitzer/Hotelbesitzerin, Restaurateur/Restaurateurin}|pw}.}}}\pend
           
\pstart
           \centering{}\textcolor{gray}{\textbf{(\textsc{Kaiserstrasse\oindex{Kaiserstrasse [Frankfurt am Main]@\textbf{Kaiserstraße [Frankfurt am Main]}, \emph{Straße (K.STR)}|pw} – Kronprinzenstrasse\oindex{Muenchener Strasse@\textbf{Münchener Straße}, \emph{Straße (K.STR)}|pw}})}}\pend
           
\pstart
           \raggedleft{}\textcolor{gray}{\textbf{Frankfurt \textsuperscript{a/}M.}}\oindex{Frankfurt am Main@\textbf{Frankfurt am Main}, \emph{P.PPLA3}|pw}{ }28. Dezember.\pend
           
\pstart{}Mein lieber Freund,\pend\vspace{0.5em}
\pstart
           Ich habe Wochen verſtreichen laſſen müſſen, ehe ich für Deinen lieben Brief, der mich
               ganz beſonders erfreut hat, weil er ſo viel Schönes über Dich ſelbſt enthielt, auch
               nur danken konnte. Eine das gewöhnliche Maß noch weit überſteigende Häufung von
               Arbeit (Du wirſt ſie ja ſelbſt in der N. Fr. Pr.\pwindex{Neue Freie Presse@\emph{Neue Freie Presse}|pw}
               beobachtet haben) war die Urſache. Hier in Frankfurt\oindex{Frankfurt am Main@\textbf{Frankfurt am Main}, \emph{P.PPLA3}|pw}, wo ich, meiner Gewohnheit gemäß, die Zeit von Weihnachten bis Neujahr
               verbringe, finde ich endlich die {\pb}Muße, Dir zu
               ſchreiben. Freilich, der ausführliche Brief, den ich plante, kommt wieder nicht zu
               Stande. Und das geſchieht deshalb nicht, weil ich ſo Fürchterliches hier erlebe, daß
               ich nicht fähig bin, zu ſchreiben. Meine Beziehungen zu der Frau\pwindex{Rottenberg, Theodore 1875-09-07 – 1945-04-05@\textsc{Rottenberg, Theodore} (1875-09-07 – 1945-04-05)|pwv}, die Du kennſt, haben in dieſen Tagen
               ihr Ende gefunden. Durch meine Schuld: Denn als ich vor drei Monaten allerlei Klatſch
               über ſie erfuhr, ſtieß ich ſie fort. Sonst iſt ſie immer wiedergekommen. Diesmal aber
               habe ich ihr offenbar ſchwer Unrecht gethan. Und das Schlimmſte: es war ein \label{K_L03231-1v}\edtext{Tröſter\pwindex{?? [Partner von Theodore Rottenberg, Ende 1902/Anfang 1903] @\textsc{?? [Partner von Theodore Rottenberg, Ende 1902/Anfang 1903]}|pwv}}{\lemma{\textnormal{\emph{Tröſter}}}\Cendnote{\textnormal{nicht ermittelt}}}\label{K_L03231-1} bei der Hand.
                  Geſtern erhielt ich den Abſchiedsbrief: »Lebe wohl!
               Du haſt ſchlecht an mir gehandelt! Ich kann Dir nicht verzeihen. Ich habe einen Beſſeren\pwindex{?? [Partner von Theodore Rottenberg, Ende 1902/Anfang 1903] @\textsc{?? [Partner von Theodore Rottenberg, Ende 1902/Anfang 1903]}|pwv} gefunden!«\pend
           
\pstart
           Und das Entſetzliche iſt, daß ich ſie jetzt liebe, – liebe, wie ich ſie nie geliebt
               habe. Und daß in meinem armen Leben nirgends ein Erſatz iſt und nie mehr ſich finden
               wird. Ich erinnere mich nicht, jemals ſo gelitten zu haben. Am Tage die Erinnerungen
               auf Schritt und Tritt – Nachts die Marter {\pb}der
               Gewiſſensvorwürfe!\pend
           
\pstart
           Liebſter Freund! Verzeih’ mir, daß ich Dir nicht mehr, – daß ich Dir nicht über Dich
               ſchreibe. Entſchuldige mich auch bei \textsc{Olga\pwindex{Schnitzler, Olga 17.01.1882 – 13.01.1970@\textsc{Schnitzler, Olga} (17.01.1882 – 13.01.1970), \emph{Schauspieler/Schauspielerin, Sänger/Sängerin}|pw}}, der ich von hier\oindex{Frankfurt am Main@\textbf{Frankfurt am Main}, \emph{P.PPLA3}|pwv} aus für
               ihren lieben Brief danken wollte.\pend
           
\pstart
           Ich wünſche Euch Beiden\pwindex{Schnitzler, Olga 17.01.1882 – 13.01.1970@\textsc{Schnitzler, Olga} (17.01.1882 – 13.01.1970), \emph{Schauspieler/Schauspielerin, Sänger/Sängerin}|pwv} ein
               glückliches neues Jahr!\pend
           
\pstart
           Viele treue Grüße! {\\[\baselineskip]}Dein {\\[\baselineskip]}\spacefill\mbox{Paul Goldmann.}\pend
           \leftskip=0em{}\selectlanguage{ngerman}\endnumbering\briefempfaengerindex{Schnitzler, Arthur@\textsc{Schnitzler, Arthur}!zzzGoldmann, Paul@\emph{von Paul Goldmann}!1902-12-281@{28. 12. {[}1902{]}}|)be}\mylabel{L03231h}  \normalsize

\doendnotes{C}
\bigskip
\vfill

\clearpage

\footnotesize

\lohead{\textsc{register}}

% Definiere theindex-Environment komplett neu ohne reledmac
\makeatletter
\renewenvironment{theindex}{%
  \section*{\indexname}%
  \setlength{\parindent}{0pt}%
  \setlength{\parskip}{0pt plus 0.3pt}%
  \let\item\@idxitem
}{%
  \clearpage
}
\makeatother

\IfFileExists{\jobname-pw.ind}{\input{\jobname-pw.ind}}{}

\end{document}

      