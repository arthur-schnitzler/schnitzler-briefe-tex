%% latex-leseansicht-vorspann.tex
%% Vorspann für die Leseansicht.
%% Lädt die gemeinsame Datei latex-vorspann.tex mit nicht gesetztem Schalter.

\newif\ifkorrekturansicht
\korrekturansichtfalse

\input{../tex-inputs/latex-vorspann}


\section[ Paul Goldmann an Arthur Schnitzler, 2. 4. [1896]]{L02770 Paul Goldmann an Arthur Schnitzler,  2. 4. [1896]}
\nopagebreak\mylabel{L02770v}
\rehead{ }\normalsize\beginnumbering\briefempfaengerindex{Schnitzler, Arthur@\textsc{Schnitzler, Arthur}!zzzGoldmann, Paul@\emph{von Paul Goldmann}!1896-04-021@{2. 4. [1896]}|(be}
\toendnotes[C]{\smallbreak\pagebreak[2]}
\correspDesc{Versand  durch Paul Goldmann am 2. 4. [1896] in Paris
\newline{}Erhalt  durch Arthur Schnitzler im Zeitraum [3. 4. 1896
                  – 7. 4. 1896?] in Wien}\toendnotes[C]{\smallbreak}
\Standort{DLA, A:Schnitzler, HS.NZ85.1.3166.}
\physDesc{Brief, 1 Blatt, 4 Seiten, 1616 Zeichen
\newline{}Handschrift: blaue Tinte, deutsche Kurrent
\newline{}Schnitzler: 1) mit Bleistift das Jahr »96« vermerkt  2) mit rotem Buntstift eine Unterstreichung}\toendnotes[C]{\smallbreak}
\pstart
           {\pb}\textcolor{gray}{\textbf{\textbf{Frankfurter Zeitung\orgindex{Frankfurter Zeitung@Frankfurter Zeitung|pw}}}}\pend
           
\pstart
           \textcolor{gray}{\textbf{(\begin{otherlanguage}{french}Gazette de Francfort\end{otherlanguage}\orgindex{Frankfurter Zeitung@Frankfurter Zeitung|pw}).}}\pend
           
\pstart
           \textcolor{gray}{\textbf{\textbf{\begin{otherlanguage}{french}Fondateur M.\end{otherlanguage}{ }L. Sonnemann\pwindex{Sonnemann, Leopold 29.\,10.\,1831 Höchberg – 30.\,10.\,1909 Frankfurt am Main@\textsc{Sonnemann, Leopold} (29.\,10.\,1831 Höchberg – 30.\,10.\,1909 Frankfurt am Main), \emph{Journalist, Herausgeber}|pw}.}}}\pend
           
\pstart
           \begin{otherlanguage}{french}\textcolor{gray}{\textbf{Journal\pwindex{Frankfurter Zeitung@\emph{Frankfurter Zeitung}|pwv} politique,
                        financier,}}\end{otherlanguage}\pend
           
\pstart
           \begin{otherlanguage}{french}\textcolor{gray}{\textbf{commercial et littéraire.}}\end{otherlanguage}\pend
           
\pstart
           \begin{otherlanguage}{french}\textcolor{gray}{\textbf{\textbf{Paraissant trois fois par jour.}}}\end{otherlanguage}\pend
           
\pstart
           \begin{otherlanguage}{french}\textcolor{gray}{\textbf{\textbf{Bureau à Paris\oindex{Paris@\textbf{Paris}, \emph{Hauptstadt}|pw}:}}}\end{otherlanguage}\pend
           
\pstart
           \begin{otherlanguage}{french}\textcolor{gray}{\textbf{\textbf{24. Rue Feydeau\oindex{rue Feydeau@\textbf{rue Feydeau}, \emph{Straße}|pw}.}}}\end{otherlanguage}\hfill \textsc{Paris\oindex{Paris@\textbf{Paris}, \emph{Hauptstadt}|pw}}, 2. April.\pend
           
\pstart\center{}Mein lieber Freund,\pend\vspace{0.5em}
\pstart
           Das iſt auch noch nicht der lange Brief,{ }ſondern nur eine Nachſchrift zum geſtrigen.
               Ich empfing geſtern{ }Nachmittag den Beſuch des \textsc{M. Schefer\pwindex{Schefer, Christian 14.\,7.\,1866 Paris – Februar 1944 Marokko@\textsc{Schefer, Christian} (14.\,7.\,1866 Paris – Februar 1944 Marokko), \emph{Journalist, Lehrer}|pw}}, eines geſcheiten und vornehmen Mannes (Profeſſor an der \textsc{École des Sciences Politiques\orgindex{École libre des sciences politiques@École libre des sciences politiques|pw} etc.}), der
               in der »\textsc{Nouvelle Revue\pwindex{Nouvelle Revue@\emph{La Nouvelle Revue}|pw}}«, die zu den angeſehenſten und geleſenſten Revuen gehört, nächſtens eine Rubrik
               über auswärtige Literatur eröffnen wird. Er will das nicht{ }ſo oberflächlich machen,
               wie dies{ }ſonſt hier geſchieht, will gründlich auf die Sache {\pb}eingehen und alle Zuſammenhänge beleuchten. Er frug
               mich um Rath wegen des deutſchen Geiſteslebens und wollte den Namen eines neuen
               Talents wiſſen, mit dem er{ }ſeine Beſprechungen über deutſche Literatur einleiten
               könnte. Du kannſt Dir denken, daß ich eifrig die Gelegenheit ergriff, um ihm von Dir
               zu{ }ſprechen. Es{ }ſcheint, daß Du gerade das biſt, was er braucht, er war ganz Feuer
               und Flamme, nahm mir mein Exemplar von der »Liebelei\pwindex{Schnitzler, Arthur 15.\,5.\,1862 Wien – 21.\,10.\,1931 ebd.@\textsc{Schnitzler, Arthur} (15.\,5.\,1862 Wien – 21.\,10.\,1931 ebd.), \emph{Schriftsteller, Mediziner}!Liebelei. Schauspiel in drei Akten@\strich\emph{Liebelei. Schauspiel in drei Akten}|pw}« weg (was er lieber hätte nicht thun {\pb}ſollen), ließ{ }ſich Deine Photographie zeigen und
               erwartet Deine Bücher\pwindex{Schnitzler, Arthur 15.\,5.\,1862 Wien – 21.\,10.\,1931 ebd.@\textsc{Schnitzler, Arthur} (15.\,5.\,1862 Wien – 21.\,10.\,1931 ebd.), \emph{Schriftsteller, Mediziner}!Sterben. Novelle@\strich\emph{Sterben. Novelle}|pwv}\pwindex{Schnitzler, Arthur 15.\,5.\,1862 Wien – 21.\,10.\,1931 ebd.@\textsc{Schnitzler, Arthur} (15.\,5.\,1862 Wien – 21.\,10.\,1931 ebd.), \emph{Schriftsteller, Mediziner}!Liebelei. Schauspiel in drei Akten@\strich\emph{Liebelei. Schauspiel in drei Akten}|pwv}\pwindex{Schnitzler, Arthur 15.\,5.\,1862 Wien – 21.\,10.\,1931 ebd.@\textsc{Schnitzler, Arthur} (15.\,5.\,1862 Wien – 21.\,10.\,1931 ebd.), \emph{Schriftsteller, Mediziner}!Anatol@\strich\emph{Anatol}|pwv}, deren Zuſendung ich ihm in Deinem Namen verſprochen habe. Bitte,{ }ſchicke ihm alſo: 1.) Sterben\pwindex{Schnitzler, Arthur 15.\,5.\,1862 Wien – 21.\,10.\,1931 ebd.@\textsc{Schnitzler, Arthur} (15.\,5.\,1862 Wien – 21.\,10.\,1931 ebd.), \emph{Schriftsteller, Mediziner}!Sterben. Novelle@\strich\emph{Sterben. Novelle}|pw} 2.) Liebelei\pwindex{Schnitzler, Arthur 15.\,5.\,1862 Wien – 21.\,10.\,1931 ebd.@\textsc{Schnitzler, Arthur} (15.\,5.\,1862 Wien – 21.\,10.\,1931 ebd.), \emph{Schriftsteller, Mediziner}!Liebelei. Schauspiel in drei Akten@\strich\emph{Liebelei. Schauspiel in drei Akten}|pw} 3.) \textsc{Anatol\pwindex{Schnitzler, Arthur 15.\,5.\,1862 Wien – 21.\,10.\,1931 ebd.@\textsc{Schnitzler, Arthur} (15.\,5.\,1862 Wien – 21.\,10.\,1931 ebd.), \emph{Schriftsteller, Mediziner}!Anatol@\strich\emph{Anatol}|pw}}. Schreibe in eines der Bücher\pwindex{Schnitzler, Arthur 15.\,5.\,1862 Wien – 21.\,10.\,1931 ebd.@\textsc{Schnitzler, Arthur} (15.\,5.\,1862 Wien – 21.\,10.\,1931 ebd.), \emph{Schriftsteller, Mediziner}!Sterben. Novelle@\strich\emph{Sterben. Novelle}|pwv}\pwindex{Schnitzler, Arthur 15.\,5.\,1862 Wien – 21.\,10.\,1931 ebd.@\textsc{Schnitzler, Arthur} (15.\,5.\,1862 Wien – 21.\,10.\,1931 ebd.), \emph{Schriftsteller, Mediziner}!Liebelei. Schauspiel in drei Akten@\strich\emph{Liebelei. Schauspiel in drei Akten}|pwv}\pwindex{Schnitzler, Arthur 15.\,5.\,1862 Wien – 21.\,10.\,1931 ebd.@\textsc{Schnitzler, Arthur} (15.\,5.\,1862 Wien – 21.\,10.\,1931 ebd.), \emph{Schriftsteller, Mediziner}!Anatol@\strich\emph{Anatol}|pwv} (oder in alle) \label{K_L02770-1v}\edtext{\begin{otherlanguage}{french}\textsc{À Monsieur Schefer,\pwindex{Schefer, Christian 14.\,7.\,1866 Paris – Februar 1944 Marokko@\textsc{Schefer, Christian} (14.\,7.\,1866 Paris – Februar 1944 Marokko), \emph{Journalist, Lehrer}|pw}
                     Hommage de l’auteur}\end{otherlanguage}}{\lemma{\textnormal{\emph{À … l’auteur}}}\Cendnote{\textnormal{französisch: an Herrn Schefer\pwindex{Schefer, Christian 14.\,7.\,1866 Paris – Februar 1944 Marokko@\textsc{Schefer, Christian} (14.\,7.\,1866 Paris – Februar 1944 Marokko), \emph{Journalist, Lehrer}|pwk}, Zueignung des Autors}}}\label{K_L02770-1}, mit Unterzeichnung Deines Namens. Ich hoffe,
               das wird gute \label{K_L02770-2v}\edtext{Früchte tragen}{\lemma{\textnormal{\emph{Früchte tragen}}}\Cendnote{\textnormal{Christian Schefer\pwindex{Schefer, Christian 14.\,7.\,1866 Paris – Februar 1944 Marokko@\textsc{Schefer, Christian} (14.\,7.\,1866 Paris – Februar 1944 Marokko), \emph{Journalist, Lehrer}|pwk}: \emph{Un jeune écrivain viennois: M. Arthur Schnitzler}\pwindex{Schefer, Christian 14.\,7.\,1866 Paris – Februar 1944 Marokko@\textsc{Schefer, Christian} (14.\,7.\,1866 Paris – Februar 1944 Marokko), \emph{Journalist, Lehrer}!Un jeune écrivain viennois: M. Arthur Schnitzler@\strich\emph{Un jeune écrivain viennois: M. Arthur Schnitzler}|pwk}. In:
                        \emph{La Nouvelle Revue}\pwindex{Nouvelle Revue@\emph{La Nouvelle Revue}|pwk}, Jg. 18, Nr. 100,
                        Mai–Juni 1896,
                     S. 855–859.}}}\label{K_L02770-2}; auch eröffnet mir das eine neue Perſpective für die
                  Überſetzung\pwindex{Schnitzler, Arthur 15.\,5.\,1862 Wien – 21.\,10.\,1931 ebd.@\textsc{Schnitzler, Arthur} (15.\,5.\,1862 Wien – 21.\,10.\,1931 ebd.), \emph{Schriftsteller, Mediziner}!Amourette. Pièce en trois actes. Adaptée de Arthur Schnitzler@\strich\emph{Amourette. Pièce en trois actes. Adaptée de Arthur Schnitzler}|pwv}s-Angelegenheit,
               und wir wollen daher dieß, wenns Dir recht iſt, {\pb}noch ein wenig aufſchieben. Adreſſe: \textsc{M. Schefer\pwindex{Schefer, Christian 14.\,7.\,1866 Paris – Februar 1944 Marokko@\textsc{Schefer, Christian} (14.\,7.\,1866 Paris – Februar 1944 Marokko), \emph{Journalist, Lehrer}|pw}, »Nouvelle
                     Revue\orgindex{Nouvelle Revue@Nouvelle Revue|pw}«, 18. Boulevard Montmartre,
                     Paris\oindex{Boulevard Montmartre@\textbf{Boulevard Montmartre}, \emph{Straße}|pw}}. (\uline{Kein} Begleitbrief.)\pend
           
\pstart
           Grüß’ Dich Gott, liebſter Freund!\pend
           
\pstart
           Dein {\\[\baselineskip]}\spacefill\mbox{Paul Goldmn}\pend
           \leftskip=0em{}\selectlanguage{ngerman}\endnumbering\briefempfaengerindex{Schnitzler, Arthur@\textsc{Schnitzler, Arthur}!zzzGoldmann, Paul@\emph{von Paul Goldmann}!1896-04-021@{2. 4. [1896]}|)be}\mylabel{L02770h}  \newcommand{\dateiname}{L02770}\newcommand{\titel}{Paul Goldmann an Arthur Schnitzler, 2. 4. [1896]}\newcommand{\editorInnen}{Martin Anton Müller und Laura Untner}%% latex-leseansicht-abspann.tex
%% Abspann für die Leseansicht.
%% Der Schalter \ifkorrekturansicht ist bereits durch den Vorspann gesetzt.

%% latex-abspann.tex
%% Gemeinsamer Abspann für Korrekturansicht und Leseansicht.
%% Setzt den Schalter \ifkorrekturansicht voraus (gesetzt in den
%% einbindenden Dateien latex-korrekturansicht-abspann.tex bzw.
%% latex-leseansicht-abspann.tex).
%% ---------------------------------------------------------------

\normalsize

% Das esempio-Environment wird nur in der Leseansicht benötigt
\ifkorrekturansicht\else
\newenvironment{esempio}[3]%
{
    \vspace{1.5ex}
    \rlap{\underline{#1}}
    \par
    \setlength{\parindent}{0cm}
    \nopagebreak
    \leftskip=#2cm
    \rightskip=#3cm
}
{
    \par
}
\fi

\doendnotes{C}
\bigskip
\vfill

\clearpage

\footnotesize

\ifkorrekturansicht
  \lohead{\textsc{register}}
\fi

% theindex-Environment neu definieren ohne reledmac
\makeatletter
\renewenvironment{theindex}{%
  \ifkorrekturansicht
    \section*{\indexname}%
  \else
    \subsubsection*{Index der erwähnten Entitäten}%
  \fi
  \setlength{\parindent}{0pt}%
  \setlength{\parskip}{0pt plus 0.3pt}%
  \let\item\@idxitem
}{%
  \ifkorrekturansicht\clearpage\fi
}
\makeatother

\IfFileExists{\jobname-pw.ind}{\input{\jobname-pw.ind}}{}

% Quellenangabe nur in der Leseansicht
\ifkorrekturansicht\else
% Fallback-Definitionen, falls die .tex-Datei \titel etc. nicht gesetzt hat
\providecommand{\titel}{}
\providecommand{\editorInnen}{}
\providecommand{\dateiname}{\jobname}

\vspace{3cm}

\vfill

\footnotesize
\textsc{Quelle}: \titel. Herausgegeben von {\editorInnen}. In: \emph{Arthur Schnitzler: Briefwechsel mit Autorinnen und Autoren}.
 Digitale Edition, https://schnitzler-briefe.acdh.oeaw.ac.at/{\dateiname}.html (Stand \today)
\fi

\end{document}


