%% latex-leseansicht-vorspann.tex
%% Vorspann für die Leseansicht.
%% Lädt die gemeinsame Datei latex-vorspann.tex mit nicht gesetztem Schalter.

\newif\ifkorrekturansicht
\korrekturansichtfalse

\input{../tex-inputs/latex-vorspann}


\section[Arthur Schnitzler an Stefan Großmann, 17. 2. 1921]{L02363 Arthur Schnitzler an Stefan Großmann, 17. 2. 1921}
\nopagebreak\mylabel{L02363v}
\rehead{ }\normalsize\beginnumbering\briefempfaengerindex{Großmann, Stefan@\textsc{Großmann, Stefan}!zzzSchnitzler, Arthur@\emph{von Arthur Schnitzler}!1921-02-171@{17. 2. 1921}|(be}
\toendnotes[C]{\smallbreak\pagebreak[2]}
\correspDesc{Versand  durch Arthur Schnitzler am 17. 2. 1921 in Wien
\newline{}Erhalt  durch Stefan Großmann im Zeitraum [18. 2. 1921
                  – 22. 2. 1921?] in Berlin}\toendnotes[C]{\smallbreak}
\Standort{DLA, A:Schnitzler, HS.NZ85.1.896.}
\physDesc{Brief, Durchschlag, 1 Blatt, 1 Seite, 1469 Zeichen
\newline{}Schreibmaschine
\newline{}Handschrift: roter Buntstift, deutsche Kurrent (\noindent{}Beschriftung: »K{[}opie{]}«, Unterstreichungen)}
\buchAbdrucke{\weitereDrucke{1) \pwindex{Schnitzler, Arthur 15.\,5.\,1862 Wien – 21.\,10.\,1931 ebd.@\textsc{Schnitzler, Arthur} (15.\,5.\,1862 Wien – 21.\,10.\,1931 ebd.), \emph{Schriftsteller, Mediziner}!Reigen der Gassenjungen@\strich\emph{Der Reigen der Gassenjungen}|pwk}\pwindex{Tage-Buch@\emph{Das Tage-Buch}|pwk}Stefan Großmann: \emph{Der Reigen der Gassenjungen.} In: \emph{Das Tage-Buch}, Jg. 2, Nr. 8, 26. 2. 1921, S. 252–253.} \weitereDrucke{2) Arthur Schnitzler: \emph{Briefe 1913–1931}. Herausgegeben von Peter Michael Braunwarth, Richard Miklin, Susanne Pertlik und Heinrich Schnitzler. Frankfurt am Main: \emph{S. Fischer} 1984, S. 234–235.} }\toendnotes[C]{\smallbreak}
\pstart
           \raggedleft{}{\pb}17. 2. 1921.\pend
           
\pstart{}Sehr verehrter Herr Grossmann.\pend\vspace{0.5em}
\pstart
           Vielen Dank für Ihr freundliches Interesse. Sie haben indess wohl meine Karte
               erhalten, in der ich Ihnen sagte, wie sehr mich Ihr parodistischer Dialog\pwindex{Großmann, Stefan 19.\,5.\,1875 Wien – 3.\,1.\,1935 ebd.@\textsc{Großmann, Stefan} (19.\,5.\,1875 Wien – 3.\,1.\,1935 ebd.), \emph{Schriftsteller, Journalist}!Hänischs Reigen. Eine unsittliche Szenenfolge@\strich\emph{Hänischs Reigen. Eine unsittliche Szenenfolge}|pwv} amüsiert hat. Ich habe vorläufig
               keine Absicht mich über den »Reigen\pwindex{Schnitzler, Arthur 15.\,5.\,1862 Wien – 21.\,10.\,1931 ebd.@\textsc{Schnitzler, Arthur} (15.\,5.\,1862 Wien – 21.\,10.\,1931 ebd.), \emph{Schriftsteller, Mediziner}!Reigen. Zehn Dialoge@\strich\emph{Reigen. Zehn Dialoge}|pw}« und die
               sogenannnte Reigen\pwindex{Schnitzler, Arthur 15.\,5.\,1862 Wien – 21.\,10.\,1931 ebd.@\textsc{Schnitzler, Arthur} (15.\,5.\,1862 Wien – 21.\,10.\,1931 ebd.), \emph{Schriftsteller, Mediziner}!Reigen. Zehn Dialoge@\strich\emph{Reigen. Zehn Dialoge}|pw}-Affaire in der
               Oeffentlichkeit weiter zu äussern. Was ich {\pb}Herrn Maximilian Harden\pwindex{Harden, Maximilian 20.\,10.\,1861 Berlin – 30.\,10.\,1927 Montana@\textsc{Harden, Maximilian} (20.\,10.\,1861 Berlin – 30.\,10.\,1927 Montana), \emph{Schriftsteller, Publizist}|pw}\pwindex{Harden, Maximilian 20.\,10.\,1861 Berlin – 30.\,10.\,1927 Montana@\textsc{Harden, Maximilian} (20.\,10.\,1861 Berlin – 30.\,10.\,1927 Montana), \emph{Schriftsteller, Publizist}!Reigen@\strich\emph{Reigen}|pwv}{ }erwidert\pwindex{Schnitzler, Arthur 15.\,5.\,1862 Wien – 21.\,10.\,1931 ebd.@\textsc{Schnitzler, Arthur} (15.\,5.\,1862 Wien – 21.\,10.\,1931 ebd.), \emph{Schriftsteller, Mediziner}!Berichtigung. Ein paar Worte zum Gutachten Maximilian Hardens über den »Reigen«@\strich\emph{Berichtigung. Ein paar Worte zum Gutachten Maximilian Hardens über den »Reigen«}|pwv} habe, ersehen Sie aus
               beiliegendem \label{K_L02363-1v}\edtext{Zeitungsblatt}{\lemma{\textnormal{\emph{Zeitungsblatt}}}\Cendnote{\textnormal{Arthur Schnitzler: \emph{Berichtigung. Ein paar Worte zum Gutachten Maximilian
                        Hardens über den »Reigen«}\pwindex{Schnitzler, Arthur 15.\,5.\,1862 Wien – 21.\,10.\,1931 ebd.@\textsc{Schnitzler, Arthur} (15.\,5.\,1862 Wien – 21.\,10.\,1931 ebd.), \emph{Schriftsteller, Mediziner}!Berichtigung. Ein paar Worte zum Gutachten Maximilian Hardens über den »Reigen«@\strich\emph{Berichtigung. Ein paar Worte zum Gutachten Maximilian Hardens über den »Reigen«}|pwk} in: \emph{Neues
                        Wiener Journal}\pwindex{Neues Wiener Journal@\emph{Neues Wiener Journal}|pwk}, Jg. 29, Nr. 9782, 30. 1. 1921,
                  S. 6.}}}\label{K_L02363-1}. Die Berichtigung war übrigens in einigen Berlin\oindex{Berlin@\textbf{Berlin}, \emph{Hauptstadt}|pw}er Blättern abgedruckt. Von den hiesigen Skandalen,
               insbesondere von dem \label{K_L02363-2v}\edtext{gestrigen}{\lemma{\textnormal{\emph{gestrigen}}}\Cendnote{\textnormal{am 16. 2. 1921}}}\label{K_L02363-2}, werden Sie wohl indess gelesen haben. Was soll man dazu sagen? Ich käme mir
               unsäglich komisch vor, wollte ich mit den Herren Kuntschak\pwindex{Kunschak, Leopold 11.\,11.\,1871 Wien – 13.\,3.\,1953 ebd.@\textsc{Kunschak, Leopold} (11.\,11.\,1871 Wien – 13.\,3.\,1953 ebd.), \emph{Politiker}|pw} oder Seipel\pwindex{Seipel, Ignaz 19.\,7.\,1876 Wien – 2.\,8.\,1932 Pernitz@\textsc{Seipel, Ignaz} (19.\,7.\,1876 Wien – 2.\,8.\,1932 Pernitz), \emph{Politiker, Prälat, Bundeskanzler}|pw} oder mit dem
               Schusterlehrling polemisieren, der das Theater stürmt, mit dem begeisterten Ruf:
               Nieder mit dem Reigen! Man schändet unsere Frauen! Nieder mit den Sozialdemokraten!
               (Es kann übrigens auch ein Stud. med. gewesen sein oder ein Tapezierergehilfe, –
               wobei meine Sympathie immerhin noch mehr bei dem Tapezierergehilfen ist als bei den
               Herren Seipel\pwindex{Seipel, Ignaz 19.\,7.\,1876 Wien – 2.\,8.\,1932 Pernitz@\textsc{Seipel, Ignaz} (19.\,7.\,1876 Wien – 2.\,8.\,1932 Pernitz), \emph{Politiker, Prälat, Bundeskanzler}|pw} und Kuntschak\pwindex{Kunschak, Leopold 11.\,11.\,1871 Wien – 13.\,3.\,1953 ebd.@\textsc{Kunschak, Leopold} (11.\,11.\,1871 Wien – 13.\,3.\,1953 ebd.), \emph{Politiker}|pw}.{[}){]} Ich habe ja schon einige
               ähnliche Sachen erlebt, wenn auch in bescheideneren Dimensionen. Erinnern Sie sich
               nur an den »Leutnant Gustl\pwindex{Schnitzler, Arthur 15.\,5.\,1862 Wien – 21.\,10.\,1931 ebd.@\textsc{Schnitzler, Arthur} (15.\,5.\,1862 Wien – 21.\,10.\,1931 ebd.), \emph{Schriftsteller, Mediziner}!Lieutenant Gustl. Novelle@\strich\emph{Lieutenant Gustl. Novelle}|pw}« und den »Professor Bernhardi\pwindex{Schnitzler, Arthur 15.\,5.\,1862 Wien – 21.\,10.\,1931 ebd.@\textsc{Schnitzler, Arthur} (15.\,5.\,1862 Wien – 21.\,10.\,1931 ebd.), \emph{Schriftsteller, Mediziner}!Professor Bernhardi. Komödie in fünf Akten@\strich\emph{Professor Bernhardi. Komödie in fünf Akten}|pw}«. Nach einigen Jahren bleibt
               von all dem Lärm nichts weiter übrig als die Bücher, die ich geschrieben und eine
               dunkle Erinnerung an die Blamage meiner Gegner. In diesem Fall wird es nicht anders
               sein.\pend
           
\pstart
           Mit herzlichem Gruss{\\[\baselineskip]}Ihr sehr ergebener{\\[\baselineskip]}\pend
           \leftskip=0em{}\selectlanguage{ngerman}\endnumbering\briefempfaengerindex{Großmann, Stefan@\textsc{Großmann, Stefan}!zzzSchnitzler, Arthur@\emph{von Arthur Schnitzler}!1921-02-171@{17. 2. 1921}|)be}\mylabel{L02363h}  \newcommand{\dateiname}{L02363}\newcommand{\titel}{Arthur Schnitzler an Stefan Großmann, 17. 2. 1921}\newcommand{\editorInnen}{Martin Anton Müller und Gerd-Hermann Susen}%% latex-leseansicht-abspann.tex
%% Abspann für die Leseansicht.
%% Der Schalter \ifkorrekturansicht ist bereits durch den Vorspann gesetzt.

%% latex-abspann.tex
%% Gemeinsamer Abspann für Korrekturansicht und Leseansicht.
%% Setzt den Schalter \ifkorrekturansicht voraus (gesetzt in den
%% einbindenden Dateien latex-korrekturansicht-abspann.tex bzw.
%% latex-leseansicht-abspann.tex).
%% ---------------------------------------------------------------

\normalsize

% Das esempio-Environment wird nur in der Leseansicht benötigt
\ifkorrekturansicht\else
\newenvironment{esempio}[3]%
{
    \vspace{1.5ex}
    \rlap{\underline{#1}}
    \par
    \setlength{\parindent}{0cm}
    \nopagebreak
    \leftskip=#2cm
    \rightskip=#3cm
}
{
    \par
}
\fi

\doendnotes{C}
\bigskip
\vfill

\clearpage

\footnotesize

\ifkorrekturansicht
  \lohead{\textsc{register}}
\fi

% theindex-Environment neu definieren ohne reledmac
\makeatletter
\renewenvironment{theindex}{%
  \ifkorrekturansicht
    \section*{\indexname}%
  \else
    \subsubsection*{Index der erwähnten Entitäten}%
  \fi
  \setlength{\parindent}{0pt}%
  \setlength{\parskip}{0pt plus 0.3pt}%
  \let\item\@idxitem
}{%
  \ifkorrekturansicht\clearpage\fi
}
\makeatother

\IfFileExists{\jobname-pw.ind}{\input{\jobname-pw.ind}}{}

% Quellenangabe nur in der Leseansicht
\ifkorrekturansicht\else
% Fallback-Definitionen, falls die .tex-Datei \titel etc. nicht gesetzt hat
\providecommand{\titel}{}
\providecommand{\editorInnen}{}
\providecommand{\dateiname}{\jobname}

\vspace{3cm}

\vfill

\footnotesize
\textsc{Quelle}: \titel. Herausgegeben von {\editorInnen}. In: \emph{Arthur Schnitzler: Briefwechsel mit Autorinnen und Autoren}.
 Digitale Edition, https://schnitzler-briefe.acdh.oeaw.ac.at/{\dateiname}.html (Stand \today)
\fi

\end{document}


