%% latex-korrekturansicht-vorspann.tex
%% Vorspann für die Korrekturansicht.
%% Lädt die gemeinsame Datei latex-vorspann.tex mit gesetztem Schalter.

\newif\ifkorrekturansicht
\korrekturansichttrue

\input{../tex-inputs/latex-vorspann}


\section[Arthur Schnitzler an Stefan Großmann, 17. 2. 1921]{L02363 Arthur Schnitzler an Stefan Großmann, 17. 2. 1921}
\nopagebreak\mylabel{L02363v}
\rehead{ }\normalsize\beginnumbering\briefempfaengerindex{Grossmann, Stefan@\textsc{Großmann, Stefan}!zzzSchnitzler, Arthur@\emph{von Arthur Schnitzler}!1921-02-171@{17. 2. 1921}|(be}
\toendnotes[C]{\smallbreak\pagebreak[2]}\Standort{DLA, A:Schnitzler, HS.NZ85.1.896.}
\physDesc{Brief, Durchschlag1 Blatt, 1 Seite, 1469 Zeichen
\newline{}Schreibmaschine
\newline{}Handschrift: roter Buntstift, deutsche Kurrent (\noindent{}Beschriftung: »K{[}opie{]}«, Unterstreichungen)}
\buchAbdrucke{\weitereDrucke{1) \pwindex{Reigen der Gassenjungen@\emph{Der Reigen der Gassenjungen}|pwk}\pwindex{Tage-Buch@\emph{Das Tage-Buch}|pwk}\emph{Das Tage-Buch}, Jg. 2, Nr. 8, 26. 2. 1921, S. 252–253.} \weitereDrucke{2) Arthur Schnitzler: \emph{Briefe 1913–1931}. Frankfurt am Main: \emph{S. Fischer} 1984, S. 234–235.} }\toendnotes[C]{\smallbreak}
\pstart
           \raggedleft{}{\pb}17. 2. 1921.\pend
           
\pstart{}Sehr verehrter Herr Grossmann.\pend\vspace{0.5em}
\pstart
           Vielen Dank für Ihr freundliches Interesse. Sie haben indess wohl meine Karte
               erhalten, in der ich Ihnen sagte, wie sehr mich Ihr parodistischer Dialog\pwindex{Haenischs Reigen. Eine unsittliche Szenenfolge@\emph{Hänischs Reigen. Eine unsittliche Szenenfolge}|pwv} amüsiert hat. Ich habe vorläufig
               keine Absicht mich über den »Reigen\pwindex{Reigen. Zehn Dialoge@\emph{Reigen. Zehn Dialoge}|pw}« und die
               sogenannnte Reigen\pwindex{Reigen. Zehn Dialoge@\emph{Reigen. Zehn Dialoge}|pw}-Affaire in der
               Oeffentlichkeit weiter zu äussern. Was ich {\pb}Herrn Maximilian Harden\pwindex{Harden, Maximilian 20.10.1861 – 30.10.1927@\textsc{Harden, Maximilian} (20.10.1861 – 30.10.1927), \emph{Schriftsteller/Schriftstellerin, Publizist/Publizistin}|pw}\pwindex{Reigen@\emph{Reigen}|pwv}{ }erwidert\pwindex{Berichtigung. Ein paar Worte zum Gutachten Maximilian Hardens ueber den »Reigen«@\emph{Berichtigung. Ein paar Worte zum Gutachten Maximilian Hardens über den »Reigen«}|pwv} habe, ersehen Sie aus
               beiliegendem \label{K_L02363-1v}\edtext{Zeitungsblatt}{\lemma{\textnormal{\emph{Zeitungsblatt}}}\Cendnote{\textnormal{Arthur Schnitzler: \emph{Berichtigung. Ein paar Worte zum Gutachten Maximilian
                        Hardens über den »Reigen«}\pwindex{Berichtigung. Ein paar Worte zum Gutachten Maximilian Hardens ueber den »Reigen«@\emph{Berichtigung. Ein paar Worte zum Gutachten Maximilian Hardens über den »Reigen«}|pwk} in: \emph{Neues
                        Wiener Journal}\pwindex{Neues Wiener Journal@\emph{Neues Wiener Journal}|pwk}, Jg. 29, Nr. 9782, 30. 1. 1921,
                  S. 6.}}}\label{K_L02363-1}. Die Berichtigung war übrigens in einigen Berlin\oindex{Berlin@\textbf{Berlin}, \emph{P.PPLC}|pw}er Blättern abgedruckt. Von den hiesigen Skandalen,
               insbesondere von dem \label{K_L02363-2v}\edtext{gestrigen}{\lemma{\textnormal{\emph{gestrigen}}}\Cendnote{\textnormal{am 16. 2. 1921}}}\label{K_L02363-2}, werden Sie wohl indess gelesen haben. Was soll man dazu sagen? Ich käme mir
               unsäglich komisch vor, wollte ich mit den Herren Kuntschak\pwindex{Kunschak, Leopold 11.11.1871 – 13.03.1953@\textsc{Kunschak, Leopold} (11.11.1871 – 13.03.1953), \emph{Politiker/Politikerin}|pw} oder Seipel\pwindex{Seipel, Ignaz 19.07.1876 – 02.08.1932@\textsc{Seipel, Ignaz} (19.07.1876 – 02.08.1932), \emph{Politiker/Politikerin, Prälat/Prälatin, Bundeskanzler/Bundeskanzlerin}|pw} oder mit dem
               Schusterlehrling polemisieren, der das Theater stürmt, mit dem begeisterten Ruf:
               Nieder mit dem Reigen! Man schändet unsere Frauen! Nieder mit den Sozialdemokraten!
               (Es kann übrigens auch ein Stud. med. gewesen sein oder ein Tapezierergehilfe, –
               wobei meine Sympathie immerhin noch mehr bei dem Tapezierergehilfen ist als bei den
               Herren Seipel\pwindex{Seipel, Ignaz 19.07.1876 – 02.08.1932@\textsc{Seipel, Ignaz} (19.07.1876 – 02.08.1932), \emph{Politiker/Politikerin, Prälat/Prälatin, Bundeskanzler/Bundeskanzlerin}|pw} und Kuntschak\pwindex{Kunschak, Leopold 11.11.1871 – 13.03.1953@\textsc{Kunschak, Leopold} (11.11.1871 – 13.03.1953), \emph{Politiker/Politikerin}|pw}.{[}){]} Ich habe ja schon einige
               ähnliche Sachen erlebt, wenn auch in bescheideneren Dimensionen. Erinnern Sie sich
               nur an den »Leutnant Gustl\pwindex{Lieutenant Gustl. Novelle@\emph{Lieutenant Gustl. Novelle}|pw}« und den »Professor Bernhardi\pwindex{Professor Bernhardi. Komoedie in fuenf Akten@\emph{Professor Bernhardi. Komödie in fünf Akten}|pw}«. Nach einigen Jahren bleibt
               von all dem Lärm nichts weiter übrig als die Bücher, die ich geschrieben und eine
               dunkle Erinnerung an die Blamage meiner Gegner. In diesem Fall wird es nicht anders
               sein.\pend
           
\pstart
           Mit herzlichem Gruss{\\[\baselineskip]}Ihr sehr ergebener{\\[\baselineskip]}\pend
           \leftskip=0em{}\selectlanguage{ngerman}\endnumbering\briefempfaengerindex{Grossmann, Stefan@\textsc{Großmann, Stefan}!zzzSchnitzler, Arthur@\emph{von Arthur Schnitzler}!1921-02-171@{17. 2. 1921}|)be}\mylabel{L02363h}  \normalsize

\doendnotes{C}
\bigskip
\vfill

\clearpage

\footnotesize

\lohead{\textsc{register}}

% Definiere theindex-Environment komplett neu ohne reledmac
\makeatletter
\renewenvironment{theindex}{%
  \section*{\indexname}%
  \setlength{\parindent}{0pt}%
  \setlength{\parskip}{0pt plus 0.3pt}%
  \let\item\@idxitem
}{%
  \clearpage
}
\makeatother

\IfFileExists{\jobname-pw.ind}{\input{\jobname-pw.ind}}{}

\end{document}

      