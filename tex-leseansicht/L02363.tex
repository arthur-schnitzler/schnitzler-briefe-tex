%% latex-leseansicht-vorspann.tex
%% Vorspann für die Leseansicht.
%% Lädt die gemeinsame Datei latex-vorspann.tex mit nicht gesetztem Schalter.

\newif\ifkorrekturansicht
\korrekturansichtfalse

\input{../tex-inputs/latex-vorspann}

\begin{center}
            \textcolor{red}{ENTWURF. ENTZIFFERUNG NOCH NICHT KORREKTURGELESEN}
                      \end{center}
            
               \section[Arthur Schnitzler an Stefan Großmann, 17. 2. 1921]{ Arthur Schnitzler an Stefan Großmann, 17. 2. 1921}\nopagebreak\mylabel{v}\rehead{ }\begin{ledgroupsized}[t]{13cm}\normalsize\beginnumbering\briefempfaengerindex{Grossmann, Stefan@\textsc{Großmann, Stefan}!zzzSchnitzler, Arthur@\emph{von Arthur Schnitzler}!1921-02-171@{17. 2. 1921}|(be} \toendnotes[C]{\smallbreak\pagebreak[2]} \Standort{DLA, A:Schnitzler, HS.NZ85.1.896.}
\physDesc{Brief, maschineller Durchschlag
\newline{}Schreibmaschine
\newline{}Handschrift: roter Buntstift, deutsche Kurrent (\noindent{}Beschriftung: »K{[}opie{]}«,
                                 Unterstreichungen)}\buchAbdrucke{\weitereDrucke{1) \pwindex{Grossmann, Stefan 19.05.1875 – 03.01.1935@\textsc{Großmann, Stefan} (19.05.1875 – 03.01.1935), \emph{Schriftsteller, Journalist}!Reigen der Gassenjungen26.2.1921 – 26.2.1921@\strich\emph{Der Reigen der Gassenjungen} {[}Hrsg., 26.2.1921 – 26.2.1921{]}|pwk}\pwindex{Tage-Buch1920-01-01 – 1933-01-01@\emph{Das Tage-Buch}|pwk}Stefan Großmann: \emph{Der Reigen der Gassenjungen.} In: \emph{Das Tage-Buch}, Jg. 2, Nr. 8, 26. 2. 1921, S. 252–253.} \weitereDrucke{2) Arthur Schnitzler: \emph{Briefe 1913–1931}. Hg. Peter Michael Braunwarth, Richard Miklin, Susanne Pertlik und Heinrich Schnitzler. Frankfurt am Main: \emph{S. Fischer} 1984, S. 234–235.} }\toendnotes[C]{\smallbreak}\pstart
           \raggedleft{}{\pb}17. 2. 1921.\pend
           \pstart{}Sehr verehrter Herr Grossmann.\pend\pstart
           Vielen Dank für Ihr freundliches Interesse. Sie haben indess wohl meine Karte
               erhalten, in der ich Ihnen sagte, wie sehr mich Ihr parodistischer Dialog\pwindex{Grossmann, Stefan 19.05.1875 – 03.01.1935@\textsc{Großmann, Stefan} (19.05.1875 – 03.01.1935), \emph{Schriftsteller, Journalist}!Haenischs Reigen. Eine unsittliche Szenenfolge1921-01-15 – 1921-01-15@\strich\emph{Hänischs Reigen. Eine unsittliche Szenenfolge} {[}1921-01-15 – 1921-01-15{]}|pwv} amüsiert hat. Ich habe vorläufig keine
               Absicht mich über den »Reigen\pwindex{Schnitzler, Arthur 15.05.1862 – 21.10.1931@\textsc{Schnitzler, Arthur} (15.05.1862 – 21.10.1931), \emph{Schriftsteller, Mediziner}!Reigen. Zehn Dialoge1900@\strich\emph{Reigen. Zehn Dialoge} {[}1900{]}|pw}« und die sogenannnte
                  Reigen\pwindex{Schnitzler, Arthur 15.05.1862 – 21.10.1931@\textsc{Schnitzler, Arthur} (15.05.1862 – 21.10.1931), \emph{Schriftsteller, Mediziner}!Reigen. Zehn Dialoge1900@\strich\emph{Reigen. Zehn Dialoge} {[}1900{]}|pw}-Affaire in der Oeffentlichkeit weiter zu
               äussern. Was ich {\pb}Herrn Maximilian Harden\pwindex{Harden, Maximilian 20.10.1861 – 30.10.1927@\textsc{Harden, Maximilian} (20.10.1861 – 30.10.1927), \emph{Schriftsteller, Publizist}|pw}\pwindex{Harden, Maximilian 20.10.1861 – 30.10.1927@\textsc{Harden, Maximilian} (20.10.1861 – 30.10.1927), \emph{Schriftsteller, Publizist}!Reigen08. 01. 1921@\strich\emph{Reigen} {[}08. 01. 1921{]}|pwv}{ }erwidert\pwindex{Schnitzler, Arthur 15.05.1862 – 21.10.1931@\textsc{Schnitzler, Arthur} (15.05.1862 – 21.10.1931), \emph{Schriftsteller, Mediziner}!Berichtigung. Ein paar Worte zum Gutachten Maximilian Hardens ueber den »Reigen«30. 01. 1921@\strich\emph{Berichtigung. Ein paar Worte zum Gutachten Maximilian Hardens über den »Reigen«} {[}30. 01. 1921{]}|pwv} habe, ersehen Sie aus
               beiliegendem \label{K_L02363_1v}\edtext{Zeitungsblatt}{\lemma{\textnormal{\emph{Zeitungsblatt}}}\Cendnote{\textnormal{Arthur
                        Schnitzler\pwindex{Schnitzler, Arthur 15.05.1862 – 21.10.1931@\textsc{Schnitzler, Arthur} (15.05.1862 – 21.10.1931), \emph{Schriftsteller, Mediziner}|pwk}: \emph{Berichtigung. Ein paar Worte
                        zum Gutachten Maximilian Hardens über den »Reigen«}\pwindex{Schnitzler, Arthur 15.05.1862 – 21.10.1931@\textsc{Schnitzler, Arthur} (15.05.1862 – 21.10.1931), \emph{Schriftsteller, Mediziner}!Berichtigung. Ein paar Worte zum Gutachten Maximilian Hardens ueber den »Reigen«30. 01. 1921@\strich\emph{Berichtigung. Ein paar Worte zum Gutachten Maximilian Hardens über den »Reigen«} {[}30. 01. 1921{]}|pwk} in: \emph{Neues Wiener Journal}\pwindex{Neues Wiener Journal1893 – 1939@\emph{Neues Wiener Journal}|pwk}, Jg. 29, Nr. 9782,
                        30. 1. 1921, S. 6.}}}\label{K_L02363_1h}. Die Berichtigung war übrigens
               in einigen Berlin\oindex{Berlin@\textbf{Berlin}|pw}er Blättern abgedruckt. Von den
               hiesigen Skandalen, insbesondere von dem \label{K_L02363_2v}\edtext{gestrigen}{\lemma{\textnormal{\emph{gestrigen}}}\Cendnote{\textnormal{am 16. 2. 1921}}}\label{K_L02363_2h}, werden Sie wohl indess gelesen
               haben. Was soll man dazu sagen? Ich käme mir unsäglich komisch vor, wollte ich mit
               den Herren Kuntschak\pwindex{Kunschak, Leopold 11.11.1871 – 13.03.1953@\textsc{Kunschak, Leopold} (11.11.1871 – 13.03.1953), \emph{Politiker}|pw} oder Seipel\pwindex{Seipel, Ignaz 19.07.1876 – 02.08.1932@\textsc{Seipel, Ignaz} (19.07.1876 – 02.08.1932), \emph{Politiker, Prälat, Bundeskanzler}|pw} oder mit dem Schusterlehrling polemisieren, der das
               Theater stürmt, mit dem begeisterten Ruf: Nieder mit dem Reigen! Man schändet unsere
               Frauen! Nieder mit den Sozialdemokraten! (Es kann übrigens auch ein Stud. med.
               gewesen sein oder ein Tapezierergehilfe, – wobei meine Sympathie immerhin noch mehr
               bei dem Tapezierergehilfen ist als bei den Herren Seipel\pwindex{Seipel, Ignaz 19.07.1876 – 02.08.1932@\textsc{Seipel, Ignaz} (19.07.1876 – 02.08.1932), \emph{Politiker, Prälat, Bundeskanzler}|pw} und Kuntschak\pwindex{Kunschak, Leopold 11.11.1871 – 13.03.1953@\textsc{Kunschak, Leopold} (11.11.1871 – 13.03.1953), \emph{Politiker}|pw}.{[}){]} Ich habe ja schon einige ähnliche Sachen
               erlebt, wenn auch in bescheideneren Dimensionen. Erinnern Sie sich nur an den »Leutnant Gustl\pwindex{Schnitzler, Arthur 15.05.1862 – 21.10.1931@\textsc{Schnitzler, Arthur} (15.05.1862 – 21.10.1931), \emph{Schriftsteller, Mediziner}!Lieutenant Gustl. Novelle25. 12. 1900@\strich\emph{Lieutenant Gustl. Novelle} {[}25. 12. 1900{]}|pw}« und den »Professor Bernhardi\pwindex{Schnitzler, Arthur 15.05.1862 – 21.10.1931@\textsc{Schnitzler, Arthur} (15.05.1862 – 21.10.1931), \emph{Schriftsteller, Mediziner}!Professor Bernhardi. Komoedie in fuenf Akten1912@\strich\emph{Professor Bernhardi. Komödie in fünf Akten} {[}1912{]}|pw}«. Nach einigen Jahren bleibt von all dem
               Lärm nichts weiter übrig als die Bücher, die ich geschrieben und eine dunkle
               Erinnerung an die Blamage meiner Gegner. In diesem Fall wird es nicht anders
               sein.\pend
           \pstart
           Mit herzlichem Gruss{\\[\baselineskip]}Ihr sehr ergebener{\\[\baselineskip]}\pend
           \leftskip=0em{}\endnumbering\briefempfaengerindex{Grossmann, Stefan@\textsc{Großmann, Stefan}!zzzSchnitzler, Arthur@\emph{von Arthur Schnitzler}!1921-02-171@{17. 2. 1921}|)be}\mylabel{h}\end{ledgroupsized}  \newcommand{\dateiname}{L02363}\newcommand{\titel}{Arthur Schnitzler an Stefan Großmann, 17. 2. 1921}\newcommand{\editorInnen}{Martin Anton Müller und Gerd-Hermann Susen}%% latex-leseansicht-abspann.tex
%% Abspann für die Leseansicht.
%% Der Schalter \ifkorrekturansicht ist bereits durch den Vorspann gesetzt.

%% latex-abspann.tex
%% Gemeinsamer Abspann für Korrekturansicht und Leseansicht.
%% Setzt den Schalter \ifkorrekturansicht voraus (gesetzt in den
%% einbindenden Dateien latex-korrekturansicht-abspann.tex bzw.
%% latex-leseansicht-abspann.tex).
%% ---------------------------------------------------------------

\normalsize

% Das esempio-Environment wird nur in der Leseansicht benötigt
\ifkorrekturansicht\else
\newenvironment{esempio}[3]%
{
    \vspace{1.5ex}
    \rlap{\underline{#1}}
    \par
    \setlength{\parindent}{0cm}
    \nopagebreak
    \leftskip=#2cm
    \rightskip=#3cm
}
{
    \par
}
\fi

\doendnotes{C}
\bigskip
\vfill

\clearpage

\footnotesize

\ifkorrekturansicht
  \lohead{\textsc{register}}
\fi

% theindex-Environment neu definieren ohne reledmac
\makeatletter
\renewenvironment{theindex}{%
  \ifkorrekturansicht
    \section*{\indexname}%
  \else
    \subsubsection*{Index der erwähnten Entitäten}%
  \fi
  \setlength{\parindent}{0pt}%
  \setlength{\parskip}{0pt plus 0.3pt}%
  \let\item\@idxitem
}{%
  \ifkorrekturansicht\clearpage\fi
}
\makeatother

\IfFileExists{\jobname-pw.ind}{\input{\jobname-pw.ind}}{}

% Quellenangabe nur in der Leseansicht
\ifkorrekturansicht\else
% Fallback-Definitionen, falls die .tex-Datei \titel etc. nicht gesetzt hat
\providecommand{\titel}{}
\providecommand{\editorInnen}{}
\providecommand{\dateiname}{\jobname}

\vspace{3cm}

\vfill

\footnotesize
\textsc{Quelle}: \titel. Herausgegeben von {\editorInnen}. In: \emph{Arthur Schnitzler: Briefwechsel mit Autorinnen und Autoren}.
 Digitale Edition, https://schnitzler-briefe.acdh.oeaw.ac.at/{\dateiname}.html (Stand \today)
\fi

\end{document}


      