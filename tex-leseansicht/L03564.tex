%% latex-leseansicht-vorspann.tex
%% Vorspann für die Leseansicht.
%% Lädt die gemeinsame Datei latex-vorspann.tex mit nicht gesetztem Schalter.

\newif\ifkorrekturansicht
\korrekturansichtfalse

\input{../tex-inputs/latex-vorspann}


         
         \renewcommand{\erwaehntePersonen}{Personen: Louis Charles de Bourbon, Felix Salten, Ottilie Salten, Olga Schnitzler, Élisabeth Vigée-Lebrun}
         \renewcommand{\erwaehnteOrte}{Orte: Berlin, Hamburg, London, Paris, Schloss Versailles, Sternwartestraße 71, Wien, Österreich}
         \renewcommand{\erwaehnteWerke}{Werke: Portrait du Dauphin Louis-Charles}
               \section[ Felix und Ottilie Salten an Arthur und Olga Schnitzler, 25. 6. 1914]{ Felix und Ottilie Salten an Arthur und Olga
               Schnitzler, 25. 6. 1914}\nopagebreak\mylabel{v}\rehead{ }\begin{ledgroupsized}[t]{13cm}\normalsize\beginnumbering\briefempfaengerindex{Schnitzler, Olga@\textsc{Schnitzler, Olga}!zzzSalten, Ottilie@\emph{von Ottilie Salten}!1914-06-251@{25. 6. 1914}|(be}\briefempfaengerindex{Schnitzler, Olga@\textsc{Schnitzler, Olga}!zzzSalten, Felix@\emph{von Felix Salten}!1914-06-251@{25. 6. 1914}|(be}\briefempfaengerindex{Schnitzler, Arthur@\textsc{Schnitzler, Arthur}!zzzSalten, Ottilie@\emph{von Ottilie Salten}!1914-06-251@{25. 6. 1914}|(be}\briefempfaengerindex{Schnitzler, Arthur@\textsc{Schnitzler, Arthur}!zzzSalten, Felix@\emph{von Felix Salten}!1914-06-251@{25. 6. 1914}|(be} \toendnotes[C]{\smallbreak\pagebreak[2]} \Standort{CUL, Schnitzler, B 89, B 2.}
\physDesc{Bildpostkarte, 311 Zeichen
\newline{}Handschrift Felix Salten: schwarze Tinte, lateinische Kurrent\newline{}Handschrift Ottilie Salten: schwarze Tinte, deutsche Kurrent
\newline{}Versand: Stempel: »\nobreak{}\oindex{Paris@\textbf{Paris}|pwk}\textcolor{gray}{Paris – 92} Boissy—D’Anglas, 25—6 14, 15 50\nobreak{}«.  
\newline{}Ordnung: mit Bleistift von unbekannter Hand nummeriert: »277« }\toendnotes[C]{\smallbreak}\pstart{}{\pb}\begin{otherlanguage}{french}Autriche\oindex{Oesterreich@\textbf{Österreich}|pw}\end{otherlanguage}\pend{}\pstart{}Herrn u. Frau D\textsuperscript{r} Arthur Schnitzler\pend{}\pstart{}Wien\oindex{Wien@\textbf{Wien}|pw}\pend{}\pstart{}XVIII. Sternwartestrasse 71\oindex{Sternwartestrasse 71@\textbf{Sternwartestraße 71}|pw}\pend{}{\bigskip}\pstart
           \noindent{}\centering{}{\pb}\textcolor{gray}{\textbf{Mme VIGÉE-LEBRUN\pwindex{Vigee-Lebrun, Elisabeth 1755-04-16 – 1842-03-30@\textsc{Vigée-Lebrun, Élisabeth} (1755-04-16 – 1842-03-30), \emph{Malerin}|pw}. – Portrait du Dauphin\pwindex{Bourbon, Louis Charles de 1785-03-27 – 1795-06-08@\textsc{Bourbon, Louis Charles de} (1785-03-27 – 1795-06-08), \emph{Dauphin}|pw}\pwindex{Portrait du Dauphin Louis-Charles1792@\emph{Portrait du Dauphin Louis-Charles} {[}1792{]}|pw}.}}\pend
           \pstart
           \noindent{}\raggedleft{}\textcolor{gray}{\textbf{MUSÉE DE VERSAILLES\oindex{Schloss Versailles@\textbf{Schloss Versailles}|pw}}}\pend
           \pstart
           {\pb}Wir fahren heute heim. In diesen kurzen Wochen Berlin\oindex{Berlin@\textbf{Berlin}|pw}, Hamburg\oindex{Hamburg@\textbf{Hamburg}|pw}, London\oindex{London@\textbf{London}|pw} und Paris\oindex{Paris@\textbf{Paris}|pw} war ein bischen viel und wir sind ein wenig müd. Aber es war sehr
               schön! \label{K_L03564-1v}\edtext{Wann kommen Sie nach
                  Hause?}{\lemma{\textnormal{\emph{Wann … Hause?}}}\Cendnote{\textnormal{Schnitzler\pwindex{Schnitzler, Arthur 15.05.1862 – 21.10.1931@\textsc{Schnitzler, Arthur} (15.05.1862 – 21.10.1931), \emph{Schriftsteller, Mediziner}|pwk} war zu diesem Zeitpunkt bereits wieder in Wien\oindex{Wien@\textbf{Wien}|pwk}.}}}\label{K_L03564-1h}\pend
           \pstart
           Viele herzliche Grüße Ihnen Beiden {\\[\baselineskip]}Ihr {\\[\baselineskip]}\spacefill\mbox{Salten}\pend
           \leftskip=0em{}\pstart
           {[}hs. Ottilie Salten:{]} herzliche Grüße {\\[\baselineskip]}\spacefill\mbox{OttilieS.}\pend
           \leftskip=0em{}
         
         \endnumbering\mylabel{h}\end{ledgroupsized}  \newcommand{\dateiname}{L03564}\newcommand{\titel}{Felix und Ottilie Salten an Arthur und Olga Schnitzler, 25. 6. 1914}\newcommand{\editorInnen}{Martin Anton Müller und Laura Untner}%% latex-leseansicht-abspann.tex
%% Abspann für die Leseansicht.
%% Der Schalter \ifkorrekturansicht ist bereits durch den Vorspann gesetzt.

%% latex-abspann.tex
%% Gemeinsamer Abspann für Korrekturansicht und Leseansicht.
%% Setzt den Schalter \ifkorrekturansicht voraus (gesetzt in den
%% einbindenden Dateien latex-korrekturansicht-abspann.tex bzw.
%% latex-leseansicht-abspann.tex).
%% ---------------------------------------------------------------

\normalsize

% Das esempio-Environment wird nur in der Leseansicht benötigt
\ifkorrekturansicht\else
\newenvironment{esempio}[3]%
{
    \vspace{1.5ex}
    \rlap{\underline{#1}}
    \par
    \setlength{\parindent}{0cm}
    \nopagebreak
    \leftskip=#2cm
    \rightskip=#3cm
}
{
    \par
}
\fi

\doendnotes{C}
\bigskip
\vfill

\clearpage

\footnotesize

\ifkorrekturansicht
  \lohead{\textsc{register}}
\fi

% theindex-Environment neu definieren ohne reledmac
\makeatletter
\renewenvironment{theindex}{%
  \ifkorrekturansicht
    \section*{\indexname}%
  \else
    \subsubsection*{Index der erwähnten Entitäten}%
  \fi
  \setlength{\parindent}{0pt}%
  \setlength{\parskip}{0pt plus 0.3pt}%
  \let\item\@idxitem
}{%
  \ifkorrekturansicht\clearpage\fi
}
\makeatother

\IfFileExists{\jobname-pw.ind}{\input{\jobname-pw.ind}}{}

% Quellenangabe nur in der Leseansicht
\ifkorrekturansicht\else
% Fallback-Definitionen, falls die .tex-Datei \titel etc. nicht gesetzt hat
\providecommand{\titel}{}
\providecommand{\editorInnen}{}
\providecommand{\dateiname}{\jobname}

\vspace{3cm}

\vfill

\footnotesize
\textsc{Quelle}: \titel. Herausgegeben von {\editorInnen}. In: \emph{Arthur Schnitzler: Briefwechsel mit Autorinnen und Autoren}.
 Digitale Edition, https://schnitzler-briefe.acdh.oeaw.ac.at/{\dateiname}.html (Stand \today)
\fi

\end{document}


      