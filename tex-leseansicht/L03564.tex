%% latex-korrekturansicht-vorspann.tex
%% Vorspann für die Korrekturansicht.
%% Lädt die gemeinsame Datei latex-vorspann.tex mit gesetztem Schalter.

\newif\ifkorrekturansicht
\korrekturansichttrue

\input{../tex-inputs/latex-vorspann}


\section[ Felix und Ottilie Salten an Arthur und Olga Schnitzler, 25. 6. 1914]{L03564 Felix und Ottilie Salten an Arthur und Olga
               Schnitzler, 25. 6. 1914}
\nopagebreak\mylabel{L03564v}
\rehead{ }\normalsize\beginnumbering\briefempfaengerindex{Schnitzler, Olga@\textsc{Schnitzler, Olga}!zzzSalten, Ottilie@\emph{von Ottilie Salten}!1914-06-251@{25. 6. 1914}|(be}\briefempfaengerindex{Schnitzler, Olga@\textsc{Schnitzler, Olga}!zzzSalten, Felix@\emph{von Felix Salten}!1914-06-251@{25. 6. 1914}|(be}\briefempfaengerindex{Schnitzler, Arthur@\textsc{Schnitzler, Arthur}!zzzSalten, Ottilie@\emph{von Ottilie Salten}!1914-06-251@{25. 6. 1914}|(be}\briefempfaengerindex{Schnitzler, Arthur@\textsc{Schnitzler, Arthur}!zzzSalten, Felix@\emph{von Felix Salten}!1914-06-251@{25. 6. 1914}|(be}
\toendnotes[C]{\smallbreak\pagebreak[2]}\Standort{CUL, Schnitzler, B 89, B 2.}
\physDesc{Bildpostkarte, 311 Zeichen
\newline{}Handschrift Felix Salten: schwarze Tinte, lateinische Kurrent
\newline{}Handschrift Ottilie Salten: schwarze Tinte, deutsche Kurrent
\newline{}Versand: Stempel: »\nobreak{}\oindex{Paris@\textbf{Paris}, \emph{P.PPLC}|pwk}\textcolor{gray}{Paris – 92} Boissy–D’Anglas, 25–6 14, 15 50\nobreak{}«.  
\newline{}Ordnung: mit Bleistift von unbekannter Hand nummeriert: »277« }\toendnotes[C]{\smallbreak}\pstart{}{\pb}\begin{otherlanguage}{french}Autriche\oindex{Oesterreich@\textbf{Österreich}, \emph{A.PCLI}|pw}\end{otherlanguage}\pend{}\pstart{}Herrn u. Frau D\textsuperscript{r} Arthur Schnitzler\pend{}\pstart{}Wien\oindex{Wien@\textbf{Wien}, \emph{A.ADM2}|pw}\pend{}\pstart{}XVIII. Sternwartestrasse 71\oindex{Sternwartestrasse 71@\textbf{Sternwartestraße 71}, \emph{Wohngebäude (K.WHS)}|pw}\pend{}{\bigskip}
\pstart
           \noindent{}\centering{}{\pb}\textcolor{gray}{\textbf{Mme VIGÉE-LEBRUN\pwindex{Vigee-Lebrun, Elisabeth 1755-04-16 – 1842-03-30@\textsc{Vigée-Lebrun, Élisabeth} (1755-04-16 – 1842-03-30), \emph{Maler/Malerin}|pw}. – Portrait du Dauphin\pwindex{Bourbon, Louis Charles de 1785-03-27 – 1795-06-08@\textsc{Bourbon, Louis Charles de} (1785-03-27 – 1795-06-08), \emph{Dauphin/Dauphine}|pw}\pwindex{Portrait du Dauphin Louis-Charles@\emph{Portrait du Dauphin Louis-Charles}|pw}.}}\pend
           
\pstart
           \raggedleft{}\textcolor{gray}{\textbf{MUSÉE DE VERSAILLES\oindex{Schloss Versailles@\textbf{Schloss Versailles}, \emph{Schloss (K.SLS)}|pw}}}\pend
           \vspace{1em}
\pstart
           \noindent{}{\pb}Wir fahren heute heim. In diesen kurzen Wochen Berlin\oindex{Berlin@\textbf{Berlin}, \emph{P.PPLC}|pw}, Hamburg\oindex{Hamburg@\textbf{Hamburg}, \emph{P.PPLA}|pw}, London\oindex{London@\textbf{London}, \emph{P.PPLC}|pw} und Paris\oindex{Paris@\textbf{Paris}, \emph{P.PPLC}|pw} war ein bischen viel und wir sind ein wenig müd. Aber es war sehr
               schön! \label{K_L03564-1v}\edtext{Wann kommen Sie nach
                  Hause?}{\lemma{\textnormal{\emph{Wann … Hause?}}}\Cendnote{\textnormal{Schnitzler war zu diesem Zeitpunkt bereits wieder in Wien\oindex{Wien@\textbf{Wien}, \emph{A.ADM2}|pwk}.}}}\label{K_L03564-1}\pend
           
\pstart
           Viele herzliche Grüße Ihnen Beiden {\\[\baselineskip]}Ihr {\\[\baselineskip]}\spacefill\mbox{Salten}\pend
           \leftskip=0em{}\selectlanguage{ngerman}\vspace{1em}
\pstart
           {[}hs. :{]} herzliche Grüße {\\[\baselineskip]}\spacefill\mbox{OttilieS.}\pend
           \leftskip=0em{}\selectlanguage{ngerman}\endnumbering\briefempfaengerindex{Schnitzler, Olga@\textsc{Schnitzler, Olga}!zzzSalten, Ottilie@\emph{von Ottilie Salten}!1914-06-251@{25. 6. 1914}|)be}\briefempfaengerindex{Schnitzler, Olga@\textsc{Schnitzler, Olga}!zzzSalten, Felix@\emph{von Felix Salten}!1914-06-251@{25. 6. 1914}|)be}\briefempfaengerindex{Schnitzler, Arthur@\textsc{Schnitzler, Arthur}!zzzSalten, Ottilie@\emph{von Ottilie Salten}!1914-06-251@{25. 6. 1914}|)be}\briefempfaengerindex{Schnitzler, Arthur@\textsc{Schnitzler, Arthur}!zzzSalten, Felix@\emph{von Felix Salten}!1914-06-251@{25. 6. 1914}|)be}\mylabel{L03564h}  \normalsize

\doendnotes{C}
\bigskip
\vfill

\clearpage

\footnotesize

\lohead{\textsc{register}}

% Definiere theindex-Environment komplett neu ohne reledmac
\makeatletter
\renewenvironment{theindex}{%
  \section*{\indexname}%
  \setlength{\parindent}{0pt}%
  \setlength{\parskip}{0pt plus 0.3pt}%
  \let\item\@idxitem
}{%
  \clearpage
}
\makeatother

\IfFileExists{\jobname-pw.ind}{\input{\jobname-pw.ind}}{}

\end{document}

      