%% latex-leseansicht-vorspann.tex
%% Vorspann für die Leseansicht.
%% Lädt die gemeinsame Datei latex-vorspann.tex mit nicht gesetztem Schalter.

\newif\ifkorrekturansicht
\korrekturansichtfalse

\input{../tex-inputs/latex-vorspann}


\section[Arthur Schnitzler an Richard Beer-Hofmann, 31. 5. 1894]{L00334 Arthur Schnitzler an Richard Beer-Hofmann, 31. 5. 1894}
\nopagebreak\mylabel{L00334v}
\rehead{ }\normalsize\beginnumbering\briefempfaengerindex{Beer-Hofmann, Richard@\textsc{Beer-Hofmann, Richard}!zzzSchnitzler, Arthur@\emph{von Arthur Schnitzler}!1894-05-311@{31. 5. 1894}|(be}
\toendnotes[C]{\smallbreak\pagebreak[2]}
\correspDesc{Versand  durch Arthur Schnitzler am 31. 5. 1894 in Wien
\newline{}Erhalt  durch Richard Beer-Hofmann am 1. 6. 1894 in Bad Ischl}\toendnotes[C]{\smallbreak}
\Standort{YCGL, MSS 31.}
\physDesc{Briefkarte, , Kuvert, 439 Zeichen
\newline{}Handschrift: schwarze Tinte, deutsche Kurrent
\newline{}Versand: 1) Stempel: »\nobreak{}\oindex{IX., Alsergrund@\textbf{IX., Alsergrund}, \emph{Verwaltungsgebiet}|pwk}Wien 9/3, 31. 5. 94, 6–7N\nobreak{}«.   2) Stempel: »\nobreak{}\oindex{Bad Ischl@\textbf{Bad Ischl}|pwk}Ischl, 1 6 94, 10 F\nobreak{}«. }
\buchAbdrucke{\weitereDrucke{Arthur Schnitzler, Richard Beer-Hofmann: \emph{Briefwechsel 1891–1931}. Herausgegeben von Konstanze Fliedl. Wien, Zürich: \emph{Europaverlag} 1992, S. 55.} }\pstart{}{\pb}Herrn \textsc{Dr. Richard
                     Beer-Hofmann}\pend{}\pstart{}\textsc{Ischl\oindex{Bad Ischl@\textbf{Bad Ischl}|pw}}\pend{}\pstart{}\textsc{Egelmoos 22\oindex{Eglmoosgasse@\textbf{Eglmoosgasse}, \emph{Bezirk}|pw}}\pend{}{\bigskip}\vspace{1em}
\pstart
           \noindent{}{\pb}Lieber Richard. Meine Abſicht iſt es,
                  Samſtag{ }Abend abzureiſen. Ich bin dann 7 Uhr früh{ }So{\geminationn}tag in München\oindex{München@\textbf{München}|pw},{ }ſteige Hotel \textsc{Maximilian}\oindex{Hotel Maximilian@\textbf{Hotel Maximilian}, \emph{Hotel}|pw} ab. Bitte um Nachricht, was Sie thun. –\pend
           
\pstart
           Hab von \textsc{Brandes}\pwindex{Brandes, Georg 4.\,2.\,1842 Kopenhagen – 19.\,2.\,1927 ebd.@\textsc{Brandes, Georg} (4.\,2.\,1842 Kopenhagen – 19.\,2.\,1927 ebd.)|pw} einen{ }ſchönen Brief über’s Märchen\pwindex{Schnitzler, Arthur 15.\,5.\,1862 Wien – 21.\,10.\,1931 ebd.@\textsc{Schnitzler, Arthur} (15.\,5.\,1862 Wien – 21.\,10.\,1931 ebd.), \emph{Schriftsteller, Mediziner}!Märchen. Schauspiel in drei Aufzügen@\strich\emph{Das Märchen. Schauspiel in drei Aufzügen}|pw} beko{\geminationm}en. – Heut einen {\pb}ſechs
               Seiten langen noch{ }ſchönern über alle möglichen Sachen von der \textsc{Lou Salomé}\pwindex{Andreas-Salomé, Lou 12.\,2.\,1861 Sankt Petersburg – 5.\,2.\,1937 Göttingen@\textsc{Andreas-Salomé, Lou} (12.\,2.\,1861 Sankt Petersburg – 5.\,2.\,1937 Göttingen), \emph{Schriftstellerin}|pw}.\pend
           
\pstart
           Herzlichen Gruſs. Ich freue mich{ }ſehr, ein paar Tage mit Ihnen zu
                  verbringen.{\\[\baselineskip]}Ihr\spacefill\mbox{Arthur}\pend
           \leftskip=0em{}\selectlanguage{ngerman}\endnumbering\briefempfaengerindex{Beer-Hofmann, Richard@\textsc{Beer-Hofmann, Richard}!zzzSchnitzler, Arthur@\emph{von Arthur Schnitzler}!1894-05-311@{31. 5. 1894}|)be}\mylabel{L00334h}  \newcommand{\dateiname}{L00334}\newcommand{\titel}{Arthur Schnitzler an Richard Beer-Hofmann, 31. 5. 1894}\newcommand{\editorInnen}{Martin Anton Müller und Gerd-Hermann Susen}%% latex-leseansicht-abspann.tex
%% Abspann für die Leseansicht.
%% Der Schalter \ifkorrekturansicht ist bereits durch den Vorspann gesetzt.

%% latex-abspann.tex
%% Gemeinsamer Abspann für Korrekturansicht und Leseansicht.
%% Setzt den Schalter \ifkorrekturansicht voraus (gesetzt in den
%% einbindenden Dateien latex-korrekturansicht-abspann.tex bzw.
%% latex-leseansicht-abspann.tex).
%% ---------------------------------------------------------------

\normalsize

% Das esempio-Environment wird nur in der Leseansicht benötigt
\ifkorrekturansicht\else
\newenvironment{esempio}[3]%
{
    \vspace{1.5ex}
    \rlap{\underline{#1}}
    \par
    \setlength{\parindent}{0cm}
    \nopagebreak
    \leftskip=#2cm
    \rightskip=#3cm
}
{
    \par
}
\fi

\doendnotes{C}
\bigskip
\vfill

\clearpage

\footnotesize

\ifkorrekturansicht
  \lohead{\textsc{register}}
\fi

% theindex-Environment neu definieren ohne reledmac
\makeatletter
\renewenvironment{theindex}{%
  \ifkorrekturansicht
    \section*{\indexname}%
  \else
    \subsubsection*{Index der erwähnten Entitäten}%
  \fi
  \setlength{\parindent}{0pt}%
  \setlength{\parskip}{0pt plus 0.3pt}%
  \let\item\@idxitem
}{%
  \ifkorrekturansicht\clearpage\fi
}
\makeatother

\IfFileExists{\jobname-pw.ind}{\input{\jobname-pw.ind}}{}

% Quellenangabe nur in der Leseansicht
\ifkorrekturansicht\else
% Fallback-Definitionen, falls die .tex-Datei \titel etc. nicht gesetzt hat
\providecommand{\titel}{}
\providecommand{\editorInnen}{}
\providecommand{\dateiname}{\jobname}

\vspace{3cm}

\vfill

\footnotesize
\textsc{Quelle}: \titel. Herausgegeben von {\editorInnen}. In: \emph{Arthur Schnitzler: Briefwechsel mit Autorinnen und Autoren}.
 Digitale Edition, https://schnitzler-briefe.acdh.oeaw.ac.at/{\dateiname}.html (Stand \today)
\fi

\end{document}


