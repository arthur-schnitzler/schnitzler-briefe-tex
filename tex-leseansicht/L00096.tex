%% latex-korrekturansicht-vorspann.tex
%% Vorspann für die Korrekturansicht.
%% Lädt die gemeinsame Datei latex-vorspann.tex mit gesetztem Schalter.

\newif\ifkorrekturansicht
\korrekturansichttrue

\input{../tex-inputs/latex-vorspann}


\section[Arthur Schnitzler an Hugo von Hofmannsthal, 24. 4. 1892]{L00096 Arthur Schnitzler an Hugo von Hofmannsthal, 24. 4. 1892}
\nopagebreak\mylabel{L00096v}
\rehead{ }\normalsize\beginnumbering\briefempfaengerindex{Hofmannsthal, Hugo von@\textsc{Hofmannsthal, Hugo von}!zzzSchnitzler, Arthur@\emph{von Arthur Schnitzler}!1892-04-242@{24. 4. 1892}|(be}
\toendnotes[C]{\smallbreak\pagebreak[2]}\Standort{FDH, Hs-30885,20.}
\physDesc{Briefkarte, 162 Zeichen
\newline{}Handschrift: schwarze Tinte, deutsche Kurrent
\newline{}Ordnung: mit Bleistift von unbekannter Hand Datum unterstrichen }
\buchAbdrucke{\weitereDrucke{Hugo von Hofmannsthal, Arthur Schnitzler: \emph{Briefwechsel}. Frankfurt am Main: \emph{S. Fischer} 1964, S. 21.} }\toendnotes[C]{\smallbreak}
\pstart
           \noindent{}{\pb}Lieber Freund,{ }Dinſtag vor 5 Uhr wird Herr Roſner\pwindex{Rosner, Karl Peter 05.02.1873 – 06.05.1951@\textsc{Rosner, Karl Peter} (05.02.1873 – 06.05.1951), \emph{Schriftsteller/Schriftstellerin}|pw}
                in meiner Wohnung eine Novelle \label{K_L00096-1v}\edtext{vorleſen}{\lemma{\textnormal{\emph{vorleſen}}}\Cendnote{\textnormal{Die Lesung
                  dürfte nicht stattgefunden haben.}}}\label{K_L00096-1}; wenn Sie Zeit haben, ſo kommen Sie
               gütigſt auch.\pend
           
\pstart
           Herzlich{\\[\baselineskip]}Ihr{\\[\baselineskip]}\spacefill\mbox{ArthurSch}\pend
           \leftskip=0em{}
\pstart
           24. 4. 92\pend
           \selectlanguage{ngerman}\endnumbering\briefempfaengerindex{Hofmannsthal, Hugo von@\textsc{Hofmannsthal, Hugo von}!zzzSchnitzler, Arthur@\emph{von Arthur Schnitzler}!1892-04-242@{24. 4. 1892}|)be}\mylabel{L00096h}  \normalsize

\doendnotes{C}
\bigskip
\vfill

\clearpage

\footnotesize

\lohead{\textsc{register}}

% Definiere theindex-Environment komplett neu ohne reledmac
\makeatletter
\renewenvironment{theindex}{%
  \section*{\indexname}%
  \setlength{\parindent}{0pt}%
  \setlength{\parskip}{0pt plus 0.3pt}%
  \let\item\@idxitem
}{%
  \clearpage
}
\makeatother

\IfFileExists{\jobname-pw.ind}{\input{\jobname-pw.ind}}{}

\end{document}

      