%% latex-leseansicht-vorspann.tex
%% Vorspann für die Leseansicht.
%% Lädt die gemeinsame Datei latex-vorspann.tex mit nicht gesetztem Schalter.

\newif\ifkorrekturansicht
\korrekturansichtfalse

\input{../tex-inputs/latex-vorspann}


         
         \renewcommand{\erwaehntePersonen}{Personen: Richard Beer-Hofmann, Paul Goldmann}
         \renewcommand{\erwaehnteOrte}{Orte: Bad Ischl, Eglmoosgasse, Hotel Bauer, Hotel und Pension Rudolfshöhe (Leopold Petter)}
         \renewcommand{\erwaehnteWerke}{Werke: Tagebuch}
               \section[Adele Sandrock und Arthur Schnitzler an Richard Beer-Hofmann, 29. 8. 1894]{ Adele Sandrock und Arthur Schnitzler an Richard Beer-Hofmann,
               29. 8. 1894}\nopagebreak\mylabel{v}\rehead{ }\begin{ledgroupsized}[t]{13cm}\normalsize\beginnumbering \toendnotes[C]{\smallbreak\pagebreak[2]} \Standort{YCGL, MSS 31.}
\physDesc{Brief, 1 Blatt, 2 Seiten, Umschlag
\newline{}Handschrift Arthur Schnitzler: schwarze Tinte\newline{}Handschrift Adele Sandrock: 1) schwarze Tinte, deutsche Kurrent\hspace{1em}2) schwarze Tinte, lateinische Kurrent (\noindent{}Umschlag)\hspace{1em}\newline{}Versand: ohne postalischen Übermittlungsvermerk }\buchAbdrucke{\weitereDrucke{1) Adele Sandrock, Arthur Schnitzler: \emph{Dilly. Geschichte einer Liebe in Briefen, Bildern und
                        Dokumenten}. Zusammengestellt von Renate Wagner. Wien, München: \emph{Amalthea} 1975, S. 183.} \weitereDrucke{2) Arthur Schnitzler, Richard Beer-Hofmann: \emph{Briefwechsel 1891–1931}. Hg. Konstanze Fliedl. Wien, Zürich: \emph{Europaverlag} 1992, S. 58.} }\toendnotes[C]{\smallbreak}\pstart{}{\pb}Herrn Dr. Richard Beer-Hofmann\pend{}\pstart{}in\pend{}\pstart{}Ischl\oindex{Bad Ischl@\textbf{Bad Ischl}|pw}\pend{}\pstart{}Egelmoos 22\oindex{Eglmoosgasse@\textbf{Eglmoosgasse}|pw}.\pend{}{\bigskip}\pstart
           \raggedleft{}{\pb}29. Aug 94{\\}Ischl\oindex{Bad Ischl@\textbf{Bad Ischl}|pw}\pend
           \pstart{}Meine \label{K_L00364-33v}\edtext{Herren}{\lemma{\textnormal{\emph{Herren}}}\Cendnote{\textnormal{Der Plural im Abgleich mit dem \emph{Tagebuch}\pwindex{Schnitzler, Arthur 15.05.1862 – 21.10.1931@\textsc{Schnitzler, Arthur} (15.05.1862 – 21.10.1931), \emph{Schriftsteller, Mediziner}!Tagebuch1981 – 2000@\strich\emph{Tagebuch} {[}1981 – 2000{]}|pwk} zeigt, dass auch Goldmann ein
                     Empfänger des Briefes ist.}}}\label{K_L00364-33h}!\pend\pstart
           \uline{Wir gehen um }\uline{6}\uline{, }\substVorne{}\textsuperscript{\uline{6}}\substDazwischen{}\uline{7}\substHinten{}\uline{ Uhr}\uline{ jedenfalls \textsc{Eglmoos 22}\oindex{Eglmoosgasse@\textbf{Eglmoosgasse}|pw} vorbei} und werden pfeifen oder auch nicht pfeifen. Sie werden zu Hauſe
               ſein oder auch nicht zu Hauſe ſein. Im Falle wir uns nicht {\pb}treffen, bin ich (die Tragödin Adele Sandrock) vor
               zehn Uhr im Hotel Bauer\oindex{Hotel Bauer@\textbf{Hotel Bauer}|pw}{ }ſoupirend anzutreffen. Ich (der Dramatiker Arthur
               Schnitzler) ſpeise \substVorne{}\textsuperscript{L}\substDazwischen{}½ 9\substHinten{} beim Leopold\oindex{Hotel und Pension Rudolfshoehe (Leopold Petter)@\textbf{Hotel und Pension Rudolfshöhe (Leopold Petter)}|pw}, wo ich Sie, meine Herren,
               jedenfalls zu ſehen hoffe.\pend
           \pstart
           Herzliche Grüße{\\[\baselineskip]}\spacefill\mbox{Sandrock A.}{\\[\baselineskip]}\spacefill\mbox{{[}hs. Schnitzler:{]} Schnitzler}\pend
           \leftskip=0em{}
         
         \endnumbering\mylabel{h}\end{ledgroupsized}  \newcommand{\dateiname}{L00364}\newcommand{\titel}{Adele Sandrock und Arthur Schnitzler an Richard Beer-Hofmann, 29. 8. 1894}\newcommand{\editorInnen}{Martin Anton Müller und Gerd-Hermann Susen}%% latex-leseansicht-abspann.tex
%% Abspann für die Leseansicht.
%% Der Schalter \ifkorrekturansicht ist bereits durch den Vorspann gesetzt.

%% latex-abspann.tex
%% Gemeinsamer Abspann für Korrekturansicht und Leseansicht.
%% Setzt den Schalter \ifkorrekturansicht voraus (gesetzt in den
%% einbindenden Dateien latex-korrekturansicht-abspann.tex bzw.
%% latex-leseansicht-abspann.tex).
%% ---------------------------------------------------------------

\normalsize

% Das esempio-Environment wird nur in der Leseansicht benötigt
\ifkorrekturansicht\else
\newenvironment{esempio}[3]%
{
    \vspace{1.5ex}
    \rlap{\underline{#1}}
    \par
    \setlength{\parindent}{0cm}
    \nopagebreak
    \leftskip=#2cm
    \rightskip=#3cm
}
{
    \par
}
\fi

\doendnotes{C}
\bigskip
\vfill

\clearpage

\footnotesize

\ifkorrekturansicht
  \lohead{\textsc{register}}
\fi

% theindex-Environment neu definieren ohne reledmac
\makeatletter
\renewenvironment{theindex}{%
  \ifkorrekturansicht
    \section*{\indexname}%
  \else
    \subsubsection*{Index der erwähnten Entitäten}%
  \fi
  \setlength{\parindent}{0pt}%
  \setlength{\parskip}{0pt plus 0.3pt}%
  \let\item\@idxitem
}{%
  \ifkorrekturansicht\clearpage\fi
}
\makeatother

\IfFileExists{\jobname-pw.ind}{\input{\jobname-pw.ind}}{}

% Quellenangabe nur in der Leseansicht
\ifkorrekturansicht\else
% Fallback-Definitionen, falls die .tex-Datei \titel etc. nicht gesetzt hat
\providecommand{\titel}{}
\providecommand{\editorInnen}{}
\providecommand{\dateiname}{\jobname}

\vspace{3cm}

\vfill

\footnotesize
\textsc{Quelle}: \titel. Herausgegeben von {\editorInnen}. In: \emph{Arthur Schnitzler: Briefwechsel mit Autorinnen und Autoren}.
 Digitale Edition, https://schnitzler-briefe.acdh.oeaw.ac.at/{\dateiname}.html (Stand \today)
\fi

\end{document}


      