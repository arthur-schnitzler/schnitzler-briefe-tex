%% latex-leseansicht-vorspann.tex
%% Vorspann für die Leseansicht.
%% Lädt die gemeinsame Datei latex-vorspann.tex mit nicht gesetztem Schalter.

\newif\ifkorrekturansicht
\korrekturansichtfalse

\input{../tex-inputs/latex-vorspann}


\section[Adele Sandrock und Arthur Schnitzler an Richard Beer-Hofmann und Paul Goldmann, 29. 8. 1894]{L00364 Adele Sandrock und Arthur Schnitzler an Richard Beer-Hofmann und Paul
               Goldmann, 29. 8. 1894}
\nopagebreak\mylabel{L00364v}
\rehead{ }\normalsize\beginnumbering\briefempfaengerindex{Goldmann, Paul@\textsc{Goldmann, Paul}!zzzSchnitzler, Arthur@\emph{von Arthur Schnitzler}!1894-08-291@{29. 8. 1894}|(be}\briefempfaengerindex{Goldmann, Paul@\textsc{Goldmann, Paul}!zzzSandrock, Adele@\emph{von Adele Sandrock}!1894-08-291@{29. 8. 1894}|(be}\briefempfaengerindex{Beer-Hofmann, Richard@\textsc{Beer-Hofmann, Richard}!zzzSchnitzler, Arthur@\emph{von Arthur Schnitzler}!1894-08-291@{29. 8. 1894}|(be}\briefempfaengerindex{Beer-Hofmann, Richard@\textsc{Beer-Hofmann, Richard}!zzzSandrock, Adele@\emph{von Adele Sandrock}!1894-08-291@{29. 8. 1894}|(be}
\toendnotes[C]{\smallbreak\pagebreak[2]}
\correspDesc{Versand  durch Adele Sandrock, Arthur Schnitzler am 29. 8. 1894 in Bad Ischl
\newline{}Erhalt  durch Richard Beer-Hofmann, Paul Goldmann am 29. 8. 1894 in Bad Ischl}\toendnotes[C]{\smallbreak}
\Standort{YCGL, MSS 31.}
\physDesc{Brief, 1 Blatt, 2 Seiten, Kuvert, 493 Zeichen
\newline{}Handschrift Arthur Schnitzler: schwarze Tinte
\newline{}Handschrift Adele Sandrock: schwarze Tinte, deutsche Kurrent
\newline{}Versand: ohne postalischen Übermittlungsvermerk }
\buchAbdrucke{\weitereDrucke{1) Adele Sandrock, Arthur Schnitzler: \emph{Dilly. Geschichte einer Liebe in Briefen, Bildern und
                        Dokumenten}. Zusammengestellt von Renate Wagner. Wien, München: \emph{Amalthea} 1975, S. 183.} \weitereDrucke{2) Arthur Schnitzler, Richard Beer-Hofmann: \emph{Briefwechsel 1891–1931}. Herausgegeben von Konstanze Fliedl. Wien, Zürich: \emph{Europaverlag} 1992, S. 58.} }\toendnotes[C]{\smallbreak}\pstart{}{\pb}Herrn Dr. Richard Beer-Hofmann\pend{}\pstart{}in\pend{}\pstart{}Ischl\oindex{Bad Ischl@\textbf{Bad Ischl}|pw}\pend{}\pstart{}Egelmoos 22\oindex{Eglmoosgasse@\textbf{Eglmoosgasse}, \emph{Bezirk}|pw}.\pend{}{\bigskip}\vspace{1em}
\pstart
           \raggedleft{}{\pb}29. Aug 94{\\}Ischl\oindex{Bad Ischl@\textbf{Bad Ischl}|pw}\pend
           
\pstart{}Meine \label{K_L00364-1v}\edtext{Herren}{\lemma{\textnormal{\emph{Herren}}}\Cendnote{\textnormal{Der Plural im Abgleich mit dem \emph{Tagebuch}\pwindex{Schnitzler, Arthur 15.\,5.\,1862 Wien – 21.\,10.\,1931 ebd.@\textsc{Schnitzler, Arthur} (15.\,5.\,1862 Wien – 21.\,10.\,1931 ebd.), \emph{Schriftsteller, Mediziner}!Tagebuch@\strich\emph{Tagebuch}|pwk} zeigt, dass auch Goldmann ein
                     Empfänger des Briefes ist.}}}\label{K_L00364-1}!\pend\vspace{0.5em}
\pstart
           \uline{Wir gehen um{ }}\uline{6}\uline{,{ }}\substVorne{}\textsuperscript{\uline{6}}\substDazwischen{}\uline{7}\substHinten{}\uline{{ }Uhr}\uline{{ }jedenfalls \textsc{Eglmoos 22}\oindex{Eglmoosgasse@\textbf{Eglmoosgasse}, \emph{Bezirk}|pw} vorbei} und werden pfeifen oder auch nicht pfeifen. Sie werden zu Hauſe{ }ſein oder auch nicht zu Hauſe{ }ſein. Im Falle wir uns nicht {\pb}treffen, bin ich (die Tragödin Adele Sandrock) vor
               zehn Uhr im Hotel Bauer\oindex{Hotel Bauer@\textbf{Hotel Bauer}, \emph{Hotel}|pw}{ }ſoupirend anzutreffen. Ich (der Dramatiker Arthur
               Schnitzler){ }ſpeise \substVorne{}\textsuperscript{L}\substDazwischen{}½ 9\substHinten{} beim Leopold\oindex{Hotel und Pension Rudolfshöhe (Leopold Petter)@\textbf{Hotel und Pension Rudolfshöhe (Leopold Petter)}, \emph{Hotel}|pw}, wo ich Sie, meine Herren,
               jedenfalls zu{ }ſehen hoffe.\pend
           
\pstart
           Herzliche Grüße{\\[\baselineskip]}\spacefill\mbox{Sandrock A.}{\\[\baselineskip]}\spacefill\mbox{{[}hs. Schnitzler:{]} Schnitzler}\pend
           \leftskip=0em{}\selectlanguage{ngerman}\endnumbering\briefempfaengerindex{Goldmann, Paul@\textsc{Goldmann, Paul}!zzzSchnitzler, Arthur@\emph{von Arthur Schnitzler}!1894-08-291@{29. 8. 1894}|)be}\briefempfaengerindex{Goldmann, Paul@\textsc{Goldmann, Paul}!zzzSandrock, Adele@\emph{von Adele Sandrock}!1894-08-291@{29. 8. 1894}|)be}\briefempfaengerindex{Beer-Hofmann, Richard@\textsc{Beer-Hofmann, Richard}!zzzSchnitzler, Arthur@\emph{von Arthur Schnitzler}!1894-08-291@{29. 8. 1894}|)be}\briefempfaengerindex{Beer-Hofmann, Richard@\textsc{Beer-Hofmann, Richard}!zzzSandrock, Adele@\emph{von Adele Sandrock}!1894-08-291@{29. 8. 1894}|)be}\mylabel{L00364h}  \newcommand{\dateiname}{L00364}\newcommand{\titel}{Adele Sandrock und Arthur Schnitzler an Richard Beer-Hofmann und Paul Goldmann, 29. 8. 1894}\newcommand{\editorInnen}{Martin Anton Müller und Gerd-Hermann Susen}%% latex-leseansicht-abspann.tex
%% Abspann für die Leseansicht.
%% Der Schalter \ifkorrekturansicht ist bereits durch den Vorspann gesetzt.

%% latex-abspann.tex
%% Gemeinsamer Abspann für Korrekturansicht und Leseansicht.
%% Setzt den Schalter \ifkorrekturansicht voraus (gesetzt in den
%% einbindenden Dateien latex-korrekturansicht-abspann.tex bzw.
%% latex-leseansicht-abspann.tex).
%% ---------------------------------------------------------------

\normalsize

% Das esempio-Environment wird nur in der Leseansicht benötigt
\ifkorrekturansicht\else
\newenvironment{esempio}[3]%
{
    \vspace{1.5ex}
    \rlap{\underline{#1}}
    \par
    \setlength{\parindent}{0cm}
    \nopagebreak
    \leftskip=#2cm
    \rightskip=#3cm
}
{
    \par
}
\fi

\doendnotes{C}
\bigskip
\vfill

\clearpage

\footnotesize

\ifkorrekturansicht
  \lohead{\textsc{register}}
\fi

% theindex-Environment neu definieren ohne reledmac
\makeatletter
\renewenvironment{theindex}{%
  \ifkorrekturansicht
    \section*{\indexname}%
  \else
    \subsubsection*{Index der erwähnten Entitäten}%
  \fi
  \setlength{\parindent}{0pt}%
  \setlength{\parskip}{0pt plus 0.3pt}%
  \let\item\@idxitem
}{%
  \ifkorrekturansicht\clearpage\fi
}
\makeatother

\IfFileExists{\jobname-pw.ind}{\input{\jobname-pw.ind}}{}

% Quellenangabe nur in der Leseansicht
\ifkorrekturansicht\else
% Fallback-Definitionen, falls die .tex-Datei \titel etc. nicht gesetzt hat
\providecommand{\titel}{}
\providecommand{\editorInnen}{}
\providecommand{\dateiname}{\jobname}

\vspace{3cm}

\vfill

\footnotesize
\textsc{Quelle}: \titel. Herausgegeben von {\editorInnen}. In: \emph{Arthur Schnitzler: Briefwechsel mit Autorinnen und Autoren}.
 Digitale Edition, https://schnitzler-briefe.acdh.oeaw.ac.at/{\dateiname}.html (Stand \today)
\fi

\end{document}


