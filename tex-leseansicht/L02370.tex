%% latex-leseansicht-vorspann.tex
%% Vorspann für die Leseansicht.
%% Lädt die gemeinsame Datei latex-vorspann.tex mit nicht gesetztem Schalter.

\newif\ifkorrekturansicht
\korrekturansichtfalse

\input{../tex-inputs/latex-vorspann}


         
         \renewcommand{\erwaehntePersonen}{Personen: Richard Beer-Hofmann, Heinrich Schnitzler, Lili Schnitzler, Olga Schnitzler}
         \renewcommand{\erwaehnteOrte}{Orte: Adriatisches Meer, Altaussee, Dachstein, Kärnten, Mallnitz, Possenhofen, Salzburg, Salzkammergut, Sarstein, Seewirt, Wien}
         \renewcommand{\erwaehnteWerke}{}
               \section[Arthur Schnitzler an Richard Beer-Hofmann, 8. 8. 1921]{ Arthur Schnitzler an Richard Beer-Hofmann, 8. 8. 1921}\nopagebreak\mylabel{v}\rehead{ }\begin{ledgroupsized}[t]{13cm}\normalsize\beginnumbering\briefempfaengerindex{Beer-Hofmann, Richard@\textsc{Beer-Hofmann, Richard}!zzzSchnitzler, Arthur@\emph{von Arthur Schnitzler}!1921-08-081@{8. 8. 1921}|(be} \toendnotes[C]{\smallbreak\pagebreak[2]} \Standort{YCGL, MSS 31.}
\physDesc{Bildpostkarte, 533 Zeichen
\newline{}Handschrift: Bleistift, deutsche Kurrent
\newline{}Versand: Stempel: »\nobreak{}\oindex{Altaussee@\textbf{Altaussee}|pwk}Alt Aussee, 8. VIII. \textcolor{gray}{2}1\nobreak{}«.  }\buchAbdrucke{\weitereDrucke{Arthur Schnitzler, Richard Beer-Hofmann: \emph{Briefwechsel 1891–1931}. Hg. Konstanze Fliedl. Wien, Zürich: \emph{Europaverlag} 1992, S. 228.} }\toendnotes[C]{\smallbreak}\pstart{}{\pb}\textsc{Hrn Dr. Rich. Beer Hofma{\geminationn}}\pend{}\pstart{}aus Wien\oindex{Wien@\textbf{Wien}|pw}\pend{}\pstart{}\textsc{Malnitz\oindex{Mallnitz@\textbf{Mallnitz}|pw} in Kärnthen\oindex{Kaernten@\textbf{Kärnten}|pw}}\pend{}{\bigskip}\pstart
           \noindent{}\centering{}{\pb}\textcolor{gray}{\textbf{Salzkammergut\oindex{Salzkammergut@\textbf{Salzkammergut}|pw}. Alt-Aussee\oindex{Altaussee@\textbf{Altaussee}|pw} mit Dachstein\oindex{Dachstein@\textbf{Dachstein}|pw} und Sarstein\oindex{Sarstein@\textbf{Sarstein}|pw}.}}\pend
           \pstart
           \raggedleft{}8. 8. 21. \textsc{Altaussee\oindex{Altaussee@\textbf{Altaussee}|pw}, Seewirth\oindex{Seewirt@\textbf{Seewirt}|pw}}\pend
           \pstart
           lieber Richard, ich hoffe dieſe Grüße erreichen Sie und treffen Sie
               wie die Ihren in guter Geſundheit und Sti{\geminationm}ung an. Ich
               behage mich hier ganz leidlich, und manchmal mehr als das; man führt heuer ein Bad u
               Strandleben, als wär man an der Adria\oindex{Adriatisches Meer@\textbf{Adriatisches Meer}|pw}, aber auch
               das ſteht dem {\pb}Altauſſee\oindex{Altaussee@\textbf{Altaussee}|pw}r ja ganz gut. Die Kinder\pwindex{Schnitzler, Heinrich 09.08.1902 – 12.07.1982@\textsc{Schnitzler, Heinrich} (09.08.1902 – 12.07.1982), \emph{Regisseur, Schauspieler}|pwv}\pwindex{Schnitzler, Lili 13.09.1909 – 26.07.1928@\textsc{Schnitzler, Lili} (13.09.1909 – 26.07.1928)|pwv}{ }ſind ſeit 10 Tagen in \textsc{Possenhofen}\oindex{Possenhofen@\textbf{Possenhofen}|pw} bei O.\pwindex{Schnitzler, Olga 17.01.1882 – 13.01.1970@\textsc{Schnitzler, Olga} (17.01.1882 – 13.01.1970), \emph{Schauspielerin, Sängerin}|pw}, ich werd wohl zwiſchen
                  25. u 3\textcolor{gray}{0}. ev. mit Aufenthalt in Salzburg\oindex{Salzburg@\textbf{Salzburg}|pw} hinfahren.\pend
           \pstart
           Schreiben Sie mir bitte ein Wort.\pend
           \pstart
           Von Herzen Ihr{\\[\baselineskip]}\spacefill\mbox{A.}\pend
           \leftskip=0em{}
         
         \endnumbering\mylabel{h}\end{ledgroupsized}  \newcommand{\dateiname}{L02370}\newcommand{\titel}{Arthur Schnitzler an Richard Beer-Hofmann, 8. 8. 1921}\newcommand{\editorInnen}{Martin Anton Müller und Gerd-Hermann Susen}%% latex-leseansicht-abspann.tex
%% Abspann für die Leseansicht.
%% Der Schalter \ifkorrekturansicht ist bereits durch den Vorspann gesetzt.

%% latex-abspann.tex
%% Gemeinsamer Abspann für Korrekturansicht und Leseansicht.
%% Setzt den Schalter \ifkorrekturansicht voraus (gesetzt in den
%% einbindenden Dateien latex-korrekturansicht-abspann.tex bzw.
%% latex-leseansicht-abspann.tex).
%% ---------------------------------------------------------------

\normalsize

% Das esempio-Environment wird nur in der Leseansicht benötigt
\ifkorrekturansicht\else
\newenvironment{esempio}[3]%
{
    \vspace{1.5ex}
    \rlap{\underline{#1}}
    \par
    \setlength{\parindent}{0cm}
    \nopagebreak
    \leftskip=#2cm
    \rightskip=#3cm
}
{
    \par
}
\fi

\doendnotes{C}
\bigskip
\vfill

\clearpage

\footnotesize

\ifkorrekturansicht
  \lohead{\textsc{register}}
\fi

% theindex-Environment neu definieren ohne reledmac
\makeatletter
\renewenvironment{theindex}{%
  \ifkorrekturansicht
    \section*{\indexname}%
  \else
    \subsubsection*{Index der erwähnten Entitäten}%
  \fi
  \setlength{\parindent}{0pt}%
  \setlength{\parskip}{0pt plus 0.3pt}%
  \let\item\@idxitem
}{%
  \ifkorrekturansicht\clearpage\fi
}
\makeatother

\IfFileExists{\jobname-pw.ind}{\input{\jobname-pw.ind}}{}

% Quellenangabe nur in der Leseansicht
\ifkorrekturansicht\else
% Fallback-Definitionen, falls die .tex-Datei \titel etc. nicht gesetzt hat
\providecommand{\titel}{}
\providecommand{\editorInnen}{}
\providecommand{\dateiname}{\jobname}

\vspace{3cm}

\vfill

\footnotesize
\textsc{Quelle}: \titel. Herausgegeben von {\editorInnen}. In: \emph{Arthur Schnitzler: Briefwechsel mit Autorinnen und Autoren}.
 Digitale Edition, https://schnitzler-briefe.acdh.oeaw.ac.at/{\dateiname}.html (Stand \today)
\fi

\end{document}


      