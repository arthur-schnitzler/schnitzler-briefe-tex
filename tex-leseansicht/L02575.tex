%% latex-korrekturansicht-vorspann.tex
%% Vorspann für die Korrekturansicht.
%% Lädt die gemeinsame Datei latex-vorspann.tex mit gesetztem Schalter.

\newif\ifkorrekturansicht
\korrekturansichttrue

\input{../tex-inputs/latex-vorspann}


\section[Therese Rie-Andro an Arthur Schnitzler, 23. 7. 1923]{L02575 Therese Rie-Andro an Arthur Schnitzler, 23. 7. 1923}
\nopagebreak\mylabel{L02575v}
\rehead{ }\normalsize\beginnumbering\briefempfaengerindex{Schnitzler, Arthur@\textsc{Schnitzler, Arthur}!zzzRie, Therese@\emph{von Therese Rie}!1923-07-231@{23. 7. 1923}|(be}
\toendnotes[C]{\smallbreak\pagebreak[2]}\Standort{CUL, Schnitzler, B 658.}
\physDesc{Brief, 1 Blatt, 2 Seiten, 1938 Zeichen
\newline{}Handschrift: blaue Tinte, lateinische Kurrent
\newline{}Schnitzler: 1) mit Bleistift beschriftet: »\textsc{Rie}«  2) mit rotem Buntstift vier Unterstreichungen}\toendnotes[C]{\smallbreak}
\pstart
           \raggedleft{}{\pb}Bernried/Starnbergerseee\oindex{Bernried@\textbf{Bernried}, \emph{P.PPLA4}|pw},
                     23. 7. 23. \pend
           
\pstart
           \raggedleft{}Altwirt.\oindex{Hotel Seeblick@\textbf{Hotel Seeblick}, \emph{Hotel (K.HTL)}|pw}\hspace*{1.5em}Oberbayern\oindex{Oberbayern@\textbf{Oberbayern}, \emph{A.ADM2}|pw}\pend
           
\pstart{}Verehrter Herr Doktor,\pend\vspace{0.5em}
\pstart
           Es iſt wirklich lieb von Ihnen, daſs Sie von meiner Literatur noch immer nicht genug
               haben; aber leider bin ich nun schon zu Ende, es exiſtieren bloß noch ein paar
               Jugendsünden und verſtreute oder ungedruckte Sachen. So schmeichelhaft es iſt – ich
               hab’ nichts mehr! – Aber \uline{nicht} schmeichelhaft, lieber
               Herr Doktor, iſt die Annahme, ich nähme meine eigenen Briefe auf die Reise mit! Das
               läßt auf düſtere Erfahrungen schließen, die Sie mit Schreibweibern gemacht haben
               müssen! Da tun Sie mir sehr leid! – Ist es nicht tausend mal schöner und wichtiger,
               zu schw{\geminationm}en\textcolor{gray}{,} zu rudern und unter alten
               Bäumen zu liegen? Ich meine, der Dichter der Lebendigen Stunden\pwindex{Lebendige Stunden@\emph{Lebendige Stunden}|pw} gibt mir da Recht!\pend
           
\pstart
           Aber da fällt mir doch ein, daſs ich noch was Schönes\pwindex{Musikalische Reise ins Land der Vergangenheit@\emph{Musikalische Reise ins Land der Vergangenheit}|pwv}{ }\introOben{}daheim\introOben{} habe: von Romain
                  Rolland\pwindex{Rolland, Romain 29.01.1866 – 30.12.1944@\textsc{Rolland, Romain} (29.01.1866 – 30.12.1944), \emph{Schriftsteller/Schriftstellerin}|pw} (von mir übersetzt.) Das beko{\geminationm}en Sie.
               Für die Reise freilich nicht mehr rechtzeitig, da ich vor dem 15. Auguſt
               kaum in Wien\oindex{Wien@\textbf{Wien}, \emph{A.ADM2}|pw} bin und Sie wol schon fort. Aber
               hoffentlich gefällt es Ihnen auch später noch. Denn es dreht sich nur um die Muſik
               und das iſt doch das Einzige, was im Leben in der Stadt \strikeout{(auch)} noch \uline{wirklich} iſt.\pend
           
\pstart
           Daß Sie mir ein Buch von sich geben wollen, iſt sehr lieb von Ihnen. Ihre gesa{\geminationm}elten Werke\pwindex{Gesammelte Werke@\emph{Gesammelte Werke}|pw}
                  (\label{K_L02575-1v}\edtext{bis zum Weiten Land\pwindex{weite Land. Tragikomoedie in fuenf Akten@\emph{Das weite Land. Tragikomödie in fünf Akten}|pw}}{\lemma{\textnormal{\emph{bis zum Weiten Land}}}\Cendnote{\textnormal{Sie besaß die Ausgabe von
                     1912 ohne die beiden Ergänzungsbände von
               1922.}}}\label{K_L02575-1}) besitze ich natürlich; ich gestehe {\pb}Ihnen eine große Zuneigung zu Fink und Fliederbuſch\pwindex{Fink und Fliederbusch. Komoedie in drei Akten@\emph{Fink und Fliederbusch. Komödie in drei Akten}|pw}, gerade weil dieses Stück
               alle wolgeölten Gemüter einmal in Aufruhr versetzt hat; aber Beate\pwindex{Frau Beate und ihr Sohn. Novelle@\emph{Frau Beate und ihr Sohn. Novelle}|pw} oder Casanova\pwindex{Casanovas Heimfahrt@\emph{Casanovas Heimfahrt}|pw}
               liebe ich nicht minder – also bitte, suchen \uline{Sie} mir
               etwas aus, dann habe ich zu der Freude des Empfangens auch noch die Ihrer
               Auswahl.\pend
           
\pstart
           Die beiden \label{K_L02575-2v}\edtext{Ausschnitte}{\lemma{\textnormal{\emph{Ausschnitte}}}\Cendnote{\textnormal{nicht überliefert}}}\label{K_L02575-2}, die ich einlege,
               sind aus einer New-York\oindex{New York City@\textbf{New York City}, \emph{P.PPL}|pw}er Revue: der eine
               enthält zwei Worte über den Casanova\pwindex{Casanovas Heimfahrt@\emph{Casanovas Heimfahrt}|pw}. Der andre
               hat mit Kunſt überhaupt nichts zu tun, iſt aber menſchlich so packend und traurig,
               daſs er Sie vielleicht intereſſirt; auch ein »Bernhardi\pwindex{Professor Bernhardi. Komoedie in fuenf Akten@\emph{Professor Bernhardi. Komödie in fünf Akten}|pwv}« hätte drüber nichts zu lachen! \pend
           
\pstart
           Und nun wünsche ich Ihnen schöne, helle, frohe So{\geminationm}ertage!\pend
           
\pstart
           Ihre{\\[\baselineskip]}\spacefill\mbox{Therese Rie.}\pend
           \leftskip=0em{}\selectlanguage{ngerman}\endnumbering\briefempfaengerindex{Schnitzler, Arthur@\textsc{Schnitzler, Arthur}!zzzRie, Therese@\emph{von Therese Rie}!1923-07-231@{23. 7. 1923}|)be}\mylabel{L02575h}  \normalsize

\doendnotes{C}
\bigskip
\vfill

\clearpage

\footnotesize

\lohead{\textsc{register}}

% Definiere theindex-Environment komplett neu ohne reledmac
\makeatletter
\renewenvironment{theindex}{%
  \section*{\indexname}%
  \setlength{\parindent}{0pt}%
  \setlength{\parskip}{0pt plus 0.3pt}%
  \let\item\@idxitem
}{%
  \clearpage
}
\makeatother

\IfFileExists{\jobname-pw.ind}{\input{\jobname-pw.ind}}{}

\end{document}

      