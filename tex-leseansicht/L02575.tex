%% latex-leseansicht-vorspann.tex
%% Vorspann für die Leseansicht.
%% Lädt die gemeinsame Datei latex-vorspann.tex mit nicht gesetztem Schalter.

\newif\ifkorrekturansicht
\korrekturansichtfalse

\input{../tex-inputs/latex-vorspann}


               \section[Therese Rie-Andro an Arthur Schnitzler, 23. 7. 1923]{ Therese Rie-Andro an Arthur Schnitzler, 23. 7. 1923}\nopagebreak\mylabel{v}\rehead{ }\begin{ledgroupsized}[t]{13cm}\normalsize\beginnumbering\briefempfaengerindex{Schnitzler, Arthur@\textsc{Schnitzler, Arthur}!zzzRie, Therese@\emph{von Therese Rie}!1923-07-231@{23. 7. 1923}|(be} \toendnotes[C]{\smallbreak\pagebreak[2]} \Standort{CUL, Schnitzler, B 658.}
\physDesc{Brief, 1 Blatt, 2 Seiten
\newline{}Handschrift: blaue Tinte, lateinische Kurrent
\newline{}Schnitzler: 1) mit Bleistift beschriftet: »\textsc{Rie}« 2) mit rotem Buntstift vier Unterstreichungen}\toendnotes[C]{\smallbreak}\pstart
           \raggedleft{}{\pb}Bernried/Starnbergerseee\oindex{Bernried@\textbf{Bernried}|pw},
                     23. 7. 23. \pend
           \pstart
           \raggedleft{}Altwirt.\oindex{Hotel Seeblick@\textbf{Hotel Seeblick}|pw}\hspace*{1.5em}Oberbayern\oindex{Oberbayern@\textbf{Oberbayern}|pw}\pend
           \pstart{}Verehrter Herr Doktor,\pend\pstart
           Es iſt wirklich lieb von Ihnen, daſs Sie von meiner Literatur noch immer nicht genug
               haben; aber leider bin ich nun schon zu Ende, es exiſtieren bloß noch ein paar
               Jugendsünden und verſtreute oder ungedruckte Sachen. So schmeichelhaft es iſt – ich
               hab’ nichts mehr! – Aber \uline{nicht} schmeichelhaft, lieber
               Herr Doktor, iſt die Annahme, ich nähme meine eigenen Briefe auf die Reise mit! Das
               läßt auf düſtere Erfahrungen schließen, die Sie mit Schreibweibern gemacht haben
               müssen! Da tun Sie mir sehr leid! – Ist es nicht tausend mal schöner und wichtiger,
               zu schw{\geminationm}en\textcolor{gray}{,} zu rudern und unter alten
               Bäumen zu liegen? Ich meine, der Dichter der Lebendigen
                  Stunden\pwindex{Schnitzler, Arthur 15.05.1862 – 21.10.1931@\textsc{Schnitzler, Arthur} (15.05.1862 – 21.10.1931), \emph{Schriftsteller, Mediziner}!Lebendige Stunden01. 12. 1901@\strich\emph{Lebendige Stunden} {[}01. 12. 1901{]}|pw} gibt mir da Recht!\pend
           \pstart
           Aber da fällt mir doch ein, daſs ich noch was Schönes\pwindex{Rolland, Romain 29.01.1866 – 30.12.1944@\textsc{Rolland, Romain} (29.01.1866 – 30.12.1944), \emph{Schriftsteller}!Musikalische Reise ins Land der Vergangenheit1919@\strich\emph{Musikalische Reise ins Land der Vergangenheit} {[}1919{]}|pwv}{ }\introOben{}daheim\introOben{} habe: von Romain
                  Rolland\pwindex{Rolland, Romain 29.01.1866 – 30.12.1944@\textsc{Rolland, Romain} (29.01.1866 – 30.12.1944), \emph{Schriftsteller}|pw} (von mir übersetzt.) Das beko{\geminationm}en Sie.
               Für die Reise freilich nicht mehr rechtzeitig, da ich vor dem 15. Auguſt
               kaum in Wien\oindex{Wien@\textbf{Wien}|pw} bin und Sie wol schon fort. Aber
               hoffentlich gefällt es Ihnen auch später noch. Denn es dreht sich nur um die Muſik
               und das iſt doch das Einzige, was im Leben in der Stadt \strikeout{(auch)} noch \uline{wirklich} iſt.\pend
           \pstart
           Daß Sie mir ein Buch von sich geben wollen, iſt sehr lieb von Ihnen. Ihre gesa{\geminationm}elten Werke\pwindex{Schnitzler, Arthur 15.05.1862 – 21.10.1931@\textsc{Schnitzler, Arthur} (15.05.1862 – 21.10.1931), \emph{Schriftsteller, Mediziner}!Gesammelte Werke1912 – 1922@\strich\emph{Gesammelte Werke} {[}1912 – 1922{]}|pw}
                  (\label{K_L02575-2v}\edtext{bis zum Weiten Land\pwindex{Schnitzler, Arthur 15.05.1862 – 21.10.1931@\textsc{Schnitzler, Arthur} (15.05.1862 – 21.10.1931), \emph{Schriftsteller, Mediziner}!weite Land. Tragikomoedie in fuenf Akten1910-10-20@\strich\emph{Das weite Land. Tragikomödie in fünf Akten} {[}1910-10-20{]}|pw}}{\lemma{\textnormal{\emph{bis zum Weiten Land}}}\Cendnote{\textnormal{Sie besitzt die Ausgabe von
                     1912 ohne die beiden Ergänzungsbände von
               1922.}}}\label{K_L02575-2h}) besitze ich natürlich; ich gestehe {\pb}Ihnen eine große Zuneigung zu Fink und Fliederbuſch\pwindex{Schnitzler, Arthur 15.05.1862 – 21.10.1931@\textsc{Schnitzler, Arthur} (15.05.1862 – 21.10.1931), \emph{Schriftsteller, Mediziner}!Fink und Fliederbusch. Komoedie in drei Akten1917@\strich\emph{Fink und Fliederbusch. Komödie in drei Akten} {[}1917{]}|pw}, gerade weil dieses Stück alle
               wolgeölten Gemüter einmal in Aufruhr versetzt hat; aber Beate\pwindex{Schnitzler, Arthur 15.05.1862 – 21.10.1931@\textsc{Schnitzler, Arthur} (15.05.1862 – 21.10.1931), \emph{Schriftsteller, Mediziner}!Frau Beate und ihr Sohn. Novelle1.2.1913 – 1.4.1913@\strich\emph{Frau Beate und ihr Sohn. Novelle} {[}1.2.1913 – 1.4.1913{]}|pw} oder Casanova\pwindex{Schnitzler, Arthur 15.05.1862 – 21.10.1931@\textsc{Schnitzler, Arthur} (15.05.1862 – 21.10.1931), \emph{Schriftsteller, Mediziner}!Casanovas Heimfahrt1.7.1918 – 1.9.1918@\strich\emph{Casanovas Heimfahrt} {[}1.7.1918 – 1.9.1918{]}|pw} liebe ich nicht minder
               – also bitte, suchen \uline{Sie} mir etwas aus, dann habe ich
               zu der Freude des Empfangens auch noch die Ihrer Auswahl.\pend
           \pstart
           Die beiden \label{K_L02575-1v}\edtext{Ausschnitte}{\lemma{\textnormal{\emph{Ausschnitte}}}\Cendnote{\textnormal{nicht überliefert}}}\label{K_L02575-1h}, die ich einlege,
               sind aus einer New-York\oindex{New York City@\textbf{New York City}|pw}er Revue: der eine enthält
               zwei Worte über den Casanova\pwindex{Schnitzler, Arthur 15.05.1862 – 21.10.1931@\textsc{Schnitzler, Arthur} (15.05.1862 – 21.10.1931), \emph{Schriftsteller, Mediziner}!Casanovas Heimfahrt1.7.1918 – 1.9.1918@\strich\emph{Casanovas Heimfahrt} {[}1.7.1918 – 1.9.1918{]}|pw}. Der andre hat mit
               Kunſt überhaupt nichts zu tun, iſt aber menſchlich so packend und traurig, daſs er
               Sie vielleicht intereſſirt; auch ein »Bernhardi\pwindex{Schnitzler, Arthur 15.05.1862 – 21.10.1931@\textsc{Schnitzler, Arthur} (15.05.1862 – 21.10.1931), \emph{Schriftsteller, Mediziner}!Professor Bernhardi. Komoedie in fuenf Akten1912@\strich\emph{Professor Bernhardi. Komödie in fünf Akten} {[}1912{]}|pwv}« hätte drüber nichts zu lachen! \pend
           \pstart
           Und nun wünsche ich Ihnen schöne, helle, frohe So{\geminationm}ertage!\pend
           \pstart
           Ihre{\\[\baselineskip]}\spacefill\mbox{Therese Rie.}\pend
           \leftskip=0em{}\endnumbering\briefempfaengerindex{Schnitzler, Arthur@\textsc{Schnitzler, Arthur}!zzzRie, Therese@\emph{von Therese Rie}!1923-07-231@{23. 7. 1923}|)be}\mylabel{h}\end{ledgroupsized}  \newcommand{\dateiname}{L02575}\newcommand{\titel}{Therese Rie-Andro an Arthur Schnitzler, 23. 7. 1923}\newcommand{\editorInnen}{Martin Anton Müller und Gerd-Hermann Susen}%% latex-leseansicht-abspann.tex
%% Abspann für die Leseansicht.
%% Der Schalter \ifkorrekturansicht ist bereits durch den Vorspann gesetzt.

%% latex-abspann.tex
%% Gemeinsamer Abspann für Korrekturansicht und Leseansicht.
%% Setzt den Schalter \ifkorrekturansicht voraus (gesetzt in den
%% einbindenden Dateien latex-korrekturansicht-abspann.tex bzw.
%% latex-leseansicht-abspann.tex).
%% ---------------------------------------------------------------

\normalsize

% Das esempio-Environment wird nur in der Leseansicht benötigt
\ifkorrekturansicht\else
\newenvironment{esempio}[3]%
{
    \vspace{1.5ex}
    \rlap{\underline{#1}}
    \par
    \setlength{\parindent}{0cm}
    \nopagebreak
    \leftskip=#2cm
    \rightskip=#3cm
}
{
    \par
}
\fi

\doendnotes{C}
\bigskip
\vfill

\clearpage

\footnotesize

\ifkorrekturansicht
  \lohead{\textsc{register}}
\fi

% theindex-Environment neu definieren ohne reledmac
\makeatletter
\renewenvironment{theindex}{%
  \ifkorrekturansicht
    \section*{\indexname}%
  \else
    \subsubsection*{Index der erwähnten Entitäten}%
  \fi
  \setlength{\parindent}{0pt}%
  \setlength{\parskip}{0pt plus 0.3pt}%
  \let\item\@idxitem
}{%
  \ifkorrekturansicht\clearpage\fi
}
\makeatother

\IfFileExists{\jobname-pw.ind}{\input{\jobname-pw.ind}}{}

% Quellenangabe nur in der Leseansicht
\ifkorrekturansicht\else
% Fallback-Definitionen, falls die .tex-Datei \titel etc. nicht gesetzt hat
\providecommand{\titel}{}
\providecommand{\editorInnen}{}
\providecommand{\dateiname}{\jobname}

\vspace{3cm}

\vfill

\footnotesize
\textsc{Quelle}: \titel. Herausgegeben von {\editorInnen}. In: \emph{Arthur Schnitzler: Briefwechsel mit Autorinnen und Autoren}.
 Digitale Edition, https://schnitzler-briefe.acdh.oeaw.ac.at/{\dateiname}.html (Stand \today)
\fi

\end{document}


      