%% latex-korrekturansicht-vorspann.tex
%% Vorspann für die Korrekturansicht.
%% Lädt die gemeinsame Datei latex-vorspann.tex mit gesetztem Schalter.

\newif\ifkorrekturansicht
\korrekturansichttrue

\input{../tex-inputs/latex-vorspann}


\section[ Felix Salten an Arthur Schnitzler, 8. 8. 1899]{L03296 Felix Salten an Arthur Schnitzler, 8. 8. 1899}
\nopagebreak\mylabel{L03296v}
\rehead{ }\normalsize\beginnumbering\briefempfaengerindex{Schnitzler, Arthur@\textsc{Schnitzler, Arthur}!zzzSalten, Felix@\emph{von Felix Salten}!1899-08-082@{8. 8. 1899}|(be}
\toendnotes[C]{\smallbreak\pagebreak[2]}\Standort{CUL, Schnitzler, B 89, A 2.}
\physDesc{Brief, 1 Blatt, 1 Seite, 281 Zeichen
\newline{}Handschrift: Bleistift, lateinische Kurrent
\newline{}Ordnung: mit Bleistift von unbekannter Hand nummeriert: »120« }\toendnotes[C]{\smallbreak}
\pstart
           \noindent{}{\pb}Lieber, vermuthlich habe ich Ihre Carte aus \label{K_L03296-1v}\edtext{Schluderbach\oindex{Carbonin@\textbf{Carbonin}, \emph{P.PPL}|pw}}{\lemma{\textnormal{\emph{Schluderbach}}}\Cendnote{\textnormal{Schnitzler hielt sich am 1. 8. 1899 sowie vom
                     5. 8. 1899 bis zum
                     6. 8. 1899 in
                     Carbonin\oindex{Carbonin@\textbf{Carbonin}, \emph{P.PPL}|pwk} (Schluderbach\oindex{Carbonin@\textbf{Carbonin}, \emph{P.PPL}|pwk}) auf. Womöglich hatte er die Karte am Beginn
                  seiner Wanderung mit Jakob Wassermann\pwindex{Wassermann, Jakob 10.03.1873 – 01.01.1934@\textsc{Wassermann, Jakob} (10.03.1873 – 01.01.1934), \emph{Schriftsteller/Schriftstellerin}|pwk} und
                     Richard Beer-Hofmann\pwindex{Beer-Hofmann, Richard 1866-07-11 – 1945-09-26@\textsc{Beer-Hofmann, Richard} (1866-07-11 – 1945-09-26), \emph{Schriftsteller/Schriftstellerin}|pwk} verfasst (siehe Felix Salten an Arthur Schnitzler, 27. 7. 1899). Nach Wien\oindex{Wien@\textbf{Wien}, \emph{A.ADM2}|pwk} kam er erst nach Wochen, am 12. 10. 1899, wieder.}}}\label{K_L03296-1} richtig gelesen, und
               Sie sind schon in Wien\oindex{Wien@\textbf{Wien}, \emph{A.ADM2}|pw}, oder kommen nächstens
               dahin. Ich reise Freitag von hier\oindex{Unterach am Attersee@\textbf{Unterach am Attersee}, \emph{P.PPL}|pwv} zurück.\pend
           
\pstart
           Wenn Sie da sind, senden Sie mir eine Zeile, wo wir uns treffen können.\pend
           
\pstart
           Wie geht es Ihnen?\pend
           
\pstart
           Herzlichst Ihr {\\[\baselineskip]}\spacefill\mbox{Salten}\pend
           \leftskip=0em{}
\pstart
           Unterach\oindex{Unterach am Attersee@\textbf{Unterach am Attersee}, \emph{P.PPL}|pw}{ }8/8 99\pend
           \selectlanguage{ngerman}\endnumbering\briefempfaengerindex{Schnitzler, Arthur@\textsc{Schnitzler, Arthur}!zzzSalten, Felix@\emph{von Felix Salten}!1899-08-082@{8. 8. 1899}|)be}\mylabel{L03296h}  \normalsize

\doendnotes{C}
\bigskip
\vfill

\clearpage

\footnotesize

\lohead{\textsc{register}}

% Definiere theindex-Environment komplett neu ohne reledmac
\makeatletter
\renewenvironment{theindex}{%
  \section*{\indexname}%
  \setlength{\parindent}{0pt}%
  \setlength{\parskip}{0pt plus 0.3pt}%
  \let\item\@idxitem
}{%
  \clearpage
}
\makeatother

\IfFileExists{\jobname-pw.ind}{\input{\jobname-pw.ind}}{}

\end{document}

      