%% latex-leseansicht-vorspann.tex
%% Vorspann für die Leseansicht.
%% Lädt die gemeinsame Datei latex-vorspann.tex mit nicht gesetztem Schalter.

\newif\ifkorrekturansicht
\korrekturansichtfalse

\input{../tex-inputs/latex-vorspann}


\section[ Felix Salten an Arthur Schnitzler, 8. 8. 1899]{L03296 Felix Salten an Arthur Schnitzler,  8. 8. 1899}
\nopagebreak\mylabel{L03296v}
\rehead{ }\normalsize\beginnumbering\briefempfaengerindex{Schnitzler, Arthur@\textsc{Schnitzler, Arthur}!zzzSalten, Felix@\emph{von Felix Salten}!1899-08-082@{8. 8. 1899}|(be}
\toendnotes[C]{\smallbreak\pagebreak[2]}
\correspDesc{Versand  durch Felix Salten am 8. 8. 1899 in Unterach am Attersee
\newline{}Weiterleitung  in Wien
\newline{}Erhalt  durch Arthur Schnitzler am [15. 8. 1899?] in Bad Ischl}\toendnotes[C]{\smallbreak}
\Standort{CUL, Schnitzler, B 89, A 2.}
\physDesc{Brief, 1 Blatt, 1 Seite, 281 Zeichen
\newline{}Handschrift: Bleistift, lateinische Kurrent
\newline{}Ordnung: mit Bleistift von unbekannter Hand nummeriert: »120« }\toendnotes[C]{\smallbreak}
\pstart
           \noindent{}{\pb}Lieber, vermuthlich habe ich Ihre Carte aus \label{K_L03296-1v}\edtext{Schluderbach\oindex{Carbonin@\textbf{Carbonin}|pw}}{\lemma{\textnormal{\emph{Schluderbach}}}\Cendnote{\textnormal{Schnitzler hielt sich am 1. 8. 1899 sowie vom
                     5. 8. 1899 bis zum
                     6. 8. 1899 in
                     Carbonin\oindex{Carbonin@\textbf{Carbonin}|pwk} (Schluderbach\oindex{Carbonin@\textbf{Carbonin}|pwk}) auf. Womöglich hatte er die Karte am Beginn
                  seiner Wanderung mit Jakob Wassermann\pwindex{Wassermann, Jakob 10.\,3.\,1873 Fürth – 1.\,1.\,1934 Altaussee@\textsc{Wassermann, Jakob} (10.\,3.\,1873 Fürth – 1.\,1.\,1934 Altaussee), \emph{Schriftsteller}|pwk} und
                     Richard Beer-Hofmann\pwindex{Beer-Hofmann, Richard 11.\,7.\,1866 Wien – 26.\,9.\,1945 New York City@\textsc{Beer-Hofmann, Richard} (11.\,7.\,1866 Wien – 26.\,9.\,1945 New York City), \emph{Schriftsteller}|pwk} verfasst (siehe XXXX Auszeichnungsfehler: Dokument L03295 nicht gefunden). Nach Wien\oindex{Wien@\textbf{Wien}, \emph{Verwaltungsgebiet}|pwk} kam er erst nach Wochen, am 12. 10. 1899, wieder.}}}\label{K_L03296-1} richtig gelesen, und
               Sie sind schon in Wien\oindex{Wien@\textbf{Wien}, \emph{Verwaltungsgebiet}|pw}, oder kommen nächstens
               dahin. Ich reise Freitag von hier\oindex{Unterach am Attersee@\textbf{Unterach am Attersee}|pwv} zurück.\pend
           
\pstart
           Wenn Sie da sind, senden Sie mir eine Zeile, wo wir uns treffen können.\pend
           
\pstart
           Wie geht es Ihnen?\pend
           
\pstart
           Herzlichst Ihr {\\[\baselineskip]}\spacefill\mbox{Salten}\pend
           \leftskip=0em{}
\pstart
           Unterach\oindex{Unterach am Attersee@\textbf{Unterach am Attersee}|pw}{ }8/8 99\pend
           \selectlanguage{ngerman}\endnumbering\briefempfaengerindex{Schnitzler, Arthur@\textsc{Schnitzler, Arthur}!zzzSalten, Felix@\emph{von Felix Salten}!1899-08-082@{8. 8. 1899}|)be}\mylabel{L03296h}  \newcommand{\dateiname}{L03296}\newcommand{\titel}{Felix Salten an Arthur Schnitzler, 8. 8. 1899}\newcommand{\editorInnen}{Martin Anton Müller und Laura Untner}%% latex-leseansicht-abspann.tex
%% Abspann für die Leseansicht.
%% Der Schalter \ifkorrekturansicht ist bereits durch den Vorspann gesetzt.

%% latex-abspann.tex
%% Gemeinsamer Abspann für Korrekturansicht und Leseansicht.
%% Setzt den Schalter \ifkorrekturansicht voraus (gesetzt in den
%% einbindenden Dateien latex-korrekturansicht-abspann.tex bzw.
%% latex-leseansicht-abspann.tex).
%% ---------------------------------------------------------------

\normalsize

% Das esempio-Environment wird nur in der Leseansicht benötigt
\ifkorrekturansicht\else
\newenvironment{esempio}[3]%
{
    \vspace{1.5ex}
    \rlap{\underline{#1}}
    \par
    \setlength{\parindent}{0cm}
    \nopagebreak
    \leftskip=#2cm
    \rightskip=#3cm
}
{
    \par
}
\fi

\doendnotes{C}
\bigskip
\vfill

\clearpage

\footnotesize

\ifkorrekturansicht
  \lohead{\textsc{register}}
\fi

% theindex-Environment neu definieren ohne reledmac
\makeatletter
\renewenvironment{theindex}{%
  \ifkorrekturansicht
    \section*{\indexname}%
  \else
    \subsubsection*{Index der erwähnten Entitäten}%
  \fi
  \setlength{\parindent}{0pt}%
  \setlength{\parskip}{0pt plus 0.3pt}%
  \let\item\@idxitem
}{%
  \ifkorrekturansicht\clearpage\fi
}
\makeatother

\IfFileExists{\jobname-pw.ind}{\input{\jobname-pw.ind}}{}

% Quellenangabe nur in der Leseansicht
\ifkorrekturansicht\else
% Fallback-Definitionen, falls die .tex-Datei \titel etc. nicht gesetzt hat
\providecommand{\titel}{}
\providecommand{\editorInnen}{}
\providecommand{\dateiname}{\jobname}

\vspace{3cm}

\vfill

\footnotesize
\textsc{Quelle}: \titel. Herausgegeben von {\editorInnen}. In: \emph{Arthur Schnitzler: Briefwechsel mit Autorinnen und Autoren}.
 Digitale Edition, https://schnitzler-briefe.acdh.oeaw.ac.at/{\dateiname}.html (Stand \today)
\fi

\end{document}


