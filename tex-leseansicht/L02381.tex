%% latex-korrekturansicht-vorspann.tex
%% Vorspann für die Korrekturansicht.
%% Lädt die gemeinsame Datei latex-vorspann.tex mit gesetztem Schalter.

\newif\ifkorrekturansicht
\korrekturansichttrue

\input{../tex-inputs/latex-vorspann}


\section[Felix Braun an Arthur Schnitzler, 13. 5. 1922]{L02381 Felix Braun an Arthur Schnitzler, 13. 5. 1922}
\nopagebreak\mylabel{L02381v}
\rehead{ }\normalsize\beginnumbering\briefempfaengerindex{Schnitzler, Arthur@\textsc{Schnitzler, Arthur}!zzzBraun, Felix@\emph{von Felix Braun}!1922-05-132@{13. 5. 1922}|(be}
\toendnotes[C]{\smallbreak\pagebreak[2]}\Standort{DLA, A:Schnitzler, HS.NZ85.1.5563.}
\physDesc{Brief, 1 Blatt, 3 Seiten, 2252 Zeichen
\newline{}Handschrift: blaue Tinte, deutsche Kurrent
\newline{}Schnitzler: 1) auf der ersten Seite mit Bleistift beschriftet: »\textsc{Felix Braun}«  2) mit rotem Buntstift zwei Unterstreichungen}
\buchAbdrucke{\weitereDrucke{Hans-Ulrich Lindken: \emph{Arthur Schnitzler. Aspekte und Akzente. Materialien zu Leben
                        und Werk}. Frankfurt am Main, Bern, Göttingen: \emph{Peter Lang} 1984, S. 410–411.} }\toendnotes[C]{\smallbreak}
\pstart
           \raggedleft{}{\pb}Wien\oindex{Wien@\textbf{Wien}, \emph{A.ADM2}|pw}, den 13. V. 1922\pend
           
\pstart
           \raggedleft{}XIX, Sieveringerstr. 191\oindex{Sieveringer Strasse@\textbf{Sieveringer Straße}, \emph{Straße (K.STR)}|pw}\pend
           
\pstart{}Verehrter Herr Doktor!\pend\vspace{0.5em}
\pstart
           Geſtatten Sie auch mir, Ihnen zu Ihrem sechzigſten Geburtstag einen herzlichen Gruß
               und Glückwunſch zu ſagen. Solche Tage haben \label{T_L02381-1v}\edtext{ihren}{\lemma{\textnormal{\emph{ihren}}}\Cendnote{\textnormal{Braun schreibt
                  fälschlich: »Ihren«.}}}\label{T_L02381-1} ſchönen Sinn darin, aus den ſonſt
               leider ſo verſchloſſenen Seelen der Menſchen hervorzuholen, was ſie aus Scheu, aus
               Trägheit, aus irgendwelchen Gebundenheiten lieber bei ſich behalten als kundgeben.
               Wie wenig wird dem Dichter doch zuteil, was er ſo ſehr nötig hat: die Verſicherung,
               daß ſeine Gaben empfangen, beherzigt, wirkſam geworden ſind. Dazu bedarf es der
               Gedenktage, die freilich allzu ſehr aufhäufen, was, weiſe verteilt, das ſchwere,
               harte Leben freudenreicher gemacht hätte. Nun, wir wollen uns deſſen darum nicht
               minder freuen.\pend
           
\pstart
           {\pb}Dem Dichter ſo vieler bedeutender, richtunggebender
               und ſchöner Werke muß nicht erſt geſagt werden, wer er iſt. Er weiß es ſelbſt und –
               wünſchen wirs! – würdigt den eignen Genius auch, der ihn ſo und nicht anders gebildet
               und geſtaltet hat. Die Fülle des Geſpendeten wird jetzt überſehen, die Ausleſe daraus
               reich genug getroffen werden können. Soviel iſt gewiß: daß die ſpätere Generation an
               das Maß Ihrer meiſterlichen Schöpfungen nicht im Entfernteſten herangereicht hat, daß
               überhaupt das ſtrenge Künſtlertum des Aufbaus und der Geſtalt von keinem der
               Nachſtrebenden eingehalten worden iſt. Möchte Sie dies Bewußtſein, verehrter Herr
               Doktor, mit Freude erfüllen und zu weiterer Dichtung und Arbeit drängen!\pend
           
\pstart
           Ich wünſche vor allem: Geſundheit und Lebensfreude, die ja doch die Grundlagen aller
               unſerer Kräfte ſind. Wenn dieſer freudige {\pb}Tag die
               letztere nur recht befeſtigte, ſo wäre er ſchon darum zu loben; die erſtere wird
               hoffentlich der Arzt in Ihnen nicht minder künſtleriſch als ein Werk zu erhalten und
               zu fördern wiſſen. Zum Dritten endlich wünſche ich, es möchte Ihnen vergönnt ſein,
               immer Schöneres hervorzubringen – dieſer Wunſch wird Ihnen wohl der liebſte ſein, dem
               jedenfalls werden Sie nicht entgegen wirken mögen. In einem Augenblick wie dieſem
               brauchen wir die Dichter – die nämlich, die es wirklich ſind – mehr als je. Wenn nur
               ſie es nicht überdrüſſig \substVorne{}\textsuperscript{ſind}\substDazwischen{}werden\substHinten{}, den immer tauben Ohren und immer blinden Augen zu geben!\pend
           
\pstart
           Herzlichſt grüßend verbleibe ich in Verehrung ſtets Ihr ergebener{\\[\baselineskip]}\spacefill\mbox{Felix Braun}\pend
           \leftskip=0em{}\selectlanguage{ngerman}\endnumbering\briefempfaengerindex{Schnitzler, Arthur@\textsc{Schnitzler, Arthur}!zzzBraun, Felix@\emph{von Felix Braun}!1922-05-132@{13. 5. 1922}|)be}\mylabel{L02381h}  \normalsize

\doendnotes{C}
\bigskip
\vfill

\clearpage

\footnotesize

\lohead{\textsc{register}}

% Definiere theindex-Environment komplett neu ohne reledmac
\makeatletter
\renewenvironment{theindex}{%
  \section*{\indexname}%
  \setlength{\parindent}{0pt}%
  \setlength{\parskip}{0pt plus 0.3pt}%
  \let\item\@idxitem
}{%
  \clearpage
}
\makeatother

\IfFileExists{\jobname-pw.ind}{\input{\jobname-pw.ind}}{}

\end{document}

      