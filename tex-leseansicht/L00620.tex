%% latex-korrekturansicht-vorspann.tex
%% Vorspann für die Korrekturansicht.
%% Lädt die gemeinsame Datei latex-vorspann.tex mit gesetztem Schalter.

\newif\ifkorrekturansicht
\korrekturansichttrue

\input{../tex-inputs/latex-vorspann}


\section[Hermann Bahr an Arthur Schnitzler, 11. 11. 1896]{L00620 Hermann Bahr an Arthur Schnitzler, 11. 11. 1896}
\nopagebreak\mylabel{L00620v}
\rehead{ }\normalsize\beginnumbering\briefempfaengerindex{Schnitzler, Arthur@\textsc{Schnitzler, Arthur}!zzzBahr, Hermann@\emph{von Hermann Bahr}!1896-11-111@{11. 11. 1896}|(be}
\toendnotes[C]{\smallbreak\pagebreak[2]}\Standort{CUL, Schnitzler, B 5b.}
\physDesc{Brief, 1 Blatt, 2 Seiten, 508 Zeichen
\newline{}Handschrift: schwarze Tinte, deutsche Kurrent
\newline{}Ordnung: mit Bleistift von unbekannter Hand nummeriert:
                                    »46« }
\buchAbdrucke{\weitereDrucke{Hermann Bahr, Arthur Schnitzler: \emph{Briefwechsel, Aufzeichnungen, Dokumente (1891–1931)}. Göttingen: \emph{Wallstein} 2018, S. 130.} }\toendnotes[C]{\smallbreak}
\pstart
           {\pb}\textcolor{gray}{\textbf{»Die Zeit\orgindex{Zeit. Wiener Wochenschrift@Die Zeit. Wiener Wochenschrift|pw}«}}\hfill \textcolor{gray}{\textbf{\textbf{Wien\oindex{Wien@\textbf{Wien}, \emph{A.ADM2}|pw}}, den }}11. November \textcolor{gray}{\textbf{189}}6\pend
           
\pstart
           \textcolor{gray}{\textbf{Wiener Wochenſchrift}}\hfill \textcolor{gray}{\textbf{IX/3, Günthergaſſe 1\oindex{Guenthergasse@\textbf{Günthergasse}, \emph{Straße (K.STR)}|pw}.}}\pend
           
\pstart
           \textcolor{gray}{\textbf{\textbf{Herausgeber}:}}{\\}\textcolor{gray}{\textbf{Profeſſor Dr. I. Singer\pwindex{Singer, Isidor 16.01.1857 – 08.12.1927@\textsc{Singer, Isidor} (16.01.1857 – 08.12.1927), \emph{Journalist/Journalistin, Herausgeber/Herausgeberin, Soziologe/Soziologin}|pw}, Hermann Bahr\pwindex{Bahr, Hermann 19.07.1863 – 15.01.1934@\textsc{Bahr, Hermann} (19.07.1863 – 15.01.1934), \emph{Schriftsteller/Schriftstellerin, Kritiker/Kritikerin}|pw},
                        Dr. Heinrich Kanner\pwindex{Kanner, Heinrich 09.11.1864 – 15.02.1930@\textsc{Kanner, Heinrich} (09.11.1864 – 15.02.1930), \emph{Herausgeber/Herausgeberin, Publizist/Publizistin}|pw}.}}\pend
           
\pstart
           \textcolor{gray}{\textbf{Telephon Nr. 6415.}}\pend
           
\pstart\center{}Lieber Arthur!\pend\vspace{0.5em}
\pstart
           Ich werde mich ſehr freuen, Dich bei mir zu ſehen. Donnerſtag, Freitag, Samſtag bin
               ich zur angegebenen Zeit, von 11–1, meiſtens nicht daheim. An den
               anderen Tagen {\pb}iſt es ziemlich ſicher, daß Du mich
               triffſt, am Sicherſten natürlich, we{\geminationn} Du noch ſo
               freundlich biſt zu telephonieren.\pend
           
\pstart
           Ich \label{K_L00620-1v}\edtext{wohne jetzt}{\lemma{\textnormal{\emph{wohne jetzt}}}\Cendnote{\textnormal{Bahrs Übersiedlung fand am
                     4./5. 11. statt.}}}\label{K_L00620-1}{ }IX Porzellangaſſe 37\oindex{Porzellangasse@\textbf{Porzellangasse}, \emph{Straße (K.STR)}|pw} 4. St., mit Aufzug. Komm
               bald; ich laß Dich dann nicht mehr fort, bis Du mir die neue \label{K_L00620-2v}\edtext{Novelle\pwindex{Frau des Weisen. Erzaehlung@\emph{Die Frau des Weisen. Erzählung}|pwv}}{\lemma{\textnormal{\emph{Novelle}}}\Cendnote{\textnormal{\emph{Die Frau des Weisen}\pwindex{Frau des Weisen. Erzaehlung@\emph{Die Frau des Weisen. Erzählung}|pwk}}}}\label{K_L00620-2} zugeſchworen haſt.\pend
           
\pstart
           Herzlichſt{\\[\baselineskip]}Dein{\\[\baselineskip]}\spacefill\mbox{hm}\pend
           \leftskip=0em{}
\pstart
           \noindent{}Herrn \textsc{Dr Arthur Schnitzler}{\\}\textsc{IX Frankgasse 1\oindex{Frankgasse 1@\textbf{Frankgasse 1}, \emph{Wohngebäude (K.WHS)}|pw}}\pend
           
\pstart
           \textcolor{gray}{\textbf{\label{T_L00620-1v}\edtext{Alle für »Die Zeit\orgindex{Zeit. Wiener Wochenschrift@Die Zeit. Wiener Wochenschrift|pw}« beſtimmten Zuſchriften und Sendungen ſind an die
                  Redaction der »Zeit\orgindex{Zeit. Wiener Wochenschrift@Die Zeit. Wiener Wochenschrift|pw}« und \textbf{nicht} an die Perſon eines der Herausgeber zu richten.}{\lemma{\textnormal{\emph{Alle … richten.}}}\Cendnote{\textnormal{am unteren Rand der ersten Seite}}}\label{T_L00620-1}}}\pend
           \selectlanguage{ngerman}\endnumbering\briefempfaengerindex{Schnitzler, Arthur@\textsc{Schnitzler, Arthur}!zzzBahr, Hermann@\emph{von Hermann Bahr}!1896-11-111@{11. 11. 1896}|)be}\mylabel{L00620h}  \normalsize

\doendnotes{C}
\bigskip
\vfill

\clearpage

\footnotesize

\lohead{\textsc{register}}

% Definiere theindex-Environment komplett neu ohne reledmac
\makeatletter
\renewenvironment{theindex}{%
  \section*{\indexname}%
  \setlength{\parindent}{0pt}%
  \setlength{\parskip}{0pt plus 0.3pt}%
  \let\item\@idxitem
}{%
  \clearpage
}
\makeatother

\IfFileExists{\jobname-pw.ind}{\input{\jobname-pw.ind}}{}

\end{document}

      