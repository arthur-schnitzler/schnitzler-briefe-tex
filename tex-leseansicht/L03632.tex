%% latex-leseansicht-vorspann.tex
%% Vorspann für die Leseansicht.
%% Lädt die gemeinsame Datei latex-vorspann.tex mit nicht gesetztem Schalter.

\newif\ifkorrekturansicht
\korrekturansichtfalse

\input{../tex-inputs/latex-vorspann}


\section[Stefan Zweig an Arthur Schnitzler, 6. 7. 1911]{L03632 Stefan Zweig an Arthur Schnitzler, 6. 7. 1911}
\nopagebreak\mylabel{L03632v}
\rehead{ }\normalsize\beginnumbering\briefempfaengerindex{Schnitzler, Arthur@\textsc{Schnitzler, Arthur}!zzzZweig, Stefan@\emph{von Stefan Zweig}!1911-07-061@{6. 7. 1911}|(be}
\toendnotes[C]{\smallbreak\pagebreak[2]}
\correspDesc{Versand  durch Stefan Zweig am 6. 7. 1911 in Hochschneeberg
\newline{}Erhalt  durch Arthur Schnitzler im Zeitraum [7. 7. 1911 –
                  8. 7. 1911?] in Wien}\toendnotes[C]{\smallbreak}
\Standort{CUL, Schnitzler, B 118.}
\physDesc{Bildpostkarte, 287 Zeichen
\newline{}Handschrift: schwarze Tinte, lateinische Kurrent
\newline{}Versand: Stempel: »\nobreak{}\oindex{Hochschneeberg@\textbf{Hochschneeberg}, \emph{Gebirge}|pwk}Hochschneeberg, 6. VII 11, VIII\nobreak{}«.  }
\buchAbdrucke{\weitereDrucke{Stefan Zweig: \emph{Briefwechsel mit Hermann Bahr, Sigmund Freud, Rainer Maria
                        Rilke und Arthur Schnitzler}. Herausgegeben von Jeffrey B. Berlin, Hans-Ulrich Lindken und Donald A. Prater. Frankfurt am Main: \emph{S. Fischer} 1987, S. 365.} }\toendnotes[C]{\smallbreak}\pstart{}{\pb}D\textsuperscript{r} Artur
                  Schnitzler\pend{}\pstart{}Wien – Cottage\oindex{Wien@\textbf{Wien}!XVIII., Währing@\textbf{XVIII., Währing}!Währinger Cottage@\textbf{Währinger Cottage}, \emph{Teil eines besiedelten Ortes}|pw}\pend{}\pstart{}Sternwartestrasse 71\oindex{Wien@\textbf{Wien}!XVIII., Währing@\textbf{XVIII., Währing}!Sternwartestraße 71@\textbf{Sternwartestraße 71}, \emph{Wohngebäude}|pw}\pend{}{\bigskip}
\pstart
           \noindent{}\centering{}{\pb}\textcolor{gray}{\textbf{Schneeberg\oindex{Schneeberg@\textbf{Schneeberg}, \emph{Berg}|pw}. Klosterwappen\oindex{Klosterwappen@\textbf{Klosterwappen}, \emph{Berg}|pw} 2076m, Kaiserstein\oindex{Kaiserstein@\textbf{Kaiserstein}, \emph{Bergspitze}|pw}
                  und Fischerhütte\oindex{Fischerhütte@\textbf{Fischerhütte}, \emph{Beherbergungsgebäude}|pw} 2036 m vom Franz Josef-Weg\oindex{Kaiser-Franz-Joseph-Promenade@\textbf{Kaiser-Franz-Joseph-Promenade}, \emph{Wanderweg}|pw}.}}\pend
           
\pstart
           \textcolor{gray}{\textbf{Klosterwappen\oindex{Klosterwappen@\textbf{Klosterwappen}, \emph{Berg}|pw}}}\hfill \textcolor{gray}{\textbf{Fischerhütte\oindex{Fischerhütte@\textbf{Fischerhütte}, \emph{Beherbergungsgebäude}|pw}}}\pend
           \vspace{1em}
\pstart
           \noindent{}{\pb}Verehrter Herr Doktor, ich sende Ihnen von hier die ergebensten
               Grüsse: mir ist’s noch lieber als der Semmering\oindex{Semmering@\textbf{Semmering}, \emph{Verwaltungsgebiet}|pw},
               eine crystallne Klarheit der Luft und wunderbare Stille. Viele Empfehlungen Ihrer
               Frau Gemahlin\pwindex{Schnitzler, Olga 17.\,1.\,1882 Wien – 13.\,1.\,1970 Lugano@\textsc{Schnitzler, Olga} (17.\,1.\,1882 Wien – 13.\,1.\,1970 Lugano), \emph{Schauspielerin, Sängerin}|pwv} von Ihrem
               getreuen\pend
           \pstart \spacefill\mbox{Stefan Zweig}\pend{}\selectlanguage{ngerman}\endnumbering\briefempfaengerindex{Schnitzler, Arthur@\textsc{Schnitzler, Arthur}!zzzZweig, Stefan@\emph{von Stefan Zweig}!1911-07-061@{6. 7. 1911}|)be}\mylabel{L03632h}  \newcommand{\dateiname}{L03632}\newcommand{\titel}{Stefan Zweig an Arthur Schnitzler, 6. 7. 1911}\newcommand{\editorInnen}{Selma Jahnke und Martin Anton Müller}%% latex-leseansicht-abspann.tex
%% Abspann für die Leseansicht.
%% Der Schalter \ifkorrekturansicht ist bereits durch den Vorspann gesetzt.

%% latex-abspann.tex
%% Gemeinsamer Abspann für Korrekturansicht und Leseansicht.
%% Setzt den Schalter \ifkorrekturansicht voraus (gesetzt in den
%% einbindenden Dateien latex-korrekturansicht-abspann.tex bzw.
%% latex-leseansicht-abspann.tex).
%% ---------------------------------------------------------------

\normalsize

% Das esempio-Environment wird nur in der Leseansicht benötigt
\ifkorrekturansicht\else
\newenvironment{esempio}[3]%
{
    \vspace{1.5ex}
    \rlap{\underline{#1}}
    \par
    \setlength{\parindent}{0cm}
    \nopagebreak
    \leftskip=#2cm
    \rightskip=#3cm
}
{
    \par
}
\fi

\doendnotes{C}
\bigskip
\vfill

\clearpage

\footnotesize

\ifkorrekturansicht
  \lohead{\textsc{register}}
\fi

% theindex-Environment neu definieren ohne reledmac
\makeatletter
\renewenvironment{theindex}{%
  \ifkorrekturansicht
    \section*{\indexname}%
  \else
    \subsubsection*{Index der erwähnten Entitäten}%
  \fi
  \setlength{\parindent}{0pt}%
  \setlength{\parskip}{0pt plus 0.3pt}%
  \let\item\@idxitem
}{%
  \ifkorrekturansicht\clearpage\fi
}
\makeatother

\IfFileExists{\jobname-pw.ind}{\input{\jobname-pw.ind}}{}

% Quellenangabe nur in der Leseansicht
\ifkorrekturansicht\else
% Fallback-Definitionen, falls die .tex-Datei \titel etc. nicht gesetzt hat
\providecommand{\titel}{}
\providecommand{\editorInnen}{}
\providecommand{\dateiname}{\jobname}

\vspace{3cm}

\vfill

\footnotesize
\textsc{Quelle}: \titel. Herausgegeben von {\editorInnen}. In: \emph{Arthur Schnitzler: Briefwechsel mit Autorinnen und Autoren}.
 Digitale Edition, https://schnitzler-briefe.acdh.oeaw.ac.at/{\dateiname}.html (Stand \today)
\fi

\end{document}


