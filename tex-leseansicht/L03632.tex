%% latex-korrekturansicht-vorspann.tex
%% Vorspann für die Korrekturansicht.
%% Lädt die gemeinsame Datei latex-vorspann.tex mit gesetztem Schalter.

\newif\ifkorrekturansicht
\korrekturansichttrue

\input{../tex-inputs/latex-vorspann}


\section[Stefan Zweig an Arthur Schnitzler, 6. 7. 1911]{L03632 Stefan Zweig an Arthur Schnitzler, 6. 7. 1911}
\nopagebreak\mylabel{L03632v}
\rehead{ }\normalsize\beginnumbering\briefempfaengerindex{Schnitzler, Arthur@\textsc{Schnitzler, Arthur}!zzzZweig, Stefan@\emph{von Stefan Zweig}!1911-07-061@{6. 7. 1911}|(be}
\toendnotes[C]{\smallbreak\pagebreak[2]}\Standort{CUL, Schnitzler, B 118.}
\physDesc{Bildpostkarte, 287 Zeichen
\newline{}Handschrift: schwarze Tinte, lateinische Kurrent
\newline{}Versand: Stempel: »\nobreak{}\oindex{Hochschneeberg@\textbf{Hochschneeberg}, \emph{T.MTS}|pwk}Hochschneeberg, 6. VII 11, VIII\nobreak{}«.  }
\buchAbdrucke{\weitereDrucke{Stefan Zweig: \emph{Briefwechsel mit Hermann Bahr, Sigmund Freud, Rainer Maria
                        Rilke und Arthur Schnitzler}. Frankfurt am Main: \emph{S. Fischer} 1987, S. 365.} }\toendnotes[C]{\smallbreak}\pstart{}{\pb}D\textsuperscript{r}
                  Artur Schnitzler\pend{}\pstart{}Wien – Cottage\oindex{Waehringer Cottage@\textbf{Währinger Cottage}, \emph{Teil eines besiedelten Ortes (A.BSOX)}|pw}\pend{}\pstart{}Sternwartestrasse 71\oindex{Sternwartestrasse 71@\textbf{Sternwartestraße 71}, \emph{Wohngebäude (K.WHS)}|pw}\pend{}{\bigskip}
\pstart
           \noindent{}\centering{}{\pb}\textcolor{gray}{\textbf{Schneeberg\oindex{Schneeberg@\textbf{Schneeberg}, \emph{Berg (N.BRG)}|pw}. Klosterwappen\oindex{Klosterwappen@\textbf{Klosterwappen}, \emph{Berg (N.BRG)}|pw}
                  2076m, Kaiserstein\oindex{Kaiserstein@\textbf{Kaiserstein}, \emph{T.PK}|pw} und Fischerhütte\oindex{Fischerhuette@\textbf{Fischerhütte}, \emph{Beherbergungsgebäude (K.BHB)}|pw} 2036 m vom Franz
               Josef-Weg\oindex{Kaiser-Franz-Joseph-Promenade@\textbf{Kaiser-Franz-Joseph-Promenade}, \emph{Wanderweg (K.WND)}|pw}.}}\pend
           
\pstart
           \textcolor{gray}{\textbf{Klosterwappen\oindex{Klosterwappen@\textbf{Klosterwappen}, \emph{Berg (N.BRG)}|pw}}}\hfill \textcolor{gray}{\textbf{Fischerhütte\oindex{Fischerhuette@\textbf{Fischerhütte}, \emph{Beherbergungsgebäude (K.BHB)}|pw}}}\pend
           \vspace{1em}
\pstart
           \noindent{}{\pb}Verehrter Herr Doktor,
               ich sende Ihnen von hier die ergebensten Grüsse: mir ist’s noch lieber als der Semmering\oindex{Semmering@\textbf{Semmering}, \emph{A.ADM3}|pw}, eine crystallne Klarheit der Luft und
               wunderbare Stille. Viele Empfehlungen Ihrer Frau Gemahlin\pwindex{Schnitzler, Olga 17.01.1882 – 13.01.1970@\textsc{Schnitzler, Olga} (17.01.1882 – 13.01.1970), \emph{Schauspieler/Schauspielerin, Sänger/Sängerin}|pwv} von Ihrem getreuen \pend
           \pstart \spacefill\mbox{Stefan Zweig}\pend{}\selectlanguage{ngerman}\endnumbering\briefempfaengerindex{Schnitzler, Arthur@\textsc{Schnitzler, Arthur}!zzzZweig, Stefan@\emph{von Stefan Zweig}!1911-07-061@{6. 7. 1911}|)be}\mylabel{L03632h}  \normalsize

\doendnotes{C}
\bigskip
\vfill

\clearpage

\footnotesize

\lohead{\textsc{register}}

% Definiere theindex-Environment komplett neu ohne reledmac
\makeatletter
\renewenvironment{theindex}{%
  \section*{\indexname}%
  \setlength{\parindent}{0pt}%
  \setlength{\parskip}{0pt plus 0.3pt}%
  \let\item\@idxitem
}{%
  \clearpage
}
\makeatother

\IfFileExists{\jobname-pw.ind}{\input{\jobname-pw.ind}}{}

\end{document}

      