%% latex-leseansicht-vorspann.tex
%% Vorspann für die Leseansicht.
%% Lädt die gemeinsame Datei latex-vorspann.tex mit nicht gesetztem Schalter.

\newif\ifkorrekturansicht
\korrekturansichtfalse

\input{../tex-inputs/latex-vorspann}


\section[Paul Goldmann an Arthur Schnitzler, 9. 6. {[}1891{]}]{L02664 Paul Goldmann an Arthur Schnitzler, 9. 6. [1891]}
\nopagebreak\mylabel{L02664v}
\rehead{ }\normalsize\beginnumbering\briefempfaengerindex{Schnitzler, Arthur@\textsc{Schnitzler, Arthur}!zzzGoldmann, Paul@\emph{von Paul Goldmann}!1891-06-091@{9. 6. [1891]}|(be}
\toendnotes[C]{\smallbreak\pagebreak[2]}
\correspDesc{Versand  durch Paul Goldmann am 9. 6. [1891] in Brüssel
\newline{}Erhalt  durch Arthur Schnitzler im Zeitraum [10. 6. 1891
                  – 14. 6. 1891?] in Wien}\toendnotes[C]{\smallbreak}
\Standort{DLA, A:Schnitzler, HS.NZ85.1.3162.}
\physDesc{Brief, 3 Blätter, 7 Seiten, 5931 Zeichen
\newline{}Handschrift: schwarze Tinte, deutsche Kurrent
\newline{}Schnitzler: mit rotem Buntstift eine Unterstreichung }\toendnotes[C]{\smallbreak}
\pstart
           \raggedleft{}{\pb}\textsc{Brüssel\oindex{Brüssel@\textbf{Brüssel}, \emph{Hauptstadt}|pw}}, 9. Juni.\pend
           
\pstart\center{}Mein lieber Arthur!\pend\vspace{0.5em}
\pstart
           Das iſt der Unterſchied zwiſchen Freundſchafts- und Liebescorreſpondenz: die Liebe
               will Gefühle, und die Freundſchaft wird durch Gefühle auf die Dauer gelangweilt und
               will Thatſachen. Dieſe Wochen, in denen ich Dir nicht geſchrieben,{ }ſollten alſo eine
               kleine Thatſachen-Sparbüchſe{ }ſein; und jetzt, wo ich meine Erſparniſſe in dieſer
               Beziehung nachſehe, finde ich nichts und kann Dir wieder nichts bieten als ein Paar{ }ſchäbige Stimmungen und Empfindungen. Der Grund für den Thatſachenmangel iſt vor
               Allem der, daß ich die Hautpzeit des Tages allein auf meinem Zimmer und mit meiner
               Arbeit verbringe. Meine Empfehlungen habe ich wohl abgegeben, aber{ }ſie haben zu
               nichts geführt; ausgeſuchte Höflichkeit überall; aber die Höflichkeit iſt ein gar
               matter Wärmeſpender;{ }ſie erwärmt nicht mehr als ein flüchtiger Händedruck, und das
               Herz kann dabei erfrieren. Da und dort hat man mich zum Diner eingeladen, und war
               froh, als der eigenthümliche Geiſt, dem man Alles Zweimal{ }ſagen mußte, um von ihm
               verſtanden zu werden, und der{ }ſelbſt ein jämmerliches Stottern vorführte, die {\pb}Thür hinter{ }ſich zumachte. Ein klein wenig näher –
               aber auch nichts weniger als intim – verkehr ich mit einem jungen \label{K_L02664-1v}\edtext{Manne\pwindex{?? [Leiter einer Glasfabrik] *~um 1869@\textsc{?? [Leiter einer Glasfabrik]} (*~um 1869)|pwv}}{\lemma{\textnormal{\emph{Manne}}}\Cendnote{\textnormal{nicht identifiziert}}}\label{K_L02664-1} (22 Jahre),
                  Erbe\pwindex{?? [Leiter einer Glasfabrik] *~um 1869@\textsc{?? [Leiter einer Glasfabrik]} (*~um 1869)|pwv} und Leiter\pwindex{?? [Leiter einer Glasfabrik] *~um 1869@\textsc{?? [Leiter einer Glasfabrik]} (*~um 1869)|pwv} einer großen \label{K_L02664-2v}\edtext{Glasfabrik\orgindex{?? [Glasfabrik in Belgien]@?? [Glasfabrik in Belgien]|pwv}}{\lemma{\textnormal{\emph{Glasfabrik}}}\Cendnote{\textnormal{nicht ermittelt}}}\label{K_L02664-2}; demgemäß ein
               wenig{ }ſtolz und \label{K_L02664-3v}\edtext{\textsc{\begin{otherlanguage}{french}hautain\end{otherlanguage}}}{\lemma{\textnormal{\emph{hautain}}}\Cendnote{\textnormal{französisch: hochmütig, unnahbar}}}\label{K_L02664-3},
               aber wohlerzogen genug, um das dem ihm warm empfohlenen Fremden nicht zu zeigen. Im
               Allgemeinen ein{ }ſehr hübſcher, \strikeout{\textcolor{gray}{aſ}} äſthetiſch angenehmer Menſch\pwindex{?? [Leiter einer Glasfabrik] *~um 1869@\textsc{?? [Leiter einer Glasfabrik]} (*~um 1869)|pwv} – eine Art \textsc{Boris Fanjung\pwindex{Van-Jung, Boris 15.\,10.\,1872 Odessa – 3.\,10.\,1899 Wien@\textsc{Van-Jung, Boris} (15.\,10.\,1872 Odessa – 3.\,10.\,1899 Wien), \emph{Mediziner}|pw}}, nur viel feiner und hochſtehender. Ein wenig Kunſtdilettant\pwindex{?? [Leiter einer Glasfabrik] *~um 1869@\textsc{?? [Leiter einer Glasfabrik]} (*~um 1869)|pwv} und reizend, wenn er{ }ſeine
               naiven Pläne entwickelt »\label{K_L02664-4v}\edtext{\textsc{\begin{otherlanguage}{french}de joindre l’art à l’industrie\end{otherlanguage}}}{\lemma{\textnormal{\emph{de … l’industrie}}}\Cendnote{\textnormal{französisch: die Kunst mit der Industrie
                  zu verbinden}}}\label{K_L02664-4}«. Vor Allem aber –{ }ſtrenggläubiger Katholik\pwindex{?? [Leiter einer Glasfabrik] *~um 1869@\textsc{?? [Leiter einer Glasfabrik]} (*~um 1869)|pwv}, der allſonntäglich zur Meſſe geht
               und{ }ſich auf nichts in der Welt mehr freut, als auf{ }ſein Fortleben nach dem Tode.
               Dazu eine blonde, äußerlich unbedeutende,{ }ſehr fromme, \strikeout{und}{ }ſehr{ }ſanfte und{ }ſehr kurzſichtige \label{K_L02664-5v}\edtext{Schweſter\pwindex{?? [Schwester eines Glasfabrikanten] @\textsc{?? [Schwester eines Glasfabrikanten]}|pwv}}{\lemma{\textnormal{\emph{Schwester}}}\Cendnote{\textnormal{nicht identifiziert}}}\label{K_L02664-5} mit einem
               ewigen \label{K_L02664-6v}\edtext{Lorgnon}{\lemma{\textnormal{\emph{Lorgnon}}}\Cendnote{\textnormal{Brille mit Haltestiel}}}\label{K_L02664-6} und mit Redensarten wie »\label{K_L02664-7v}\edtext{\textsc{\begin{otherlanguage}{french}Jésus es mon ami intime\end{otherlanguage}}}{\lemma{\textnormal{\emph{Jésus es mon ami intime}}}\Cendnote{\textnormal{französisch: Jesus ist mein enger
                  Freund}}}\label{K_L02664-7}«. Fürſtlicher Haushalt, nicht ohne Stimmung das Ganze – aber doch
               ohne rechte Wärme{\dots} Außerdem iſt da in Brüſſel\oindex{Brüssel@\textbf{Brüssel}, \emph{Hauptstadt}|pw} der Chefredacteur\pwindex{Tardieu, Charles 9.\,2.\,1838 – 1909@\textsc{Tardieu, Charles} (9.\,2.\,1838 – 1909), \emph{Journalist, Chefredakteur}|pwv} der »\textsc{Indépendance belge\orgindex{Indépendance Belge@L’Indépendance Belge|pw}}« (Geograph wie Du biſt, wirſt Du fragen, {\pb}wieſo Brüſſel\oindex{Brüssel@\textbf{Brüssel}, \emph{Hauptstadt}|pw} zu Belgien\oindex{Belgien@\textbf{Belgien}|pw} kommt, aber ich kann Dir verrathen, daß es die Hauptſtadt\oindex{Brüssel@\textbf{Brüssel}, \emph{Hauptstadt}|pwv} davon iſt). Dieſer also, \textsc{M. Tardieu\pwindex{Tardieu, Charles 9.\,2.\,1838 – 1909@\textsc{Tardieu, Charles} (9.\,2.\,1838 – 1909), \emph{Journalist, Chefredakteur}|pw}}, iſt ein durchaus charmanter Menſch\pwindex{Tardieu, Charles 9.\,2.\,1838 – 1909@\textsc{Tardieu, Charles} (9.\,2.\,1838 – 1909), \emph{Journalist, Chefredakteur}|pwv}, der einzige echte Franzoſe\pwindex{Tardieu, Charles 9.\,2.\,1838 – 1909@\textsc{Tardieu, Charles} (9.\,2.\,1838 – 1909), \emph{Journalist, Chefredakteur}|pwv}, den ich bisher kennen gelernt, Cavalier\pwindex{Tardieu, Charles 9.\,2.\,1838 – 1909@\textsc{Tardieu, Charles} (9.\,2.\,1838 – 1909), \emph{Journalist, Chefredakteur}|pwv}, unermüdlicher und geiſtſprühender
                  Plauderer\pwindex{Tardieu, Charles 9.\,2.\,1838 – 1909@\textsc{Tardieu, Charles} (9.\,2.\,1838 – 1909), \emph{Journalist, Chefredakteur}|pwv} und profunder
                  Kunſtkenner\pwindex{Tardieu, Charles 9.\,2.\,1838 – 1909@\textsc{Tardieu, Charles} (9.\,2.\,1838 – 1909), \emph{Journalist, Chefredakteur}|pwv}, Specialist\pwindex{Tardieu, Charles 9.\,2.\,1838 – 1909@\textsc{Tardieu, Charles} (9.\,2.\,1838 – 1909), \emph{Journalist, Chefredakteur}|pwv} für niederländiſche\oindex{Niederlande@\textbf{Niederlande}|pw} Malerei und \label{K_L02664-8v}\edtext{enragirter}{\lemma{\textnormal{\emph{enragirter}}}\Cendnote{\textnormal{begeisterter}}}\label{K_L02664-8}Wagner\pwindex{Wagner, Richard 22.\,5.\,1813 Leipzig – 13.\,2.\,1883 Venedig@\textsc{Wagner, Richard} (22.\,5.\,1813 Leipzig – 13.\,2.\,1883 Venedig), \emph{Komponist}|pw}ianer. Der Chefredacteur\pwindex{Tardieu, Charles 9.\,2.\,1838 – 1909@\textsc{Tardieu, Charles} (9.\,2.\,1838 – 1909), \emph{Journalist, Chefredakteur}|pwv} der »\textsc{Indépendance\orgindex{Indépendance Belge@L’Indépendance Belge|pw}}« iſt natürlich in Brüſſel\oindex{Brüssel@\textbf{Brüssel}, \emph{Hauptstadt}|pw} ein großer Mann\pwindex{Tardieu, Charles 9.\,2.\,1838 – 1909@\textsc{Tardieu, Charles} (9.\,2.\,1838 – 1909), \emph{Journalist, Chefredakteur}|pwv} – wenn \introOben{}er\introOben{} auch von dem Größenwahn der Wien\oindex{Wien@\textbf{Wien}, \emph{Verwaltungsgebiet}|pw}er Zeitungsſ\textcolor{gray}{a}ujuden keine Spur beſitzt – und hat
               Beſſeres zu thun, als mit dem Correſpondenten der »Frankfurter Zeitung\orgindex{Frankfurter Zeitung@Frankfurter Zeitung|pw}« zu verkehren; aber alle 8 Tage ergibt{ }ſich doch eine
               Plauder-Viertelſtunde auf{ }ſeiner Redaction\orgindex{Indépendance Belge@L’Indépendance Belge|pwv}sſtube, die ich dann immer höchlich angeregt verlaſſe. Und dann iſt
                  Brüſſel\oindex{Brüssel@\textbf{Brüssel}, \emph{Hauptstadt}|pw}{ }ſelbſt – elegante und{ }ſympathiſche Stadt\oindex{Brüssel@\textbf{Brüssel}, \emph{Hauptstadt}|pwv}. Schöne Leute. Und vor
               Allem eine große hiſtoriſche Vergangenheit – die gewiſſe gothiſche Bettdecke, die man{ }ſich über die \strikeout{Ohr} Ohren zieht, wenn man von der
               Gegenwart nichts wiſſen will. {\pb}Viel Kunst –
               herrliche alte und elende neue: Ein Muſeum\orgindex{Musées royaux des Beaux-Arts de Belgique@Musées royaux des Beaux-Arts de Belgique|pwv} mit \textsc{Rubens\pwindex{Rubens, Peter Paul 28.\,6.\,1577 Siegen – 30.\,5.\,1640 Antwerpen@\textsc{Rubens, Peter Paul} (28.\,6.\,1577 Siegen – 30.\,5.\,1640 Antwerpen), \emph{Maler}|pw}} und \textsc{Jordaens\pwindex{Jordaens, Jacob 19.\,5.\,1593 Antwerpen – 18.\,10.\,1678 ebd.@\textsc{Jordaens, Jacob} (19.\,5.\,1593 Antwerpen – 18.\,10.\,1678 ebd.), \emph{Maler}|pw}}, wie ich{ }ſie{ }ſo{ }ſchön noch nirgend geſehen und die mich gründlich v\substVorne{}\textsuperscript{\textcolor{gray}{or}}\substDazwischen{}om\substHinten{} »Modernen« kurirt haben,{ }ſo daß ich allmälig anfange, mir die Gegenwart
               abzugewöhnen. Kurzum: Eindrücke genug; aber doch der ewig wiederkehrende Grundton,
               der in Alles hineinſummt:{ }ſremd, fremd und fremd! Ach, mein liebes Wien\oindex{Wien@\textbf{Wien}, \emph{Verwaltungsgebiet}|pw}! {\dotsfive}\pend
           
\pstart
           Und zu thun habe ich! Du{ }ſelbſt wirſt zwar kaum meine Arbeiten verfolgen können, was
               ich im Übrigen ganz begreiflich finde. Soviel ich mich erinnere, haſt Du nie eine
               beſondere Vorliebe für belgiſche\oindex{Belgien@\textbf{Belgien}|pw} Politik
               beſeſſen. Und was die Feuilletons anlangt, die ich{ }ſchreibe, die{ }ſollſt Du erſt nicht
               leſen, weil{ }ſie eh’ nichts taugen. Aber immerhin, es gibt gewaltige Arbeit. Allein
               die Lectüre der 14 freitäglich erſcheinenden großen Blätter nimmt mir vier bis fünf
               Stunden pro Tag. Aber die Arbeit iſt gut – Du weißt ja, nicht? – und jetzt beſonders,
               denn{ }ſie richtet{ }ſich als eine \label{K_L02664-9v}\edtext{ſpaniſche\oindex{Spanien@\textbf{Spanien}|pw} Wand}{\lemma{\textnormal{\emph{spanische Wand}}}\Cendnote{\textnormal{bewegliche Wand zur Raumtrennung}}}\label{K_L02664-9} auf, die mir das {\pb}ewig unzufriedene, traurige und hoffnungsloſe
               Geſicht eines eigenen Selbſt verbirgt {\dots} Fürchterliche
               Schwierigkeiten macht mir die Sprache. Seit ich hier bin, habe ich nicht eine Sylbe
               zugelernt. Und wenn man in der Regel{ }ſagt, man{ }ſolle in ein fremdes Land gehen, um
               die fremde Sprache zu lernen,{ }ſo{ }ſage ich dementgegen aus eigener Erfahrung, daß der
               Aufenthalt im fremden Land nur dazu nütze iſt, Einen von Woche zu Woche mehr zu
               überzeugen, daß man von der fremden Sprache keinen Dunſt hat und nie einen bekommen
                  wird{\dotsfour}\pend
           
\pstart
           Ja richtig, der \label{K_L02664-10v}\edtext{Koffer}{\lemma{\textnormal{\emph{Koffer}}}\Cendnote{\textnormal{Goldmann\pwindex{Goldmann, Paul 31.\,1.\,1865 Breslau – 25.\,9.\,1935 Wien@\textsc{Goldmann, Paul} (31.\,1.\,1865 Breslau – 25.\,9.\,1935 Wien), \emph{Schriftsteller, Journalist}|pwk} dürfte bei Schnitzler für die Reise nach Frankfurt\oindex{Frankfurt am Main@\textbf{Frankfurt am Main}, \emph{Hauptstadt}|pwk} einen Koffer ausgeliehen haben.}}}\label{K_L02664-10}! Damit iſt
               es mir gut gegangen. Ich laſſe ihn in Frankfurt\oindex{Frankfurt am Main@\textbf{Frankfurt am Main}, \emph{Hauptstadt}|pw}
               und bitte meine Mutter\pwindex{Goldmann, Clementine 15.\,5.\,1842 Breslau – 24.\,2.\,1924 Frankfurt am Main@\textsc{Goldmann, Clementine} (15.\,5.\,1842 Breslau – 24.\,2.\,1924 Frankfurt am Main)|pwv}, ihn
               Dir zu überſenden. Meine Mutter\pwindex{Goldmann, Clementine 15.\,5.\,1842 Breslau – 24.\,2.\,1924 Frankfurt am Main@\textsc{Goldmann, Clementine} (15.\,5.\,1842 Breslau – 24.\,2.\,1924 Frankfurt am Main)|pwv}, die in’s Land geht, vergißt im Eifer der Reiſe. Und mein Onkel\pwindex{Mamroth, Fedor 21.\,2.\,1851 Breslau – 25.\,6.\,1907 Frankfurt am Main@\textsc{Mamroth, Fedor} (21.\,2.\,1851 Breslau – 25.\,6.\,1907 Frankfurt am Main), \emph{Journalist, Kritiker}|pwv}{ }ſchreibt mir dieſer
               Tage: er habe mir den Koffer, den ich in Frankfurt\oindex{Frankfurt am Main@\textbf{Frankfurt am Main}, \emph{Hauptstadt}|pw} gelaſſen, nach Brüſſel\oindex{Brüssel@\textbf{Brüssel}, \emph{Hauptstadt}|pw}
               nachgeſchickt. Ich muß alſo wohl oder übel warten {\pb}bis der Koffer hier ankommt, und dann werde ich den Vielgereiſten{ }ſofort nach Wien\oindex{Wien@\textbf{Wien}, \emph{Verwaltungsgebiet}|pw}{ }ſpediren. Sei mir nicht böſe, bitte, deswegen!
               Haſt Du irgend einen Wunſch, bezüglich irgend eines Gegenſtandes, den man bei dieſer
               Gelegenheit in Brüſſel\oindex{Brüssel@\textbf{Brüssel}, \emph{Hauptstadt}|pw} erwerben und mitſchicken
               könnte? Litteratur, Kunſt, Muſik, Crawatten, Eßwaren oder{ }ſo etwas? Bitte, denke
               nach. Mir iſt leid darum, den Koffer leer zu expediren{\dotsfour}\pend
           
\pstart
           Und nun bekomme ich wohl einen recht langen Brief? Befinden, Arbeiten, Verkehr,
               Stimmung, Sommerpläne, Tages- und Abendeintheilung \textsc{etc}. Ich
               bin heißhungrig nach jedem Biſſen Neuigkeit von Dir, von Wien\oindex{Wien@\textbf{Wien}, \emph{Verwaltungsgebiet}|pw} und den anderen Freunden. »\textsc{Es\pwindex{Glümer, Marie 3.\,7.\,1867 Wien – 16.\,11.\,1925 München@\textsc{Glümer, Marie} (3.\,7.\,1867 Wien – 16.\,11.\,1925 München), \emph{Schauspielerin}|pwv}}« iſt in Brünn\oindex{Brünn@\textbf{Brünn}|pw}? Und \textsc{Madame Olga\pwindex{Waissnix, Olga 3.\,11.\,1862 Wien – 4.\,11.\,1897 ebd.@\textsc{Waissnix, Olga} (3.\,11.\,1862 Wien – 4.\,11.\,1897 ebd.), \emph{Hotelière}|pw}}? Ich kann Dir{ }ſagen, die echten \textsc{\begin{otherlanguage}{french}Mondainen\end{otherlanguage}}, die man hier sieht,{ }ſehen doch noch ganz anders aus{\dots} Bitte grüße vielma{[}l{]}s \textsc{Kapper\pwindex{Kapper, Friedrich 21.\,4.\,1861 Wien – 22.\,7.\,1939 ebd.@\textsc{Kapper, Friedrich} (21.\,4.\,1861 Wien – 22.\,7.\,1939 ebd.), \emph{Mediziner}|pw}}, \textsc{Beer-Hofmann\pwindex{Beer-Hofmann, Richard 11.\,7.\,1866 Wien – 26.\,9.\,1945 New York City@\textsc{Beer-Hofmann, Richard} (11.\,7.\,1866 Wien – 26.\,9.\,1945 New York City), \emph{Schriftsteller}|pw}} und \textsc{Loris\pwindex{Hofmannsthal, Hugo von 1.\,2.\,1874 Wien – 15.\,7.\,1929 Rodaun@\textsc{Hofmannsthal, Hugo von} (1.\,2.\,1874 Wien – 15.\,7.\,1929 Rodaun), \emph{Schriftsteller}|pw}}. Und{ }ſei Du{ }ſelbſt gegrüßt, von Herzen und in Treue!\pend
           
\pstart
           Dein {\\[\baselineskip]}\spacefill\mbox{Paul Goldmann.}\pend
           \leftskip=0em{}
\pstart
           \noindent{}\uline{Adreſſe umſtehend:}\pend
           
\pstart
           {\pb}\textsc{Brüssel – St. Josse ten Noode, 21. rue des
                        plantes\oindex{rue des Plantes@\textbf{rue des Plantes}, \emph{Straße}|pw}}.\pend
           
\pstart
           Meine ergebenen Empfehlungen an die Deinen!\pend
           \selectlanguage{ngerman}\endnumbering\briefempfaengerindex{Schnitzler, Arthur@\textsc{Schnitzler, Arthur}!zzzGoldmann, Paul@\emph{von Paul Goldmann}!1891-06-091@{9. 6. [1891]}|)be}\mylabel{L02664h}  \newcommand{\dateiname}{L02664}\newcommand{\titel}{Paul Goldmann an Arthur Schnitzler, 9. 6. [1891]}\newcommand{\editorInnen}{Martin Anton Müller und Laura Untner}%% latex-leseansicht-abspann.tex
%% Abspann für die Leseansicht.
%% Der Schalter \ifkorrekturansicht ist bereits durch den Vorspann gesetzt.

%% latex-abspann.tex
%% Gemeinsamer Abspann für Korrekturansicht und Leseansicht.
%% Setzt den Schalter \ifkorrekturansicht voraus (gesetzt in den
%% einbindenden Dateien latex-korrekturansicht-abspann.tex bzw.
%% latex-leseansicht-abspann.tex).
%% ---------------------------------------------------------------

\normalsize

% Das esempio-Environment wird nur in der Leseansicht benötigt
\ifkorrekturansicht\else
\newenvironment{esempio}[3]%
{
    \vspace{1.5ex}
    \rlap{\underline{#1}}
    \par
    \setlength{\parindent}{0cm}
    \nopagebreak
    \leftskip=#2cm
    \rightskip=#3cm
}
{
    \par
}
\fi

\doendnotes{C}
\bigskip
\vfill

\clearpage

\footnotesize

\ifkorrekturansicht
  \lohead{\textsc{register}}
\fi

% theindex-Environment neu definieren ohne reledmac
\makeatletter
\renewenvironment{theindex}{%
  \ifkorrekturansicht
    \section*{\indexname}%
  \else
    \subsubsection*{Index der erwähnten Entitäten}%
  \fi
  \setlength{\parindent}{0pt}%
  \setlength{\parskip}{0pt plus 0.3pt}%
  \let\item\@idxitem
}{%
  \ifkorrekturansicht\clearpage\fi
}
\makeatother

\IfFileExists{\jobname-pw.ind}{\input{\jobname-pw.ind}}{}

% Quellenangabe nur in der Leseansicht
\ifkorrekturansicht\else
% Fallback-Definitionen, falls die .tex-Datei \titel etc. nicht gesetzt hat
\providecommand{\titel}{}
\providecommand{\editorInnen}{}
\providecommand{\dateiname}{\jobname}

\vspace{3cm}

\vfill

\footnotesize
\textsc{Quelle}: \titel. Herausgegeben von {\editorInnen}. In: \emph{Arthur Schnitzler: Briefwechsel mit Autorinnen und Autoren}.
 Digitale Edition, https://schnitzler-briefe.acdh.oeaw.ac.at/{\dateiname}.html (Stand \today)
\fi

\end{document}


