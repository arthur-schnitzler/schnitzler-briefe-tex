%% latex-leseansicht-vorspann.tex
%% Vorspann für die Leseansicht.
%% Lädt die gemeinsame Datei latex-vorspann.tex mit nicht gesetztem Schalter.

\newif\ifkorrekturansicht
\korrekturansichtfalse

\input{../tex-inputs/latex-vorspann}


         
         \newcommand{\erwaehntePersonen}{Personen:  ?? [Leiter einer Glasfabrik],  ?? [Schwester eines Glasfabrikanten], Richard Beer-Hofmann, Marie Glümer, Clementine Goldmann, Hugo von Hofmannsthal, Jacob Jordaens, Friedrich Kapper, Fedor Mamroth, Peter Paul Rubens, Charles Tardieu, Boris Van-Jung, Richard Wagner, Olga Waissnix}
         \newcommand{\erwaehnteInstitutionen}{Institutionen: ?? [Glasfabrik in Belgien], Frankfurter Zeitung, L’Indépendance Belge, Musées royaux des Beaux-Arts de Belgique}
         \newcommand{\erwaehnteOrte}{Orte: Belgien, Brünn, Brüssel, Frankfurt am Main, Niederlande, Spanien, Wien, rue des Plantes}
         \newcommand{\erwaehnteWerke}{
               \section[Paul Goldmann an Arthur Schnitzler, 9. 6. {[}1891{]}]{ Paul Goldmann an Arthur Schnitzler, 9. 6. {[}1891{]}}\nopagebreak\mylabel{v}\rehead{ }\begin{ledgroupsized}[t]{13cm}\normalsize\beginnumbering \toendnotes[C]{\smallbreak\pagebreak[2]} \Standort{DLA, A:Schnitzler, HS.NZ85.1.3162.}
\physDesc{Brief, 3 Blätter, 7 Seiten
\newline{}Handschrift: schwarze Tinte, deutsche Kurrent
\newline{}Schnitzler: mit rotem Buntstift eine Unterstreichung }\toendnotes[C]{\smallbreak}\pstart
           \raggedleft{}{\pb}\textsc{Brüssel\oindex{Bruessel@\textbf{Brüssel}|pw}}, 9. Juni.\pend
           \pstart\center{}Mein lieber Arthur!\pend\pstart
           Das iſt der Unterſchied zwiſchen Freundſchafts- und Liebescorreſpondenz: die Liebe
               will Gefühle, und die Freundſchaft wird durch Gefühle auf die Dauer gelangweilt und
               will Thatſachen. Dieſe Wochen, in denen ich Dir nicht geſchrieben, ſollten alſo eine
               kleine Thatſachen-Sparbüchſe ſein; und jetzt, wo ich meine Erſparniſſe in dieſer
               Beziehung nachſehe, finde ich nichts und kann Dir wieder nichts bieten als ein Paar
               ſchäbige Stimmungen und Empfindungen. Der Grund für den Thatſachenmangel iſt vor
               Allem der, daß ich die Hautpzeit des Tages allein auf meinem Zimmer und mit meiner
               Arbeit verbringe. Meine Empfehlungen habe ich wohl abgegeben, aber ſie haben zu
               nichts geführt; ausgeſuchte Höflichkeit überall; aber die Höflichkeit iſt ein gar
               matter Wärmeſpender; ſie erwärmt nicht mehr als ein flüchtiger Händedruck, und das
               Herz kann dabei erfrieren. Da und dort hat man mich zum Diner eingeladen, und war
               froh, als der eigenthümliche Geiſt, dem man Alles Zweimal ſagen mußte, um von ihm
               verſtanden zu werden, und der ſelbſt ein jämmerliches Stottern vorführte, die {\pb}Thür hinter ſich zumachte. Ein klein wenig näher –
               aber auch nichts weniger als intim – verkehr ich mit einem jungen \label{K_L02664-5v}\edtext{Manne\pwindex{?? [Leiter einer Glasfabrik] *~um 1869@\textsc{?? [Leiter einer Glasfabrik]} (*~um 1869)|pwv}}{\lemma{\textnormal{\emph{Manne}}}\Cendnote{\textnormal{nicht identifiziert}}}\label{K_L02664-5h} (22 Jahre),
                  Erbe\pwindex{?? [Leiter einer Glasfabrik] *~um 1869@\textsc{?? [Leiter einer Glasfabrik]} (*~um 1869)|pwv} und Leiter\pwindex{?? [Leiter einer Glasfabrik] *~um 1869@\textsc{?? [Leiter einer Glasfabrik]} (*~um 1869)|pwv} einer großen \label{K_L02664-6v}\edtext{Glasfabrik\orgindex{?? [Glasfabrik in Belgien]@?? [Glasfabrik in Belgien]|pwv}}{\lemma{\textnormal{\emph{Glasfabrik}}}\Cendnote{\textnormal{nicht ermittelt}}}\label{K_L02664-6h}; demgemäß ein
               wenig ſtolz und \label{K_L02664-1v}\edtext{\textsc{\begin{otherlanguage}{french}hautain\end{otherlanguage}}}{\lemma{\textnormal{\emph{hautain}}}\Cendnote{\textnormal{französisch: hochmütig, unnahbar}}}\label{K_L02664-1h},
               aber wohlerzogen genug, um das dem ihm warm empfohlenen Fremden nicht zu zeigen. Im
               Allgemeinen ein ſehr hübſcher, \strikeout{\textcolor{gray}{aſ}} äſthetiſch angenehmer Menſch\pwindex{?? [Leiter einer Glasfabrik] *~um 1869@\textsc{?? [Leiter einer Glasfabrik]} (*~um 1869)|pwv} – eine Art \textsc{Boris Fanjung\pwindex{Van-Jung, Boris 15.10.1872 – 03.10.1899@\textsc{Van-Jung, Boris} (15.10.1872 – 03.10.1899), \emph{Mediziner}|pw}}, nur viel feiner und hochſtehender. Ein wenig Kunſtdilettant\pwindex{?? [Leiter einer Glasfabrik] *~um 1869@\textsc{?? [Leiter einer Glasfabrik]} (*~um 1869)|pwv} und reizend, wenn er ſeine
               naiven Pläne entwickelt »\label{K_L02664-2v}\edtext{\textsc{\begin{otherlanguage}{french}de joindre l’art à l’industrie\end{otherlanguage}}}{\lemma{\textnormal{\emph{de … l’industrie}}}\Cendnote{\textnormal{französisch: die Kunst mit der Industrie
                  zu verbinden}}}\label{K_L02664-2h}«. Vor Allem aber – ſtrenggläubiger Katholik\pwindex{?? [Leiter einer Glasfabrik] *~um 1869@\textsc{?? [Leiter einer Glasfabrik]} (*~um 1869)|pwv}, der allſonntäglich zur Meſſe geht
               und ſich auf nichts in der Welt mehr freut, als auf ſein Fortleben nach dem Tode.
               Dazu eine blonde, äußerlich unbedeutende, ſehr fromme, \strikeout{und} ſehr ſanfte und ſehr kurzſichtige \label{K_L02664-7v}\edtext{Schweſter\pwindex{?? [Schwester eines Glasfabrikanten] @\textsc{?? [Schwester eines Glasfabrikanten]}|pwv}}{\lemma{\textnormal{\emph{Schweſter}}}\Cendnote{\textnormal{nicht identifiziert}}}\label{K_L02664-7h} mit einem
               ewigen \label{K_L02664-3v}\edtext{Lorgnon}{\lemma{\textnormal{\emph{Lorgnon}}}\Cendnote{\textnormal{Brille mit Haltestiel}}}\label{K_L02664-3h} und mit Redensarten wie »\label{K_L02664-4v}\edtext{\textsc{\begin{otherlanguage}{french}Jésus es mon ami intime\end{otherlanguage}}}{\lemma{\textnormal{\emph{Jésus es mon ami intime}}}\Cendnote{\textnormal{französisch: Jesus ist mein enger
                  Freund}}}\label{K_L02664-4h}«. Fürſtlicher Haushalt, nicht ohne Stimmung das Ganze – aber doch
               ohne rechte Wärme{\dots} Außerdem iſt da in Brüſſel\oindex{Bruessel@\textbf{Brüssel}|pw} der Chefredacteur\pwindex{Tardieu, Charles 1838-02-09 – 1909@\textsc{Tardieu, Charles} (1838-02-09 – 1909), \emph{Journalist, Chefredakteur}|pwv} der »\textsc{Indépendance belge\orgindex{Independance Belge@L’Indépendance Belge|pw}}« (Geograph wie Du biſt, wirſt Du fragen, {\pb}wieſo Brüſſel\oindex{Bruessel@\textbf{Brüssel}|pw} zu Belgien\oindex{Belgien@\textbf{Belgien}|pw} kommt, aber ich kann Dir verrathen, daß es die Hauptſtadt\oindex{Bruessel@\textbf{Brüssel}|pwv} davon iſt). Dieſer also, \textsc{M. Tardieu\pwindex{Tardieu, Charles 1838-02-09 – 1909@\textsc{Tardieu, Charles} (1838-02-09 – 1909), \emph{Journalist, Chefredakteur}|pw}}, iſt ein durchaus charmanter Menſch\pwindex{Tardieu, Charles 1838-02-09 – 1909@\textsc{Tardieu, Charles} (1838-02-09 – 1909), \emph{Journalist, Chefredakteur}|pwv}, der einzige echte Franzoſe\pwindex{Tardieu, Charles 1838-02-09 – 1909@\textsc{Tardieu, Charles} (1838-02-09 – 1909), \emph{Journalist, Chefredakteur}|pwv}, den ich bisher kennen gelernt, Cavalier\pwindex{Tardieu, Charles 1838-02-09 – 1909@\textsc{Tardieu, Charles} (1838-02-09 – 1909), \emph{Journalist, Chefredakteur}|pwv}, unermüdlicher und geiſtſprühender
                  Plauderer\pwindex{Tardieu, Charles 1838-02-09 – 1909@\textsc{Tardieu, Charles} (1838-02-09 – 1909), \emph{Journalist, Chefredakteur}|pwv} und profunder
                  Kunſtkenner\pwindex{Tardieu, Charles 1838-02-09 – 1909@\textsc{Tardieu, Charles} (1838-02-09 – 1909), \emph{Journalist, Chefredakteur}|pwv}, Specialist\pwindex{Tardieu, Charles 1838-02-09 – 1909@\textsc{Tardieu, Charles} (1838-02-09 – 1909), \emph{Journalist, Chefredakteur}|pwv} für niederländiſche\oindex{Niederlande@\textbf{Niederlande}|pw} Malerei und \label{K_L02664-10v}\edtext{enragirter}{\lemma{\textnormal{\emph{enragirter}}}\Cendnote{\textnormal{begeisteter}}}\label{K_L02664-10h}Wagner\pwindex{Wagner, Richard 22.05.1813 – 13.02.1883@\textsc{Wagner, Richard} (22.05.1813 – 13.02.1883), \emph{Komponist}|pw}ianer. Der Chefredacteur\pwindex{Tardieu, Charles 1838-02-09 – 1909@\textsc{Tardieu, Charles} (1838-02-09 – 1909), \emph{Journalist, Chefredakteur}|pwv} der »\textsc{Indépendance\orgindex{Independance Belge@L’Indépendance Belge|pw}}« iſt natürlich in Brüſſel\oindex{Bruessel@\textbf{Brüssel}|pw} ein großer Mann\pwindex{Tardieu, Charles 1838-02-09 – 1909@\textsc{Tardieu, Charles} (1838-02-09 – 1909), \emph{Journalist, Chefredakteur}|pwv} – wenn \introOben{}er\introOben{} auch von dem Größenwahn der Wien\oindex{Wien@\textbf{Wien}|pw}er Zeitungsſ\textcolor{gray}{a}ujuden keine Spur beſitzt – und hat
               Beſſeres zu thun, als mit dem Correſpondenten der »Frankfurter Zeitung\orgindex{Frankfurter Zeitung@Frankfurter Zeitung|pw}« zu verkehren; aber alle 8 Tage ergibt ſich doch eine
               Plauder-Viertelſtunde auf ſeiner Redaction\orgindex{Independance Belge@L’Indépendance Belge|pwv}sſtube, die ich dann immer höchlich angeregt verlaſſe. Und dann iſt
                  Brüſſel\oindex{Bruessel@\textbf{Brüssel}|pw} ſelbſt – elegante und ſympathiſche Stadt\oindex{Bruessel@\textbf{Brüssel}|pwv}. Schöne Leute. Und vor
               Allem eine große hiſtoriſche Vergangenheit – die gewiſſe gothiſche Bettdecke, die man
               ſich über die \strikeout{Ohr} Ohren zieht, wenn man von der
               Gegenwart nichts wiſſen will. {\pb}Viel Kunst –
               herrliche alte und elende neue: Ein Muſeum\orgindex{Musees royaux des Beaux-Arts de Belgique@Musées royaux des Beaux-Arts de Belgique|pwv} mit \textsc{Rubens\pwindex{Rubens, Peter Paul 1577-06-28 – 1640-05-30@\textsc{Rubens, Peter Paul} (1577-06-28 – 1640-05-30), \emph{Maler}|pw}} und \textsc{Jordaens\pwindex{Jordaens, Jacob 1593-05-19 – 1678-10-18@\textsc{Jordaens, Jacob} (1593-05-19 – 1678-10-18), \emph{Maler}|pw}}, wie ich ſie ſo ſchön noch nirgend geſehen und die mich gründlich v\substVorne{}\textsuperscript{\textcolor{gray}{or}}\substDazwischen{}om\substHinten{} »Modernen« kurirt haben, ſo daß ich allmälig anfange, mir die Gegenwart
               abzugewöhnen. Kurzum: Eindrücke genug; aber doch der ewig wiederkehrende Grundton,
               der in Alles hineinſummt: ſremd, fremd und fremd! Ach, mein liebes Wien\oindex{Wien@\textbf{Wien}|pw}! {\dotsfive}\pend
           \pstart
           Und zu thun habe ich! Du ſelbſt wirſt zwar kaum meine Arbeiten verfolgen können, was
               ich im Übrigen ganz begreiflich finde. Soviel ich mich erinnere, haſt Du nie eine
               beſondere Vorliebe für belgiſche\oindex{Belgien@\textbf{Belgien}|pw} Politik
               beſeſſen. Und was die Feuilletons anlangt, die ich ſchreibe, die ſollſt Du erſt nicht
               leſen, weil ſie eh’ nichts taugen. Aber immerhin, es gibt gewaltige Arbeit. Allein
               die Lectüre der 14 freitäglich erſcheinenden großen Blätter nimmt mir vier bis fünf
               Stunden pro Tag. Aber die Arbeit iſt gut – Du weißt ja, nicht? – und jetzt beſonders,
               denn ſie richtet ſich als eine \label{K_L02664-8v}\edtext{ſpaniſche\oindex{Spanien@\textbf{Spanien}|pw} Wand}{\lemma{\textnormal{\emph{ſpaniſche Wand}}}\Cendnote{\textnormal{bewegliche Wand zur Raumtrennung}}}\label{K_L02664-8h} auf, die mir das {\pb}ewig unzufriedene, traurige und hoffnungsloſe
               Geſicht eines eigenen Selbſt verbirgt {\dots} Fürchterliche
               Schwierigkeiten macht mir die Sprache. Seit ich hier bin, habe ich nicht eine Sylbe
               zugelernt. Und wenn man in der Regel ſagt, man ſolle in ein fremdes Land gehen, um
               die fremde Sprache zu lernen, ſo ſage ich dementgegen aus eigener Erfahrung, daß der
               Aufenthalt im fremden Land nur dazu nütze iſt, Einen von Woche zu Woche mehr zu
               überzeugen, daß man von der fremden Sprache keinen Dunſt hat und nie einen bekommen
                  wird{\dotsfour}\pend
           \pstart
           Ja richtig, der \label{K_L02664-9v}\edtext{Koffer}{\lemma{\textnormal{\emph{Koffer}}}\Cendnote{\textnormal{Goldmann\pwindex{Goldmann, Paul 31.01.1865 – 25.09.1935@\textsc{Goldmann, Paul} (31.01.1865 – 25.09.1935), \emph{Schriftsteller, Journalist}|pwk} dürfte bei Schnitzler\pwindex{Schnitzler, Arthur 15.05.1862 – 21.10.1931@\textsc{Schnitzler, Arthur} (15.05.1862 – 21.10.1931), \emph{Schriftsteller, Mediziner}|pwk} für die Reise nach Frankfurt\oindex{Frankfurt am Main@\textbf{Frankfurt am Main}|pwk} einen Koffer ausgeliehen haben.}}}\label{K_L02664-9h}! Damit iſt
               es mir gut gegangen. Ich laſſe ihn in Frankfurt\oindex{Frankfurt am Main@\textbf{Frankfurt am Main}|pw}
               und bitte meine Mutter\pwindex{Goldmann, Clementine 1842-05-15 – 1924-02-24@\textsc{Goldmann, Clementine} (1842-05-15 – 1924-02-24)|pwv}, ihn
               Dir zu überſenden. Meine Mutter\pwindex{Goldmann, Clementine 1842-05-15 – 1924-02-24@\textsc{Goldmann, Clementine} (1842-05-15 – 1924-02-24)|pwv}, die in’s Land geht, vergißt im Eifer der Reiſe. Und mein Onkel\pwindex{Mamroth, Fedor 21.02.1851 – 25.06.1907@\textsc{Mamroth, Fedor} (21.02.1851 – 25.06.1907), \emph{Journalist, Kritiker}|pwv} ſchreibt mir dieſer
               Tage: er habe mir den Koffer, den ich in Frankfurt\oindex{Frankfurt am Main@\textbf{Frankfurt am Main}|pw} gelaſſen, nach Brüſſel\oindex{Bruessel@\textbf{Brüssel}|pw}
               nachgeſchickt. Ich muß alſo wohl oder übel warten {\pb}bis der Koffer hier ankommt, und dann werde ich den Vielgereiſten ſofort nach Wien\oindex{Wien@\textbf{Wien}|pw} ſpediren. Sei mir nicht böſe, bitte, deswegen!
               Haſt Du irgend einen Wunſch, bezüglich irgend eines Gegenſtandes, den man bei dieſer
               Gelegenheit in Brüſſel\oindex{Bruessel@\textbf{Brüssel}|pw} erwerben und mitſchicken
               könnte? Litteratur, Kunſt, Muſik, Crawatten, Eßwaren oder ſo etwas? Bitte, denke
               nach. Mir iſt leid darum, den Koffer leer zu expediren{\dotsfour}\pend
           \pstart
           Und nun bekomme ich wohl einen recht langen Brief? Befinden, Arbeiten, Verkehr,
               Stimmung, Sommerpläne, Tages- und Abendeintheilung \textsc{etc}. Ich
               bin heißhungrig nach jedem Biſſen Neuigkeit von Dir, von Wien\oindex{Wien@\textbf{Wien}|pw} und den anderen Freunden. »\textsc{Es\pwindex{Gluemer, Marie 03.07.1867 – 16.11.1925@\textsc{Glümer, Marie} (03.07.1867 – 16.11.1925), \emph{Schauspielerin}|pwv}}« iſt in Brünn\oindex{Bruenn@\textbf{Brünn}|pw}? Und \textsc{Madame Olga\pwindex{Waissnix, Olga 03.11.1862 – 04.11.1897@\textsc{Waissnix, Olga} (03.11.1862 – 04.11.1897), \emph{Hotelière}|pw}}? Ich kann Dir ſagen, die echten \textsc{\begin{otherlanguage}{french}Mondainen\end{otherlanguage}}, die man hier sieht, ſehen doch noch ganz anders aus{\dots} Bitte grüße vielma{[}l{]}s \textsc{Kapper\pwindex{Kapper, Friedrich 21.04.1861 – 22.07.1939@\textsc{Kapper, Friedrich} (21.04.1861 – 22.07.1939), \emph{Mediziner}|pw}}, \textsc{Beer-Hofmann\pwindex{Beer-Hofmann, Richard 1866-07-11 – 1945-09-26@\textsc{Beer-Hofmann, Richard} (1866-07-11 – 1945-09-26), \emph{Schriftsteller}|pw}} und \textsc{Loris\pwindex{Hofmannsthal, Hugo von 1874-02-01 – 1929-07-15@\textsc{Hofmannsthal, Hugo von} (1874-02-01 – 1929-07-15), \emph{Schriftsteller}|pw}}. Und ſei Du ſelbſt gegrüßt, von Herzen und in Treue!\pend
           \pstart
           Dein {\\[\baselineskip]}\spacefill\mbox{Paul Goldmann.}\pend
           \leftskip=0em{}\pstart
           \noindent{}\uline{Adreſſe umſtehend:}\pend
           \pstart
           {\pb}\textsc{Brüssel – St. Josse ten Noode, 21. rue des
                        plantes\oindex{rue des Plantes@\textbf{rue des Plantes}|pw}}.\pend
           \pstart
           Meine ergebenen Empfehlungen an die Deinen!\pend
           
         
         \endnumbering\mylabel{h}\end{ledgroupsized}  \newcommand{\dateiname}{L02664}\newcommand{\titel}{Paul Goldmann an Arthur Schnitzler, 9. 6. [1891]}\newcommand{\editorInnen}{Martin Anton Müller und Laura Untner}%% latex-leseansicht-abspann.tex
%% Abspann für die Leseansicht.
%% Der Schalter \ifkorrekturansicht ist bereits durch den Vorspann gesetzt.

%% latex-abspann.tex
%% Gemeinsamer Abspann für Korrekturansicht und Leseansicht.
%% Setzt den Schalter \ifkorrekturansicht voraus (gesetzt in den
%% einbindenden Dateien latex-korrekturansicht-abspann.tex bzw.
%% latex-leseansicht-abspann.tex).
%% ---------------------------------------------------------------

\normalsize

% Das esempio-Environment wird nur in der Leseansicht benötigt
\ifkorrekturansicht\else
\newenvironment{esempio}[3]%
{
    \vspace{1.5ex}
    \rlap{\underline{#1}}
    \par
    \setlength{\parindent}{0cm}
    \nopagebreak
    \leftskip=#2cm
    \rightskip=#3cm
}
{
    \par
}
\fi

\doendnotes{C}
\bigskip
\vfill

\clearpage

\footnotesize

\ifkorrekturansicht
  \lohead{\textsc{register}}
\fi

% theindex-Environment neu definieren ohne reledmac
\makeatletter
\renewenvironment{theindex}{%
  \ifkorrekturansicht
    \section*{\indexname}%
  \else
    \subsubsection*{Index der erwähnten Entitäten}%
  \fi
  \setlength{\parindent}{0pt}%
  \setlength{\parskip}{0pt plus 0.3pt}%
  \let\item\@idxitem
}{%
  \ifkorrekturansicht\clearpage\fi
}
\makeatother

\IfFileExists{\jobname-pw.ind}{\input{\jobname-pw.ind}}{}

% Quellenangabe nur in der Leseansicht
\ifkorrekturansicht\else
% Fallback-Definitionen, falls die .tex-Datei \titel etc. nicht gesetzt hat
\providecommand{\titel}{}
\providecommand{\editorInnen}{}
\providecommand{\dateiname}{\jobname}

\vspace{3cm}

\vfill

\footnotesize
\textsc{Quelle}: \titel. Herausgegeben von {\editorInnen}. In: \emph{Arthur Schnitzler: Briefwechsel mit Autorinnen und Autoren}.
 Digitale Edition, https://schnitzler-briefe.acdh.oeaw.ac.at/{\dateiname}.html (Stand \today)
\fi

\end{document}


      