%% latex-korrekturansicht-vorspann.tex
%% Vorspann für die Korrekturansicht.
%% Lädt die gemeinsame Datei latex-vorspann.tex mit gesetztem Schalter.

\newif\ifkorrekturansicht
\korrekturansichttrue

\input{../tex-inputs/latex-vorspann}


\section[Arthur Schnitzler an Marie Herzfeld, 7. 3. 1931]{L02598 Arthur Schnitzler an Marie Herzfeld, 7. 3. 1931}
\nopagebreak\mylabel{L02598v}
\rehead{ }\normalsize\beginnumbering\briefempfaengerindex{Herzfeld, Marie@\textsc{Herzfeld, Marie}!zzzSchnitzler, Arthur@\emph{von Arthur Schnitzler}!1931-03-071@{7. 3. 1931}|(be}
\toendnotes[C]{\smallbreak\pagebreak[2]}\Standort{DLA, A:Schnitzler, HS.1985.1.993.}
\physDesc{Brief, Durchschlag1 Blatt, 1 Seite, 1034 Zeichen
\newline{}Schreibmaschine
\newline{}Handschrift: roter Buntstift, lateinische Kurrent (\noindent{}mit rotem Buntstift Vermerk »\textsc{\uline{Herzfeld}}« und sieben Unterstreichungen)}\toendnotes[C]{\smallbreak}
\pstart
           \raggedleft{}{\pb}7. 3. 1931\pend
           
\pstart{}Verehrtes Fräulein.\pend\vspace{0.5em}
\pstart
           Dass es sich bei dem in Hofmannsthals\pwindex{Hofmannsthal, Hugo von 1874-02-01 – 1929-07-15@\textsc{Hofmannsthal, Hugo von} (1874-02-01 – 1929-07-15), \emph{Schriftsteller/Schriftstellerin}|pw}{ }\label{K_L02598-1v}\edtext{Brief vom 19. Juli 92}{\lemma{\textnormal{\emph{Brief vom 19. Juli 92}}}\Cendnote{\textnormal{Siehe Marie Herzfeld an Arthur Schnitzler, 5. 3. 1931.
               }}}\label{K_L02598-1} und am \label{K_L02598-2v}\edtext{4. August}{\lemma{\textnormal{\emph{4. August}}}\Cendnote{\textnormal{Siehe Hugo von Hofmannsthal an Arthur Schnitzler, 4. 8. [1892].
               }}}\label{K_L02598-2} erwähnten \label{K_L02598-3v}\edtext{Renaissancedrama\pwindex{Ascanio und Gioconda@\emph{Ascanio und Gioconda}|pwv}}{\lemma{\textnormal{\emph{Renaissancedrama}}}\Cendnote{\textnormal{Gemeint ist das zu Lebzeiten
                  unveröffentlicht gebliebene Drama \emph{Ascanio und
                     Gioconda}\pwindex{Ascanio und Gioconda@\emph{Ascanio und Gioconda}|pwk}.}}}\label{K_L02598-3} schon um die Vorarbeiten zum »Geretteten Venedig\pwindex{gerettete Venedig. Trauerspiel in fuenf Aufzuegen@\emph{Das gerettete Venedig. Trauerspiel in fünf Aufzügen}|pw}« handeln könnte, halte ich für durchaus
               unwahrscheinlich; Positives kann ich freilich nicht behaupten. Ich vermag mich auch
               nicht zu erinnern, dass Hofmannsthal\pwindex{Hofmannsthal, Hugo von 1874-02-01 – 1929-07-15@\textsc{Hofmannsthal, Hugo von} (1874-02-01 – 1929-07-15), \emph{Schriftsteller/Schriftstellerin}|pw} mir
               später von dieser fünfaktigen Renaissancetragödie\pwindex{Ascanio und Gioconda@\emph{Ascanio und Gioconda}|pwv}{ }\label{T_L02598-1v}\edtext{»dramatisierter
                  Novelle{[}«{]}}{\lemma{\textnormal{\emph{»dramatisierter
                  Novelle«}}}\Cendnote{\textnormal{Das eine Anführungszeichen ist mit
                  Schreibmaschine genau in den Leerraum zwischen
                  »Renaissancetragödie« und »dramatisierter« gesetzt,
                  sodass das Anführungszeichen alternativ auch das schließende der
                     »Renaissancetragödie« sein könnte.}}}\label{T_L02598-1}, äusserlich im Stil
               von »Romeo und Julie\pwindex{Romeo and Juliet@\emph{Romeo and Juliet}|pw}« später wieder gesprochen
               oder mir Verse daraus vorgelesen hätte. Immerhin wäre es denkbar, dass Stellen aus
               dem Entwurf in andere Werke von ihm übergegangen sind, vielleicht sogar ins »Gerettete Venedig\pwindex{gerettete Venedig. Trauerspiel in fuenf Aufzuegen@\emph{Das gerettete Venedig. Trauerspiel in fünf Aufzügen}|pw}«.\pend
           
\pstart
           Möglich auch, dass er mir seinerzeit mehr von jener Tragödie\pwindex{Ascanio und Gioconda@\emph{Ascanio und Gioconda}|pwv} erzählt oder mir manchmal auch daraus vorgelesen
               hätte; – das wäre ja bald 40 Jahre her und man hat ja leider mancherlei
               vergessen.\pend
           
\pstart
           Ich freue mich, nach so langer Zeit wieder einmal direkt von
                  {[}Ihnen{]} etwas gehört zu haben und bin mit herzlichen
               Grüssen\pend
           \pstart Ihr aufrichtig ergebener\pend{}{\vspace{1\baselineskip}}
\pstart
           \noindent{}Fräulein Marie Herzfeld,{\\}Wien III.\oindex{III., Landstrasse@\textbf{III., Landstraße}, \emph{A.ADM3}|pw}{\\}Oetzeltg. 1\oindex{Oelzeltgasse@\textbf{Ölzeltgasse}, \emph{Straße (K.STR)}|pw}.\pend
           \selectlanguage{ngerman}\endnumbering\briefempfaengerindex{Herzfeld, Marie@\textsc{Herzfeld, Marie}!zzzSchnitzler, Arthur@\emph{von Arthur Schnitzler}!1931-03-071@{7. 3. 1931}|)be}\mylabel{L02598h}  \normalsize

\doendnotes{C}
\bigskip
\vfill

\clearpage

\footnotesize

\lohead{\textsc{register}}

% Definiere theindex-Environment komplett neu ohne reledmac
\makeatletter
\renewenvironment{theindex}{%
  \section*{\indexname}%
  \setlength{\parindent}{0pt}%
  \setlength{\parskip}{0pt plus 0.3pt}%
  \let\item\@idxitem
}{%
  \clearpage
}
\makeatother

\IfFileExists{\jobname-pw.ind}{\input{\jobname-pw.ind}}{}

\end{document}

      