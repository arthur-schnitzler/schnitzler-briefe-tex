%% latex-leseansicht-vorspann.tex
%% Vorspann für die Leseansicht.
%% Lädt die gemeinsame Datei latex-vorspann.tex mit nicht gesetztem Schalter.

\newif\ifkorrekturansicht
\korrekturansichtfalse

\input{../tex-inputs/latex-vorspann}


               \section[Arthur Schnitzler an Marie Herzfeld, 7. 3. 1931]{ Arthur Schnitzler an Marie Herzfeld, 7. 3. 1931}\nopagebreak\mylabel{v}\rehead{ }\begin{ledgroupsized}[t]{13cm}\normalsize\beginnumbering\briefempfaengerindex{Herzfeld, Marie@\textsc{Herzfeld, Marie}!zzzSchnitzler, Arthur@\emph{von Arthur Schnitzler}!1931-03-071@{7. 3. 1931}|(be} \toendnotes[C]{\smallbreak\pagebreak[2]} \Standort{DLA, A:Schnitzler, HS.1985.1.993.}
\physDesc{Brief, 1 Blatt, 1 Seite, maschineller Durchschlag
\newline{}Schreibmaschine
\newline{}Handschrift: roter Buntstift, lateinische Kurrent (\noindent{}mit rotem Buntstift Vermerk »\textsc{\uline{Herzfeld}}« und
                                 sieben Unterstreichungen)}\toendnotes[C]{\smallbreak}\pstart
           \raggedleft{}{\pb}7. 3. 1931\pend
           \pstart{}Verehrtes Fräulein.\pend\pstart
           Dass es sich bei dem in Hofmannsthal\pwindex{Hofmannsthal, Hugo von 01.02.1874 – 15.07.1929@\textsc{Hofmannsthal, Hugo von} (01.02.1874 – 15.07.1929), \emph{Schriftsteller}|pw}s \label{K_L02598-1v}\edtext{Brief vom
                  19. Juli 92}{\lemma{\textnormal{\emph{Brief vom
                  19. Juli 92}}}\Cendnote{\textnormal{siehe Marie Herzfeld an Arthur Schnitzler, 5. 3. 1931}}}\label{K_L02598-1h} und am \label{K_L02598-2v}\edtext{4. August}{\lemma{\textnormal{\emph{4. August}}}\Cendnote{\textnormal{siehe Hugo von Hofmannsthal an Arthur Schnitzler, 4. 8. [1892]}}}\label{K_L02598-2h} erwähnten \label{K_L02598-3v}\edtext{Renaissancedrama\pwindex{Hofmannsthal, Hugo von 01.02.1874 – 15.07.1929@\textsc{Hofmannsthal, Hugo von} (01.02.1874 – 15.07.1929), \emph{Schriftsteller}!Ascanio und Gioconda1979@\strich\emph{Ascanio und Gioconda} {[}1979{]}|pwv}}{\lemma{\textnormal{\emph{Renaissancedrama}}}\Cendnote{\textnormal{Gemeint ist das zu Lebzeiten
                  unveröffentlicht gebliebene Drama \emph{Ascanio und
                  Gioconda}\pwindex{Hofmannsthal, Hugo von 01.02.1874 – 15.07.1929@\textsc{Hofmannsthal, Hugo von} (01.02.1874 – 15.07.1929), \emph{Schriftsteller}!Ascanio und Gioconda1979@\strich\emph{Ascanio und Gioconda} {[}1979{]}|pwk}.}}}\label{K_L02598-3h} schon um die Vorarbeiten zum »Geretteten Venedig\pwindex{Hofmannsthal, Hugo von 01.02.1874 – 15.07.1929@\textsc{Hofmannsthal, Hugo von} (01.02.1874 – 15.07.1929), \emph{Schriftsteller}!gerettete Venedig. Trauerspiel in fuenf Aufzuegen1905@\strich\emph{Das gerettete Venedig. Trauerspiel in fünf Aufzügen} {[}1905{]}|pw}« handeln könnte, halte ich für durchaus unwahrscheinlich;
               Positives kann ich freilich nicht behaupten. Ich vermag mich auch nicht zu erinnern,
               dass Hofmannsthal\pwindex{Hofmannsthal, Hugo von 01.02.1874 – 15.07.1929@\textsc{Hofmannsthal, Hugo von} (01.02.1874 – 15.07.1929), \emph{Schriftsteller}|pw} mir später von dieser
               fünfaktigen Renaissancetragödie\pwindex{Hofmannsthal, Hugo von 01.02.1874 – 15.07.1929@\textsc{Hofmannsthal, Hugo von} (01.02.1874 – 15.07.1929), \emph{Schriftsteller}!Ascanio und Gioconda1979@\strich\emph{Ascanio und Gioconda} {[}1979{]}|pwv}{ }\label{T_L02598-1v}\edtext{»dramatisierter
                  Novelle{[}«{]}}{\lemma{\textnormal{\emph{»dramatisierter
                  Novelle«}}}\Cendnote{\textnormal{Das eine
                  Anführungszeichen ist mit Schreibmaschine genau in den Leerraum zwischen
                     »Renaissancetragödie« und »dramatisierter«
                  gesetzt, so dass das Anführungszeichen alternativ auch das schließende der
                     »Renaissancetragödie« sein könnte.}}}\label{T_L02598-1h}, äusserlich im Stil
               von »Romeo und Julie\pwindex{\textcolor{red}{\textsuperscript{XXXX1 indx}}!Romeo und Julia1594@\strich\emph{Romeo und Julia} {[}1594{]}|pw}« später wieder gesprochen oder
               mir Verse daraus vorgelesen hätte. Immerhin wäre es denkbar, dass Stellen aus dem
               Entwurf in andere Werke von ihm übergegangen sind, vielleicht sogar ins »Gerettete Venedig\pwindex{Hofmannsthal, Hugo von 01.02.1874 – 15.07.1929@\textsc{Hofmannsthal, Hugo von} (01.02.1874 – 15.07.1929), \emph{Schriftsteller}!gerettete Venedig. Trauerspiel in fuenf Aufzuegen1905@\strich\emph{Das gerettete Venedig. Trauerspiel in fünf Aufzügen} {[}1905{]}|pw}«.\pend
           \pstart
           Möglich auch, dass er mir seinerzeit mehr von jener Tragödie\pwindex{Hofmannsthal, Hugo von 01.02.1874 – 15.07.1929@\textsc{Hofmannsthal, Hugo von} (01.02.1874 – 15.07.1929), \emph{Schriftsteller}!Ascanio und Gioconda1979@\strich\emph{Ascanio und Gioconda} {[}1979{]}|pwv} erzählt oder mir manchmal auch daraus vorgelesen
               hätte; – das wäre ja bald 40 Jahre her und man hat ja leider mancherlei
               vergessen.\pend
           \pstart
           Ich freue mich, nach so langer Zeit wieder einmal direkt von
                  {[}Ihnen{]} etwas gehört zu haben und bin mit herzlichen
               Grüssen\pend
           \pstart Ihr aufrichtig ergebener\pend{}{\bigskip}\pstart
           \noindent{}Fräulein Marie Herzfeld,{\\}Wien III.\oindex{III., Landstrasse@\textbf{III., Landstraße}|pw}{\\}Oetzeltg. 1\oindex{Oelzeltgasse@\textbf{Ölzeltgasse}|pw}.\pend
           \endnumbering\briefempfaengerindex{Herzfeld, Marie@\textsc{Herzfeld, Marie}!zzzSchnitzler, Arthur@\emph{von Arthur Schnitzler}!1931-03-071@{7. 3. 1931}|)be}\mylabel{h}\end{ledgroupsized}  \newcommand{\dateiname}{L02598}\newcommand{\titel}{Arthur Schnitzler an Marie Herzfeld, 7. 3. 1931}\newcommand{\editorInnen}{Martin Anton Müller und Laura Untner}%% latex-leseansicht-abspann.tex
%% Abspann für die Leseansicht.
%% Der Schalter \ifkorrekturansicht ist bereits durch den Vorspann gesetzt.

%% latex-abspann.tex
%% Gemeinsamer Abspann für Korrekturansicht und Leseansicht.
%% Setzt den Schalter \ifkorrekturansicht voraus (gesetzt in den
%% einbindenden Dateien latex-korrekturansicht-abspann.tex bzw.
%% latex-leseansicht-abspann.tex).
%% ---------------------------------------------------------------

\normalsize

% Das esempio-Environment wird nur in der Leseansicht benötigt
\ifkorrekturansicht\else
\newenvironment{esempio}[3]%
{
    \vspace{1.5ex}
    \rlap{\underline{#1}}
    \par
    \setlength{\parindent}{0cm}
    \nopagebreak
    \leftskip=#2cm
    \rightskip=#3cm
}
{
    \par
}
\fi

\doendnotes{C}
\bigskip
\vfill

\clearpage

\footnotesize

\ifkorrekturansicht
  \lohead{\textsc{register}}
\fi

% theindex-Environment neu definieren ohne reledmac
\makeatletter
\renewenvironment{theindex}{%
  \ifkorrekturansicht
    \section*{\indexname}%
  \else
    \subsubsection*{Index der erwähnten Entitäten}%
  \fi
  \setlength{\parindent}{0pt}%
  \setlength{\parskip}{0pt plus 0.3pt}%
  \let\item\@idxitem
}{%
  \ifkorrekturansicht\clearpage\fi
}
\makeatother

\IfFileExists{\jobname-pw.ind}{\input{\jobname-pw.ind}}{}

% Quellenangabe nur in der Leseansicht
\ifkorrekturansicht\else
% Fallback-Definitionen, falls die .tex-Datei \titel etc. nicht gesetzt hat
\providecommand{\titel}{}
\providecommand{\editorInnen}{}
\providecommand{\dateiname}{\jobname}

\vspace{3cm}

\vfill

\footnotesize
\textsc{Quelle}: \titel. Herausgegeben von {\editorInnen}. In: \emph{Arthur Schnitzler: Briefwechsel mit Autorinnen und Autoren}.
 Digitale Edition, https://schnitzler-briefe.acdh.oeaw.ac.at/{\dateiname}.html (Stand \today)
\fi

\end{document}


      