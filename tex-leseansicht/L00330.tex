%% latex-korrekturansicht-vorspann.tex
%% Vorspann für die Korrekturansicht.
%% Lädt die gemeinsame Datei latex-vorspann.tex mit gesetztem Schalter.

\newif\ifkorrekturansicht
\korrekturansichttrue

\input{../tex-inputs/latex-vorspann}


\section[Richard Beer-Hofmann an Arthur Schnitzler, {[}26. 5. 1894{]}]{L00330 Richard Beer-Hofmann an Arthur Schnitzler, {[}26. 5. 1894{]}}
\nopagebreak\mylabel{L00330v}
\rehead{ }\normalsize\beginnumbering\briefempfaengerindex{Schnitzler, Arthur@\textsc{Schnitzler, Arthur}!zzzBeer-Hofmann, Richard@\emph{von Richard Beer-Hofmann}!1894-05-261@{{[}26. 5. 1894{]}}|(be}
\toendnotes[C]{\smallbreak\pagebreak[2]}\Standort{CUL, Schnitzler, B 8.}
\physDesc{Brief, 1 Blatt, 1 Seite, 71 Zeichen
\newline{}Handschrift: blauer Buntstift, lateinische Kurrent
\newline{}Schnitzler: mit Bleistift datiert: »26/5 94« und nummeriert: »29.« }
\pstart
           \noindent{}{\pb}Lieber! Bin
                  Samstag{ }Abends – nicht zu spät \uline{Caffée Central}\oindex{Cafe Central@\textbf{Café Central}, \emph{Kaffeehaus (K.KAF)}|pw}.\pend
           \pstart Herzlichst\spacefill\mbox{Richard}\pend{}\selectlanguage{ngerman}\endnumbering\briefempfaengerindex{Schnitzler, Arthur@\textsc{Schnitzler, Arthur}!zzzBeer-Hofmann, Richard@\emph{von Richard Beer-Hofmann}!1894-05-261@{{[}26. 5. 1894{]}}|)be}\mylabel{L00330h}  \normalsize

\doendnotes{C}
\bigskip
\vfill

\clearpage

\footnotesize

\lohead{\textsc{register}}

% Definiere theindex-Environment komplett neu ohne reledmac
\makeatletter
\renewenvironment{theindex}{%
  \section*{\indexname}%
  \setlength{\parindent}{0pt}%
  \setlength{\parskip}{0pt plus 0.3pt}%
  \let\item\@idxitem
}{%
  \clearpage
}
\makeatother

\IfFileExists{\jobname-pw.ind}{\input{\jobname-pw.ind}}{}

\end{document}

      