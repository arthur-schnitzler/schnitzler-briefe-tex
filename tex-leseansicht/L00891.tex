%% latex-leseansicht-vorspann.tex
%% Vorspann für die Leseansicht.
%% Lädt die gemeinsame Datei latex-vorspann.tex mit nicht gesetztem Schalter.

\newif\ifkorrekturansicht
\korrekturansichtfalse

\input{../tex-inputs/latex-vorspann}


\section[Hugo von Hofmannsthal an Arthur Schnitzler, {[}17. 2. 1899{]}]{L00891 Hugo von Hofmannsthal an Arthur Schnitzler, {[}17. 2. 1899{]}}
\nopagebreak\mylabel{L00891v}
\rehead{ }\normalsize\beginnumbering\briefempfaengerindex{Schnitzler, Arthur@\textsc{Schnitzler, Arthur}!zzzHofmannsthal, Hugo von@\emph{von Hugo von Hofmannsthal}!1899-02-171@{{[}17. 2. 1899{]}}|(be}
\toendnotes[C]{\smallbreak\pagebreak[2]}
\correspDesc{Versand  durch Hugo von Hofmannsthal am [17. 2. 1899] in Wien
\newline{}Erhalt  durch Arthur Schnitzler im Zeitraum [17. 2. 1899
                  – 21. 2. 1899?] in Wien}\toendnotes[C]{\smallbreak}
\Standort{CUL, Schnitzler, B 43.}
\physDesc{Brief, 1 Blatt, 4 Seiten, 907 Zeichen
\newline{}Handschrift: schwarze Tinte, deutsche Kurrent
\newline{}Schnitzler: mit Bleistift datiert: »Feber 99« 
\newline{}Ordnung: mit Bleistift von unbekannter Hand nummeriert:
                                    »138« }
\buchAbdrucke{\weitereDrucke{Hugo von Hofmannsthal, Arthur Schnitzler: \emph{Briefwechsel}. Herausgegeben von Therese Nickl und Heinrich Schnitzler. Frankfurt am Main: \emph{S. Fischer} 1964, S. 118.} }\toendnotes[C]{\smallbreak}
\pstart
           \raggedleft{}{\pb}Freitag{ }Früh\pend
           \vspace{0.5em}
\pstart
           lieber, ich höre von Roſenbaum\pwindex{Rosenbaum, Richard 4.\,11.\,1867 Žikov – 25.\,6.\,1942 Konzentrationslager Theresienstadt@\textsc{Rosenbaum, Richard} (4.\,11.\,1867 Žikov – 25.\,6.\,1942 Konzentrationslager Theresienstadt), \emph{Dramaturg, Verleger}|pw} daſs Sonnenthal\pwindex{Sonnenthal, Adolf von 21.\,12.\,1834 Budapest – 4.\,4.\,1909 Prag@\textsc{Sonnenthal, Adolf von} (21.\,12.\,1834 Budapest – 4.\,4.\,1909 Prag), \emph{Schauspieler}|pw} auch den
                  Henry\pwindex{Schnitzler, Arthur 15.\,5.\,1862 Wien – 21.\,10.\,1931 ebd.@\textsc{Schnitzler, Arthur} (15.\,5.\,1862 Wien – 21.\,10.\,1931 ebd.), \emph{Schriftsteller, Mediziner}!grüne Kakadu. Groteske in einem Akt@\strich\emph{Der grüne Kakadu. Groteske in einem Akt}|pwv}{ }ſpielt, was ich{ }ſehr geſcheidt und richtig finde.
               Nur möchte ich doch nicht, daſs die nachträgliche Folge davon wäre, daſs er auch
               nicht einmal die eine Rolle des Kaufmanns\pwindex{Hofmannsthal, Hugo von 1.\,2.\,1874 Wien – 15.\,7.\,1929 Rodaun@\textsc{Hofmannsthal, Hugo von} (1.\,2.\,1874 Wien – 15.\,7.\,1929 Rodaun), \emph{Schriftsteller}!Hochzeit der Sobeide@\strich\emph{Die Hochzeit der Sobeide}|pwv} in meinen Stücken\pwindex{Hofmannsthal, Hugo von 1.\,2.\,1874 Wien – 15.\,7.\,1929 Rodaun@\textsc{Hofmannsthal, Hugo von} (1.\,2.\,1874 Wien – 15.\,7.\,1929 Rodaun), \emph{Schriftsteller}!Hochzeit der Sobeide@\strich\emph{Die Hochzeit der Sobeide}|pwv}\pwindex{Hofmannsthal, Hugo von 1.\,2.\,1874 Wien – 15.\,7.\,1929 Rodaun@\textsc{Hofmannsthal, Hugo von} (1.\,2.\,1874 Wien – 15.\,7.\,1929 Rodaun), \emph{Schriftsteller}!Abenteurer und die Sängerin oder Die Geschenke des Lebens@\strich\emph{Der Abenteurer und die Sängerin oder Die Geschenke des Lebens}|pwv}{ }{\pb}lernen kann oder will, weil ja
               auf dieſe Art der Abend immer mehr gefährdet würde. Ich meine alſo, daſs Sie – wenn
               einmal Ihre Proben\pwindex{Schnitzler, Arthur 15.\,5.\,1862 Wien – 21.\,10.\,1931 ebd.@\textsc{Schnitzler, Arthur} (15.\,5.\,1862 Wien – 21.\,10.\,1931 ebd.), \emph{Schriftsteller, Mediziner}!grüne Kakadu – Paracelsus – Die Gefährtin. Drei Einakter@\strich\emph{Der grüne Kakadu – Paracelsus – Die Gefährtin. Drei Einakter}|pwv} in Gang{ }ſind, nicht früher – bei ihm\pwindex{Sonnenthal, Adolf von 21.\,12.\,1834 Budapest – 4.\,4.\,1909 Prag@\textsc{Sonnenthal, Adolf von} (21.\,12.\,1834 Budapest – 4.\,4.\,1909 Prag), \emph{Schauspieler}|pwv}
               und Schlenther\pwindex{Schlenther, Paul 20.\,8.\,1854 Chernyakhovsk – 30.\,4.\,1916 Berlin@\textsc{Schlenther, Paul} (20.\,8.\,1854 Chernyakhovsk – 30.\,4.\,1916 Berlin), \emph{Schriftsteller, Kritiker, Theaterleiter}|pw} dahin wirken könnten, daſs {\pb}er\pwindex{Sonnenthal, Adolf von 21.\,12.\,1834 Budapest – 4.\,4.\,1909 Prag@\textsc{Sonnenthal, Adolf von} (21.\,12.\,1834 Budapest – 4.\,4.\,1909 Prag), \emph{Schauspieler}|pwv}{ }ſich bereit erklärt, nach Ihrer Premiere\pwindex{Schnitzler, Arthur 15.\,5.\,1862 Wien – 21.\,10.\,1931 ebd.@\textsc{Schnitzler, Arthur} (15.\,5.\,1862 Wien – 21.\,10.\,1931 ebd.), \emph{Schriftsteller, Mediziner}!grüne Kakadu – Paracelsus – Die Gefährtin. Drei Einakter@\strich\emph{Der grüne Kakadu – Paracelsus – Die Gefährtin. Drei Einakter}|pwv} nicht plötzlich ermüdet zu{ }ſein
               und{ }ſicher die gar nicht anſtrengende Rolle, in der er mir unentbehrlich{ }ſcheint, zu
               übernehmen.\pend
           
\pstart
           Herzlich Ihr{\\[\baselineskip]}\spacefill\mbox{Hugo}\pend
           \leftskip=0em{}
\pstart
           \label{K_L00891-1v}\edtext{Samstag{ }Rebhuhn\oindex{Wien@\textbf{Wien}!I., Innere Stadt@\textbf{I., Innere Stadt}!Café Rebhuhn@\textbf{Café Rebhuhn}, \emph{Kaffeehaus}|pw}}{\lemma{\textnormal{\emph{Samstag Rebhuhn}}}\Cendnote{\textnormal{Vgl. A. S.: \emph{Tagebuch}, 18. 2. 1899.
                  }}}\label{K_L00891-1}!\pend
           
\pstart
           {\pb}Ich möchte,{ }ſolang{ }ſich kein
                  greifbares Hindernis{ }ſondern nur die allgemeine Indolenz entgegenſtellt, natürlich
                  an dem Datum des \label{K_L00891-2v}\edtext{11\textsuperscript{ten} März}{\lemma{\textnormal{\emph{11\textsuperscript{ten} März}}}\Cendnote{\textnormal{Tatsächlich fand sie am
                        18. 3. 1899 statt.}}}\label{K_L00891-2} feſthalten und dazu iſt natürlich{ }ſehr nöthig, daſs Ihre Aufführung\pwindex{Schnitzler, Arthur 15.\,5.\,1862 Wien – 21.\,10.\,1931 ebd.@\textsc{Schnitzler, Arthur} (15.\,5.\,1862 Wien – 21.\,10.\,1931 ebd.), \emph{Schriftsteller, Mediziner}!grüne Kakadu – Paracelsus – Die Gefährtin. Drei Einakter@\strich\emph{Der grüne Kakadu – Paracelsus – Die Gefährtin. Drei Einakter}|pwv} nicht über den \label{K_L00891-3v}\edtext{25\textsuperscript{ten} dieſes}{\lemma{\textnormal{\emph{25\textsuperscript{ten} dieses}}}\Cendnote{\textnormal{Diese verzögerte sich
                     auf den 1. 3. 1899.}}}\label{K_L00891-3} verzögert wird.\pend
           \selectlanguage{ngerman}\endnumbering\briefempfaengerindex{Schnitzler, Arthur@\textsc{Schnitzler, Arthur}!zzzHofmannsthal, Hugo von@\emph{von Hugo von Hofmannsthal}!1899-02-171@{{[}17. 2. 1899{]}}|)be}\mylabel{L00891h}  \newcommand{\dateiname}{L00891}\newcommand{\titel}{Hugo von Hofmannsthal an Arthur Schnitzler, [17. 2. 1899]}\newcommand{\editorInnen}{Martin Anton Müller und Gerd-Hermann Susen}%% latex-leseansicht-abspann.tex
%% Abspann für die Leseansicht.
%% Der Schalter \ifkorrekturansicht ist bereits durch den Vorspann gesetzt.

%% latex-abspann.tex
%% Gemeinsamer Abspann für Korrekturansicht und Leseansicht.
%% Setzt den Schalter \ifkorrekturansicht voraus (gesetzt in den
%% einbindenden Dateien latex-korrekturansicht-abspann.tex bzw.
%% latex-leseansicht-abspann.tex).
%% ---------------------------------------------------------------

\normalsize

% Das esempio-Environment wird nur in der Leseansicht benötigt
\ifkorrekturansicht\else
\newenvironment{esempio}[3]%
{
    \vspace{1.5ex}
    \rlap{\underline{#1}}
    \par
    \setlength{\parindent}{0cm}
    \nopagebreak
    \leftskip=#2cm
    \rightskip=#3cm
}
{
    \par
}
\fi

\doendnotes{C}
\bigskip
\vfill

\clearpage

\footnotesize

\ifkorrekturansicht
  \lohead{\textsc{register}}
\fi

% theindex-Environment neu definieren ohne reledmac
\makeatletter
\renewenvironment{theindex}{%
  \ifkorrekturansicht
    \section*{\indexname}%
  \else
    \subsubsection*{Index der erwähnten Entitäten}%
  \fi
  \setlength{\parindent}{0pt}%
  \setlength{\parskip}{0pt plus 0.3pt}%
  \let\item\@idxitem
}{%
  \ifkorrekturansicht\clearpage\fi
}
\makeatother

\IfFileExists{\jobname-pw.ind}{\input{\jobname-pw.ind}}{}

% Quellenangabe nur in der Leseansicht
\ifkorrekturansicht\else
% Fallback-Definitionen, falls die .tex-Datei \titel etc. nicht gesetzt hat
\providecommand{\titel}{}
\providecommand{\editorInnen}{}
\providecommand{\dateiname}{\jobname}

\vspace{3cm}

\vfill

\footnotesize
\textsc{Quelle}: \titel. Herausgegeben von {\editorInnen}. In: \emph{Arthur Schnitzler: Briefwechsel mit Autorinnen und Autoren}.
 Digitale Edition, https://schnitzler-briefe.acdh.oeaw.ac.at/{\dateiname}.html (Stand \today)
\fi

\end{document}


