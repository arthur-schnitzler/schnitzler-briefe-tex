%% latex-korrekturansicht-vorspann.tex
%% Vorspann für die Korrekturansicht.
%% Lädt die gemeinsame Datei latex-vorspann.tex mit gesetztem Schalter.

\newif\ifkorrekturansicht
\korrekturansichttrue

\input{../tex-inputs/latex-vorspann}


\section[Hugo von Hofmannsthal an Arthur Schnitzler, {[}17. 2. 1899{]}]{L00891 Hugo von Hofmannsthal an Arthur Schnitzler, {[}17. 2. 1899{]}}
\nopagebreak\mylabel{L00891v}
\rehead{ }\normalsize\beginnumbering\briefempfaengerindex{Schnitzler, Arthur@\textsc{Schnitzler, Arthur}!zzzHofmannsthal, Hugo von@\emph{von Hugo von Hofmannsthal}!1899-02-171@{{[}17. 2. 1899{]}}|(be}
\toendnotes[C]{\smallbreak\pagebreak[2]}\Standort{CUL, Schnitzler, B 43.}
\physDesc{Brief, 1 Blatt, 4 Seiten, 907 Zeichen
\newline{}Handschrift: schwarze Tinte, deutsche Kurrent
\newline{}Schnitzler: mit Bleistift datiert: »Feber 99« 
\newline{}Ordnung: mit Bleistift von unbekannter Hand nummeriert:
                                    »138« }
\buchAbdrucke{\weitereDrucke{Hugo von Hofmannsthal, Arthur Schnitzler: \emph{Briefwechsel}. Frankfurt am Main: \emph{S. Fischer} 1964, S. 118.} }\toendnotes[C]{\smallbreak}
\pstart
           \raggedleft{}{\pb}Freitag{ }Früh\pend
           \vspace{0.5em}
\pstart
           lieber, ich höre von Roſenbaum\pwindex{Rosenbaum, Richard 04.11.1867 – 25.06.1942@\textsc{Rosenbaum, Richard} (04.11.1867 – 25.06.1942), \emph{Dramaturg/Dramaturgin, Verleger/Verlegerin}|pw} daſs Sonnenthal\pwindex{Sonnenthal, Adolf von 1834-12-21 – 1909-04-04@\textsc{Sonnenthal, Adolf von} (1834-12-21 – 1909-04-04), \emph{Schauspieler/Schauspielerin}|pw} auch den
                  Henry\pwindex{gruene Kakadu. Groteske in einem Akt@\emph{Der grüne Kakadu. Groteske in einem Akt}|pwv}{ }ſpielt, was ich ſehr geſcheidt und richtig finde.
               Nur möchte ich doch nicht, daſs die nachträgliche Folge davon wäre, daſs er auch
               nicht einmal die eine Rolle des Kaufmanns\pwindex{Hochzeit der Sobeide@\emph{Die Hochzeit der Sobeide}|pwv} in meinen Stücken\pwindex{Hochzeit der Sobeide@\emph{Die Hochzeit der Sobeide}|pwv}\pwindex{Abenteurer und die Saengerin oder Die Geschenke des Lebens@\emph{Der Abenteurer und die Sängerin oder Die Geschenke des Lebens}|pwv}{ }{\pb}lernen kann oder will, weil ja
               auf dieſe Art der Abend immer mehr gefährdet würde. Ich meine alſo, daſs Sie – wenn
               einmal Ihre Proben\pwindex{gruene Kakadu – Paracelsus – Die Gefaehrtin. Drei Einakter@\emph{Der grüne Kakadu – Paracelsus – Die Gefährtin. Drei Einakter}|pwv} in Gang
               ſind, nicht früher – bei ihm\pwindex{Sonnenthal, Adolf von 1834-12-21 – 1909-04-04@\textsc{Sonnenthal, Adolf von} (1834-12-21 – 1909-04-04), \emph{Schauspieler/Schauspielerin}|pwv}
               und Schlenther\pwindex{Schlenther, Paul 20.08.1854 – 30.04.1916@\textsc{Schlenther, Paul} (20.08.1854 – 30.04.1916), \emph{Schriftsteller/Schriftstellerin, Kritiker/Kritikerin, Theaterleiter/Theaterleiterin}|pw} dahin wirken könnten, daſs {\pb}er\pwindex{Sonnenthal, Adolf von 1834-12-21 – 1909-04-04@\textsc{Sonnenthal, Adolf von} (1834-12-21 – 1909-04-04), \emph{Schauspieler/Schauspielerin}|pwv}{ }ſich bereit erklärt, nach Ihrer Premiere\pwindex{gruene Kakadu – Paracelsus – Die Gefaehrtin. Drei Einakter@\emph{Der grüne Kakadu – Paracelsus – Die Gefährtin. Drei Einakter}|pwv} nicht plötzlich ermüdet zu ſein
               und ſicher die gar nicht anſtrengende Rolle, in der er mir unentbehrlich ſcheint, zu
               übernehmen.\pend
           
\pstart
           Herzlich Ihr{\\[\baselineskip]}\spacefill\mbox{Hugo}\pend
           \leftskip=0em{}
\pstart
           \label{K_L00891-1v}\edtext{Samstag{ }Rebhuhn\oindex{Cafe Rebhuhn@\textbf{Café Rebhuhn}, \emph{Kaffeehaus (K.KAF)}|pw}}{\lemma{\textnormal{\emph{Samstag Rebhuhn}}}\Cendnote{\textnormal{Vgl. A. S.: \emph{Tagebuch}, 18. 2. 1899.
                  }}}\label{K_L00891-1}!\pend
           
\pstart
           {\pb}Ich möchte, ſolang ſich kein
                  greifbares Hindernis ſondern nur die allgemeine Indolenz entgegenſtellt, natürlich
                  an dem Datum des \label{K_L00891-2v}\edtext{11\textsuperscript{ten} März}{\lemma{\textnormal{\emph{11\textsuperscript{ten} März}}}\Cendnote{\textnormal{Tatsächlich fand sie am
                        18. 3. 1899 statt.}}}\label{K_L00891-2} feſthalten und dazu iſt natürlich
                  ſehr nöthig, daſs Ihre Aufführung\pwindex{gruene Kakadu – Paracelsus – Die Gefaehrtin. Drei Einakter@\emph{Der grüne Kakadu – Paracelsus – Die Gefährtin. Drei Einakter}|pwv} nicht über den \label{K_L00891-3v}\edtext{25\textsuperscript{ten} dieſes}{\lemma{\textnormal{\emph{25\textsuperscript{ten} dieſes}}}\Cendnote{\textnormal{Diese verzögerte sich
                     auf den 1. 3. 1899.}}}\label{K_L00891-3} verzögert wird.\pend
           \selectlanguage{ngerman}\endnumbering\briefempfaengerindex{Schnitzler, Arthur@\textsc{Schnitzler, Arthur}!zzzHofmannsthal, Hugo von@\emph{von Hugo von Hofmannsthal}!1899-02-171@{{[}17. 2. 1899{]}}|)be}\mylabel{L00891h}  \normalsize

\doendnotes{C}
\bigskip
\vfill

\clearpage

\footnotesize

\lohead{\textsc{register}}

% Definiere theindex-Environment komplett neu ohne reledmac
\makeatletter
\renewenvironment{theindex}{%
  \section*{\indexname}%
  \setlength{\parindent}{0pt}%
  \setlength{\parskip}{0pt plus 0.3pt}%
  \let\item\@idxitem
}{%
  \clearpage
}
\makeatother

\IfFileExists{\jobname-pw.ind}{\input{\jobname-pw.ind}}{}

\end{document}

      