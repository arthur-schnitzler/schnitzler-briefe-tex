%% latex-korrekturansicht-vorspann.tex
%% Vorspann für die Korrekturansicht.
%% Lädt die gemeinsame Datei latex-vorspann.tex mit gesetztem Schalter.

\newif\ifkorrekturansicht
\korrekturansichttrue

\input{../tex-inputs/latex-vorspann}


\section[Arthur Schnitzler an Gerhart Hauptmann, 24. 10. 1904]{L01460 Arthur Schnitzler an Gerhart Hauptmann, 24. 10. 1904}
\nopagebreak\mylabel{L01460v}
\rehead{ }\normalsize\beginnumbering\briefempfaengerindex{Hauptmann, Gerhart@\textsc{Hauptmann, Gerhart}!zzzSchnitzler, Arthur@\emph{von Arthur Schnitzler}!1904-10-241@{24. 10. 1904}|(be}
\toendnotes[C]{\smallbreak\pagebreak[2]}\Standort{Staatsbibliothek Berlin – Preußischer Kulturbesitz, GHBrBl A:Schnitzler (9).}
\physDesc{Telegramm, 203 Zeichen
\newline{}maschinell
\newline{}Versand: 1) auf der Rückseite ein handschriftlicher Vermerk mit Tinte:
                                    »Adrſ. Im Lessing-Theater\orgindex{Lessing-Theater@Lessing-Theater|pw} nicht anweſend, Nachſ. nach Hotel de Rome\oindex{Hotel de Rome@\textbf{Hotel de Rome}, \emph{Hotel (K.HTL)}|pw}. \textcolor{gray}{Litmica}\pwindex{Litmica @\textsc{Litmica}, \emph{Briefträger/Briefträgerin}|pw}«  2) Stempel: »\nobreak{}\oindex{Berlin@\textbf{Berlin}, \emph{P.PPLC}|pwk}Berlin N. W., 24 V 04, 1\textcolor{gray}{8}\textsuperscript{20}N\nobreak{}«.  3) Stempel: »\nobreak{}Ausgefertigt, 24 Oct., 1\textcolor{gray}{×}\textsuperscript{\textcolor{gray}{×}2}\nobreak{}«.  4) »\textcolor{gray}{\textbf{\textbf{Aufgenommen} von}}{ }\textcolor{gray}{IW}{ }\textcolor{gray}{\textbf{den}}{ }24\textcolor{gray}{\textbf{/}}11{ }\textcolor{gray}{\textbf{um}}{ }11 \textcolor{gray}{\textbf{Uhr}} 19 \textcolor{gray}{\textbf{M.}}m{ }\textcolor{gray}{\textbf{durch}}{ }\textcolor{gray}{Grm}« 5) mit Bleistift unterhalb des Empfängers: »Hotel Rom{ }U d Linden 39\oindex{Unter den Linden@\textbf{Unter den Linden}, \emph{P.PPLX}|pw}«
\newline{}Ordnung: Lochung }\pstart{}{\pb}= {[}g{]}erhard hauptmann
                     berlin\oindex{Berlin@\textbf{Berlin}, \emph{P.PPLC}|pw}\pend{}\pstart{}lessingtheater\oindex{Lessing-Theater@\textbf{Lessing-Theater}, \emph{Theater (K.THE)}|pw} +\pend{}{\bigskip}\vspace{1em}
\pstart
           {\pb}\textcolor{gray}{\textbf{Telegramm}} de wien\oindex{Wien@\textbf{Wien}, \emph{A.ADM2}|pw}
                  111.+723 21 24{ }10 40m{ }\textcolor{gray}{\textbf{W.}}{ }\textcolor{gray}{\textbf{190}}4{ }\pend
           \vspace{0.5em}
\pstart
           jch beglueckwuensche sie und brahm\pwindex{Brahm, Otto 05.02.1856 – 28.11.1912@\textsc{Brahm, Otto} (05.02.1856 – 28.11.1912), \emph{Theaterleiter/Theaterleiterin, Regisseur/Regisseurin}|pw} herzlich
               zur ruhmreichen auferstehung des florian geyer\pwindex{Florian Geyer. Die Tragoedie des Bauernkrieges@\emph{Florian Geyer. Die Tragödie des Bauernkrieges}|pw} –
               herzlichst gruessend jhr \spacefill\mbox{arthur schnitzler +}\pend
           \selectlanguage{ngerman}\endnumbering\briefempfaengerindex{Hauptmann, Gerhart@\textsc{Hauptmann, Gerhart}!zzzSchnitzler, Arthur@\emph{von Arthur Schnitzler}!1904-10-241@{24. 10. 1904}|)be}\mylabel{L01460h}  \normalsize

\doendnotes{C}
\bigskip
\vfill

\clearpage

\footnotesize

\lohead{\textsc{register}}

% Definiere theindex-Environment komplett neu ohne reledmac
\makeatletter
\renewenvironment{theindex}{%
  \section*{\indexname}%
  \setlength{\parindent}{0pt}%
  \setlength{\parskip}{0pt plus 0.3pt}%
  \let\item\@idxitem
}{%
  \clearpage
}
\makeatother

\IfFileExists{\jobname-pw.ind}{\input{\jobname-pw.ind}}{}

\end{document}

      