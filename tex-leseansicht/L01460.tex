\input{../tex-inputs/latex-pdf-vorspann}
\begin{center}
            \textcolor{red}{ENTWURF. ENTZIFFERUNG NOCH NICHT KORREKTURGELESEN}
                      \end{center}
            
               \section[Arthur Schnitzler an Gerhart Hauptmann, 24. 10. 1904]{ Arthur Schnitzler an Gerhart Hauptmann, 24. 10. 1904}\nopagebreak\mylabel{v}\rehead{ }\begin{ledgroupsized}[t]{13cm}\normalsize\beginnumbering\briefempfaengerindex{Hauptmann, Gerhart@\textsc{Hauptmann, Gerhart}!zzzSchnitzler, Arthur@\emph{von Arthur Schnitzler}!1904-10-241@{24. 10. 1904}|(be} \toendnotes[C]{\smallbreak\pagebreak[2]} \Standort{Staatsbibliothek Berlin – Preußischer Kulturbesitz, GHBrBl A:Schnitzler (9).}
\physDesc{Telegramm
\newline{}maschinell\newline{}Versand: 1) auf der Rückseite ein handschriftlicher Vermerk mit Tinte:
                                    »Adrſ. Im Lessing-Theater\orgindex{Lessing-Theater@Lessing-Theater|pw} nicht anweſend, Nachſ. nach Hotel de Rome\oindex{Hotel de Rome@\textbf{Hotel de Rome}|pw}. \textcolor{gray}{Litmica}\pwindex{Litmica *~24.11.1904@\textsc{Litmica} (*~24.11.1904), \emph{Briefträger/Briefträgerin}|pw}« 2) Stempel: »\nobreak{}\oindex{Berlin@\textbf{Berlin}|pwk}Berlin N.W., 24 V 04, 1\textcolor{gray}{8}\textsuperscript{20}N\nobreak{}«. 3) Stempel: »\nobreak{}Ausgefertigt, 24 Oct., 1\textcolor{gray}{×}\textsuperscript{\textcolor{gray}{×}2}\nobreak{}«. 4) »\textcolor{gray}{\textbf{\textbf{Aufgenommen} von}}{ }\textcolor{gray}{IW}{ }\textcolor{gray}{\textbf{den}}{ }24\textcolor{gray}{\textbf{/}}11{ }\textcolor{gray}{\textbf{um}}{ }11 \textcolor{gray}{\textbf{Uhr}} 19 \textcolor{gray}{\textbf{M.}}m{ }\textcolor{gray}{\textbf{durch}}{ }\textcolor{gray}{Grm}«5) mit Bleistift unterhalb des Empfängers: »Hotel Rom{ }U d Linden 39\oindex{Unter den Linden@\textbf{Unter den Linden}|pw}«\newline{}Ordnung: Lochung }\pstart{}{\pb}= {[}g{]}erhard hauptmann
                     berlin\oindex{Berlin@\textbf{Berlin}|pw}\pend{}\pstart{}lessingtheater\oindex{Lessing-Theater@\textbf{Lessing-Theater}|pw} +\pend{}{\bigskip}\pstart
           {\pb}\textcolor{gray}{\textbf{Telegramm}} de wien\oindex{Wien@\textbf{Wien}|pw}
                  111.+723 21 24{ }10 40m{ }\textcolor{gray}{\textbf{W.}}{ }\textcolor{gray}{\textbf{190}}4{ }\pend
           \pstart
           jch beglueckwuensche sie und brahm\pwindex{Brahm, Otto 05.02.1856 – 28.11.1912@\textsc{Brahm, Otto} (05.02.1856 – 28.11.1912), \emph{Theaterleiter, Regisseur}|pw} herzlich zur
               ruhmreichen auferstehung des florian geyer\pwindex{Hauptmann, Gerhart 15.11.1862 – 06.06.1946@\textsc{Hauptmann, Gerhart} (15.11.1862 – 06.06.1946), \emph{Schriftsteller}!Florian Geyer. Die Tragoedie des Bauernkrieges1896 – 1896@\strich\emph{Florian Geyer. Die Tragödie des Bauernkrieges} {[}1896 – 1896{]}|pw} –
               herzlichst gruessend jhr \spacefill\mbox{arthur schnitzler +}\pend
           \endnumbering\briefempfaengerindex{Hauptmann, Gerhart@\textsc{Hauptmann, Gerhart}!zzzSchnitzler, Arthur@\emph{von Arthur Schnitzler}!1904-10-241@{24. 10. 1904}|)be}\mylabel{h}\end{ledgroupsized}  \newcommand{\dateiname}{L01460}\newcommand{\titel}{Arthur Schnitzler an Gerhart Hauptmann, 24. 10. 1904}\newcommand{\editorInnen}{ Martin Anton Müller und Gerd-Hermann Susen}\input{../tex-inputs/latex-pdf-abspann}
      