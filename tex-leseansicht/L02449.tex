%% latex-leseansicht-vorspann.tex
%% Vorspann für die Leseansicht.
%% Lädt die gemeinsame Datei latex-vorspann.tex mit nicht gesetztem Schalter.

\newif\ifkorrekturansicht
\korrekturansichtfalse

\input{../tex-inputs/latex-vorspann}


         
         \renewcommand{\erwaehntePersonen}{Personen: Leopold Schwarzschild}
         \renewcommand{\erwaehnteInstitutionen}{Institutionen: Das Tage-Buch, Frankfurter Zeitung}
         \renewcommand{\erwaehnteOrte}{Orte: Berlin, Beuthstrasse, Sternwartestraße, Wien, XVIII., Währing, Österreich}
         \renewcommand{\erwaehnteWerke}{}
               \section[Stefan Großmann an Arthur Schnitzler, 21. 9. 1925]{ Stefan Großmann an Arthur Schnitzler, 21. 9. 1925}\nopagebreak\mylabel{v}\rehead{ }\begin{ledgroupsized}[t]{13cm}\normalsize\beginnumbering \toendnotes[C]{\smallbreak\pagebreak[2]} \Standort{DLA, A:Schnitzler, HS.NZ85.1.3232.}
\physDesc{Brief, 1 Blatt, 2 Seiten, 1156 Zeichen
\newline{}Schreibmaschine
\newline{}Handschrift: schwarze Tinte, deutsche Kurrent (\noindent{}Unterschrift)
\newline{}Schnitzler: mit rotem Buntstift vier Unterstreichungen }\toendnotes[C]{\smallbreak}\pstart
           \noindent{}\centering{}{\pb}\textcolor{gray}{\textbf{Das Tage-Buch\orgindex{Tage-Buch@Das Tage-Buch|pw}}}\pend
           \pstart
           \noindent{}\centering{}\textcolor{gray}{\textbf{\emph{Herausgeber: Stefan Großmann und Leopold Schwarzschild\pwindex{Schwarzschild, Leopold 1891-12-08 – 1950-10-02@\textsc{Schwarzschild, Leopold} (1891-12-08 – 1950-10-02), \emph{Publizist}|pw}}}}\pend
           \pstart
           \noindent{}\centering{}\textcolor{gray}{\textbf{Tagebuchverlag m. b. H., Berlin SW 19\oindex{Berlin@\textbf{Berlin}|pw}}}\pend
           \pstart
           \noindent{}\centering{}\textcolor{gray}{\textbf{BEUTHSTRASSE 19\oindex{Beuthstrasse@\textbf{Beuthstrasse}|pw}}}\pend
           \pstart
           \noindent{}\centering{}\textcolor{gray}{\textbf{\emph{Telegramm-Adresse: Tagebuch Berlin\oindex{Berlin@\textbf{Berlin}|pw} ⋅ Fernsprecher: Merkur 8790–8792}}}\pend
           \pstart
           \noindent{}\centering{}\textcolor{gray}{\textbf{\emph{\so{Sprechstunde der Redaktion: 12–1 Uhr}}}}\pend
           \pstart
           \noindent{}\centering{}\textcolor{gray}{\textbf{*}}\pend
           \pstart
           \noindent{}Tgb./Gr./Schl.\hfill Berlin\oindex{Berlin@\textbf{Berlin}|pw}, den 21. September
                     1925.\pend
           \pstart
           \raggedleft{}Herrn\pend
           \pstart
           \noindent{}\raggedleft{}Dr. Arthur \so{Schnitzler}\pend
           \pstart
           \noindent{}\raggedleft{}\so{Wien } XVIII\oindex{XVIII., Waehring@\textbf{XVIII., Währing}|pw}\pend
           {\bigskip}\pstart
           \noindent{}\raggedleft{}Sternwartestr. 71\oindex{Sternwartestrasse@\textbf{Sternwartestraße}|pw}. \pend
           \pstart\center{}Verehrter Herr Doktor Schnitzler!\pend\pstart
           Ich bemühe mich, meinem TAGE-BUCH\orgindex{Tage-Buch@Das Tage-Buch|pw} einen leichten österreich\oindex{Oesterreich@\textbf{Österreich}|pw}ischen Anstrich
               zu geben. Sie würden mir eine sehr grosse Freude machen und mich zu grossem Dank
               verpflichten, wenn Sie mir für eine der nächsten Nummern des TAGE-BUCH\orgindex{Tage-Buch@Das Tage-Buch|pw}ES einen Beitrag schicken würden. Gäbe es nicht in einer Ihrer Mappen irgendwo
               eine kleine Novelle, die Sie mir überlassen könnten? Ich würde mich, da sich das TAGE-BUCH\orgindex{Tage-Buch@Das Tage-Buch|pw} ja jetzt durchgesetzt hat, zu dem höchsten Honorar entschliessen, das ich
               aufbringen kann. Aber selbst wenn Sie mir diese Bitte abschlagen müssen – ich hoffe,
               dass es nicht geschehen muss –, weiss ich aus den \label{K_L02449_1v}\edtext{Veröffentlichungen}{\lemma{\textnormal{\emph{Veröffentlichungen}}}\Cendnote{\textnormal{Gegenwärtig ist kein Abdruck eines Textes im Jahre 1925
                  bekannt.}}}\label{K_L02449_1h} in der Frankfurter Zeitung\orgindex{Frankfurter Zeitung@Frankfurter Zeitung|pw},
               dass Sie eine grosse Mappe mit Reflexionen haben. Ich bitte Sie sehr, öffnen Sie
               diese Mappe und schicken Sie mir einige Seiten daraus, die ich im {\pb}TAGE-BUCH\orgindex{Tage-Buch@Das Tage-Buch|pw} veröffentlichen kann. Ich weiss, dass Sie viele solche Bitten abschlagen,
               dennoch glaube ich, dass Sie mir in mein Berlin\oindex{Berlin@\textbf{Berlin}|pw}er
               Exil diesmal keine Absage schicken werden.\pend
           \pstart
           Ich bin mit dankbaren Grüssen{\\[\baselineskip]}Ihr sehr ergebener{\\[\baselineskip]}\spacefill\mbox{{[}hs.:{]} Stefan Großmann}\pend
           \leftskip=0em{}
         
         \endnumbering\mylabel{h}\end{ledgroupsized}  \newcommand{\dateiname}{L02449}\newcommand{\titel}{Stefan Großmann an Arthur Schnitzler, 21. 9. 1925}\newcommand{\editorInnen}{ Martin Anton Müller und Gerd-Hermann Susen}%% latex-leseansicht-abspann.tex
%% Abspann für die Leseansicht.
%% Der Schalter \ifkorrekturansicht ist bereits durch den Vorspann gesetzt.

%% latex-abspann.tex
%% Gemeinsamer Abspann für Korrekturansicht und Leseansicht.
%% Setzt den Schalter \ifkorrekturansicht voraus (gesetzt in den
%% einbindenden Dateien latex-korrekturansicht-abspann.tex bzw.
%% latex-leseansicht-abspann.tex).
%% ---------------------------------------------------------------

\normalsize

% Das esempio-Environment wird nur in der Leseansicht benötigt
\ifkorrekturansicht\else
\newenvironment{esempio}[3]%
{
    \vspace{1.5ex}
    \rlap{\underline{#1}}
    \par
    \setlength{\parindent}{0cm}
    \nopagebreak
    \leftskip=#2cm
    \rightskip=#3cm
}
{
    \par
}
\fi

\doendnotes{C}
\bigskip
\vfill

\clearpage

\footnotesize

\ifkorrekturansicht
  \lohead{\textsc{register}}
\fi

% theindex-Environment neu definieren ohne reledmac
\makeatletter
\renewenvironment{theindex}{%
  \ifkorrekturansicht
    \section*{\indexname}%
  \else
    \subsubsection*{Index der erwähnten Entitäten}%
  \fi
  \setlength{\parindent}{0pt}%
  \setlength{\parskip}{0pt plus 0.3pt}%
  \let\item\@idxitem
}{%
  \ifkorrekturansicht\clearpage\fi
}
\makeatother

\IfFileExists{\jobname-pw.ind}{\input{\jobname-pw.ind}}{}

% Quellenangabe nur in der Leseansicht
\ifkorrekturansicht\else
% Fallback-Definitionen, falls die .tex-Datei \titel etc. nicht gesetzt hat
\providecommand{\titel}{}
\providecommand{\editorInnen}{}
\providecommand{\dateiname}{\jobname}

\vspace{3cm}

\vfill

\footnotesize
\textsc{Quelle}: \titel. Herausgegeben von {\editorInnen}. In: \emph{Arthur Schnitzler: Briefwechsel mit Autorinnen und Autoren}.
 Digitale Edition, https://schnitzler-briefe.acdh.oeaw.ac.at/{\dateiname}.html (Stand \today)
\fi

\end{document}


      