%% latex-korrekturansicht-vorspann.tex
%% Vorspann für die Korrekturansicht.
%% Lädt die gemeinsame Datei latex-vorspann.tex mit gesetztem Schalter.

\newif\ifkorrekturansicht
\korrekturansichttrue

\input{../tex-inputs/latex-vorspann}


\section[ Paul Goldmann an Arthur Schnitzler, 7. 2. 1906]{L03239 Paul Goldmann an Arthur Schnitzler, 7. 2. 1906}
\nopagebreak\mylabel{L03239v}
\rehead{ }\normalsize\beginnumbering\briefempfaengerindex{Schnitzler, Arthur@\textsc{Schnitzler, Arthur}!zzzGoldmann, Paul@\emph{von Paul Goldmann}!1906-02-071@{7. 2. 1906}|(be}
\toendnotes[C]{\smallbreak\pagebreak[2]}\Standort{DLA, A:Schnitzler, HS.NZ85.1.3175.}
\physDesc{Postkarte, 410 Zeichen
\newline{}Handschrift: 1) blaue Tinte, deutsche Kurrent\hspace{1em}2) blaue Tinte, lateinische Kurrent (\noindent{}Adresse)\hspace{1em}
\newline{}Versand: 1) Stempel: »\nobreak{}\oindex{Berlin@\textbf{Berlin}, \emph{P.PPLC}|pwk}Berlin, S.W. 11, 7. 2. 06, 10–11N.\nobreak{}«.   2) Stempel: »\nobreak{}\oindex{VIII., Josefstadt@\textbf{VIII., Josefstadt}, \emph{A.ADM3}|pwk}18⁄\textsubscript{1} Wi{[}en{]}, 8. II. 06, 5, Bestellt\nobreak{}«. }\toendnotes[C]{\smallbreak}\pstart{}{\pb}Herrn\pend{}\pstart{}Dr. Arthur Schnitzler\pend{}\pstart{}Wien\oindex{Wien@\textbf{Wien}, \emph{A.ADM2}|pw}\pend{}\pstart{}XVIII. Spöttelgaſse 7\oindex{Edmund-Weiss-Gasse 7@\textbf{Edmund-Weiß-Gasse 7}, \emph{Wohngebäude (K.WHS)}|pw}.\pend{}{\bigskip}\vspace{1em}
\pstart
           \noindent{}{\pb}Berlin\oindex{Berlin@\textbf{Berlin}, \emph{P.PPLC}|pw}, 7. Februar. Lieber Freund, Als ich heut um 5 Uhr im \textsc{Hotel Continental\oindex{Hotel Continental [Berlin]@\textbf{Hotel Continental [Berlin]}, \emph{Hotel (K.HTL)}|pw}} vorſprach, mußte ich leider vom Portier\pwindex{?? [Portier des Hotel Continental Berlin] 1906-02-07 – 1906-02-07@\textsc{?? [Portier des Hotel Continental Berlin]} (1906-02-07 – 1906-02-07)|pw}
               erfahren, daß Du bereits \label{K_L03239-1v}\edtext{abgereiſt}{\lemma{\textnormal{\emph{abgereiſt}}}\Cendnote{\textnormal{Schnitzler war seit 4. 2. 1906 in Berlin\oindex{Berlin@\textbf{Berlin}, \emph{P.PPLC}|pwk}. Er reiste am 7. 2. 1906 zurück
                  nach Wien\oindex{Wien@\textbf{Wien}, \emph{A.ADM2}|pwk}, wo er am 8. 2. 1906
                  ankam.}}}\label{K_L03239-1} ſeieſt. Es thut mir unendlich leid, Dich heut und geſtern{ }\label{K_L03239-2v}\edtext{verfehlt}{\lemma{\textnormal{\emph{verfehlt}}}\Cendnote{\textnormal{Vgl. A. S.: \emph{Tagebuch}, 6. 2. 1906.
               }}}\label{K_L03239-2} zu haben. Ich danke Dir für Deinen lieben Beſuch, hoffe, Dich bald \label{K_L03239-3v}\edtext{wieder hier\oindex{Berlin@\textbf{Berlin}, \emph{P.PPLC}|pwv}}{\lemma{\textnormal{\emph{wieder hier}}}\Cendnote{\textnormal{Schnitzler traf Goldmann\pwindex{Goldmann, Paul 31.01.1865 – 25.09.1935@\textsc{Goldmann, Paul} (31.01.1865 – 25.09.1935), \emph{Schriftsteller/Schriftstellerin, Journalist/Journalistin}|pwk} am 21. 2. 1906 in
                     Berlin\oindex{Berlin@\textbf{Berlin}, \emph{P.PPLC}|pwk} wieder.}}}\label{K_L03239-3} zu ſehen, und bin mit
               herzlichen Grüßen an Dich und Deine Frau\pwindex{Schnitzler, Olga 17.01.1882 – 13.01.1970@\textsc{Schnitzler, Olga} (17.01.1882 – 13.01.1970), \emph{Schauspieler/Schauspielerin, Sänger/Sängerin}|pwv}{\\}Dein {\\}\spacefill\mbox{Paul Goldmann.}\pend
           \selectlanguage{ngerman}\endnumbering\briefempfaengerindex{Schnitzler, Arthur@\textsc{Schnitzler, Arthur}!zzzGoldmann, Paul@\emph{von Paul Goldmann}!1906-02-071@{7. 2. 1906}|)be}\mylabel{L03239h}  \normalsize

\doendnotes{C}
\bigskip
\vfill

\clearpage

\footnotesize

\lohead{\textsc{register}}

% Definiere theindex-Environment komplett neu ohne reledmac
\makeatletter
\renewenvironment{theindex}{%
  \section*{\indexname}%
  \setlength{\parindent}{0pt}%
  \setlength{\parskip}{0pt plus 0.3pt}%
  \let\item\@idxitem
}{%
  \clearpage
}
\makeatother

\IfFileExists{\jobname-pw.ind}{\input{\jobname-pw.ind}}{}

\end{document}

      