%% latex-leseansicht-vorspann.tex
%% Vorspann für die Leseansicht.
%% Lädt die gemeinsame Datei latex-vorspann.tex mit nicht gesetztem Schalter.

\newif\ifkorrekturansicht
\korrekturansichtfalse

\input{../tex-inputs/latex-vorspann}


\section[ Paul Goldmann an Arthur Schnitzler, 7. 2. 1906]{L03239 Paul Goldmann an Arthur Schnitzler,  7. 2. 1906}
\nopagebreak\mylabel{L03239v}
\rehead{ }\normalsize\beginnumbering\briefempfaengerindex{Schnitzler, Arthur@\textsc{Schnitzler, Arthur}!zzzGoldmann, Paul@\emph{von Paul Goldmann}!1906-02-071@{7. 2. 1906}|(be}
\toendnotes[C]{\smallbreak\pagebreak[2]}
\correspDesc{Versand  durch Paul Goldmann am 7. 2. 1906 in Berlin
\newline{}Erhalt  durch Arthur Schnitzler am 8. 2. 1906 in Wien}\toendnotes[C]{\smallbreak}
\Standort{DLA, A:Schnitzler, HS.NZ85.1.3175.}
\physDesc{Postkarte, 410 Zeichen
\newline{}Handschrift: blaue Tinte, deutsche Kurrent
\newline{}Versand: 1) Stempel: »\nobreak{}\oindex{Berlin@\textbf{Berlin}, \emph{Hauptstadt}|pwk}Berlin, S.W. 11, 7. 2. 06, 10–11N.\nobreak{}«.   2) Stempel: »\nobreak{}\oindex{VIII., Josefstadt@\textbf{VIII., Josefstadt}, \emph{Verwaltungsgebiet}|pwk}18⁄\textsubscript{1} Wi{[}en{]}, 8. II. 06, 5, Bestellt\nobreak{}«. }\toendnotes[C]{\smallbreak}\pstart{}\textsc{{\pb}Herrn}\pend{}\pstart{}\textsc{Dr. Arthur Schnitzler}\pend{}\pstart{}\textsc{Wien\oindex{Wien@\textbf{Wien}, \emph{Verwaltungsgebiet}|pw}}\pend{}\pstart{}\textsc{XVIII. Spöttelgaſse 7\oindex{Wien@\textbf{Wien}!XVIII., Währing@\textbf{XVIII., Währing}!Edmund-Weiß-Gasse 7@\textbf{Edmund-Weiß-Gasse 7}, \emph{Wohngebäude}|pw}.}\pend{}{\bigskip}\vspace{1em}
\pstart
           \noindent{}{\pb}Berlin\oindex{Berlin@\textbf{Berlin}, \emph{Hauptstadt}|pw}, 7. Februar. Lieber Freund, Als ich heut um 5 Uhr im \textsc{Hotel Continental\oindex{Hotel Continental [Berlin]@\textbf{Hotel Continental [Berlin]}, \emph{Hotel}|pw}} vorſprach, mußte ich leider vom Portier\pwindex{?? [Portier des Hotel Continental Berlin] 7.\,2.\,1906 – 7.\,2.\,1906@\textsc{?? [Portier des Hotel Continental Berlin]} (7.\,2.\,1906 – 7.\,2.\,1906)|pw}
               erfahren, daß Du bereits \label{K_L03239-1v}\edtext{abgereiſt}{\lemma{\textnormal{\emph{abgereist}}}\Cendnote{\textnormal{Schnitzler war seit 4. 2. 1906 in Berlin\oindex{Berlin@\textbf{Berlin}, \emph{Hauptstadt}|pwk}. Er reiste am 7. 2. 1906 zurück
                  nach Wien\oindex{Wien@\textbf{Wien}, \emph{Verwaltungsgebiet}|pwk}, wo er am 8. 2. 1906
                  ankam.}}}\label{K_L03239-1}{ }ſeieſt. Es thut mir unendlich leid, Dich heut und geſtern{ }\label{K_L03239-2v}\edtext{verfehlt}{\lemma{\textnormal{\emph{verfehlt}}}\Cendnote{\textnormal{Vgl. A. S.: \emph{Tagebuch}, 6. 2. 1906.
               }}}\label{K_L03239-2} zu haben. Ich danke Dir für Deinen lieben Beſuch, hoffe, Dich bald \label{K_L03239-3v}\edtext{wieder hier\oindex{Berlin@\textbf{Berlin}, \emph{Hauptstadt}|pwv}}{\lemma{\textnormal{\emph{wieder hier}}}\Cendnote{\textnormal{Schnitzler traf Goldmann\pwindex{Goldmann, Paul 31.\,1.\,1865 Breslau – 25.\,9.\,1935 Wien@\textsc{Goldmann, Paul} (31.\,1.\,1865 Breslau – 25.\,9.\,1935 Wien), \emph{Schriftsteller, Journalist}|pwk} am 21. 2. 1906 in
                     Berlin\oindex{Berlin@\textbf{Berlin}, \emph{Hauptstadt}|pwk} wieder.}}}\label{K_L03239-3} zu{ }ſehen, und bin mit
               herzlichen Grüßen an Dich und Deine Frau\pwindex{Schnitzler, Olga 17.\,1.\,1882 Wien – 13.\,1.\,1970 Lugano@\textsc{Schnitzler, Olga} (17.\,1.\,1882 Wien – 13.\,1.\,1970 Lugano), \emph{Schauspielerin, Sängerin}|pwv}{\\}Dein {\\}\spacefill\mbox{Paul Goldmann.}\pend
           \selectlanguage{ngerman}\endnumbering\briefempfaengerindex{Schnitzler, Arthur@\textsc{Schnitzler, Arthur}!zzzGoldmann, Paul@\emph{von Paul Goldmann}!1906-02-071@{7. 2. 1906}|)be}\mylabel{L03239h}  \newcommand{\dateiname}{L03239}\newcommand{\titel}{Paul Goldmann an Arthur Schnitzler, 7. 2. 1906}\newcommand{\editorInnen}{Martin Anton Müller und Laura Untner}%% latex-leseansicht-abspann.tex
%% Abspann für die Leseansicht.
%% Der Schalter \ifkorrekturansicht ist bereits durch den Vorspann gesetzt.

%% latex-abspann.tex
%% Gemeinsamer Abspann für Korrekturansicht und Leseansicht.
%% Setzt den Schalter \ifkorrekturansicht voraus (gesetzt in den
%% einbindenden Dateien latex-korrekturansicht-abspann.tex bzw.
%% latex-leseansicht-abspann.tex).
%% ---------------------------------------------------------------

\normalsize

% Das esempio-Environment wird nur in der Leseansicht benötigt
\ifkorrekturansicht\else
\newenvironment{esempio}[3]%
{
    \vspace{1.5ex}
    \rlap{\underline{#1}}
    \par
    \setlength{\parindent}{0cm}
    \nopagebreak
    \leftskip=#2cm
    \rightskip=#3cm
}
{
    \par
}
\fi

\doendnotes{C}
\bigskip
\vfill

\clearpage

\footnotesize

\ifkorrekturansicht
  \lohead{\textsc{register}}
\fi

% theindex-Environment neu definieren ohne reledmac
\makeatletter
\renewenvironment{theindex}{%
  \ifkorrekturansicht
    \section*{\indexname}%
  \else
    \subsubsection*{Index der erwähnten Entitäten}%
  \fi
  \setlength{\parindent}{0pt}%
  \setlength{\parskip}{0pt plus 0.3pt}%
  \let\item\@idxitem
}{%
  \ifkorrekturansicht\clearpage\fi
}
\makeatother

\IfFileExists{\jobname-pw.ind}{\input{\jobname-pw.ind}}{}

% Quellenangabe nur in der Leseansicht
\ifkorrekturansicht\else
% Fallback-Definitionen, falls die .tex-Datei \titel etc. nicht gesetzt hat
\providecommand{\titel}{}
\providecommand{\editorInnen}{}
\providecommand{\dateiname}{\jobname}

\vspace{3cm}

\vfill

\footnotesize
\textsc{Quelle}: \titel. Herausgegeben von {\editorInnen}. In: \emph{Arthur Schnitzler: Briefwechsel mit Autorinnen und Autoren}.
 Digitale Edition, https://schnitzler-briefe.acdh.oeaw.ac.at/{\dateiname}.html (Stand \today)
\fi

\end{document}


