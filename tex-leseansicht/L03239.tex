%% latex-leseansicht-vorspann.tex
%% Vorspann für die Leseansicht.
%% Lädt die gemeinsame Datei latex-vorspann.tex mit nicht gesetztem Schalter.

\newif\ifkorrekturansicht
\korrekturansichtfalse

\input{../tex-inputs/latex-vorspann}


         
         \renewcommand{\erwaehntePersonen}{Personen:  ?? [Portier des Hotel Continental Berlin], Olga Schnitzler}
         \renewcommand{\erwaehnteOrte}{Orte: Berlin, Edmund-Weiß-Gasse, Hotel Continental (Berlin), VIII., Josefstadt, Wien}
         \renewcommand{\erwaehnteWerke}{}
               \section[ Paul Goldmann an Arthur Schnitzler, 7. 2. 1906]{ Paul Goldmann an Arthur Schnitzler, 7. 2. 1906}\nopagebreak\mylabel{v}\rehead{ }\begin{ledgroupsized}[t]{13cm}\normalsize\beginnumbering \toendnotes[C]{\smallbreak\pagebreak[2]} \Standort{DLA, A:Schnitzler, HS.NZ85.1.3175.}
\physDesc{Postkarte, 410 Zeichen
\newline{}Handschrift: 1) blaue Tinte, deutsche Kurrent\hspace{1em}2) blaue Tinte, lateinische Kurrent (\noindent{}Adresse)\hspace{1em}
\newline{}Versand: 1) Stempel: »\nobreak{}\oindex{Berlin@\textbf{Berlin}|pwk}Berlin, S.W. 11, 7. 2. 06, 10–11N.\nobreak{}«.   2) Stempel: »\nobreak{}\oindex{VIII., Josefstadt@\textbf{VIII., Josefstadt}|pwk}18⁄\textsubscript{1} Wi{[}en{]}, 8. II. 06, 5, Bestellt\nobreak{}«. }\toendnotes[C]{\smallbreak}\pstart{}{\pb}Herrn\pend{}\pstart{}Dr. Arthur Schnitzler\pend{}\pstart{}Wien\oindex{Wien@\textbf{Wien}|pw}\pend{}\pstart{}XVIII. Spöttelgaſse 7\oindex{Edmund-Weiss-Gasse@\textbf{Edmund-Weiß-Gasse}|pw}.\pend{}{\bigskip}\pstart
           \noindent{}{\pb}Berlin\oindex{Berlin@\textbf{Berlin}|pw}, 7. Februar. Lieber Freund, Als ich heut um 5 Uhr im \textsc{Hotel Continental\oindex{Hotel Continental (Berlin)@\textbf{Hotel Continental (Berlin)}|pw}} vorſprach, mußte ich leider vom Portier\pwindex{?? [Portier des Hotel Continental Berlin] 1906-02-07 – 1906-02-07@\textsc{?? [Portier des Hotel Continental Berlin]} (1906-02-07 – 1906-02-07)|pw}
               erfahren, daß Du bereits \label{K-L03239-1v}\edtext{abgereiſt}{\lemma{\textnormal{\emph{abgereiſt}}}\Cendnote{\textnormal{Schnitzler\pwindex{Schnitzler, Arthur 15.05.1862 – 21.10.1931@\textsc{Schnitzler, Arthur} (15.05.1862 – 21.10.1931), \emph{Schriftsteller, Mediziner}|pwk} war seit 4. 2. 1906 in Berlin\oindex{Berlin@\textbf{Berlin}|pwk}. Er reiste am 7. 2. 1906 zurück
                  nach Wien\oindex{Wien@\textbf{Wien}|pwk}, wo er am 8. 2. 1906
                  ankam.}}}\label{K-L03239-1h} ſeieſt. Es thut mir unendlich leid, Dich heut und geſtern{ }\label{K-L03239-2v}\edtext{verfehlt}{\lemma{\textnormal{\emph{verfehlt}}}\Cendnote{\textnormal{vgl. A. S.: \emph{Tagebuch}, 6. 2. 1906}}}\label{K-L03239-2h} zu haben. Ich danke Dir für Deinen lieben Beſuch, hoffe, Dich bald \label{K-L03239-3v}\edtext{wieder hier\oindex{Berlin@\textbf{Berlin}|pwv}}{\lemma{\textnormal{\emph{wieder hier}}}\Cendnote{\textnormal{Schnitzler\pwindex{Schnitzler, Arthur 15.05.1862 – 21.10.1931@\textsc{Schnitzler, Arthur} (15.05.1862 – 21.10.1931), \emph{Schriftsteller, Mediziner}|pwk} traf Goldmann\pwindex{Goldmann, Paul 31.01.1865 – 25.09.1935@\textsc{Goldmann, Paul} (31.01.1865 – 25.09.1935), \emph{Schriftsteller, Journalist}|pwk} am 21. 2. 1906 in
                     Berlin\oindex{Berlin@\textbf{Berlin}|pwk} wieder.}}}\label{K-L03239-3h} zu ſehen, und bin mit
               herzlichen Grüßen an Dich und Deine Frau\pwindex{Schnitzler, Olga 17.01.1882 – 13.01.1970@\textsc{Schnitzler, Olga} (17.01.1882 – 13.01.1970), \emph{Schauspielerin, Sängerin}|pwv}{\\}Dein {\\}\spacefill\mbox{Paul Goldmann.}\pend
           
         
         \endnumbering\mylabel{h}\end{ledgroupsized}  \newcommand{\dateiname}{L03239}\newcommand{\titel}{Paul Goldmann an Arthur Schnitzler, 7. 2. 1906}\newcommand{\editorInnen}{Martin Anton Müller und Laura Untner}%% latex-leseansicht-abspann.tex
%% Abspann für die Leseansicht.
%% Der Schalter \ifkorrekturansicht ist bereits durch den Vorspann gesetzt.

%% latex-abspann.tex
%% Gemeinsamer Abspann für Korrekturansicht und Leseansicht.
%% Setzt den Schalter \ifkorrekturansicht voraus (gesetzt in den
%% einbindenden Dateien latex-korrekturansicht-abspann.tex bzw.
%% latex-leseansicht-abspann.tex).
%% ---------------------------------------------------------------

\normalsize

% Das esempio-Environment wird nur in der Leseansicht benötigt
\ifkorrekturansicht\else
\newenvironment{esempio}[3]%
{
    \vspace{1.5ex}
    \rlap{\underline{#1}}
    \par
    \setlength{\parindent}{0cm}
    \nopagebreak
    \leftskip=#2cm
    \rightskip=#3cm
}
{
    \par
}
\fi

\doendnotes{C}
\bigskip
\vfill

\clearpage

\footnotesize

\ifkorrekturansicht
  \lohead{\textsc{register}}
\fi

% theindex-Environment neu definieren ohne reledmac
\makeatletter
\renewenvironment{theindex}{%
  \ifkorrekturansicht
    \section*{\indexname}%
  \else
    \subsubsection*{Index der erwähnten Entitäten}%
  \fi
  \setlength{\parindent}{0pt}%
  \setlength{\parskip}{0pt plus 0.3pt}%
  \let\item\@idxitem
}{%
  \ifkorrekturansicht\clearpage\fi
}
\makeatother

\IfFileExists{\jobname-pw.ind}{\input{\jobname-pw.ind}}{}

% Quellenangabe nur in der Leseansicht
\ifkorrekturansicht\else
% Fallback-Definitionen, falls die .tex-Datei \titel etc. nicht gesetzt hat
\providecommand{\titel}{}
\providecommand{\editorInnen}{}
\providecommand{\dateiname}{\jobname}

\vspace{3cm}

\vfill

\footnotesize
\textsc{Quelle}: \titel. Herausgegeben von {\editorInnen}. In: \emph{Arthur Schnitzler: Briefwechsel mit Autorinnen und Autoren}.
 Digitale Edition, https://schnitzler-briefe.acdh.oeaw.ac.at/{\dateiname}.html (Stand \today)
\fi

\end{document}


      