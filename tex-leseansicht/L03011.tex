%% latex-leseansicht-vorspann.tex
%% Vorspann für die Leseansicht.
%% Lädt die gemeinsame Datei latex-vorspann.tex mit nicht gesetztem Schalter.

\newif\ifkorrekturansicht
\korrekturansichtfalse

\input{../tex-inputs/latex-vorspann}

\begin{center}
            \textcolor{red}{ENTWURF, NICHT FERTIG KORRIGIERT}
                      \end{center}
            
         
         \renewcommand{\erwaehntePersonen}{Personen: Richard Beer-Hofmann, Samuel Fischer, Anna Katharina Rehmann, Felix Salten, Ottilie Salten, Paul Salten, Olga Schnitzler, Heinrich Schnitzler, Louise Schnitzler}
         \renewcommand{\erwaehnteInstitutionen}{Institutionen: Franz-Grillparzer-Preis}
         \renewcommand{\erwaehnteOrte}{Orte: Semmering, Wien}
         \renewcommand{\erwaehnteWerke}{Werke: Der Weg ins Freie. Roman}
               \section[ Arthur Schnitzler an Felix Salten, 25. 1. 1908]{ Arthur Schnitzler an Felix Salten, 25. 1. 1908}\nopagebreak\mylabel{v}\rehead{ }\begin{ledgroupsized}[t]{13cm}\normalsize\beginnumbering \toendnotes[C]{\smallbreak\pagebreak[2]} \Standort{Wienbibliothek im Rathaus, ZPH 1681, 2.1.516.}
\physDesc{Brief, 1 Blatt, 4 Seiten, 2049 Zeichen
\newline{}Handschrift: blaue Tinte, lateinische Kurrent
\newline{}Ordnung: mit Bleistift von unbekannter Hand Nummerierung der Blätter des Konvoluts:
                                    »6« }\toendnotes[C]{\smallbreak}\pstart
           \raggedleft{}{\pb}25. 1. 908\pend
           \pstart
           lieber, es ist \label{K_L03011-1v}\edtext{desinfizirt}{\lemma{\textnormal{\emph{desinfizirt}}}\Cendnote{\textnormal{siehe Felix Salten an Arthur Schnitzler, [10. 12. 1907]}}}\label{K_L03011-1h}, Wohnung, Kleider, Olga\pwindex{Schnitzler, Olga 17.01.1882 – 13.01.1970@\textsc{Schnitzler, Olga} (17.01.1882 – 13.01.1970), \emph{Schauspielerin, Sängerin}|pw} ist meist außer
               Bett, also die Zustände sind annähernd zur Norm zurückgekehrt. Der \label{K_L03011-2v}\edtext{Bub\pwindex{Schnitzler, Heinrich 09.08.1902 – 12.07.1982@\textsc{Schnitzler, Heinrich} (09.08.1902 – 12.07.1982), \emph{Regisseur, Schauspieler}|pwv} ist noch nicht
                  daheim}{\lemma{\textnormal{\emph{Bub … daheim}}}\Cendnote{\textnormal{Heinrich\pwindex{Schnitzler, Heinrich 09.08.1902 – 12.07.1982@\textsc{Schnitzler, Heinrich} (09.08.1902 – 12.07.1982), \emph{Regisseur, Schauspieler}|pwk} war während der Erkrankung seiner
                     Mutter\pwindex{Schnitzler, Olga 17.01.1882 – 13.01.1970@\textsc{Schnitzler, Olga} (17.01.1882 – 13.01.1970), \emph{Schauspielerin, Sängerin}|pwkv} bei seiner Großmutter\pwindex{Schnitzler, Louise 1840-07-08 – 1911-09-09@\textsc{Schnitzler, Louise} (1840-07-08 – 1911-09-09)|pwkv}
                  väterlicherseits.}}}\label{K_L03011-2h}, doch hab ich mit ihm Zusa{\geminationm}enkünfte, auch macht er uns \label{K_L03011-3v}\edtext{Fensterpromenaden}{\lemma{\textnormal{\emph{Fensterpromenaden}}}\Cendnote{\textnormal{Heißt: Er spaziert
                  am Fenster vorbei und winkt seiner Mutter\pwindex{Schnitzler, Olga 17.01.1882 – 13.01.1970@\textsc{Schnitzler, Olga} (17.01.1882 – 13.01.1970), \emph{Schauspielerin, Sängerin}|pwkv}, die weiterhin in Quarantäne ist.}}}\label{K_L03011-3h}. Wir wollen
               in etwa 10 Tagen, bis Olga\pwindex{Schnitzler, Olga 17.01.1882 – 13.01.1970@\textsc{Schnitzler, Olga} (17.01.1882 – 13.01.1970), \emph{Schauspielerin, Sängerin}|pw} ganz gehtüchtig und
               die Influenzagerüchte – oder -wahrheiten vom Semmering\oindex{Semmering@\textbf{Semmering}|pw} geschwunden sind, \label{K_L03011-4v}\edtext{auf besagten Südbahngipfel\oindex{Semmering@\textbf{Semmering}|pwv}
                  reisen}{\lemma{\textnormal{\emph{auf … reisen}}}\Cendnote{\textnormal{Arthur\pwindex{Schnitzler, Arthur 15.05.1862 – 21.10.1931@\textsc{Schnitzler, Arthur} (15.05.1862 – 21.10.1931), \emph{Schriftsteller, Mediziner}|pwk} und Olga Schnitzler\pwindex{Schnitzler, Olga 17.01.1882 – 13.01.1970@\textsc{Schnitzler, Olga} (17.01.1882 – 13.01.1970), \emph{Schauspielerin, Sängerin}|pwk} reisten am 4. 2. 1908 auf den Semmering\oindex{Semmering@\textbf{Semmering}|pwk} und trafen dabei im Zug auf Salten\pwindex{Salten, Felix 06.09.1869 – 08.10.1945@\textsc{Salten, Felix} (06.09.1869 – 08.10.1945), \emph{Schriftsteller, Journalist}|pwk}. Am 22. 2. 1908 reisten sie zurück nach Wien\oindex{Wien@\textbf{Wien}|pwk}.}}}\label{K_L03011-4h} und dort mit Heini\pwindex{Schnitzler, Heinrich 09.08.1902 – 12.07.1982@\textsc{Schnitzler, Heinrich} (09.08.1902 – 12.07.1982), \emph{Regisseur, Schauspieler}|pw} etwa 8 Tage verbringen. Dies unser Progra{\geminationm}. Da{\geminationn} erst gedenk ich
               Freundes- und andre Häuser wieder zu betreten und das {\pb}unsre zu eröffnen.\pend
           \pstart
           Trotzdem möcht ich Sie gerne sehen{[},{]} früher sehen; – we{\geminationn} Sie nicht (was ich Ihnen beim Hi{\geminationm}el keinen Moment lang verübeln kö{\geminationn}te!) zu ängstlich sind. Jedenfalls schreiben Sie mir
               zum Trost\textcolor{gray}{,} wie es Ihnen Allen geht; von Richard\pwindex{Beer-Hofmann, Richard 1866-07-11 – 1945-09-26@\textsc{Beer-Hofmann, Richard} (1866-07-11 – 1945-09-26), \emph{Schriftsteller}|pw} hört ich, dass Sie sich noch i{\geminationm}er nicht ganz wohl befinden.\pend
           \pstart
           Hinsichtlich des Vorausdrucks des Roman\pwindex{Schnitzler, Arthur 15.05.1862 – 21.10.1931@\textsc{Schnitzler, Arthur} (15.05.1862 – 21.10.1931), \emph{Schriftsteller, Mediziner}!Weg ins Freie. Roman1.1.1908 – 1.6.1908@\strich\emph{Der Weg ins Freie. Roman} {[}1.1.1908 – 1.6.1908{]}|pwv}s hab ich mit Fischer\pwindex{Fischer, Samuel 24.12.1859 – 15.10.1934@\textsc{Fischer, Samuel} (24.12.1859 – 15.10.1934), \emph{Verleger}|pw} schon vor
               Monaten correspondirt; aus \textcolor{gray}{irgend}welchen techn. Gründen läßt sich
               die Sache nicht machen. Ich habe in den letzten Wochen noch viel daran corrigirt, so
               daß die Manuscripte immer ungastlicher aussehen, überdies werden Sie lieber kein {\pb}Papierconvolut aus unsrer Wohnung in Ihre
               hinübernehmen wollen – was bleibt mir also übrig? Sie bitten, das Ding\pwindex{Schnitzler, Arthur 15.05.1862 – 21.10.1931@\textsc{Schnitzler, Arthur} (15.05.1862 – 21.10.1931), \emph{Schriftsteller, Mediziner}!Weg ins Freie. Roman1.1.1908 – 1.6.1908@\strich\emph{Der Weg ins Freie. Roman} {[}1.1.1908 – 1.6.1908{]}|pwv} nicht in Forsetzungen zu lesen,
               sondern warten, bis das Buch\pwindex{Schnitzler, Arthur 15.05.1862 – 21.10.1931@\textsc{Schnitzler, Arthur} (15.05.1862 – 21.10.1931), \emph{Schriftsteller, Mediziner}!Weg ins Freie. Roman1.1.1908 – 1.6.1908@\strich\emph{Der Weg ins Freie. Roman} {[}1.1.1908 – 1.6.1908{]}|pwv}
               da ist, um es, womöglich an einem – zwei schönen Sommertagen in \uline{einem} Zug (eventuell auch in einem \uline{Zug},
               aber besser, im Freien) hinunterzuschlucken. Der Nachgeschmack wird kein übler sein;
               heut trau ich mich es zu sagen.–\pend
           \pstart
           Ich danke Ihnen sehr für Ihre lieben Grillparzerpreis\orgindex{Franz-Grillparzer-Preis@Franz-Grillparzer-Preis|pw}glückwünsche. Anfangs war ich sehr erstaunt, da{\geminationn} eher (aus allerlei, complicirten und oberflächlichen
               Gründen) heruntergesti{\geminationm}t – jetzt überwiegt die Freude,
               woran die {\pb}\label{K_L03011-5v}\edtext{5 Mille}{\lemma{\textnormal{\emph{5 Mille}}}\Cendnote{\textnormal{5000 Kronen im Jahr 1908 entsprechen
                     2023 etwa 38.000 Euro.}}}\label{K_L03011-5h} nicht ganz unbetheiligt
               sind. Nach dem Arbeiten sehn ich mich, hab manches vorbereitet und \substVorne{}\textsuperscript{\textcolor{gray}{au}}\substDazwischen{}bin\substHinten{} neugierig, was zuerst fertig sein wird. So stellt man sich frech wieder
               mitten ins Leben hinein.\pend
           \pstart
           Seien Sie, Otti\pwindex{Salten, Ottilie 07.03.1868 – 22.06.1942@\textsc{Salten, Ottilie} (07.03.1868 – 22.06.1942), \emph{Schauspielerin}|pw} und die Kinder\pwindex{Salten, Paul 11.08.1903 – 08.05.1937@\textsc{Salten, Paul} (11.08.1903 – 08.05.1937), \emph{Filmcutter}|pwv}\pwindex{Rehmann, Anna Katharina 18.08.1904 – 27.03.1977@\textsc{Rehmann, Anna Katharina} (18.08.1904 – 27.03.1977), \emph{Schauspielerin, Übersetzerin}|pwv} herzlichst gegrüßt und
               lassen mindestens was von sich \uline{hören}. Auch von Olga\pwindex{Schnitzler, Olga 17.01.1882 – 13.01.1970@\textsc{Schnitzler, Olga} (17.01.1882 – 13.01.1970), \emph{Schauspielerin, Sängerin}|pw} alles schöne.\pend
           \pstart
           Ihr {\\[\baselineskip]}\spacefill\mbox{Arthur}\pend
           \leftskip=0em{}
         
         \endnumbering\mylabel{h}\end{ledgroupsized}  \newcommand{\dateiname}{L03011}\newcommand{\titel}{Arthur Schnitzler an Felix Salten, 25. 1. 1908}\newcommand{\editorInnen}{Martin Anton Müller und Laura Untner}%% latex-leseansicht-abspann.tex
%% Abspann für die Leseansicht.
%% Der Schalter \ifkorrekturansicht ist bereits durch den Vorspann gesetzt.

%% latex-abspann.tex
%% Gemeinsamer Abspann für Korrekturansicht und Leseansicht.
%% Setzt den Schalter \ifkorrekturansicht voraus (gesetzt in den
%% einbindenden Dateien latex-korrekturansicht-abspann.tex bzw.
%% latex-leseansicht-abspann.tex).
%% ---------------------------------------------------------------

\normalsize

% Das esempio-Environment wird nur in der Leseansicht benötigt
\ifkorrekturansicht\else
\newenvironment{esempio}[3]%
{
    \vspace{1.5ex}
    \rlap{\underline{#1}}
    \par
    \setlength{\parindent}{0cm}
    \nopagebreak
    \leftskip=#2cm
    \rightskip=#3cm
}
{
    \par
}
\fi

\doendnotes{C}
\bigskip
\vfill

\clearpage

\footnotesize

\ifkorrekturansicht
  \lohead{\textsc{register}}
\fi

% theindex-Environment neu definieren ohne reledmac
\makeatletter
\renewenvironment{theindex}{%
  \ifkorrekturansicht
    \section*{\indexname}%
  \else
    \subsubsection*{Index der erwähnten Entitäten}%
  \fi
  \setlength{\parindent}{0pt}%
  \setlength{\parskip}{0pt plus 0.3pt}%
  \let\item\@idxitem
}{%
  \ifkorrekturansicht\clearpage\fi
}
\makeatother

\IfFileExists{\jobname-pw.ind}{\input{\jobname-pw.ind}}{}

% Quellenangabe nur in der Leseansicht
\ifkorrekturansicht\else
% Fallback-Definitionen, falls die .tex-Datei \titel etc. nicht gesetzt hat
\providecommand{\titel}{}
\providecommand{\editorInnen}{}
\providecommand{\dateiname}{\jobname}

\vspace{3cm}

\vfill

\footnotesize
\textsc{Quelle}: \titel. Herausgegeben von {\editorInnen}. In: \emph{Arthur Schnitzler: Briefwechsel mit Autorinnen und Autoren}.
 Digitale Edition, https://schnitzler-briefe.acdh.oeaw.ac.at/{\dateiname}.html (Stand \today)
\fi

\end{document}


      