%% latex-leseansicht-vorspann.tex
%% Vorspann für die Leseansicht.
%% Lädt die gemeinsame Datei latex-vorspann.tex mit nicht gesetztem Schalter.

\newif\ifkorrekturansicht
\korrekturansichtfalse

\input{../tex-inputs/latex-vorspann}


\section[ Arthur Schnitzler an Felix Salten, 25. 1. 1908]{L03011 Arthur Schnitzler an Felix Salten,  25. 1. 1908}
\nopagebreak\mylabel{L03011v}
\rehead{ }\normalsize\beginnumbering\briefempfaengerindex{Salten, Felix@\textsc{Salten, Felix}!zzzSchnitzler, Arthur@\emph{von Arthur Schnitzler}!1908-01-252@{25. 1. 1908}|(be}
\toendnotes[C]{\smallbreak\pagebreak[2]}
\correspDesc{Versand  durch Arthur Schnitzler am 25. 1. 1908 in Wien
\newline{}Erhalt  durch Felix Salten am [26. 1. 1908] in Semmering}\toendnotes[C]{\smallbreak}
\Standort{Wienbibliothek im Rathaus, ZPH 1681, 2.1.516.}
\physDesc{Brief, 1 Blatt, 4 Seiten, 2045 Zeichen
\newline{}Handschrift: blaue Tinte, lateinische Kurrent
\newline{}Ordnung: mit Bleistift von unbekannter Hand nummeriert: »6« }\toendnotes[C]{\smallbreak}
\pstart
           \raggedleft{}{\pb}25. 1. 908\pend
           \vspace{0.5em}
\pstart
           lieber, es ist \label{K_L03011-1v}\edtext{desinfizirt}{\lemma{\textnormal{\emph{desinfizirt}}}\Cendnote{\textnormal{Siehe XXXX Auszeichnungsfehler: Dokument L03494 nicht gefunden.
               }}}\label{K_L03011-1}, Wohnung, Kleider, Olga\pwindex{Schnitzler, Olga 17.\,1.\,1882 Wien – 13.\,1.\,1970 Lugano@\textsc{Schnitzler, Olga} (17.\,1.\,1882 Wien – 13.\,1.\,1970 Lugano), \emph{Schauspielerin, Sängerin}|pw} ist meist außer
               Bett, also die Zustände sind annähernd zur Norm zurückgekehrt. Der \label{K_L03011-2v}\edtext{Bub\pwindex{Schnitzler, Heinrich 9.\,8.\,1902 Hinterbrühl – 12.\,7.\,1982 Wien@\textsc{Schnitzler, Heinrich} (9.\,8.\,1902 Hinterbrühl – 12.\,7.\,1982 Wien), \emph{Regisseur, Schauspieler}|pwv} ist noch nicht
                  daheim}{\lemma{\textnormal{\emph{Bub … daheim}}}\Cendnote{\textnormal{Heinrich\pwindex{Schnitzler, Heinrich 9.\,8.\,1902 Hinterbrühl – 12.\,7.\,1982 Wien@\textsc{Schnitzler, Heinrich} (9.\,8.\,1902 Hinterbrühl – 12.\,7.\,1982 Wien), \emph{Regisseur, Schauspieler}|pwk} war während der Erkrankung seiner
                     Mutter\pwindex{Schnitzler, Olga 17.\,1.\,1882 Wien – 13.\,1.\,1970 Lugano@\textsc{Schnitzler, Olga} (17.\,1.\,1882 Wien – 13.\,1.\,1970 Lugano), \emph{Schauspielerin, Sängerin}|pwkv} bei Schnitzlers Mutter Louise\pwindex{Schnitzler, Louise 8.\,7.\,1840 Kőszeg – 9.\,9.\,1911 Wien@\textsc{Schnitzler, Louise} (8.\,7.\,1840 Kőszeg – 9.\,9.\,1911 Wien)|pwk}.}}}\label{K_L03011-2}, doch hab ich mit ihm Zusa{\geminationm}enkünfte, auch macht er uns \label{K_L03011-3v}\edtext{Fensterpromenaden}{\lemma{\textnormal{\emph{Fensterpromenaden}}}\Cendnote{\textnormal{Eigentlich wird damit die Praxis von Verliebten bezeichnet, in der Hoffnung, von der Angebeteten gesehen zu werden, 
                        wenn man vor dem Fenster vorbeispaziert. Für Heinrich\pwindex{Schnitzler, Heinrich 9.\,8.\,1902 Hinterbrühl – 12.\,7.\,1982 Wien@\textsc{Schnitzler, Heinrich} (9.\,8.\,1902 Hinterbrühl – 12.\,7.\,1982 Wien), \emph{Regisseur, Schauspieler}|pwk} war das
                  während Olgas\pwindex{Schnitzler, Olga 17.\,1.\,1882 Wien – 13.\,1.\,1970 Lugano@\textsc{Schnitzler, Olga} (17.\,1.\,1882 Wien – 13.\,1.\,1970 Lugano), \emph{Schauspielerin, Sängerin}|pwk} Scharlacherkrankung die
                  einzige Möglichkeit, seine Mutter zu sehen.}}}\label{K_L03011-3}. Wir wollen in etwa 10 Tagen,
               bis Olga\pwindex{Schnitzler, Olga 17.\,1.\,1882 Wien – 13.\,1.\,1970 Lugano@\textsc{Schnitzler, Olga} (17.\,1.\,1882 Wien – 13.\,1.\,1970 Lugano), \emph{Schauspielerin, Sängerin}|pw} ganz gehtüchtig und die
               Influenzagerüchte – oder -wahrheiten vom Semmering\oindex{Semmering@\textbf{Semmering}, \emph{Verwaltungsgebiet}|pw} geschwunden sind, \label{K_L03011-4v}\edtext{auf besagten Südbahngipfel\oindex{Semmering@\textbf{Semmering}, \emph{Verwaltungsgebiet}|pwv}
                  reisen}{\lemma{\textnormal{\emph{auf … reisen}}}\Cendnote{\textnormal{Arthur und Olga Schnitzler\pwindex{Schnitzler, Olga 17.\,1.\,1882 Wien – 13.\,1.\,1970 Lugano@\textsc{Schnitzler, Olga} (17.\,1.\,1882 Wien – 13.\,1.\,1970 Lugano), \emph{Schauspielerin, Sängerin}|pwk} reisten am 4. 2. 1908 auf den Semmering\oindex{Semmering@\textbf{Semmering}, \emph{Verwaltungsgebiet}|pwk} und trafen dabei im Zug auf Salten\pwindex{Salten, Felix 6.\,9.\,1869 Budapest – 8.\,10.\,1945 Zürich@\textsc{Salten, Felix} (6.\,9.\,1869 Budapest – 8.\,10.\,1945 Zürich), \emph{Schriftsteller, Journalist, Chefredakteur}|pwk}. Am 22. 2. 1908 reisten sie zurück nach Wien\oindex{Wien@\textbf{Wien}, \emph{Verwaltungsgebiet}|pwk}.}}}\label{K_L03011-4} und dort mit Heini\pwindex{Schnitzler, Heinrich 9.\,8.\,1902 Hinterbrühl – 12.\,7.\,1982 Wien@\textsc{Schnitzler, Heinrich} (9.\,8.\,1902 Hinterbrühl – 12.\,7.\,1982 Wien), \emph{Regisseur, Schauspieler}|pw} etwa 8 Tage verbringen. Dies unser Progra{\geminationm}. Da{\geminationn} erst gedenk ich
               Freundes- und andre Häuser wieder zu betreten und das {\pb}unsre zu eröffnen.\pend
           
\pstart
           Trotzdem möcht ich Sie gerne sehen\textcolor{gray}{,} früher sehen; – we{\geminationn} Sie nicht (was ich Ihnen beim Hi{\geminationm}el keinen Moment lang verübeln kö{\geminationn}te!) zu ängstlich sind. Jedenfalls schreiben Sie mir
               zum Trost\textcolor{gray}{,} wie es Ihnen Allen geht; \label{K_L03011-5v}\edtext{von Richard\pwindex{Beer-Hofmann, Richard 11.\,7.\,1866 Wien – 26.\,9.\,1945 New York City@\textsc{Beer-Hofmann, Richard} (11.\,7.\,1866 Wien – 26.\,9.\,1945 New York City), \emph{Schriftsteller}|pw} hört
                  ich}{\lemma{\textnormal{\emph{von Richard hört
                  ich}}}\Cendnote{\textnormal{Schnitzler erfuhr das vermutlich beim gemeinsamen Spaziergang am 23. 1. 1908.}}}\label{K_L03011-5},
               dass Sie sich noch i{\geminationm}er nicht ganz wohl befinden.\pend
           
\pstart
           Hinsichtlich des Vorausdrucks des Romans\pwindex{Schnitzler, Arthur 15.\,5.\,1862 Wien – 21.\,10.\,1931 ebd.@\textsc{Schnitzler, Arthur} (15.\,5.\,1862 Wien – 21.\,10.\,1931 ebd.), \emph{Schriftsteller, Mediziner}!Weg ins Freie. Roman@\strich\emph{Der Weg ins Freie. Roman}|pwv} hab ich \label{K_L03011-6v}\edtext{mit Fischer\pwindex{Fischer, Samuel 24.\,12.\,1859 Liptovský Mikuláš – 15.\,10.\,1934 Berlin@\textsc{Fischer, Samuel} (24.\,12.\,1859 Liptovský Mikuláš – 15.\,10.\,1934 Berlin), \emph{Verleger}|pw} schon vor
               Monaten correspondirt}{\lemma{\textnormal{\emph{mit … correspondirt}}}\Cendnote{\textnormal{Siehe 
                  Samuel Fischer\pwindex{Fischer, Samuel 24.\,12.\,1859 Liptovský Mikuláš – 15.\,10.\,1934 Berlin@\textsc{Fischer, Samuel} (24.\,12.\,1859 Liptovský Mikuláš – 15.\,10.\,1934 Berlin), \emph{Verleger}|pwk}, Hedwig Fischer\pwindex{Fischer, Hedwig 8.\,9.\,1871 Szczecin – 11.\,4.\,1952 Königstein im Taunus@\textsc{Fischer, Hedwig} (8.\,9.\,1871 Szczecin – 11.\,4.\,1952 Königstein im Taunus)|pwk}: \emph{Briefwechsel mit
                     Autoren}. Herausgegeben von Dierk Rodewald und Corinna Fiedler. Mit
                  einer Einführung von Bernhard Zeller. Frankfurt am Main:
                  \emph{S. Fischer}{ }1989, S. 76–77.}}}\label{K_L03011-6}; aus irgen\textcolor{gray}{d}welchen techn. Gründen läßt sich
               die Sache nicht machen. Ich habe in den letzten Wochen noch viel daran corrigirt, so
               daß die Manuscripte immer ungastlicher aussehen, überdies werden Sie lieber kein {\pb}Papierconvolut aus unsrer Wohnung in Ihre
               hinübernehmen wollen – was bleibt mir also übrig? Sie bitten, das Ding\pwindex{Schnitzler, Arthur 15.\,5.\,1862 Wien – 21.\,10.\,1931 ebd.@\textsc{Schnitzler, Arthur} (15.\,5.\,1862 Wien – 21.\,10.\,1931 ebd.), \emph{Schriftsteller, Mediziner}!Weg ins Freie. Roman@\strich\emph{Der Weg ins Freie. Roman}|pwv} nicht in Forsetzungen zu lesen,
               sondern warten, bis das Buch\pwindex{Schnitzler, Arthur 15.\,5.\,1862 Wien – 21.\,10.\,1931 ebd.@\textsc{Schnitzler, Arthur} (15.\,5.\,1862 Wien – 21.\,10.\,1931 ebd.), \emph{Schriftsteller, Mediziner}!Weg ins Freie. Roman@\strich\emph{Der Weg ins Freie. Roman}|pwv}
               da ist, um es, womöglich an einem – zwei schönen Sommertagen in \uline{einem} Zug (eventuell auch in einem \uline{Zug},
               aber besser, im Freien) hinunterzuschlucken. Der Nachgeschmack wird kein übler sein;
               heut trau ich mich es zu sagen. –\pend
           
\pstart
           Ich danke Ihnen sehr für Ihre lieben Grillparzerpreis\orgindex{Franz-Grillparzer-Preis@Franz-Grillparzer-Preis|pw}glückwünsche. Anfangs war ich sehr erstaunt, da{\geminationn} eher (aus allerlei, complicirten und oberflächlichen
               Gründen) heruntergesti{\geminationm}t – jetzt überwiegt die Freude,
               woran die {\pb}\label{K_L03011-7v}\edtext{5 Mille}{\lemma{\textnormal{\emph{5 Mille}}}\Cendnote{\textnormal{Schnitzler verwendet
                  das italienische Wort »mille« für tausend. Das Preisgeld von 5000 Kronen im Jahr
                     1908 entspricht 2024 etwa
                  40.000 Euro.}}}\label{K_L03011-7} nicht ganz unbetheiligt sind. Nach dem Arbeiten sehn ich mich,
               hab manches vorbereitet und \substVorne{}\textsuperscript{\textcolor{gray}{au}}\substDazwischen{}bin\substHinten{} neugierig, was zuerst fertig sein wird. So stellt man sich frech wieder
               mitten ins Leben hinein.\pend
           
\pstart
           Seien Sie, Otti\pwindex{Salten, Ottilie 7.\,3.\,1868 Prag – 22.\,6.\,1942 Zürich@\textsc{Salten, Ottilie} (7.\,3.\,1868 Prag – 22.\,6.\,1942 Zürich), \emph{Schauspielerin}|pw} und die Kinder\pwindex{Salten, Paul 11.\,8.\,1903 Wien – 8.\,5.\,1937 ebd.@\textsc{Salten, Paul} (11.\,8.\,1903 Wien – 8.\,5.\,1937 ebd.), \emph{Filmcutter}|pwv}\pwindex{Rehmann, Anna Katharina 18.\,8.\,1904 Wien – 27.\,3.\,1977 Zürich@\textsc{Rehmann, Anna Katharina} (18.\,8.\,1904 Wien – 27.\,3.\,1977 Zürich), \emph{Schauspielerin, Übersetzerin}|pwv} herzlichst gegrüßt und
               lassen mindestens was von sich \uline{hören}. Auch von Olga\pwindex{Schnitzler, Olga 17.\,1.\,1882 Wien – 13.\,1.\,1970 Lugano@\textsc{Schnitzler, Olga} (17.\,1.\,1882 Wien – 13.\,1.\,1970 Lugano), \emph{Schauspielerin, Sängerin}|pw} alles schöne.\pend
           
\pstart
           Ihr {\\[\baselineskip]}\spacefill\mbox{Arthur}\pend
           \leftskip=0em{}\selectlanguage{ngerman}\endnumbering\briefempfaengerindex{Salten, Felix@\textsc{Salten, Felix}!zzzSchnitzler, Arthur@\emph{von Arthur Schnitzler}!1908-01-252@{25. 1. 1908}|)be}\mylabel{L03011h}  \newcommand{\dateiname}{L03011}\newcommand{\titel}{Arthur Schnitzler an Felix Salten, 25. 1. 1908}\newcommand{\editorInnen}{Martin Anton Müller und Laura Untner}%% latex-leseansicht-abspann.tex
%% Abspann für die Leseansicht.
%% Der Schalter \ifkorrekturansicht ist bereits durch den Vorspann gesetzt.

%% latex-abspann.tex
%% Gemeinsamer Abspann für Korrekturansicht und Leseansicht.
%% Setzt den Schalter \ifkorrekturansicht voraus (gesetzt in den
%% einbindenden Dateien latex-korrekturansicht-abspann.tex bzw.
%% latex-leseansicht-abspann.tex).
%% ---------------------------------------------------------------

\normalsize

% Das esempio-Environment wird nur in der Leseansicht benötigt
\ifkorrekturansicht\else
\newenvironment{esempio}[3]%
{
    \vspace{1.5ex}
    \rlap{\underline{#1}}
    \par
    \setlength{\parindent}{0cm}
    \nopagebreak
    \leftskip=#2cm
    \rightskip=#3cm
}
{
    \par
}
\fi

\doendnotes{C}
\bigskip
\vfill

\clearpage

\footnotesize

\ifkorrekturansicht
  \lohead{\textsc{register}}
\fi

% theindex-Environment neu definieren ohne reledmac
\makeatletter
\renewenvironment{theindex}{%
  \ifkorrekturansicht
    \section*{\indexname}%
  \else
    \subsubsection*{Index der erwähnten Entitäten}%
  \fi
  \setlength{\parindent}{0pt}%
  \setlength{\parskip}{0pt plus 0.3pt}%
  \let\item\@idxitem
}{%
  \ifkorrekturansicht\clearpage\fi
}
\makeatother

\IfFileExists{\jobname-pw.ind}{\input{\jobname-pw.ind}}{}

% Quellenangabe nur in der Leseansicht
\ifkorrekturansicht\else
% Fallback-Definitionen, falls die .tex-Datei \titel etc. nicht gesetzt hat
\providecommand{\titel}{}
\providecommand{\editorInnen}{}
\providecommand{\dateiname}{\jobname}

\vspace{3cm}

\vfill

\footnotesize
\textsc{Quelle}: \titel. Herausgegeben von {\editorInnen}. In: \emph{Arthur Schnitzler: Briefwechsel mit Autorinnen und Autoren}.
 Digitale Edition, https://schnitzler-briefe.acdh.oeaw.ac.at/{\dateiname}.html (Stand \today)
\fi

\end{document}


