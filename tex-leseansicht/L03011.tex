%% latex-korrekturansicht-vorspann.tex
%% Vorspann für die Korrekturansicht.
%% Lädt die gemeinsame Datei latex-vorspann.tex mit gesetztem Schalter.

\newif\ifkorrekturansicht
\korrekturansichttrue

\input{../tex-inputs/latex-vorspann}


\section[ Arthur Schnitzler an Felix Salten, 25. 1. 1908]{L03011 Arthur Schnitzler an Felix Salten, 25. 1. 1908}
\nopagebreak\mylabel{L03011v}
\rehead{ }\normalsize\beginnumbering\briefempfaengerindex{Salten, Felix@\textsc{Salten, Felix}!zzzSchnitzler, Arthur@\emph{von Arthur Schnitzler}!1908-01-252@{25. 1. 1908}|(be}
\toendnotes[C]{\smallbreak\pagebreak[2]}\Standort{Wienbibliothek im Rathaus, ZPH 1681, 2.1.516.}
\physDesc{Brief, 1 Blatt, 4 Seiten, 2045 Zeichen
\newline{}Handschrift: blaue Tinte, lateinische Kurrent
\newline{}Ordnung: mit Bleistift von unbekannter Hand nummeriert: »6« }\toendnotes[C]{\smallbreak}
\pstart
           \raggedleft{}{\pb}25. 1. 908\pend
           \vspace{0.5em}
\pstart
           lieber, es ist \label{K_L03011-1v}\edtext{desinfizirt}{\lemma{\textnormal{\emph{desinfizirt}}}\Cendnote{\textnormal{Siehe Felix Salten an Arthur Schnitzler, [10. 12. 1907].
               }}}\label{K_L03011-1}, Wohnung, Kleider, Olga\pwindex{Schnitzler, Olga 17.01.1882 – 13.01.1970@\textsc{Schnitzler, Olga} (17.01.1882 – 13.01.1970), \emph{Schauspieler/Schauspielerin, Sänger/Sängerin}|pw} ist meist außer
               Bett, also die Zustände sind annähernd zur Norm zurückgekehrt. Der \label{K_L03011-2v}\edtext{Bub\pwindex{Schnitzler, Heinrich 09.08.1902 – 12.07.1982@\textsc{Schnitzler, Heinrich} (09.08.1902 – 12.07.1982), \emph{Regisseur/Regisseurin, Schauspieler/Schauspielerin}|pwv} ist noch nicht
                  daheim}{\lemma{\textnormal{\emph{Bub … daheim}}}\Cendnote{\textnormal{Heinrich\pwindex{Schnitzler, Heinrich 09.08.1902 – 12.07.1982@\textsc{Schnitzler, Heinrich} (09.08.1902 – 12.07.1982), \emph{Regisseur/Regisseurin, Schauspieler/Schauspielerin}|pwk} war während der Erkrankung seiner
                     Mutter\pwindex{Schnitzler, Olga 17.01.1882 – 13.01.1970@\textsc{Schnitzler, Olga} (17.01.1882 – 13.01.1970), \emph{Schauspieler/Schauspielerin, Sänger/Sängerin}|pwkv} bei Schnitzlers Mutter Louise\pwindex{Schnitzler, Louise 1840-07-08 – 1911-09-09@\textsc{Schnitzler, Louise} (1840-07-08 – 1911-09-09)|pwk}.}}}\label{K_L03011-2}, doch hab ich mit ihm Zusa{\geminationm}enkünfte, auch macht er uns \label{K_L03011-3v}\edtext{Fensterpromenaden}{\lemma{\textnormal{\emph{Fensterpromenaden}}}\Cendnote{\textnormal{Eigentlich wird damit die Praxis von Verliebten bezeichnet, in der Hoffnung, von der Angebeteten gesehen zu werden, 
                        wenn man vor dem Fenster vorbeispaziert. Für Heinrich\pwindex{Schnitzler, Heinrich 09.08.1902 – 12.07.1982@\textsc{Schnitzler, Heinrich} (09.08.1902 – 12.07.1982), \emph{Regisseur/Regisseurin, Schauspieler/Schauspielerin}|pwk} war das
                  während Olgas\pwindex{Schnitzler, Olga 17.01.1882 – 13.01.1970@\textsc{Schnitzler, Olga} (17.01.1882 – 13.01.1970), \emph{Schauspieler/Schauspielerin, Sänger/Sängerin}|pwk} Scharlacherkrankung die
                  einzige Möglichkeit, seine Mutter zu sehen.}}}\label{K_L03011-3}. Wir wollen in etwa 10 Tagen,
               bis Olga\pwindex{Schnitzler, Olga 17.01.1882 – 13.01.1970@\textsc{Schnitzler, Olga} (17.01.1882 – 13.01.1970), \emph{Schauspieler/Schauspielerin, Sänger/Sängerin}|pw} ganz gehtüchtig und die
               Influenzagerüchte – oder -wahrheiten vom Semmering\oindex{Semmering@\textbf{Semmering}, \emph{A.ADM3}|pw} geschwunden sind, \label{K_L03011-4v}\edtext{auf besagten Südbahngipfel\oindex{Semmering@\textbf{Semmering}, \emph{A.ADM3}|pwv}
                  reisen}{\lemma{\textnormal{\emph{auf … reisen}}}\Cendnote{\textnormal{Arthur und Olga Schnitzler\pwindex{Schnitzler, Olga 17.01.1882 – 13.01.1970@\textsc{Schnitzler, Olga} (17.01.1882 – 13.01.1970), \emph{Schauspieler/Schauspielerin, Sänger/Sängerin}|pwk} reisten am 4. 2. 1908 auf den Semmering\oindex{Semmering@\textbf{Semmering}, \emph{A.ADM3}|pwk} und trafen dabei im Zug auf Salten\pwindex{Salten, Felix 06.09.1869 – 08.10.1945@\textsc{Salten, Felix} (06.09.1869 – 08.10.1945), \emph{Schriftsteller/Schriftstellerin, Journalist/Journalistin, Chefredakteur/Chefredakteurin}|pwk}. Am 22. 2. 1908 reisten sie zurück nach Wien\oindex{Wien@\textbf{Wien}, \emph{A.ADM2}|pwk}.}}}\label{K_L03011-4} und dort mit Heini\pwindex{Schnitzler, Heinrich 09.08.1902 – 12.07.1982@\textsc{Schnitzler, Heinrich} (09.08.1902 – 12.07.1982), \emph{Regisseur/Regisseurin, Schauspieler/Schauspielerin}|pw} etwa 8 Tage verbringen. Dies unser Progra{\geminationm}. Da{\geminationn} erst gedenk ich
               Freundes- und andre Häuser wieder zu betreten und das {\pb}unsre zu eröffnen.\pend
           
\pstart
           Trotzdem möcht ich Sie gerne sehen\textcolor{gray}{,} früher sehen; – we{\geminationn} Sie nicht (was ich Ihnen beim Hi{\geminationm}el keinen Moment lang verübeln kö{\geminationn}te!) zu ängstlich sind. Jedenfalls schreiben Sie mir
               zum Trost\textcolor{gray}{,} wie es Ihnen Allen geht; \label{K_L03011-5v}\edtext{von Richard\pwindex{Beer-Hofmann, Richard 1866-07-11 – 1945-09-26@\textsc{Beer-Hofmann, Richard} (1866-07-11 – 1945-09-26), \emph{Schriftsteller/Schriftstellerin}|pw} hört
                  ich}{\lemma{\textnormal{\emph{von Richard hört
                  ich}}}\Cendnote{\textnormal{Schnitzler erfuhr das vermutlich beim gemeinsamen Spaziergang am 23. 1. 1908.}}}\label{K_L03011-5},
               dass Sie sich noch i{\geminationm}er nicht ganz wohl befinden.\pend
           
\pstart
           Hinsichtlich des Vorausdrucks des Romans\pwindex{Weg ins Freie. Roman@\emph{Der Weg ins Freie. Roman}|pwv} hab ich \label{K_L03011-6v}\edtext{mit Fischer\pwindex{Fischer, Samuel 24.12.1859 – 15.10.1934@\textsc{Fischer, Samuel} (24.12.1859 – 15.10.1934), \emph{Verleger/Verlegerin}|pw} schon vor
               Monaten correspondirt}{\lemma{\textnormal{\emph{mit … correspondirt}}}\Cendnote{\textnormal{Siehe 
                  Samuel Fischer\pwindex{Fischer, Samuel 24.12.1859 – 15.10.1934@\textsc{Fischer, Samuel} (24.12.1859 – 15.10.1934), \emph{Verleger/Verlegerin}|pwk}, Hedwig Fischer\pwindex{Fischer, Hedwig 08.09.1871 – 11.04.1952@\textsc{Fischer, Hedwig} (08.09.1871 – 11.04.1952)|pwk}: \emph{Briefwechsel mit
                     Autoren}. Herausgegeben von Dierk Rodewald und Corinna Fiedler. Mit
                  einer Einführung von Bernhard Zeller. Frankfurt am Main:
                  \emph{S. Fischer}{ }1989, S. 76–77.}}}\label{K_L03011-6}; aus irgen\textcolor{gray}{d}welchen techn. Gründen läßt sich
               die Sache nicht machen. Ich habe in den letzten Wochen noch viel daran corrigirt, so
               daß die Manuscripte immer ungastlicher aussehen, überdies werden Sie lieber kein {\pb}Papierconvolut aus unsrer Wohnung in Ihre
               hinübernehmen wollen – was bleibt mir also übrig? Sie bitten, das Ding\pwindex{Weg ins Freie. Roman@\emph{Der Weg ins Freie. Roman}|pwv} nicht in Forsetzungen zu lesen,
               sondern warten, bis das Buch\pwindex{Weg ins Freie. Roman@\emph{Der Weg ins Freie. Roman}|pwv}
               da ist, um es, womöglich an einem – zwei schönen Sommertagen in \uline{einem} Zug (eventuell auch in einem \uline{Zug},
               aber besser, im Freien) hinunterzuschlucken. Der Nachgeschmack wird kein übler sein;
               heut trau ich mich es zu sagen. –\pend
           
\pstart
           Ich danke Ihnen sehr für Ihre lieben Grillparzerpreis\orgindex{Franz-Grillparzer-Preis@Franz-Grillparzer-Preis|pw}glückwünsche. Anfangs war ich sehr erstaunt, da{\geminationn} eher (aus allerlei, complicirten und oberflächlichen
               Gründen) heruntergesti{\geminationm}t – jetzt überwiegt die Freude,
               woran die {\pb}\label{K_L03011-7v}\edtext{5 Mille}{\lemma{\textnormal{\emph{5 Mille}}}\Cendnote{\textnormal{Schnitzler verwendet
                  das italienische Wort »mille« für tausend. Das Preisgeld von 5000 Kronen im Jahr
                     1908 entspricht 2024 etwa
                  40.000 Euro.}}}\label{K_L03011-7} nicht ganz unbetheiligt sind. Nach dem Arbeiten sehn ich mich,
               hab manches vorbereitet und \substVorne{}\textsuperscript{\textcolor{gray}{au}}\substDazwischen{}bin\substHinten{} neugierig, was zuerst fertig sein wird. So stellt man sich frech wieder
               mitten ins Leben hinein.\pend
           
\pstart
           Seien Sie, Otti\pwindex{Salten, Ottilie 07.03.1868 – 22.06.1942@\textsc{Salten, Ottilie} (07.03.1868 – 22.06.1942), \emph{Schauspieler/Schauspielerin}|pw} und die Kinder\pwindex{Salten, Paul 11.08.1903 – 08.05.1937@\textsc{Salten, Paul} (11.08.1903 – 08.05.1937), \emph{Filmcutter/Filmcutterin}|pwv}\pwindex{Rehmann, Anna Katharina 18.08.1904 – 27.03.1977@\textsc{Rehmann, Anna Katharina} (18.08.1904 – 27.03.1977), \emph{Schauspieler/Schauspielerin, Übersetzer/Übersetzerin}|pwv} herzlichst gegrüßt und
               lassen mindestens was von sich \uline{hören}. Auch von Olga\pwindex{Schnitzler, Olga 17.01.1882 – 13.01.1970@\textsc{Schnitzler, Olga} (17.01.1882 – 13.01.1970), \emph{Schauspieler/Schauspielerin, Sänger/Sängerin}|pw} alles schöne.\pend
           
\pstart
           Ihr {\\[\baselineskip]}\spacefill\mbox{Arthur}\pend
           \leftskip=0em{}\selectlanguage{ngerman}\endnumbering\briefempfaengerindex{Salten, Felix@\textsc{Salten, Felix}!zzzSchnitzler, Arthur@\emph{von Arthur Schnitzler}!1908-01-252@{25. 1. 1908}|)be}\mylabel{L03011h}  \normalsize

\doendnotes{C}
\bigskip
\vfill

\clearpage

\footnotesize

\lohead{\textsc{register}}

% Definiere theindex-Environment komplett neu ohne reledmac
\makeatletter
\renewenvironment{theindex}{%
  \section*{\indexname}%
  \setlength{\parindent}{0pt}%
  \setlength{\parskip}{0pt plus 0.3pt}%
  \let\item\@idxitem
}{%
  \clearpage
}
\makeatother

\IfFileExists{\jobname-pw.ind}{\input{\jobname-pw.ind}}{}

\end{document}

      