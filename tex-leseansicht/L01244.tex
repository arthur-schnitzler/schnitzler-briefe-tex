%% latex-leseansicht-vorspann.tex
%% Vorspann für die Leseansicht.
%% Lädt die gemeinsame Datei latex-vorspann.tex mit nicht gesetztem Schalter.

\newif\ifkorrekturansicht
\korrekturansichtfalse

\input{../tex-inputs/latex-vorspann}

\begin{center}
            \textcolor{red}{ENTWURF. ENTZIFFERUNG NOCH NICHT KORREKTURGELESEN}
                      \end{center}
            
               \section[Hugo von Hofmannsthal an Arthur Schnitzler, 23. 10. {[}1902{]}]{ Hugo von Hofmannsthal an Arthur Schnitzler, 23. 10. {[}1902{]}}\nopagebreak\mylabel{v}\rehead{ }\begin{ledgroupsized}[t]{13cm}\normalsize\beginnumbering\briefempfaengerindex{Schnitzler, Arthur@\textsc{Schnitzler, Arthur}!zzzHofmannsthal, Hugo von@\emph{von Hugo von Hofmannsthal}!1902-10-231@{23. 10. {[}1902{]}}|(be} \toendnotes[C]{\smallbreak\pagebreak[2]} \Standort{CUL, Schnitzler, B 43.}
\physDesc{Brief, 1 Blatt, 4 Seiten
\newline{}Handschrift: schwarze Tinte, deutsche Kurrent
\newline{}Schnitzler: mit Bleistift die Jahreszahl ergänzt: »902« \newline{}Ordnung: 1) mit Bleistift von unbekannter Hand nummeriert: »\strikeout{204}« 2) mit Bleistift von unbekannter Hand nummeriert: »187«}\buchAbdrucke{\weitereDrucke{Hugo von Hofmannsthal, Arthur Schnitzler: \emph{Briefwechsel}. Hg. Therese Nickl und Heinrich Schnitzler. Frankfurt am Main: \emph{S. Fischer} 1964, S. 162–163.} }\toendnotes[C]{\smallbreak}\pstart
           \raggedleft{}{\pb}23 X\hspace*{1.5em}Rom\oindex{Rom@\textbf{Rom}|pw}.\pend
           \pstart
           lieber, ich danke Ihnen herzlich für Ihre Karte und noch mehr für
               den frühern lieben und guten Brief, der mir damals in einem Moment, wo mich ſelbſt
                  Goethe\pwindex{Goethe, Johann Wolfgang von 28.08.1749 – 22.03.1832@\textsc{Goethe, Johann Wolfgang von} (28.08.1749 – 22.03.1832), \emph{Schriftsteller}|pw} im Stich gelaſſen hatte, ungemein wohl
               gethan hat. Ich bin die erſten 14 Tage hier in einer ſinnloſen Depreſſion und
               Hilfloſigkeit herum{\pb}gelaufen.
               Plötzlich am morgen des 15\textsuperscript{ten}, hab ich gefühlt daſs etwas in mir da iſt. Und zwar nicht das »Leben ein Traum\pwindex{Hofmannsthal, Hugo von 01.02.1874 – 15.07.1929@\textsc{Hofmannsthal, Hugo von} (01.02.1874 – 15.07.1929), \emph{Schriftsteller}!Turm. Ein Trauerspiel1925@\strich\emph{Der Turm. Ein Trauerspiel} {[}1925{]}|pw}«, nicht die Elektra\pwindex{Hofmannsthal, Hugo von 01.02.1874 – 15.07.1929@\textsc{Hofmannsthal, Hugo von} (01.02.1874 – 15.07.1929), \emph{Schriftsteller}!Elektra. Tragoedie in einem Aufzug1903@\strich\emph{Elektra. Tragödie in einem Aufzug} {[}1903{]}|pw}, ſondern ein anderer Stoff\pwindex{\textcolor{red}{\textsuperscript{XXXX1 indx}}!gerettete Venedig9.2.1682 – 9.2.1682@\strich\emph{Das gerettete Venedig} {[}9.2.1682 – 9.2.1682{]}|pwv} den ich mir einmal flüchtig zurechtgelegt hatte,
               gleichfalls \strikeout{h} nach einem ältern Vorbild. Seither hab
               ich meinen Arbeitstiſch, {\pb}der je
               nach dem Wetter entweder auf dem flachen Dach oder in meinem Zimmer ſteht, kaum mehr
               viel verlaſſen und heute den erſten Act\pwindex{Hofmannsthal, Hugo von 01.02.1874 – 15.07.1929@\textsc{Hofmannsthal, Hugo von} (01.02.1874 – 15.07.1929), \emph{Schriftsteller}!gerettete Venedig. Trauerspiel in fuenf Aufzuegen1905@\strich\emph{Das gerettete Venedig. Trauerspiel in fünf Aufzügen} {[}1905{]}|pwv}, den weitaus längſten, mit 695 Verſen abgeſchloſſen.\pend
           \pstart
           Kommt von außen nichts Schlimmes, ſo glaub ich faſt ſicher gegen Ende
                  November mit dem Stück\pwindex{Hofmannsthal, Hugo von 01.02.1874 – 15.07.1929@\textsc{Hofmannsthal, Hugo von} (01.02.1874 – 15.07.1929), \emph{Schriftsteller}!gerettete Venedig. Trauerspiel in fuenf Aufzuegen1905@\strich\emph{Das gerettete Venedig. Trauerspiel in fünf Aufzügen} {[}1905{]}|pwv}
               fertig zu {\pb}ſein. Laſſen Sie mich
               nicht ohne einige Nachricht, auch über Ihre Arbeit. In ſolchen glücklicheren Tagen
               empfinde ich das freundliche ſolcher lieber Briefe doppelt ſtark. Von Herzen Ihr\pend
           \pstart \spacefill\mbox{Hugo}\pend{}\pstart
           \noindent{}P. S. Wir müſſen wieder eine Radtour zuſammen machen!\pend
           \pstart
           \numberlinefalse{}\centering{}–\numberlinetrue{}\pend
           \pstart
           \noindent{}Eiſenſtein\orgindex{J. Eisenstein und Co.@J. Eisenstein {\kaufmannsund}  Co.|pw} wird das Exemplar »Tod d. T.\pwindex{Hofmannsthal, Hugo von 01.02.1874 – 15.07.1929@\textsc{Hofmannsthal, Hugo von} (01.02.1874 – 15.07.1929), \emph{Schriftsteller}!Tod des Tizian1892.10@\strich\emph{Der Tod des Tizian} {[}1892.10{]}|pw}« an Sie ſchicken!!\pend
           \endnumbering\briefempfaengerindex{Schnitzler, Arthur@\textsc{Schnitzler, Arthur}!zzzHofmannsthal, Hugo von@\emph{von Hugo von Hofmannsthal}!1902-10-231@{23. 10. {[}1902{]}}|)be}\mylabel{h}\end{ledgroupsized}  \newcommand{\dateiname}{L01244}\newcommand{\titel}{Hugo von Hofmannsthal an Arthur Schnitzler, 23. 10. [1902]}\newcommand{\editorInnen}{Martin Anton Müller und Gerd-Hermann Susen}%% latex-leseansicht-abspann.tex
%% Abspann für die Leseansicht.
%% Der Schalter \ifkorrekturansicht ist bereits durch den Vorspann gesetzt.

%% latex-abspann.tex
%% Gemeinsamer Abspann für Korrekturansicht und Leseansicht.
%% Setzt den Schalter \ifkorrekturansicht voraus (gesetzt in den
%% einbindenden Dateien latex-korrekturansicht-abspann.tex bzw.
%% latex-leseansicht-abspann.tex).
%% ---------------------------------------------------------------

\normalsize

% Das esempio-Environment wird nur in der Leseansicht benötigt
\ifkorrekturansicht\else
\newenvironment{esempio}[3]%
{
    \vspace{1.5ex}
    \rlap{\underline{#1}}
    \par
    \setlength{\parindent}{0cm}
    \nopagebreak
    \leftskip=#2cm
    \rightskip=#3cm
}
{
    \par
}
\fi

\doendnotes{C}
\bigskip
\vfill

\clearpage

\footnotesize

\ifkorrekturansicht
  \lohead{\textsc{register}}
\fi

% theindex-Environment neu definieren ohne reledmac
\makeatletter
\renewenvironment{theindex}{%
  \ifkorrekturansicht
    \section*{\indexname}%
  \else
    \subsubsection*{Index der erwähnten Entitäten}%
  \fi
  \setlength{\parindent}{0pt}%
  \setlength{\parskip}{0pt plus 0.3pt}%
  \let\item\@idxitem
}{%
  \ifkorrekturansicht\clearpage\fi
}
\makeatother

\IfFileExists{\jobname-pw.ind}{\input{\jobname-pw.ind}}{}

% Quellenangabe nur in der Leseansicht
\ifkorrekturansicht\else
% Fallback-Definitionen, falls die .tex-Datei \titel etc. nicht gesetzt hat
\providecommand{\titel}{}
\providecommand{\editorInnen}{}
\providecommand{\dateiname}{\jobname}

\vspace{3cm}

\vfill

\footnotesize
\textsc{Quelle}: \titel. Herausgegeben von {\editorInnen}. In: \emph{Arthur Schnitzler: Briefwechsel mit Autorinnen und Autoren}.
 Digitale Edition, https://schnitzler-briefe.acdh.oeaw.ac.at/{\dateiname}.html (Stand \today)
\fi

\end{document}


      