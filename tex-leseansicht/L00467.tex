%% latex-korrekturansicht-vorspann.tex
%% Vorspann für die Korrekturansicht.
%% Lädt die gemeinsame Datei latex-vorspann.tex mit gesetztem Schalter.

\newif\ifkorrekturansicht
\korrekturansichttrue

\input{../tex-inputs/latex-vorspann}


\section[Arthur Schnitzler an Hugo von Hofmannsthal, 28. 7. 1895]{L00467 Arthur Schnitzler an Hugo von Hofmannsthal, 28. 7. 1895}
\nopagebreak\mylabel{L00467v}
\rehead{ }\normalsize\beginnumbering\briefempfaengerindex{Hofmannsthal, Hugo von@\textsc{Hofmannsthal, Hugo von}!zzzSchnitzler, Arthur@\emph{von Arthur Schnitzler}!1895-07-281@{28. 7. 1895}|(be}
\toendnotes[C]{\smallbreak\pagebreak[2]}\Standort{FDH, Hs-30885,44.}
\physDesc{Brief, 2 Blätter, 6 Seiten, 2409 Zeichen
\newline{}Handschrift: schwarze Tinte, deutsche Kurrent
\newline{}Ordnung: mit rotem Buntstift von unbekannter Hand Vermerk:
                                    »X« }
\buchAbdrucke{\weitereDrucke{Hugo von Hofmannsthal, Arthur Schnitzler: \emph{Briefwechsel}. Frankfurt am Main: \emph{S. Fischer} 1964, S. 57–58.} }\toendnotes[C]{\smallbreak}
\pstart
           \raggedleft{}{\pb}Ischl\oindex{Bad Ischl@\textbf{Bad Ischl}, \emph{P.PPL}|pw}, 28/7 95\pend
           \vspace{0.5em}
\pstart
           Mein lieber Hugo, ich habe mich ſehr gefreut, gleich nachdem ich
               hier angekommen war, Nachricht von Ihnen zu bekommen, und will Sie heute vor Allem
               herzlich grüßen u Sie bitten, mir recht bald wieder ſo einen Sti{\geminationm}ungsextract herzuſchicken, denn ſolch deutliche Zeichen
               eines In Verbindungbleibens tragen zum allgemeinen Lebensgefühl, bei mir wenigſtens,
               recht viel bei, und ſo ſollen {\pb}Ihre Briefe mit zum
               Sommer, zum »Erholen« und zu meiner guten Luft gehören. Treffen Sie dieſe Worte noch
               in Göding\oindex{Hodonín@\textbf{Hodonín}, \emph{P.PPL}|pw}? Für alle Fälle ſchickt man Ihnen ja
               nach, denk’ ich. – Mir geht es hier, bis jetzt, ganz behaglich; ich fahre \textsc{Bicycle}, bade in Strobl\oindex{Strobl@\textbf{Strobl}, \emph{A.ADM3}|pw},
               geh ins Theater, bin nicht wenig allein, leſe \textsc{Chartreuse de Parme}\pwindex{Kartause von Parma@\emph{Die Kartause von Parma}|pw}, weſtöſtl. Divan\pwindex{West-oestlicher Divan@\emph{West-östlicher Divan}|pw}, Schopenhauer\pwindex{Schopenhauer, Arthur 22.02.1788 – 21.09.1860@\textsc{Schopenhauer, Arthur} (22.02.1788 – 21.09.1860), \emph{Philosoph/Philosophin}|pw}ſche Briefe\pwindex{Schopenhauer-Briefe@\emph{Schopenhauer-Briefe}|pw},
               habe was kleines\pwindex{Frau des Weisen. Erzaehlung@\emph{Die Frau des Weisen. Erzählung}|pwv} geſchrieben
               und geh langſam an das neue Stück\pwindex{Freiwild. Schauspiel in 3 Akten@\emph{Freiwild. Schauspiel in 3 Akten}|pwv}, wovon etwa ein halber Akt da iſt und das mir im Schreiben noch ſehr
               lieb werden wird.\pend
           
\pstart
           {\pb}Vor den \textsc{Schopenh}\pwindex{Schopenhauer, Arthur 22.02.1788 – 21.09.1860@\textsc{Schopenhauer, Arthur} (22.02.1788 – 21.09.1860), \emph{Philosoph/Philosophin}|pw}. Briefen\pwindex{Schopenhauer-Briefe@\emph{Schopenhauer-Briefe}|pw} möcht ich beinahe warnen; ſie
               machen traurig – ich bin auf Seite 350 oder weiter und finde nichts als eine ſtete
               Beſchäftigung mit allem Kleinlichen, das um den »Ehrgeiz« herum iſt. Jede kleinſte
               Recenſion, die da oder dort über ihn erſchienen, wird erwähnt; – und alle Menſchen
                  un{[}d{]} Dinge nur in Betracht gezogen, inſofern ſie ſich zu
               ſeiner Philoſophie, nein, vielmehr zu der Anerke{\geminationn}ung
               ſeiner Philoſophie in Beziehung bringen laſſen. Es iſt nichts über das Leben, nichts
               über die Kunſt darin zu {\pb}finden; etwas ſo papierenes hab
               ich nie geleſen. Federkratzen, Knittern, Geruch von Büchern – es iſt als hätte die
               Welt, nachdem er ſie einmal in eine Formel gebracht, aufgehört für ihn zu exiſtiren,
                  un{[}d{]} es handelte ſich nur mehr darum, dieſe Formel von der
               Menſchheit gekannt, bewundert u angebetet zu ſehn. – In dieſer ganzen Unheimlichkeit
               war die Eitelkeit noch nicht da – und ſo iſt vielleicht auch das wieder groſs? – Eine
               Stelle lautet ungefähr: »\label{K_L00467-1v}\edtext{Ich werde
               geradezu melancholiſch, {\pb}wenn ich denke, daſs ich kaum
               ein Viertel von dem zu leſen beko{\geminationm}e, was \strikeout{ich} über mich gedruckt wird.}{\lemma{\textnormal{\emph{Ich … wird.}}}\Cendnote{\textnormal{Mehrfach im Buch geäußerter Gedanke, obzwar für gewöhnlich »die
                  Hälfte« und nicht ein Viertel entgeht. \emph{Schopenhauer-Briefe. Sammlung meist
                        ungedruckter oder schwer zugänglicher Briefe von, an und über Schopenhauer.
                        Mit Anmerkungen und biographischen Analekten}\pwindex{Schopenhauer-Briefe@\emph{Schopenhauer-Briefe}|pwk}. Herausgegeben von Ludwig Schemann\pwindex{Schemann, Ludwig 16.10.1852 – 13.02.1938@\textsc{Schemann, Ludwig} (16.10.1852 – 13.02.1938), \emph{Philosoph/Philosophin, Rassentheoretiker/Rassentheoretikerin, Bibliothekar/Bibliothekarin}|pwk}. Leipzig:
                        \emph{Brockhaus}{ }1893, S. 292, S. 324. Denkbar wäre auch, dass er eine
                  frühere Ausgabe von Briefen gelesen hat. An Julius
                     Frauenstädt\pwindex{Frauenstaedt, Julius 1813-04-17 – 1879-01-13@\textsc{Frauenstädt, Julius} (1813-04-17 – 1879-01-13)|pwk}{ }schrieb Schopenhauer\pwindex{Schopenhauer, Arthur 22.02.1788 – 21.09.1860@\textsc{Schopenhauer, Arthur} (22.02.1788 – 21.09.1860), \emph{Philosoph/Philosophin}|pwk}: »Trotz Ihrer und meiner Vigilanz glaube ich, daß von
                     Dem, was über mich gedruckt wird, etwan ¼ uns ganz entgeht.« (\emph{Arthur Schopenhauer. Von ihm. Über ihn. Ein
                        Wort der Vertheidigung}\pwindex{Arthur Schopenhauer. Von ihm. Ueber ihn@\emph{Arthur Schopenhauer. Von ihm. Über ihn}|pwk} von Ernst Otto
                        Lindner\pwindex{Lindner, Ernst Otto 28.11.1820 – 07.08.1867@\textsc{Lindner, Ernst Otto} (28.11.1820 – 07.08.1867), \emph{Musikwissenschaftler/Musikwissenschaftlerin}|pwk} und \emph{Memorabilien, Briefe und
                        Nachlassstücke}\pwindex{Arthur Schopenhauer. Von ihm. Ueber ihn@\emph{Arthur Schopenhauer. Von ihm. Über ihn}|pwk} von Julius
                        Frauenstädt\pwindex{Frauenstaedt, Julius 1813-04-17 – 1879-01-13@\textsc{Frauenstädt, Julius} (1813-04-17 – 1879-01-13)|pwk}. Berlin: \emph{A. W. Hayn}{ }1863, S. 584.)}}}\label{K_L00467-1}« Das iſt als Motto aufs Buch zu
               ſetzen. –\pend
           
\pstart
           Goldma{\geminationn}\pwindex{Goldmann, Paul 31.01.1865 – 25.09.1935@\textsc{Goldmann, Paul} (31.01.1865 – 25.09.1935), \emph{Schriftsteller/Schriftstellerin, Journalist/Journalistin}|pw} werden wir heuer wohl wieder ſehn; es ſcheint, Anfang September,
               aber alles das, wie auch \textsc{Kopenhagen}\oindex{Kopenhagen@\textbf{Kopenhagen}, \emph{P.PPLC}|pw} iſt nicht ganz ſicher. Sehr wahrſcheinlich werde ich gegen Mitte
                  Auguſt auf ein paar Tage nach Wien\oindex{Wien@\textbf{Wien}, \emph{A.ADM2}|pw}; und
               Sie? Ko{\geminationm}en Sie auch noch einmal vor den großen Manövern
               nach Wien\oindex{Wien@\textbf{Wien}, \emph{A.ADM2}|pw}? Das {\pb}laſſen
               Sie mich für alle Fälle wiſſen. –\pend
           
\pstart
           Leben Sie wohl und ſeien Sie vielmals gegrüßt.\pend
           \pstart Ihr \spacefill\mbox{Arthur.}\pend{}\selectlanguage{ngerman}\endnumbering\briefempfaengerindex{Hofmannsthal, Hugo von@\textsc{Hofmannsthal, Hugo von}!zzzSchnitzler, Arthur@\emph{von Arthur Schnitzler}!1895-07-281@{28. 7. 1895}|)be}\mylabel{L00467h}  \normalsize

\doendnotes{C}
\bigskip
\vfill

\clearpage

\footnotesize

\lohead{\textsc{register}}

% Definiere theindex-Environment komplett neu ohne reledmac
\makeatletter
\renewenvironment{theindex}{%
  \section*{\indexname}%
  \setlength{\parindent}{0pt}%
  \setlength{\parskip}{0pt plus 0.3pt}%
  \let\item\@idxitem
}{%
  \clearpage
}
\makeatother

\IfFileExists{\jobname-pw.ind}{\input{\jobname-pw.ind}}{}

\end{document}

      