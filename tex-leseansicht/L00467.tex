%% latex-leseansicht-vorspann.tex
%% Vorspann für die Leseansicht.
%% Lädt die gemeinsame Datei latex-vorspann.tex mit nicht gesetztem Schalter.

\newif\ifkorrekturansicht
\korrekturansichtfalse

\input{../tex-inputs/latex-vorspann}


\section[Arthur Schnitzler an Hugo von Hofmannsthal, 28. 7. 1895]{L00467 Arthur Schnitzler an Hugo von Hofmannsthal, 28. 7. 1895}
\nopagebreak\mylabel{L00467v}
\rehead{ }\normalsize\beginnumbering\briefempfaengerindex{Hofmannsthal, Hugo von@\textsc{Hofmannsthal, Hugo von}!zzzSchnitzler, Arthur@\emph{von Arthur Schnitzler}!1895-07-281@{28. 7. 1895}|(be}
\toendnotes[C]{\smallbreak\pagebreak[2]}
\correspDesc{Versand  durch Arthur Schnitzler am 28. 7. 1895 in Bad Ischl
\newline{}Erhalt  durch Hugo von Hofmannsthal im Zeitraum [29. 7. 1895
                  – 2. 8. 1895?] in Wien}\toendnotes[C]{\smallbreak}
\Standort{FDH, Hs-30885,44.}
\physDesc{Brief, 2 Blätter, 6 Seiten, 2409 Zeichen
\newline{}Handschrift: schwarze Tinte, deutsche Kurrent
\newline{}Ordnung: mit rotem Buntstift von unbekannter Hand Vermerk:
                                    »X« }
\buchAbdrucke{\weitereDrucke{Hugo von Hofmannsthal, Arthur Schnitzler: \emph{Briefwechsel}. Herausgegeben von Therese Nickl und Heinrich Schnitzler. Frankfurt am Main: \emph{S. Fischer} 1964, S. 57–58.} }\toendnotes[C]{\smallbreak}
\pstart
           \raggedleft{}{\pb}Ischl\oindex{Bad Ischl@\textbf{Bad Ischl}|pw}, 28/7 95\pend
           \vspace{0.5em}
\pstart
           Mein lieber Hugo, ich habe mich{ }ſehr gefreut, gleich nachdem ich
               hier angekommen war, Nachricht von Ihnen zu bekommen, und will Sie heute vor Allem
               herzlich grüßen u Sie bitten, mir recht bald wieder{ }ſo einen Sti{\geminationm}ungsextract herzuſchicken, denn{ }ſolch deutliche Zeichen
               eines In Verbindungbleibens tragen zum allgemeinen Lebensgefühl, bei mir wenigſtens,
               recht viel bei, und{ }ſo{ }ſollen {\pb}Ihre Briefe mit zum
               Sommer, zum »Erholen« und zu meiner guten Luft gehören. Treffen Sie dieſe Worte noch
               in Göding\oindex{Hodonín@\textbf{Hodonín}|pw}? Für alle Fälle{ }ſchickt man Ihnen ja
               nach, denk’ ich. – Mir geht es hier, bis jetzt, ganz behaglich; ich fahre \textsc{Bicycle}, bade in Strobl\oindex{Strobl@\textbf{Strobl}, \emph{Verwaltungsgebiet}|pw},
               geh ins Theater, bin nicht wenig allein, leſe \textsc{Chartreuse de Parme}\pwindex{\textcolor{red}{\textsuperscript{XXXX indx1}}!Kartause von Parma@\strich\emph{Die Kartause von Parma}|pw}, weſtöſtl. Divan\pwindex{\textcolor{red}{\textsuperscript{XXXX indx1}}!West-östlicher Divan@\strich\emph{West-östlicher Divan}|pw}, Schopenhauer\pwindex{Schopenhauer, Arthur 22.\,2.\,1788 Danzig – 21.\,9.\,1860 Frankfurt am Main@\textsc{Schopenhauer, Arthur} (22.\,2.\,1788 Danzig – 21.\,9.\,1860 Frankfurt am Main), \emph{Philosoph}|pw}ſche Briefe\pwindex{Schopenhauer, Arthur 22.\,2.\,1788 Danzig – 21.\,9.\,1860 Frankfurt am Main@\textsc{Schopenhauer, Arthur} (22.\,2.\,1788 Danzig – 21.\,9.\,1860 Frankfurt am Main), \emph{Philosoph}!Schopenhauer-Briefe@\strich\emph{Schopenhauer-Briefe}|pw},
               habe was kleines\pwindex{Schnitzler, Arthur 15.\,5.\,1862 Wien – 21.\,10.\,1931 ebd.@\textsc{Schnitzler, Arthur} (15.\,5.\,1862 Wien – 21.\,10.\,1931 ebd.), \emph{Schriftsteller, Mediziner}!Frau des Weisen. Erzählung@\strich\emph{Die Frau des Weisen. Erzählung}|pwv} geſchrieben
               und geh langſam an das neue Stück\pwindex{Schnitzler, Arthur 15.\,5.\,1862 Wien – 21.\,10.\,1931 ebd.@\textsc{Schnitzler, Arthur} (15.\,5.\,1862 Wien – 21.\,10.\,1931 ebd.), \emph{Schriftsteller, Mediziner}!Freiwild. Schauspiel in 3 Akten@\strich\emph{Freiwild. Schauspiel in 3 Akten}|pwv}, wovon etwa ein halber Akt da iſt und das mir im Schreiben noch{ }ſehr
               lieb werden wird.\pend
           
\pstart
           {\pb}Vor den \textsc{Schopenh}\pwindex{Schopenhauer, Arthur 22.\,2.\,1788 Danzig – 21.\,9.\,1860 Frankfurt am Main@\textsc{Schopenhauer, Arthur} (22.\,2.\,1788 Danzig – 21.\,9.\,1860 Frankfurt am Main), \emph{Philosoph}|pw}. Briefen\pwindex{Schopenhauer, Arthur 22.\,2.\,1788 Danzig – 21.\,9.\,1860 Frankfurt am Main@\textsc{Schopenhauer, Arthur} (22.\,2.\,1788 Danzig – 21.\,9.\,1860 Frankfurt am Main), \emph{Philosoph}!Schopenhauer-Briefe@\strich\emph{Schopenhauer-Briefe}|pw} möcht ich beinahe warnen;{ }ſie
               machen traurig – ich bin auf Seite 350 oder weiter und finde nichts als eine{ }ſtete
               Beſchäftigung mit allem Kleinlichen, das um den »Ehrgeiz« herum iſt. Jede kleinſte
               Recenſion, die da oder dort über ihn erſchienen, wird erwähnt; – und alle Menſchen
                  un{[}d{]} Dinge nur in Betracht gezogen, inſofern{ }ſie{ }ſich zu{ }ſeiner Philoſophie, nein, vielmehr zu der Anerke{\geminationn}ung{ }ſeiner Philoſophie in Beziehung bringen laſſen. Es iſt nichts über das Leben, nichts
               über die Kunſt darin zu {\pb}finden; etwas{ }ſo papierenes hab
               ich nie geleſen. Federkratzen, Knittern, Geruch von Büchern – es iſt als hätte die
               Welt, nachdem er{ }ſie einmal in eine Formel gebracht, aufgehört für ihn zu exiſtiren,
                  un{[}d{]} es handelte{ }ſich nur mehr darum, dieſe Formel von der
               Menſchheit gekannt, bewundert u angebetet zu{ }ſehn. – In dieſer ganzen Unheimlichkeit
               war die Eitelkeit noch nicht da – und{ }ſo iſt vielleicht auch das wieder groſs? – Eine
               Stelle lautet ungefähr: »\label{K_L00467-1v}\edtext{Ich werde
               geradezu melancholiſch, {\pb}wenn ich denke, daſs ich kaum
               ein Viertel von dem zu leſen beko{\geminationm}e, was \strikeout{ich} über mich gedruckt wird.}{\lemma{\textnormal{\emph{Ich … wird.}}}\Cendnote{\textnormal{Mehrfach im Buch geäußerter Gedanke, obzwar für gewöhnlich »die
                  Hälfte« und nicht ein Viertel entgeht. \emph{Schopenhauer-Briefe. Sammlung meist
                        ungedruckter oder schwer zugänglicher Briefe von, an und über Schopenhauer.
                        Mit Anmerkungen und biographischen Analekten}\pwindex{Schopenhauer, Arthur 22.\,2.\,1788 Danzig – 21.\,9.\,1860 Frankfurt am Main@\textsc{Schopenhauer, Arthur} (22.\,2.\,1788 Danzig – 21.\,9.\,1860 Frankfurt am Main), \emph{Philosoph}!Schopenhauer-Briefe@\strich\emph{Schopenhauer-Briefe}|pwk}. Herausgegeben von Ludwig Schemann\pwindex{Schemann, Ludwig 16.\,10.\,1852 Köln – 13.\,2.\,1938 Freiburg im Breisgau@\textsc{Schemann, Ludwig} (16.\,10.\,1852 Köln – 13.\,2.\,1938 Freiburg im Breisgau), \emph{Philosoph, Rassentheoretiker, Bibliothekar}|pwk}. Leipzig:
                        \emph{Brockhaus}{ }1893, S. 292, S. 324. Denkbar wäre auch, dass er eine
                  frühere Ausgabe von Briefen gelesen hat. An Julius
                     Frauenstädt\pwindex{Frauenstädt, Julius 17.\,4.\,1813 Bojanowo – 13.\,1.\,1879 Berlin@\textsc{Frauenstädt, Julius} (17.\,4.\,1813 Bojanowo – 13.\,1.\,1879 Berlin)|pwk}{ }schrieb Schopenhauer\pwindex{Schopenhauer, Arthur 22.\,2.\,1788 Danzig – 21.\,9.\,1860 Frankfurt am Main@\textsc{Schopenhauer, Arthur} (22.\,2.\,1788 Danzig – 21.\,9.\,1860 Frankfurt am Main), \emph{Philosoph}|pwk}: »Trotz Ihrer und meiner Vigilanz glaube ich, daß von
                     Dem, was über mich gedruckt wird, etwan ¼ uns ganz entgeht.« (\emph{Arthur Schopenhauer. Von ihm. Über ihn. Ein
                        Wort der Vertheidigung}\pwindex{Schopenhauer, Arthur 22.\,2.\,1788 Danzig – 21.\,9.\,1860 Frankfurt am Main@\textsc{Schopenhauer, Arthur} (22.\,2.\,1788 Danzig – 21.\,9.\,1860 Frankfurt am Main), \emph{Philosoph}!Arthur Schopenhauer. Von ihm. Über ihn@\strich\emph{Arthur Schopenhauer. Von ihm. Über ihn}|pwk}\pwindex{Lindner, Ernst Otto 28.\,11.\,1820 Breslau – 7.\,8.\,1867 Berlin@\textsc{Lindner, Ernst Otto} (28.\,11.\,1820 Breslau – 7.\,8.\,1867 Berlin), \emph{Musikwissenschaftler}!Arthur Schopenhauer. Von ihm. Über ihn@\strich\emph{Arthur Schopenhauer. Von ihm. Über ihn}|pwk}\pwindex{Frauenstädt, Julius 17.\,4.\,1813 Bojanowo – 13.\,1.\,1879 Berlin@\textsc{Frauenstädt, Julius} (17.\,4.\,1813 Bojanowo – 13.\,1.\,1879 Berlin)!Arthur Schopenhauer. Von ihm. Über ihn@\strich\emph{Arthur Schopenhauer. Von ihm. Über ihn}|pwk} von Ernst Otto
                        Lindner\pwindex{Lindner, Ernst Otto 28.\,11.\,1820 Breslau – 7.\,8.\,1867 Berlin@\textsc{Lindner, Ernst Otto} (28.\,11.\,1820 Breslau – 7.\,8.\,1867 Berlin), \emph{Musikwissenschaftler}|pwk} und \emph{Memorabilien, Briefe und
                        Nachlassstücke}\pwindex{Schopenhauer, Arthur 22.\,2.\,1788 Danzig – 21.\,9.\,1860 Frankfurt am Main@\textsc{Schopenhauer, Arthur} (22.\,2.\,1788 Danzig – 21.\,9.\,1860 Frankfurt am Main), \emph{Philosoph}!Arthur Schopenhauer. Von ihm. Über ihn@\strich\emph{Arthur Schopenhauer. Von ihm. Über ihn}|pwk}\pwindex{Lindner, Ernst Otto 28.\,11.\,1820 Breslau – 7.\,8.\,1867 Berlin@\textsc{Lindner, Ernst Otto} (28.\,11.\,1820 Breslau – 7.\,8.\,1867 Berlin), \emph{Musikwissenschaftler}!Arthur Schopenhauer. Von ihm. Über ihn@\strich\emph{Arthur Schopenhauer. Von ihm. Über ihn}|pwk}\pwindex{Frauenstädt, Julius 17.\,4.\,1813 Bojanowo – 13.\,1.\,1879 Berlin@\textsc{Frauenstädt, Julius} (17.\,4.\,1813 Bojanowo – 13.\,1.\,1879 Berlin)!Arthur Schopenhauer. Von ihm. Über ihn@\strich\emph{Arthur Schopenhauer. Von ihm. Über ihn}|pwk} von Julius
                        Frauenstädt\pwindex{Frauenstädt, Julius 17.\,4.\,1813 Bojanowo – 13.\,1.\,1879 Berlin@\textsc{Frauenstädt, Julius} (17.\,4.\,1813 Bojanowo – 13.\,1.\,1879 Berlin)|pwk}. Berlin: \emph{A. W. Hayn}{ }1863, S. 584.)}}}\label{K_L00467-1}« Das iſt als Motto aufs Buch zu{ }ſetzen. –\pend
           
\pstart
           Goldma{\geminationn}\pwindex{Goldmann, Paul 31.\,1.\,1865 Breslau – 25.\,9.\,1935 Wien@\textsc{Goldmann, Paul} (31.\,1.\,1865 Breslau – 25.\,9.\,1935 Wien), \emph{Schriftsteller, Journalist}|pw} werden wir heuer wohl wieder{ }ſehn; es{ }ſcheint, Anfang September,
               aber alles das, wie auch \textsc{Kopenhagen}\oindex{Kopenhagen@\textbf{Kopenhagen}, \emph{Hauptstadt}|pw} iſt nicht ganz{ }ſicher. Sehr wahrſcheinlich werde ich gegen Mitte
                  Auguſt auf ein paar Tage nach Wien\oindex{Wien@\textbf{Wien}, \emph{Verwaltungsgebiet}|pw}; und
               Sie? Ko{\geminationm}en Sie auch noch einmal vor den großen Manövern
               nach Wien\oindex{Wien@\textbf{Wien}, \emph{Verwaltungsgebiet}|pw}? Das {\pb}laſſen
               Sie mich für alle Fälle wiſſen. –\pend
           
\pstart
           Leben Sie wohl und{ }ſeien Sie vielmals gegrüßt.\pend
           \pstart Ihr \spacefill\mbox{Arthur.}\pend{}\selectlanguage{ngerman}\endnumbering\briefempfaengerindex{Hofmannsthal, Hugo von@\textsc{Hofmannsthal, Hugo von}!zzzSchnitzler, Arthur@\emph{von Arthur Schnitzler}!1895-07-281@{28. 7. 1895}|)be}\mylabel{L00467h}  \newcommand{\dateiname}{L00467}\newcommand{\titel}{Arthur Schnitzler an Hugo von Hofmannsthal, 28. 7. 1895}\newcommand{\editorInnen}{Martin Anton Müller und Gerd-Hermann Susen}%% latex-leseansicht-abspann.tex
%% Abspann für die Leseansicht.
%% Der Schalter \ifkorrekturansicht ist bereits durch den Vorspann gesetzt.

%% latex-abspann.tex
%% Gemeinsamer Abspann für Korrekturansicht und Leseansicht.
%% Setzt den Schalter \ifkorrekturansicht voraus (gesetzt in den
%% einbindenden Dateien latex-korrekturansicht-abspann.tex bzw.
%% latex-leseansicht-abspann.tex).
%% ---------------------------------------------------------------

\normalsize

% Das esempio-Environment wird nur in der Leseansicht benötigt
\ifkorrekturansicht\else
\newenvironment{esempio}[3]%
{
    \vspace{1.5ex}
    \rlap{\underline{#1}}
    \par
    \setlength{\parindent}{0cm}
    \nopagebreak
    \leftskip=#2cm
    \rightskip=#3cm
}
{
    \par
}
\fi

\doendnotes{C}
\bigskip
\vfill

\clearpage

\footnotesize

\ifkorrekturansicht
  \lohead{\textsc{register}}
\fi

% theindex-Environment neu definieren ohne reledmac
\makeatletter
\renewenvironment{theindex}{%
  \ifkorrekturansicht
    \section*{\indexname}%
  \else
    \subsubsection*{Index der erwähnten Entitäten}%
  \fi
  \setlength{\parindent}{0pt}%
  \setlength{\parskip}{0pt plus 0.3pt}%
  \let\item\@idxitem
}{%
  \ifkorrekturansicht\clearpage\fi
}
\makeatother

\IfFileExists{\jobname-pw.ind}{\input{\jobname-pw.ind}}{}

% Quellenangabe nur in der Leseansicht
\ifkorrekturansicht\else
% Fallback-Definitionen, falls die .tex-Datei \titel etc. nicht gesetzt hat
\providecommand{\titel}{}
\providecommand{\editorInnen}{}
\providecommand{\dateiname}{\jobname}

\vspace{3cm}

\vfill

\footnotesize
\textsc{Quelle}: \titel. Herausgegeben von {\editorInnen}. In: \emph{Arthur Schnitzler: Briefwechsel mit Autorinnen und Autoren}.
 Digitale Edition, https://schnitzler-briefe.acdh.oeaw.ac.at/{\dateiname}.html (Stand \today)
\fi

\end{document}


