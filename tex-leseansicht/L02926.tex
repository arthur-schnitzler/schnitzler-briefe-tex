%% latex-korrekturansicht-vorspann.tex
%% Vorspann für die Korrekturansicht.
%% Lädt die gemeinsame Datei latex-vorspann.tex mit gesetztem Schalter.

\newif\ifkorrekturansicht
\korrekturansichttrue

\input{../tex-inputs/latex-vorspann}


\section[ Paul Goldmann an Arthur Schnitzler, 2. 8. {[}1900{]}]{L02926 Paul Goldmann an Arthur Schnitzler, 2. 8. {[}1900{]}}
\nopagebreak\mylabel{L02926v}
\rehead{ }\normalsize\beginnumbering\briefempfaengerindex{Schnitzler, Arthur@\textsc{Schnitzler, Arthur}!zzzGoldmann, Paul@\emph{von Paul Goldmann}!1900-08-022@{2. 8. {[}1900{]}}|(be}
\toendnotes[C]{\smallbreak\pagebreak[2]}\Standort{DLA, A:Schnitzler, HS.NZ85.1.3170.}
\physDesc{Brief, 1 Blatt, 2 Seiten, 493 Zeichen
\newline{}Handschrift: blaue Tinte, deutsche Kurrent
\newline{}Schnitzler: mit Bleistift das Jahr »900« vermerkt }\toendnotes[C]{\smallbreak}
\pstart
           \raggedleft{}{\pb}\textcolor{gray}{\textbf{DESSAUERSTRASSE 19}}\oindex{Dessauer Strasse@\textbf{Dessauer Straße}, \emph{Straße (K.STR)}|pw}\pend
           
\pstart
           Berlin\oindex{Berlin@\textbf{Berlin}, \emph{P.PPLC}|pw}, 2. Auguſt.\pend
           
\pstart\center{}Mein lieber Freund,\pend\vspace{0.5em}
\pstart
           Ich bin mit dem \label{K_L02926-1v}\edtext{Reiſeprogramm}{\lemma{\textnormal{\emph{Reiſeprogramm}}}\Cendnote{\textnormal{Siehe Paul Goldmann an Arthur Schnitzler, 16. 6. [1900].
               }}}\label{K_L02926-1} einverſtanden und hoffe, am 10. Auguſt hier
               wegzukönnen. Aber beſtimmt iſt es noch nicht. \strikeout{Die} Es
               kann früher und auch ſpäter ſein. Der Termin verſchiebt ſich jeden Tag. Bleibſt Du
               bis dahin in \label{K_L02926-2v}\edtext{\textsc{Ischl\oindex{Bad Ischl@\textbf{Bad Ischl}, \emph{P.PPL}|pw}}}{\lemma{\textnormal{\emph{Ischl}}}\Cendnote{\textnormal{Schnitzler war vom 1. 8. 1900 bis zum 10. 8. 1900 in Ischl\oindex{Bad Ischl@\textbf{Bad Ischl}, \emph{P.PPL}|pwk}.}}}\label{K_L02926-2}? Damit ich Dich von meiner
               Abreiſe verſtändigen kann.\pend
           
\pstart
           Ich muß eine Nacht in Wien\oindex{Wien@\textbf{Wien}, \emph{A.ADM2}|pw}{ }{\pb}bleiben. In welchem billigen \textsc{Hotel} kann ich wohnen?\pend
           
\pstart
           Viele Grüße! {\\[\baselineskip]}Dein {\\[\baselineskip]}\spacefill\mbox{Paul Goldmn}\pend
           \leftskip=0em{}
\pstart
           \noindent{}\label{K_L02926-3v}\edtext{Deinen Brief ſchicke ich an \textsc{Kerr\pwindex{Kerr, Alfred 25.12.1867 – 12.10.1948@\textsc{Kerr, Alfred} (25.12.1867 – 12.10.1948), \emph{Schriftsteller/Schriftstellerin, Kritiker/Kritikerin}|pw}}}{\lemma{\textnormal{\emph{Deinen … Kerr}}}\Cendnote{\textnormal{Deshalb ist der Brief auch im
                     Nachlass Kerrs\pwindex{Kerr, Alfred 25.12.1867 – 12.10.1948@\textsc{Kerr, Alfred} (25.12.1867 – 12.10.1948), \emph{Schriftsteller/Schriftstellerin, Kritiker/Kritikerin}|pwk} überliefert, siehe Arthur Schnitzler an Paul Goldmann, 1. 7. 1900.}}}\label{K_L02926-3}, deſſen
                  letzte Adreſſe \textsc{Toblach\oindex{Toblach@\textbf{Toblach}, \emph{A.ADM3}|pw}{ }\begin{otherlanguage}{french}Poste restante\end{otherlanguage}} iſt.\pend
           \selectlanguage{ngerman}\endnumbering\briefempfaengerindex{Schnitzler, Arthur@\textsc{Schnitzler, Arthur}!zzzGoldmann, Paul@\emph{von Paul Goldmann}!1900-08-022@{2. 8. {[}1900{]}}|)be}\mylabel{L02926h}  \normalsize

\doendnotes{C}
\bigskip
\vfill

\clearpage

\footnotesize

\lohead{\textsc{register}}

% Definiere theindex-Environment komplett neu ohne reledmac
\makeatletter
\renewenvironment{theindex}{%
  \section*{\indexname}%
  \setlength{\parindent}{0pt}%
  \setlength{\parskip}{0pt plus 0.3pt}%
  \let\item\@idxitem
}{%
  \clearpage
}
\makeatother

\IfFileExists{\jobname-pw.ind}{\input{\jobname-pw.ind}}{}

\end{document}

      