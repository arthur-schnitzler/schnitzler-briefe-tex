%% latex-leseansicht-vorspann.tex
%% Vorspann für die Leseansicht.
%% Lädt die gemeinsame Datei latex-vorspann.tex mit nicht gesetztem Schalter.

\newif\ifkorrekturansicht
\korrekturansichtfalse

\input{../tex-inputs/latex-vorspann}


         
         \renewcommand{\erwaehntePersonen}{Personen: Hugo von Hofmannsthal, Ida Nacht, Felix Salten, Paul Salten}
         \renewcommand{\erwaehnteOrte}{Orte: Berlin, Wien}
         \renewcommand{\erwaehnteWerke}{Werke: Der Schrei der Liebe. Novelle}
               \section[ Felix Salten an Arthur Schnitzler, {[}16.? 10. 1903{]}]{ Felix Salten an Arthur Schnitzler, {[}16.? 10. 1903{]}}\nopagebreak\mylabel{v}\rehead{ }\begin{ledgroupsized}[t]{13cm}\normalsize\beginnumbering\briefempfaengerindex{Schnitzler, Arthur@\textsc{Schnitzler, Arthur}!zzzSalten, Felix@\emph{von Felix Salten}!1903-10-241@{{[}16.? 10. 1903{]}}|(be} \toendnotes[C]{\smallbreak\pagebreak[2]} \Standort{CUL, Schnitzler, B 89, A 2.}
\physDesc{Karte, 497 Zeichen
\newline{}Handschrift: schwarze Tinte, lateinische Kurrent
\newline{}Schnitzler: mit Bleistift datiert: »Oct 903« 
\newline{}Ordnung: mit Bleistift von unbekannter Hand nummeriert: »{\pb}176« }\toendnotes[C]{\smallbreak}\pstart
           \noindent{}{\pb}Lieber, da wir die Amme\pwindex{Nacht, Ida @\textsc{Nacht, Ida}, \emph{Hebamme}|pwv} und das Kleine\pwindex{Salten, Paul 11.08.1903 – 08.05.1937@\textsc{Salten, Paul} (11.08.1903 – 08.05.1937), \emph{Filmcutter}|pwv} nicht so lange allein laßen wollen, \label{K_L03350-1v}\edtext{kommen wir Sonntag nicht zum Essen}{\lemma{\textnormal{\emph{kommen … Essen}}}\Cendnote{\textnormal{Hier dürfte es sich
                     um die Antwort auf Schnitzler\pwindex{Schnitzler, Arthur 15.05.1862 – 21.10.1931@\textsc{Schnitzler, Arthur} (15.05.1862 – 21.10.1931), \emph{Schriftsteller, Mediziner}|pwk}s Brief vom 15. 10. 1903
                        handeln, was eine genauere zeitliche Eingrenzung des undatierten Korrespondenzstücks über Schnitzler\pwindex{Schnitzler, Arthur 15.05.1862 – 21.10.1931@\textsc{Schnitzler, Arthur} (15.05.1862 – 21.10.1931), \emph{Schriftsteller, Mediziner}|pwk}s
                        Angabe »Oct 903« hinaus in den Zeitraum vor den Sonntag erlaubt. Zudem dürfte am Vortag des Treffens, dem 17. 10. 1903, bereits
                        von ›morgen‹ die Rede gewesen sein und wäre es eine sehr kurzfristige Absage des Mittagessens gewesen, weswegen dieser
                        Tag ebenfalls nicht in Frage kommt.}}}\label{K_L03350-1h}, sondern um 3 od.
                  ½ 4 zum Kaffee, wenn wir einen kriegen.\pend
           \pstart
           Hfthl.\pwindex{Hofmannsthal, Hugo von 1874-02-01 – 1929-07-15@\textsc{Hofmannsthal, Hugo von} (1874-02-01 – 1929-07-15), \emph{Schriftsteller}|pw} bittet mich am \label{K_L03350-2v}\edtext{Dienstag vorzulesen, weil er 
               Mittwoch}{\lemma{\textnormal{\emph{Dienstag … Mittwoch}}}\Cendnote{\textnormal{Es dürfte sich dabei um einen weiteren Schlenker beim Versuch handeln, 
                  die private Lesung von \emph{Der Schrei der Liebe}\pwindex{Salten, Felix 06.09.1869 – 08.10.1945@\textsc{Salten, Felix} (06.09.1869 – 08.10.1945), \emph{Schriftsteller, Journalist}!Schrei der Liebe. Novelle1904-10-22@\strich\emph{Der Schrei der Liebe. Novelle} {[}1904-10-22{]}|pwk} zu terminisieren,
                  die dann trotz der Ankündigung im vorliegenden Korrespondenzstück am Mittwoch, dem 21. 10. 1903
                  stattfand. Dass der Termin am Mittwoch hielt, dürfte daran liegen, dass Hofmannsthal\pwindex{Hofmannsthal, Hugo von 1874-02-01 – 1929-07-15@\textsc{Hofmannsthal, Hugo von} (1874-02-01 – 1929-07-15), \emph{Schriftsteller}|pwk} erst am 26. 10. 1903 nach Berlin\oindex{Berlin@\textbf{Berlin}|pwk} reiste.}}}\label{K_L03350-2h} abreist. Also Dienstag.
               Ich hoffe sehr, dass Sie nicht verhindert sind, denn ich möchte es jetzt nicht mehr
               verschieben. Sonst müßte die Sache bis November bleiben,
               weil H.\pwindex{Hofmannsthal, Hugo von 1874-02-01 – 1929-07-15@\textsc{Hofmannsthal, Hugo von} (1874-02-01 – 1929-07-15), \emph{Schriftsteller}|pw} dabei sein will, und ein so langer
               Aufschub wäre mir jetzt mehr als unangenehm.\pend
           \pstart
           Also zunächst auf Sonntag.\pend
           \pstart
           herzlichst {\\[\baselineskip]}Ihr {\\[\baselineskip]}\spacefill\mbox{S.}\pend
           \leftskip=0em{}
         
         \endnumbering\mylabel{h}\end{ledgroupsized}  \newcommand{\dateiname}{L03350}\newcommand{\titel}{Felix Salten an Arthur Schnitzler, [16.? 10. 1903]}\newcommand{\editorInnen}{Martin Anton Müller und Laura Untner}%% latex-leseansicht-abspann.tex
%% Abspann für die Leseansicht.
%% Der Schalter \ifkorrekturansicht ist bereits durch den Vorspann gesetzt.

%% latex-abspann.tex
%% Gemeinsamer Abspann für Korrekturansicht und Leseansicht.
%% Setzt den Schalter \ifkorrekturansicht voraus (gesetzt in den
%% einbindenden Dateien latex-korrekturansicht-abspann.tex bzw.
%% latex-leseansicht-abspann.tex).
%% ---------------------------------------------------------------

\normalsize

% Das esempio-Environment wird nur in der Leseansicht benötigt
\ifkorrekturansicht\else
\newenvironment{esempio}[3]%
{
    \vspace{1.5ex}
    \rlap{\underline{#1}}
    \par
    \setlength{\parindent}{0cm}
    \nopagebreak
    \leftskip=#2cm
    \rightskip=#3cm
}
{
    \par
}
\fi

\doendnotes{C}
\bigskip
\vfill

\clearpage

\footnotesize

\ifkorrekturansicht
  \lohead{\textsc{register}}
\fi

% theindex-Environment neu definieren ohne reledmac
\makeatletter
\renewenvironment{theindex}{%
  \ifkorrekturansicht
    \section*{\indexname}%
  \else
    \subsubsection*{Index der erwähnten Entitäten}%
  \fi
  \setlength{\parindent}{0pt}%
  \setlength{\parskip}{0pt plus 0.3pt}%
  \let\item\@idxitem
}{%
  \ifkorrekturansicht\clearpage\fi
}
\makeatother

\IfFileExists{\jobname-pw.ind}{\input{\jobname-pw.ind}}{}

% Quellenangabe nur in der Leseansicht
\ifkorrekturansicht\else
% Fallback-Definitionen, falls die .tex-Datei \titel etc. nicht gesetzt hat
\providecommand{\titel}{}
\providecommand{\editorInnen}{}
\providecommand{\dateiname}{\jobname}

\vspace{3cm}

\vfill

\footnotesize
\textsc{Quelle}: \titel. Herausgegeben von {\editorInnen}. In: \emph{Arthur Schnitzler: Briefwechsel mit Autorinnen und Autoren}.
 Digitale Edition, https://schnitzler-briefe.acdh.oeaw.ac.at/{\dateiname}.html (Stand \today)
\fi

\end{document}


      