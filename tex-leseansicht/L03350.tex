%% latex-korrekturansicht-vorspann.tex
%% Vorspann für die Korrekturansicht.
%% Lädt die gemeinsame Datei latex-vorspann.tex mit gesetztem Schalter.

\newif\ifkorrekturansicht
\korrekturansichttrue

\input{../tex-inputs/latex-vorspann}


\section[ Felix Salten an Arthur Schnitzler, {[}16.? 10. 1903{]}]{L03350 Felix Salten an Arthur Schnitzler, {[}16.? 10. 1903{]}}
\nopagebreak\mylabel{L03350v}
\rehead{ }\normalsize\beginnumbering\briefempfaengerindex{Schnitzler, Arthur@\textsc{Schnitzler, Arthur}!zzzSalten, Felix@\emph{von Felix Salten}!1903-10-241@{{[}16.? 10. 1903{]}}|(be}
\toendnotes[C]{\smallbreak\pagebreak[2]}\Standort{CUL, Schnitzler, B 89, A 2.}
\physDesc{Karte, 497 Zeichen
\newline{}Handschrift: schwarze Tinte, lateinische Kurrent
\newline{}Schnitzler: mit Bleistift datiert: »Oct 903« 
\newline{}Ordnung: mit Bleistift von unbekannter Hand nummeriert: »{\pb}176« }\toendnotes[C]{\smallbreak}
\pstart
           \noindent{}{\pb}Lieber, da wir die Amme\pwindex{Nacht, Ida @\textsc{Nacht, Ida}, \emph{männliche Hebamme/Hebamme}|pwv} und das Kleine\pwindex{Salten, Paul 11.08.1903 – 08.05.1937@\textsc{Salten, Paul} (11.08.1903 – 08.05.1937), \emph{Filmcutter/Filmcutterin}|pwv} nicht so lange allein laßen wollen, \label{K_L03350-1v}\edtext{kommen wir Sonntag nicht zum Essen}{\lemma{\textnormal{\emph{kommen … Essen}}}\Cendnote{\textnormal{Hier
                  dürfte es sich um die Antwort auf Schnitzlers Brief vom 15. 10. 1903 handeln, was eine genauere zeitliche Eingrenzung des
                  undatierten Korrespondenzstücks über Schnitzlers Angabe »Oct 903« hinaus in den Zeitraum vor dem Sonntag erlaubt. Zudem wäre am Vortag
                  des Treffens, dem 17. 10. 1903, bereits von ›morgen‹ die Rede gewesen, weswegen
                  dieser Tag ebenfalls nicht infrage kommt.}}}\label{K_L03350-1}, sondern um 3 od.
                  ½ 4 zum Kaffee, wenn wir einen kriegen.\pend
           
\pstart
           Hfthl.\pwindex{Hofmannsthal, Hugo von 1874-02-01 – 1929-07-15@\textsc{Hofmannsthal, Hugo von} (1874-02-01 – 1929-07-15), \emph{Schriftsteller/Schriftstellerin}|pw} bittet mich am \label{K_L03350-2v}\edtext{Dienstag vorzulesen, weil er Mittwoch}{\lemma{\textnormal{\emph{Dienstag … Mittwoch}}}\Cendnote{\textnormal{Es dürfte sich dabei um einen weiteren
                  Schlenker bei dem Versuch handeln, die private Lesung von \emph{Der Schrei der Liebe}\pwindex{Schrei der Liebe. Novelle@\emph{Der Schrei der Liebe. Novelle}|pwk} terminlich zu fixieren, die dann trotz der
                  Ankündigung im vorliegenden Korrespondenzstück am Mittwoch, dem 21. 10. 1903 stattfand.
                  Dass der Termin am Mittwoch hielt, dürfte daran liegen, dass Hofmannsthal\pwindex{Hofmannsthal, Hugo von 1874-02-01 – 1929-07-15@\textsc{Hofmannsthal, Hugo von} (1874-02-01 – 1929-07-15), \emph{Schriftsteller/Schriftstellerin}|pwk} erst am 26. 10. 1903 nach Berlin\oindex{Berlin@\textbf{Berlin}, \emph{P.PPLC}|pwk}
                  reiste.}}}\label{K_L03350-2} abreist. Also Dienstag. Ich hoffe
               sehr, dass Sie nicht verhindert sind, denn ich möchte es jetzt nicht mehr
               verschieben. Sonst müßte die Sache bis November bleiben,
               weil H.\pwindex{Hofmannsthal, Hugo von 1874-02-01 – 1929-07-15@\textsc{Hofmannsthal, Hugo von} (1874-02-01 – 1929-07-15), \emph{Schriftsteller/Schriftstellerin}|pw} dabei sein will, und ein so langer
               Aufschub wäre mir jetzt mehr als unangenehm.\pend
           
\pstart
           Also zunächst auf Sonntag.\pend
           
\pstart
           herzlichst {\\[\baselineskip]}Ihr {\\[\baselineskip]}\spacefill\mbox{S.}\pend
           \leftskip=0em{}\selectlanguage{ngerman}\endnumbering\briefempfaengerindex{Schnitzler, Arthur@\textsc{Schnitzler, Arthur}!zzzSalten, Felix@\emph{von Felix Salten}!1903-10-161@{{[}16.? 10. 1903{]}}|)be}\mylabel{L03350h}  \normalsize

\doendnotes{C}
\bigskip
\vfill

\clearpage

\footnotesize

\lohead{\textsc{register}}

% Definiere theindex-Environment komplett neu ohne reledmac
\makeatletter
\renewenvironment{theindex}{%
  \section*{\indexname}%
  \setlength{\parindent}{0pt}%
  \setlength{\parskip}{0pt plus 0.3pt}%
  \let\item\@idxitem
}{%
  \clearpage
}
\makeatother

\IfFileExists{\jobname-pw.ind}{\input{\jobname-pw.ind}}{}

\end{document}

      