%% latex-leseansicht-vorspann.tex
%% Vorspann für die Leseansicht.
%% Lädt die gemeinsame Datei latex-vorspann.tex mit nicht gesetztem Schalter.

\newif\ifkorrekturansicht
\korrekturansichtfalse

\input{../tex-inputs/latex-vorspann}


         
         \renewcommand{\erwaehntePersonen}{Personen: Moritz Coschell, Auguste Glümer, Marie Glümer}
         \renewcommand{\erwaehnteOrte}{Orte: Berlin, Café Grunewald, Dessauer Straße, Grunewald, Ivano-Frankivsk}
         \renewcommand{\erwaehnteWerke}{Werke: ?? [Jüdisches Gemälde]}
               \section[ Paul Goldmann an Arthur Schnitzler, 14. 10. {[}1902{]}]{ Paul Goldmann an Arthur Schnitzler, 14. 10. {[}1902{]}}\nopagebreak\mylabel{v}\rehead{ }\begin{ledgroupsized}[t]{13cm}\normalsize\beginnumbering \toendnotes[C]{\smallbreak\pagebreak[2]} \Standort{DLA, A:Schnitzler, HS.NZ85.1.3172.}
\physDesc{Brief, 1 Blatt, 2 Seiten, 298 Zeichen
\newline{}Handschrift: blaue Tinte, deutsche Kurrent
\newline{}Schnitzler: 1) mit Bleistift das Jahr »902« vermerkt  2) mit rotem Buntstift drei Unterstreichungen}\toendnotes[C]{\smallbreak}\pstart
           \noindent{}\raggedleft{}{\pb}\textcolor{gray}{\textbf{DESSAUERSTRASSE 19}}\oindex{Dessauer Strasse@\textbf{Dessauer Straße}|pw}\pend
           \pstart
           Berlin\oindex{Berlin@\textbf{Berlin}|pw}, 14. Okt.\pend
           \pstart\center{}Mein lieber Freund,\pend\pstart
           \textsc{Coschell\pwindex{Coschell, Moritz 1872-09-18 – 1943-07-11@\textsc{Coschell, Moritz} (1872-09-18 – 1943-07-11), \emph{Maler}|pw}} iſt gar nicht in Berlin\oindex{Berlin@\textbf{Berlin}|pw}. Er macht Studien
               zu ſeinem \label{K_L03227-1v}\edtext{jüdiſchen Gemälde\pwindex{Coschell, Moritz 1872-09-18 – 1943-07-11@\textsc{Coschell, Moritz} (1872-09-18 – 1943-07-11), \emph{Maler}!?? [Juedisches Gemaelde]@\strich\emph{?? [Jüdisches Gemälde]}|pwv}}{\lemma{\textnormal{\emph{jüdiſchen Gemälde}}}\Cendnote{\textnormal{nicht ermittelt}}}\label{K_L03227-1h} in \textsc{Stanislau\oindex{Ivano-Frankivsk@\textbf{Ivano-Frankivsk}|pw}}.\pend
           \pstart
           \label{K_L03227-2v}\edtext{\textsc{Gusti\pwindex{Gluemer, Auguste 16.03.1862 – 1956@\textsc{Glümer, Auguste} (16.03.1862 – 1956)|pw}}}{\lemma{\textnormal{\emph{Gusti}}}\Cendnote{\textnormal{Schnitzler\pwindex{Schnitzler, Arthur 15.05.1862 – 21.10.1931@\textsc{Schnitzler, Arthur} (15.05.1862 – 21.10.1931), \emph{Schriftsteller, Mediziner}|pwk} traf Auguste Glümer\pwindex{Gluemer, Auguste 16.03.1862 – 1956@\textsc{Glümer, Auguste} (16.03.1862 – 1956)|pwk} am Folgetag, dem 15. 10. 1902.}}}\label{K_L03227-2h}
               wird ſich mit Dir in Verbindung setzen.\pend
           \pstart
           \textsc{Mizzi\pwindex{Gluemer, Marie 03.07.1867 – 16.11.1925@\textsc{Glümer, Marie} (03.07.1867 – 16.11.1925), \emph{Schauspielerin}|pw}} iſt krank. Sie {\pb}hat ihre alten Kopfſchmerzen
               u. wohnt im \textsc{Grunewald\oindex{Grunewald@\textbf{Grunewald}|pw}}, \textsc{Café Grunewald\oindex{Cafe Grunewald@\textbf{Café Grunewald}|pw}}.\pend
           \pstart
           Auf Mittwoch{ }Abend, 7 Uhr!\pend
           \pstart
           Herzlichſt {\\[\baselineskip]}Dein {\\[\baselineskip]}\spacefill\mbox{Paul Goldmn}\pend
           \leftskip=0em{}
         
         \endnumbering\mylabel{h}\end{ledgroupsized}  \newcommand{\dateiname}{L03227}\newcommand{\titel}{Paul Goldmann an Arthur Schnitzler, 14. 10. [1902]}\newcommand{\editorInnen}{Martin Anton Müller und Laura Untner}%% latex-leseansicht-abspann.tex
%% Abspann für die Leseansicht.
%% Der Schalter \ifkorrekturansicht ist bereits durch den Vorspann gesetzt.

%% latex-abspann.tex
%% Gemeinsamer Abspann für Korrekturansicht und Leseansicht.
%% Setzt den Schalter \ifkorrekturansicht voraus (gesetzt in den
%% einbindenden Dateien latex-korrekturansicht-abspann.tex bzw.
%% latex-leseansicht-abspann.tex).
%% ---------------------------------------------------------------

\normalsize

% Das esempio-Environment wird nur in der Leseansicht benötigt
\ifkorrekturansicht\else
\newenvironment{esempio}[3]%
{
    \vspace{1.5ex}
    \rlap{\underline{#1}}
    \par
    \setlength{\parindent}{0cm}
    \nopagebreak
    \leftskip=#2cm
    \rightskip=#3cm
}
{
    \par
}
\fi

\doendnotes{C}
\bigskip
\vfill

\clearpage

\footnotesize

\ifkorrekturansicht
  \lohead{\textsc{register}}
\fi

% theindex-Environment neu definieren ohne reledmac
\makeatletter
\renewenvironment{theindex}{%
  \ifkorrekturansicht
    \section*{\indexname}%
  \else
    \subsubsection*{Index der erwähnten Entitäten}%
  \fi
  \setlength{\parindent}{0pt}%
  \setlength{\parskip}{0pt plus 0.3pt}%
  \let\item\@idxitem
}{%
  \ifkorrekturansicht\clearpage\fi
}
\makeatother

\IfFileExists{\jobname-pw.ind}{\input{\jobname-pw.ind}}{}

% Quellenangabe nur in der Leseansicht
\ifkorrekturansicht\else
% Fallback-Definitionen, falls die .tex-Datei \titel etc. nicht gesetzt hat
\providecommand{\titel}{}
\providecommand{\editorInnen}{}
\providecommand{\dateiname}{\jobname}

\vspace{3cm}

\vfill

\footnotesize
\textsc{Quelle}: \titel. Herausgegeben von {\editorInnen}. In: \emph{Arthur Schnitzler: Briefwechsel mit Autorinnen und Autoren}.
 Digitale Edition, https://schnitzler-briefe.acdh.oeaw.ac.at/{\dateiname}.html (Stand \today)
\fi

\end{document}


      