%% latex-korrekturansicht-vorspann.tex
%% Vorspann für die Korrekturansicht.
%% Lädt die gemeinsame Datei latex-vorspann.tex mit gesetztem Schalter.

\newif\ifkorrekturansicht
\korrekturansichttrue

\input{../tex-inputs/latex-vorspann}


\section[ Paul Goldmann an Arthur Schnitzler, 14. 10. {[}1902{]}]{L03227 Paul Goldmann an Arthur Schnitzler, 14. 10. {[}1902{]}}
\nopagebreak\mylabel{L03227v}
\rehead{ }\normalsize\beginnumbering\briefempfaengerindex{Schnitzler, Arthur@\textsc{Schnitzler, Arthur}!zzzGoldmann, Paul@\emph{von Paul Goldmann}!1902-10-141@{14. 10. {[}1902{]}}|(be}
\toendnotes[C]{\smallbreak\pagebreak[2]}\Standort{DLA, A:Schnitzler, HS.NZ85.1.3172.}
\physDesc{Brief, 1 Blatt, 2 Seiten, 298 Zeichen
\newline{}Handschrift: blaue Tinte, deutsche Kurrent
\newline{}Schnitzler: 1) mit Bleistift das Jahr »902« vermerkt  2) mit rotem Buntstift drei Unterstreichungen}\toendnotes[C]{\smallbreak}
\pstart
           \raggedleft{}{\pb}\textcolor{gray}{\textbf{DESSAUERSTRASSE 19}}\oindex{Dessauer Strasse@\textbf{Dessauer Straße}, \emph{Straße (K.STR)}|pw}\pend
           
\pstart
           Berlin\oindex{Berlin@\textbf{Berlin}, \emph{P.PPLC}|pw}, 14. Okt.\pend
           
\pstart\center{}Mein lieber Freund,\pend\vspace{0.5em}
\pstart
           \textsc{Coschell\pwindex{Coschell, Moritz 1872-09-18 – 1943-07-11@\textsc{Coschell, Moritz} (1872-09-18 – 1943-07-11), \emph{Maler/Malerin}|pw}} iſt gar nicht in Berlin\oindex{Berlin@\textbf{Berlin}, \emph{P.PPLC}|pw}. Er macht Studien
               zu ſeinem \label{K_L03227-1v}\edtext{jüdiſchen Gemälde\pwindex{?? [Juedisches Gemaelde]@\emph{?? [Jüdisches Gemälde]}|pwv}}{\lemma{\textnormal{\emph{jüdiſchen Gemälde}}}\Cendnote{\textnormal{nicht ermittelt}}}\label{K_L03227-1} in \textsc{Stanislau\oindex{Iwano-Frankiwsk@\textbf{Iwano-Frankiwsk}, \emph{P.PPLA}|pw}}.\pend
           
\pstart
           \label{K_L03227-2v}\edtext{\textsc{Gusti\pwindex{Gluemer, Auguste 1862-03-16 – 1956@\textsc{Glümer, Auguste} (1862-03-16 – 1956), \emph{Lehrer/Lehrerin}|pw}}}{\lemma{\textnormal{\emph{Gusti}}}\Cendnote{\textnormal{Schnitzler traf Auguste Glümer\pwindex{Gluemer, Auguste 1862-03-16 – 1956@\textsc{Glümer, Auguste} (1862-03-16 – 1956), \emph{Lehrer/Lehrerin}|pwk} am Folgetag, dem 15. 10. 1902.}}}\label{K_L03227-2}
               wird ſich mit Dir in Verbindung setzen.\pend
           
\pstart
           \textsc{Mizzi\pwindex{Gluemer, Marie 03.07.1867 – 16.11.1925@\textsc{Glümer, Marie} (03.07.1867 – 16.11.1925), \emph{Schauspieler/Schauspielerin}|pw}} iſt krank. Sie {\pb}hat ihre alten Kopfſchmerzen
               u. wohnt im \textsc{Grunewald\oindex{Grunewald@\textbf{Grunewald}, \emph{P.PPLX}|pw}}, \textsc{Café Grunewald\oindex{Cafe Grunewald@\textbf{Café Grunewald}, \emph{Kaffeehaus (K.KAF)}|pw}}.\pend
           
\pstart
           Auf Mittwoch{ }Abend, 7 Uhr!\pend
           
\pstart
           Herzlichſt {\\[\baselineskip]}Dein {\\[\baselineskip]}\spacefill\mbox{Paul Goldmn}\pend
           \leftskip=0em{}\selectlanguage{ngerman}\endnumbering\briefempfaengerindex{Schnitzler, Arthur@\textsc{Schnitzler, Arthur}!zzzGoldmann, Paul@\emph{von Paul Goldmann}!1902-10-141@{14. 10. {[}1902{]}}|)be}\mylabel{L03227h}  \normalsize

\doendnotes{C}
\bigskip
\vfill

\clearpage

\footnotesize

\lohead{\textsc{register}}

% Definiere theindex-Environment komplett neu ohne reledmac
\makeatletter
\renewenvironment{theindex}{%
  \section*{\indexname}%
  \setlength{\parindent}{0pt}%
  \setlength{\parskip}{0pt plus 0.3pt}%
  \let\item\@idxitem
}{%
  \clearpage
}
\makeatother

\IfFileExists{\jobname-pw.ind}{\input{\jobname-pw.ind}}{}

\end{document}

      