%% latex-leseansicht-vorspann.tex
%% Vorspann für die Leseansicht.
%% Lädt die gemeinsame Datei latex-vorspann.tex mit nicht gesetztem Schalter.

\newif\ifkorrekturansicht
\korrekturansichtfalse

\input{../tex-inputs/latex-vorspann}


\section[ Paul Goldmann an Arthur Schnitzler, 14. 10. [1902]]{L03227 Paul Goldmann an Arthur Schnitzler,  14. 10. [1902]}
\nopagebreak\mylabel{L03227v}
\rehead{ }\normalsize\beginnumbering\briefempfaengerindex{Schnitzler, Arthur@\textsc{Schnitzler, Arthur}!zzzGoldmann, Paul@\emph{von Paul Goldmann}!1902-10-141@{14. 10. [1902]}|(be}
\toendnotes[C]{\smallbreak\pagebreak[2]}
\correspDesc{Versand  durch Paul Goldmann am 14. 10. [1902] in Berlin
\newline{}Erhalt  durch Arthur Schnitzler im Zeitraum [14. 10. 1902 – 15. 10. 1902?] in Berlin}\toendnotes[C]{\smallbreak}
\Standort{DLA, A:Schnitzler, HS.NZ85.1.3172.}
\physDesc{Brief, 1 Blatt, 2 Seiten, 298 Zeichen
\newline{}Handschrift: blaue Tinte, deutsche Kurrent
\newline{}Schnitzler: 1) mit Bleistift das Jahr »902« vermerkt  2) mit rotem Buntstift drei Unterstreichungen}\toendnotes[C]{\smallbreak}
\pstart
           \raggedleft{}{\pb}\textcolor{gray}{\textbf{DESSAUERSTRASSE 19}}\oindex{Dessauer Straße@\textbf{Dessauer Straße}, \emph{Straße}|pw}\pend
           
\pstart
           Berlin\oindex{Berlin@\textbf{Berlin}, \emph{Hauptstadt}|pw}, 14. Okt.\pend
           
\pstart\center{}Mein lieber Freund,\pend\vspace{0.5em}
\pstart
           \textsc{Coschell\pwindex{Coschell, Moritz 18.\,9.\,1872 Wien – 11.\,7.\,1943 ebd.@\textsc{Coschell, Moritz} (18.\,9.\,1872 Wien – 11.\,7.\,1943 ebd.), \emph{Maler}|pw}} iſt gar nicht in Berlin\oindex{Berlin@\textbf{Berlin}, \emph{Hauptstadt}|pw}. Er macht Studien
               zu{ }ſeinem \label{K_L03227-1v}\edtext{jüdiſchen Gemälde\pwindex{Coschell, Moritz 18.\,9.\,1872 Wien – 11.\,7.\,1943 ebd.@\textsc{Coschell, Moritz} (18.\,9.\,1872 Wien – 11.\,7.\,1943 ebd.), \emph{Maler}!?? [Jüdisches Gemälde]@\strich\emph{?? [Jüdisches Gemälde]}|pwv}}{\lemma{\textnormal{\emph{jüdischen Gemälde}}}\Cendnote{\textnormal{nicht ermittelt}}}\label{K_L03227-1} in \textsc{Stanislau\oindex{Iwano-Frankiwsk@\textbf{Iwano-Frankiwsk}|pw}}.\pend
           
\pstart
           \label{K_L03227-2v}\edtext{\textsc{Gusti\pwindex{Glümer, Auguste 16.\,3.\,1862 Wien – 1956@\textsc{Glümer, Auguste} (16.\,3.\,1862 Wien – 1956), \emph{Lehrerin}|pw}}}{\lemma{\textnormal{\emph{Gusti}}}\Cendnote{\textnormal{Schnitzler traf Auguste Glümer\pwindex{Glümer, Auguste 16.\,3.\,1862 Wien – 1956@\textsc{Glümer, Auguste} (16.\,3.\,1862 Wien – 1956), \emph{Lehrerin}|pwk} am Folgetag, dem 15. 10. 1902.}}}\label{K_L03227-2}
               wird{ }ſich mit Dir in Verbindung setzen.\pend
           
\pstart
           \textsc{Mizzi\pwindex{Glümer, Marie 3.\,7.\,1867 Wien – 16.\,11.\,1925 München@\textsc{Glümer, Marie} (3.\,7.\,1867 Wien – 16.\,11.\,1925 München), \emph{Schauspielerin}|pw}} iſt krank. Sie {\pb}hat ihre alten Kopfſchmerzen
               u. wohnt im \textsc{Grunewald\oindex{Grunewald@\textbf{Grunewald}, \emph{Ehemaliger Ort}|pw}}, \textsc{Café Grunewald\oindex{Café Grunewald@\textbf{Café Grunewald}, \emph{Kaffeehaus}|pw}}.\pend
           
\pstart
           Auf Mittwoch{ }Abend, 7 Uhr!\pend
           
\pstart
           Herzlichſt {\\[\baselineskip]}Dein {\\[\baselineskip]}\spacefill\mbox{Paul Goldmn}\pend
           \leftskip=0em{}\selectlanguage{ngerman}\endnumbering\briefempfaengerindex{Schnitzler, Arthur@\textsc{Schnitzler, Arthur}!zzzGoldmann, Paul@\emph{von Paul Goldmann}!1902-10-141@{14. 10. [1902]}|)be}\mylabel{L03227h}  \newcommand{\dateiname}{L03227}\newcommand{\titel}{Paul Goldmann an Arthur Schnitzler, 14. 10. [1902]}\newcommand{\editorInnen}{Martin Anton Müller und Laura Untner}%% latex-leseansicht-abspann.tex
%% Abspann für die Leseansicht.
%% Der Schalter \ifkorrekturansicht ist bereits durch den Vorspann gesetzt.

%% latex-abspann.tex
%% Gemeinsamer Abspann für Korrekturansicht und Leseansicht.
%% Setzt den Schalter \ifkorrekturansicht voraus (gesetzt in den
%% einbindenden Dateien latex-korrekturansicht-abspann.tex bzw.
%% latex-leseansicht-abspann.tex).
%% ---------------------------------------------------------------

\normalsize

% Das esempio-Environment wird nur in der Leseansicht benötigt
\ifkorrekturansicht\else
\newenvironment{esempio}[3]%
{
    \vspace{1.5ex}
    \rlap{\underline{#1}}
    \par
    \setlength{\parindent}{0cm}
    \nopagebreak
    \leftskip=#2cm
    \rightskip=#3cm
}
{
    \par
}
\fi

\doendnotes{C}
\bigskip
\vfill

\clearpage

\footnotesize

\ifkorrekturansicht
  \lohead{\textsc{register}}
\fi

% theindex-Environment neu definieren ohne reledmac
\makeatletter
\renewenvironment{theindex}{%
  \ifkorrekturansicht
    \section*{\indexname}%
  \else
    \subsubsection*{Index der erwähnten Entitäten}%
  \fi
  \setlength{\parindent}{0pt}%
  \setlength{\parskip}{0pt plus 0.3pt}%
  \let\item\@idxitem
}{%
  \ifkorrekturansicht\clearpage\fi
}
\makeatother

\IfFileExists{\jobname-pw.ind}{\input{\jobname-pw.ind}}{}

% Quellenangabe nur in der Leseansicht
\ifkorrekturansicht\else
% Fallback-Definitionen, falls die .tex-Datei \titel etc. nicht gesetzt hat
\providecommand{\titel}{}
\providecommand{\editorInnen}{}
\providecommand{\dateiname}{\jobname}

\vspace{3cm}

\vfill

\footnotesize
\textsc{Quelle}: \titel. Herausgegeben von {\editorInnen}. In: \emph{Arthur Schnitzler: Briefwechsel mit Autorinnen und Autoren}.
 Digitale Edition, https://schnitzler-briefe.acdh.oeaw.ac.at/{\dateiname}.html (Stand \today)
\fi

\end{document}


