%% latex-leseansicht-vorspann.tex
%% Vorspann für die Leseansicht.
%% Lädt die gemeinsame Datei latex-vorspann.tex mit nicht gesetztem Schalter.

\newif\ifkorrekturansicht
\korrekturansichtfalse

\input{../tex-inputs/latex-vorspann}


\section[Robert Adam an Arthur Schnitzler, 1. 10. 1918]{L02306 Robert Adam an Arthur Schnitzler, 1. 10. 1918}
\nopagebreak\mylabel{L02306v}
\rehead{ }\normalsize\beginnumbering\briefempfaengerindex{Schnitzler, Arthur@\textsc{Schnitzler, Arthur}!zzzAdam, Robert@\emph{von Robert Adam}!1918-10-011@{1. 10. 1918}|(be}
\toendnotes[C]{\smallbreak\pagebreak[2]}
\correspDesc{Versand  durch Robert Adam am 1. 10. 1918 in Wien
\newline{}Erhalt  durch Arthur Schnitzler im Zeitraum [1. 10. 1918
                  – 5. 10. 1918?] in Wien}\toendnotes[C]{\smallbreak}
\Standort{CUL, Schnitzler, B 1.}
\physDesc{Brief, 1 Blatt, 3 Seiten, 1006 Zeichen
\newline{}Handschrift: schwarze Tinte, deutsche Kurrent
\newline{}Schnitzler: 1) mit Bleistift beschriftet: »\textsc{Adam}«  2) mit rotem Buntstift zwei Unterstreichungen
\newline{}Ordnung: von unbekannter Hand nummeriert: »7« }\Standort{Wien, Österreichische Nationalbibliothek, Cod.ser. 52.269, 223 recto.}
\physDesc{Brief, maschinenschriftliche Abschrift, 1 Blatt, 1 Seite, 1006 Zeichen
\newline{}Schreibmaschine}\toendnotes[C]{\smallbreak}
\pstart
           \raggedleft{}{\pb}Wien\oindex{Wien@\textbf{Wien}, \emph{Verwaltungsgebiet}|pw}, am 1. Oktober 1918\pend
           
\pstart\center{}Hochverehrter Doktor!\pend\vspace{0.5em}
\pstart
           Ich vermute Sie von Ihrer Reiſe, die Ihnen hoffentlich Erholung gebracht hat, bereits
               nach Wien\oindex{Wien@\textbf{Wien}, \emph{Verwaltungsgebiet}|pw} zurückgekehrt und frage mich an, ob und
               wann Sie ein Beſuch nicht{ }ſtören würde. Es wäre mir{ }ſehr lieb, wenn ich über das
               Stück »Yppl\pwindex{Adam, Robert 20.\,4.\,1877 Wien – 16.\,10.\,1961 Baden bei Wien@\textsc{Adam, Robert} (20.\,4.\,1877 Wien – 16.\,10.\,1961 Baden bei Wien), \emph{Schriftsteller, Richter}!Yppl. Idylle in fünf Akten@\strich\emph{Yppl. Idylle in fünf Akten}|pw}« und über die Frage, ob nicht jetzt
               Schritte möglich wären, den »Neidhard\pwindex{Adam, Robert 20.\,4.\,1877 Wien – 16.\,10.\,1961 Baden bei Wien@\textsc{Adam, Robert} (20.\,4.\,1877 Wien – 16.\,10.\,1961 Baden bei Wien), \emph{Schriftsteller, Richter}!Neidhard@\strich\emph{Neidhard}|pw}« dem Burgtheater\oindex{Wien@\textbf{Wien}!I., Innere Stadt@\textbf{I., Innere Stadt}!Burgtheater@\textbf{Burgtheater}, \emph{Theater}|pw} näherzubringen, mit Ihnen{ }ſprechen
               könnte. Darf ich Ihnen hiebei eines der Bücher\pwindex{Geistesstörung und Verbrechen im Kindesalter@\emph{Geistesstörung und Verbrechen im Kindesalter}|pwv}\pwindex{\textcolor{red}{\textsuperscript{XXXX indx1}}!Minderjährige Verbrecher. (Versuch einer strafgerichtlichen Psychologie) mit Original-Gutachten von Berenini – Brusa – Colajanni – Negri – Nordau – Pierantoni@\strich\emph{Minderjährige Verbrecher. (Versuch einer strafgerichtlichen Psychologie) mit Original-Gutachten von Berenini – Brusa – Colajanni – Negri – Nordau – Pierantoni}|pwv} über jugend{\pb}liche Verbrecher (und welches?)
               mitbringen?\pend
           
\pstart
           Meine Urlaubswoche verlebte ich, vom Wetter nicht{ }ſehr begünſtigt, in der Welſ\oindex{Wels@\textbf{Wels}, \emph{Hauptstadt}|pw}er und Linz\oindex{Linz@\textbf{Linz}|pw}er Gegend; die Wanderungen waren, da ich zwei Laib Brot im Ruckſack
               mitſchleppen mußte, einigermaßen beſchwerlich, die Ernährungs- und Unterkunftsfragen
               nicht immer leicht zu löſen. Immerhin gab es{ }ſchöne Stunden in Wilhering\oindex{Wilhering@\textbf{Wilhering}, \emph{Verwaltungsgebiet}|pw}, Ottensheim\oindex{Ottensheim@\textbf{Ottensheim}, \emph{Verwaltungsgebiet}|pw}, Eberſtall-Zell\oindex{Eberstalzell@\textbf{Eberstalzell}|pw}, Vorchdorf\oindex{Vorchdorf@\textbf{Vorchdorf}, \emph{Verwaltungsgebiet}|pw}, St. Florian\oindex{Sankt Florian@\textbf{Sankt Florian}|pw} und auf dem Pöſtlingberg\oindex{Pöstlingberg@\textbf{Pöstlingberg}|pw}. Näheres – falls Sie es
               intereſſieren{ }ſollte – hoffe ich Ihnen münd{\pb}lich mitteilen zu können.\pend
           
\pstart
           Mit den ergebenſten Grüßen Ihr\pend
           \pstart \spacefill\mbox{D\textsuperscript{r}RAdam}\pend{}\selectlanguage{ngerman}\endnumbering\briefempfaengerindex{Schnitzler, Arthur@\textsc{Schnitzler, Arthur}!zzzAdam, Robert@\emph{von Robert Adam}!1918-10-011@{1. 10. 1918}|)be}\mylabel{L02306h}  \newcommand{\dateiname}{L02306}\newcommand{\titel}{Robert Adam an Arthur Schnitzler, 1. 10. 1918}\newcommand{\editorInnen}{Martin Anton Müller und Gerd-Hermann Susen}%% latex-leseansicht-abspann.tex
%% Abspann für die Leseansicht.
%% Der Schalter \ifkorrekturansicht ist bereits durch den Vorspann gesetzt.

%% latex-abspann.tex
%% Gemeinsamer Abspann für Korrekturansicht und Leseansicht.
%% Setzt den Schalter \ifkorrekturansicht voraus (gesetzt in den
%% einbindenden Dateien latex-korrekturansicht-abspann.tex bzw.
%% latex-leseansicht-abspann.tex).
%% ---------------------------------------------------------------

\normalsize

% Das esempio-Environment wird nur in der Leseansicht benötigt
\ifkorrekturansicht\else
\newenvironment{esempio}[3]%
{
    \vspace{1.5ex}
    \rlap{\underline{#1}}
    \par
    \setlength{\parindent}{0cm}
    \nopagebreak
    \leftskip=#2cm
    \rightskip=#3cm
}
{
    \par
}
\fi

\doendnotes{C}
\bigskip
\vfill

\clearpage

\footnotesize

\ifkorrekturansicht
  \lohead{\textsc{register}}
\fi

% theindex-Environment neu definieren ohne reledmac
\makeatletter
\renewenvironment{theindex}{%
  \ifkorrekturansicht
    \section*{\indexname}%
  \else
    \subsubsection*{Index der erwähnten Entitäten}%
  \fi
  \setlength{\parindent}{0pt}%
  \setlength{\parskip}{0pt plus 0.3pt}%
  \let\item\@idxitem
}{%
  \ifkorrekturansicht\clearpage\fi
}
\makeatother

\IfFileExists{\jobname-pw.ind}{\input{\jobname-pw.ind}}{}

% Quellenangabe nur in der Leseansicht
\ifkorrekturansicht\else
% Fallback-Definitionen, falls die .tex-Datei \titel etc. nicht gesetzt hat
\providecommand{\titel}{}
\providecommand{\editorInnen}{}
\providecommand{\dateiname}{\jobname}

\vspace{3cm}

\vfill

\footnotesize
\textsc{Quelle}: \titel. Herausgegeben von {\editorInnen}. In: \emph{Arthur Schnitzler: Briefwechsel mit Autorinnen und Autoren}.
 Digitale Edition, https://schnitzler-briefe.acdh.oeaw.ac.at/{\dateiname}.html (Stand \today)
\fi

\end{document}


