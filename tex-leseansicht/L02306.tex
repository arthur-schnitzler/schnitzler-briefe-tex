%% latex-korrekturansicht-vorspann.tex
%% Vorspann für die Korrekturansicht.
%% Lädt die gemeinsame Datei latex-vorspann.tex mit gesetztem Schalter.

\newif\ifkorrekturansicht
\korrekturansichttrue

\input{../tex-inputs/latex-vorspann}


\section[Robert Adam an Arthur Schnitzler, 1. 10. 1918]{L02306 Robert Adam an Arthur Schnitzler, 1. 10. 1918}
\nopagebreak\mylabel{L02306v}
\rehead{ }\normalsize\beginnumbering\briefempfaengerindex{Schnitzler, Arthur@\textsc{Schnitzler, Arthur}!zzzAdam, Robert@\emph{von Robert Adam}!1918-10-011@{1. 10. 1918}|(be}
\toendnotes[C]{\smallbreak\pagebreak[2]}\Standort{CUL, Schnitzler, B 1.}
\physDesc{Brief, 1 Blatt, 3 Seiten, 1006 Zeichen
\newline{}Handschrift: schwarze Tinte, deutsche Kurrent
\newline{}Schnitzler: 1) mit Bleistift beschriftet: »\textsc{Adam}«  2) mit rotem Buntstift zwei Unterstreichungen
\newline{}Ordnung: von unbekannter Hand nummeriert: »7« }\Standort{Wien, Österreichische Nationalbibliothek, Cod.ser. 52.269, 223 recto.}
\physDesc{Brief, maschinenschriftliche Abschrift1 Blatt, 1 Seite, 1006 Zeichen
\newline{}Schreibmaschine}\toendnotes[C]{\smallbreak}
\pstart
           \raggedleft{}{\pb}Wien\oindex{Wien@\textbf{Wien}, \emph{A.ADM2}|pw}, am 1. Oktober 1918\pend
           
\pstart\center{}Hochverehrter Doktor!\pend\vspace{0.5em}
\pstart
           Ich vermute Sie von Ihrer Reiſe, die Ihnen hoffentlich Erholung gebracht hat, bereits
               nach Wien\oindex{Wien@\textbf{Wien}, \emph{A.ADM2}|pw} zurückgekehrt und frage mich an, ob und
               wann Sie ein Beſuch nicht ſtören würde. Es wäre mir ſehr lieb, wenn ich über das
               Stück »Yppl\pwindex{Yppl. Idylle in fuenf Akten@\emph{Yppl. Idylle in fünf Akten}|pw}« und über die Frage, ob nicht jetzt
               Schritte möglich wären, den »Neidhard\pwindex{Neidhard@\emph{Neidhard}|pw}« dem Burgtheater\oindex{Burgtheater@\textbf{Burgtheater}, \emph{S.THTR}|pw} näherzubringen, mit Ihnen ſprechen
               könnte. Darf ich Ihnen hiebei eines der Bücher\pwindex{Geistesstoerung und Verbrechen im Kindesalter@\emph{Geistesstörung und Verbrechen im Kindesalter}|pwv}\pwindex{Minderjaehrige Verbrecher. (Versuch einer strafgerichtlichen Psychologie) mit Original-Gutachten von Berenini – Brusa – Colajanni – Negri – Nordau – Pierantoni@\emph{Minderjährige Verbrecher. (Versuch einer strafgerichtlichen Psychologie) mit Original-Gutachten von Berenini – Brusa – Colajanni – Negri – Nordau – Pierantoni}|pwv} über jugend{\pb}liche Verbrecher (und welches?)
               mitbringen?\pend
           
\pstart
           Meine Urlaubswoche verlebte ich, vom Wetter nicht ſehr begünſtigt, in der Welſ\oindex{Wels@\textbf{Wels}, \emph{P.PPLA2}|pw}er und Linz\oindex{Linz@\textbf{Linz}, \emph{P.PPLA}|pw}er Gegend; die Wanderungen waren, da ich zwei Laib Brot im Ruckſack
               mitſchleppen mußte, einigermaßen beſchwerlich, die Ernährungs- und Unterkunftsfragen
               nicht immer leicht zu löſen. Immerhin gab es ſchöne Stunden in Wilhering\oindex{Wilhering@\textbf{Wilhering}, \emph{A.ADM3}|pw}, Ottensheim\oindex{Ottensheim@\textbf{Ottensheim}, \emph{A.ADM3}|pw}, Eberſtall-Zell\oindex{Eberstalzell@\textbf{Eberstalzell}, \emph{P.PPL}|pw}, Vorchdorf\oindex{Vorchdorf@\textbf{Vorchdorf}, \emph{A.ADM3}|pw}, St. Florian\oindex{Sankt Florian@\textbf{Sankt Florian}, \emph{P.PPL}|pw} und auf dem Pöſtlingberg\oindex{Poestlingberg@\textbf{Pöstlingberg}, \emph{P.PPL}|pw}. Näheres – falls Sie es
               intereſſieren ſollte – hoffe ich Ihnen münd{\pb}lich mitteilen zu können.\pend
           
\pstart
           Mit den ergebenſten Grüßen Ihr\pend
           \pstart \spacefill\mbox{D\textsuperscript{r}RAdam}\pend{}\selectlanguage{ngerman}\endnumbering\briefempfaengerindex{Schnitzler, Arthur@\textsc{Schnitzler, Arthur}!zzzAdam, Robert@\emph{von Robert Adam}!1918-10-011@{1. 10. 1918}|)be}\mylabel{L02306h}  \normalsize

\doendnotes{C}
\bigskip
\vfill

\clearpage

\footnotesize

\lohead{\textsc{register}}

% Definiere theindex-Environment komplett neu ohne reledmac
\makeatletter
\renewenvironment{theindex}{%
  \section*{\indexname}%
  \setlength{\parindent}{0pt}%
  \setlength{\parskip}{0pt plus 0.3pt}%
  \let\item\@idxitem
}{%
  \clearpage
}
\makeatother

\IfFileExists{\jobname-pw.ind}{\input{\jobname-pw.ind}}{}

\end{document}

      