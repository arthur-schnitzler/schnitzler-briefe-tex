%% latex-leseansicht-vorspann.tex
%% Vorspann für die Leseansicht.
%% Lädt die gemeinsame Datei latex-vorspann.tex mit nicht gesetztem Schalter.

\newif\ifkorrekturansicht
\korrekturansichtfalse

\input{../tex-inputs/latex-vorspann}

\begin{center}
            \textcolor{red}{ENTWURF, NICHT FERTIG KORRIGIERT}
                      \end{center}
            
         
         \renewcommand{\erwaehntePersonen}{Personen:  ?? [Arzt von Ottilie Salten], Hermann Bahr, Anna Bahr-Mildenburg, Hugo Ganz, Robert Hirschfeld, Eduard Pötzl, Paul Salten, Gustav Schwarzkopf, Adalbert Franz Seligmann, Friedrich Viktor Spitzer}
         \renewcommand{\erwaehnteOrte}{Orte: Armbrustergasse, Athen, Heiligenstadt, Hotel Quisisana, Hotel Waldmühle, I., Innere Stadt, Karlsbad, Lido, Marienbad, Oper, Pustertal, Welsberg-Taisten, Wien, Wildbad Waldbrunn, XIX., Döbling}
         \renewcommand{\erwaehnteWerke}{Werke: Das gelobte Wien, Der Wiener Korrespondent, Morgen. Wochenschrift für deutsche Kultur, Neues Wiener Tagblatt, Wien. Mit acht Vollbildern}
               \section[Felix und Ottilie Salten an Arthur Schnitzler, 3. 8. 1907]{ Felix und Ottilie Salten an Arthur Schnitzler, 3. 8. 1907}\nopagebreak\mylabel{v}\rehead{ }\begin{ledgroupsized}[t]{13cm}\normalsize\beginnumbering \toendnotes[C]{\smallbreak\pagebreak[2]} \Standort{CUL, Schnitzler, B 89, B 1.}
\physDesc{Postkarte, 2016 Zeichen
\newline{}Handschrift: schwarze Tinte, lateinische Kurrent
\newline{}Versand: 1) Stempel: »\nobreak{}\oindex{I., Innere Stadt@\textbf{I., Innere Stadt}|pwk}1/\textsubscript{1} Wien
                                       13, 3. VIII. 07, 6\nobreak{}«.   2) Stempel: »\nobreak{}\oindex{Welsberg-Taisten@\textbf{Welsberg-Taisten}|pwk}{[}Welsberg{]}, 4. 8. \textcolor{gray}{07}\nobreak{}«. 
\newline{}Schnitzler: mit Bleistift sechs Unterstreichungen 
\newline{}Ordnung: mit Bleistift von unbekannter Hand nummeriert:
                                    »232« }\toendnotes[C]{\smallbreak}\pstart{}{\pb}Salten\pend{}\pstart{}Wien XIX.\oindex{XIX., Doebling@\textbf{XIX., Döbling}|pw}\pend{}\pstart{}Armbrustergasse 6\oindex{Armbrustergasse@\textbf{Armbrustergasse}|pw}\pend{}{\bigskip}\pstart{}Herrn D\textsuperscript{r} Arthur Schnitzler\pend{}\pstart{}Wildbald Waldbrunn bei/ Welsberg\oindex{Wildbad Waldbrunn@\textbf{Wildbad Waldbrunn}|pw}\pend{}\pstart{}Pustertal\oindex{Pustertal@\textbf{Pustertal}|pw}\pend{}{\bigskip}\pstart
           \raggedleft{}{\pb}Heiligenstadt\oindex{Heiligenstadt@\textbf{Heiligenstadt}|pw}, 3. VIII. 07\pend
           \pstart
           Lieber, ich habe Ihre letzte Karte nicht gut lesen können, glaube
               aber dass Sie noch in Waldbrunn\oindex{Wildbad Waldbrunn@\textbf{Wildbad Waldbrunn}|pw} sind. Uns ist es
               nicht besonders gegangen.Otti\pwindex{Salten, Ottilie 07.03.1868 – 22.06.1942@\textsc{Salten, Ottilie} (07.03.1868 – 22.06.1942), \emph{Schauspielerin}|pw} mußte operirt
               werden, was zu Hause geschah. Sie hat sich bis heute noch nicht völlig erholt. Der
                  Arzt\pwindex{?? [Arzt von Ottilie Salten] @\textsc{?? [Arzt von Ottilie Salten]}|pwv} will, dass sie
               jetzt noch eine Kur brauchen soll. So gehen wir nächster Tage auf 4 Wochen nach Marienbad\oindex{Marienbad@\textbf{Marienbad}|pw}. Ich komme eben von dort, wo ich Wohnung
               genommen habe. Vorher war ich ein paar Tage in Karlsbad\oindex{Karlsbad@\textbf{Karlsbad}|pw}. Unsere Adreſse ist dann\textcolor{gray}{,} (wahrscheinlich vom 8.
               an) »Quisiana\oindex{Hotel Quisisana@\textbf{Hotel Quisisana}|pw}«. Ein sehr hübsches Haus, oben im
               Wald bei der Waldmühle\oindex{Hotel Waldmuehle@\textbf{Hotel Waldmühle}|pw}. Paul\pwindex{Salten, Paul 11.08.1903 – 08.05.1937@\textsc{Salten, Paul} (11.08.1903 – 08.05.1937), \emph{Filmcutter}|pw} ist dieser Tage auch wieder krank geworden, hoffentlich
               wird er sich in Marienbad\oindex{Marienbad@\textbf{Marienbad}|pw} vollständig erholen.
               Wann kommen Sie nach Wien\oindex{Wien@\textbf{Wien}|pw} zurück? Spielen Sie dort
               Tennis? Haben Sie gearbeitet? Haben Sie für den September Reisepläne? Ich möchte im
               September irgend eine Meerfahrt machen. Athen\oindex{Athen@\textbf{Athen}|pw}
               oder so was ähnliches. Bahr\pwindex{Bahr, Hermann 19.07.1863 – 15.01.1934@\textsc{Bahr, Hermann} (19.07.1863 – 15.01.1934), \emph{Schriftsteller, Kritiker}|pw} hat mir vom \uline{Lido}\oindex{Lido@\textbf{Lido}|pw} einen entrüsteten Brief geschrieben, weil mich der \label{K_L03489-1v}\edtext{Pötzl\pwindex{Poetzl, Eduard 17.03.1851 – 20.08.1914@\textsc{Pötzl, Eduard} (17.03.1851 – 20.08.1914), \emph{Schriftsteller, Journalist}|pw} im Tagblatt\pwindex{?? Werk@Nicht ermittelte Verfasserinnen und Verfasser!Neues Wiener Tagblatt1867 – 1945@\emph{Neues Wiener Tagblatt} {[}1867 – 1945{]}|pw}gelobt\pwindex{Poetzl, Eduard 17.03.1851 – 20.08.1914@\textsc{Pötzl, Eduard} (17.03.1851 – 20.08.1914), \emph{Schriftsteller, Journalist}!gelobte Wien28. 07. 1907@\strich\emph{Das gelobte Wien} {[}28. 07. 1907{]}|pwv}}{\lemma{\textnormal{\emph{Pötzl im Tagblattgelobt}}}\Cendnote{\textnormal{Ed. Pötzl\pwindex{Poetzl, Eduard 17.03.1851 – 20.08.1914@\textsc{Pötzl, Eduard} (17.03.1851 – 20.08.1914), \emph{Schriftsteller, Journalist}|pwk}: \emph{Das gelobte Wien}\pwindex{Poetzl, Eduard 17.03.1851 – 20.08.1914@\textsc{Pötzl, Eduard} (17.03.1851 – 20.08.1914), \emph{Schriftsteller, Journalist}!gelobte Wien28. 07. 1907@\strich\emph{Das gelobte Wien} {[}28. 07. 1907{]}|pwk}. In: \emph{Neues Wiener Tagblatt}\pwindex{?? Werk@Nicht ermittelte Verfasserinnen und Verfasser!Neues Wiener Tagblatt1867 – 1945@\emph{Neues Wiener Tagblatt} {[}1867 – 1945{]}|pwk}, Jg. 41, Nr. 204, 28. 7. 1907,
                     S. 1–3.}}}\label{K_L03489-1h} hat. Und der Pötzl\pwindex{Poetzl, Eduard 17.03.1851 – 20.08.1914@\textsc{Pötzl, Eduard} (17.03.1851 – 20.08.1914), \emph{Schriftsteller, Journalist}|pw}
               hat mich gelobt\pwindex{Poetzl, Eduard 17.03.1851 – 20.08.1914@\textsc{Pötzl, Eduard} (17.03.1851 – 20.08.1914), \emph{Schriftsteller, Journalist}!gelobte Wien28. 07. 1907@\strich\emph{Das gelobte Wien} {[}28. 07. 1907{]}|pwv}, weil ich im
                  »Morgen\pwindex{Morgen. Wochenschrift fuer deutsche Kultur1907 – 1908@\emph{Morgen. Wochenschrift für deutsche Kultur} {[}1907 – 1908{]}|pw}« \label{K_L03489-2v}\edtext{Wien\pwindex{Bahr, Hermann 19.07.1863 – 15.01.1934@\textsc{Bahr, Hermann} (19.07.1863 – 15.01.1934), \emph{Schriftsteller, Kritiker}!Wien. Mit acht Vollbildern1907@\strich\emph{Wien. Mit acht Vollbildern} {[}1907{]}|pw} gelobt}{\lemma{\textnormal{\emph{Wien gelobt}}}\Cendnote{\textnormal{Das Lob für Bahr\pwindex{Bahr, Hermann 19.07.1863 – 15.01.1934@\textsc{Bahr, Hermann} (19.07.1863 – 15.01.1934), \emph{Schriftsteller, Kritiker}|pwk}s
                  Abrechnungsbuch \emph{Wien}\pwindex{Bahr, Hermann 19.07.1863 – 15.01.1934@\textsc{Bahr, Hermann} (19.07.1863 – 15.01.1934), \emph{Schriftsteller, Kritiker}!Wien. Mit acht Vollbildern1907@\strich\emph{Wien. Mit acht Vollbildern} {[}1907{]}|pwk} nur implizit in Felix Salten\pwindex{Salten, Felix 06.09.1869 – 08.10.1945@\textsc{Salten, Felix} (06.09.1869 – 08.10.1945), \emph{Schriftsteller, Journalist}|pwk}: \emph{Der Wiener Korrespondent}\pwindex{Salten, Felix 06.09.1869 – 08.10.1945@\textsc{Salten, Felix} (06.09.1869 – 08.10.1945), \emph{Schriftsteller, Journalist}!Wiener Korrespondent05. 07. 1907@\strich\emph{Der Wiener Korrespondent} {[}05. 07. 1907{]}|pwk}. In: \emph{Der Morgen}\pwindex{Morgen. Wochenschrift fuer deutsche Kultur1907 – 1908@\emph{Morgen. Wochenschrift für deutsche Kultur} {[}1907 – 1908{]}|pwk}, Jg. 1, H. 4, 5. 7. 1907, S. 113–116.}}}\label{K_L03489-2h} habe. Es ist eine düstere Sache, wie
               Sie sehen. Aber was soll ich thun? Ich zittere, dass mich am Ende nächstens auch noch
               der Seligmann\pwindex{Seligmann, Adalbert Franz 02.04.1862 – 13.12.1945@\textsc{Seligmann, Adalbert Franz} (02.04.1862 – 13.12.1945), \emph{Bildender Künstler, Publizist}|pw} lobt, oder der Hugo Ganz\pwindex{Ganz, Hugo 24.04.1862 – 02.01.1922@\textsc{Ganz, Hugo} (24.04.1862 – 02.01.1922), \emph{Schriftsteller, Journalist}|pw} und dann wird mich Bahr\pwindex{Bahr, Hermann 19.07.1863 – 15.01.1934@\textsc{Bahr, Hermann} (19.07.1863 – 15.01.1934), \emph{Schriftsteller, Kritiker}|pw} sicherlich total verachten, und komme ich einmal in die Oper\oindex{Oper@\textbf{Oper}|pw}, wird die M.\pwindex{Bahr-Mildenburg, Anna 29.11.1872 – 27.01.1947@\textsc{Bahr-Mildenburg, Anna} (29.11.1872 – 27.01.1947), \emph{Sängerin}|pw} zu singen aufhören, weil ich da bin. Mir fehlt zu meinem gänzlichen
               Untergang nur noch, dass Robert Hirschfeld\pwindex{Hirschfeld, Robert 17.09.1857 – 02.04.1914@\textsc{Hirschfeld, Robert} (17.09.1857 – 02.04.1914), \emph{Journalist, Kritiker}|pw} ein
               Feuilleton über mich schreibt, und dem Gustav
                  S-Kopf\pwindex{Schwarzkopf, Gustav 07.11.1853 – 13.11.1939@\textsc{Schwarzkopf, Gustav} (07.11.1853 – 13.11.1939), \emph{Schriftsteller}|pw} in einem Aufruf die Wien\oindex{Wien@\textbf{Wien}|pw}er einlädt,
               meine Bücher fleißiger zu kaufen. Dann bin ich ganz kaput\textcolor{gray}{,} und
               kann mich von D\textsuperscript{r} Spitzer\pwindex{Spitzer, Friedrich Viktor 05.02.1854 – 19.02.1922@\textsc{Spitzer, Friedrich Viktor} (05.02.1854 – 19.02.1922), \emph{Bildender Künstler, Gesangspädagoge, Industrieller}|pw} ehrlicher Weise nicht einmal mehr fotografiren laſsen. Ich habe
               trübe Ahnungen und bin auf das Schlimmste gefaſst. Aber, wenn’s mir bestimmt ist,
               kann ich garnichts machen. – Hoffentlich geht es Ihnen allen gut. \pend
           \pstart
           Leben Sie wol und schreiben Sie bald wieder eine Zeile. Herzliche Grüße von uns zu
               Ihnen.\pend
           \pstart Ihr \spacefill\mbox{FSalten}\pend{}\pstart
           \noindent{}{[}hs. Salten:{]} Viele herzliche Grüße\pend
           \pstart \spacefill\mbox{Ottilie S.}\pend{}
         
         \endnumbering\mylabel{h}\end{ledgroupsized}\begin{anhang}\end{anhang}\newcommand{\dateiname}{L03489}\newcommand{\titel}{Felix und Ottilie Salten an Arthur Schnitzler, 3. 8. 1907}\newcommand{\editorInnen}{Martin Anton Müller und Laura Untner}%% latex-leseansicht-abspann.tex
%% Abspann für die Leseansicht.
%% Der Schalter \ifkorrekturansicht ist bereits durch den Vorspann gesetzt.

%% latex-abspann.tex
%% Gemeinsamer Abspann für Korrekturansicht und Leseansicht.
%% Setzt den Schalter \ifkorrekturansicht voraus (gesetzt in den
%% einbindenden Dateien latex-korrekturansicht-abspann.tex bzw.
%% latex-leseansicht-abspann.tex).
%% ---------------------------------------------------------------

\normalsize

% Das esempio-Environment wird nur in der Leseansicht benötigt
\ifkorrekturansicht\else
\newenvironment{esempio}[3]%
{
    \vspace{1.5ex}
    \rlap{\underline{#1}}
    \par
    \setlength{\parindent}{0cm}
    \nopagebreak
    \leftskip=#2cm
    \rightskip=#3cm
}
{
    \par
}
\fi

\doendnotes{C}
\bigskip
\vfill

\clearpage

\footnotesize

\ifkorrekturansicht
  \lohead{\textsc{register}}
\fi

% theindex-Environment neu definieren ohne reledmac
\makeatletter
\renewenvironment{theindex}{%
  \ifkorrekturansicht
    \section*{\indexname}%
  \else
    \subsubsection*{Index der erwähnten Entitäten}%
  \fi
  \setlength{\parindent}{0pt}%
  \setlength{\parskip}{0pt plus 0.3pt}%
  \let\item\@idxitem
}{%
  \ifkorrekturansicht\clearpage\fi
}
\makeatother

\IfFileExists{\jobname-pw.ind}{\input{\jobname-pw.ind}}{}

% Quellenangabe nur in der Leseansicht
\ifkorrekturansicht\else
% Fallback-Definitionen, falls die .tex-Datei \titel etc. nicht gesetzt hat
\providecommand{\titel}{}
\providecommand{\editorInnen}{}
\providecommand{\dateiname}{\jobname}

\vspace{3cm}

\vfill

\footnotesize
\textsc{Quelle}: \titel. Herausgegeben von {\editorInnen}. In: \emph{Arthur Schnitzler: Briefwechsel mit Autorinnen und Autoren}.
 Digitale Edition, https://schnitzler-briefe.acdh.oeaw.ac.at/{\dateiname}.html (Stand \today)
\fi

\end{document}


      