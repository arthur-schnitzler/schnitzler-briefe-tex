%% latex-korrekturansicht-vorspann.tex
%% Vorspann für die Korrekturansicht.
%% Lädt die gemeinsame Datei latex-vorspann.tex mit gesetztem Schalter.

\newif\ifkorrekturansicht
\korrekturansichttrue

\input{../tex-inputs/latex-vorspann}


\section[ Felix und Ottilie Salten an Arthur Schnitzler, 3. 8. 1907]{L03489 Felix und Ottilie Salten an Arthur Schnitzler, 3. 8. 1907}
\nopagebreak\mylabel{L03489v}
\rehead{ }\normalsize\beginnumbering\briefempfaengerindex{Schnitzler, Arthur@\textsc{Schnitzler, Arthur}!zzzSalten, Ottilie@\emph{von Ottilie Salten}!1907-08-032@{3. 8. 1907}|(be}\briefempfaengerindex{Schnitzler, Arthur@\textsc{Schnitzler, Arthur}!zzzSalten, Felix@\emph{von Felix Salten}!1907-08-032@{3. 8. 1907}|(be}
\toendnotes[C]{\smallbreak\pagebreak[2]}\Standort{CUL, Schnitzler, B 89, B 1.}
\physDesc{Postkarte, 2006 Zeichen
\newline{}Handschrift Felix Salten: schwarze Tinte, lateinische Kurrent
\newline{}Handschrift Ottilie Salten: schwarze Tinte, lateinische Kurrent
\newline{}Versand: 1) Stempel: »\nobreak{}\oindex{I., Innere Stadt@\textbf{I., Innere Stadt}, \emph{A.ADM3}|pwk}1/1 Wien 13, 3. VIII. 07, 6\nobreak{}«. Stempel: »\nobreak{}\oindex{Welsberg-Taisten@\textbf{Welsberg-Taisten}, \emph{A.ADM3}|pwk}We{[}lsber{]}g, 4. 8. \textcolor{gray}{07}\nobreak{}«.   2) mit Bleistift beschriftet: »III 9–\substVorne{}\textsuperscript{\textcolor{gray}{11}}\substDazwischen{}4\substHinten{}«
\newline{}Schnitzler: mit Bleistift sechs Unterstreichungen 
\newline{}Ordnung: mit Bleistift von unbekannter Hand nummeriert: »232« }
\buchAbdrucke{\weitereDrucke{Hermann Bahr, Arthur Schnitzler: \emph{Briefwechsel, Aufzeichnungen, Dokumente (1891–1931)}. Göttingen: \emph{Wallstein} 2018, S. 394–395.} }\toendnotes[C]{\smallbreak}\pstart{}{\pb}Salten\pend{}\pstart{}Wien XIX.\oindex{XIX., Doebling@\textbf{XIX., Döbling}, \emph{A.ADM3}|pw}\pend{}\pstart{}Armbrustergasse 6\oindex{Armbrustergasse@\textbf{Armbrustergasse}, \emph{R.ST}|pw}\pend{}{\bigskip}\pstart{}Herrn D\textsuperscript{r} Arthur Schnitzler\pend{}\pstart{}Wildbald Waldbrunn bei/ Welsberg\oindex{Wildbad Waldbrunn@\textbf{Wildbad Waldbrunn}, \emph{S.SPA}|pw}\pend{}\pstart{}i Pustertal\oindex{Pustertal@\textbf{Pustertal}, \emph{T.VAL}|pw}\pend{}{\bigskip}\vspace{1em}
\pstart
           \raggedleft{}{\pb}Heiligenstadt\oindex{Heiligenstadt@\textbf{Heiligenstadt}, \emph{P.PPL}|pw}, 3. VIII. 07\pend
           \vspace{0.5em}
\pstart
           Lieber, ich habe Ihre letzte Karte nicht gut lesen können, glaube
               aber dass Sie noch in \label{K_L03489-1v}\edtext{Waldbrunn\oindex{Wildbad Waldbrunn@\textbf{Wildbad Waldbrunn}, \emph{S.SPA}|pw}}{\lemma{\textnormal{\emph{Waldbrunn}}}\Cendnote{\textnormal{Siehe Felix Salten an Arthur Schnitzler, 15. 7. 1907.
               }}}\label{K_L03489-1} sind. Uns ist es nicht besonders gegangen. Otti\pwindex{Salten, Ottilie 07.03.1868 – 22.06.1942@\textsc{Salten, Ottilie} (07.03.1868 – 22.06.1942), \emph{Schauspieler/Schauspielerin}|pw} mußte operirt werden, was zu Hause geschah. Sie hat sich bis heute noch nicht völlig erholt. Der Arzt\pwindex{?? [Arzt von Ottilie Salten] @\textsc{?? [Arzt von Ottilie Salten]}|pwv} will, dass sie jetzt noch eine Kur
               brauchen soll. So gehen wir nächster Tage auf 4 Wochen nach Marienbad\oindex{Marienbad@\textbf{Marienbad}, \emph{P.PPL}|pw}. Ich komme eben von dort, wo ich Wohnung genommen
               habe. Vorher war ich ein paar Tage in Karlsbad\oindex{Karlsbad@\textbf{Karlsbad}, \emph{P.PPLA}|pw}.
               Unsere Adreße ist dann (wahrscheinlich vom 8\textsuperscript{\textcolor{gray}{ten}} an) »Quisisana\oindex{Hotel Quisisana@\textbf{Hotel Quisisana}, \emph{Hotel (K.HTL)}|pw}«. Ein sehr hübsches Haus,
               oben im Wald bei der Waldmühle\oindex{Hotel Waldmuehle [Karlsbad]@\textbf{Hotel Waldmühle [Karlsbad]}, \emph{Hotel (K.HTL)}|pw}. Paul\pwindex{Salten, Paul 11.08.1903 – 08.05.1937@\textsc{Salten, Paul} (11.08.1903 – 08.05.1937), \emph{Filmcutter/Filmcutterin}|pw} ist dieser Tage auch wieder krank gewesen, hoffentlich
               wird er sich in Marienbad\oindex{Marienbad@\textbf{Marienbad}, \emph{P.PPL}|pw} vollständig erholen.
               Wann kommen Sie \label{K_L03489-2v}\edtext{nach Wien\oindex{Wien@\textbf{Wien}, \emph{A.ADM2}|pw}}{\lemma{\textnormal{\emph{nach Wien}}}\Cendnote{\textnormal{Schnitzler kehrte am 12. 9. 1907 nach Wien\oindex{Wien@\textbf{Wien}, \emph{A.ADM2}|pwk} zurück.}}}\label{K_L03489-2} zurück? \label{K_L03489-3v}\edtext{Spielen Sie dort\oindex{Welsberg-Taisten@\textbf{Welsberg-Taisten}, \emph{A.ADM3}|pwv}
               Tennis}{\lemma{\textnormal{\emph{Spielen Sie dort
               Tennis}}}\Cendnote{\textnormal{Ja, 
                  siehe A. S.: \emph{Tagebuch}, 3. 8. 1907 und 5. 8. 1907.
               }}}\label{K_L03489-3}? Haben Sie gearbeitet? Haben Sie für den \label{K_L03489-4v}\edtext{September Reisepläne}{\lemma{\textnormal{\emph{September Reisepläne}}}\Cendnote{\textnormal{Arthur und Olga Schnitzler\pwindex{Schnitzler, Olga 17.01.1882 – 13.01.1970@\textsc{Schnitzler, Olga} (17.01.1882 – 13.01.1970), \emph{Schauspieler/Schauspielerin, Sänger/Sängerin}|pwk} reisten am 26. 8. 1907 von Welsberg\oindex{Welsberg-Taisten@\textbf{Welsberg-Taisten}, \emph{A.ADM3}|pwk} weiter durch Südtirol\oindex{Suedtirol@\textbf{Südtirol}, \emph{A.ADM2}|pwk}.}}}\label{K_L03489-4}? Ich
               möchte im September irgend eine Meerfahrt machen. Athen\oindex{Athen@\textbf{Athen}, \emph{P.PPLC}|pw} oder so was ähnliches. Bahr\pwindex{Bahr, Hermann 19.07.1863 – 15.01.1934@\textsc{Bahr, Hermann} (19.07.1863 – 15.01.1934), \emph{Schriftsteller/Schriftstellerin, Kritiker/Kritikerin}|pw} hat mir vom Lido\oindex{Lido@\textbf{Lido}, \emph{P.PPL}|pw}
               einen entrüsteten Brief geschrieben, weil mich der \label{K_L03489-5v}\edtext{Pötzl\pwindex{Poetzl, Eduard 17.03.1851 – 20.08.1914@\textsc{Pötzl, Eduard} (17.03.1851 – 20.08.1914), \emph{Schriftsteller/Schriftstellerin, Journalist/Journalistin}|pw} im Tagblatt\pwindex{Neues Wiener Tagblatt@\emph{Neues Wiener Tagblatt}|pw}{ }gelobt\pwindex{gelobte Wien@\emph{Das gelobte Wien}|pwv}}{\lemma{\textnormal{\emph{Pötzl im Tagblatt gelobt}}}\Cendnote{\textnormal{Ed. [ = Eduard] Pötzl\pwindex{Poetzl, Eduard 17.03.1851 – 20.08.1914@\textsc{Pötzl, Eduard} (17.03.1851 – 20.08.1914), \emph{Schriftsteller/Schriftstellerin, Journalist/Journalistin}|pwk}: \emph{Das gelobte Wien}\pwindex{gelobte Wien@\emph{Das gelobte Wien}|pwk}. In: \emph{Neues Wiener Tagblatt}\pwindex{Neues Wiener Tagblatt@\emph{Neues Wiener Tagblatt}|pwk}, Jg. 41, Nr. 204, 28. 7. 1907, S. 1–3. Pötzl\pwindex{Poetzl, Eduard 17.03.1851 – 20.08.1914@\textsc{Pötzl, Eduard} (17.03.1851 – 20.08.1914), \emph{Schriftsteller/Schriftstellerin, Journalist/Journalistin}|pwk}
                  war der Intimfeind Bahrs\pwindex{Bahr, Hermann 19.07.1863 – 15.01.1934@\textsc{Bahr, Hermann} (19.07.1863 – 15.01.1934), \emph{Schriftsteller/Schriftstellerin, Kritiker/Kritikerin}|pwk} beim \emph{Neuen Wiener Tagblatt}\orgindex{Neues Wiener Tagblatt@Neues Wiener Tagblatt|pwk}, vgl. Hermann Bahr an Arthur Schnitzler, 5. 7. 1901.
                  }}}\label{K_L03489-5} hat. Und der Pötzl\pwindex{Poetzl, Eduard 17.03.1851 – 20.08.1914@\textsc{Pötzl, Eduard} (17.03.1851 – 20.08.1914), \emph{Schriftsteller/Schriftstellerin, Journalist/Journalistin}|pw} hat mich gelobt\pwindex{gelobte Wien@\emph{Das gelobte Wien}|pwv}, weil ich im »Morgen\pwindex{Morgen. Wochenschrift fuer deutsche Kultur@\emph{Morgen. Wochenschrift für deutsche Kultur}|pw}« \label{K_L03489-6v}\edtext{Wien\pwindex{Wien. Mit acht Vollbildern@\emph{Wien. Mit acht Vollbildern}|pw}{ }gelobt\pwindex{Wiener Korrespondent@\emph{Der Wiener Korrespondent}|pwv}}{\lemma{\textnormal{\emph{Wien gelobt}}}\Cendnote{\textnormal{Das Lob für Bahrs\pwindex{Bahr, Hermann 19.07.1863 – 15.01.1934@\textsc{Bahr, Hermann} (19.07.1863 – 15.01.1934), \emph{Schriftsteller/Schriftstellerin, Kritiker/Kritikerin}|pwk} Abrechnungsbuch \emph{Wien}\pwindex{Wien. Mit acht Vollbildern@\emph{Wien. Mit acht Vollbildern}|pwk} findet sich nur implizit in Felix Salten\pwindex{Salten, Felix 06.09.1869 – 08.10.1945@\textsc{Salten, Felix} (06.09.1869 – 08.10.1945), \emph{Schriftsteller/Schriftstellerin, Journalist/Journalistin, Chefredakteur/Chefredakteurin}|pwk}: \emph{Der Wiener Korrespondent}\pwindex{Wiener Korrespondent@\emph{Der Wiener Korrespondent}|pwk}. In: \emph{Wochenschrift für deutsche Kultur}\pwindex{Morgen. Wochenschrift fuer deutsche Kultur@\emph{Morgen. Wochenschrift für deutsche Kultur}|pwk}, Jg. 1, H. 4, 5. 7. 1907, S. 113–116.}}}\label{K_L03489-6} habe. Es ist eine düstere
               Sache, wie Sie sehen. Aber was soll ich thun? Ich zittere, dass mich am Ende
               nächstens auch noch der Seligmann\pwindex{Seligmann, Adalbert Franz 02.04.1862 – 13.12.1945@\textsc{Seligmann, Adalbert Franz} (02.04.1862 – 13.12.1945), \emph{Maler/Malerin, Publizist/Publizistin}|pw} lobt, oder
               der Hugo Ganz\pwindex{Ganz, Hugo 24.04.1862 – 02.01.1922@\textsc{Ganz, Hugo} (24.04.1862 – 02.01.1922), \emph{Schriftsteller/Schriftstellerin, Journalist/Journalistin}|pw} und dann wird mich Bahr\pwindex{Bahr, Hermann 19.07.1863 – 15.01.1934@\textsc{Bahr, Hermann} (19.07.1863 – 15.01.1934), \emph{Schriftsteller/Schriftstellerin, Kritiker/Kritikerin}|pw} sicherlich total verachten, und komme ich
               einmal in die Oper\oindex{Oper@\textbf{Oper}, \emph{Oper (K.OPR)}|pw}, wird die \label{K_L03489-7v}\edtext{M.\pwindex{Bahr-Mildenburg, Anna 29.11.1872 – 27.01.1947@\textsc{Bahr-Mildenburg, Anna} (29.11.1872 – 27.01.1947), \emph{Sänger/Sängerin}|pw}}{\lemma{\textnormal{\emph{M.}}}\Cendnote{\textnormal{Anna Mildenburg\pwindex{Bahr-Mildenburg, Anna 29.11.1872 – 27.01.1947@\textsc{Bahr-Mildenburg, Anna} (29.11.1872 – 27.01.1947), \emph{Sänger/Sängerin}|pwk}, Hermann Bahrs\pwindex{Bahr, Hermann 19.07.1863 – 15.01.1934@\textsc{Bahr, Hermann} (19.07.1863 – 15.01.1934), \emph{Schriftsteller/Schriftstellerin, Kritiker/Kritikerin}|pwk} Partnerin und spätere zweite Ehefrau}}}\label{K_L03489-7}
               zu singen aufhören, weil ich da bin. Mir fehlt zu meinem gänzlichen Untergang nur
               noch, dass Robert Hirschfeld\pwindex{Hirschfeld, Robert 17.09.1857 – 02.04.1914@\textsc{Hirschfeld, Robert} (17.09.1857 – 02.04.1914), \emph{Journalist/Journalistin, Musikkritiker/Musikkritikerin}|pw} ein Feuilleton
               über mich schreibt, und dein Gustav S-kopf\pwindex{Schwarzkopf, Gustav 07.11.1853 – 13.11.1939@\textsc{Schwarzkopf, Gustav} (07.11.1853 – 13.11.1939), \emph{Schriftsteller/Schriftstellerin}|pw} in
               einem Aufruf die Wien\oindex{Wien@\textbf{Wien}, \emph{A.ADM2}|pw}er einlädt, meine Bücher
               fleißiger zu kaufen. Dann bin ich ganz kaput \textcolor{gray}{–} und kann mich von
                  D\textsuperscript{r}{ }Spitzer\pwindex{Spitzer, Friedrich Viktor 05.02.1854 – 19.02.1922@\textsc{Spitzer, Friedrich Viktor} (05.02.1854 – 19.02.1922), \emph{Fotograf/Fotografin, Gesangspädagoge/Gesangspädagogin, Industrieller/Industrielle}|pw} ehrlicher Weise nicht einmal mehr
               fotografiren laßen. Ich habe trübe Ahnungen und bin auf das Schlimmste gefaßt. Aber,
               wenn’s mir bestimmt ist, kann ich garnichts machen. – Hoffentlich geht es Ihnen allen
               gut.\pend
           
\pstart
           Leben Sie wol und schreiben Sie bald wieder eine Zeile. Herzliche Grüße von uns zu
                  Ihnen\pwindex{Schnitzler, Olga 17.01.1882 – 13.01.1970@\textsc{Schnitzler, Olga} (17.01.1882 – 13.01.1970), \emph{Schauspieler/Schauspielerin, Sänger/Sängerin}|pwv}.\pend
           \pstart Ihr \spacefill\mbox{FSalten}\pend{}\selectlanguage{ngerman}\vspace{1em}
\pstart
           \noindent{}{[}hs. :{]} Viele herzliche Grüße \spacefill\mbox{Ottilie S.}\pend
           \selectlanguage{ngerman}\endnumbering\briefempfaengerindex{Schnitzler, Arthur@\textsc{Schnitzler, Arthur}!zzzSalten, Ottilie@\emph{von Ottilie Salten}!1907-08-032@{3. 8. 1907}|)be}\briefempfaengerindex{Schnitzler, Arthur@\textsc{Schnitzler, Arthur}!zzzSalten, Felix@\emph{von Felix Salten}!1907-08-032@{3. 8. 1907}|)be}\mylabel{L03489h}  \normalsize

\doendnotes{C}
\bigskip
\vfill

\clearpage

\footnotesize

\lohead{\textsc{register}}

% Definiere theindex-Environment komplett neu ohne reledmac
\makeatletter
\renewenvironment{theindex}{%
  \section*{\indexname}%
  \setlength{\parindent}{0pt}%
  \setlength{\parskip}{0pt plus 0.3pt}%
  \let\item\@idxitem
}{%
  \clearpage
}
\makeatother

\IfFileExists{\jobname-pw.ind}{\input{\jobname-pw.ind}}{}

\end{document}

      