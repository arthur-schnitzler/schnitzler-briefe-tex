%% latex-korrekturansicht-vorspann.tex
%% Vorspann für die Korrekturansicht.
%% Lädt die gemeinsame Datei latex-vorspann.tex mit gesetztem Schalter.

\newif\ifkorrekturansicht
\korrekturansichttrue

\input{../tex-inputs/latex-vorspann}


\section[ Felix Salten an Arthur Schnitzler, 27. 12. 1917]{L03567 Felix Salten an Arthur Schnitzler, 27. 12. 1917}
\nopagebreak\mylabel{L03567v}
\rehead{ }\normalsize\beginnumbering\briefempfaengerindex{Schnitzler, Arthur@\textsc{Schnitzler, Arthur}!zzzSalten, Felix@\emph{von Felix Salten}!1917-12-271@{27. 12. 1917}|(be}
\toendnotes[C]{\smallbreak\pagebreak[2]}\Standort{CUL, Schnitzler, B 89, B 2.}
\physDesc{Postkarte, 340 Zeichen
\newline{}Handschrift: schwarze Tinte, lateinische Kurrent
\newline{}Versand: Stempel: »\nobreak{}\oindex{XVIII., Waehring@\textbf{XVIII., Währing}, \emph{A.ADM3}|pwk}18/\textsubscript{1} Wien 110, 27. XII. 17, 4\textsuperscript{\textcolor{gray}{2}0}\nobreak{}«.  
\newline{}Ordnung: 1) mit Bleistift von Frieda Pollak\pwindex{Pollak, Frieda 08.12.1881 – 13.07.1937@\textsc{Pollak, Frieda} (08.12.1881 – 13.07.1937), \emph{Sekretär/Sekretärin}|pw} (?) mit
                                 dem Buchstaben »A« (Abgeschrieben/Abschrift)
                                 gekennzeichnet  2) mit Bleistift von unbekannter Hand nummeriert: »280«}\toendnotes[C]{\smallbreak}\pstart{}{\pb}\textcolor{gray}{\textbf{\textit{FELIX SALTEN}}}\pend{}\pstart{}\textcolor{gray}{\textbf{\textit{WIEN, XVIII.}}}\oindex{XVIII., Waehring@\textbf{XVIII., Währing}, \emph{A.ADM3}|pw}\pend{}\pstart{}\textcolor{gray}{\textbf{\textit{COTTAGEGASSE 37}}}\oindex{Cottagegasse@\textbf{Cottagegasse}, \emph{Straße (K.STR)}|pw}\pend{}{\bigskip}\pstart{}Herrn\pend{}\pstart{}D\textsuperscript{r} Arthur Schnitzler\pend{}\pstart{}Wien\oindex{Wien@\textbf{Wien}, \emph{A.ADM2}|pw}\pend{}\pstart{}XVIII. Sternwartestrasse 71\oindex{Sternwartestrasse 71@\textbf{Sternwartestraße 71}, \emph{Wohngebäude (K.WHS)}|pw}\pend{}{\bigskip}\vspace{1em}
\pstart
           \raggedleft{}{\pb}27. XII. 17\pend
           
\pstart{}Lieber Arthur,\pend\vspace{0.5em}
\pstart
           gestern{ }Vormittag war ich bei Ihnen, habe Sie aber nicht zu Hause getroffen; so
               muss ich Ihnen nun auf diesem Weg für Ihre \label{K_L03567-1v}\edtext{freundlichen Zeilen}{\lemma{\textnormal{\emph{freundlichen Zeilen}}}\Cendnote{\textnormal{Am 22. 12. 1917 hatten Saltens\pwindex{Salten, Felix 06.09.1869 – 08.10.1945@\textsc{Salten, Felix} (06.09.1869 – 08.10.1945), \emph{Schriftsteller/Schriftstellerin, Journalist/Journalistin, Chefredakteur/Chefredakteurin}|pwk} drei Einakter \emph{Kinder der Freude}\pwindex{Kinder der Freude. Drei Einakter@\emph{Kinder der Freude. Drei Einakter}|pwk} die Uraufführung am \emph{Deutschen Volkstheater}\orgindex{Volkstheater@Volkstheater|pwk}. Die Regie verantwortete ebenfalls
                     Salten\pwindex{Salten, Felix 06.09.1869 – 08.10.1945@\textsc{Salten, Felix} (06.09.1869 – 08.10.1945), \emph{Schriftsteller/Schriftstellerin, Journalist/Journalistin, Chefredakteur/Chefredakteurin}|pwk}. Schnitzler las den Text am 12. 11. 1917 und fand ihn furchtbar. Die Premiere besuchte er
                     nicht, dürfte Salten\pwindex{Salten, Felix 06.09.1869 – 08.10.1945@\textsc{Salten, Felix} (06.09.1869 – 08.10.1945), \emph{Schriftsteller/Schriftstellerin, Journalist/Journalistin, Chefredakteur/Chefredakteurin}|pwk}
                     trotzdem mit den »freundlichen Zeilen« gratuliert haben. Erst am
                     18. 1. 1918 besuchte
                     er die Aufführung.}}}\label{K_L03567-1}
               danken. Ich hätte es gern \label{K_L03567-2v}\edtext{mündlich}{\lemma{\textnormal{\emph{mündlich}}}\Cendnote{\textnormal{Das
               nächste belegte Zusammentreffen zwischen Schnitzler und Salten\pwindex{Salten, Felix 06.09.1869 – 08.10.1945@\textsc{Salten, Felix} (06.09.1869 – 08.10.1945), \emph{Schriftsteller/Schriftstellerin, Journalist/Journalistin, Chefredakteur/Chefredakteurin}|pwk} fand
                  am 8. 1. 1918 statt.}}}\label{K_L03567-2} getan.\pend
           
\pstart
           Viele Grüße von uns\pwindex{Salten, Ottilie 07.03.1868 – 22.06.1942@\textsc{Salten, Ottilie} (07.03.1868 – 22.06.1942), \emph{Schauspieler/Schauspielerin}|pwv} zu
               Ihnen. {\\[\baselineskip]}Ihr {\\[\baselineskip]}\spacefill\mbox{Felix Salten}\pend
           \leftskip=0em{}\selectlanguage{ngerman}\endnumbering\briefempfaengerindex{Schnitzler, Arthur@\textsc{Schnitzler, Arthur}!zzzSalten, Felix@\emph{von Felix Salten}!1917-12-271@{27. 12. 1917}|)be}\mylabel{L03567h}  \normalsize

\doendnotes{C}
\bigskip
\vfill

\clearpage

\footnotesize

\lohead{\textsc{register}}

% Definiere theindex-Environment komplett neu ohne reledmac
\makeatletter
\renewenvironment{theindex}{%
  \section*{\indexname}%
  \setlength{\parindent}{0pt}%
  \setlength{\parskip}{0pt plus 0.3pt}%
  \let\item\@idxitem
}{%
  \clearpage
}
\makeatother

\IfFileExists{\jobname-pw.ind}{\input{\jobname-pw.ind}}{}

\end{document}

      