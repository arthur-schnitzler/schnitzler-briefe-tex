%% latex-leseansicht-vorspann.tex
%% Vorspann für die Leseansicht.
%% Lädt die gemeinsame Datei latex-vorspann.tex mit nicht gesetztem Schalter.

\newif\ifkorrekturansicht
\korrekturansichtfalse

\input{../tex-inputs/latex-vorspann}


\section[ Arthur Schnitzler an Felix Salten, 8. 8. 191[0]]{L03018 Arthur Schnitzler an Felix Salten,  8. 8. 191[0]}
\nopagebreak\mylabel{L03018v}
\rehead{ }\normalsize\beginnumbering\briefempfaengerindex{Salten, Felix@\textsc{Salten, Felix}!zzzSchnitzler, Arthur@\emph{von Arthur Schnitzler}!1910-08-081@{8. 8. 191[0]}|(be}
\toendnotes[C]{\smallbreak\pagebreak[2]}
\correspDesc{Versand  durch Arthur Schnitzler am 8. 8. 191[0] in Wien
\newline{}Erhalt  durch Felix Salten im Zeitraum [9. 8. 1910
                  – 13. 8. 1910?] in Unterach am Attersee}\toendnotes[C]{\smallbreak}
\Standort{Wienbibliothek im Rathaus, ZPH 1681, 2.1.516.}
\physDesc{Brief, 1 Blatt, 3 Seiten, 864 Zeichen
\newline{}Handschrift: Bleistift, deutsche Kurrent
\newline{}Ordnung: mit Bleistift von unbekannter Hand Nummerierung der Doppelseiten des
                                 Konvoluts: »6«–»7« }\toendnotes[C]{\smallbreak}
\pstart
           {\pb}\textcolor{gray}{\textbf{Dr. Arthur Schnitzler}}\hfill \label{K_L03018-1v}\edtext{8. 8. 1911}{\lemma{\textnormal{\emph{8. 8. 1911}}}\Cendnote{\textnormal{Schnitzlers Datierung ins Jahr 1911 ist falsch. Mindestens vier Argumente lassen sich
                        finden: die handschriftliche Angabe der neuen Adresse neben dem gedruckten
                        Briefkopf mit der alten Adresse (vgl. XXXX Auszeichnungsfehler: Dokument L01952 nicht gefunden); der Brief bezieht sich auf ein Glückwünschtelegramm,
                        womit wohl jenes zum Einzug in der Sternwartestraße 71\oindex{Wien@\textbf{Wien}!XVIII., Währing@\textbf{XVIII., Währing}!Sternwartestraße 71@\textbf{Sternwartestraße 71}, \emph{Wohngebäude}|pwk} gemeint ist (XXXX Auszeichnungsfehler: Dokument L03550 nicht gefunden); die inhaltliche Übereinstimmung mit dem
                        (Antwort-)Brief Saltens\pwindex{Salten, Felix 6.\,9.\,1869 Budapest – 8.\,10.\,1945 Zürich@\textsc{Salten, Felix} (6.\,9.\,1869 Budapest – 8.\,10.\,1945 Zürich), \emph{Schriftsteller, Journalist, Chefredakteur}|pwk} (XXXX Auszeichnungsfehler: Dokument L03551 nicht gefunden), worin auch auf die Anwesenheit von Samuel\pwindex{Fischer, Samuel 24.\,12.\,1859 Liptovský Mikuláš – 15.\,10.\,1934 Berlin@\textsc{Fischer, Samuel} (24.\,12.\,1859 Liptovský Mikuláš – 15.\,10.\,1934 Berlin), \emph{Verleger}|pwk} und Hedwig
                                 Fischer\pwindex{Fischer, Hedwig 8.\,9.\,1871 Szczecin – 11.\,4.\,1952 Königstein im Taunus@\textsc{Fischer, Hedwig} (8.\,9.\,1871 Szczecin – 11.\,4.\,1952 Königstein im Taunus)|pwk} in Unterach\oindex{Unterach am Attersee@\textbf{Unterach am Attersee}|pwk} eingegangen wird; die Erwähung von Elisabeth Steinrücks\pwindex{Steinrück, Elisabeth 19.\,11.\,1885 – 7.\,4.\,1920 Partenkirchen@\textsc{Steinrück, Elisabeth} (19.\,11.\,1885 – 7.\,4.\,1920 Partenkirchen)|pwk}
                        Rippenfellentzündung (vgl. A. S.: \emph{Tagebuch}, 2. 8. 1910). Die Bezugnahme
                        auf \emph{Olga Frohgemuth}\pwindex{Salten, Felix 6.\,9.\,1869 Budapest – 8.\,10.\,1945 Zürich@\textsc{Salten, Felix} (6.\,9.\,1869 Budapest – 8.\,10.\,1945 Zürich), \emph{Schriftsteller, Journalist, Chefredakteur}!Olga Frohgemuth. Erzählung@\strich\emph{Olga Frohgemuth. Erzählung}|pwk} weist zudem auf die
                        bevorstehende Buchpublikation (vgl. XXXX Auszeichnungsfehler: Dokument L03047 nicht gefunden).}}}\label{K_L03018-1}\pend
           
\pstart
           \textcolor{gray}{\textbf{Wien XVIII. Spoettelgasse 7\oindex{Wien@\textbf{Wien}!XVIII., Währing@\textbf{XVIII., Währing}!Edmund-Weiß-Gasse 7@\textbf{Edmund-Weiß-Gasse 7}, \emph{Wohngebäude}|pw}.}}\hfill \textsc{XVIII. Sternwartestr 71\oindex{Wien@\textbf{Wien}!XVIII., Währing@\textbf{XVIII., Währing}!Sternwartestraße 71@\textbf{Sternwartestraße 71}, \emph{Wohngebäude}|pw}}\pend
           \vspace{0.5em}
\pstart
           lieber,{ }wir\pwindex{Schnitzler, Olga 17.\,1.\,1882 Wien – 13.\,1.\,1970 Lugano@\textsc{Schnitzler, Olga} (17.\,1.\,1882 Wien – 13.\,1.\,1970 Lugano), \emph{Schauspielerin, Sängerin}|pwv} danken herzlich für das
               liebe Glückwunſchtelegramm. Nun{ }ſind wir in leidlicher Ordnung; und dieſer Tage
                  \label{K_L03018-2v}\edtext{fahren wir nach Partenkirchen\oindex{Partenkirchen@\textbf{Partenkirchen}, \emph{Teil eines besiedelten Ortes}|pw}}{\lemma{\textnormal{\emph{fahren … Partenkirchen}}}\Cendnote{\textnormal{Schnitzler war zwischen 20. 8. 1910 und 26. 8. 1910 in Partenkirchen\oindex{Partenkirchen@\textbf{Partenkirchen}, \emph{Teil eines besiedelten Ortes}|pwk}.}}}\label{K_L03018-2}, wo \textsc{Liesl\pwindex{Steinrück, Elisabeth 19.\,11.\,1885 – 7.\,4.\,1920 Partenkirchen@\textsc{Steinrück, Elisabeth} (19.\,11.\,1885 – 7.\,4.\,1920 Partenkirchen)|pw}} an einer Rippenfellentzündg erkrankt liegt. Wir waren{ }ſchon vor 3 Tagen daran
               hinzufahren, {\pb}da bat uns der Arzt\pwindex{?? [Arzt von Elisabeth Steinrück] @\textsc{?? [Arzt von Elisabeth Steinrück]}|pwv} telegraphiſch die Reiſe
               aufzuſchieben, da unſer Erſcheinen bei dem augenblicklich\textcolor{gray}{en}
               Zuſtand der Kranken einen nicht ungefährlichen \textsc{Chok}
               bedeuten müßte. Nun{ }ſcheint es etwas beſſer zu gehen. Ob wir von P.\oindex{Partenkirchen@\textbf{Partenkirchen}, \emph{Teil eines besiedelten Ortes}|pw} aus noch \label{K_L03018-3v}\edtext{ins
                  \textsc{Salzkgut\oindex{Salzkammergut@\textbf{Salzkammergut}, \emph{Region}|pw}} gelangen}{\lemma{\textnormal{\emph{ins
                  Salzkgut gelangen}}}\Cendnote{\textnormal{Zwischen 29. 8. 1910 und 5. 9. 1910 war Schnitzler in Bad Ischl\oindex{Bad Ischl@\textbf{Bad Ischl}|pwk}.}}}\label{K_L03018-3}, wie es unſere Abſicht war, läßt{ }ſich heute noch nicht voraus{\pb}ſehen; wollen Sie mir gelegentlich{ }ſagen, wie
               lange Sie un\textcolor{gray}{d} wie lange \textsc{Fischers\pwindex{Fischer, Samuel 24.\,12.\,1859 Liptovský Mikuláš – 15.\,10.\,1934 Berlin@\textsc{Fischer, Samuel} (24.\,12.\,1859 Liptovský Mikuláš – 15.\,10.\,1934 Berlin), \emph{Verleger}|pw}\pwindex{Fischer, Hedwig 8.\,9.\,1871 Szczecin – 11.\,4.\,1952 Königstein im Taunus@\textsc{Fischer, Hedwig} (8.\,9.\,1871 Szczecin – 11.\,4.\,1952 Königstein im Taunus)|pw}} noch in \textsc{Unterach\oindex{Unterach am Attersee@\textbf{Unterach am Attersee}|pw}} bleiben?\pend
           
\pstart
           Ihren Nachrichten und dem weiteren Schickſale Ihres reizumfloſſenen Frohgemuth\pwindex{Salten, Felix 6.\,9.\,1869 Budapest – 8.\,10.\,1945 Zürich@\textsc{Salten, Felix} (6.\,9.\,1869 Budapest – 8.\,10.\,1945 Zürich), \emph{Schriftsteller, Journalist, Chefredakteur}!Olga Frohgemuth. Erzählung@\strich\emph{Olga Frohgemuth. Erzählung}|pw}{ }ſeh ich mit Spa{\geminationn}ung entgegen und hoffe Sie{ }ſind alle wohl u vergnügt.
               Herzlichſt mit Grüßen von uns Allen {\\[\baselineskip]}Ihr {\\[\baselineskip]}\spacefill\mbox{A.}\pend
           \leftskip=0em{}\selectlanguage{ngerman}\endnumbering\briefempfaengerindex{Salten, Felix@\textsc{Salten, Felix}!zzzSchnitzler, Arthur@\emph{von Arthur Schnitzler}!1910-08-081@{8. 8. 191[0]}|)be}\mylabel{L03018h}  \newcommand{\dateiname}{L03018}\newcommand{\titel}{Arthur Schnitzler an Felix Salten, 8. 8. 191[0]}\newcommand{\editorInnen}{Martin Anton Müller und Laura Untner}%% latex-leseansicht-abspann.tex
%% Abspann für die Leseansicht.
%% Der Schalter \ifkorrekturansicht ist bereits durch den Vorspann gesetzt.

%% latex-abspann.tex
%% Gemeinsamer Abspann für Korrekturansicht und Leseansicht.
%% Setzt den Schalter \ifkorrekturansicht voraus (gesetzt in den
%% einbindenden Dateien latex-korrekturansicht-abspann.tex bzw.
%% latex-leseansicht-abspann.tex).
%% ---------------------------------------------------------------

\normalsize

% Das esempio-Environment wird nur in der Leseansicht benötigt
\ifkorrekturansicht\else
\newenvironment{esempio}[3]%
{
    \vspace{1.5ex}
    \rlap{\underline{#1}}
    \par
    \setlength{\parindent}{0cm}
    \nopagebreak
    \leftskip=#2cm
    \rightskip=#3cm
}
{
    \par
}
\fi

\doendnotes{C}
\bigskip
\vfill

\clearpage

\footnotesize

\ifkorrekturansicht
  \lohead{\textsc{register}}
\fi

% theindex-Environment neu definieren ohne reledmac
\makeatletter
\renewenvironment{theindex}{%
  \ifkorrekturansicht
    \section*{\indexname}%
  \else
    \subsubsection*{Index der erwähnten Entitäten}%
  \fi
  \setlength{\parindent}{0pt}%
  \setlength{\parskip}{0pt plus 0.3pt}%
  \let\item\@idxitem
}{%
  \ifkorrekturansicht\clearpage\fi
}
\makeatother

\IfFileExists{\jobname-pw.ind}{\input{\jobname-pw.ind}}{}

% Quellenangabe nur in der Leseansicht
\ifkorrekturansicht\else
% Fallback-Definitionen, falls die .tex-Datei \titel etc. nicht gesetzt hat
\providecommand{\titel}{}
\providecommand{\editorInnen}{}
\providecommand{\dateiname}{\jobname}

\vspace{3cm}

\vfill

\footnotesize
\textsc{Quelle}: \titel. Herausgegeben von {\editorInnen}. In: \emph{Arthur Schnitzler: Briefwechsel mit Autorinnen und Autoren}.
 Digitale Edition, https://schnitzler-briefe.acdh.oeaw.ac.at/{\dateiname}.html (Stand \today)
\fi

\end{document}


