%% latex-leseansicht-vorspann.tex
%% Vorspann für die Leseansicht.
%% Lädt die gemeinsame Datei latex-vorspann.tex mit nicht gesetztem Schalter.

\newif\ifkorrekturansicht
\korrekturansichtfalse

\input{../tex-inputs/latex-vorspann}

\begin{center}
            \textcolor{red}{ENTWURF, NICHT FERTIG KORRIGIERT}
                      \end{center}
            
         
         \renewcommand{\erwaehntePersonen}{Personen:  ?? [Arzt von Elisabeth Steinrück], Samuel Fischer, Hedwig Fischer, Felix Salten, Elisabeth Steinrück}
         \renewcommand{\erwaehnteOrte}{Orte: Edmund-Weiß-Gasse, Partenkirchen, Salzkammergut, Sternwartestraße, Unterach am Attersee, Wien}
         \renewcommand{\erwaehnteWerke}{Werke: Olga Frohgemuth. Erzählung}
               \section[Arthur Schnitzler an Felix Salten, 8. 8. 1911]{ Arthur Schnitzler an Felix Salten, 8. 8. 1911}\nopagebreak\mylabel{v}\rehead{ }\begin{ledgroupsized}[t]{13cm}\normalsize\beginnumbering \toendnotes[C]{\smallbreak\pagebreak[2]} \Standort{Wienbibliothek im Rathaus, ZPH 1681, 2.1.516.}
\physDesc{Brief, 1 Blatt, 3 Seiten, 870 Zeichen
\newline{}Handschrift: Bleistift, deutsche Kurrent
\newline{}Ordnung: mit Bleistift von unbekannter Hand Nummerierung der Blätter des
                                 Konvoluts: »6« }\toendnotes[C]{\smallbreak}\pstart
           \noindent{}{\pb}\textcolor{gray}{\textbf{Dr Arthur Schnitzler}}\hfill 8. 8. 1911\pend
           \pstart
           \textcolor{gray}{\textbf{Wien XVIII. Spoettelgasse 7\oindex{Edmund-Weiss-Gasse@\textbf{Edmund-Weiß-Gasse}|pw}.}}\hfill \textsc{XVIII. Sternwartestr 71\oindex{Sternwartestrasse@\textbf{Sternwartestraße}|pw}}\pend
           \pstart
           lieber, wir danken herzlich für das liebe Glückwunſchtelegramm. Nun
               ſind wir in leidlicher Ordnung; und dieſer Tage fahren wir nach Partenkirchen\oindex{Partenkirchen@\textbf{Partenkirchen}|pw}, wo \textsc{Liesl\pwindex{Steinrueck, Elisabeth 19.11.1885 – 07.04.1920@\textsc{Steinrück, Elisabeth} (19.11.1885 – 07.04.1920)|pw}} an einer Rippenfellentzündg erkrankt liegt. Wir waren ſchon vor 3 Tagen daran
               hinzufahren, {\pb}da bat uns der Arzt\pwindex{?? [Arzt von Elisabeth Steinrueck] @\textsc{?? [Arzt von Elisabeth Steinrück]}|pwv} telegraphiſch, die
               Reiſe aufzuſchieben, da unſer Erſcheinen bei dem augenblicklichen Zuſtand der Kranken
               einen nicht ungefährlichen \textsc{Chok} bedeuten müßte, Nun ſcheint
               es etwas beſſer zu gehen. Ob wir von P.\oindex{Partenkirchen@\textbf{Partenkirchen}|pw} aus noch
               ins \textsc{Salzkgut\oindex{Salzkammergut@\textbf{Salzkammergut}|pw}} gelangen, wie es unſere Abſicht war, läßt ſich heute noch nicht voraus{\pb}ſehen; wollen Sie mir gelegentlich ſagen, wie
               lange Sie un\textcolor{gray}{d} wie lange \textsc{Fischers\pwindex{Fischer, Samuel 24.12.1859 – 15.10.1934@\textsc{Fischer, Samuel} (24.12.1859 – 15.10.1934), \emph{Verleger}|pw}\pwindex{Fischer, Hedwig 08.09.1871 – 11.04.1952@\textsc{Fischer, Hedwig} (08.09.1871 – 11.04.1952)|pw}} noch in \textsc{Unterach\oindex{Unterach am Attersee@\textbf{Unterach am Attersee}|pw}} bleiben? \pend
           \pstart
           Ihren Nachrichten und dem weiteren Schickſale Ihres reizumfloſſenen Frohgemuth\pwindex{Salten, Felix 06.09.1869 – 08.10.1945@\textsc{Salten, Felix} (06.09.1869 – 08.10.1945), \emph{Schriftsteller, Journalist}!Olga Frohgemuth. Erzaehlung1910@\strich\emph{Olga Frohgemuth. Erzählung} {[}1910{]}|pw} ſeh ich mit Spa{\geminationn}ung entgegen, und hoffe, Sie ſind alle wohl u vergnügt,
               {\\[\baselineskip]}mit Grüßen von uns Allen {\\[\baselineskip]}Herzlichſt {\\[\baselineskip]}Ihr {\\[\baselineskip]}\spacefill\mbox{A.}\pend
           \leftskip=0em{}
         
         \endnumbering\mylabel{h}\end{ledgroupsized}\begin{anhang}\end{anhang}\newcommand{\dateiname}{L03018}\newcommand{\titel}{Arthur Schnitzler an Felix Salten, 8. 8. 1911}\newcommand{\editorInnen}{Martin Anton Müller und Laura Untner}%% latex-leseansicht-abspann.tex
%% Abspann für die Leseansicht.
%% Der Schalter \ifkorrekturansicht ist bereits durch den Vorspann gesetzt.

%% latex-abspann.tex
%% Gemeinsamer Abspann für Korrekturansicht und Leseansicht.
%% Setzt den Schalter \ifkorrekturansicht voraus (gesetzt in den
%% einbindenden Dateien latex-korrekturansicht-abspann.tex bzw.
%% latex-leseansicht-abspann.tex).
%% ---------------------------------------------------------------

\normalsize

% Das esempio-Environment wird nur in der Leseansicht benötigt
\ifkorrekturansicht\else
\newenvironment{esempio}[3]%
{
    \vspace{1.5ex}
    \rlap{\underline{#1}}
    \par
    \setlength{\parindent}{0cm}
    \nopagebreak
    \leftskip=#2cm
    \rightskip=#3cm
}
{
    \par
}
\fi

\doendnotes{C}
\bigskip
\vfill

\clearpage

\footnotesize

\ifkorrekturansicht
  \lohead{\textsc{register}}
\fi

% theindex-Environment neu definieren ohne reledmac
\makeatletter
\renewenvironment{theindex}{%
  \ifkorrekturansicht
    \section*{\indexname}%
  \else
    \subsubsection*{Index der erwähnten Entitäten}%
  \fi
  \setlength{\parindent}{0pt}%
  \setlength{\parskip}{0pt plus 0.3pt}%
  \let\item\@idxitem
}{%
  \ifkorrekturansicht\clearpage\fi
}
\makeatother

\IfFileExists{\jobname-pw.ind}{\input{\jobname-pw.ind}}{}

% Quellenangabe nur in der Leseansicht
\ifkorrekturansicht\else
% Fallback-Definitionen, falls die .tex-Datei \titel etc. nicht gesetzt hat
\providecommand{\titel}{}
\providecommand{\editorInnen}{}
\providecommand{\dateiname}{\jobname}

\vspace{3cm}

\vfill

\footnotesize
\textsc{Quelle}: \titel. Herausgegeben von {\editorInnen}. In: \emph{Arthur Schnitzler: Briefwechsel mit Autorinnen und Autoren}.
 Digitale Edition, https://schnitzler-briefe.acdh.oeaw.ac.at/{\dateiname}.html (Stand \today)
\fi

\end{document}


      