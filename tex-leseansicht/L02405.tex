%% latex-leseansicht-vorspann.tex
%% Vorspann für die Leseansicht.
%% Lädt die gemeinsame Datei latex-vorspann.tex mit nicht gesetztem Schalter.

\newif\ifkorrekturansicht
\korrekturansichtfalse

\input{../tex-inputs/latex-vorspann}


         
         \renewcommand{\erwaehntePersonen}{Personen: Katia Mann}
         \renewcommand{\erwaehnteOrte}{Orte: München, Poschingerstraße, Wien}
         \renewcommand{\erwaehnteWerke}{Werke: Bekenntnisse des Hochstaplers Felix Krull}
               \section[Thomas Mann an Arthur Schnitzler, 22. 11. 1923]{ Thomas Mann an Arthur Schnitzler, 22. 11. 1923}\nopagebreak\mylabel{v}\rehead{ }\begin{ledgroupsized}[t]{13cm}\normalsize\beginnumbering \toendnotes[C]{\smallbreak\pagebreak[2]} \Standort{Düsseldorf, Heinrich-Heine-Institut, HHI.94.5036.397.}
\physDesc{Briefkarte
\newline{}Handschrift: schwarze Tinte, deutsche Kurrent
\newline{}Schnitzler: mit rotem Buntstift eine Unterstreichung }\toendnotes[C]{\smallbreak}\pstart
           \noindent{}{\pb}\textcolor{gray}{\textbf{\textsc{Thomas Mann}}}\hfill \textcolor{gray}{\textbf{MÜNCHEN\oindex{Muenchen@\textbf{München}|pw}, den}}{ }22. XI. 23.\pend
           \pstart
           \raggedleft{}\textcolor{gray}{\textbf{POSCHINGERSTR. 1\oindex{Poschingerstrasse@\textbf{Poschingerstraße}|pw}}}\pend
           \pstart{}Lieber, verehrter Herr Dr. Schnitzler,\pend\pstart
           ich bin wahrhaft gerührt durch Ihr gütiges Eingehen auf den »Krull\pwindex{Mann, Thomas 06.06.1875 – 12.08.1955@\textsc{Mann, Thomas} (06.06.1875 – 12.08.1955), \emph{Schriftsteller}!Bekenntnisse des Hochstaplers Felix Krull1922@\strich\emph{Bekenntnisse des Hochstaplers Felix Krull} {[}1922{]}|pw}« und danke Ihnen herzlich. Ich weiß nicht, warum ich
                    damals ſtecken blieb, – vielleicht, weil der extrem individualiſtiſche und
                    unſoziale Charakter des Buches mir nicht zeitgemäß ſchien, vielleicht auch, weil
                    ich das Gefühl hatte, in dieſem erſten Teil alles We{\pb}ſentliche
                    eigentlich ſchon gegeben zu haben. Immerhin habe ich den Plan nie ganz aus den
                    Augen verloren, und wenn ich abgewälzt habe, woran ich jetzt ſchleppe, findet
                    ſich wohl einmal die Laune, das abſonderliche Ding zu beenden.\pend
           \pstart
           Ich freue mich auf Wien\oindex{Wien@\textbf{Wien}|pw}, wohin ich – diesmal wohl
                    mit meiner Frau\pwindex{Mann, Katia 24.07.1883 – 25.04.1980@\textsc{Mann, Katia} (24.07.1883 – 25.04.1980)|pwv}, die Ihnen
                    herzlich verehrungsvolle Grüße ſendet – Ende des Winters, im März
                    etwa, zu kommen hoffe, freue mich auf die Freunde dort und vor Allem auf
                    Sie.\pend
           \pstart
           Ihr ergebener{\\[\baselineskip]}\spacefill\mbox{Thomas Mann.}\pend
           \leftskip=0em{}
         
         \endnumbering\mylabel{h}\end{ledgroupsized}  \newcommand{\dateiname}{L02405}\newcommand{\titel}{Thomas Mann an Arthur Schnitzler, 22. 11. 1923}\newcommand{\editorInnen}{Martin Anton Müller und Gerd-Hermann Susen}%% latex-leseansicht-abspann.tex
%% Abspann für die Leseansicht.
%% Der Schalter \ifkorrekturansicht ist bereits durch den Vorspann gesetzt.

%% latex-abspann.tex
%% Gemeinsamer Abspann für Korrekturansicht und Leseansicht.
%% Setzt den Schalter \ifkorrekturansicht voraus (gesetzt in den
%% einbindenden Dateien latex-korrekturansicht-abspann.tex bzw.
%% latex-leseansicht-abspann.tex).
%% ---------------------------------------------------------------

\normalsize

% Das esempio-Environment wird nur in der Leseansicht benötigt
\ifkorrekturansicht\else
\newenvironment{esempio}[3]%
{
    \vspace{1.5ex}
    \rlap{\underline{#1}}
    \par
    \setlength{\parindent}{0cm}
    \nopagebreak
    \leftskip=#2cm
    \rightskip=#3cm
}
{
    \par
}
\fi

\doendnotes{C}
\bigskip
\vfill

\clearpage

\footnotesize

\ifkorrekturansicht
  \lohead{\textsc{register}}
\fi

% theindex-Environment neu definieren ohne reledmac
\makeatletter
\renewenvironment{theindex}{%
  \ifkorrekturansicht
    \section*{\indexname}%
  \else
    \subsubsection*{Index der erwähnten Entitäten}%
  \fi
  \setlength{\parindent}{0pt}%
  \setlength{\parskip}{0pt plus 0.3pt}%
  \let\item\@idxitem
}{%
  \ifkorrekturansicht\clearpage\fi
}
\makeatother

\IfFileExists{\jobname-pw.ind}{\input{\jobname-pw.ind}}{}

% Quellenangabe nur in der Leseansicht
\ifkorrekturansicht\else
% Fallback-Definitionen, falls die .tex-Datei \titel etc. nicht gesetzt hat
\providecommand{\titel}{}
\providecommand{\editorInnen}{}
\providecommand{\dateiname}{\jobname}

\vspace{3cm}

\vfill

\footnotesize
\textsc{Quelle}: \titel. Herausgegeben von {\editorInnen}. In: \emph{Arthur Schnitzler: Briefwechsel mit Autorinnen und Autoren}.
 Digitale Edition, https://schnitzler-briefe.acdh.oeaw.ac.at/{\dateiname}.html (Stand \today)
\fi

\end{document}


      