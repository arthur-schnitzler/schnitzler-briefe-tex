%% latex-leseansicht-vorspann.tex
%% Vorspann für die Leseansicht.
%% Lädt die gemeinsame Datei latex-vorspann.tex mit nicht gesetztem Schalter.

\newif\ifkorrekturansicht
\korrekturansichtfalse

\input{../tex-inputs/latex-vorspann}


\section[Thomas Mann an Arthur Schnitzler, 22. 11. 1923]{L02405 Thomas Mann an Arthur Schnitzler, 22. 11. 1923}
\nopagebreak\mylabel{L02405v}
\rehead{ }\normalsize\beginnumbering\briefempfaengerindex{Schnitzler, Arthur@\textsc{Schnitzler, Arthur}!zzzMann, Thomas@\emph{von Thomas Mann}!1923-11-221@{22. 11. 1923}|(be}
\toendnotes[C]{\smallbreak\pagebreak[2]}
\correspDesc{Versand  durch Thomas Mann am 22. 11. 1923 in München
\newline{}Erhalt  durch Arthur Schnitzler im Zeitraum [23. 11. 1923 – 27. 11. 1923?] in Wien}\toendnotes[C]{\smallbreak}
\Standort{Düsseldorf, Heinrich-Heine-Institut, HHI.94.5036.397.}
\physDesc{Briefkarte, 848 Zeichen
\newline{}Handschrift: schwarze Tinte, deutsche Kurrent
\newline{}Schnitzler: mit rotem Buntstift eine Unterstreichung }\toendnotes[C]{\smallbreak}
\pstart
           {\pb}\textcolor{gray}{\textbf{\textsc{Thomas Mann}}}\hfill \textcolor{gray}{\textbf{MÜNCHEN\oindex{München@\textbf{München}|pw}, den}}{ }22. XI. 23.\pend
           
\pstart
           \raggedleft{}\textcolor{gray}{\textbf{POSCHINGERSTR. 1\oindex{Poschingerstraße@\textbf{Poschingerstraße}, \emph{Straße}|pw}}}\pend
           
\pstart{}Lieber, verehrter Herr Dr. Schnitzler,\pend\vspace{0.5em}
\pstart
           ich bin wahrhaft gerührt durch Ihr gütiges Eingehen auf den »Krull\pwindex{Mann, Thomas 6.\,6.\,1875 Lübeck – 12.\,8.\,1955 Zürich@\textsc{Mann, Thomas} (6.\,6.\,1875 Lübeck – 12.\,8.\,1955 Zürich), \emph{Schriftsteller}!Bekenntnisse des Hochstaplers Felix Krull@\strich\emph{Bekenntnisse des Hochstaplers Felix Krull}|pw}« und danke Ihnen herzlich. Ich weiß nicht, warum ich damals{ }ſtecken blieb, – vielleicht, weil der extrem individualiſtiſche und unſoziale
               Charakter des Buches mir nicht zeitgemäß{ }ſchien, vielleicht auch, weil ich das Gefühl
               hatte, in dieſem erſten Teil alles We{\pb}ſentliche eigentlich{ }ſchon gegeben zu haben. Immerhin habe ich den Plan nie ganz aus den Augen verloren,
               und wenn ich abgewälzt habe, woran ich jetzt{ }ſchleppe, findet{ }ſich wohl einmal die
               Laune, das abſonderliche Ding zu beenden.\pend
           
\pstart
           Ich freue mich auf Wien\oindex{Wien@\textbf{Wien}, \emph{Verwaltungsgebiet}|pw}, wohin ich – diesmal wohl
               mit meiner Frau\pwindex{Mann, Katia 24.\,7.\,1883 Feldafing – 25.\,4.\,1980 Kilchberg@\textsc{Mann, Katia} (24.\,7.\,1883 Feldafing – 25.\,4.\,1980 Kilchberg)|pwv}, die Ihnen
               herzlich verehrungsvolle Grüße{ }ſendet – Ende des Winters, im März etwa,
               zu kommen hoffe, freue mich auf die Freunde dort und vor Allem auf Sie.\pend
           
\pstart
           Ihr ergebener{\\[\baselineskip]}\spacefill\mbox{Thomas Mann.}\pend
           \leftskip=0em{}\selectlanguage{ngerman}\endnumbering\briefempfaengerindex{Schnitzler, Arthur@\textsc{Schnitzler, Arthur}!zzzMann, Thomas@\emph{von Thomas Mann}!1923-11-221@{22. 11. 1923}|)be}\mylabel{L02405h}  \newcommand{\dateiname}{L02405}\newcommand{\titel}{Thomas Mann an Arthur Schnitzler, 22. 11. 1923}\newcommand{\editorInnen}{Martin Anton Müller und Gerd-Hermann Susen}%% latex-leseansicht-abspann.tex
%% Abspann für die Leseansicht.
%% Der Schalter \ifkorrekturansicht ist bereits durch den Vorspann gesetzt.

%% latex-abspann.tex
%% Gemeinsamer Abspann für Korrekturansicht und Leseansicht.
%% Setzt den Schalter \ifkorrekturansicht voraus (gesetzt in den
%% einbindenden Dateien latex-korrekturansicht-abspann.tex bzw.
%% latex-leseansicht-abspann.tex).
%% ---------------------------------------------------------------

\normalsize

% Das esempio-Environment wird nur in der Leseansicht benötigt
\ifkorrekturansicht\else
\newenvironment{esempio}[3]%
{
    \vspace{1.5ex}
    \rlap{\underline{#1}}
    \par
    \setlength{\parindent}{0cm}
    \nopagebreak
    \leftskip=#2cm
    \rightskip=#3cm
}
{
    \par
}
\fi

\doendnotes{C}
\bigskip
\vfill

\clearpage

\footnotesize

\ifkorrekturansicht
  \lohead{\textsc{register}}
\fi

% theindex-Environment neu definieren ohne reledmac
\makeatletter
\renewenvironment{theindex}{%
  \ifkorrekturansicht
    \section*{\indexname}%
  \else
    \subsubsection*{Index der erwähnten Entitäten}%
  \fi
  \setlength{\parindent}{0pt}%
  \setlength{\parskip}{0pt plus 0.3pt}%
  \let\item\@idxitem
}{%
  \ifkorrekturansicht\clearpage\fi
}
\makeatother

\IfFileExists{\jobname-pw.ind}{\input{\jobname-pw.ind}}{}

% Quellenangabe nur in der Leseansicht
\ifkorrekturansicht\else
% Fallback-Definitionen, falls die .tex-Datei \titel etc. nicht gesetzt hat
\providecommand{\titel}{}
\providecommand{\editorInnen}{}
\providecommand{\dateiname}{\jobname}

\vspace{3cm}

\vfill

\footnotesize
\textsc{Quelle}: \titel. Herausgegeben von {\editorInnen}. In: \emph{Arthur Schnitzler: Briefwechsel mit Autorinnen und Autoren}.
 Digitale Edition, https://schnitzler-briefe.acdh.oeaw.ac.at/{\dateiname}.html (Stand \today)
\fi

\end{document}


