%% latex-korrekturansicht-vorspann.tex
%% Vorspann für die Korrekturansicht.
%% Lädt die gemeinsame Datei latex-vorspann.tex mit gesetztem Schalter.

\newif\ifkorrekturansicht
\korrekturansichttrue

\input{../tex-inputs/latex-vorspann}


\section[ Arthur Schnitzler an Felix Salten, 10. 8. 1901]{L02969 Arthur Schnitzler an Felix Salten, 10. 8. 1901}
\nopagebreak\mylabel{L02969v}
\rehead{ }\normalsize\beginnumbering\briefempfaengerindex{Salten, Felix@\textsc{Salten, Felix}!zzzSchnitzler, Arthur@\emph{von Arthur Schnitzler}!1901-08-103@{10. 8. 1901}|(be}
\toendnotes[C]{\smallbreak\pagebreak[2]}\Standort{Wienbibliothek im Rathaus, ZPH 1681, 2.1.516.}
\physDesc{Brief, 1 Blatt, 4 Seiten, 919 Zeichen
\newline{}Handschrift: Bleistift, deutsche Kurrent
\newline{}Ordnung: mit Bleistift von unbekannter Hand Nummerierung der Doppelseiten des
                                 Konvoluts: »24«–»25« }\toendnotes[C]{\smallbreak}
\pstart
           \raggedleft{}{\pb}\textsc{Vahrn\oindex{Vahrn@\textbf{Vahrn}, \emph{P.PPLA3}|pw}}, 10. 8. 901\pend
           \vspace{0.5em}
\pstart
           Mein lieber Freund,{ }heut ſinds 4 Wochen, dſs ich hier\oindex{Vahrn@\textbf{Vahrn}, \emph{P.PPLA3}|pwv} bin, habe mich ſehr wohlgefühlt; Montag nach Bozen\oindex{Bozen@\textbf{Bozen}, \emph{P.PPLA2}|pw},
               woſelbſt Paul Goldma{\geminationn}\pwindex{Goldmann, Paul 31.01.1865 – 25.09.1935@\textsc{Goldmann, Paul} (31.01.1865 – 25.09.1935), \emph{Schriftsteller/Schriftstellerin, Journalist/Journalistin}|pw}, dann Trient\oindex{Trient@\textbf{Trient}, \emph{P.PPLA}|pw}, aber wir haben uns nicht zum
                  Gardaſee\oindex{Lago di Garda@\textbf{Lago di Garda}, \emph{See (N.SEE)}|pw}, ſondern zu einem ſehr schönen Ort
               im Puſterthal\oindex{Pustertal@\textbf{Pustertal}, \emph{T.VAL}|pw} entſchloſſen, Welsberg\oindex{Welsberg-Taisten@\textbf{Welsberg-Taisten}, \emph{A.ADM3}|pw}, Penſion {\pb}Waldbrunn\oindex{Wildbad Waldbrunn@\textbf{Wildbad Waldbrunn}, \emph{S.SPA}|pw}; woſelbſt wir etwa bis
                  Ende Auguſt verbleiben um da{\geminationn} direct \label{K_L02969-1v}\edtext{nach Wien\oindex{Wien@\textbf{Wien}, \emph{A.ADM2}|pw} zurückzukehren}{\lemma{\textnormal{\emph{nach Wien zurückzukehren}}}\Cendnote{\textnormal{Nach einem kurzen Aufenthalt in Pörtschach am Wörthersee\oindex{Poertschach am Woerthersee@\textbf{Pörtschach am Wörthersee}, \emph{P.PPL}|pwk} (27. 8. 1901 bis 29. 8. 1901) kehrte Schnitzler am 30. 8. 1901 nach Wien\oindex{Wien@\textbf{Wien}, \emph{A.ADM2}|pwk} zurück. Nachweislich sahen sich Salten\pwindex{Salten, Felix 06.09.1869 – 08.10.1945@\textsc{Salten, Felix} (06.09.1869 – 08.10.1945), \emph{Schriftsteller/Schriftstellerin, Journalist/Journalistin, Chefredakteur/Chefredakteurin}|pwk} und Schnitzler dort am 1. 9. 1901
                  wieder.}}}\label{K_L02969-1}. So treff’ ich Sie wahrſcheinlich dort noch an, bevor Sie nach \textsc{Verona\oindex{Verona@\textbf{Verona}, \emph{P.PPLA2}|pw}} oder \textsc{Venedig\oindex{Venedig@\textbf{Venedig}, \emph{P.PPLA}|pw}} fahren. Wollen Sie mir das Inſelheft\pwindex{Insel. Monatsschrift mit Buchschmuck und Illustrationen@\emph{Die Insel. Monatsschrift mit Buchschmuck und Illustrationen}|pwv}\pwindex{Gedenktafel der Prinzessin Anna@\emph{Die Gedenktafel der Prinzessin Anna}|pwv} nach \textsc{Welsberg\oindex{Welsberg-Taisten@\textbf{Welsberg-Taisten}, \emph{A.ADM3}|pw}} ſchicken? wäre Ihnen ſehr dankbar. – Das \label{K_L02969-2v}\edtext{Brettl\orgindex{Jung-Wiener Theater zum Lieben Augustin@Jung-Wiener Theater zum Lieben Augustin|pwv}}{\lemma{\textnormal{\emph{Brettl}}}\Cendnote{\textnormal{Synonym für ›Kabarett‹; das \emph{Jung-Wiener Theater zum Lieben Augustin}\orgindex{Jung-Wiener Theater zum Lieben Augustin@Jung-Wiener Theater zum Lieben Augustin|pwk} hatte das Berlin\oindex{Berlin@\textbf{Berlin}, \emph{P.PPLC}|pwk}er \emph{Überbrettl}\orgindex{Ueberbrettl@Überbrettl|pwk} als Vorbild.}}}\label{K_L02969-2} macht Ihnen natürlich viel
                  Mühe\textcolor{gray}{;} –{ }{\pb}– daſs der Erfolg nicht von Wien\oindex{Wien@\textbf{Wien}, \emph{A.ADM2}|pw} beſtritten werden kann, war vom erſten Moment an klar.
               Könnten Sie mir die Nummer der Allg. (Münchner)\pwindex{Allgemeine Zeitung@\emph{Allgemeine Zeitung}|pw}
               verſchaffen, wo dieſer \label{K_L02969-3v}\edtext{Bettelheim\pwindex{Bettelheim, Anton 18.11.1851 – 29.03.1930@\textsc{Bettelheim, Anton} (18.11.1851 – 29.03.1930), \emph{Kritiker/Kritikerin, Lexikograf/Lexikografin}|pw} uns beflegelt\pwindex{Zum Saekulartag Eduard Devrients@\emph{Zum Säkulartag Eduard Devrients}|pwuv}}{\lemma{\textnormal{\emph{Bettelheim uns beflegelt}}}\Cendnote{\textnormal{Am Tag des Briefes erschien in der
                  Beilage ein längerer Text über Eduard
                     Devrient\pwindex{Devrient, Eduard 11.08.1801 – 04.10.1877@\textsc{Devrient, Eduard} (11.08.1801 – 04.10.1877), \emph{Theaterleiter/Theaterleiterin, Schauspieler/Schauspielerin}|pwk}, der mehrere Seitenhiebe auf populäres Theater enthielt. Ob Schnitzler davon schon Kenntnis gehabt haben konnte, ist zweifelhaft. Vgl. Anton Bettelheim\pwindex{Bettelheim, Anton 18.11.1851 – 29.03.1930@\textsc{Bettelheim, Anton} (18.11.1851 – 29.03.1930), \emph{Kritiker/Kritikerin, Lexikograf/Lexikografin}|pwk}: \emph{Zum
                        Säkulartag Eduard Devrients}\pwindex{Zum Saekulartag Eduard Devrients@\emph{Zum Säkulartag Eduard Devrients}|pwk}. In: \emph{Allgemeine Zeitung}\pwindex{Allgemeine Zeitung@\emph{Allgemeine Zeitung}|pwk}, Beilage, Nr. 182, 10. 8. 1901, S. 1–6.}}}\label{K_L02969-3} haben ſoll? –\pend
           
\pstart
           Leben Sie wohl und ſeien Sie herzlich gegrüßt.\pend
           
\pstart
           Das neue \label{K_L02969-4v}\edtext{Stück\pwindex{einsame Weg. Schauspiel in fuenf Akten@\emph{Der einsame Weg. Schauspiel in fünf Akten}|pwv}}{\lemma{\textnormal{\emph{Stück}}}\Cendnote{\textnormal{\emph{Der einsame Weg}\pwindex{einsame Weg. Schauspiel in fuenf Akten@\emph{Der einsame Weg. Schauspiel in fünf Akten}|pwk}, den Schnitzler am 21. 7. 1901 vorläufig abgeschlossen hatte und am
                     20. 11. 1901 neu
                  zu bearbeiten begann}}}\label{K_L02969-4} iſt doch nicht fertig, ka{\geminationn} es aber bald ſein. {\pb}Dafür \label{K_L02969-5v}\edtext{2 Einakter\pwindex{Lebendige Stunden@\emph{Lebendige Stunden}|pwv}\pwindex{Frau mit dem Dolche@\emph{Die Frau mit dem Dolche}|pwv}}{\lemma{\textnormal{\emph{2 Einakter}}}\Cendnote{\textnormal{\emph{Lebendige Stunden}\pwindex{Lebendige Stunden@\emph{Lebendige Stunden}|pwk} hatte er am 28. 7. 1901
                  und \emph{Die Frau mit dem Dolche}\pwindex{Frau mit dem Dolche@\emph{Die Frau mit dem Dolche}|pwk} am 3. 8. 1901 fertiggestellt.}}}\label{K_L02969-5},
               die zu »Literatur\pwindex{Literatur@\emph{Literatur}|pw}« dazu gegeben werden
               ſollen.\pend
           \pstart Ihr \spacefill\mbox{A.}\pend{}\selectlanguage{ngerman}\endnumbering\briefempfaengerindex{Salten, Felix@\textsc{Salten, Felix}!zzzSchnitzler, Arthur@\emph{von Arthur Schnitzler}!1901-08-103@{10. 8. 1901}|)be}\mylabel{L02969h}  \normalsize

\doendnotes{C}
\bigskip
\vfill

\clearpage

\footnotesize

\lohead{\textsc{register}}

% Definiere theindex-Environment komplett neu ohne reledmac
\makeatletter
\renewenvironment{theindex}{%
  \section*{\indexname}%
  \setlength{\parindent}{0pt}%
  \setlength{\parskip}{0pt plus 0.3pt}%
  \let\item\@idxitem
}{%
  \clearpage
}
\makeatother

\IfFileExists{\jobname-pw.ind}{\input{\jobname-pw.ind}}{}

\end{document}

      