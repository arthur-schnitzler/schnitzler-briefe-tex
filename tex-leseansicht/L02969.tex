%% latex-leseansicht-vorspann.tex
%% Vorspann für die Leseansicht.
%% Lädt die gemeinsame Datei latex-vorspann.tex mit nicht gesetztem Schalter.

\newif\ifkorrekturansicht
\korrekturansichtfalse

\input{../tex-inputs/latex-vorspann}


\section[ Arthur Schnitzler an Felix Salten, 10. 8. 1901]{L02969 Arthur Schnitzler an Felix Salten,  10. 8. 1901}
\nopagebreak\mylabel{L02969v}
\rehead{ }\normalsize\beginnumbering\briefempfaengerindex{Salten, Felix@\textsc{Salten, Felix}!zzzSchnitzler, Arthur@\emph{von Arthur Schnitzler}!1901-08-103@{10. 8. 1901}|(be}
\toendnotes[C]{\smallbreak\pagebreak[2]}
\correspDesc{Versand  durch Arthur Schnitzler am 10. 8. 1901 in Vahrn
\newline{}Erhalt  durch Felix Salten im Zeitraum [11. 8. 1901
                  – 15. 8. 1901?] in Wien}\toendnotes[C]{\smallbreak}
\Standort{Wienbibliothek im Rathaus, ZPH 1681, 2.1.516.}
\physDesc{Brief, 1 Blatt, 4 Seiten, 919 Zeichen
\newline{}Handschrift: Bleistift, deutsche Kurrent
\newline{}Ordnung: mit Bleistift von unbekannter Hand Nummerierung der Doppelseiten des
                                 Konvoluts: »24«–»25« }\toendnotes[C]{\smallbreak}
\pstart
           \raggedleft{}{\pb}\textsc{Vahrn\oindex{Vahrn@\textbf{Vahrn}, \emph{Hauptstadt}|pw}}, 10. 8. 901\pend
           \vspace{0.5em}
\pstart
           Mein lieber Freund,{ }heut{ }ſinds 4 Wochen, dſs ich hier\oindex{Vahrn@\textbf{Vahrn}, \emph{Hauptstadt}|pwv} bin, habe mich{ }ſehr wohlgefühlt; Montag nach Bozen\oindex{Bozen@\textbf{Bozen}, \emph{Hauptstadt}|pw},
               woſelbſt Paul Goldma{\geminationn}\pwindex{Goldmann, Paul 31.\,1.\,1865 Breslau – 25.\,9.\,1935 Wien@\textsc{Goldmann, Paul} (31.\,1.\,1865 Breslau – 25.\,9.\,1935 Wien), \emph{Schriftsteller, Journalist}|pw}, dann Trient\oindex{Trient@\textbf{Trient}|pw}, aber wir haben uns nicht zum
                  Gardaſee\oindex{Lago di Garda@\textbf{Lago di Garda}, \emph{See}|pw},{ }ſondern zu einem{ }ſehr schönen Ort
               im Puſterthal\oindex{Pustertal@\textbf{Pustertal}, \emph{Tal}|pw} entſchloſſen, Welsberg\oindex{Welsberg-Taisten@\textbf{Welsberg-Taisten}, \emph{Verwaltungsgebiet}|pw}, Penſion {\pb}Waldbrunn\oindex{Wildbad Waldbrunn@\textbf{Wildbad Waldbrunn}, \emph{Spa}|pw}; woſelbſt wir etwa bis
                  Ende Auguſt verbleiben um da{\geminationn} direct \label{K_L02969-1v}\edtext{nach Wien\oindex{Wien@\textbf{Wien}, \emph{Verwaltungsgebiet}|pw} zurückzukehren}{\lemma{\textnormal{\emph{nach Wien zurückzukehren}}}\Cendnote{\textnormal{Nach einem kurzen Aufenthalt in Pörtschach am Wörthersee\oindex{Pörtschach am Wörthersee@\textbf{Pörtschach am Wörthersee}|pwk} (27. 8. 1901 bis 29. 8. 1901) kehrte Schnitzler am 30. 8. 1901 nach Wien\oindex{Wien@\textbf{Wien}, \emph{Verwaltungsgebiet}|pwk} zurück. Nachweislich sahen sich Salten\pwindex{Salten, Felix 6.\,9.\,1869 Budapest – 8.\,10.\,1945 Zürich@\textsc{Salten, Felix} (6.\,9.\,1869 Budapest – 8.\,10.\,1945 Zürich), \emph{Schriftsteller, Journalist, Chefredakteur}|pwk} und Schnitzler dort am 1. 9. 1901
                  wieder.}}}\label{K_L02969-1}. So treff’ ich Sie wahrſcheinlich dort noch an, bevor Sie nach \textsc{Verona\oindex{Verona@\textbf{Verona}, \emph{Hauptstadt}|pw}} oder \textsc{Venedig\oindex{Venedig@\textbf{Venedig}|pw}} fahren. Wollen Sie mir das Inſelheft\pwindex{Insel. Monatsschrift mit Buchschmuck und Illustrationen@\emph{Die Insel. Monatsschrift mit Buchschmuck und Illustrationen}|pwv}\pwindex{Salten, Felix 6.\,9.\,1869 Budapest – 8.\,10.\,1945 Zürich@\textsc{Salten, Felix} (6.\,9.\,1869 Budapest – 8.\,10.\,1945 Zürich), \emph{Schriftsteller, Journalist, Chefredakteur}!Gedenktafel der Prinzessin Anna@\strich\emph{Die Gedenktafel der Prinzessin Anna}|pwv} nach \textsc{Welsberg\oindex{Welsberg-Taisten@\textbf{Welsberg-Taisten}, \emph{Verwaltungsgebiet}|pw}}{ }ſchicken? wäre Ihnen{ }ſehr dankbar. – Das \label{K_L02969-2v}\edtext{Brettl\orgindex{Jung-Wiener Theater zum Lieben Augustin@Jung-Wiener Theater zum Lieben Augustin|pwv}}{\lemma{\textnormal{\emph{Brettl}}}\Cendnote{\textnormal{Synonym für ›Kabarett‹; das \emph{Jung-Wiener Theater zum Lieben Augustin}\orgindex{Jung-Wiener Theater zum Lieben Augustin@Jung-Wiener Theater zum Lieben Augustin|pwk} hatte das Berlin\oindex{Berlin@\textbf{Berlin}, \emph{Hauptstadt}|pwk}er \emph{Überbrettl}\orgindex{Überbrettl@Überbrettl|pwk} als Vorbild.}}}\label{K_L02969-2} macht Ihnen natürlich viel
                  Mühe\textcolor{gray}{;} –{ }{\pb}– daſs der Erfolg nicht von Wien\oindex{Wien@\textbf{Wien}, \emph{Verwaltungsgebiet}|pw} beſtritten werden kann, war vom erſten Moment an klar.
               Könnten Sie mir die Nummer der Allg. (Münchner)\pwindex{Allgemeine Zeitung@\emph{Allgemeine Zeitung}|pw}
               verſchaffen, wo dieſer \label{K_L02969-3v}\edtext{Bettelheim\pwindex{Bettelheim, Anton 18.\,11.\,1851 Wien – 29.\,3.\,1930 ebd.@\textsc{Bettelheim, Anton} (18.\,11.\,1851 Wien – 29.\,3.\,1930 ebd.), \emph{Kritiker, Lexikograf}|pw} uns beflegelt\pwindex{Bettelheim, Anton 18.\,11.\,1851 Wien – 29.\,3.\,1930 ebd.@\textsc{Bettelheim, Anton} (18.\,11.\,1851 Wien – 29.\,3.\,1930 ebd.), \emph{Kritiker, Lexikograf}!Zum Säkulartag Eduard Devrients@\strich\emph{Zum Säkulartag Eduard Devrients}|pwuv}}{\lemma{\textnormal{\emph{Bettelheim uns beflegelt}}}\Cendnote{\textnormal{Am Tag des Briefes erschien in der
                  Beilage ein längerer Text über Eduard
                     Devrient\pwindex{Devrient, Eduard 11.\,8.\,1801 Berlin – 4.\,10.\,1877 Karlsruhe@\textsc{Devrient, Eduard} (11.\,8.\,1801 Berlin – 4.\,10.\,1877 Karlsruhe), \emph{Theaterleiter, Schauspieler}|pwk}, der mehrere Seitenhiebe auf populäres Theater enthielt. Ob Schnitzler davon schon Kenntnis gehabt haben konnte, ist zweifelhaft. Vgl. Anton Bettelheim\pwindex{Bettelheim, Anton 18.\,11.\,1851 Wien – 29.\,3.\,1930 ebd.@\textsc{Bettelheim, Anton} (18.\,11.\,1851 Wien – 29.\,3.\,1930 ebd.), \emph{Kritiker, Lexikograf}|pwk}: \emph{Zum
                        Säkulartag Eduard Devrients}\pwindex{Bettelheim, Anton 18.\,11.\,1851 Wien – 29.\,3.\,1930 ebd.@\textsc{Bettelheim, Anton} (18.\,11.\,1851 Wien – 29.\,3.\,1930 ebd.), \emph{Kritiker, Lexikograf}!Zum Säkulartag Eduard Devrients@\strich\emph{Zum Säkulartag Eduard Devrients}|pwk}. In: \emph{Allgemeine Zeitung}\pwindex{Allgemeine Zeitung@\emph{Allgemeine Zeitung}|pwk}, Beilage, Nr. 182, 10. 8. 1901, S. 1–6.}}}\label{K_L02969-3} haben{ }ſoll? –\pend
           
\pstart
           Leben Sie wohl und{ }ſeien Sie herzlich gegrüßt.\pend
           
\pstart
           Das neue \label{K_L02969-4v}\edtext{Stück\pwindex{Schnitzler, Arthur 15.\,5.\,1862 Wien – 21.\,10.\,1931 ebd.@\textsc{Schnitzler, Arthur} (15.\,5.\,1862 Wien – 21.\,10.\,1931 ebd.), \emph{Schriftsteller, Mediziner}!einsame Weg. Schauspiel in fünf Akten@\strich\emph{Der einsame Weg. Schauspiel in fünf Akten}|pwv}}{\lemma{\textnormal{\emph{Stück}}}\Cendnote{\textnormal{\emph{Der einsame Weg}\pwindex{Schnitzler, Arthur 15.\,5.\,1862 Wien – 21.\,10.\,1931 ebd.@\textsc{Schnitzler, Arthur} (15.\,5.\,1862 Wien – 21.\,10.\,1931 ebd.), \emph{Schriftsteller, Mediziner}!einsame Weg. Schauspiel in fünf Akten@\strich\emph{Der einsame Weg. Schauspiel in fünf Akten}|pwk}, den Schnitzler am 21. 7. 1901 vorläufig abgeschlossen hatte und am
                     20. 11. 1901 neu
                  zu bearbeiten begann}}}\label{K_L02969-4} iſt doch nicht fertig, ka{\geminationn} es aber bald{ }ſein. {\pb}Dafür \label{K_L02969-5v}\edtext{2 Einakter\pwindex{Schnitzler, Arthur 15.\,5.\,1862 Wien – 21.\,10.\,1931 ebd.@\textsc{Schnitzler, Arthur} (15.\,5.\,1862 Wien – 21.\,10.\,1931 ebd.), \emph{Schriftsteller, Mediziner}!Lebendige Stunden@\strich\emph{Lebendige Stunden}|pwv}\pwindex{Schnitzler, Arthur 15.\,5.\,1862 Wien – 21.\,10.\,1931 ebd.@\textsc{Schnitzler, Arthur} (15.\,5.\,1862 Wien – 21.\,10.\,1931 ebd.), \emph{Schriftsteller, Mediziner}!Frau mit dem Dolche@\strich\emph{Die Frau mit dem Dolche}|pwv}}{\lemma{\textnormal{\emph{2 Einakter}}}\Cendnote{\textnormal{\emph{Lebendige Stunden}\pwindex{Schnitzler, Arthur 15.\,5.\,1862 Wien – 21.\,10.\,1931 ebd.@\textsc{Schnitzler, Arthur} (15.\,5.\,1862 Wien – 21.\,10.\,1931 ebd.), \emph{Schriftsteller, Mediziner}!Lebendige Stunden@\strich\emph{Lebendige Stunden}|pwk} hatte er am 28. 7. 1901
                  und \emph{Die Frau mit dem Dolche}\pwindex{Schnitzler, Arthur 15.\,5.\,1862 Wien – 21.\,10.\,1931 ebd.@\textsc{Schnitzler, Arthur} (15.\,5.\,1862 Wien – 21.\,10.\,1931 ebd.), \emph{Schriftsteller, Mediziner}!Frau mit dem Dolche@\strich\emph{Die Frau mit dem Dolche}|pwk} am 3. 8. 1901 fertiggestellt.}}}\label{K_L02969-5},
               die zu »Literatur\pwindex{Schnitzler, Arthur 15.\,5.\,1862 Wien – 21.\,10.\,1931 ebd.@\textsc{Schnitzler, Arthur} (15.\,5.\,1862 Wien – 21.\,10.\,1931 ebd.), \emph{Schriftsteller, Mediziner}!Literatur@\strich\emph{Literatur}|pw}« dazu gegeben werden{ }ſollen.\pend
           \pstart Ihr \spacefill\mbox{A.}\pend{}\selectlanguage{ngerman}\endnumbering\briefempfaengerindex{Salten, Felix@\textsc{Salten, Felix}!zzzSchnitzler, Arthur@\emph{von Arthur Schnitzler}!1901-08-103@{10. 8. 1901}|)be}\mylabel{L02969h}  \newcommand{\dateiname}{L02969}\newcommand{\titel}{Arthur Schnitzler an Felix Salten, 10. 8. 1901}\newcommand{\editorInnen}{Martin Anton Müller und Laura Untner}%% latex-leseansicht-abspann.tex
%% Abspann für die Leseansicht.
%% Der Schalter \ifkorrekturansicht ist bereits durch den Vorspann gesetzt.

%% latex-abspann.tex
%% Gemeinsamer Abspann für Korrekturansicht und Leseansicht.
%% Setzt den Schalter \ifkorrekturansicht voraus (gesetzt in den
%% einbindenden Dateien latex-korrekturansicht-abspann.tex bzw.
%% latex-leseansicht-abspann.tex).
%% ---------------------------------------------------------------

\normalsize

% Das esempio-Environment wird nur in der Leseansicht benötigt
\ifkorrekturansicht\else
\newenvironment{esempio}[3]%
{
    \vspace{1.5ex}
    \rlap{\underline{#1}}
    \par
    \setlength{\parindent}{0cm}
    \nopagebreak
    \leftskip=#2cm
    \rightskip=#3cm
}
{
    \par
}
\fi

\doendnotes{C}
\bigskip
\vfill

\clearpage

\footnotesize

\ifkorrekturansicht
  \lohead{\textsc{register}}
\fi

% theindex-Environment neu definieren ohne reledmac
\makeatletter
\renewenvironment{theindex}{%
  \ifkorrekturansicht
    \section*{\indexname}%
  \else
    \subsubsection*{Index der erwähnten Entitäten}%
  \fi
  \setlength{\parindent}{0pt}%
  \setlength{\parskip}{0pt plus 0.3pt}%
  \let\item\@idxitem
}{%
  \ifkorrekturansicht\clearpage\fi
}
\makeatother

\IfFileExists{\jobname-pw.ind}{\input{\jobname-pw.ind}}{}

% Quellenangabe nur in der Leseansicht
\ifkorrekturansicht\else
% Fallback-Definitionen, falls die .tex-Datei \titel etc. nicht gesetzt hat
\providecommand{\titel}{}
\providecommand{\editorInnen}{}
\providecommand{\dateiname}{\jobname}

\vspace{3cm}

\vfill

\footnotesize
\textsc{Quelle}: \titel. Herausgegeben von {\editorInnen}. In: \emph{Arthur Schnitzler: Briefwechsel mit Autorinnen und Autoren}.
 Digitale Edition, https://schnitzler-briefe.acdh.oeaw.ac.at/{\dateiname}.html (Stand \today)
\fi

\end{document}


