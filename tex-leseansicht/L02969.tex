%% latex-leseansicht-vorspann.tex
%% Vorspann für die Leseansicht.
%% Lädt die gemeinsame Datei latex-vorspann.tex mit nicht gesetztem Schalter.

\newif\ifkorrekturansicht
\korrekturansichtfalse

\input{../tex-inputs/latex-vorspann}

\begin{center}
            \textcolor{red}{ENTWURF, NICHT FERTIG KORRIGIERT}
                      \end{center}
            
         
         \newcommand{\erwaehntePersonen}{Personen: Anton Bettelheim, Eduard Devrient, Paul Goldmann, Felix Salten}
         \newcommand{\erwaehnteInstitutionen}{}
         \newcommand{\erwaehnteOrte}{Orte: Bozen, Lago di Garda, Pustertal, Trient, Vahrn, Venedig, Verona, Welsberg-Taisten, Wien, Wildbad Waldbrunn}
         \newcommand{\erwaehnteWerke}{Werke: Allgemeine Zeitung, Literatur, Zum Säkulartag Eduard Devrients}
               \section[Arthur Schnitzler an Felix Salten, 10. 8. 1901]{ Arthur Schnitzler an Felix Salten, 10. 8. 1901}\nopagebreak\mylabel{v}\rehead{ }\begin{ledgroupsized}[t]{13cm}\normalsize\beginnumbering \toendnotes[C]{\smallbreak\pagebreak[2]} \Standort{Wienbibliothek im Rathaus, ZPH 1681, 2.1.516.}
\physDesc{
\newline{}Handschrift: , deutsche Kurrent}\toendnotes[C]{\smallbreak}\pstart
           \raggedleft{}{\pb}\textsc{Vahrn\oindex{Vahrn@\textbf{Vahrn}|pw}}, 10. 8. 901\pend
           \pstart
           mein lieber Freund, heute ſinds 4 Wochen, dſs ich hier\oindex{Vahrn@\textbf{Vahrn}|pwv} bin, habe mich ſehr wohlgefühlt;
                  Montag nach Bozen\oindex{Bozen@\textbf{Bozen}|pw}, woſelbſt Paul Goldma{\geminationn}\pwindex{Goldmann, Paul 31.01.1865 – 25.09.1935@\textsc{Goldmann, Paul} (31.01.1865 – 25.09.1935), \emph{Schriftsteller, Journalist}|pw}, dann
               Trient\oindex{Trient@\textbf{Trient}|pw}, aber wir haben uns nicht zum Gardaſee\oindex{Lago di Garda@\textbf{Lago di Garda}|pw} ſondern zu einem ſehr schönen Ort imPuſterthal\oindex{Pustertal@\textbf{Pustertal}|pw} entſchloſſen, Welsberg\oindex{Welsberg-Taisten@\textbf{Welsberg-Taisten}|pw}, Penſion {\pb}Waldbrunn\oindex{Wildbad Waldbrunn@\textbf{Wildbad Waldbrunn}|pw}, woſelbſt wir
               etwa bis Ende Auguſt verbleiben um da{\geminationn}
               direct nach Wien\oindex{Wien@\textbf{Wien}|pw} zurückzukehren. So treff’ ich Sie
               wahrſcheinlich dort noch an, bevor Sie nach \textsc{Verona\oindex{Verona@\textbf{Verona}|pw}} oder \textsc{Venedig\oindex{Venedig@\textbf{Venedig}|pw}} fahren. Wollen Sie mir das Inſelheft\textcolor{red}{\textsuperscript{\textbf{KEY}}} nach \textsc{Welsberg\oindex{Welsberg-Taisten@\textbf{Welsberg-Taisten}|pw}} ſchicken? wäre Ihnen ſehr dankbar. Das Brettl\textcolor{red}{\textsuperscript{\textbf{KEY}}} macht Ihnen natürlich viel Mühe, {\pb}– daſs der Erfolg nicht Wien\oindex{Wien@\textbf{Wien}|pw} beſtritten werden kann, war vom erſten Moment
               an klar. Könnten Sie mir die Nummer der Allg.
                  (Münchner)\pwindex{?? Werk@Nicht ermittelte Verfasserinnen und Verfasser!Allgemeine Zeitung1798 – 30.6.1929@\emph{Allgemeine Zeitung} {[}1798 – 30.6.1929{]}|pw} verſchaffen, wo dieſer \label{K_L02969-11v}\edtext{Bettelheim\pwindex{Bettelheim, Anton 18.11.1851 – 29.03.1930@\textsc{Bettelheim, Anton} (18.11.1851 – 29.03.1930), \emph{Kritiker, Lexikograf}|pw} uns beflegelt\pwindex{Bettelheim, Anton 18.11.1851 – 29.03.1930@\textsc{Bettelheim, Anton} (18.11.1851 – 29.03.1930), \emph{Kritiker, Lexikograf}!Zum Saekulartag Eduard Devrients1901-08-10@\strich\emph{Zum Säkulartag Eduard Devrients} {[}1901-08-10{]}|pwuv}}{\lemma{\textnormal{\emph{Bettelheim uns beflegelt}}}\Cendnote{\textnormal{unklar. Am Tag des Briefes erschien in der
                  Beilage ein längerer Text über Eduard
                     Devrient\pwindex{Devrient, Eduard 11.08.1801 – 04.10.1877@\textsc{Devrient, Eduard} (11.08.1801 – 04.10.1877), \emph{Theaterleiter, Schauspieler}|pwk}, der mehrere Seitenhiebe auf populäres Theater enthält, doch ob
                     Schnitzler\pwindex{Schnitzler, Arthur 15.05.1862 – 21.10.1931@\textsc{Schnitzler, Arthur} (15.05.1862 – 21.10.1931), \emph{Schriftsteller, Mediziner}|pwk} davon schon Kenntnis gehabt
                  und sich angesprochen gefühlt hätte, ist zweifelhaft. (Anton Bettelheim\pwindex{Bettelheim, Anton 18.11.1851 – 29.03.1930@\textsc{Bettelheim, Anton} (18.11.1851 – 29.03.1930), \emph{Kritiker, Lexikograf}|pwk}: \emph{Zum
                        Säkulartag Eduard Devrients}\pwindex{Bettelheim, Anton 18.11.1851 – 29.03.1930@\textsc{Bettelheim, Anton} (18.11.1851 – 29.03.1930), \emph{Kritiker, Lexikograf}!Zum Saekulartag Eduard Devrients1901-08-10@\strich\emph{Zum Säkulartag Eduard Devrients} {[}1901-08-10{]}|pwk}. In: \emph{Allgemeine Zeitung}\pwindex{?? Werk@Nicht ermittelte Verfasserinnen und Verfasser!Allgemeine Zeitung1798 – 30.6.1929@\emph{Allgemeine Zeitung} {[}1798 – 30.6.1929{]}|pwk}, Beilage, Nr. 182,
                        10. 8. 1901, S. 1–6.)}}}\label{K_L02969-11h} haben ſoll?– \pend
           \pstart
           Leben Sie wohl und ſeien Sie herzlich gegrüßt. \pend
           \pstart
           Das neue Stück\textcolor{red}{\textsuperscript{\textbf{KEY}}} iſt doch nicht fertig,
               ka{\geminationn} es aber bald ſein. {\pb}Dafür 2 Einakter\textcolor{red}{\textsuperscript{\textbf{KEY}}}, die zu »Literatur\pwindex{Schnitzler, Arthur 15.05.1862 – 21.10.1931@\textsc{Schnitzler, Arthur} (15.05.1862 – 21.10.1931), \emph{Schriftsteller, Mediziner}!Literatur1901@\strich\emph{Literatur} {[}1901{]}|pw}« dazu gegeben werden ſollen. \pend
           \pstart Ihr \spacefill\mbox{A.}\pend{}
         
         \endnumbering\mylabel{h}\end{ledgroupsized}\begin{anhang}\end{anhang}\newcommand{\dateiname}{L02969}\newcommand{\titel}{Arthur Schnitzler an Felix Salten, 10. 8. 1901}\newcommand{\editorInnen}{Martin Anton Müller und Laura Untner}%% latex-leseansicht-abspann.tex
%% Abspann für die Leseansicht.
%% Der Schalter \ifkorrekturansicht ist bereits durch den Vorspann gesetzt.

%% latex-abspann.tex
%% Gemeinsamer Abspann für Korrekturansicht und Leseansicht.
%% Setzt den Schalter \ifkorrekturansicht voraus (gesetzt in den
%% einbindenden Dateien latex-korrekturansicht-abspann.tex bzw.
%% latex-leseansicht-abspann.tex).
%% ---------------------------------------------------------------

\normalsize

% Das esempio-Environment wird nur in der Leseansicht benötigt
\ifkorrekturansicht\else
\newenvironment{esempio}[3]%
{
    \vspace{1.5ex}
    \rlap{\underline{#1}}
    \par
    \setlength{\parindent}{0cm}
    \nopagebreak
    \leftskip=#2cm
    \rightskip=#3cm
}
{
    \par
}
\fi

\doendnotes{C}
\bigskip
\vfill

\clearpage

\footnotesize

\ifkorrekturansicht
  \lohead{\textsc{register}}
\fi

% theindex-Environment neu definieren ohne reledmac
\makeatletter
\renewenvironment{theindex}{%
  \ifkorrekturansicht
    \section*{\indexname}%
  \else
    \subsubsection*{Index der erwähnten Entitäten}%
  \fi
  \setlength{\parindent}{0pt}%
  \setlength{\parskip}{0pt plus 0.3pt}%
  \let\item\@idxitem
}{%
  \ifkorrekturansicht\clearpage\fi
}
\makeatother

\IfFileExists{\jobname-pw.ind}{\input{\jobname-pw.ind}}{}

% Quellenangabe nur in der Leseansicht
\ifkorrekturansicht\else
% Fallback-Definitionen, falls die .tex-Datei \titel etc. nicht gesetzt hat
\providecommand{\titel}{}
\providecommand{\editorInnen}{}
\providecommand{\dateiname}{\jobname}

\vspace{3cm}

\vfill

\footnotesize
\textsc{Quelle}: \titel. Herausgegeben von {\editorInnen}. In: \emph{Arthur Schnitzler: Briefwechsel mit Autorinnen und Autoren}.
 Digitale Edition, https://schnitzler-briefe.acdh.oeaw.ac.at/{\dateiname}.html (Stand \today)
\fi

\end{document}


      