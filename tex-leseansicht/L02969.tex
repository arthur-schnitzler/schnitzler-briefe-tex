%% latex-leseansicht-vorspann.tex
%% Vorspann für die Leseansicht.
%% Lädt die gemeinsame Datei latex-vorspann.tex mit nicht gesetztem Schalter.

\newif\ifkorrekturansicht
\korrekturansichtfalse

\input{../tex-inputs/latex-vorspann}


         
         \renewcommand{\erwaehntePersonen}{Personen: Anton Bettelheim, Eduard Devrient, Paul Goldmann, Felix Salten}
         \renewcommand{\erwaehnteInstitutionen}{Institutionen: Jung-Wiener Theater zum Lieben Augustin, Überbrettl}
         \renewcommand{\erwaehnteOrte}{Orte: Berlin, Bozen, Lago di Garda, Pustertal, Pörtschach, Trient, Vahrn, Venedig, Verona, Welsberg-Taisten, Wien, Wildbad Waldbrunn}
         \renewcommand{\erwaehnteWerke}{Werke: Allgemeine Zeitung, Der einsame Weg. Schauspiel in fünf Akten, Die Frau mit dem Dolche, Die Gedenktafel der Prinzessin Anna, Die Insel. Monatsschrift mit Buchschmuck und Illustrationen, Lebendige Stunden, Literatur, Zum Säkulartag Eduard Devrients}
               \section[ Arthur Schnitzler an Felix Salten, 10. 8. 1901]{ Arthur Schnitzler an Felix Salten, 10. 8. 1901}\nopagebreak\mylabel{v}\rehead{ }\begin{ledgroupsized}[t]{13cm}\normalsize\beginnumbering\briefempfaengerindex{Salten, Felix@\textsc{Salten, Felix}!zzzSchnitzler, Arthur@\emph{von Arthur Schnitzler}!1901-08-103@{10. 8. 1901}|(be} \toendnotes[C]{\smallbreak\pagebreak[2]} \Standort{Wienbibliothek im Rathaus, ZPH 1681, 2.1.516.}
\physDesc{Brief, 1 Blatt, 4 Seiten, 919 Zeichen
\newline{}Handschrift: Bleistift, deutsche Kurrent
\newline{}Ordnung: mit Bleistift von unbekannter Hand Nummerierung der Doppelseiten des
                                 Konvoluts: »24«–»25« }\toendnotes[C]{\smallbreak}\pstart
           \raggedleft{}{\pb}\textsc{Vahrn\oindex{Vahrn@\textbf{Vahrn}|pw}}, 10. 8. 901\pend
           \pstart
           Mein lieber Freund,{ }heut ſinds 4 Wochen, dſs ich hier\oindex{Vahrn@\textbf{Vahrn}|pwv} bin, habe mich ſehr wohlgefühlt; Montag nach Bozen\oindex{Bozen@\textbf{Bozen}|pw},
               woſelbſt Paul Goldma{\geminationn}\pwindex{Goldmann, Paul 31.01.1865 – 25.09.1935@\textsc{Goldmann, Paul} (31.01.1865 – 25.09.1935), \emph{Schriftsteller, Journalist}|pw}, dann Trient\oindex{Trient@\textbf{Trient}|pw}, aber wir haben uns nicht zum
                  Gardaſee\oindex{Lago di Garda@\textbf{Lago di Garda}|pw}, ſondern zu einem ſehr schönen Ort
               im Puſterthal\oindex{Pustertal@\textbf{Pustertal}|pw} entſchloſſen, Welsberg\oindex{Welsberg-Taisten@\textbf{Welsberg-Taisten}|pw}, Penſion {\pb}Waldbrunn\oindex{Wildbad Waldbrunn@\textbf{Wildbad Waldbrunn}|pw}; woſelbſt wir etwa bis
                  Ende Auguſt verbleiben um da{\geminationn} direct \label{K_L02969-1v}\edtext{nach Wien\oindex{Wien@\textbf{Wien}|pw} zurückzukehren}{\lemma{\textnormal{\emph{nach Wien zurückzukehren}}}\Cendnote{\textnormal{Nach einem kurzen Aufenthalt in Pörtschach am Wörthersee\oindex{Poertschach@\textbf{Pörtschach}|pwk} (27. 8. 1901 bis 29. 8. 1901) kehrte Schnitzler\pwindex{Schnitzler, Arthur 15.05.1862 – 21.10.1931@\textsc{Schnitzler, Arthur} (15.05.1862 – 21.10.1931), \emph{Schriftsteller, Mediziner}|pwk} am 30. 8. 1901 nach Wien\oindex{Wien@\textbf{Wien}|pwk} zurück. Nachweislich sahen sich Salten\pwindex{Salten, Felix 06.09.1869 – 08.10.1945@\textsc{Salten, Felix} (06.09.1869 – 08.10.1945), \emph{Schriftsteller, Journalist}|pwk} und Schnitzler\pwindex{Schnitzler, Arthur 15.05.1862 – 21.10.1931@\textsc{Schnitzler, Arthur} (15.05.1862 – 21.10.1931), \emph{Schriftsteller, Mediziner}|pwk} dort am 1. 9. 1901
                  wieder.}}}\label{K_L02969-1h}. So treff’ ich Sie wahrſcheinlich dort noch an, bevor Sie nach \textsc{Verona\oindex{Verona@\textbf{Verona}|pw}} oder \textsc{Venedig\oindex{Venedig@\textbf{Venedig}|pw}} fahren. Wollen Sie mir das Inſelheft\pwindex{Insel. Monatsschrift mit Buchschmuck und Illustrationen1899 – 1902@\emph{Die Insel. Monatsschrift mit Buchschmuck und Illustrationen} {[}1899 – 1902{]}|pwv}\pwindex{Salten, Felix 06.09.1869 – 08.10.1945@\textsc{Salten, Felix} (06.09.1869 – 08.10.1945), \emph{Schriftsteller, Journalist}!Gedenktafel der Prinzessin Anna1901-07-01@\strich\emph{Die Gedenktafel der Prinzessin Anna} {[}1901-07-01{]}|pwv} nach \textsc{Welsberg\oindex{Welsberg-Taisten@\textbf{Welsberg-Taisten}|pw}} ſchicken? wäre Ihnen ſehr dankbar. – Das \label{K_L02969-2v}\edtext{Brettl\orgindex{Jung-Wiener Theater zum Lieben Augustin@Jung-Wiener Theater zum Lieben Augustin|pwv}}{\lemma{\textnormal{\emph{Brettl}}}\Cendnote{\textnormal{Synonym für ›Kabarett‹; das \emph{Jung-Wiener Theater zum Lieben Augustin}\orgindex{Jung-Wiener Theater zum Lieben Augustin@Jung-Wiener Theater zum Lieben Augustin|pwk} hatte das Berlin\oindex{Berlin@\textbf{Berlin}|pwk}er \emph{Überbrettl}\orgindex{Ueberbrettl@Überbrettl|pwk} als Vorbild.}}}\label{K_L02969-2h} macht Ihnen natürlich viel
                  Mühe\textcolor{gray}{;} –{ }{\pb}– daſs der Erfolg nicht von Wien\oindex{Wien@\textbf{Wien}|pw} beſtritten werden kann, war vom erſten Moment an klar.
               Könnten Sie mir die Nummer der Allg. (Münchner)\pwindex{?? Werk@Nicht ermittelte Verfasserinnen und Verfasser!Allgemeine Zeitung1798 – 30.6.1929@\emph{Allgemeine Zeitung} {[}1798 – 30.6.1929{]}|pw}
               verſchaffen, wo dieſer \label{K_L02969-3v}\edtext{Bettelheim\pwindex{Bettelheim, Anton 18.11.1851 – 29.03.1930@\textsc{Bettelheim, Anton} (18.11.1851 – 29.03.1930), \emph{Kritiker, Lexikograf}|pw} uns beflegelt\pwindex{Bettelheim, Anton 18.11.1851 – 29.03.1930@\textsc{Bettelheim, Anton} (18.11.1851 – 29.03.1930), \emph{Kritiker, Lexikograf}!Zum Saekulartag Eduard Devrients1901-08-10@\strich\emph{Zum Säkulartag Eduard Devrients} {[}1901-08-10{]}|pwuv}}{\lemma{\textnormal{\emph{Bettelheim uns beflegelt}}}\Cendnote{\textnormal{Am Tag des Briefes erschien in der
                  Beilage ein längerer Text über Eduard
                     Devrient\pwindex{Devrient, Eduard 11.08.1801 – 04.10.1877@\textsc{Devrient, Eduard} (11.08.1801 – 04.10.1877), \emph{Theaterleiter, Schauspieler}|pwk}, der mehrere Seitenhiebe auf populäres Theater enthielt. Ob Schnitzler\pwindex{Schnitzler, Arthur 15.05.1862 – 21.10.1931@\textsc{Schnitzler, Arthur} (15.05.1862 – 21.10.1931), \emph{Schriftsteller, Mediziner}|pwk} davon schon Kenntnis gehabt haben konnte, ist zweifelhaft. Vgl. Anton Bettelheim\pwindex{Bettelheim, Anton 18.11.1851 – 29.03.1930@\textsc{Bettelheim, Anton} (18.11.1851 – 29.03.1930), \emph{Kritiker, Lexikograf}|pwk}: \emph{Zum
                        Säkulartag Eduard Devrients}\pwindex{Bettelheim, Anton 18.11.1851 – 29.03.1930@\textsc{Bettelheim, Anton} (18.11.1851 – 29.03.1930), \emph{Kritiker, Lexikograf}!Zum Saekulartag Eduard Devrients1901-08-10@\strich\emph{Zum Säkulartag Eduard Devrients} {[}1901-08-10{]}|pwk}. In: \emph{Allgemeine Zeitung}\pwindex{?? Werk@Nicht ermittelte Verfasserinnen und Verfasser!Allgemeine Zeitung1798 – 30.6.1929@\emph{Allgemeine Zeitung} {[}1798 – 30.6.1929{]}|pwk}, Beilage, Nr. 182, 10. 8. 1901, S. 1–6.}}}\label{K_L02969-3h} haben ſoll? –\pend
           \pstart
           Leben Sie wohl und ſeien Sie herzlich gegrüßt.\pend
           \pstart
           Das neue \label{K_L02969-4v}\edtext{Stück\pwindex{Schnitzler, Arthur 15.05.1862 – 21.10.1931@\textsc{Schnitzler, Arthur} (15.05.1862 – 21.10.1931), \emph{Schriftsteller, Mediziner}!einsame Weg. Schauspiel in fuenf Akten1904@\strich\emph{Der einsame Weg. Schauspiel in fünf Akten} {[}1904{]}|pwv}}{\lemma{\textnormal{\emph{Stück}}}\Cendnote{\textnormal{\emph{Der einsame Weg}\pwindex{Schnitzler, Arthur 15.05.1862 – 21.10.1931@\textsc{Schnitzler, Arthur} (15.05.1862 – 21.10.1931), \emph{Schriftsteller, Mediziner}!einsame Weg. Schauspiel in fuenf Akten1904@\strich\emph{Der einsame Weg. Schauspiel in fünf Akten} {[}1904{]}|pwk}, den Schnitzler\pwindex{Schnitzler, Arthur 15.05.1862 – 21.10.1931@\textsc{Schnitzler, Arthur} (15.05.1862 – 21.10.1931), \emph{Schriftsteller, Mediziner}|pwk} am 21. 7. 1901 vorläufig abgeschlossen hatte und am
                     20. 11. 1901 neu
                  zu bearbeiten begann}}}\label{K_L02969-4h} iſt doch nicht fertig, ka{\geminationn} es aber bald ſein. {\pb}Dafür \label{K_L02969-5v}\edtext{2 Einakter\pwindex{Schnitzler, Arthur 15.05.1862 – 21.10.1931@\textsc{Schnitzler, Arthur} (15.05.1862 – 21.10.1931), \emph{Schriftsteller, Mediziner}!Lebendige Stunden01. 12. 1901@\strich\emph{Lebendige Stunden} {[}01. 12. 1901{]}|pwv}\pwindex{Schnitzler, Arthur 15.05.1862 – 21.10.1931@\textsc{Schnitzler, Arthur} (15.05.1862 – 21.10.1931), \emph{Schriftsteller, Mediziner}!Frau mit dem Dolche1901@\strich\emph{Die Frau mit dem Dolche} {[}1901{]}|pwv}}{\lemma{\textnormal{\emph{2 Einakter}}}\Cendnote{\textnormal{\emph{Lebendige Stunden}\pwindex{Schnitzler, Arthur 15.05.1862 – 21.10.1931@\textsc{Schnitzler, Arthur} (15.05.1862 – 21.10.1931), \emph{Schriftsteller, Mediziner}!Lebendige Stunden01. 12. 1901@\strich\emph{Lebendige Stunden} {[}01. 12. 1901{]}|pwk} hatte er am 28. 7. 1901
                  und \emph{Die Frau mit dem Dolche}\pwindex{Schnitzler, Arthur 15.05.1862 – 21.10.1931@\textsc{Schnitzler, Arthur} (15.05.1862 – 21.10.1931), \emph{Schriftsteller, Mediziner}!Frau mit dem Dolche1901@\strich\emph{Die Frau mit dem Dolche} {[}1901{]}|pwk} am 3. 8. 1901 fertiggestellt.}}}\label{K_L02969-5h},
               die zu »Literatur\pwindex{Schnitzler, Arthur 15.05.1862 – 21.10.1931@\textsc{Schnitzler, Arthur} (15.05.1862 – 21.10.1931), \emph{Schriftsteller, Mediziner}!Literatur1901@\strich\emph{Literatur} {[}1901{]}|pw}« dazu gegeben werden
               ſollen.\pend
           \pstart Ihr \spacefill\mbox{A.}\pend{}
         
         \endnumbering\mylabel{h}\end{ledgroupsized}  \newcommand{\dateiname}{L02969}\newcommand{\titel}{Arthur Schnitzler an Felix Salten, 10. 8. 1901}\newcommand{\editorInnen}{Martin Anton Müller und Laura Untner}%% latex-leseansicht-abspann.tex
%% Abspann für die Leseansicht.
%% Der Schalter \ifkorrekturansicht ist bereits durch den Vorspann gesetzt.

%% latex-abspann.tex
%% Gemeinsamer Abspann für Korrekturansicht und Leseansicht.
%% Setzt den Schalter \ifkorrekturansicht voraus (gesetzt in den
%% einbindenden Dateien latex-korrekturansicht-abspann.tex bzw.
%% latex-leseansicht-abspann.tex).
%% ---------------------------------------------------------------

\normalsize

% Das esempio-Environment wird nur in der Leseansicht benötigt
\ifkorrekturansicht\else
\newenvironment{esempio}[3]%
{
    \vspace{1.5ex}
    \rlap{\underline{#1}}
    \par
    \setlength{\parindent}{0cm}
    \nopagebreak
    \leftskip=#2cm
    \rightskip=#3cm
}
{
    \par
}
\fi

\doendnotes{C}
\bigskip
\vfill

\clearpage

\footnotesize

\ifkorrekturansicht
  \lohead{\textsc{register}}
\fi

% theindex-Environment neu definieren ohne reledmac
\makeatletter
\renewenvironment{theindex}{%
  \ifkorrekturansicht
    \section*{\indexname}%
  \else
    \subsubsection*{Index der erwähnten Entitäten}%
  \fi
  \setlength{\parindent}{0pt}%
  \setlength{\parskip}{0pt plus 0.3pt}%
  \let\item\@idxitem
}{%
  \ifkorrekturansicht\clearpage\fi
}
\makeatother

\IfFileExists{\jobname-pw.ind}{\input{\jobname-pw.ind}}{}

% Quellenangabe nur in der Leseansicht
\ifkorrekturansicht\else
% Fallback-Definitionen, falls die .tex-Datei \titel etc. nicht gesetzt hat
\providecommand{\titel}{}
\providecommand{\editorInnen}{}
\providecommand{\dateiname}{\jobname}

\vspace{3cm}

\vfill

\footnotesize
\textsc{Quelle}: \titel. Herausgegeben von {\editorInnen}. In: \emph{Arthur Schnitzler: Briefwechsel mit Autorinnen und Autoren}.
 Digitale Edition, https://schnitzler-briefe.acdh.oeaw.ac.at/{\dateiname}.html (Stand \today)
\fi

\end{document}


      