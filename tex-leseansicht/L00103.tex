%% latex-korrekturansicht-vorspann.tex
%% Vorspann für die Korrekturansicht.
%% Lädt die gemeinsame Datei latex-vorspann.tex mit gesetztem Schalter.

\newif\ifkorrekturansicht
\korrekturansichttrue

\input{../tex-inputs/latex-vorspann}


\section[Hugo von Hofmannsthal an Arthur Schnitzler, {[}2. 7. 1892{]}]{L00103 Hugo von Hofmannsthal an Arthur Schnitzler, {[}2. 7. 1892{]}}
\nopagebreak\mylabel{L00103v}
\rehead{ }\normalsize\beginnumbering\briefempfaengerindex{Schnitzler, Arthur@\textsc{Schnitzler, Arthur}!zzzHofmannsthal, Hugo von@\emph{von Hugo von Hofmannsthal}!1892-07-021@{{[}2. 7. 1892{]}}|(be}
\toendnotes[C]{\smallbreak\pagebreak[2]}\Standort{CUL, Schnitzler, B 43.}
\physDesc{Brief, 1 Blatt, 1 Seite, 270 Zeichen
\newline{}Handschrift: schwarze Tinte, deutsche Kurrent
\newline{}Schnitzler: mit Bleistift datiert »2/7 92« und nummeriert: »25« }
\buchAbdrucke{\weitereDrucke{Hugo von Hofmannsthal, Arthur Schnitzler: \emph{Briefwechsel}. Frankfurt am Main: \emph{S. Fischer} 1964, S. 21–22.} }\toendnotes[C]{\smallbreak}
\pstart{}{\pb}Lieber Arthur.\pend\vspace{0.5em}
\pstart
           Beſten Dank. Mittwoch{ }abend bin ich \label{K_L00103-1v}\edtext{fertig}{\lemma{\textnormal{\emph{fertig}}}\Cendnote{\textnormal{mit der Matura (Reifeprüfung)}}}\label{K_L00103-1}. Ich
               möchte ſehr gern den Donnerstag- oder Freitagabend mit Ihnen und Salten\pwindex{Salten, Felix 06.09.1869 – 08.10.1945@\textsc{Salten, Felix} (06.09.1869 – 08.10.1945), \emph{Schriftsteller/Schriftstellerin, Journalist/Journalistin, Chefredakteur/Chefredakteurin}|pw}
               zubringen, incluſive \textsc{Souper}, (Ausſtellung\oindex{Internationales Ausstellungstheater im k.k. Prater@\textbf{Internationales Ausstellungstheater im k.k. Prater}, \emph{Theater (K.THE)}|pw}?) und bitte um baldige freundliche Entſcheidung,
               damit ich mir das übrige danach einrichten kann.\pend
           \pstart \spacefill\mbox{Loris.}\pend{}\selectlanguage{ngerman}\endnumbering\briefempfaengerindex{Schnitzler, Arthur@\textsc{Schnitzler, Arthur}!zzzHofmannsthal, Hugo von@\emph{von Hugo von Hofmannsthal}!1892-07-021@{{[}2. 7. 1892{]}}|)be}\mylabel{L00103h}  \normalsize

\doendnotes{C}
\bigskip
\vfill

\clearpage

\footnotesize

\lohead{\textsc{register}}

% Definiere theindex-Environment komplett neu ohne reledmac
\makeatletter
\renewenvironment{theindex}{%
  \section*{\indexname}%
  \setlength{\parindent}{0pt}%
  \setlength{\parskip}{0pt plus 0.3pt}%
  \let\item\@idxitem
}{%
  \clearpage
}
\makeatother

\IfFileExists{\jobname-pw.ind}{\input{\jobname-pw.ind}}{}

\end{document}

      