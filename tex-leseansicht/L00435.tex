%% latex-korrekturansicht-vorspann.tex
%% Vorspann für die Korrekturansicht.
%% Lädt die gemeinsame Datei latex-vorspann.tex mit gesetztem Schalter.

\newif\ifkorrekturansicht
\korrekturansichttrue

\input{../tex-inputs/latex-vorspann}


\section[Lou Andreas-Salomé an Arthur Schnitzler, {[}1. 5. 1895{]}]{L00435 Lou Andreas-Salomé an Arthur Schnitzler, {[}1. 5. 1895{]}}
\nopagebreak\mylabel{L00435v}
\rehead{ }\normalsize\beginnumbering\briefempfaengerindex{Schnitzler, Arthur@\textsc{Schnitzler, Arthur}!zzzAndreas-Salome, Lou@\emph{von Lou Andreas-Salomé}!1895-05-011@{{[}1. 5. 1895{]}}|(be}
\toendnotes[C]{\smallbreak\pagebreak[2]}\Standort{CUL, Schnitzler, B 3.}
\physDesc{Briefkarte, 331 Zeichen
\newline{}Handschrift: schwarze Tinte, deutsche Kurrent
\newline{}Schnitzler: mit Bleistift datiert »1/5 95« 
\newline{}Ordnung: mit rotem Buntstift von unbekannter Hand nummeriert
                                    »4« }\toendnotes[C]{\smallbreak}
\pstart{}{\pb}Sehr geehrter Herr \textsc{D\textsuperscript{r}},\pend\vspace{0.5em}
\pstart
           Wollen wir einen Abend zuſammen verplaudern? Vielleicht morgen, \label{K_L00435-1v}\edtext{Donnerstag}{\lemma{\textnormal{\emph{Donnerstag}}}\Cendnote{\textnormal{Das erste durch Schnitzlers{ }\emph{Tagebuch}\pwindex{Tagebuch@\emph{Tagebuch}|pwk}
                   belegbare Treffen fand im Frühjahr 1895 am Freitag, dem 3. 5. 1895 statt.
                  Folglich dürfte die Datierung Schnitzlers stimmen.}}}\label{K_L00435-1}? Ich bin wieder wohl und werde daher ein
               beſſerer Geſellſchafter ſein als Montag.\pend
           
\pstart
           Mein Zimmer iſt jetzt N\textsuperscript{o} 58, III, am Lift, Haupttreppe;
               es iſt ſehr eng, wir brauchen aber nicht darin ſitzen zu bleiben.\pend
           
\pstart
           Mit herzl. Gruß{\\[\baselineskip]}\spacefill\mbox{Lou Andreas-Salomé.}\pend
           \leftskip=0em{}\selectlanguage{ngerman}\endnumbering\briefempfaengerindex{Schnitzler, Arthur@\textsc{Schnitzler, Arthur}!zzzAndreas-Salome, Lou@\emph{von Lou Andreas-Salomé}!1895-05-011@{{[}1. 5. 1895{]}}|)be}\mylabel{L00435h}  \normalsize

\doendnotes{C}
\bigskip
\vfill

\clearpage

\footnotesize

\lohead{\textsc{register}}

% Definiere theindex-Environment komplett neu ohne reledmac
\makeatletter
\renewenvironment{theindex}{%
  \section*{\indexname}%
  \setlength{\parindent}{0pt}%
  \setlength{\parskip}{0pt plus 0.3pt}%
  \let\item\@idxitem
}{%
  \clearpage
}
\makeatother

\IfFileExists{\jobname-pw.ind}{\input{\jobname-pw.ind}}{}

\end{document}

      