\input{../tex-inputs/latex-pdf-vorspann}
\begin{center}
            \textcolor{red}{ENTWURF. ENTZIFFERUNG NOCH NICHT KORREKTURGELESEN}
                      \end{center}
            
               \section[Hermann Bahr an Arthur Schnitzler, {[}17.? 7. 1894{]}]{ Hermann Bahr an Arthur Schnitzler, {[}17.? 7. 1894{]}}\nopagebreak\mylabel{v}\rehead{ }\begin{ledgroupsized}[t]{13cm}\normalsize\beginnumbering\briefempfaengerindex{Schnitzler, Arthur@\textsc{Schnitzler, Arthur}!zzzBahr, Hermann@\emph{von Hermann Bahr}!1894-07-171@{{[}17.? 7. 1894{]}}|(be} \toendnotes[C]{\smallbreak\pagebreak[2]} \Standort{CUL, Schnitzler, B 5b.}
\physDesc{Briefkarte
\newline{}Handschrift: Bleistift, lateinische Kurrent
\newline{}Schnitzler: mit Bleistift datiert: »Juli 94« \newline{}Ordnung: mit rotem Buntstift von unbekannter Hand und mit Bleistift
                           jeweils nummeriert: »23« }\buchAbdrucke{\weitereDrucke{Hermann Bahr, Arthur Schnitzler: \emph{Briefwechsel, Aufzeichnungen, Dokumente (1891–1931)}. Hg. Kurt Ifkovits und Martin Anton Müller. Göttingen: \emph{Wallstein} 2018, S. 75.} }\toendnotes[C]{\smallbreak}\pstart
           \noindent{}{\pb}Die Fahrt nach Salzburg\oindex{Salzburg@\textbf{Salzburg}|pw} werde ich wol \label{K_L00353_1v}\edtext{nicht
               mitmachen können}{\lemma{\textnormal{\emph{nicht
               mitmachen können}}}\Cendnote{\textnormal{Auch Hofmannsthal\pwindex{Hofmannsthal, Hugo von 01.02.1874 – 15.07.1929@\textsc{Hofmannsthal, Hugo von} (01.02.1874 – 15.07.1929), \emph{Schriftsteller}|pwk} reagiert am 18. 7. auf ein
                  nicht erhalten gebliebenes Schreiben, in dem ihm Bahr\pwindex{Bahr, Hermann 19.07.1863 – 15.01.1934@\textsc{Bahr, Hermann} (19.07.1863 – 15.01.1934), \emph{Schriftsteller, Kritiker}|pwk} mitteilte, ihn nicht in Salzburg\oindex{Salzburg@\textbf{Salzburg}|pwk}
                  zu treffen (\emph{Briefwechsel} Bahr/Hofmannsthal 52–53).}}}\label{K_L00353_1h}. Möchte aber
               gern in Ischl\oindex{Bad Ischl@\textbf{Bad Ischl}|pw} mit Dir zusa{\geminationm}en sein. Paßt Dir\strikeout{s},
               wenn ich Samstag den 21. in der Früh komme u. bis
               Abends bleibe?\pend
           \pstart
           Herzlichst grüßt{\\[\baselineskip]}\spacefill\mbox{Hermann}\pend
           \leftskip=0em{}\endnumbering\briefempfaengerindex{Schnitzler, Arthur@\textsc{Schnitzler, Arthur}!zzzBahr, Hermann@\emph{von Hermann Bahr}!1894-07-171@{{[}17.? 7. 1894{]}}|)be}\mylabel{h}\end{ledgroupsized}  \newcommand{\dateiname}{L00353}\newcommand{\titel}{Hermann Bahr an Arthur Schnitzler, [17.? 7. 1894]}\newcommand{\editorInnen}{ Kurt Ifkovits,  Martin Anton Müller}\input{../tex-inputs/latex-pdf-abspann}
      