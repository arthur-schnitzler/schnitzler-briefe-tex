%% latex-korrekturansicht-vorspann.tex
%% Vorspann für die Korrekturansicht.
%% Lädt die gemeinsame Datei latex-vorspann.tex mit gesetztem Schalter.

\newif\ifkorrekturansicht
\korrekturansichttrue

\input{../tex-inputs/latex-vorspann}


\section[ Paul Goldmann an Arthur Schnitzler, 20. 2. 1900]{L02905 Paul Goldmann an Arthur Schnitzler, 20. 2. 1900}
\nopagebreak\mylabel{L02905v}
\rehead{ }\normalsize\beginnumbering\briefempfaengerindex{Schnitzler, Arthur@\textsc{Schnitzler, Arthur}!zzzGoldmann, Paul@\emph{von Paul Goldmann}!1900-02-201@{20. 2. 1900}|(be}
\toendnotes[C]{\smallbreak\pagebreak[2]}\Standort{DLA, A:Schnitzler, HS.NZ85.1.3170.}
\physDesc{Brief, 2 Blätter, 7 Seiten, 4085 Zeichen
\newline{}Handschrift: schwarze Tinte, deutsche Kurrent
\newline{}Schnitzler: mit rotem Buntstift drei Unterstreichungen }\toendnotes[C]{\smallbreak}
\pstart
           \centering{}{\pb}\textcolor{gray}{\textbf{\textbf{HOTEL SAXONIA\oindex{Hotel Saxonia@\textbf{Hotel Saxonia}, \emph{Hotel (K.HTL)}|pw}}}}\pend
           
\pstart
           \raggedleft{}\textcolor{gray}{\textbf{am Potsdamer Platz\oindex{Potsdamer Platz@\textbf{Potsdamer Platz}, \emph{Platz (K.PLT)}|pw} und
                        Thiergarten\oindex{Tiergarten@\textbf{Tiergarten}, \emph{P.PPLX}|pw}}}\pend
           
\pstart
           \centering{}\textcolor{gray}{\textbf{D. W. SCHRÖDER\pwindex{Schroeder, D. W. @\textsc{Schröder, D. W.}, \emph{Hotelbesitzer/Hotelbesitzerin}|pw}.}}\pend
           
\pstart
           \textcolor{gray}{\textbf{Fernsprecher:}}\pend
           
\pstart
           \textcolor{gray}{\textbf{\textbf{Amt VI. No. 2838.}}}\pend
           
\pstart
           \raggedleft{}\textcolor{gray}{\textbf{\emph{BERLIN W.}\oindex{Berlin@\textbf{Berlin}, \emph{P.PPLC}|pw}, den}}{ }20. Februar \textcolor{gray}{\textbf{1}}900. \pend
           
\pstart
           \raggedleft{}\textcolor{gray}{\textbf{Königgrätzerstrasse 10\oindex{Stresemannstrasse@\textbf{Stresemannstraße}, \emph{Straße (K.STR)}|pw}.}}\pend
           
\pstart{}Mein lieber Freund,\pend\vspace{0.5em}
\pstart
           Ich will gleich auf Deinen lieben Brief antworten, ſonſt komme ich lange nicht
               dazu.\pend
           
\pstart
           Es freut mich ſehr, daß Du mit meiner \label{K_L02905-1v}\edtext{Anſicht}{\lemma{\textnormal{\emph{Anſicht}}}\Cendnote{\textnormal{Siehe Paul Goldmann an Arthur Schnitzler, 11. 2. 1900.
               }}}\label{K_L02905-1} über dein Stück\pwindex{Schleier der Beatrice. Schauspiel in fuenf Akten@\emph{Der Schleier der Beatrice. Schauspiel in fünf Akten}|pwv} zum
               Theil einverſtanden biſt. Ich habe noch einmal Dieſes und Jenes geleſen\strikeout{,} und kann Dir nur ſagen: Seit \textsc{Grillparzer\pwindex{Grillparzer, Franz 15.01.1791 – 21.01.1872@\textsc{Grillparzer, Franz} (15.01.1791 – 21.01.1872), \emph{Schriftsteller/Schriftstellerin, Beamter/Beamte}|pw}} hat man auf dem Wien\oindex{Wien@\textbf{Wien}, \emph{A.ADM2}|pw}er Theater ſolche Verſe\pwindex{Schleier der Beatrice. Schauspiel in fuenf Akten@\emph{Der Schleier der Beatrice. Schauspiel in fünf Akten}|pwv} nicht gehört. Das ſoll
               aber nicht bedeuten, daß es \textsc{Grillparzer\pwindex{Grillparzer, Franz 15.01.1791 – 21.01.1872@\textsc{Grillparzer, Franz} (15.01.1791 – 21.01.1872), \emph{Schriftsteller/Schriftstellerin, Beamter/Beamte}|pw}ische} Verſe ſind. Nein, ſie
               ſind durchaus \textsc{Schnitzlerisch}, und nur der weiche Wien\oindex{Wien@\textbf{Wien}, \emph{A.ADM2}|pw}er Wohllaut iſt den beiden Dichtern\pwindex{Grillparzer, Franz 15.01.1791 – 21.01.1872@\textsc{Grillparzer, Franz} (15.01.1791 – 21.01.1872), \emph{Schriftsteller/Schriftstellerin, Beamter/Beamte}|pwv} gemeinſam. Was die Aufführung
               anlangt, ſo {\pb}möchte ich Streichungen empfehlen.
               Vielleicht auch einige \label{K_L02905-2v}\edtext{Umarbeitungen}{\lemma{\textnormal{\emph{Umarbeitungen}}}\Cendnote{\textnormal{Entsprechende
                  Umarbeitungen sind keine bekannt.}}}\label{K_L02905-2}. Ich bleibe dabei: die Geſtalt des Herzogs\pwindex{Schleier der Beatrice. Schauspiel in fuenf Akten@\emph{Der Schleier der Beatrice. Schauspiel in fünf Akten}|pwv} erſcheint mir in zu unklaren
               Umriſſen. Wenn da auch nur ein wenig mit feſter Hand nachgezeichnet würde, könnte das
               dem Drama\pwindex{Schleier der Beatrice. Schauspiel in fuenf Akten@\emph{Der Schleier der Beatrice. Schauspiel in fünf Akten}|pwv} ſehr zum Vortheil
               gereichen. Wäre es nicht doch möglich, daß die Hochzeit nur ein im Voraus
               beabſichtigter Carnevals-Scherz ſein könnte? Wenn der Herzog\pwindex{Schleier der Beatrice. Schauspiel in fuenf Akten@\emph{Der Schleier der Beatrice. Schauspiel in fünf Akten}|pwv} durchaus edel ſein muß, ſo könnte
               der Edelmuth ja nachher erwachen. Mich hat übrigens in Deinem Briefe das Wort »Größe«
               ſtutzig gemacht. Warum ſoll der Herzog\pwindex{Schleier der Beatrice. Schauspiel in fuenf Akten@\emph{Der Schleier der Beatrice. Schauspiel in fünf Akten}|pwv} »groß« ſein? Mir ſcheint, dieſes Streben nach Größe, dieſe abſtrakt
               hinzugedachte Eigenſchaft, iſt an der Unklarheit ſchuld. Hätteſt Du ihn nur (wie es
               ſonſt Deine Gewohnheit iſt) ruhig und \substVorne{}\textsuperscript{natürlich}\substDazwischen{}natürlich\substHinten{} leben laſſen, wie er leben mochte, ſo wäre \strikeout{\textcolor{gray}{er}} er deutlicher und wahrer geworden. Im Übrigen, vielleicht haſt Du Recht, und
                  {\pb}auf der Bühne zeigt ſich vielleicht, daß die Figur\pwindex{Schleier der Beatrice. Schauspiel in fuenf Akten@\emph{Der Schleier der Beatrice. Schauspiel in fünf Akten}|pwv} richtig gedacht war.\pend
           
\pstart
           Welche Rolle \label{K_L02905-3v}\edtext{\textsc{Kainz\pwindex{Kainz, Josef 02.01.1858 – 20.09.1910@\textsc{Kainz, Josef} (02.01.1858 – 20.09.1910), \emph{Schauspieler/Schauspielerin}|pw}}}{\lemma{\textnormal{\emph{Kainz}}}\Cendnote{\textnormal{Josef Kainz\pwindex{Kainz, Josef 02.01.1858 – 20.09.1910@\textsc{Kainz, Josef} (02.01.1858 – 20.09.1910), \emph{Schauspieler/Schauspielerin}|pwk} war ein von Schnitzler vielgeschätzter Schauspieler und mehrmals an
                  Inszenierungen seiner Dramen beteiligt. Für die geplante Uraufführung von \emph{Der Schleier der Beatrice}\pwindex{Schleier der Beatrice. Schauspiel in fuenf Akten@\emph{Der Schleier der Beatrice. Schauspiel in fünf Akten}|pwk} im Burgtheater\oindex{Burgtheater@\textbf{Burgtheater}, \emph{S.THTR}|pwk} wollte Schnitzler{ }Kainz\pwindex{Kainz, Josef 02.01.1858 – 20.09.1910@\textsc{Kainz, Josef} (02.01.1858 – 20.09.1910), \emph{Schauspieler/Schauspielerin}|pwk} in der Rolle des Filippo\pwindex{Schleier der Beatrice. Schauspiel in fuenf Akten@\emph{Der Schleier der Beatrice. Schauspiel in fünf Akten}|pwkv} sehen (vgl. Arthur Schnitzler an Richard Beer-Hofmann, 17. 2. 1900). Zu dieser Aufführung
                  kam es aber nicht (vgl. Paul Goldmann an Arthur Schnitzler, 12. 11. [1899]).}}}\label{K_L02905-3} ſpielen ſoll, kann ich Dir nicht ſagen. Denn ich kenne \textsc{Kainz\pwindex{Kainz, Josef 02.01.1858 – 20.09.1910@\textsc{Kainz, Josef} (02.01.1858 – 20.09.1910), \emph{Schauspieler/Schauspielerin}|pw}} nicht. Der Herzog\pwindex{Schleier der Beatrice. Schauspiel in fuenf Akten@\emph{Der Schleier der Beatrice. Schauspiel in fünf Akten}|pwv} muß
               jedenfalls ein vollendeter \uline{Sprecher} ſein, und mir
               ſcheint, daß \textsc{Kainz\pwindex{Kainz, Josef 02.01.1858 – 20.09.1910@\textsc{Kainz, Josef} (02.01.1858 – 20.09.1910), \emph{Schauspieler/Schauspielerin}|pw}} das nicht iſt. Für die \textsc{Beatrice\pwindex{Schleier der Beatrice. Schauspiel in fuenf Akten@\emph{Der Schleier der Beatrice. Schauspiel in fünf Akten}|pwv}} aber gibt es meiner Anſicht nach nur \uline{eine} auf
               den deutſchen Theatern: Die \label{K_L02905-4v}\edtext{\textsc{Triesch\pwindex{Triesch, Irene 13.04.1877 – 24.11.1964@\textsc{Triesch, Irene} (13.04.1877 – 24.11.1964), \emph{Schauspieler/Schauspielerin}|pw}}}{\lemma{\textnormal{\emph{Triesch}}}\Cendnote{\textnormal{Irene Triesch\pwindex{Triesch, Irene 13.04.1877 – 24.11.1964@\textsc{Triesch, Irene} (13.04.1877 – 24.11.1964), \emph{Schauspieler/Schauspielerin}|pwk} gestaltete erst 1903 die Beatrice\pwindex{Schleier der Beatrice. Schauspiel in fuenf Akten@\emph{Der Schleier der Beatrice. Schauspiel in fünf Akten}|pwkv} am \emph{Deutschen Theater Berlin}\orgindex{Deutsches Theater Berlin@Deutsches Theater Berlin|pwk}
                  aus. Schnitzler missfiel, wie sie die Rolle anlegte
                     (vgl. A. S.: \emph{Tagebuch}, 23. 2. 1903).}}}\label{K_L02905-4} in
                  Frankfurt\oindex{Frankfurt am Main@\textbf{Frankfurt am Main}, \emph{P.PPLA3}|pw}. Sie hat geniale Kunſt-Inſtinkte,
               iſt ſelbſt ein ſo unberechenbares Luder, wie Deine \textsc{Beatrice\pwindex{Schleier der Beatrice. Schauspiel in fuenf Akten@\emph{Der Schleier der Beatrice. Schauspiel in fünf Akten}|pwv}}, hat außerdem die Jugend und das ſüdliche Feuer. Damit wäre jede Frage über die
               Bühnenwirkſamkeit der Figur mit einem Schlage beſeitigt. Die \textsc{Triesch\pwindex{Triesch, Irene 13.04.1877 – 24.11.1964@\textsc{Triesch, Irene} (13.04.1877 – 24.11.1964), \emph{Schauspieler/Schauspielerin}|pw}} würde etwas Unerhörtes daraus machen. Wenn Du mir folgteſt, würdeſt Du alle
               Mittel aufbieten, um die Perſon für dieſe Rolle zu gewinnen. Aber leider folgſt Du
               mir ja niemals. In Berlin\oindex{Berlin@\textbf{Berlin}, \emph{P.PPLC}|pw} könnte meiner Anſicht
               nach nur {\pb}das \label{K_L02905-5v}\edtext{»Deutſche Theater\orgindex{Deutsches Theater Berlin@Deutsches Theater Berlin|pw}«}{\lemma{\textnormal{\emph{»Deutſche Theater«}}}\Cendnote{\textnormal{Zwei Jahre nach der Uraufführung in Breslau\oindex{Breslau@\textbf{Breslau}, \emph{P.PPLA}|pwk} (1. 12. 1900) fand am 7. 3. 1903 die
                  Premiere am \emph{Deutschen Theater Berlin}\orgindex{Deutsches Theater Berlin@Deutsches Theater Berlin|pwk} statt.
                     Otto Brahm\pwindex{Brahm, Otto 05.02.1856 – 28.11.1912@\textsc{Brahm, Otto} (05.02.1856 – 28.11.1912), \emph{Theaterleiter/Theaterleiterin, Regisseur/Regisseurin}|pwk} kannte das Stück\pwindex{Schleier der Beatrice. Schauspiel in fuenf Akten@\emph{Der Schleier der Beatrice. Schauspiel in fünf Akten}|pwkv} bereits seit 7. 10. 1899.}}}\label{K_L02905-5} in Betracht kommen. \textsc{Brahms\pwindex{Brahm, Otto 05.02.1856 – 28.11.1912@\textsc{Brahm, Otto} (05.02.1856 – 28.11.1912), \emph{Theaterleiter/Theaterleiterin, Regisseur/Regisseurin}|pw}}{ }\strikeout{iſt} zeigt ſich ſehr urtheilslos, wenn er nach dem Stück\pwindex{Schleier der Beatrice. Schauspiel in fuenf Akten@\emph{Der Schleier der Beatrice. Schauspiel in fünf Akten}|pwv} nicht mit beiden Händen
               greift. Wenn es in Wien\oindex{Wien@\textbf{Wien}, \emph{A.ADM2}|pw} Erfolg hat, wird er es
               übrigens ſchon thun. An das \label{K_L02905-6v}\edtext{Schauſpielhaus\orgindex{Schauspielhaus Berlin@Schauspielhaus Berlin|pw}}{\lemma{\textnormal{\emph{Schauſpielhaus}}}\Cendnote{\textnormal{Zu einer Inszenierung am \emph{Schauspielhaus Berlin}\orgindex{Schauspielhaus Berlin@Schauspielhaus Berlin|pwk} kam es nicht.}}}\label{K_L02905-6} iſt bei der
               jetzt herrſchenden Sittlichkeits-Manie nicht zu denken. Man würde Dein Drama\pwindex{Schleier der Beatrice. Schauspiel in fuenf Akten@\emph{Der Schleier der Beatrice. Schauspiel in fünf Akten}|pwv} entweder überhaupt nicht
               nehmen oder Dir zumuthen, die Hälfte wegzulaſſen. Im Nothfall könnte man es auch mit
               dem \label{K_L02905-7v}\edtext{»Berliner Theater\orgindex{Berliner Theater@Berliner Theater|pw}«}{\lemma{\textnormal{\emph{»Berliner Theater«}}}\Cendnote{\textnormal{Zu einer
                  Inszenierung von \emph{Der Schleier der Beatrice}\pwindex{Schleier der Beatrice. Schauspiel in fuenf Akten@\emph{Der Schleier der Beatrice. Schauspiel in fünf Akten}|pwk} am
                     \emph{Berliner Theater}\orgindex{Berliner Theater@Berliner Theater|pwk} kam es nicht.}}}\label{K_L02905-7}
               (Direktion \textsc{Paul Lindau\pwindex{Lindau, Paul 03.06.1839 – 31.01.1919@\textsc{Lindau, Paul} (03.06.1839 – 31.01.1919), \emph{Schriftsteller/Schriftstellerin, Kritiker/Kritikerin, Theaterleiter/Theaterleiterin}|pw}}) verſuchen, wo nicht ſchlecht geſpielt wird; nur die Ausſtattung würde hier
               armſeelig ſein.\pend
           
\pstart
           Deine \label{K_L02905-8v}\edtext{Aufträge}{\lemma{\textnormal{\emph{Aufträge}}}\Cendnote{\textnormal{Bezug unklar}}}\label{K_L02905-8} an \textsc{Gusti\pwindex{Gluemer, Auguste 1862-03-16 – 1956@\textsc{Glümer, Auguste} (1862-03-16 – 1956), \emph{Lehrer/Lehrerin}|pwv}} u. die \label{K_L02905-9v}\edtext{Frau
                  Rechtsanwalt\pwindex{Freudenthal, Rosa 1862 – 18.06.1905@\textsc{Freudenthal, Rosa} (1862 – 18.06.1905)|pwv}}{\lemma{\textnormal{\emph{Frau
                  Rechtsanwalt}}}\Cendnote{\textnormal{Schnitzlers ehemalige Geliebte Rosa Freudenthal\pwindex{Freudenthal, Rosa 1862 – 18.06.1905@\textsc{Freudenthal, Rosa} (1862 – 18.06.1905)|pwk} war mit dem Rechtsanwalt Hermann Freudenthal\pwindex{Freudenthal, Hermann 1852/1853 – 12.09.1925@\textsc{Freudenthal, Hermann} (1852/1853 – 12.09.1925), \emph{Rechtsanwalt/Rechtsanwältin}|pwk} verheiratet. Goldmann\pwindex{Goldmann, Paul 31.01.1865 – 25.09.1935@\textsc{Goldmann, Paul} (31.01.1865 – 25.09.1935), \emph{Schriftsteller/Schriftstellerin, Journalist/Journalistin}|pwk} hatte sich bereits 1897 mit einer ähnlichen Formulierung auf sie bezogen (vgl. Paul Goldmann an Arthur Schnitzler, 4. 9. 1897).}}}\label{K_L02905-9} werde ich
               beſorgen.\pend
           
\pstart
           Das Theaterreferat\orgindex{Neue Freie Presse@Neue Freie Presse|pwv} von hier aus
               hat ſeine Schwierigkeiten. Ich muß doch alle Deine \label{K_L02905-10v}\edtext{Geliebten}{\lemma{\textnormal{\emph{Geliebten}}}\Cendnote{\textnormal{Das
                  dürfte vor allem als Anspielung auf Marie
                     Glümer\pwindex{Gluemer, Marie 03.07.1867 – 16.11.1925@\textsc{Glümer, Marie} (03.07.1867 – 16.11.1925), \emph{Schauspieler/Schauspielerin}|pwk} zu lesen sein.}}}\label{K_L02905-10} loben. Um Irrthümer auszuſchließen, werde ich
               Dich demnächſt um einen Katalog bitten.\pend
           
\pstart
           {\pb}Von mir willſt Du hören? Siehſt Du, ich habe wenig \substVorne{}\textsuperscript{\textcolor{gray}{h}}\substDazwischen{}Z\substHinten{}eit zum Schreiben. Ich muß alſo wählen: ſoll ich Dir von Dir ſchreiben oder
               von mir? Und Du wirſt doch nicht leugnen, daß es Dich mehr intereſſirt, wenn ich Dir
               über Dein Stück\pwindex{Schleier der Beatrice. Schauspiel in fuenf Akten@\emph{Der Schleier der Beatrice. Schauspiel in fünf Akten}|pwv} ſchreibe, als
               über meine Schmerzen und \substVorne{}\textsuperscript{\textcolor{gray}{ſ}}\substDazwischen{}S\substHinten{}orgen. Oder vielmehr, Du wirſt es leugnen, aber ich werde Dir nicht
               glauben.\pend
           
\pstart
           Auf Umwegen höre ich, daß Dein \label{K_L02905-11v}\edtext{Bruder\pwindex{Schnitzler, Julius 13.07.1865 – 29.06.1939@\textsc{Schnitzler, Julius} (13.07.1865 – 29.06.1939), \emph{Chirurg/Chirurgin}|pwv} ein Mädchen\pwindex{Donath, Anna 1900-01-23 – 1995-12-27@\textsc{Donath, Anna} (1900-01-23 – 1995-12-27)|pwv} bekommen}{\lemma{\textnormal{\emph{Bruder … bekommen}}}\Cendnote{\textnormal{Anna\pwindex{Donath, Anna 1900-01-23 – 1995-12-27@\textsc{Donath, Anna} (1900-01-23 – 1995-12-27)|pwk}, das dritte Kind von Julius\pwindex{Schnitzler, Julius 13.07.1865 – 29.06.1939@\textsc{Schnitzler, Julius} (13.07.1865 – 29.06.1939), \emph{Chirurg/Chirurgin}|pwk} und Helene
                     Schnitzler\pwindex{Schnitzler, Helene 16.07.1871 – September 1941@\textsc{Schnitzler, Helene} (16.07.1871 – September 1941)|pwk}, war am 23. 1. 1900 geboren worden.}}}\label{K_L02905-11} hat. Bitte, übermittle den Eltern\pwindex{Schnitzler, Julius 13.07.1865 – 29.06.1939@\textsc{Schnitzler, Julius} (13.07.1865 – 29.06.1939), \emph{Chirurg/Chirurgin}|pwv}\pwindex{Schnitzler, Helene 16.07.1871 – September 1941@\textsc{Schnitzler, Helene} (16.07.1871 – September 1941)|pwv} meine {\pb}Glückwünſche zugleich mit meinen herzlichen Grüßen.
               Auch Deine übrigen Angehörigen bitte ich zu grüßen.\pend
           
\pstart
           Eine Wien\oindex{Wien@\textbf{Wien}, \emph{A.ADM2}|pw}er Jüdin, ein Frl. \textsc{Schreiber\pwindex{Schreiber, Adele 1872-04-29 – 1957-02-20@\textsc{Schreiber, Adele} (1872-04-29 – 1957-02-20), \emph{Schriftsteller/Schriftstellerin, Politiker/Politikerin, Pädagoge/Pädagogin}|pw}}, iſt mir mit einer Empfehlung von \textsc{Hanslick\pwindex{Hanslick, Eduard 11.09.1825 – 06.08.1904@\textsc{Hanslick, Eduard} (11.09.1825 – 06.08.1904), \emph{Musikkritiker/Musikkritikerin}|pw}} ins Haus gekommen. Sie will hier\oindex{Berlin@\textbf{Berlin}, \emph{P.PPLC}|pwv} einen \label{K_L02905-12v}\edtext{Vortrag\pwindex{Vortrag ueber Arthur Schnitzler]@\emph{[Vortrag über Arthur Schnitzler]}|pwv}}{\lemma{\textnormal{\emph{Vortrag}}}\Cendnote{\textnormal{Der Vortrag\pwindex{Vortrag ueber Arthur Schnitzler]@\emph{[Vortrag über Arthur Schnitzler]}|pwkv} von Adele
                     Schreiber\pwindex{Schreiber, Adele 1872-04-29 – 1957-02-20@\textsc{Schreiber, Adele} (1872-04-29 – 1957-02-20), \emph{Schriftsteller/Schriftstellerin, Politiker/Politikerin, Pädagoge/Pädagogin}|pwk}, veranstaltet von der \emph{Gesellschaft für Kunst und Wissenschaft}\orgindex{Lessing-Gesellschaft fuer Kunst und Wissenschaft@Lessing-Gesellschaft für Kunst und Wissenschaft|pwk} in Berlin\oindex{Berlin@\textbf{Berlin}, \emph{P.PPLC}|pwk}, fand am 28. 3. 1900 statt.}}}\label{K_L02905-12}
               über Dich halten (was ich bedaure, denn der Vortrag\pwindex{Vortrag ueber Arthur Schnitzler]@\emph{[Vortrag über Arthur Schnitzler]}|pwv} wird ſchlecht ſein) und hat mir inzwiſchen im
               Geſpräch werthvolle literariſche Aufſchlüſſe über Dich gegeben.\pend
           
\pstart
           Viele treue Grüße! {\\[\baselineskip]}Dein {\\[\baselineskip]}\spacefill\mbox{Paul Goldmann.}\pend
           \leftskip=0em{}
\pstart
           \noindent{}Ja, eine Bitte habe ich doch. Ich habe den Eindruck, daß ich in der N. Fr. Preſſe\orgindex{Neue Freie Presse@Neue Freie Presse|pw}, im Gegenſatz zur \label{K_L02905-13v}\edtext{Frankfurter {\pb}Zeitung\orgindex{Frankfurter Zeitung@Frankfurter Zeitung|pw}}{\lemma{\textnormal{\emph{Frankfurter Zeitung}}}\Cendnote{\textnormal{Goldmann\pwindex{Goldmann, Paul 31.01.1865 – 25.09.1935@\textsc{Goldmann, Paul} (31.01.1865 – 25.09.1935), \emph{Schriftsteller/Schriftstellerin, Journalist/Journalistin}|pwk} hatte bis Dezember 1899 für die
                     \emph{Frankfurter Zeitung}\orgindex{Frankfurter Zeitung@Frankfurter Zeitung|pwk}
                     gearbeitet.}}}\label{K_L02905-13}, vollſtändig verſchwinde. Merkt irgend Jemand, außer
                  Dir, daß ich vorhanden bin? Bitte, ſchreib’ mir ein Wort darüber!\pend
           \selectlanguage{ngerman}\endnumbering\briefempfaengerindex{Schnitzler, Arthur@\textsc{Schnitzler, Arthur}!zzzGoldmann, Paul@\emph{von Paul Goldmann}!1900-02-201@{20. 2. 1900}|)be}\mylabel{L02905h}  \normalsize

\doendnotes{C}
\bigskip
\vfill

\clearpage

\footnotesize

\lohead{\textsc{register}}

% Definiere theindex-Environment komplett neu ohne reledmac
\makeatletter
\renewenvironment{theindex}{%
  \section*{\indexname}%
  \setlength{\parindent}{0pt}%
  \setlength{\parskip}{0pt plus 0.3pt}%
  \let\item\@idxitem
}{%
  \clearpage
}
\makeatother

\IfFileExists{\jobname-pw.ind}{\input{\jobname-pw.ind}}{}

\end{document}

      