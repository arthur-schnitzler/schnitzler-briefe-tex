%% latex-leseansicht-vorspann.tex
%% Vorspann für die Leseansicht.
%% Lädt die gemeinsame Datei latex-vorspann.tex mit nicht gesetztem Schalter.

\newif\ifkorrekturansicht
\korrekturansichtfalse

\input{../tex-inputs/latex-vorspann}


         
         \renewcommand{\erwaehntePersonen}{Personen: Otto Brahm, Anna Donath, Rosa Freudenthal, Hermann Freudenthal, Auguste Glümer, Marie Glümer, Franz Grillparzer, Eduard Hanslick, Josef Kainz, Paul Lindau, Julius Schnitzler, Helene Schnitzler, Adele Schreiber, D. W. Schröder, Irene Triesch}
         \renewcommand{\erwaehnteInstitutionen}{Institutionen: Berliner Theater, Deutsches Theater Berlin, Frankfurter Zeitung, Lessing-Gesellschaft für Kunst und Wissenschaft, Neue Freie Presse, Schauspielhaus Berlin}
         \renewcommand{\erwaehnteOrte}{Orte: Berlin, Breslau, Burgtheater, Deutsches Theater Berlin, Frankfurt am Main, Hotel Saxonia, Potsdamer Platz, Stresemannstraße, Tiergarten, Wien}
         \renewcommand{\erwaehnteWerke}{Werke: Der Schleier der Beatrice. Schauspiel in fünf Akten, [Vortrag über Arthur Schnitzler]}
               \section[ Paul Goldmann an Arthur Schnitzler, 20. 2. 1900]{ Paul Goldmann an Arthur Schnitzler, 20. 2. 1900}\nopagebreak\mylabel{v}\rehead{ }\begin{ledgroupsized}[t]{13cm}\normalsize\beginnumbering \toendnotes[C]{\smallbreak\pagebreak[2]} \Standort{DLA, A:Schnitzler, HS.NZ85.1.3170.}
\physDesc{Brief, 2 Blätter, 7 Seiten, 4085 Zeichen
\newline{}Handschrift: schwarze Tinte, deutsche Kurrent
\newline{}Schnitzler: mit rotem Buntstift drei Unterstreichungen }\toendnotes[C]{\smallbreak}\pstart
           \noindent{}\centering{}{\pb}\textcolor{gray}{\textbf{\textbf{HOTEL SAXONIA\oindex{Hotel Saxonia@\textbf{Hotel Saxonia}|pw}}}}\pend
           \pstart
           \noindent{}\raggedleft{}\textcolor{gray}{\textbf{am Potsdamer Platz\oindex{Potsdamer Platz@\textbf{Potsdamer Platz}|pw} und
                        Thiergarten\oindex{Tiergarten@\textbf{Tiergarten}|pw}}}\pend
           \pstart
           \noindent{}\centering{}\textcolor{gray}{\textbf{D. W. SCHRÖDER\pwindex{Schroeder, D. W. @\textsc{Schröder, D. W.}, \emph{Hotelbesitzer/Hotelbesitzerin}|pw}.}}\pend
           \pstart
           \noindent{}\textcolor{gray}{\textbf{Fernsprecher:}}\pend
           \pstart
           \textcolor{gray}{\textbf{\textbf{Amt VI. No. 2838.}}}\pend
           \pstart
           \raggedleft{}\textcolor{gray}{\textbf{\emph{BERLIN W.}\oindex{Berlin@\textbf{Berlin}|pw}, den}}{ }20. Februar \textcolor{gray}{\textbf{1}}900. \pend
           \pstart
           \raggedleft{}\textcolor{gray}{\textbf{Königgrätzerstrasse 10\oindex{Stresemannstrasse@\textbf{Stresemannstraße}|pw}.}}\pend
           \pstart{}Mein lieber Freund,\pend\pstart
           Ich will gleich auf Deinen lieben Brief antworten, ſonſt komme ich lange nicht
               dazu.\pend
           \pstart
           Es freut mich ſehr, daß Du mit meiner \label{K_L02905-1v}\edtext{Anſicht}{\lemma{\textnormal{\emph{Anſicht}}}\Cendnote{\textnormal{siehe Paul Goldmann an Arthur Schnitzler, 11. 2. 1900}}}\label{K_L02905-1h} über dein Stück\pwindex{Schnitzler, Arthur 15.05.1862 – 21.10.1931@\textsc{Schnitzler, Arthur} (15.05.1862 – 21.10.1931), \emph{Schriftsteller, Mediziner}!Schleier der Beatrice. Schauspiel in fuenf Akten1900-12-01@\strich\emph{Der Schleier der Beatrice. Schauspiel in fünf Akten} {[}1900-12-01{]}|pwv} zum
               Theil einverſtanden biſt. Ich habe noch einmal Dieſes und Jenes geleſen\strikeout{,} und kann Dir nur ſagen: Seit \textsc{Grillparzer\pwindex{Grillparzer, Franz 15.01.1791 – 21.01.1872@\textsc{Grillparzer, Franz} (15.01.1791 – 21.01.1872), \emph{Schriftsteller, Beamter}|pw}} hat man auf dem Wien\oindex{Wien@\textbf{Wien}|pw}er Theater ſolche Verſe\pwindex{Schnitzler, Arthur 15.05.1862 – 21.10.1931@\textsc{Schnitzler, Arthur} (15.05.1862 – 21.10.1931), \emph{Schriftsteller, Mediziner}!Schleier der Beatrice. Schauspiel in fuenf Akten1900-12-01@\strich\emph{Der Schleier der Beatrice. Schauspiel in fünf Akten} {[}1900-12-01{]}|pwv} nicht gehört. Das ſoll
               aber nicht bedeuten, daß es \textsc{Grillparzer\pwindex{Grillparzer, Franz 15.01.1791 – 21.01.1872@\textsc{Grillparzer, Franz} (15.01.1791 – 21.01.1872), \emph{Schriftsteller, Beamter}|pw}ische} Verſe ſind. Nein, ſie
               ſind durchaus \textsc{Schnitzlerisch}, und nur der weiche Wien\oindex{Wien@\textbf{Wien}|pw}er Wohllaut iſt den beiden Dichtern\pwindex{Grillparzer, Franz 15.01.1791 – 21.01.1872@\textsc{Grillparzer, Franz} (15.01.1791 – 21.01.1872), \emph{Schriftsteller, Beamter}|pwv} gemeinſam. Was die Aufführung
               anlangt, ſo {\pb}möchte ich Streichungen empfehlen.
               Vielleicht auch einige \label{K_L02905-2v}\edtext{Umarbeitungen}{\lemma{\textnormal{\emph{Umarbeitungen}}}\Cendnote{\textnormal{keine entsprechenden
                  Umarbeitungen bekannt}}}\label{K_L02905-2h}. Ich bleibe dabei: die Geſtalt des Herzog\pwindex{Schnitzler, Arthur 15.05.1862 – 21.10.1931@\textsc{Schnitzler, Arthur} (15.05.1862 – 21.10.1931), \emph{Schriftsteller, Mediziner}!Schleier der Beatrice. Schauspiel in fuenf Akten1900-12-01@\strich\emph{Der Schleier der Beatrice. Schauspiel in fünf Akten} {[}1900-12-01{]}|pwv}s erſcheint mir in zu unklaren
               Umriſſen. Wenn da auch nur ein wenig mit feſter Hand nachgezeichnet würde, könnte das
               dem Drama\pwindex{Schnitzler, Arthur 15.05.1862 – 21.10.1931@\textsc{Schnitzler, Arthur} (15.05.1862 – 21.10.1931), \emph{Schriftsteller, Mediziner}!Schleier der Beatrice. Schauspiel in fuenf Akten1900-12-01@\strich\emph{Der Schleier der Beatrice. Schauspiel in fünf Akten} {[}1900-12-01{]}|pwv} ſehr zum Vortheil
               gereichen. Wäre es nicht doch möglich, daß die Hochzeit nur ein im Voraus
               beabſichtigter Carnevals-Scherz ſein könnte? Wenn der Herzog\pwindex{Schnitzler, Arthur 15.05.1862 – 21.10.1931@\textsc{Schnitzler, Arthur} (15.05.1862 – 21.10.1931), \emph{Schriftsteller, Mediziner}!Schleier der Beatrice. Schauspiel in fuenf Akten1900-12-01@\strich\emph{Der Schleier der Beatrice. Schauspiel in fünf Akten} {[}1900-12-01{]}|pwv} durchaus edel ſein muß, ſo könnte
               der Edelmuth ja nachher erwachen. Mich hat übrigens in Deinem Briefe das Wort »Größe«
               ſtutzig gemacht. Warum ſoll der Herzog\pwindex{Schnitzler, Arthur 15.05.1862 – 21.10.1931@\textsc{Schnitzler, Arthur} (15.05.1862 – 21.10.1931), \emph{Schriftsteller, Mediziner}!Schleier der Beatrice. Schauspiel in fuenf Akten1900-12-01@\strich\emph{Der Schleier der Beatrice. Schauspiel in fünf Akten} {[}1900-12-01{]}|pwv} »groß« ſein? Mir ſcheint, dieſes Streben nach Größe, dieſe abſtrakt
               hinzugedachte Eigenſchaft, iſt an der Unklarheit ſchuld. Hätteſt Du ihn nur (wie es
               ſonſt Deine Gewohnheit iſt) ruhig und \substVorne{}\textsuperscript{natürlich}{\allowbreak}\substDazwischen{}natürlich\substHinten{} leben laſſen, wie er leben mochte, ſo wäre \strikeout{\textcolor{gray}{er}} er deutlicher und wahrer geworden. Im Übrigen, vielleicht haſt Du Recht, und
                  {\pb}auf der Bühne zeigt ſich vielleicht, daß die Figur\pwindex{Schnitzler, Arthur 15.05.1862 – 21.10.1931@\textsc{Schnitzler, Arthur} (15.05.1862 – 21.10.1931), \emph{Schriftsteller, Mediziner}!Schleier der Beatrice. Schauspiel in fuenf Akten1900-12-01@\strich\emph{Der Schleier der Beatrice. Schauspiel in fünf Akten} {[}1900-12-01{]}|pwv} richtig gedacht war.\pend
           \pstart
           Welche Rolle \label{K_L02905-3v}\edtext{\textsc{Kainz\pwindex{Kainz, Josef 02.01.1858 – 20.09.1910@\textsc{Kainz, Josef} (02.01.1858 – 20.09.1910), \emph{Schauspieler}|pw}}}{\lemma{\textnormal{\emph{Kainz}}}\Cendnote{\textnormal{Josef Kainz\pwindex{Kainz, Josef 02.01.1858 – 20.09.1910@\textsc{Kainz, Josef} (02.01.1858 – 20.09.1910), \emph{Schauspieler}|pwk} war ein von Schnitzler\pwindex{Schnitzler, Arthur 15.05.1862 – 21.10.1931@\textsc{Schnitzler, Arthur} (15.05.1862 – 21.10.1931), \emph{Schriftsteller, Mediziner}|pwk} vielgeschätzter Schauspieler und war mehrmals an
                  Inszenierungen seiner Dramen beteiligt. Für die geplante Uraufführung des \emph{Schleiers der Beatrice}\pwindex{Schnitzler, Arthur 15.05.1862 – 21.10.1931@\textsc{Schnitzler, Arthur} (15.05.1862 – 21.10.1931), \emph{Schriftsteller, Mediziner}!Schleier der Beatrice. Schauspiel in fuenf Akten1900-12-01@\strich\emph{Der Schleier der Beatrice. Schauspiel in fünf Akten} {[}1900-12-01{]}|pwk} im Burgtheater\oindex{Burgtheater@\textbf{Burgtheater}|pwk} wollte Schnitzler\pwindex{Schnitzler, Arthur 15.05.1862 – 21.10.1931@\textsc{Schnitzler, Arthur} (15.05.1862 – 21.10.1931), \emph{Schriftsteller, Mediziner}|pwk}{ }Kainz\pwindex{Kainz, Josef 02.01.1858 – 20.09.1910@\textsc{Kainz, Josef} (02.01.1858 – 20.09.1910), \emph{Schauspieler}|pwk} in der Rolle des Filippo\pwindex{Schnitzler, Arthur 15.05.1862 – 21.10.1931@\textsc{Schnitzler, Arthur} (15.05.1862 – 21.10.1931), \emph{Schriftsteller, Mediziner}!Schleier der Beatrice. Schauspiel in fuenf Akten1900-12-01@\strich\emph{Der Schleier der Beatrice. Schauspiel in fünf Akten} {[}1900-12-01{]}|pwkv} sehen (vgl. Arthur Schnitzler an Richard Beer-Hofmann, 17. 2. 1900). Zu dieser Aufführung
                  kam es aber nicht (vgl. Paul Goldmann an Arthur Schnitzler, 12. 11. [1899]).}}}\label{K_L02905-3h} ſpielen ſoll, kann ich Dir nicht ſagen. Denn ich kenne \textsc{Kainz\pwindex{Kainz, Josef 02.01.1858 – 20.09.1910@\textsc{Kainz, Josef} (02.01.1858 – 20.09.1910), \emph{Schauspieler}|pw}} nicht. Der Herzog\pwindex{Schnitzler, Arthur 15.05.1862 – 21.10.1931@\textsc{Schnitzler, Arthur} (15.05.1862 – 21.10.1931), \emph{Schriftsteller, Mediziner}!Schleier der Beatrice. Schauspiel in fuenf Akten1900-12-01@\strich\emph{Der Schleier der Beatrice. Schauspiel in fünf Akten} {[}1900-12-01{]}|pwv} muß
               jedenfalls ein vollendeter \uline{Sprecher} ſein, und mir
               ſcheint, daß \textsc{Kainz\pwindex{Kainz, Josef 02.01.1858 – 20.09.1910@\textsc{Kainz, Josef} (02.01.1858 – 20.09.1910), \emph{Schauspieler}|pw}} das nicht iſt. Für die \textsc{Beatrice\pwindex{Schnitzler, Arthur 15.05.1862 – 21.10.1931@\textsc{Schnitzler, Arthur} (15.05.1862 – 21.10.1931), \emph{Schriftsteller, Mediziner}!Schleier der Beatrice. Schauspiel in fuenf Akten1900-12-01@\strich\emph{Der Schleier der Beatrice. Schauspiel in fünf Akten} {[}1900-12-01{]}|pwv}} aber gibt es meiner Anſicht nach nur \uline{eine} auf
               den deutſchen Theatern: Die \label{K_L02905-4v}\edtext{\textsc{Triesch\pwindex{Triesch, Irene 13.04.1877 – 24.11.1964@\textsc{Triesch, Irene} (13.04.1877 – 24.11.1964), \emph{Schauspielerin}|pw}}}{\lemma{\textnormal{\emph{Triesch}}}\Cendnote{\textnormal{Irene Triesch\pwindex{Triesch, Irene 13.04.1877 – 24.11.1964@\textsc{Triesch, Irene} (13.04.1877 – 24.11.1964), \emph{Schauspielerin}|pwk} gestaltete erst 1903 die Beatrice\pwindex{Schnitzler, Arthur 15.05.1862 – 21.10.1931@\textsc{Schnitzler, Arthur} (15.05.1862 – 21.10.1931), \emph{Schriftsteller, Mediziner}!Schleier der Beatrice. Schauspiel in fuenf Akten1900-12-01@\strich\emph{Der Schleier der Beatrice. Schauspiel in fünf Akten} {[}1900-12-01{]}|pwkv} am Deutschen Theater Berlin\oindex{Deutsches Theater Berlin@\textbf{Deutsches Theater Berlin}|pwk}
                  aus. Schnitzler\pwindex{Schnitzler, Arthur 15.05.1862 – 21.10.1931@\textsc{Schnitzler, Arthur} (15.05.1862 – 21.10.1931), \emph{Schriftsteller, Mediziner}|pwk} missfiel sie darin jedoch
                     (vgl. A. S.: \emph{Tagebuch}, 23. 2. 1903).}}}\label{K_L02905-4h} in
                  Frankfurt\oindex{Frankfurt am Main@\textbf{Frankfurt am Main}|pw}. Sie hat geniale Kunſt-Inſtinkte,
               iſt ſelbſt ein ſo unberechenbares Luder, wie Deine \textsc{Beatrice\pwindex{Schnitzler, Arthur 15.05.1862 – 21.10.1931@\textsc{Schnitzler, Arthur} (15.05.1862 – 21.10.1931), \emph{Schriftsteller, Mediziner}!Schleier der Beatrice. Schauspiel in fuenf Akten1900-12-01@\strich\emph{Der Schleier der Beatrice. Schauspiel in fünf Akten} {[}1900-12-01{]}|pwv}}, hat außerdem die Jugend und das ſüdliche Feuer. Damit wäre jede Frage über die
               Bühnenwirkſamkeit der Figur mit einem Schlage beſeitigt. Die \textsc{Triesch\pwindex{Triesch, Irene 13.04.1877 – 24.11.1964@\textsc{Triesch, Irene} (13.04.1877 – 24.11.1964), \emph{Schauspielerin}|pw}} würde etwas Unerhörtes daraus machen. Wenn Du mir folgteſt, würdeſt Du alle
               Mittel aufbieten, um die Perſon für dieſe Rolle zu gewinnen. Aber leider folgſt Du
               mir ja niemals. In Berlin\oindex{Berlin@\textbf{Berlin}|pw} könnte meiner Anſicht
               nach nur {\pb}das \label{K_L02905-5v}\edtext{»Deutſche Theater\orgindex{Deutsches Theater Berlin@Deutsches Theater Berlin|pw}«}{\lemma{\textnormal{\emph{»Deutſche Theater«}}}\Cendnote{\textnormal{Zwei Jahre nach der Uraufführung in Breslau\oindex{Breslau@\textbf{Breslau}|pwk} (1. 12. 1900) fand am 7. 3. 1903 die
                  Premiere am \emph{Deutschen Theater Berlin}\orgindex{Deutsches Theater Berlin@Deutsches Theater Berlin|pwk} statt.
                     Otto Brahm\pwindex{Brahm, Otto 05.02.1856 – 28.11.1912@\textsc{Brahm, Otto} (05.02.1856 – 28.11.1912), \emph{Theaterleiter, Regisseur}|pwk} hatte das Stück\pwindex{Schnitzler, Arthur 15.05.1862 – 21.10.1931@\textsc{Schnitzler, Arthur} (15.05.1862 – 21.10.1931), \emph{Schriftsteller, Mediziner}!Schleier der Beatrice. Schauspiel in fuenf Akten1900-12-01@\strich\emph{Der Schleier der Beatrice. Schauspiel in fünf Akten} {[}1900-12-01{]}|pwkv} bereits seit 7. 10. 1899
                  gekannt.}}}\label{K_L02905-5h} in Betracht kommen. \textsc{Brahms\pwindex{Brahm, Otto 05.02.1856 – 28.11.1912@\textsc{Brahm, Otto} (05.02.1856 – 28.11.1912), \emph{Theaterleiter, Regisseur}|pw}}{ }\strikeout{iſt} zeigt ſich ſehr urtheilslos, wenn er nach dem Stück\pwindex{Schnitzler, Arthur 15.05.1862 – 21.10.1931@\textsc{Schnitzler, Arthur} (15.05.1862 – 21.10.1931), \emph{Schriftsteller, Mediziner}!Schleier der Beatrice. Schauspiel in fuenf Akten1900-12-01@\strich\emph{Der Schleier der Beatrice. Schauspiel in fünf Akten} {[}1900-12-01{]}|pwv} nicht mit beiden Händen
               greift. Wenn es in Wien\oindex{Wien@\textbf{Wien}|pw} Erfolg hat, wird er es
               übrigens ſchon thun. An das \label{K_L02905-6v}\edtext{Schauſpielhaus\orgindex{Schauspielhaus Berlin@Schauspielhaus Berlin|pw}}{\lemma{\textnormal{\emph{Schauſpielhaus}}}\Cendnote{\textnormal{Zu einer Inszenierung am \emph{Schauspielhaus Berlin}\orgindex{Schauspielhaus Berlin@Schauspielhaus Berlin|pwk} kam es nicht.}}}\label{K_L02905-6h} iſt bei der
               jetzt herrſchenden Sittlichkeits-Manie nicht zu denken. Man würde Dein Drama\pwindex{Schnitzler, Arthur 15.05.1862 – 21.10.1931@\textsc{Schnitzler, Arthur} (15.05.1862 – 21.10.1931), \emph{Schriftsteller, Mediziner}!Schleier der Beatrice. Schauspiel in fuenf Akten1900-12-01@\strich\emph{Der Schleier der Beatrice. Schauspiel in fünf Akten} {[}1900-12-01{]}|pwv} entweder überhaupt nicht
               nehmen oder Dir zumuthen, die Hälfte wegzulaſſen. Im Nothfall könnte man es auch mit
               dem \label{K_L02905-7v}\edtext{»Berliner Theater\orgindex{Berliner Theater@Berliner Theater|pw}«}{\lemma{\textnormal{\emph{»Berliner Theater«}}}\Cendnote{\textnormal{Zu einer
                  Inszenierung von \emph{Der Schleier der Beatrice}\pwindex{Schnitzler, Arthur 15.05.1862 – 21.10.1931@\textsc{Schnitzler, Arthur} (15.05.1862 – 21.10.1931), \emph{Schriftsteller, Mediziner}!Schleier der Beatrice. Schauspiel in fuenf Akten1900-12-01@\strich\emph{Der Schleier der Beatrice. Schauspiel in fünf Akten} {[}1900-12-01{]}|pwk} am
                     \emph{Berliner Theater}\orgindex{Berliner Theater@Berliner Theater|pwk} kam es nicht.}}}\label{K_L02905-7h}
               (Direktion \textsc{Paul Lindau\pwindex{Lindau, Paul 03.06.1839 – 31.01.1919@\textsc{Lindau, Paul} (03.06.1839 – 31.01.1919), \emph{Schriftsteller, Kritiker, Theaterleiter}|pw}}) verſuchen, wo nicht ſchlecht geſpielt wird; nur die Ausſtattung würde hier
               armſeelig ſein.\pend
           \pstart
           Deine \label{K_L02905-8v}\edtext{Aufträge}{\lemma{\textnormal{\emph{Aufträge}}}\Cendnote{\textnormal{Bezug unklar}}}\label{K_L02905-8h} an \textsc{Gusti\pwindex{Gluemer, Auguste 16.03.1862 – 1956@\textsc{Glümer, Auguste} (16.03.1862 – 1956)|pwv}} u. die \label{K_L02905-9v}\edtext{Frau
                  Rechtsanwalt\pwindex{Freudenthal, Rosa 1862 – 18.06.1905@\textsc{Freudenthal, Rosa} (1862 – 18.06.1905)|pwv}}{\lemma{\textnormal{\emph{Frau
                  Rechtsanwalt}}}\Cendnote{\textnormal{höchstwahrscheinlich Schnitzler\pwindex{Schnitzler, Arthur 15.05.1862 – 21.10.1931@\textsc{Schnitzler, Arthur} (15.05.1862 – 21.10.1931), \emph{Schriftsteller, Mediziner}|pwk}s ehemalige Geliebte Rosa Freudenthal\pwindex{Freudenthal, Rosa 1862 – 18.06.1905@\textsc{Freudenthal, Rosa} (1862 – 18.06.1905)|pwk}, die mit dem Rechtsanwalt Hermann Freudenthal\pwindex{Freudenthal, Hermann 1852/1853 – 12.09.1925@\textsc{Freudenthal, Hermann} (1852/1853 – 12.09.1925), \emph{Rechtsanwalt}|pwk} verheiratet war; Goldmann\pwindex{Goldmann, Paul 31.01.1865 – 25.09.1935@\textsc{Goldmann, Paul} (31.01.1865 – 25.09.1935), \emph{Schriftsteller, Journalist}|pwk} bezog sich bereits 1897 mit einer ähnlichen Formulierung auf sie (vgl. Paul Goldmann an Arthur Schnitzler, 4. 9. 1897)}}}\label{K_L02905-9h} werde ich
               beſorgen.\pend
           \pstart
           Das Theaterreferat\orgindex{Neue Freie Presse@Neue Freie Presse|pwv} von hier aus
               hat ſeine Schwierigkeiten. Ich muß doch alle Deine \label{K_L02905-10v}\edtext{Geliebten}{\lemma{\textnormal{\emph{Geliebten}}}\Cendnote{\textnormal{Das
                  dürfte vor allem als Anspielung auf Marie
                     Glümer\pwindex{Gluemer, Marie 03.07.1867 – 16.11.1925@\textsc{Glümer, Marie} (03.07.1867 – 16.11.1925), \emph{Schauspielerin}|pwk} zu lesen sein.}}}\label{K_L02905-10h} loben. Um Irrthümer auszuſchließen, werde ich
               Dich demnächſt um einen Katalog bitten.\pend
           \pstart
           {\pb}Von mir willſt Du hören? Siehſt Du, ich habe wenig \substVorne{}\textsuperscript{\textcolor{gray}{h}}\substDazwischen{}Z\substHinten{}eit zum Schreiben. Ich muß alſo wählen: ſoll ich Dir von Dir ſchreiben oder
               von mir? Und Du wirſt doch nicht leugnen, daß es Dich mehr intereſſirt, wenn ich Dir
               über Dein Stück\pwindex{Schnitzler, Arthur 15.05.1862 – 21.10.1931@\textsc{Schnitzler, Arthur} (15.05.1862 – 21.10.1931), \emph{Schriftsteller, Mediziner}!Schleier der Beatrice. Schauspiel in fuenf Akten1900-12-01@\strich\emph{Der Schleier der Beatrice. Schauspiel in fünf Akten} {[}1900-12-01{]}|pwv} ſchreibe, als
               über meine Schmerzen und \substVorne{}\textsuperscript{\textcolor{gray}{ſ}}\substDazwischen{}S\substHinten{}orgen. Oder vielmehr, Du wirſt es leugnen, aber ich werde Dir nicht
               glauben.\pend
           \pstart
           Auf Umwegen höre ich, daß Dein \label{K_L02905-11v}\edtext{Bruder\pwindex{Schnitzler, Julius 13.07.1865 – 29.06.1939@\textsc{Schnitzler, Julius} (13.07.1865 – 29.06.1939), \emph{Mediziner}|pwv} ein Mädchen\pwindex{Donath, Anna 1900-01-23 – 1995-12-27@\textsc{Donath, Anna} (1900-01-23 – 1995-12-27)|pwv} bekommen}{\lemma{\textnormal{\emph{Bruder … bekommen}}}\Cendnote{\textnormal{Anna\pwindex{Donath, Anna 1900-01-23 – 1995-12-27@\textsc{Donath, Anna} (1900-01-23 – 1995-12-27)|pwk}, das dritte Kind von Julius\pwindex{Schnitzler, Julius 13.07.1865 – 29.06.1939@\textsc{Schnitzler, Julius} (13.07.1865 – 29.06.1939), \emph{Mediziner}|pwk} und Helene
                     Schnitzler\pwindex{Schnitzler, Helene 16.07.1871 – September 1941@\textsc{Schnitzler, Helene} (16.07.1871 – September 1941)|pwk}, war am 23. 1. 1900 geboren worden.}}}\label{K_L02905-11h} hat. Bitte, übermittle den Eltern\pwindex{Schnitzler, Julius 13.07.1865 – 29.06.1939@\textsc{Schnitzler, Julius} (13.07.1865 – 29.06.1939), \emph{Mediziner}|pwv}\pwindex{Schnitzler, Helene 16.07.1871 – September 1941@\textsc{Schnitzler, Helene} (16.07.1871 – September 1941)|pwv} meine {\pb}Glückwünſche zugleich mit meinen herzlichen Grüßen.
               Auch Deine übrigen Angehörigen bitte ich zu grüßen.\pend
           \pstart
           Eine Wien\oindex{Wien@\textbf{Wien}|pw}er Jüdin, ein Frl. \textsc{Schreiber\pwindex{Schreiber, Adele 1872-04-29 – 1957-02-20@\textsc{Schreiber, Adele} (1872-04-29 – 1957-02-20), \emph{Schriftstellerin, Politikerin, Pädagogin}|pw}}, iſt mir mit einer Empfehlung von \textsc{Hanslick\pwindex{Hanslick, Eduard 11.09.1825 – 06.08.1904@\textsc{Hanslick, Eduard} (11.09.1825 – 06.08.1904), \emph{Kritiker}|pw}} ins Haus gekommen. Sie will hier\oindex{Berlin@\textbf{Berlin}|pwv} einen \label{K_L02905-12v}\edtext{Vortrag\pwindex{Schreiber, Adele 1872-04-29 – 1957-02-20@\textsc{Schreiber, Adele} (1872-04-29 – 1957-02-20), \emph{Schriftstellerin, Politikerin, Pädagogin}!Vortrag ueber Arthur Schnitzler]1900-03-28@\strich\emph{[Vortrag über Arthur Schnitzler]} {[}1900-03-28{]}|pwv}}{\lemma{\textnormal{\emph{Vortrag}}}\Cendnote{\textnormal{Der Vortrag\pwindex{Schreiber, Adele 1872-04-29 – 1957-02-20@\textsc{Schreiber, Adele} (1872-04-29 – 1957-02-20), \emph{Schriftstellerin, Politikerin, Pädagogin}!Vortrag ueber Arthur Schnitzler]1900-03-28@\strich\emph{[Vortrag über Arthur Schnitzler]} {[}1900-03-28{]}|pwkv} von Adele
                     Schreiber\pwindex{Schreiber, Adele 1872-04-29 – 1957-02-20@\textsc{Schreiber, Adele} (1872-04-29 – 1957-02-20), \emph{Schriftstellerin, Politikerin, Pädagogin}|pwk}, veranstaltet von der \emph{Gesellschaft für Kunst und Wissenschaft}\orgindex{Lessing-Gesellschaft fuer Kunst und Wissenschaft@Lessing-Gesellschaft für Kunst und Wissenschaft|pwk} in Berlin\oindex{Berlin@\textbf{Berlin}|pwk}, fand am 28. 3. 1900 statt.}}}\label{K_L02905-12h}
               über Dich halten (was ich bedaure, denn der Vortrag\pwindex{Schreiber, Adele 1872-04-29 – 1957-02-20@\textsc{Schreiber, Adele} (1872-04-29 – 1957-02-20), \emph{Schriftstellerin, Politikerin, Pädagogin}!Vortrag ueber Arthur Schnitzler]1900-03-28@\strich\emph{[Vortrag über Arthur Schnitzler]} {[}1900-03-28{]}|pwv} wird ſchlecht ſein) und hat mir inzwiſchen im
               Geſpräch werthvolle literariſche Aufſchlüſſe über Dich gegeben.\pend
           \pstart
           Viele treue Grüße! {\\[\baselineskip]}Dein {\\[\baselineskip]}\spacefill\mbox{Paul Goldmann.}\pend
           \leftskip=0em{}\pstart
           \noindent{}Ja, eine Bitte habe ich doch. Ich habe den Eindruck, daß ich in der N. Fr. Preſſe\orgindex{Neue Freie Presse@Neue Freie Presse|pw}, im Gegenſatz zur \label{K_L02905-13v}\edtext{Frankfurter {\pb}Zeitung\orgindex{Frankfurter Zeitung@Frankfurter Zeitung|pw}}{\lemma{\textnormal{\emph{Frankfurter Zeitung}}}\Cendnote{\textnormal{für die Goldmann\pwindex{Goldmann, Paul 31.01.1865 – 25.09.1935@\textsc{Goldmann, Paul} (31.01.1865 – 25.09.1935), \emph{Schriftsteller, Journalist}|pwk} bis Dezember 1899
                     gearbeitet hatte}}}\label{K_L02905-13h}, vollſtändig verſchwinde. Merkt irgend Jemand, außer
                  Dir, daß ich vorhanden bin? Bitte, ſchreib’ mir ein Wort darüber!\pend
           
         
         \endnumbering\mylabel{h}\end{ledgroupsized}  \newcommand{\dateiname}{L02905}\newcommand{\titel}{Paul Goldmann an Arthur Schnitzler, 20. 2. 1900}\newcommand{\editorInnen}{Martin Anton Müller und Laura Untner}%% latex-leseansicht-abspann.tex
%% Abspann für die Leseansicht.
%% Der Schalter \ifkorrekturansicht ist bereits durch den Vorspann gesetzt.

%% latex-abspann.tex
%% Gemeinsamer Abspann für Korrekturansicht und Leseansicht.
%% Setzt den Schalter \ifkorrekturansicht voraus (gesetzt in den
%% einbindenden Dateien latex-korrekturansicht-abspann.tex bzw.
%% latex-leseansicht-abspann.tex).
%% ---------------------------------------------------------------

\normalsize

% Das esempio-Environment wird nur in der Leseansicht benötigt
\ifkorrekturansicht\else
\newenvironment{esempio}[3]%
{
    \vspace{1.5ex}
    \rlap{\underline{#1}}
    \par
    \setlength{\parindent}{0cm}
    \nopagebreak
    \leftskip=#2cm
    \rightskip=#3cm
}
{
    \par
}
\fi

\doendnotes{C}
\bigskip
\vfill

\clearpage

\footnotesize

\ifkorrekturansicht
  \lohead{\textsc{register}}
\fi

% theindex-Environment neu definieren ohne reledmac
\makeatletter
\renewenvironment{theindex}{%
  \ifkorrekturansicht
    \section*{\indexname}%
  \else
    \subsubsection*{Index der erwähnten Entitäten}%
  \fi
  \setlength{\parindent}{0pt}%
  \setlength{\parskip}{0pt plus 0.3pt}%
  \let\item\@idxitem
}{%
  \ifkorrekturansicht\clearpage\fi
}
\makeatother

\IfFileExists{\jobname-pw.ind}{\input{\jobname-pw.ind}}{}

% Quellenangabe nur in der Leseansicht
\ifkorrekturansicht\else
% Fallback-Definitionen, falls die .tex-Datei \titel etc. nicht gesetzt hat
\providecommand{\titel}{}
\providecommand{\editorInnen}{}
\providecommand{\dateiname}{\jobname}

\vspace{3cm}

\vfill

\footnotesize
\textsc{Quelle}: \titel. Herausgegeben von {\editorInnen}. In: \emph{Arthur Schnitzler: Briefwechsel mit Autorinnen und Autoren}.
 Digitale Edition, https://schnitzler-briefe.acdh.oeaw.ac.at/{\dateiname}.html (Stand \today)
\fi

\end{document}


      