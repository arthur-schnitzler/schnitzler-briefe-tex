%% latex-leseansicht-vorspann.tex
%% Vorspann für die Leseansicht.
%% Lädt die gemeinsame Datei latex-vorspann.tex mit nicht gesetztem Schalter.

\newif\ifkorrekturansicht
\korrekturansichtfalse

\input{../tex-inputs/latex-vorspann}


\section[ Paul Goldmann an Arthur Schnitzler, 20. 2. 1900]{L02905 Paul Goldmann an Arthur Schnitzler,  20. 2. 1900}
\nopagebreak\mylabel{L02905v}
\rehead{ }\normalsize\beginnumbering\briefempfaengerindex{Schnitzler, Arthur@\textsc{Schnitzler, Arthur}!zzzGoldmann, Paul@\emph{von Paul Goldmann}!1900-02-201@{20. 2. 1900}|(be}
\toendnotes[C]{\smallbreak\pagebreak[2]}
\correspDesc{Versand  durch Paul Goldmann am 20. 2. 1900 in Berlin
\newline{}Erhalt  durch Arthur Schnitzler im Zeitraum [21. 2. 1900
                  – 25. 2. 1900?] in Wien}\toendnotes[C]{\smallbreak}
\Standort{DLA, A:Schnitzler, HS.NZ85.1.3170.}
\physDesc{Brief, 2 Blätter, 7 Seiten, 4085 Zeichen
\newline{}Handschrift: schwarze Tinte, deutsche Kurrent
\newline{}Schnitzler: mit rotem Buntstift drei Unterstreichungen }\toendnotes[C]{\smallbreak}
\pstart
           \centering{}{\pb}\textcolor{gray}{\textbf{\textbf{HOTEL SAXONIA\oindex{Hotel Saxonia@\textbf{Hotel Saxonia}, \emph{Hotel}|pw}}}}\pend
           
\pstart
           \raggedleft{}\textcolor{gray}{\textbf{am Potsdamer Platz\oindex{Potsdamer Platz@\textbf{Potsdamer Platz}, \emph{Platz}|pw} und
                        Thiergarten\oindex{Tiergarten@\textbf{Tiergarten}, \emph{Ehemaliger Ort}|pw}}}\pend
           
\pstart
           \centering{}\textcolor{gray}{\textbf{D. W. SCHRÖDER\pwindex{Schröder, D. W. @\textsc{Schröder, D. W.}, \emph{Hotelbesitzer/Hotelbesitzerin}|pw}.}}\pend
           
\pstart
           \textcolor{gray}{\textbf{Fernsprecher:}}\pend
           
\pstart
           \textcolor{gray}{\textbf{\textbf{Amt VI. No. 2838.}}}\pend
           
\pstart
           \raggedleft{}\textcolor{gray}{\textbf{\emph{BERLIN W.}\oindex{Berlin@\textbf{Berlin}, \emph{Hauptstadt}|pw}, den}}{ }20. Februar \textcolor{gray}{\textbf{1}}900.\pend
           
\pstart
           \raggedleft{}\textcolor{gray}{\textbf{Königgrätzerstrasse 10\oindex{Stresemannstraße@\textbf{Stresemannstraße}, \emph{Straße}|pw}.}}\pend
           
\pstart{}Mein lieber Freund,\pend\vspace{0.5em}
\pstart
           Ich will gleich auf Deinen lieben Brief antworten,{ }ſonſt komme ich lange nicht
               dazu.\pend
           
\pstart
           Es freut mich{ }ſehr, daß Du mit meiner \label{K_L02905-1v}\edtext{Anſicht}{\lemma{\textnormal{\emph{Ansicht}}}\Cendnote{\textnormal{Siehe XXXX Auszeichnungsfehler: Dokument L02904 nicht gefunden.
               }}}\label{K_L02905-1} über dein Stück\pwindex{Schnitzler, Arthur 15.\,5.\,1862 Wien – 21.\,10.\,1931 ebd.@\textsc{Schnitzler, Arthur} (15.\,5.\,1862 Wien – 21.\,10.\,1931 ebd.), \emph{Schriftsteller, Mediziner}!Schleier der Beatrice. Schauspiel in fünf Akten@\strich\emph{Der Schleier der Beatrice. Schauspiel in fünf Akten}|pwv} zum
               Theil einverſtanden biſt. Ich habe noch einmal Dieſes und Jenes geleſen\strikeout{,} und kann Dir nur{ }ſagen: Seit \textsc{Grillparzer\pwindex{Grillparzer, Franz 15.\,1.\,1791 Wien – 21.\,1.\,1872 ebd.@\textsc{Grillparzer, Franz} (15.\,1.\,1791 Wien – 21.\,1.\,1872 ebd.), \emph{Schriftsteller, Beamter}|pw}} hat man auf dem Wien\oindex{Wien@\textbf{Wien}, \emph{Verwaltungsgebiet}|pw}er Theater{ }ſolche Verſe\pwindex{Schnitzler, Arthur 15.\,5.\,1862 Wien – 21.\,10.\,1931 ebd.@\textsc{Schnitzler, Arthur} (15.\,5.\,1862 Wien – 21.\,10.\,1931 ebd.), \emph{Schriftsteller, Mediziner}!Schleier der Beatrice. Schauspiel in fünf Akten@\strich\emph{Der Schleier der Beatrice. Schauspiel in fünf Akten}|pwv} nicht gehört. Das{ }ſoll
               aber nicht bedeuten, daß es \textsc{Grillparzer\pwindex{Grillparzer, Franz 15.\,1.\,1791 Wien – 21.\,1.\,1872 ebd.@\textsc{Grillparzer, Franz} (15.\,1.\,1791 Wien – 21.\,1.\,1872 ebd.), \emph{Schriftsteller, Beamter}|pw}ische} Verſe{ }ſind. Nein,{ }ſie{ }ſind durchaus \textsc{Schnitzlerisch}, und nur der weiche Wien\oindex{Wien@\textbf{Wien}, \emph{Verwaltungsgebiet}|pw}er Wohllaut iſt den beiden Dichtern\pwindex{Grillparzer, Franz 15.\,1.\,1791 Wien – 21.\,1.\,1872 ebd.@\textsc{Grillparzer, Franz} (15.\,1.\,1791 Wien – 21.\,1.\,1872 ebd.), \emph{Schriftsteller, Beamter}|pwv} gemeinſam. Was die Aufführung
               anlangt,{ }ſo {\pb}möchte ich Streichungen empfehlen.
               Vielleicht auch einige \label{K_L02905-2v}\edtext{Umarbeitungen}{\lemma{\textnormal{\emph{Umarbeitungen}}}\Cendnote{\textnormal{Entsprechende
                  Umarbeitungen sind keine bekannt.}}}\label{K_L02905-2}. Ich bleibe dabei: die Geſtalt des Herzogs\pwindex{Schnitzler, Arthur 15.\,5.\,1862 Wien – 21.\,10.\,1931 ebd.@\textsc{Schnitzler, Arthur} (15.\,5.\,1862 Wien – 21.\,10.\,1931 ebd.), \emph{Schriftsteller, Mediziner}!Schleier der Beatrice. Schauspiel in fünf Akten@\strich\emph{Der Schleier der Beatrice. Schauspiel in fünf Akten}|pwv} erſcheint mir in zu unklaren
               Umriſſen. Wenn da auch nur ein wenig mit feſter Hand nachgezeichnet würde, könnte das
               dem Drama\pwindex{Schnitzler, Arthur 15.\,5.\,1862 Wien – 21.\,10.\,1931 ebd.@\textsc{Schnitzler, Arthur} (15.\,5.\,1862 Wien – 21.\,10.\,1931 ebd.), \emph{Schriftsteller, Mediziner}!Schleier der Beatrice. Schauspiel in fünf Akten@\strich\emph{Der Schleier der Beatrice. Schauspiel in fünf Akten}|pwv}{ }ſehr zum Vortheil
               gereichen. Wäre es nicht doch möglich, daß die Hochzeit nur ein im Voraus
               beabſichtigter Carnevals-Scherz{ }ſein könnte? Wenn der Herzog\pwindex{Schnitzler, Arthur 15.\,5.\,1862 Wien – 21.\,10.\,1931 ebd.@\textsc{Schnitzler, Arthur} (15.\,5.\,1862 Wien – 21.\,10.\,1931 ebd.), \emph{Schriftsteller, Mediziner}!Schleier der Beatrice. Schauspiel in fünf Akten@\strich\emph{Der Schleier der Beatrice. Schauspiel in fünf Akten}|pwv} durchaus edel{ }ſein muß,{ }ſo könnte
               der Edelmuth ja nachher erwachen. Mich hat übrigens in Deinem Briefe das Wort »Größe«{ }ſtutzig gemacht. Warum{ }ſoll der Herzog\pwindex{Schnitzler, Arthur 15.\,5.\,1862 Wien – 21.\,10.\,1931 ebd.@\textsc{Schnitzler, Arthur} (15.\,5.\,1862 Wien – 21.\,10.\,1931 ebd.), \emph{Schriftsteller, Mediziner}!Schleier der Beatrice. Schauspiel in fünf Akten@\strich\emph{Der Schleier der Beatrice. Schauspiel in fünf Akten}|pwv} »groß«{ }ſein? Mir{ }ſcheint, dieſes Streben nach Größe, dieſe abſtrakt
               hinzugedachte Eigenſchaft, iſt an der Unklarheit{ }ſchuld. Hätteſt Du ihn nur (wie es{ }ſonſt Deine Gewohnheit iſt) ruhig und \substVorne{}\textsuperscript{natürlich}\substDazwischen{}natürlich\substHinten{} leben laſſen, wie er leben mochte,{ }ſo wäre \strikeout{\textcolor{gray}{er}} er deutlicher und wahrer geworden. Im Übrigen, vielleicht haſt Du Recht, und
                  {\pb}auf der Bühne zeigt{ }ſich vielleicht, daß die Figur\pwindex{Schnitzler, Arthur 15.\,5.\,1862 Wien – 21.\,10.\,1931 ebd.@\textsc{Schnitzler, Arthur} (15.\,5.\,1862 Wien – 21.\,10.\,1931 ebd.), \emph{Schriftsteller, Mediziner}!Schleier der Beatrice. Schauspiel in fünf Akten@\strich\emph{Der Schleier der Beatrice. Schauspiel in fünf Akten}|pwv} richtig gedacht war.\pend
           
\pstart
           Welche Rolle \label{K_L02905-3v}\edtext{\textsc{Kainz\pwindex{Kainz, Josef 2.\,1.\,1858 Mosonmagyaróvár – 20.\,9.\,1910 Wien@\textsc{Kainz, Josef} (2.\,1.\,1858 Mosonmagyaróvár – 20.\,9.\,1910 Wien), \emph{Schauspieler}|pw}}}{\lemma{\textnormal{\emph{Kainz}}}\Cendnote{\textnormal{Josef Kainz\pwindex{Kainz, Josef 2.\,1.\,1858 Mosonmagyaróvár – 20.\,9.\,1910 Wien@\textsc{Kainz, Josef} (2.\,1.\,1858 Mosonmagyaróvár – 20.\,9.\,1910 Wien), \emph{Schauspieler}|pwk} war ein von Schnitzler vielgeschätzter Schauspieler und mehrmals an
                  Inszenierungen seiner Dramen beteiligt. Für die geplante Uraufführung von \emph{Der Schleier der Beatrice}\pwindex{Schnitzler, Arthur 15.\,5.\,1862 Wien – 21.\,10.\,1931 ebd.@\textsc{Schnitzler, Arthur} (15.\,5.\,1862 Wien – 21.\,10.\,1931 ebd.), \emph{Schriftsteller, Mediziner}!Schleier der Beatrice. Schauspiel in fünf Akten@\strich\emph{Der Schleier der Beatrice. Schauspiel in fünf Akten}|pwk} im Burgtheater\oindex{Wien@\textbf{Wien}!I., Innere Stadt@\textbf{I., Innere Stadt}!Burgtheater@\textbf{Burgtheater}, \emph{Theater}|pwk} wollte Schnitzler{ }Kainz\pwindex{Kainz, Josef 2.\,1.\,1858 Mosonmagyaróvár – 20.\,9.\,1910 Wien@\textsc{Kainz, Josef} (2.\,1.\,1858 Mosonmagyaróvár – 20.\,9.\,1910 Wien), \emph{Schauspieler}|pwk} in der Rolle des Filippo\pwindex{Schnitzler, Arthur 15.\,5.\,1862 Wien – 21.\,10.\,1931 ebd.@\textsc{Schnitzler, Arthur} (15.\,5.\,1862 Wien – 21.\,10.\,1931 ebd.), \emph{Schriftsteller, Mediziner}!Schleier der Beatrice. Schauspiel in fünf Akten@\strich\emph{Der Schleier der Beatrice. Schauspiel in fünf Akten}|pwkv} sehen (vgl. XXXX Auszeichnungsfehler: Dokument L01014 nicht gefunden). Zu dieser Aufführung
                  kam es aber nicht (vgl. XXXX Auszeichnungsfehler: Dokument L02893 nicht gefunden).}}}\label{K_L02905-3}{ }ſpielen{ }ſoll, kann ich Dir nicht{ }ſagen. Denn ich kenne \textsc{Kainz\pwindex{Kainz, Josef 2.\,1.\,1858 Mosonmagyaróvár – 20.\,9.\,1910 Wien@\textsc{Kainz, Josef} (2.\,1.\,1858 Mosonmagyaróvár – 20.\,9.\,1910 Wien), \emph{Schauspieler}|pw}} nicht. Der Herzog\pwindex{Schnitzler, Arthur 15.\,5.\,1862 Wien – 21.\,10.\,1931 ebd.@\textsc{Schnitzler, Arthur} (15.\,5.\,1862 Wien – 21.\,10.\,1931 ebd.), \emph{Schriftsteller, Mediziner}!Schleier der Beatrice. Schauspiel in fünf Akten@\strich\emph{Der Schleier der Beatrice. Schauspiel in fünf Akten}|pwv} muß
               jedenfalls ein vollendeter \uline{Sprecher}{ }ſein, und mir{ }ſcheint, daß \textsc{Kainz\pwindex{Kainz, Josef 2.\,1.\,1858 Mosonmagyaróvár – 20.\,9.\,1910 Wien@\textsc{Kainz, Josef} (2.\,1.\,1858 Mosonmagyaróvár – 20.\,9.\,1910 Wien), \emph{Schauspieler}|pw}} das nicht iſt. Für die \textsc{Beatrice\pwindex{Schnitzler, Arthur 15.\,5.\,1862 Wien – 21.\,10.\,1931 ebd.@\textsc{Schnitzler, Arthur} (15.\,5.\,1862 Wien – 21.\,10.\,1931 ebd.), \emph{Schriftsteller, Mediziner}!Schleier der Beatrice. Schauspiel in fünf Akten@\strich\emph{Der Schleier der Beatrice. Schauspiel in fünf Akten}|pwv}} aber gibt es meiner Anſicht nach nur \uline{eine} auf
               den deutſchen Theatern: Die \label{K_L02905-4v}\edtext{\textsc{Triesch\pwindex{Triesch, Irene 13.\,4.\,1877 Wien – 24.\,11.\,1964 Basel@\textsc{Triesch, Irene} (13.\,4.\,1877 Wien – 24.\,11.\,1964 Basel), \emph{Schauspielerin}|pw}}}{\lemma{\textnormal{\emph{Triesch}}}\Cendnote{\textnormal{Irene Triesch\pwindex{Triesch, Irene 13.\,4.\,1877 Wien – 24.\,11.\,1964 Basel@\textsc{Triesch, Irene} (13.\,4.\,1877 Wien – 24.\,11.\,1964 Basel), \emph{Schauspielerin}|pwk} gestaltete erst 1903 die Beatrice\pwindex{Schnitzler, Arthur 15.\,5.\,1862 Wien – 21.\,10.\,1931 ebd.@\textsc{Schnitzler, Arthur} (15.\,5.\,1862 Wien – 21.\,10.\,1931 ebd.), \emph{Schriftsteller, Mediziner}!Schleier der Beatrice. Schauspiel in fünf Akten@\strich\emph{Der Schleier der Beatrice. Schauspiel in fünf Akten}|pwkv} am \emph{Deutschen Theater Berlin}\orgindex{Deutsches Theater Berlin@Deutsches Theater Berlin|pwk}
                  aus. Schnitzler missfiel, wie sie die Rolle anlegte
                     (vgl. A. S.: \emph{Tagebuch}, 23. 2. 1903).}}}\label{K_L02905-4} in
                  Frankfurt\oindex{Frankfurt am Main@\textbf{Frankfurt am Main}, \emph{Hauptstadt}|pw}. Sie hat geniale Kunſt-Inſtinkte,
               iſt{ }ſelbſt ein{ }ſo unberechenbares Luder, wie Deine \textsc{Beatrice\pwindex{Schnitzler, Arthur 15.\,5.\,1862 Wien – 21.\,10.\,1931 ebd.@\textsc{Schnitzler, Arthur} (15.\,5.\,1862 Wien – 21.\,10.\,1931 ebd.), \emph{Schriftsteller, Mediziner}!Schleier der Beatrice. Schauspiel in fünf Akten@\strich\emph{Der Schleier der Beatrice. Schauspiel in fünf Akten}|pwv}}, hat außerdem die Jugend und das{ }ſüdliche Feuer. Damit wäre jede Frage über die
               Bühnenwirkſamkeit der Figur mit einem Schlage beſeitigt. Die \textsc{Triesch\pwindex{Triesch, Irene 13.\,4.\,1877 Wien – 24.\,11.\,1964 Basel@\textsc{Triesch, Irene} (13.\,4.\,1877 Wien – 24.\,11.\,1964 Basel), \emph{Schauspielerin}|pw}} würde etwas Unerhörtes daraus machen. Wenn Du mir folgteſt, würdeſt Du alle
               Mittel aufbieten, um die Perſon für dieſe Rolle zu gewinnen. Aber leider folgſt Du
               mir ja niemals. In Berlin\oindex{Berlin@\textbf{Berlin}, \emph{Hauptstadt}|pw} könnte meiner Anſicht
               nach nur {\pb}das \label{K_L02905-5v}\edtext{»Deutſche Theater\orgindex{Deutsches Theater Berlin@Deutsches Theater Berlin|pw}«}{\lemma{\textnormal{\emph{»Deutsche Theater«}}}\Cendnote{\textnormal{Zwei Jahre nach der Uraufführung\eventindex{Lobe-Theater@\textbf{Lobe-Theater}!Uraufführung von Der Schleier der Beatrice, 1.12.1900@Uraufführung von Der Schleier der Beatrice, 1.12.1900|pwkv} in Breslau\oindex{Breslau@\textbf{Breslau}|pwk} (1. 12. 1900) fand am 7. 3. 1903 die
                  Premiere\eventindex{Deutsches Theater Berlin@\textbf{Deutsches Theater Berlin}!Premiere von Der Schleier der Beatrice, 7.3.1903@Premiere von Der Schleier der Beatrice, 7.3.1903|pwkv} am \emph{Deutschen Theater Berlin}\orgindex{Deutsches Theater Berlin@Deutsches Theater Berlin|pwk} statt.
                     Otto Brahm\pwindex{Brahm, Otto 5.\,2.\,1856 Hamburg – 28.\,11.\,1912 Berlin@\textsc{Brahm, Otto} (5.\,2.\,1856 Hamburg – 28.\,11.\,1912 Berlin), \emph{Theaterleiter, Regisseur}|pwk} kannte das Stück\pwindex{Schnitzler, Arthur 15.\,5.\,1862 Wien – 21.\,10.\,1931 ebd.@\textsc{Schnitzler, Arthur} (15.\,5.\,1862 Wien – 21.\,10.\,1931 ebd.), \emph{Schriftsteller, Mediziner}!Schleier der Beatrice. Schauspiel in fünf Akten@\strich\emph{Der Schleier der Beatrice. Schauspiel in fünf Akten}|pwkv} bereits seit 7. 10. 1899.}}}\label{K_L02905-5} in Betracht kommen. \textsc{Brahms\pwindex{Brahm, Otto 5.\,2.\,1856 Hamburg – 28.\,11.\,1912 Berlin@\textsc{Brahm, Otto} (5.\,2.\,1856 Hamburg – 28.\,11.\,1912 Berlin), \emph{Theaterleiter, Regisseur}|pw}}{ }\strikeout{iſt} zeigt{ }ſich{ }ſehr urtheilslos, wenn er nach dem Stück\pwindex{Schnitzler, Arthur 15.\,5.\,1862 Wien – 21.\,10.\,1931 ebd.@\textsc{Schnitzler, Arthur} (15.\,5.\,1862 Wien – 21.\,10.\,1931 ebd.), \emph{Schriftsteller, Mediziner}!Schleier der Beatrice. Schauspiel in fünf Akten@\strich\emph{Der Schleier der Beatrice. Schauspiel in fünf Akten}|pwv} nicht mit beiden Händen
               greift. Wenn es in Wien\oindex{Wien@\textbf{Wien}, \emph{Verwaltungsgebiet}|pw} Erfolg hat, wird er es
               übrigens{ }ſchon thun. An das \label{K_L02905-6v}\edtext{Schauſpielhaus\orgindex{Schauspielhaus Berlin@Schauspielhaus Berlin|pw}}{\lemma{\textnormal{\emph{Schauspielhaus}}}\Cendnote{\textnormal{Zu einer Inszenierung am \emph{Schauspielhaus Berlin}\orgindex{Schauspielhaus Berlin@Schauspielhaus Berlin|pwk} kam es nicht.}}}\label{K_L02905-6} iſt bei der
               jetzt herrſchenden Sittlichkeits-Manie nicht zu denken. Man würde Dein Drama\pwindex{Schnitzler, Arthur 15.\,5.\,1862 Wien – 21.\,10.\,1931 ebd.@\textsc{Schnitzler, Arthur} (15.\,5.\,1862 Wien – 21.\,10.\,1931 ebd.), \emph{Schriftsteller, Mediziner}!Schleier der Beatrice. Schauspiel in fünf Akten@\strich\emph{Der Schleier der Beatrice. Schauspiel in fünf Akten}|pwv} entweder überhaupt nicht
               nehmen oder Dir zumuthen, die Hälfte wegzulaſſen. Im Nothfall könnte man es auch mit
               dem \label{K_L02905-7v}\edtext{»Berliner Theater\orgindex{Berliner Theater@Berliner Theater|pw}«}{\lemma{\textnormal{\emph{»Berliner Theater«}}}\Cendnote{\textnormal{Zu einer
                  Inszenierung von \emph{Der Schleier der Beatrice}\pwindex{Schnitzler, Arthur 15.\,5.\,1862 Wien – 21.\,10.\,1931 ebd.@\textsc{Schnitzler, Arthur} (15.\,5.\,1862 Wien – 21.\,10.\,1931 ebd.), \emph{Schriftsteller, Mediziner}!Schleier der Beatrice. Schauspiel in fünf Akten@\strich\emph{Der Schleier der Beatrice. Schauspiel in fünf Akten}|pwk} am
                     \emph{Berliner Theater}\orgindex{Berliner Theater@Berliner Theater|pwk} kam es nicht.}}}\label{K_L02905-7}
               (Direktion \textsc{Paul Lindau\pwindex{Lindau, Paul 3.\,6.\,1839 Magdeburg – 31.\,1.\,1919 Berlin@\textsc{Lindau, Paul} (3.\,6.\,1839 Magdeburg – 31.\,1.\,1919 Berlin), \emph{Schriftsteller, Kritiker, Theaterleiter}|pw}}) verſuchen, wo nicht{ }ſchlecht geſpielt wird; nur die Ausſtattung würde hier
               armſeelig{ }ſein.\pend
           
\pstart
           Deine \label{K_L02905-8v}\edtext{Aufträge}{\lemma{\textnormal{\emph{Aufträge}}}\Cendnote{\textnormal{Bezug unklar}}}\label{K_L02905-8} an \textsc{Gusti\pwindex{Glümer, Auguste 16.\,3.\,1862 Wien – 1956@\textsc{Glümer, Auguste} (16.\,3.\,1862 Wien – 1956), \emph{Lehrerin}|pwv}} u. die \label{K_L02905-9v}\edtext{Frau
                  Rechtsanwalt\pwindex{Freudenthal, Rosa 1862 – 18.\,6.\,1905 Berlin@\textsc{Freudenthal, Rosa} (1862 – 18.\,6.\,1905 Berlin)|pwv}}{\lemma{\textnormal{\emph{Frau
                  Rechtsanwalt}}}\Cendnote{\textnormal{Schnitzlers ehemalige Geliebte Rosa Freudenthal\pwindex{Freudenthal, Rosa 1862 – 18.\,6.\,1905 Berlin@\textsc{Freudenthal, Rosa} (1862 – 18.\,6.\,1905 Berlin)|pwk} war mit dem Rechtsanwalt Hermann Freudenthal\pwindex{Freudenthal, Hermann 1852/1853 – 12.\,9.\,1925 Berlin@\textsc{Freudenthal, Hermann} (1852/1853 – 12.\,9.\,1925 Berlin), \emph{Rechtsanwalt}|pwk} verheiratet. Goldmann\pwindex{Goldmann, Paul 31.\,1.\,1865 Breslau – 25.\,9.\,1935 Wien@\textsc{Goldmann, Paul} (31.\,1.\,1865 Breslau – 25.\,9.\,1935 Wien), \emph{Schriftsteller, Journalist}|pwk} hatte sich bereits 1897 mit einer ähnlichen Formulierung auf sie bezogen (vgl. XXXX Auszeichnungsfehler: Dokument L02822 nicht gefunden).}}}\label{K_L02905-9} werde ich
               beſorgen.\pend
           
\pstart
           Das Theaterreferat\orgindex{Neue Freie Presse@Neue Freie Presse|pwv} von hier aus
               hat{ }ſeine Schwierigkeiten. Ich muß doch alle Deine \label{K_L02905-10v}\edtext{Geliebten}{\lemma{\textnormal{\emph{Geliebten}}}\Cendnote{\textnormal{Das
                  dürfte vor allem als Anspielung auf Marie
                     Glümer\pwindex{Glümer, Marie 3.\,7.\,1867 Wien – 16.\,11.\,1925 München@\textsc{Glümer, Marie} (3.\,7.\,1867 Wien – 16.\,11.\,1925 München), \emph{Schauspielerin}|pwk} zu lesen sein.}}}\label{K_L02905-10} loben. Um Irrthümer auszuſchließen, werde ich
               Dich demnächſt um einen Katalog bitten.\pend
           
\pstart
           {\pb}Von mir willſt Du hören? Siehſt Du, ich habe wenig \substVorne{}\textsuperscript{\textcolor{gray}{h}}\substDazwischen{}Z\substHinten{}eit zum Schreiben. Ich muß alſo wählen:{ }ſoll ich Dir von Dir{ }ſchreiben oder
               von mir? Und Du wirſt doch nicht leugnen, daß es Dich mehr intereſſirt, wenn ich Dir
               über Dein Stück\pwindex{Schnitzler, Arthur 15.\,5.\,1862 Wien – 21.\,10.\,1931 ebd.@\textsc{Schnitzler, Arthur} (15.\,5.\,1862 Wien – 21.\,10.\,1931 ebd.), \emph{Schriftsteller, Mediziner}!Schleier der Beatrice. Schauspiel in fünf Akten@\strich\emph{Der Schleier der Beatrice. Schauspiel in fünf Akten}|pwv}{ }ſchreibe, als
               über meine Schmerzen und \substVorne{}\textsuperscript{\textcolor{gray}{ſ}}\substDazwischen{}S\substHinten{}orgen. Oder vielmehr, Du wirſt es leugnen, aber ich werde Dir nicht
               glauben.\pend
           
\pstart
           Auf Umwegen höre ich, daß Dein \label{K_L02905-11v}\edtext{Bruder\pwindex{Schnitzler, Julius 13.\,7.\,1865 Wien – 29.\,6.\,1939 ebd.@\textsc{Schnitzler, Julius} (13.\,7.\,1865 Wien – 29.\,6.\,1939 ebd.), \emph{Chirurg}|pwv} ein Mädchen\pwindex{Donath, Anna 23.\,1.\,1900 Wien – 27.\,12.\,1995 Cincinnati@\textsc{Donath, Anna} (23.\,1.\,1900 Wien – 27.\,12.\,1995 Cincinnati)|pwv} bekommen}{\lemma{\textnormal{\emph{Bruder … bekommen}}}\Cendnote{\textnormal{Anna\pwindex{Donath, Anna 23.\,1.\,1900 Wien – 27.\,12.\,1995 Cincinnati@\textsc{Donath, Anna} (23.\,1.\,1900 Wien – 27.\,12.\,1995 Cincinnati)|pwk}, das dritte Kind von Julius\pwindex{Schnitzler, Julius 13.\,7.\,1865 Wien – 29.\,6.\,1939 ebd.@\textsc{Schnitzler, Julius} (13.\,7.\,1865 Wien – 29.\,6.\,1939 ebd.), \emph{Chirurg}|pwk} und Helene
                     Schnitzler\pwindex{Schnitzler, Helene 16.\,7.\,1871 Budapest – September 1941 Atlantischer Ozean@\textsc{Schnitzler, Helene} (16.\,7.\,1871 Budapest – September 1941 Atlantischer Ozean)|pwk}, war am 23. 1. 1900 geboren worden.}}}\label{K_L02905-11} hat. Bitte, übermittle den Eltern\pwindex{Schnitzler, Julius 13.\,7.\,1865 Wien – 29.\,6.\,1939 ebd.@\textsc{Schnitzler, Julius} (13.\,7.\,1865 Wien – 29.\,6.\,1939 ebd.), \emph{Chirurg}|pwv}\pwindex{Schnitzler, Helene 16.\,7.\,1871 Budapest – September 1941 Atlantischer Ozean@\textsc{Schnitzler, Helene} (16.\,7.\,1871 Budapest – September 1941 Atlantischer Ozean)|pwv} meine {\pb}Glückwünſche zugleich mit meinen herzlichen Grüßen.
               Auch Deine übrigen Angehörigen bitte ich zu grüßen.\pend
           
\pstart
           Eine Wien\oindex{Wien@\textbf{Wien}, \emph{Verwaltungsgebiet}|pw}er Jüdin, ein Frl. \textsc{Schreiber\pwindex{Schreiber, Adele 29.\,4.\,1872 Wien – 20.\,2.\,1957 Herrliberg@\textsc{Schreiber, Adele} (29.\,4.\,1872 Wien – 20.\,2.\,1957 Herrliberg), \emph{Schriftstellerin, Politikerin, Pädagogin}|pw}}, iſt mir mit einer Empfehlung von \textsc{Hanslick\pwindex{Hanslick, Eduard 11.\,9.\,1825 Prag – 6.\,8.\,1904 Baden bei Wien@\textsc{Hanslick, Eduard} (11.\,9.\,1825 Prag – 6.\,8.\,1904 Baden bei Wien), \emph{Musikkritiker}|pw}} ins Haus gekommen. Sie will hier\oindex{Berlin@\textbf{Berlin}, \emph{Hauptstadt}|pwv} einen \label{K_L02905-12v}\edtext{Vortrag\pwindex{Schreiber, Adele 29.\,4.\,1872 Wien – 20.\,2.\,1957 Herrliberg@\textsc{Schreiber, Adele} (29.\,4.\,1872 Wien – 20.\,2.\,1957 Herrliberg), \emph{Schriftstellerin, Politikerin, Pädagogin}!Vortrag über Arthur Schnitzler]@\strich\emph{[Vortrag über Arthur Schnitzler]}|pwv}}{\lemma{\textnormal{\emph{Vortrag}}}\Cendnote{\textnormal{Der Vortrag\pwindex{Schreiber, Adele 29.\,4.\,1872 Wien – 20.\,2.\,1957 Herrliberg@\textsc{Schreiber, Adele} (29.\,4.\,1872 Wien – 20.\,2.\,1957 Herrliberg), \emph{Schriftstellerin, Politikerin, Pädagogin}!Vortrag über Arthur Schnitzler]@\strich\emph{[Vortrag über Arthur Schnitzler]}|pwkv} von Adele
                     Schreiber\pwindex{Schreiber, Adele 29.\,4.\,1872 Wien – 20.\,2.\,1957 Herrliberg@\textsc{Schreiber, Adele} (29.\,4.\,1872 Wien – 20.\,2.\,1957 Herrliberg), \emph{Schriftstellerin, Politikerin, Pädagogin}|pwk}, veranstaltet von der \emph{Gesellschaft für Kunst und Wissenschaft}\orgindex{Lessing-Gesellschaft für Kunst und Wissenschaft@Lessing-Gesellschaft für Kunst und Wissenschaft|pwk} in Berlin\oindex{Berlin@\textbf{Berlin}, \emph{Hauptstadt}|pwk}, fand am 28. 3. 1900 statt.}}}\label{K_L02905-12}
               über Dich halten (was ich bedaure, denn der Vortrag\pwindex{Schreiber, Adele 29.\,4.\,1872 Wien – 20.\,2.\,1957 Herrliberg@\textsc{Schreiber, Adele} (29.\,4.\,1872 Wien – 20.\,2.\,1957 Herrliberg), \emph{Schriftstellerin, Politikerin, Pädagogin}!Vortrag über Arthur Schnitzler]@\strich\emph{[Vortrag über Arthur Schnitzler]}|pwv} wird{ }ſchlecht{ }ſein) und hat mir inzwiſchen im
               Geſpräch werthvolle literariſche Aufſchlüſſe über Dich gegeben.\pend
           
\pstart
           Viele treue Grüße! {\\[\baselineskip]}Dein {\\[\baselineskip]}\spacefill\mbox{Paul Goldmann.}\pend
           \leftskip=0em{}
\pstart
           \noindent{}Ja, eine Bitte habe ich doch. Ich habe den Eindruck, daß ich in der N. Fr. Preſſe\orgindex{Neue Freie Presse@Neue Freie Presse|pw}, im Gegenſatz zur \label{K_L02905-13v}\edtext{Frankfurter {\pb}Zeitung\orgindex{Frankfurter Zeitung@Frankfurter Zeitung|pw}}{\lemma{\textnormal{\emph{Frankfurter Zeitung}}}\Cendnote{\textnormal{Goldmann\pwindex{Goldmann, Paul 31.\,1.\,1865 Breslau – 25.\,9.\,1935 Wien@\textsc{Goldmann, Paul} (31.\,1.\,1865 Breslau – 25.\,9.\,1935 Wien), \emph{Schriftsteller, Journalist}|pwk} hatte bis Dezember 1899 für die
                     \emph{Frankfurter Zeitung}\orgindex{Frankfurter Zeitung@Frankfurter Zeitung|pwk}
                     gearbeitet.}}}\label{K_L02905-13}, vollſtändig verſchwinde. Merkt irgend Jemand, außer
                  Dir, daß ich vorhanden bin? Bitte,{ }ſchreib’ mir ein Wort darüber!\pend
           \selectlanguage{ngerman}\endnumbering\briefempfaengerindex{Schnitzler, Arthur@\textsc{Schnitzler, Arthur}!zzzGoldmann, Paul@\emph{von Paul Goldmann}!1900-02-201@{20. 2. 1900}|)be}\mylabel{L02905h}  \newcommand{\dateiname}{L02905}\newcommand{\titel}{Paul Goldmann an Arthur Schnitzler, 20. 2. 1900}\newcommand{\editorInnen}{Martin Anton Müller und Laura Untner}%% latex-leseansicht-abspann.tex
%% Abspann für die Leseansicht.
%% Der Schalter \ifkorrekturansicht ist bereits durch den Vorspann gesetzt.

%% latex-abspann.tex
%% Gemeinsamer Abspann für Korrekturansicht und Leseansicht.
%% Setzt den Schalter \ifkorrekturansicht voraus (gesetzt in den
%% einbindenden Dateien latex-korrekturansicht-abspann.tex bzw.
%% latex-leseansicht-abspann.tex).
%% ---------------------------------------------------------------

\normalsize

% Das esempio-Environment wird nur in der Leseansicht benötigt
\ifkorrekturansicht\else
\newenvironment{esempio}[3]%
{
    \vspace{1.5ex}
    \rlap{\underline{#1}}
    \par
    \setlength{\parindent}{0cm}
    \nopagebreak
    \leftskip=#2cm
    \rightskip=#3cm
}
{
    \par
}
\fi

\doendnotes{C}
\bigskip
\vfill

\clearpage

\footnotesize

\ifkorrekturansicht
  \lohead{\textsc{register}}
\fi

% theindex-Environment neu definieren ohne reledmac
\makeatletter
\renewenvironment{theindex}{%
  \ifkorrekturansicht
    \section*{\indexname}%
  \else
    \subsubsection*{Index der erwähnten Entitäten}%
  \fi
  \setlength{\parindent}{0pt}%
  \setlength{\parskip}{0pt plus 0.3pt}%
  \let\item\@idxitem
}{%
  \ifkorrekturansicht\clearpage\fi
}
\makeatother

\IfFileExists{\jobname-pw.ind}{\input{\jobname-pw.ind}}{}

% Quellenangabe nur in der Leseansicht
\ifkorrekturansicht\else
% Fallback-Definitionen, falls die .tex-Datei \titel etc. nicht gesetzt hat
\providecommand{\titel}{}
\providecommand{\editorInnen}{}
\providecommand{\dateiname}{\jobname}

\vspace{3cm}

\vfill

\footnotesize
\textsc{Quelle}: \titel. Herausgegeben von {\editorInnen}. In: \emph{Arthur Schnitzler: Briefwechsel mit Autorinnen und Autoren}.
 Digitale Edition, https://schnitzler-briefe.acdh.oeaw.ac.at/{\dateiname}.html (Stand \today)
\fi

\end{document}


