%% latex-leseansicht-vorspann.tex
%% Vorspann für die Leseansicht.
%% Lädt die gemeinsame Datei latex-vorspann.tex mit nicht gesetztem Schalter.

\newif\ifkorrekturansicht
\korrekturansichtfalse

\input{../tex-inputs/latex-vorspann}


\section[Arthur Schnitzler an Gustav Schwarzkopf, {{[}}zwischen 21. 3. 1892 und 30. 3. 1892?{{]}}]{L04216 Arthur Schnitzler an Gustav Schwarzkopf, {[}zwischen 21. 3. 1892 und 30. 3. 1892?{]}}
\nopagebreak\mylabel{L04216v}
\rehead{ }\normalsize\beginnumbering\briefempfaengerindex{Schwarzkopf, Gustav@\textsc{Schwarzkopf, Gustav}!zzzSchnitzler, Arthur@\emph{von Arthur Schnitzler}!1892-03-302@{{[}zwischen 21. 3. 1892 und 30. 3. 1892?{]}}|(be}
\toendnotes[C]{\smallbreak\pagebreak[2]}
\correspDesc{Versand  durch Arthur Schnitzler im Zeitraum [zwischen 21. 3. 1892 und 30. 3. 1892?] in Wien
\newline{}Erhalt  durch Gustav Schwarzkopf in Wien}\toendnotes[C]{\smallbreak}
\Standort{CUL, Schnitzler, B 96.}
\physDesc{Brief, 1 Blatt, 2 Seiten, 611 Zeichen
\newline{}Handschrift: schwarze Tinte, deutsche Kurrent}\toendnotes[C]{\smallbreak}
\pstart{}{\pb}Verehrteſter Herr
                  Schwarzkopf,\pend\vspace{0.5em}
\pstart
           \label{K_L04099-1v}\edtext{hier ist das Manuscript\pwindex{Schnitzler, Arthur 15. 5. 1862 Wien – 21. 10. 1931 ebd.@\textsc{Schnitzler, Arthur} (15. 5. 1862 Wien – 21. 10. 1931 ebd.), \emph{Schriftsteller, Mediziner}!Anatol@\strich\emph{Anatol}|pwv}}{\lemma{\textnormal{\emph{hier ist das Manuscript}}}\Cendnote{\textnormal{Der
                  vorliegende Brief ist undatiert und kann nur in einen bestimmten Zeitraum verortet werden. Am XXXX Auszeichnungsfehler: Dokument L00083 nicht gefunden erbat
                  sich Schnitzler von Hugo von Hofmannsthal\pwindex{Hofmannsthal, Hugo von 1.\,2.\,1874 Wien – 15.\,7.\,1929 Rodaun@\textsc{Hofmannsthal, Hugo von} (1.\,2.\,1874 Wien – 15.\,7.\,1929 Rodaun), \emph{Schriftsteller}|pwk} die Adresse von Schwarzkopf\pwindex{Schwarzkopf, Gustav 7.\,11.\,1853 Wien – 13.\,11.\,1939 ebd.@\textsc{Schwarzkopf, Gustav} (7.\,11.\,1853 Wien – 13.\,11.\,1939 ebd.), \emph{Schriftsteller}|pwk},
                  die dieser ihm in seiner Antwort vom XXXX Auszeichnungsfehler: Dokument L00083 nicht gefunden nicht übermittelt, sondern auf 
                  ein persönliches Gespräch verweist, das dann am 21. 3. 1892 stattfand. Am
                  selben Tag lieh sich Schnitzler bei Salten\pwindex{Salten, Felix 6.\,9.\,1869 Budapest – 8.\,10.\,1945 Zürich@\textsc{Salten, Felix} (6.\,9.\,1869 Budapest – 8.\,10.\,1945 Zürich), \emph{Schriftsteller, Journalist, Chefredakteur}|pwk} ein
                  von Schwarzkopf\pwindex{Schwarzkopf, Gustav 7.\,11.\,1853 Wien – 13.\,11.\,1939 ebd.@\textsc{Schwarzkopf, Gustav} (7.\,11.\,1853 Wien – 13.\,11.\,1939 ebd.), \emph{Schriftsteller}|pwk} verfasstes Buch\pwindex{Schwarzkopf, Gustav 7.\,11.\,1853 Wien – 13.\,11.\,1939 ebd.@\textsc{Schwarzkopf, Gustav} (7.\,11.\,1853 Wien – 13.\,11.\,1939 ebd.), \emph{Schriftsteller}!Bilanz der Ehe. Novellistische Studien. 2 Bde.@\strich\emph{Die Bilanz der Ehe. Novellistische Studien. 2 Bde.}|pwkv} aus (XXXX Auszeichnungsfehler: Dokument L02955 nicht gefunden),
                  so dass dieser oder die folgenden Tage für das Schreiben des Briefes anzunehmen sind.}}}\label{K_L04099-1} des
                  Buchs\pwindex{Schnitzler, Arthur 15. 5. 1862 Wien – 21. 10. 1931 ebd.@\textsc{Schnitzler, Arthur} (15. 5. 1862 Wien – 21. 10. 1931 ebd.), \emph{Schriftsteller, Mediziner}!Anatol@\strich\emph{Anatol}|pwv}. Die von mir gedachte
               Reihenfolge liegt bei. Die Entſcheidung über die \label{K_L04099-2v}\edtext{Gedichte\pwindex{Schnitzler, Arthur 15. 5. 1862 Wien – 21. 10. 1931 ebd.@\textsc{Schnitzler, Arthur} (15. 5. 1862 Wien – 21. 10. 1931 ebd.), \emph{Schriftsteller, Mediziner}!Blasirte@\strich\emph{Der Blasirte}|pwv}\pwindex{Schnitzler, Arthur 15. 5. 1862 Wien – 21. 10. 1931 ebd.@\textsc{Schnitzler, Arthur} (15. 5. 1862 Wien – 21. 10. 1931 ebd.), \emph{Schriftsteller, Mediziner}!Liebesgeständnis@\strich\emph{Liebesgeständnis}|pwv}\pwindex{Schnitzler, Arthur 15. 5. 1862 Wien – 21. 10. 1931 ebd.@\textsc{Schnitzler, Arthur} (15. 5. 1862 Wien – 21. 10. 1931 ebd.), \emph{Schriftsteller, Mediziner}!Lieder eines Nervösen@\strich\emph{Lieder eines Nervösen}|pwv}\pwindex{Schnitzler, Arthur 15. 5. 1862 Wien – 21. 10. 1931 ebd.@\textsc{Schnitzler, Arthur} (15. 5. 1862 Wien – 21. 10. 1931 ebd.), \emph{Schriftsteller, Mediziner}!An die Alten@\strich\emph{An die Alten}|pwv}\pwindex{Schnitzler, Arthur 15. 5. 1862 Wien – 21. 10. 1931 ebd.@\textsc{Schnitzler, Arthur} (15. 5. 1862 Wien – 21. 10. 1931 ebd.), \emph{Schriftsteller, Mediziner}!gar Manche@\strich\emph{An gar Manche}|pwv}\pwindex{Schnitzler, Arthur 15. 5. 1862 Wien – 21. 10. 1931 ebd.@\textsc{Schnitzler, Arthur} (15. 5. 1862 Wien – 21. 10. 1931 ebd.), \emph{Schriftsteller, Mediziner}!And’re@\strich\emph{Der And’re}|pwv}\pwindex{Schnitzler, Arthur 15. 5. 1862 Wien – 21. 10. 1931 ebd.@\textsc{Schnitzler, Arthur} (15. 5. 1862 Wien – 21. 10. 1931 ebd.), \emph{Schriftsteller, Mediziner}!Intermezzo@\strich\emph{Intermezzo}|pwv}\pwindex{Schnitzler, Arthur 15. 5. 1862 Wien – 21. 10. 1931 ebd.@\textsc{Schnitzler, Arthur} (15. 5. 1862 Wien – 21. 10. 1931 ebd.), \emph{Schriftsteller, Mediziner}!Wildenstein@\strich\emph{Wildenstein}|pwv}}{\lemma{\textnormal{\emph{Gedichte}}}\Cendnote{\textnormal{Schnitzler hatte
                  mehrere Gedichte unter seinem Pseudonym ›Anatol‹ veröffentlicht und überlegte zu
                     diesem Zeitpunkt also noch, diese in die Textzusammenstellung \emph{Anatol}\pwindex{Schnitzler, Arthur 15. 5. 1862 Wien – 21. 10. 1931 ebd.@\textsc{Schnitzler, Arthur} (15. 5. 1862 Wien – 21. 10. 1931 ebd.), \emph{Schriftsteller, Mediziner}!Anatol@\strich\emph{Anatol}|pwk} aufzunehmen. 
                     Letztlich entschied er sich, wie hier schon absehbar, dagegen.
                  }}}\label{K_L04099-2} überlaſſe ich vollko{\geminationm}en Ihnen. Sind Sie der Anſicht (der ich mich übrigens auch ſchon zuzuneigen
               beginne,) daß ſie nicht hineingehören, ſo behalten Sie ſie einfach zurück. – Das Gedicht\pwindex{Hofmannsthal, Hugo von 1.\,2.\,1874 Wien – 15.\,7.\,1929 Rodaun@\textsc{Hofmannsthal, Hugo von} (1.\,2.\,1874 Wien – 15.\,7.\,1929 Rodaun), \emph{Schriftsteller}!Prolog [zum Anatol]@\strich\emph{Prolog [zum Anatol]}|pwv} von \label{K_L04099-3v}\edtext{Morren\pwindex{Hofmannsthal, Hugo von 1.\,2.\,1874 Wien – 15.\,7.\,1929 Rodaun@\textsc{Hofmannsthal, Hugo von} (1.\,2.\,1874 Wien – 15.\,7.\,1929 Rodaun), \emph{Schriftsteller}|pw}}{\lemma{\textnormal{\emph{Morren}}}\Cendnote{\textnormal{In der gedruckten Ausgabe
                  ist der \emph{Prolog}\pwindex{Hofmannsthal, Hugo von 1.\,2.\,1874 Wien – 15.\,7.\,1929 Rodaun@\textsc{Hofmannsthal, Hugo von} (1.\,2.\,1874 Wien – 15.\,7.\,1929 Rodaun), \emph{Schriftsteller}!Prolog [zum Anatol]@\strich\emph{Prolog [zum Anatol]}|pwk}{ }Hofmannsthals\pwindex{Hofmannsthal, Hugo von 1.\,2.\,1874 Wien – 15.\,7.\,1929 Rodaun@\textsc{Hofmannsthal, Hugo von} (1.\,2.\,1874 Wien – 15.\,7.\,1929 Rodaun), \emph{Schriftsteller}|pwk} mit dessen
                  anderem Pseudonym, Loris\pwindex{Hofmannsthal, Hugo von 1.\,2.\,1874 Wien – 15.\,7.\,1929 Rodaun@\textsc{Hofmannsthal, Hugo von} (1.\,2.\,1874 Wien – 15.\,7.\,1929 Rodaun), \emph{Schriftsteller}|pwk}
                  gekennzeichnet.}}}\label{K_L04099-3} ko{\geminationm}t natürlich jedenfalls als
               Motto. Darf ich {\pb}Sie auch daran eri{\geminationn}ern, an \textsc{\label{K_L04099-4v}\edtext{Minden\orgindex{Heinrich Minden@Heinrich Minden|pw}}{\lemma{\textnormal{\emph{Minden}}}\Cendnote{\textnormal{Das 
                        Verlagshaus 
                        \emph{Heinrich Minden}\orgindex{Heinrich Minden@Heinrich Minden|pwk} in Dresden\oindex{Dresden@\textbf{Dresden}|pwk} hatte die 
                        meisten der Werke Schwarzkopfs\pwindex{Schwarzkopf, Gustav 7.\,11.\,1853 Wien – 13.\,11.\,1939 ebd.@\textsc{Schwarzkopf, Gustav} (7.\,11.\,1853 Wien – 13.\,11.\,1939 ebd.), \emph{Schriftsteller}|pwk} herausgebracht. An seinem dreißigsten Geburtstag, am 15. 5. 1892
                        fand Schnitzler bei der abendlichen Heimkehr das Manuskript mit der Ablehnung 
                        durch \emph{Minden}\orgindex{Heinrich Minden@Heinrich Minden|pwk} vor.}}}\label{K_L04099-4}} eine Anfrage betreffs den Illustrationen \introOben{}machen\introOben{} zu wollen?–\pend
           
\pstart
           – Nochmals will ich Ihnen aufs allermärmſte für Ihre große Freundlichkeit
               danken!\pend
           
\pstart
           Mit herzlichem Gruße{\\[\baselineskip]} Ihr Sie hochſchätzender{\\[\baselineskip]}\spacefill\mbox{Dr. Arthur Schnitzler}\pend
           \leftskip=0em{}\selectlanguage{ngerman}\endnumbering\briefempfaengerindex{Schwarzkopf, Gustav@\textsc{Schwarzkopf, Gustav}!zzzSchnitzler, Arthur@\emph{von Arthur Schnitzler}!1892-03-212@{{[}zwischen 21. 3. 1892 und 30. 3. 1892?{]}}|)be}\mylabel{L04216h}
\begin{anhang}
\end{anhang}\newcommand{\dateiname}{L04216}\newcommand{\titel}{Arthur Schnitzler an Gustav Schwarzkopf, [zwischen 21. 3. 1892 und 30. 3. 1892?]}\newcommand{\editorInnen}{Herausgegeben von Jahnke, SelmaMüller, Martin Anton}%% latex-leseansicht-abspann.tex
%% Abspann für die Leseansicht.
%% Der Schalter \ifkorrekturansicht ist bereits durch den Vorspann gesetzt.

%% latex-abspann.tex
%% Gemeinsamer Abspann für Korrekturansicht und Leseansicht.
%% Setzt den Schalter \ifkorrekturansicht voraus (gesetzt in den
%% einbindenden Dateien latex-korrekturansicht-abspann.tex bzw.
%% latex-leseansicht-abspann.tex).
%% ---------------------------------------------------------------

\normalsize

% Das esempio-Environment wird nur in der Leseansicht benötigt
\ifkorrekturansicht\else
\newenvironment{esempio}[3]%
{
    \vspace{1.5ex}
    \rlap{\underline{#1}}
    \par
    \setlength{\parindent}{0cm}
    \nopagebreak
    \leftskip=#2cm
    \rightskip=#3cm
}
{
    \par
}
\fi

\doendnotes{C}
\bigskip
\vfill

\clearpage

\footnotesize

\ifkorrekturansicht
  \lohead{\textsc{register}}
\fi

% theindex-Environment neu definieren ohne reledmac
\makeatletter
\renewenvironment{theindex}{%
  \ifkorrekturansicht
    \section*{\indexname}%
  \else
    \subsubsection*{Index der erwähnten Entitäten}%
  \fi
  \setlength{\parindent}{0pt}%
  \setlength{\parskip}{0pt plus 0.3pt}%
  \let\item\@idxitem
}{%
  \ifkorrekturansicht\clearpage\fi
}
\makeatother

\IfFileExists{\jobname-pw.ind}{\input{\jobname-pw.ind}}{}

% Quellenangabe nur in der Leseansicht
\ifkorrekturansicht\else
% Fallback-Definitionen, falls die .tex-Datei \titel etc. nicht gesetzt hat
\providecommand{\titel}{}
\providecommand{\editorInnen}{}
\providecommand{\dateiname}{\jobname}

\vspace{3cm}

\vfill

\footnotesize
\textsc{Quelle}: \titel. Herausgegeben von {\editorInnen}. In: \emph{Arthur Schnitzler: Briefwechsel mit Autorinnen und Autoren}.
 Digitale Edition, https://schnitzler-briefe.acdh.oeaw.ac.at/{\dateiname}.html (Stand \today)
\fi

\end{document}


