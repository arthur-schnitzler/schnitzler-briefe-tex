%% latex-korrekturansicht-vorspann.tex
%% Vorspann für die Korrekturansicht.
%% Lädt die gemeinsame Datei latex-vorspann.tex mit gesetztem Schalter.

\newif\ifkorrekturansicht
\korrekturansichttrue

\input{../tex-inputs/latex-vorspann}


\section[Hermann Bahr an Arthur Schnitzler, 4. 7. 1906]{L01607 Hermann Bahr an Arthur Schnitzler, 4. 7. 1906}
\nopagebreak\mylabel{L01607v}
\rehead{ }\normalsize\beginnumbering\briefempfaengerindex{Schnitzler, Arthur@\textsc{Schnitzler, Arthur}!zzzBahr, Hermann@\emph{von Hermann Bahr}!1906-07-041@{4. 7. 1906}|(be}
\toendnotes[C]{\smallbreak\pagebreak[2]}\Standort{CUL, Schnitzler, B 5b.}
\physDesc{Postkarte, 466 Zeichen
\newline{}Handschrift: 1) Bleistift, deutsche Kurrent\hspace{1em}2) Bleistift, lateinische Kurrent (\noindent{}Adresse)\hspace{1em}
\newline{}Versand: 1) Stempel: »\nobreak{}\oindex{Stazione di Venezia Santa Lucia@\textbf{Stazione di Venezia Santa Lucia}, \emph{Bahnhofsgebäude (K.BHF)}|pwk}Venezia Ferrovia, {[}4. 7.{]} 06, 2S\nobreak{}«.   2) Stempel: »\nobreak{}\oindex{XVIII., Waehring@\textbf{XVIII., Währing}, \emph{A.ADM3}|pwk}18/1 Wien, 6. VII. 06, Bestellt\nobreak{}«. 
\newline{}Ordnung: mit Bleistift von unbekannter Hand nummeriert:
                                    »140« }
\buchAbdrucke{\weitereDrucke{Hermann Bahr, Arthur Schnitzler: \emph{Briefwechsel, Aufzeichnungen, Dokumente (1891–1931)}. Göttingen: \emph{Wallstein} 2018, S. 380.} }\toendnotes[C]{\smallbreak}\pstart{}{\pb}D\textsuperscript{r} Artur
                  Schnitzler\pend{}\pstart{}XVIII Spöttelgasse 7\oindex{XVIII., Waehring@\textbf{XVIII., Währing}, \emph{A.ADM3}|pw}\pend{}\pstart{}Wien\oindex{Wien@\textbf{Wien}, \emph{A.ADM2}|pw}\pend{}\pstart{}Austria\oindex{Oesterreich@\textbf{Österreich}, \emph{A.PCLI}|pw}\pend{}{\bigskip}\vspace{1em}
\pstart
           \raggedleft{}{\pb}\textsc{Venezia}\oindex{Venedig@\textbf{Venedig}, \emph{P.PPLA}|pw}{ }4. 7. 06{\\}\textsc{Casa\oindex{Casa Petrarca@\textbf{Casa Petrarca}, \emph{Gebäude (K.GBD)}|pw}{ }Petrarca\pwindex{Petrarca, Francesco 1304-07-19 – 1374@\textsc{Petrarca, Francesco} (1304-07-19 – 1374), \emph{Schriftsteller/Schriftstellerin}|pwv}}\pend
           \vspace{0.5em}
\pstart
           Dank ſchön, lieber Artur. Dein Brief hat mir eine große Freude
               gemacht, und Luſt, ſolchen zweiten und dritten Akt\pwindex{Faun. Ein Akt@\emph{Der Faun. Ein Akt}|pwv} wirklich zu ſchreiben. Neugierig, was Brahm\pwindex{Brahm, Otto 05.02.1856 – 28.11.1912@\textsc{Brahm, Otto} (05.02.1856 – 28.11.1912), \emph{Theaterleiter/Theaterleiterin, Regisseur/Regisseurin}|pw}{ }ſagen wird. – Hier herrlichſt, obwol mir die Sonne
               die Beine ſo verbrannt hat, daß ſie zwei Tage in Bleiwaſſer gelegt werden mußten. –
               Grüß Frau Olga\pwindex{Schnitzler, Olga 17.01.1882 – 13.01.1970@\textsc{Schnitzler, Olga} (17.01.1882 – 13.01.1970), \emph{Schauspieler/Schauspielerin, Sänger/Sängerin}|pw} herzlichſt und laßt es Euch gut
               gehen und ſchreib Deine Adreſſe\pend
           
\pstart
           Deinem alten{\\[\baselineskip]}\spacefill\mbox{Hermann{ }Faun\pwindex{Faun. Ein Akt@\emph{Der Faun. Ein Akt}|pwv}}\pend
           \leftskip=0em{}\selectlanguage{ngerman}\endnumbering\briefempfaengerindex{Schnitzler, Arthur@\textsc{Schnitzler, Arthur}!zzzBahr, Hermann@\emph{von Hermann Bahr}!1906-07-041@{4. 7. 1906}|)be}\mylabel{L01607h}  \normalsize

\doendnotes{C}
\bigskip
\vfill

\clearpage

\footnotesize

\lohead{\textsc{register}}

% Definiere theindex-Environment komplett neu ohne reledmac
\makeatletter
\renewenvironment{theindex}{%
  \section*{\indexname}%
  \setlength{\parindent}{0pt}%
  \setlength{\parskip}{0pt plus 0.3pt}%
  \let\item\@idxitem
}{%
  \clearpage
}
\makeatother

\IfFileExists{\jobname-pw.ind}{\input{\jobname-pw.ind}}{}

\end{document}

      