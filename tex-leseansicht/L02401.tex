%% latex-korrekturansicht-vorspann.tex
%% Vorspann für die Korrekturansicht.
%% Lädt die gemeinsame Datei latex-vorspann.tex mit gesetztem Schalter.

\newif\ifkorrekturansicht
\korrekturansichttrue

\input{../tex-inputs/latex-vorspann}


\section[Georg Brandes an Arthur Schnitzler, 11. 6. 1923]{L02401 Georg Brandes an Arthur Schnitzler, 11. 6. 1923}
\nopagebreak\mylabel{L02401v}
\rehead{ }\normalsize\beginnumbering\briefempfaengerindex{Schnitzler, Arthur@\textsc{Schnitzler, Arthur}!zzzBrandes, Georg@\emph{von Georg Brandes}!1923-06-111@{11. 6. 1923}|(be}
\toendnotes[C]{\smallbreak\pagebreak[2]}\Standort{CUL, Schnitzler, B 17.}
\physDesc{Postkarte, 1185 Zeichen
\newline{}Handschrift: schwarze Tinte, lateinische Kurrent
\newline{}Versand: Stempel: »\nobreak{}\oindex{Hornbæk@\textbf{Hornbæk}, \emph{P.PPL}|pwk}Hornbæk, 11. 6. 23, 6–8 E\nobreak{}«.  
\newline{}Schnitzler: 1) Markierung  (?) mit Bleistift: »\textsc{\uline{A}}« (für: Abgeschrieben?)  2) mit rotem Buntstift zwei Unterstreichungen
\newline{}Ordnung: mit Bleistift von unbekannter Hand nummeriert:
                                    »54« }
\buchAbdrucke{\weitereDrucke{Georg Brandes, Arthur Schnitzler: \emph{Ein Briefwechsel}. Bern: \emph{Francke} 1956, S. 138–139.} }\toendnotes[C]{\smallbreak}\pstart{}{\pb}Herrn Dr. Arthur
                  Schnitzler\pend{}\pstart{}Sternwartestrasse 71\oindex{Sternwartestrasse 71@\textbf{Sternwartestraße 71}, \emph{Wohngebäude (K.WHS)}|pw}\pend{}\pstart{}Wien XVIII\oindex{XVIII., Waehring@\textbf{XVIII., Währing}, \emph{A.ADM3}|pw}\pend{}{\bigskip}\vspace{1em}
\pstart
           \raggedleft{}{\pb}Kopenhagen\oindex{Kopenhagen@\textbf{Kopenhagen}, \emph{P.PPLC}|pw}{ }11 Juni 23\pend
           \vspace{0.5em}
\pstart
           Liebster Schnitzler\hspace*{3.5em}Seien Sie bedankt für die Güte, die Sie nicht
               weniger als drei mal einen Patienten aufsuchen lies. Ich war und bin Ihnen von ganzem
               Herzen dankbar. Ich hoffe dass Sie in Stockholm\oindex{Stockholm@\textbf{Stockholm}, \emph{P.PPLC}|pw}
               gute Erfahrungen machte{[}n{]}. Ich habe leider keine schwedische\oindex{Schweden@\textbf{Schweden}, \emph{A.PCLI}|pw} Zeitung gesehen. Ich habe den
               Wunsch, dass es Ihnen in der hübschen Stadt\oindex{Stockholm@\textbf{Stockholm}, \emph{P.PPLC}|pwv} gut ging und dass Sie was verdienten. Die schwedische\oindex{Schweden@\textbf{Schweden}, \emph{A.PCLI}|pw} Krone ist viel mehr werth als die
                  dänische\oindex{Daenemark@\textbf{Dänemark}, \emph{A.PCLI}|pw}.\pend
           
\pstart
           Ich bin augenblicklich auf dem Lande (Hornbæk\oindex{Hornbæk@\textbf{Hornbæk}, \emph{P.PPL}|pw},
                  Villa Iris\oindex{Villa Iris@\textbf{Villa Iris}, \emph{Wohngebäude (K.WHS)}|pw}) um mich zu erholen, und es geht
               mir sehr gut, wäre nur nicht der Sommer so schlecht, das Wetter so kalt und
               regnerisch. Ich habe recht viel gearbeitet, gebe die 6\textsuperscript{te}
               Ausgabe meiner alten vor halbhundert Jahren geschriebenen Hauptströmungen\pwindex{Hauptstroemungen der Literatur des neunzehnten Jahrhunderts@\emph{Hauptströmungen der Literatur des neunzehnten Jahrhunderts}|pw} heraus, in vermehrter und verbesserter Gestalt,
               merze {\pb}Irrthümer aus und füge
               Binsenwahrheiten hinzu.\pend
           
\pstart
           Es war eine wahre Freude für mich, Sie wiederzusehen, anscheinend unangefochten von
               all dem Ungemach, das sich über Ihr Land\oindex{Oesterreich@\textbf{Österreich}, \emph{A.PCLI}|pwv} wie über ganz Europa\oindex{Europa@\textbf{Europa}, \emph{Kontinent (A.KNT)}|pw} gestürzt hat. Sie haben augenscheinlich nicht weniger
               Widerstandskraft als Ihr jugendlicher Verehrer\pend
           \pstart \spacefill\mbox{G. B.}\pend{}
\pstart
           \noindent{}\label{T_L02401-1v}\edtext{Grüssen Sie den Sohn\pwindex{Schnitzler, Heinrich 09.08.1902 – 12.07.1982@\textsc{Schnitzler, Heinrich} (09.08.1902 – 12.07.1982), \emph{Regisseur/Regisseurin, Schauspieler/Schauspielerin}|pwv}, von dem Sie mir sprachen}{\lemma{\textnormal{\emph{Grüssen … sprachen}}}\Cendnote{\textnormal{am linken Rand}}}\label{T_L02401-1}\pend
           \selectlanguage{ngerman}\endnumbering\briefempfaengerindex{Schnitzler, Arthur@\textsc{Schnitzler, Arthur}!zzzBrandes, Georg@\emph{von Georg Brandes}!1923-06-111@{11. 6. 1923}|)be}\mylabel{L02401h}  \normalsize

\doendnotes{C}
\bigskip
\vfill

\clearpage

\footnotesize

\lohead{\textsc{register}}

% Definiere theindex-Environment komplett neu ohne reledmac
\makeatletter
\renewenvironment{theindex}{%
  \section*{\indexname}%
  \setlength{\parindent}{0pt}%
  \setlength{\parskip}{0pt plus 0.3pt}%
  \let\item\@idxitem
}{%
  \clearpage
}
\makeatother

\IfFileExists{\jobname-pw.ind}{\input{\jobname-pw.ind}}{}

\end{document}

      