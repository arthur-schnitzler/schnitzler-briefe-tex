%% latex-leseansicht-vorspann.tex
%% Vorspann für die Leseansicht.
%% Lädt die gemeinsame Datei latex-vorspann.tex mit nicht gesetztem Schalter.

\newif\ifkorrekturansicht
\korrekturansichtfalse

\input{../tex-inputs/latex-vorspann}


         
         \renewcommand{\erwaehntePersonen}{Personen: Heinrich Schnitzler}
         \renewcommand{\erwaehnteOrte}{Orte: Dänemark, Europa, Hornbæk, Kopenhagen, Schweden, Sternwartestraße, Stockholm, Villa Iris, Wien, XVIII., Währing, Österreich}
         \renewcommand{\erwaehnteWerke}{Werke: Hauptströmungen der Literatur des neunzehnten Jahrhunderts}
               \section[Georg Brandes an Arthur Schnitzler, 11. 6. 1923]{ Georg Brandes an Arthur Schnitzler, 11. 6. 1923}\nopagebreak\mylabel{v}\rehead{ }\begin{ledgroupsized}[t]{13cm}\normalsize\beginnumbering \toendnotes[C]{\smallbreak\pagebreak[2]} \Standort{CUL, Schnitzler, B 17.}
\physDesc{Postkarte, 1185 Zeichen
\newline{}Handschrift: schwarze Tinte, lateinische Kurrent
\newline{}Versand: Stempel: »\nobreak{}\oindex{Hornbæk@\textbf{Hornbæk}|pwk}Hornbæk, 11. 6. 23, 6–8 E\nobreak{}«.  
\newline{}Schnitzler: 1) Markierung  (?) mit Bleistift: »\textsc{\uline{A}}« (für: Abgeschrieben?)  2) mit rotem Buntstift zwei Unterstreichungen
\newline{}Ordnung: mit Bleistift von unbekannter Hand nummeriert:
                                    »54« }\buchAbdrucke{\weitereDrucke{Georg Brandes, Arthur Schnitzler: \emph{Ein Briefwechsel}. Hg. Kurt Bergel. Bern: \emph{Francke} 1956, S. 138–139.} }\toendnotes[C]{\smallbreak}\pstart{}{\pb}Herrn Dr. Arthur
                  Schnitzler\pend{}\pstart{}Sternwartestrasse 71\oindex{XXXX Ortsangabe fehlt|pw}\pend{}\pstart{}Wien XVIII\oindex{XVIII., Waehring@\textbf{XVIII., Währing}|pw}\pend{}{\bigskip}\pstart
           \raggedleft{}{\pb}Kopenhagen\oindex{Kopenhagen@\textbf{Kopenhagen}|pw}{ }11 Juni 23\pend
           \pstart
           Liebster Schnitzler\hspace*{3.5em}Seien Sie bedankt für die Güte, die Sie nicht
               weniger als drei mal einen Patienten aufsuchen lies. Ich war und bin Ihnen von ganzem
               Herzen dankbar. Ich hoffe dass Sie in Stockholm\oindex{Stockholm@\textbf{Stockholm}|pw}
               gute Erfahrungen machte{[}n{]}. Ich habe leider keine schwedische\oindex{Schweden@\textbf{Schweden}|pw} Zeitung gesehen. Ich habe den
               Wunsch, dass es Ihnen in der hübschen Stadt\oindex{Stockholm@\textbf{Stockholm}|pwv} gut ging und dass Sie was verdienten. Die schwedische\oindex{Schweden@\textbf{Schweden}|pw} Krone ist viel mehr werth als die
                  dänische\oindex{Daenemark@\textbf{Dänemark}|pw}.\pend
           \pstart
           Ich bin augenblicklich auf dem Lande (Hornbæk\oindex{Hornbæk@\textbf{Hornbæk}|pw},
                  Villa Iris\oindex{Villa Iris@\textbf{Villa Iris}|pw}) um mich zu erholen, und es geht
               mir sehr gut, wäre nur nicht der Sommer so schlecht, das Wetter so kalt und
               regnerisch. Ich habe recht viel gearbeitet, gebe die 6\textsuperscript{te}
               Ausgabe meiner alten vor halbhundert Jahren geschriebenen Hauptströmungen\pwindex{Brandes, Georg 04.02.1842 – 19.02.1927@\textsc{Brandes, Georg} (04.02.1842 – 19.02.1927)!Hauptstroemungen der Literatur des neunzehnten Jahrhunderts1872@\strich\emph{Hauptströmungen der Literatur des neunzehnten Jahrhunderts} {[}1872{]}|pw} heraus, in vermehrter und verbesserter Gestalt,
               merze {\pb}Irrthümer aus und füge
               Binsenwahrheiten hinzu.\pend
           \pstart
           Es war eine wahre Freude für mich, Sie wiederzusehen, anscheinend unangefochten von
               all dem Ungemach, das sich über Ihr Land\oindex{Oesterreich@\textbf{Österreich}|pwv} wie über ganz Europa\oindex{Europa@\textbf{Europa}|pw} gestürzt hat. Sie haben augenscheinlich nicht weniger
               Widerstandskraft als Ihr jugendlicher Verehrer\pend
           \pstart \spacefill\mbox{G. B.}\pend{}\pstart
           \noindent{}\label{T_L02401-1v}\edtext{Grüssen Sie den Sohn\pwindex{Schnitzler, Heinrich 09.08.1902 – 12.07.1982@\textsc{Schnitzler, Heinrich} (09.08.1902 – 12.07.1982), \emph{Regisseur, Schauspieler}|pwv}, von dem Sie mir sprachen}{\lemma{\textnormal{\emph{Grüssen … sprachen}}}\Cendnote{\textnormal{am linken Rand}}}\label{T_L02401-1h}\pend
           
         
         \endnumbering\mylabel{h}\end{ledgroupsized}  \newcommand{\dateiname}{L02401}\newcommand{\titel}{Georg Brandes an Arthur Schnitzler, 11. 6. 1923}\newcommand{\editorInnen}{Martin Anton Müller und Gerd-Hermann Susen}%% latex-leseansicht-abspann.tex
%% Abspann für die Leseansicht.
%% Der Schalter \ifkorrekturansicht ist bereits durch den Vorspann gesetzt.

%% latex-abspann.tex
%% Gemeinsamer Abspann für Korrekturansicht und Leseansicht.
%% Setzt den Schalter \ifkorrekturansicht voraus (gesetzt in den
%% einbindenden Dateien latex-korrekturansicht-abspann.tex bzw.
%% latex-leseansicht-abspann.tex).
%% ---------------------------------------------------------------

\normalsize

% Das esempio-Environment wird nur in der Leseansicht benötigt
\ifkorrekturansicht\else
\newenvironment{esempio}[3]%
{
    \vspace{1.5ex}
    \rlap{\underline{#1}}
    \par
    \setlength{\parindent}{0cm}
    \nopagebreak
    \leftskip=#2cm
    \rightskip=#3cm
}
{
    \par
}
\fi

\doendnotes{C}
\bigskip
\vfill

\clearpage

\footnotesize

\ifkorrekturansicht
  \lohead{\textsc{register}}
\fi

% theindex-Environment neu definieren ohne reledmac
\makeatletter
\renewenvironment{theindex}{%
  \ifkorrekturansicht
    \section*{\indexname}%
  \else
    \subsubsection*{Index der erwähnten Entitäten}%
  \fi
  \setlength{\parindent}{0pt}%
  \setlength{\parskip}{0pt plus 0.3pt}%
  \let\item\@idxitem
}{%
  \ifkorrekturansicht\clearpage\fi
}
\makeatother

\IfFileExists{\jobname-pw.ind}{\input{\jobname-pw.ind}}{}

% Quellenangabe nur in der Leseansicht
\ifkorrekturansicht\else
% Fallback-Definitionen, falls die .tex-Datei \titel etc. nicht gesetzt hat
\providecommand{\titel}{}
\providecommand{\editorInnen}{}
\providecommand{\dateiname}{\jobname}

\vspace{3cm}

\vfill

\footnotesize
\textsc{Quelle}: \titel. Herausgegeben von {\editorInnen}. In: \emph{Arthur Schnitzler: Briefwechsel mit Autorinnen und Autoren}.
 Digitale Edition, https://schnitzler-briefe.acdh.oeaw.ac.at/{\dateiname}.html (Stand \today)
\fi

\end{document}


      