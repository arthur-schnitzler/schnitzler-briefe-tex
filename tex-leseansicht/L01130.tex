%% latex-leseansicht-vorspann.tex
%% Vorspann für die Leseansicht.
%% Lädt die gemeinsame Datei latex-vorspann.tex mit nicht gesetztem Schalter.

\newif\ifkorrekturansicht
\korrekturansichtfalse

\input{../tex-inputs/latex-vorspann}

\begin{center}
            \textcolor{red}{ENTWURF. ENTZIFFERUNG NOCH NICHT KORREKTURGELESEN}
                      \end{center}
            
               \section[Ferdinand von Saar an Arthur Schnitzler, 19. 6. 1901]{ Ferdinand von Saar an Arthur Schnitzler, 19. 6. 1901}\nopagebreak\mylabel{v}\rehead{ }\begin{ledgroupsized}[t]{13cm}\normalsize\beginnumbering\briefempfaengerindex{Schnitzler, Arthur@\textsc{Schnitzler, Arthur}!zzzSaar, Ferdinand von@\emph{von Ferdinand von Saar}!1901-06-191@{19. 6. 1901}|(be} \toendnotes[C]{\smallbreak\pagebreak[2]} \Standort{CUL, Schnitzler, B 88.}
\physDesc{Brief, 1 Blatt, 2 Seiten
\newline{}Handschrift: schwarze Tinte, deutsche Kurrent
\newline{}Schnitzler: mit Bleistift nummeriert: »9« }\toendnotes[C]{\smallbreak}\pstart
           \raggedleft{}{\pb}\textsc{Wien-Döbling}\oindex{XIX., Doebling@\textbf{XIX., Döbling}|pw}, 19/6. 1901.\pend
           \pstart{}Sehr verehrter Herr Doctor!\pend\pstart
           Ihre neueſten Bücher\pwindex{Schnitzler, Arthur 15.05.1862 – 21.10.1931@\textsc{Schnitzler, Arthur} (15.05.1862 – 21.10.1931), \emph{Schriftsteller, Mediziner}!Lieutenant Gustl. Novelle25. 12. 1900@\strich\emph{Lieutenant Gustl. Novelle} {[}25. 12. 1900{]}|pwv}\pwindex{Schnitzler, Arthur 15.05.1862 – 21.10.1931@\textsc{Schnitzler, Arthur} (15.05.1862 – 21.10.1931), \emph{Schriftsteller, Mediziner}!Frau Bertha Garlan. Roman15.1.1901 – 15.3.1901@\strich\emph{Frau Bertha Garlan. Roman} {[}15.1.1901 – 15.3.1901{]}|pwv} habe
               ich mit großer Aufmerkſamkeit geleſen, habe ſie in mir nachwirken laſſen – und ſo
               gelange ich erſt heute dazu, Ihnen für die ſo freundliche Überſendung zu danken. An
               beiden habe ich wieder Ihre bewährte Kraft der Seelenanalyſe und Milieuſchilderung
               bewundert. »Lieutenant Guſtl\pwindex{Schnitzler, Arthur 15.05.1862 – 21.10.1931@\textsc{Schnitzler, Arthur} (15.05.1862 – 21.10.1931), \emph{Schriftsteller, Mediziner}!Lieutenant Gustl. Novelle25. 12. 1900@\strich\emph{Lieutenant Gustl. Novelle} {[}25. 12. 1900{]}|pw}« iſt freilich mehr ein
               Virtuoſenſtück; hingegen erſcheint aber »Frau Bertha
                  Garlan\pwindex{Schnitzler, Arthur 15.05.1862 – 21.10.1931@\textsc{Schnitzler, Arthur} (15.05.1862 – 21.10.1931), \emph{Schriftsteller, Mediziner}!Frau Bertha Garlan. Roman15.1.1901 – 15.3.1901@\strich\emph{Frau Bertha Garlan. Roman} {[}15.1.1901 – 15.3.1901{]}|pw}« als ein umſo echteres Kunſtwerk. Man athmet die Luft der kleinen
               Landſtadt und lebt die öden, gedrückten Verhältniſſe mit, als befände man ſich dort.
               Daher kommt es auch, dſs man ſich ungefähr in der Mitte des Buches\pwindex{Schnitzler, Arthur 15.05.1862 – 21.10.1931@\textsc{Schnitzler, Arthur} (15.05.1862 – 21.10.1931), \emph{Schriftsteller, Mediziner}!Frau Bertha Garlan. Roman15.1.1901 – 15.3.1901@\strich\emph{Frau Bertha Garlan. Roman} {[}15.1.1901 – 15.3.1901{]}|pwv} fragt, ob dieſe Zuſtände ſo eingehender
               Behandlung auch wirklich werth ſeien – und man fängt an, ein bißchen ungeduldig zu
               werden. Aber die zweite Hälfte wirkt mit dem ergreifenden Schluß nach rückwärts wie
               ein mächtiger elektriſcher Lichtſtrom, der allein und vor allem der Heldin vollen
               Reiz und volle Bedeu{\pb}tung verleiht. Jeder Zug in dieſem ſtillen, ſtill
               verlangenden und eigentlich nichts erlebenden Frauenleben wird als nothwendig
               empfunden, prägt ſich tief ein, und ſo wird »Frau Bertha
                  Garlan\pwindex{Schnitzler, Arthur 15.05.1862 – 21.10.1931@\textsc{Schnitzler, Arthur} (15.05.1862 – 21.10.1931), \emph{Schriftsteller, Mediziner}!Frau Bertha Garlan. Roman15.1.1901 – 15.3.1901@\strich\emph{Frau Bertha Garlan. Roman} {[}15.1.1901 – 15.3.1901{]}|pw}« zu den Büchern gehören, die man niemals aus dem Gedächtniſſe
               verliert. Man hat ſie, wenn ich nicht irre, zu \label{K_L01130_1v}\edtext{Madame Bovary\pwindex{\textcolor{red}{\textsuperscript{XXXX1 indx}}!Madame Bovary. Mœurs de province1857@\strich\emph{Madame Bovary. Mœurs de province} {[}1857{]}|pw} in Beziehung}{\lemma{\textnormal{\emph{Madame … Beziehung}}}\Cendnote{\textnormal{Auf \emph{Madame
                     Bovary}\pwindex{\textcolor{red}{\textsuperscript{XXXX1 indx}}!Madame Bovary. Mœurs de province1857@\strich\emph{Madame Bovary. Mœurs de province} {[}1857{]}|pwk} – Schnitzler\pwindex{Schnitzler, Arthur 15.05.1862 – 21.10.1931@\textsc{Schnitzler, Arthur} (15.05.1862 – 21.10.1931), \emph{Schriftsteller, Mediziner}|pwk} hatte den Roman mit
                  achtzehn Jahren gelesen (siehe A. S.: \emph{Tagebuch}, 14. 5. 1880) – als literarische »Vorlage« verweisen viele Rezensenten der
                  Novelle, vgl. z. B. Alfred Gold\pwindex{Gold, Alfred 28.06.1874 – 24.10.1958@\textsc{Gold, Alfred} (28.06.1874 – 24.10.1958), \emph{Schriftsteller, Journalist, Kunsthändler}|pwk}: \emph{Arthur Schnitzler: Frau Bertha Garlan}\pwindex{Gold, Alfred 28.06.1874 – 24.10.1958@\textsc{Gold, Alfred} (28.06.1874 – 24.10.1958), \emph{Schriftsteller, Journalist, Kunsthändler}!Arthur Schnitzler: Frau Bertha Garlan1901-05-04 – 1901-05-04@\strich\emph{Arthur Schnitzler: Frau Bertha Garlan} {[}1901-05-04 – 1901-05-04{]}|pwk}. In: \emph{Die Zeit}\pwindex{Zeit. Wiener Wochenschrift1894 – 1904@\emph{Die Zeit. Wiener Wochenschrift}|pwk}, Nr. 344, 4. 5. 1901, S. 78
                  und [Joseph Victor Widmann\pwindex{Widmann, Joseph Victor 20.02.1842 – 06.11.1911@\textsc{Widmann, Joseph Victor} (20.02.1842 – 06.11.1911), \emph{Schriftsteller, Journalist}|pwk}?]: \emph{Kunst und Litteratur. Frau Bertha Garlan}\pwindex{Kunst und Litteratur. Frau Bertha Garlan1901-05-05@\emph{Kunst und Litteratur. Frau Bertha Garlan} {[}1901-05-05{]}|pwk}. In:
                        \emph{Sonntagsblatt des Bund}\pwindex{Sonntagsblatt des Bund@\emph{Sonntagsblatt des Bund}|pwk}, Nr. 18,
                        5. 5. 1901, S. 141–142.}}}\label{K_L01130_1h} bringen wollen. Höchſt
               ungerechtfertigt! Denn es ist \uline{alles ganz} anders. Die
               einzige Ähnlichkeit, die man aber an den Haaren herbeiziehen müßte, beſteht darin:
               dſs beide Romane in der Provinz ſpielen. Aber ſo ſind die Menſchen: ſie können eben
               immer nur vergleichen! \pend
           \pstart
           Indem ich mich Ihnen mit wahrer Hochachtung empfehle, bin ich{\\[\baselineskip]}Ihr alt ergebener{\\[\baselineskip]}\spacefill\mbox{Ferdinand von Saar.}\pend
           \leftskip=0em{}\endnumbering\briefempfaengerindex{Schnitzler, Arthur@\textsc{Schnitzler, Arthur}!zzzSaar, Ferdinand von@\emph{von Ferdinand von Saar}!1901-06-191@{19. 6. 1901}|)be}\mylabel{h}\end{ledgroupsized}  \newcommand{\dateiname}{L01130}\newcommand{\titel}{Ferdinand von Saar an Arthur Schnitzler, 19. 6. 1901}\newcommand{\editorInnen}{Martin Anton Müller und Gerd-Hermann Susen}%% latex-leseansicht-abspann.tex
%% Abspann für die Leseansicht.
%% Der Schalter \ifkorrekturansicht ist bereits durch den Vorspann gesetzt.

%% latex-abspann.tex
%% Gemeinsamer Abspann für Korrekturansicht und Leseansicht.
%% Setzt den Schalter \ifkorrekturansicht voraus (gesetzt in den
%% einbindenden Dateien latex-korrekturansicht-abspann.tex bzw.
%% latex-leseansicht-abspann.tex).
%% ---------------------------------------------------------------

\normalsize

% Das esempio-Environment wird nur in der Leseansicht benötigt
\ifkorrekturansicht\else
\newenvironment{esempio}[3]%
{
    \vspace{1.5ex}
    \rlap{\underline{#1}}
    \par
    \setlength{\parindent}{0cm}
    \nopagebreak
    \leftskip=#2cm
    \rightskip=#3cm
}
{
    \par
}
\fi

\doendnotes{C}
\bigskip
\vfill

\clearpage

\footnotesize

\ifkorrekturansicht
  \lohead{\textsc{register}}
\fi

% theindex-Environment neu definieren ohne reledmac
\makeatletter
\renewenvironment{theindex}{%
  \ifkorrekturansicht
    \section*{\indexname}%
  \else
    \subsubsection*{Index der erwähnten Entitäten}%
  \fi
  \setlength{\parindent}{0pt}%
  \setlength{\parskip}{0pt plus 0.3pt}%
  \let\item\@idxitem
}{%
  \ifkorrekturansicht\clearpage\fi
}
\makeatother

\IfFileExists{\jobname-pw.ind}{\input{\jobname-pw.ind}}{}

% Quellenangabe nur in der Leseansicht
\ifkorrekturansicht\else
% Fallback-Definitionen, falls die .tex-Datei \titel etc. nicht gesetzt hat
\providecommand{\titel}{}
\providecommand{\editorInnen}{}
\providecommand{\dateiname}{\jobname}

\vspace{3cm}

\vfill

\footnotesize
\textsc{Quelle}: \titel. Herausgegeben von {\editorInnen}. In: \emph{Arthur Schnitzler: Briefwechsel mit Autorinnen und Autoren}.
 Digitale Edition, https://schnitzler-briefe.acdh.oeaw.ac.at/{\dateiname}.html (Stand \today)
\fi

\end{document}


      