%% latex-leseansicht-vorspann.tex
%% Vorspann für die Leseansicht.
%% Lädt die gemeinsame Datei latex-vorspann.tex mit nicht gesetztem Schalter.

\newif\ifkorrekturansicht
\korrekturansichtfalse

\input{../tex-inputs/latex-vorspann}


\section[Ferdinand von Saar an Arthur Schnitzler, 19. 6. 1901]{L01130 Ferdinand von Saar an Arthur Schnitzler, 19. 6. 1901}
\nopagebreak\mylabel{L01130v}
\rehead{ }\normalsize\beginnumbering\briefempfaengerindex{Schnitzler, Arthur@\textsc{Schnitzler, Arthur}!zzzSaar, Ferdinand von@\emph{von Ferdinand von Saar}!1901-06-191@{19. 6. 1901}|(be}
\toendnotes[C]{\smallbreak\pagebreak[2]}
\correspDesc{Versand  durch Ferdindand von Saar am 19. 6. 1901 in Wien
\newline{}Erhalt  durch Arthur Schnitzler im Zeitraum [19. 6. 1901
                  – 23. 6. 1901?] in Wien}\toendnotes[C]{\smallbreak}
\Standort{CUL, Schnitzler, B 88.}
\physDesc{Brief, 1 Blatt, 2 Seiten, 1646 Zeichen
\newline{}Handschrift: schwarze Tinte, deutsche Kurrent
\newline{}Schnitzler: mit Bleistift nummeriert: »9« }\toendnotes[C]{\smallbreak}
\pstart
           \raggedleft{}{\pb}\textsc{Wien-Döbling}\oindex{XIX., Döbling@\textbf{XIX., Döbling}, \emph{Verwaltungsgebiet}|pw}, 19/6. 1901.\pend
           
\pstart{}Sehr verehrter Herr Doctor!\pend\vspace{0.5em}
\pstart
           Ihre neueſten Bücher\pwindex{Schnitzler, Arthur 15.\,5.\,1862 Wien – 21.\,10.\,1931 ebd.@\textsc{Schnitzler, Arthur} (15.\,5.\,1862 Wien – 21.\,10.\,1931 ebd.), \emph{Schriftsteller, Mediziner}!Lieutenant Gustl. Novelle@\strich\emph{Lieutenant Gustl. Novelle}|pwv}\pwindex{Schnitzler, Arthur 15.\,5.\,1862 Wien – 21.\,10.\,1931 ebd.@\textsc{Schnitzler, Arthur} (15.\,5.\,1862 Wien – 21.\,10.\,1931 ebd.), \emph{Schriftsteller, Mediziner}!Frau Bertha Garlan. Roman@\strich\emph{Frau Bertha Garlan. Roman}|pwv}
               habe ich mit großer Aufmerkſamkeit geleſen, habe{ }ſie in mir nachwirken laſſen – und{ }ſo gelange ich erſt heute dazu, Ihnen für die{ }ſo freundliche Überſendung zu danken.
               An beiden habe ich wieder Ihre bewährte Kraft der Seelenanalyſe und Milieuſchilderung
               bewundert. »Lieutenant Guſtl\pwindex{Schnitzler, Arthur 15.\,5.\,1862 Wien – 21.\,10.\,1931 ebd.@\textsc{Schnitzler, Arthur} (15.\,5.\,1862 Wien – 21.\,10.\,1931 ebd.), \emph{Schriftsteller, Mediziner}!Lieutenant Gustl. Novelle@\strich\emph{Lieutenant Gustl. Novelle}|pw}« iſt freilich mehr
               ein Virtuoſenſtück; hingegen erſcheint aber »Frau
                  Bertha Garlan\pwindex{Schnitzler, Arthur 15.\,5.\,1862 Wien – 21.\,10.\,1931 ebd.@\textsc{Schnitzler, Arthur} (15.\,5.\,1862 Wien – 21.\,10.\,1931 ebd.), \emph{Schriftsteller, Mediziner}!Frau Bertha Garlan. Roman@\strich\emph{Frau Bertha Garlan. Roman}|pw}« als ein umſo echteres Kunſtwerk. Man athmet die Luft der
               kleinen Landſtadt und lebt die öden, gedrückten Verhältniſſe mit, als befände man{ }ſich dort. Daher kommt es auch, dſs man{ }ſich ungefähr in der Mitte des Buches\pwindex{Schnitzler, Arthur 15.\,5.\,1862 Wien – 21.\,10.\,1931 ebd.@\textsc{Schnitzler, Arthur} (15.\,5.\,1862 Wien – 21.\,10.\,1931 ebd.), \emph{Schriftsteller, Mediziner}!Frau Bertha Garlan. Roman@\strich\emph{Frau Bertha Garlan. Roman}|pwv} fragt, ob dieſe Zuſtände{ }ſo eingehender Behandlung auch wirklich werth{ }ſeien – und man fängt an, ein bißchen
               ungeduldig zu werden. Aber die zweite Hälfte wirkt mit dem ergreifenden Schluß nach
               rückwärts wie ein mächtiger elektriſcher Lichtſtrom, der allein und vor allem der
               Heldin vollen Reiz und volle Bedeu{\pb}tung verleiht. Jeder Zug in dieſem{ }ſtillen,{ }ſtill verlangenden und eigentlich nichts erlebenden Frauenleben wird als
               nothwendig empfunden, prägt{ }ſich tief ein, und{ }ſo wird »Frau Bertha Garlan\pwindex{Schnitzler, Arthur 15.\,5.\,1862 Wien – 21.\,10.\,1931 ebd.@\textsc{Schnitzler, Arthur} (15.\,5.\,1862 Wien – 21.\,10.\,1931 ebd.), \emph{Schriftsteller, Mediziner}!Frau Bertha Garlan. Roman@\strich\emph{Frau Bertha Garlan. Roman}|pw}« zu den Büchern gehören, die man niemals
               aus dem Gedächtniſſe verliert. Man hat{ }ſie, wenn ich nicht irre, zu \label{K_L01130-1v}\edtext{Madame Bovary\pwindex{\textcolor{red}{\textsuperscript{XXXX indx1}}!Madame Bovary. Mœurs de province@\strich\emph{Madame Bovary. Mœurs de province}|pw} in Beziehung}{\lemma{\textnormal{\emph{Madame … Beziehung}}}\Cendnote{\textnormal{Auf \emph{Madame
                     Bovary}\pwindex{\textcolor{red}{\textsuperscript{XXXX indx1}}!Madame Bovary. Mœurs de province@\strich\emph{Madame Bovary. Mœurs de province}|pwk} – Schnitzler hatte den Roman
                  mit achtzehn Jahren gelesen (siehe A. S.: \emph{Tagebuch}, 14. 5. 1880) – als literarische »Vorlage« verweisen viele Rezensenten der
                  Novelle,  vgl. z. B. 
                     Alfred Gold\pwindex{Gold, Alfred 28.\,6.\,1874 Wien – 24.\,10.\,1958 New York City@\textsc{Gold, Alfred} (28.\,6.\,1874 Wien – 24.\,10.\,1958 New York City), \emph{Schriftsteller, Journalist, Kunsthändler}|pwk}: \emph{Arthur Schnitzler: Frau Bertha Garlan}\pwindex{Gold, Alfred 28.\,6.\,1874 Wien – 24.\,10.\,1958 New York City@\textsc{Gold, Alfred} (28.\,6.\,1874 Wien – 24.\,10.\,1958 New York City), \emph{Schriftsteller, Journalist, Kunsthändler}!Arthur Schnitzler: Frau Bertha Garlan@\strich\emph{Arthur Schnitzler: Frau Bertha Garlan}|pwk}. In: \emph{Die Zeit}\pwindex{Zeit. Wiener Wochenschrift@\emph{Die Zeit. Wiener Wochenschrift}|pwk}, Nr. 344, 4. 5. 1901,
                     S. 78 und [Joseph Victor
                        Widmann\pwindex{Widmann, Joseph Victor 20.\,2.\,1842 Brněnské Ivanovice – 6.\,11.\,1911 Bern@\textsc{Widmann, Joseph Victor} (20.\,2.\,1842 Brněnské Ivanovice – 6.\,11.\,1911 Bern), \emph{Schriftsteller, Journalist}|pwk}?]: \emph{Kunst und Litteratur. Frau
                        Bertha Garlan}\pwindex{Kunst und Litteratur. Frau Bertha Garlan@\emph{Kunst und Litteratur. Frau Bertha Garlan}|pwk}. In: \emph{Sonntagsblatt des
                        Bund}\pwindex{Sonntagsblatt des Bund@\emph{Sonntagsblatt des Bund}|pwk}, Nr. 18, 5. 5. 1901, S. 141–142.}}}\label{K_L01130-1}
               bringen wollen. Höchſt ungerechtfertigt! Denn es ist \uline{alles
                  ganz} anders. Die einzige Ähnlichkeit, die man aber an den Haaren herbeiziehen
               müßte, beſteht darin: dſs beide Romane in der Provinz{ }ſpielen. Aber{ }ſo{ }ſind die
               Menſchen:{ }ſie können eben immer nur vergleichen!\pend
           
\pstart
           Indem ich mich Ihnen mit wahrer Hochachtung empfehle, bin ich{\\[\baselineskip]}Ihr alt ergebener{\\[\baselineskip]}\spacefill\mbox{Ferdinand von Saar.}\pend
           \leftskip=0em{}\selectlanguage{ngerman}\endnumbering\briefempfaengerindex{Schnitzler, Arthur@\textsc{Schnitzler, Arthur}!zzzSaar, Ferdinand von@\emph{von Ferdinand von Saar}!1901-06-191@{19. 6. 1901}|)be}\mylabel{L01130h}  \newcommand{\dateiname}{L01130}\newcommand{\titel}{Ferdinand von Saar an Arthur Schnitzler, 19. 6. 1901}\newcommand{\editorInnen}{Martin Anton Müller und Gerd-Hermann Susen}%% latex-leseansicht-abspann.tex
%% Abspann für die Leseansicht.
%% Der Schalter \ifkorrekturansicht ist bereits durch den Vorspann gesetzt.

%% latex-abspann.tex
%% Gemeinsamer Abspann für Korrekturansicht und Leseansicht.
%% Setzt den Schalter \ifkorrekturansicht voraus (gesetzt in den
%% einbindenden Dateien latex-korrekturansicht-abspann.tex bzw.
%% latex-leseansicht-abspann.tex).
%% ---------------------------------------------------------------

\normalsize

% Das esempio-Environment wird nur in der Leseansicht benötigt
\ifkorrekturansicht\else
\newenvironment{esempio}[3]%
{
    \vspace{1.5ex}
    \rlap{\underline{#1}}
    \par
    \setlength{\parindent}{0cm}
    \nopagebreak
    \leftskip=#2cm
    \rightskip=#3cm
}
{
    \par
}
\fi

\doendnotes{C}
\bigskip
\vfill

\clearpage

\footnotesize

\ifkorrekturansicht
  \lohead{\textsc{register}}
\fi

% theindex-Environment neu definieren ohne reledmac
\makeatletter
\renewenvironment{theindex}{%
  \ifkorrekturansicht
    \section*{\indexname}%
  \else
    \subsubsection*{Index der erwähnten Entitäten}%
  \fi
  \setlength{\parindent}{0pt}%
  \setlength{\parskip}{0pt plus 0.3pt}%
  \let\item\@idxitem
}{%
  \ifkorrekturansicht\clearpage\fi
}
\makeatother

\IfFileExists{\jobname-pw.ind}{\input{\jobname-pw.ind}}{}

% Quellenangabe nur in der Leseansicht
\ifkorrekturansicht\else
% Fallback-Definitionen, falls die .tex-Datei \titel etc. nicht gesetzt hat
\providecommand{\titel}{}
\providecommand{\editorInnen}{}
\providecommand{\dateiname}{\jobname}

\vspace{3cm}

\vfill

\footnotesize
\textsc{Quelle}: \titel. Herausgegeben von {\editorInnen}. In: \emph{Arthur Schnitzler: Briefwechsel mit Autorinnen und Autoren}.
 Digitale Edition, https://schnitzler-briefe.acdh.oeaw.ac.at/{\dateiname}.html (Stand \today)
\fi

\end{document}


