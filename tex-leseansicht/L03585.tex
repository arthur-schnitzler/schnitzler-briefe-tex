%% latex-leseansicht-vorspann.tex
%% Vorspann für die Leseansicht.
%% Lädt die gemeinsame Datei latex-vorspann.tex mit nicht gesetztem Schalter.

\newif\ifkorrekturansicht
\korrekturansichtfalse

\input{../tex-inputs/latex-vorspann}


\section[ Felix Salten an Arthur Schnitzler, 23. 2. 1926]{L03585 Felix Salten an Arthur Schnitzler,  23. 2. 1926}
\nopagebreak\mylabel{L03585v}
\rehead{ }\normalsize\beginnumbering\briefempfaengerindex{Schnitzler, Arthur@\textsc{Schnitzler, Arthur}!zzzSalten, Felix@\emph{von Felix Salten}!1926-02-231@{23. 2. 1926}|(be}
\toendnotes[C]{\smallbreak\pagebreak[2]}
\correspDesc{Versand  durch Felix Salten am 23. 2. 1926 in Wien
\newline{}Erhalt  durch Arthur Schnitzler im Zeitraum [23. 2. 1926
                  – 27. 2. 1926?] in Wien}\toendnotes[C]{\smallbreak}
\Standort{CUL, Schnitzler, B 89, B 2.}
\physDesc{Brief, 1 Blatt, 1 Seite, 607 Zeichen
\newline{}Schreibmaschine
\newline{}Handschrift: schwarze Tinte, lateinische Kurrent (\noindent{}Unterschrift)
\newline{}Ordnung: 1) mit Bleistift von unbekannter Hand beschriftet: »{\pb}Salten«  2) mit Bleistift von unbekannter Hand nummeriert: »297«}\toendnotes[C]{\smallbreak}
\pstart
           \raggedleft{}{\pb}Wien\oindex{Wien@\textbf{Wien}, \emph{Verwaltungsgebiet}|pw}, 23. Februar 1926\pend
           
\pstart{}Lieber Schnitzler,\pend\vspace{0.5em}
\pstart
           Sie werden nächster Tage von der Gräfin Hartenau\pwindex{Hartenau, Johanna von 18.\,4.\,1865 Bratislava – 12.\,7.\,1951 Wien@\textsc{Hartenau, Johanna von} (18.\,4.\,1865 Bratislava – 12.\,7.\,1951 Wien), \emph{Sängerin}|pw} ein Albumblatt erhalten und dazu das Ersuchen, dieses Blatt mit
               einer Widmung für das Ehepaar Isidor\pwindex{Mautner, Isidor 7.\,10.\,1852 Náchod – 13.\,4.\,1930 Wien@\textsc{Mautner, Isidor} (7.\,10.\,1852 Náchod – 13.\,4.\,1930 Wien), \emph{Industrieller}|pw} und Jenny Mautner\pwindex{Mautner, Jenny 3.\,5.\,1856 Bratislava – 9.\,4.\,1938 Wien@\textsc{Mautner, Jenny} (3.\,5.\,1856 Bratislava – 9.\,4.\,1938 Wien)|pw} zu versehen, da die beiden im
                  \label{K_L03585-1v}\edtext{nächsten Monat ihre goldene
                  Hochzeit}{\lemma{\textnormal{\emph{nächsten … Hochzeit}}}\Cendnote{\textnormal{Die Hochzeit hatte am 19. 3. 1876 in Wien\oindex{Wien@\textbf{Wien}, \emph{Verwaltungsgebiet}|pwk}
                  stattgefunden.}}}\label{K_L03585-1} feiern. Der \label{K_L03585-2v}\edtext{Tod ihres Schwiegersohnes, des Dr. Hans
                  Breuer\pwindex{Breuer, Hans 8.\,10.\,1876 Wien – 27.\,1.\,1926 ebd.@\textsc{Breuer, Hans} (8.\,10.\,1876 Wien – 27.\,1.\,1926 ebd.), \emph{Rechtsanwalt}|pw}}{\lemma{\textnormal{\emph{Tod … Breuer}}}\Cendnote{\textnormal{Hans Breuer\pwindex{Breuer, Hans 8.\,10.\,1876 Wien – 27.\,1.\,1926 ebd.@\textsc{Breuer, Hans} (8.\,10.\,1876 Wien – 27.\,1.\,1926 ebd.), \emph{Rechtsanwalt}|pwk} war am 27. 1. 1926 verstorben.}}}\label{K_L03585-2}, hat jedes Fest, das geplant wurde,
               unmöglich gemacht und das Album soll, wie mir Gräfin Hartenau\pwindex{Hartenau, Johanna von 18.\,4.\,1865 Bratislava – 12.\,7.\,1951 Wien@\textsc{Hartenau, Johanna von} (18.\,4.\,1865 Bratislava – 12.\,7.\,1951 Wien), \emph{Sängerin}|pw} sagt, die einzige Freude sein, die man Mautners\pwindex{Mautner, Isidor 7.\,10.\,1852 Náchod – 13.\,4.\,1930 Wien@\textsc{Mautner, Isidor} (7.\,10.\,1852 Náchod – 13.\,4.\,1930 Wien), \emph{Industrieller}|pw}\pwindex{Mautner, Jenny 3.\,5.\,1856 Bratislava – 9.\,4.\,1938 Wien@\textsc{Mautner, Jenny} (3.\,5.\,1856 Bratislava – 9.\,4.\,1938 Wien)|pw} bereiten kann. Die Gräfin\pwindex{Hartenau, Johanna von 18.\,4.\,1865 Bratislava – 12.\,7.\,1951 Wien@\textsc{Hartenau, Johanna von} (18.\,4.\,1865 Bratislava – 12.\,7.\,1951 Wien), \emph{Sängerin}|pwv} hat mich ersucht, bei
               Ihnen wegen Ausfüllung des Albumblattes vorstellig zu werden.\pend
           
\pstart
           Mit herzlichem Gruss {\\[\baselineskip]}Ihr {\\[\baselineskip]}{[}hs.:{]} \spacefill\mbox{Felix Salten}\pend
           \leftskip=0em{}\selectlanguage{ngerman}\endnumbering\briefempfaengerindex{Schnitzler, Arthur@\textsc{Schnitzler, Arthur}!zzzSalten, Felix@\emph{von Felix Salten}!1926-02-231@{23. 2. 1926}|)be}\mylabel{L03585h}  \newcommand{\dateiname}{L03585}\newcommand{\titel}{Felix Salten an Arthur Schnitzler, 23. 2. 1926}\newcommand{\editorInnen}{Martin Anton Müller und Laura Untner}%% latex-leseansicht-abspann.tex
%% Abspann für die Leseansicht.
%% Der Schalter \ifkorrekturansicht ist bereits durch den Vorspann gesetzt.

%% latex-abspann.tex
%% Gemeinsamer Abspann für Korrekturansicht und Leseansicht.
%% Setzt den Schalter \ifkorrekturansicht voraus (gesetzt in den
%% einbindenden Dateien latex-korrekturansicht-abspann.tex bzw.
%% latex-leseansicht-abspann.tex).
%% ---------------------------------------------------------------

\normalsize

% Das esempio-Environment wird nur in der Leseansicht benötigt
\ifkorrekturansicht\else
\newenvironment{esempio}[3]%
{
    \vspace{1.5ex}
    \rlap{\underline{#1}}
    \par
    \setlength{\parindent}{0cm}
    \nopagebreak
    \leftskip=#2cm
    \rightskip=#3cm
}
{
    \par
}
\fi

\doendnotes{C}
\bigskip
\vfill

\clearpage

\footnotesize

\ifkorrekturansicht
  \lohead{\textsc{register}}
\fi

% theindex-Environment neu definieren ohne reledmac
\makeatletter
\renewenvironment{theindex}{%
  \ifkorrekturansicht
    \section*{\indexname}%
  \else
    \subsubsection*{Index der erwähnten Entitäten}%
  \fi
  \setlength{\parindent}{0pt}%
  \setlength{\parskip}{0pt plus 0.3pt}%
  \let\item\@idxitem
}{%
  \ifkorrekturansicht\clearpage\fi
}
\makeatother

\IfFileExists{\jobname-pw.ind}{\input{\jobname-pw.ind}}{}

% Quellenangabe nur in der Leseansicht
\ifkorrekturansicht\else
% Fallback-Definitionen, falls die .tex-Datei \titel etc. nicht gesetzt hat
\providecommand{\titel}{}
\providecommand{\editorInnen}{}
\providecommand{\dateiname}{\jobname}

\vspace{3cm}

\vfill

\footnotesize
\textsc{Quelle}: \titel. Herausgegeben von {\editorInnen}. In: \emph{Arthur Schnitzler: Briefwechsel mit Autorinnen und Autoren}.
 Digitale Edition, https://schnitzler-briefe.acdh.oeaw.ac.at/{\dateiname}.html (Stand \today)
\fi

\end{document}


