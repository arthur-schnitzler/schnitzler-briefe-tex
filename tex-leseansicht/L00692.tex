%% latex-leseansicht-vorspann.tex
%% Vorspann für die Leseansicht.
%% Lädt die gemeinsame Datei latex-vorspann.tex mit nicht gesetztem Schalter.

\newif\ifkorrekturansicht
\korrekturansichtfalse

\input{../tex-inputs/latex-vorspann}


\section[Richard Beer-Hofmann an Arthur Schnitzler, 25. 6. 1897]{L00692 Richard Beer-Hofmann an Arthur Schnitzler, 25. 6. 1897}
\nopagebreak\mylabel{L00692v}
\rehead{ }\normalsize\beginnumbering\briefempfaengerindex{Schnitzler, Arthur@\textsc{Schnitzler, Arthur}!zzzBeer-Hofmann, Richard@\emph{von Richard Beer-Hofmann}!1897-06-251@{25. 6. 1897}|(be}
\toendnotes[C]{\smallbreak\pagebreak[2]}
\correspDesc{Versand  durch Richard Beer-Hofmann am 25. 6. 1897 in Bad Ischl
\newline{}Erhalt  durch Arthur Schnitzler im Zeitraum [26. 6. 1897
                  – 30. 6. 1897?] in Wien}\toendnotes[C]{\smallbreak}
\Standort{CUL, Schnitzler, B 8.}
\physDesc{Briefkarte, 367 Zeichen
\newline{}Handschrift: Bleistift, lateinische Kurrent
\newline{}Ordnung: mit Bleistift von unbekannter Hand nummeriert:
                                    »101« }
\buchAbdrucke{\weitereDrucke{Arthur Schnitzler, Richard Beer-Hofmann: \emph{Briefwechsel 1891–1931}. Herausgegeben von Konstanze Fliedl. Wien, Zürich: \emph{Europaverlag} 1992, S. 111.} }\toendnotes[C]{\smallbreak}
\pstart
           \raggedleft{}{\pb}Ischl\oindex{Bad Ischl@\textbf{Bad Ischl}|pw}{ }25/VI 97\pend
           \vspace{0.5em}
\pstart
           Lieber Arthur ich habe kein Zimmer für Sie gewählt weil Herr Petter\pwindex{Petter, Leopold 17.\,11.\,1850 Bad Ischl – 3.\,7.\,1917 ebd.@\textsc{Petter, Leopold} (17.\,11.\,1850 Bad Ischl – 3.\,7.\,1917 ebd.), \emph{Hotelier}|pw} mir sagt er hätte 30 zu Ihrer Verfügung.
               Ich selbst mache \uline{morgen} – \uline{Samstag} – mit Papa\pwindex{Beer, Hermann 10.\,8.\,1835 Radiměř – 3.\,10.\,1902 Wien@\textsc{Beer, Hermann} (10.\,8.\,1835 Radiměř – 3.\,10.\,1902 Wien), \emph{Rechtsanwalt}|pwv}, Onkel\pwindex{Beer, Sigmund 14.\,9.\,1826 – 7.\,5.\,1905@\textsc{Beer, Sigmund} (14.\,9.\,1826 – 7.\,5.\,1905)|pwv}, Tante\pwindex{Beer, Agnes 12.\,2.\,1833 – 27.\,7.\,1909 Wien@\textsc{Beer, Agnes} (12.\,2.\,1833 – 27.\,7.\,1909 Wien)|pwv} einen Ausflug nach Gmunden\oindex{Gmunden@\textbf{Gmunden}|pw}{ }{\pb}und bin um 6 oder
                  8 Abends wieder in Ischl\oindex{Bad Ischl@\textbf{Bad Ischl}|pw}. Um
                  8 nachtmalen wir und um ½ 9 gehe ich weg – Wollen Sie
               mich also noch Samstag sehn, dann sind Sie zwischen 8 u
                  ½ 9 bei mir. Von Herzen Ihr \spacefill\mbox{R}\pend
           \selectlanguage{ngerman}\endnumbering\briefempfaengerindex{Schnitzler, Arthur@\textsc{Schnitzler, Arthur}!zzzBeer-Hofmann, Richard@\emph{von Richard Beer-Hofmann}!1897-06-251@{25. 6. 1897}|)be}\mylabel{L00692h}  \newcommand{\dateiname}{L00692}\newcommand{\titel}{Richard Beer-Hofmann an Arthur Schnitzler, 25. 6. 1897}\newcommand{\editorInnen}{Martin Anton Müller und Gerd-Hermann Susen}%% latex-leseansicht-abspann.tex
%% Abspann für die Leseansicht.
%% Der Schalter \ifkorrekturansicht ist bereits durch den Vorspann gesetzt.

%% latex-abspann.tex
%% Gemeinsamer Abspann für Korrekturansicht und Leseansicht.
%% Setzt den Schalter \ifkorrekturansicht voraus (gesetzt in den
%% einbindenden Dateien latex-korrekturansicht-abspann.tex bzw.
%% latex-leseansicht-abspann.tex).
%% ---------------------------------------------------------------

\normalsize

% Das esempio-Environment wird nur in der Leseansicht benötigt
\ifkorrekturansicht\else
\newenvironment{esempio}[3]%
{
    \vspace{1.5ex}
    \rlap{\underline{#1}}
    \par
    \setlength{\parindent}{0cm}
    \nopagebreak
    \leftskip=#2cm
    \rightskip=#3cm
}
{
    \par
}
\fi

\doendnotes{C}
\bigskip
\vfill

\clearpage

\footnotesize

\ifkorrekturansicht
  \lohead{\textsc{register}}
\fi

% theindex-Environment neu definieren ohne reledmac
\makeatletter
\renewenvironment{theindex}{%
  \ifkorrekturansicht
    \section*{\indexname}%
  \else
    \subsubsection*{Index der erwähnten Entitäten}%
  \fi
  \setlength{\parindent}{0pt}%
  \setlength{\parskip}{0pt plus 0.3pt}%
  \let\item\@idxitem
}{%
  \ifkorrekturansicht\clearpage\fi
}
\makeatother

\IfFileExists{\jobname-pw.ind}{\input{\jobname-pw.ind}}{}

% Quellenangabe nur in der Leseansicht
\ifkorrekturansicht\else
% Fallback-Definitionen, falls die .tex-Datei \titel etc. nicht gesetzt hat
\providecommand{\titel}{}
\providecommand{\editorInnen}{}
\providecommand{\dateiname}{\jobname}

\vspace{3cm}

\vfill

\footnotesize
\textsc{Quelle}: \titel. Herausgegeben von {\editorInnen}. In: \emph{Arthur Schnitzler: Briefwechsel mit Autorinnen und Autoren}.
 Digitale Edition, https://schnitzler-briefe.acdh.oeaw.ac.at/{\dateiname}.html (Stand \today)
\fi

\end{document}


