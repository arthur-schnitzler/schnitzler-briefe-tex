%% latex-korrekturansicht-vorspann.tex
%% Vorspann für die Korrekturansicht.
%% Lädt die gemeinsame Datei latex-vorspann.tex mit gesetztem Schalter.

\newif\ifkorrekturansicht
\korrekturansichttrue

\input{../tex-inputs/latex-vorspann}


\section[Richard Beer-Hofmann an Arthur Schnitzler, 25. 6. 1897]{L00692 Richard Beer-Hofmann an Arthur Schnitzler, 25. 6. 1897}
\nopagebreak\mylabel{L00692v}
\rehead{ }\normalsize\beginnumbering\briefempfaengerindex{Schnitzler, Arthur@\textsc{Schnitzler, Arthur}!zzzBeer-Hofmann, Richard@\emph{von Richard Beer-Hofmann}!1897-06-251@{25. 6. 1897}|(be}
\toendnotes[C]{\smallbreak\pagebreak[2]}\Standort{CUL, Schnitzler, B 8.}
\physDesc{Briefkarte, 367 Zeichen
\newline{}Handschrift: Bleistift, lateinische Kurrent
\newline{}Ordnung: mit Bleistift von unbekannter Hand nummeriert:
                                    »101« }
\buchAbdrucke{\weitereDrucke{Arthur Schnitzler, Richard Beer-Hofmann: \emph{Briefwechsel 1891–1931}. Wien, Zürich: \emph{Europaverlag} 1992, S. 111.} }\toendnotes[C]{\smallbreak}
\pstart
           \raggedleft{}{\pb}Ischl\oindex{Bad Ischl@\textbf{Bad Ischl}, \emph{P.PPL}|pw}{ }25/VI 97\pend
           \vspace{0.5em}
\pstart
           Lieber Arthur ich habe kein Zimmer für Sie gewählt weil Herr Petter\pwindex{Petter, Leopold 17.11.1850 – 03.07.1917@\textsc{Petter, Leopold} (17.11.1850 – 03.07.1917), \emph{Hotelier/Hotelière}|pw} mir sagt er hätte 30 zu Ihrer Verfügung.
               Ich selbst mache \uline{morgen} – \uline{Samstag} – mit Papa\pwindex{Beer, Hermann 10.8.1835 – 03.10.1902@\textsc{Beer, Hermann} (10.8.1835 – 03.10.1902), \emph{Rechtsanwalt/Rechtsanwältin}|pwv}, Onkel\pwindex{Beer, Sigmund 1826-09-14 – 1905-05-07@\textsc{Beer, Sigmund} (1826-09-14 – 1905-05-07)|pwv}, Tante\pwindex{Beer, Agnes 1833-02-12 – 27.7.1909@\textsc{Beer, Agnes} (1833-02-12 – 27.7.1909)|pwv} einen Ausflug nach Gmunden\oindex{Gmunden@\textbf{Gmunden}, \emph{P.PPL}|pw}{ }{\pb}und bin um 6 oder
                  8 Abends wieder in Ischl\oindex{Bad Ischl@\textbf{Bad Ischl}, \emph{P.PPL}|pw}. Um
                  8 nachtmalen wir und um ½ 9 gehe ich weg – Wollen Sie
               mich also noch Samstag sehn, dann sind Sie zwischen 8 u
                  ½ 9 bei mir. Von Herzen Ihr \spacefill\mbox{R}\pend
           \selectlanguage{ngerman}\endnumbering\briefempfaengerindex{Schnitzler, Arthur@\textsc{Schnitzler, Arthur}!zzzBeer-Hofmann, Richard@\emph{von Richard Beer-Hofmann}!1897-06-251@{25. 6. 1897}|)be}\mylabel{L00692h}  \normalsize

\doendnotes{C}
\bigskip
\vfill

\clearpage

\footnotesize

\lohead{\textsc{register}}

% Definiere theindex-Environment komplett neu ohne reledmac
\makeatletter
\renewenvironment{theindex}{%
  \section*{\indexname}%
  \setlength{\parindent}{0pt}%
  \setlength{\parskip}{0pt plus 0.3pt}%
  \let\item\@idxitem
}{%
  \clearpage
}
\makeatother

\IfFileExists{\jobname-pw.ind}{\input{\jobname-pw.ind}}{}

\end{document}

      