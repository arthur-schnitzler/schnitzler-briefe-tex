%% latex-leseansicht-vorspann.tex
%% Vorspann für die Leseansicht.
%% Lädt die gemeinsame Datei latex-vorspann.tex mit nicht gesetztem Schalter.

\newif\ifkorrekturansicht
\korrekturansichtfalse

\input{../tex-inputs/latex-vorspann}


\section[Richard Beer-Hofmann an Arthur Schnitzler, 23. 10. 1894]{L00391 Richard Beer-Hofmann an Arthur Schnitzler, 23. 10. 1894}
\nopagebreak\mylabel{L00391v}
\rehead{ }\normalsize\beginnumbering\briefempfaengerindex{Schnitzler, Arthur@\textsc{Schnitzler, Arthur}!zzzBeer-Hofmann, Richard@\emph{von Richard Beer-Hofmann}!1894-10-231@{23. 10. 1894}|(be}
\toendnotes[C]{\smallbreak\pagebreak[2]}
\correspDesc{Versand  durch Richard Beer-Hofmann am 23. 10. 1894 in Neapel
\newline{}Erhalt  durch Arthur Schnitzler am 26. 10. 1894 in Wien}\toendnotes[C]{\smallbreak}
\Standort{CUL, Schnitzler, B 8.}
\physDesc{Brief, 1 Blatt, 4 Seiten, 927 Zeichen
\newline{}Handschrift: Bleistift, lateinische Kurrent
\newline{}Schnitzler: mit Bleistift datiert: »23/10 94« und nummeriert »51« }
\buchAbdrucke{\weitereDrucke{Arthur Schnitzler, Richard Beer-Hofmann: \emph{Briefwechsel 1891–1931}. Herausgegeben von Konstanze Fliedl. Wien, Zürich: \emph{Europaverlag} 1992, S. 68.} }\toendnotes[C]{\smallbreak}
\pstart
           \noindent{}{\pb}Lieber Arthur! Soeben
               erhalte ich Ihren »Sudermann\pwindex{Sudermann, Hermann 30.\,9.\,1857 Macikai – 21.\,11.\,1928 Berlin@\textsc{Sudermann, Hermann} (30.\,9.\,1857 Macikai – 21.\,11.\,1928 Berlin), \emph{Schriftsteller}|pw}«brief, er hat
               sich mit meinem gestrigen gekreuzt, wo ich von »Schmetterlingsschlacht\pwindex{Sudermann, Hermann 30.\,9.\,1857 Macikai – 21.\,11.\,1928 Berlin@\textsc{Sudermann, Hermann} (30.\,9.\,1857 Macikai – 21.\,11.\,1928 Berlin), \emph{Schriftsteller}!Schmetterlingsschlacht. Komödie in 4 Akten@\strich\emph{Die Schmetterlingsschlacht. Komödie in 4 Akten}|pw}« sprach. Also ich habe richtig empfunden. Schön wär
               es wenn »Liebelei\pwindex{Schnitzler, Arthur 15.\,5.\,1862 Wien – 21.\,10.\,1931 ebd.@\textsc{Schnitzler, Arthur} (15.\,5.\,1862 Wien – 21.\,10.\,1931 ebd.), \emph{Schriftsteller, Mediziner}!Liebelei. Schauspiel in drei Akten@\strich\emph{Liebelei. Schauspiel in drei Akten}|pw}« am Burgtheater\orgindex{Burgtheater@Burgtheater|pw} drankäme – sehr {\pb}schön, der Erfolg der Aufführung
               wäre beinahe nebensächlich \strikeout{neb} gegenüber dem Erfolg
               der Annahme. Freilich, Schönthan\pwindex{Schönthan-Pernwald, Franz von 20.\,6.\,1849 Wien – 2.\,12.\,1913 ebd.@\textsc{Schönthan-Pernwald, Franz von} (20.\,6.\,1849 Wien – 2.\,12.\,1913 ebd.), \emph{Schriftsteller}|pw} und Rudolf Lothar\pwindex{Lothar, Rudolf 23.\,2.\,1865 Budapest – 2.\,10.\,1943 ebd.@\textsc{Lothar, Rudolf} (23.\,2.\,1865 Budapest – 2.\,10.\,1943 ebd.), \emph{Schriftsteller, Journalist, Theaterdirektor}|pw} und das Buch Hiob\pwindex{\textcolor{red}{\textsuperscript{XXXX indx1}}!Buch Hiob. Schauspiel in einem Akt@\strich\emph{Das Buch Hiob. Schauspiel in einem Akt}|pw}, spielt man auch am Burgtea{\pb}ter\orgindex{Burgtheater@Burgtheater|pw}. Nur \uline{wir} würden eigentlich erstaunt sein daß »Liebelei\pwindex{Schnitzler, Arthur 15.\,5.\,1862 Wien – 21.\,10.\,1931 ebd.@\textsc{Schnitzler, Arthur} (15.\,5.\,1862 Wien – 21.\,10.\,1931 ebd.), \emph{Schriftsteller, Mediziner}!Liebelei. Schauspiel in drei Akten@\strich\emph{Liebelei. Schauspiel in drei Akten}|pw}« angeno{\geminationm}en
               wird, und finden die Annahme all’ des Andern begreiflich. Nein arrogant sind wir
               nicht. In Pompei\strikeout{j}\oindex{Pompeji@\textbf{Pompeji}, \emph{Ausgrabung}|pw} war ich heute; ich bin ganz krank \strikeout{nach} vor
               Sehnsucht nach {\pb}wirklichen römischen\oindex{Rom@\textbf{Rom}, \emph{Hauptstadt}|pw} Bädern. Im Culturraffinement sind wir
               noch alle Barbaren. Ja – Theater wollten Sie wissen?\pend
           \settowidth{\longeste}{La martire(Samarra)}\settowidth{\longestz}{Mailand}\settowidth{\longestd}{}\settowidth{\longestv}{}\settowidth{\longestf}{}\addtolength\longeste{1em}
        \addtolength\longestz{1em}
      \pstart\noindent\makebox[\the\longeste][l]{La martire\pwindex{Samaras, Spyros 29.\,11.\,1861 Korfu – 7.\,4.\,1917 Athen@\textsc{Samaras, Spyros} (29.\,11.\,1861 Korfu – 7.\,4.\,1917 Athen), \emph{Komponist}!Martire@\strich\emph{La Martire}|pw} (Samarra\pwindex{Samaras, Spyros 29.\,11.\,1861 Korfu – 7.\,4.\,1917 Athen@\textsc{Samaras, Spyros} (29.\,11.\,1861 Korfu – 7.\,4.\,1917 Athen), \emph{Komponist}|pw})}\makebox[\the\longestz][l]{Mailand\oindex{Mailand@\textbf{Mailand}|pw}}
                  \pend\pstart\noindent\makebox[\the\longeste][l]{Medici\pwindex{\textcolor{red}{\textsuperscript{XXXX indx1}}!I Medici@\strich\emph{I Medici}|pw}}\makebox[\the\longestz][l]{}
                  \pend\settowidth{\longeste}{Premiere von}\settowidth{\longestz}{Puppenfeela fata del bambolm}\settowidth{\longestd}{Rom}\settowidth{\longestv}{}\settowidth{\longestf}{}\addtolength\longeste{1em}
        \addtolength\longestz{1em}
        \addtolength\longestd{1em}
      \pstart\noindent\makebox[\the\longeste][l]{Premiere von}\makebox[\the\longestz][l]{Ennemico del popolo\pwindex{\textcolor{red}{\textsuperscript{XXXX indx1}}!Volksfeind@\strich\emph{Ein Volksfeind}|pw}}
                  \makebox[\the\longestd][l]{Rom\oindex{Rom@\textbf{Rom}, \emph{Hauptstadt}|pw}}\pend\pstart\noindent\makebox[\the\longeste][l]{\hspace*{2.5em}“\hspace*{1.5em}“\hspace*{1em}}\makebox[\the\longestz][l]{Puppenfee\pwindex{\textcolor{red}{\textsuperscript{XXXX indx1}}!Puppenfee. Pantomimisches Divertissement@\strich\emph{Die Puppenfee. Pantomimisches Divertissement}|pw}\pwindex{\textcolor{red}{\textsuperscript{XXXX indx1}}!Puppenfee. Pantomimisches Divertissement@\strich\emph{Die Puppenfee. Pantomimisches Divertissement}|pw}{ }\label{K_L00391-1v}\edtext{la fata del bambol\pwindex{\textcolor{red}{\textsuperscript{XXXX indx1}}!Puppenfee. Pantomimisches Divertissement@\strich\emph{Die Puppenfee. Pantomimisches Divertissement}|pw}\pwindex{\textcolor{red}{\textsuperscript{XXXX indx1}}!Puppenfee. Pantomimisches Divertissement@\strich\emph{Die Puppenfee. Pantomimisches Divertissement}|pw}}{\lemma{\textnormal{\emph{la fata del bambol}}}\Cendnote{\textnormal{italienisch richtig: \emph{La fata delle bambole}\pwindex{\textcolor{red}{\textsuperscript{XXXX indx1}}!Puppenfee. Pantomimisches Divertissement@\strich\emph{Die Puppenfee. Pantomimisches Divertissement}|pwk}\pwindex{\textcolor{red}{\textsuperscript{XXXX indx1}}!Puppenfee. Pantomimisches Divertissement@\strich\emph{Die Puppenfee. Pantomimisches Divertissement}|pwk}}}}\label{K_L00391-1}}
                  \makebox[\the\longestd][l]{}\pend
\pstart
           Varietés, Operetten etc. überall.\pend
           \pstart Herzlichst Ihr \spacefill\mbox{Richard.}\pend{}
\pstart
           \noindent{}der sich auf Sie freut\pend
           
\pstart
           \raggedleft{}Neapel\oindex{Neapel@\textbf{Neapel}|pw}{ }23/X 94.\pend
           \selectlanguage{ngerman}\endnumbering\briefempfaengerindex{Schnitzler, Arthur@\textsc{Schnitzler, Arthur}!zzzBeer-Hofmann, Richard@\emph{von Richard Beer-Hofmann}!1894-10-231@{23. 10. 1894}|)be}\mylabel{L00391h}  \newcommand{\dateiname}{L00391}\newcommand{\titel}{Richard Beer-Hofmann an Arthur Schnitzler, 23. 10. 1894}\newcommand{\editorInnen}{Martin Anton Müller und Gerd-Hermann Susen}%% latex-leseansicht-abspann.tex
%% Abspann für die Leseansicht.
%% Der Schalter \ifkorrekturansicht ist bereits durch den Vorspann gesetzt.

%% latex-abspann.tex
%% Gemeinsamer Abspann für Korrekturansicht und Leseansicht.
%% Setzt den Schalter \ifkorrekturansicht voraus (gesetzt in den
%% einbindenden Dateien latex-korrekturansicht-abspann.tex bzw.
%% latex-leseansicht-abspann.tex).
%% ---------------------------------------------------------------

\normalsize

% Das esempio-Environment wird nur in der Leseansicht benötigt
\ifkorrekturansicht\else
\newenvironment{esempio}[3]%
{
    \vspace{1.5ex}
    \rlap{\underline{#1}}
    \par
    \setlength{\parindent}{0cm}
    \nopagebreak
    \leftskip=#2cm
    \rightskip=#3cm
}
{
    \par
}
\fi

\doendnotes{C}
\bigskip
\vfill

\clearpage

\footnotesize

\ifkorrekturansicht
  \lohead{\textsc{register}}
\fi

% theindex-Environment neu definieren ohne reledmac
\makeatletter
\renewenvironment{theindex}{%
  \ifkorrekturansicht
    \section*{\indexname}%
  \else
    \subsubsection*{Index der erwähnten Entitäten}%
  \fi
  \setlength{\parindent}{0pt}%
  \setlength{\parskip}{0pt plus 0.3pt}%
  \let\item\@idxitem
}{%
  \ifkorrekturansicht\clearpage\fi
}
\makeatother

\IfFileExists{\jobname-pw.ind}{\input{\jobname-pw.ind}}{}

% Quellenangabe nur in der Leseansicht
\ifkorrekturansicht\else
% Fallback-Definitionen, falls die .tex-Datei \titel etc. nicht gesetzt hat
\providecommand{\titel}{}
\providecommand{\editorInnen}{}
\providecommand{\dateiname}{\jobname}

\vspace{3cm}

\vfill

\footnotesize
\textsc{Quelle}: \titel. Herausgegeben von {\editorInnen}. In: \emph{Arthur Schnitzler: Briefwechsel mit Autorinnen und Autoren}.
 Digitale Edition, https://schnitzler-briefe.acdh.oeaw.ac.at/{\dateiname}.html (Stand \today)
\fi

\end{document}


