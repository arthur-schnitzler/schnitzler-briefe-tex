%% latex-leseansicht-vorspann.tex
%% Vorspann für die Leseansicht.
%% Lädt die gemeinsame Datei latex-vorspann.tex mit nicht gesetztem Schalter.

\newif\ifkorrekturansicht
\korrekturansichtfalse

\input{../tex-inputs/latex-vorspann}


         
         \renewcommand{\erwaehntePersonen}{Personen: Rudolf Lothar, Spyros Samaras, Franz von Schönthan-Pernwald, Hermann Sudermann}
         \renewcommand{\erwaehnteOrte}{Orte: Burgtheater, Mailand, Neapel, Pompei, Rom, Wien}
         \renewcommand{\erwaehnteWerke}{Werke: Das Buch Hiob. Schauspiel in einem Akt, Die Puppenfee, Die Schmetterlingsschlacht. Komödie in 4 Akten, Ein Volksfeind, I Medici, La Martire, Liebelei. Schauspiel in drei Akten}
               \section[Richard Beer-Hofmann an Arthur Schnitzler, 23. 10. 1894]{ Richard Beer-Hofmann an Arthur Schnitzler, 23. 10. 1894}\nopagebreak\mylabel{v}\rehead{ }\begin{ledgroupsized}[t]{13cm}\normalsize\beginnumbering \toendnotes[C]{\smallbreak\pagebreak[2]} \Standort{CUL, Schnitzler, B 8.}
\physDesc{Brief, 1 Blatt, 4 Seiten
\newline{}Handschrift: Bleistift, lateinische Kurrent
\newline{}Schnitzler: mit Bleistift datiert: »23/10 94« und nummeriert »51« }\buchAbdrucke{\weitereDrucke{Arthur Schnitzler, Richard Beer-Hofmann: \emph{Briefwechsel 1891–1931}. Hg. Konstanze Fliedl. Wien, Zürich: \emph{Europaverlag} 1992, S. 68.} }\toendnotes[C]{\smallbreak}\pstart
           \noindent{}{\pb}Lieber Arthur! Soeben
               erhalte ich Ihren »Sudermann\pwindex{Sudermann, Hermann 30.09.1857 – 21.11.1928@\textsc{Sudermann, Hermann} (30.09.1857 – 21.11.1928), \emph{Schriftsteller}|pw}«brief, er hat sich
               mit meinem gestrigen gekreuzt, wo ich von »Schmetterlingsschlacht\pwindex{Sudermann, Hermann 30.09.1857 – 21.11.1928@\textsc{Sudermann, Hermann} (30.09.1857 – 21.11.1928), \emph{Schriftsteller}!Schmetterlingsschlacht. Komoedie in 4 Akten1894-10-06@\strich\emph{Die Schmetterlingsschlacht. Komödie in 4 Akten} {[}1894-10-06{]}|pw}« sprach. Also ich habe richtig empfunden. Schön
               wär es wenn »Liebelei\pwindex{Schnitzler, Arthur 15.05.1862 – 21.10.1931@\textsc{Schnitzler, Arthur} (15.05.1862 – 21.10.1931), \emph{Schriftsteller, Mediziner}!Liebelei. Schauspiel in drei Akten1895-10-09@\strich\emph{Liebelei. Schauspiel in drei Akten} {[}1895-10-09{]}|pw}« am Burgtheater\oindex{Burgtheater@\textbf{Burgtheater}|pw} drankäme – sehr {\pb}schön, der Erfolg der Aufführung
               wäre beinahe nebensächlich \strikeout{neb} gegenüber dem Erfolg
               der Annahme. Freilich, Schönthan\pwindex{Schoenthan-Pernwald, Franz von 20.06.1849 – 02.12.1913@\textsc{Schönthan-Pernwald, Franz von} (20.06.1849 – 02.12.1913), \emph{Schriftsteller}|pw} und Rudolf Lothar\pwindex{Lothar, Rudolf 23.2.1865 – 2.10.1943@\textsc{Lothar, Rudolf} (23.2.1865 – 2.10.1943), \emph{Schriftsteller, Journalist, Theaterdirektor}|pw} und das Buch
                  Hiob\pwindex{\textcolor{red}{\textsuperscript{XXXX1 indx}}!Buch Hiob. Schauspiel in einem Akt1891@\strich\emph{Das Buch Hiob. Schauspiel in einem Akt} {[}1891{]}|pw}, spielt man auch am Burgtea{\pb}ter\oindex{Burgtheater@\textbf{Burgtheater}|pw}. Nur \uline{wir} würden eigentlich erstaunt sein daß »Liebelei\pwindex{Schnitzler, Arthur 15.05.1862 – 21.10.1931@\textsc{Schnitzler, Arthur} (15.05.1862 – 21.10.1931), \emph{Schriftsteller, Mediziner}!Liebelei. Schauspiel in drei Akten1895-10-09@\strich\emph{Liebelei. Schauspiel in drei Akten} {[}1895-10-09{]}|pw}« angeno{\geminationm}en wird,
               und finden die Annahme all’ des Andern begreiflich. Nein arrogant sind wir nicht. In
                  Pompei\strikeout{j}\oindex{Pompei@\textbf{Pompei}|pw} war ich heute; ich bin ganz krank \strikeout{nach} vor Sehnsucht nach {\pb}wirklichen römischen\oindex{Rom@\textbf{Rom}|pw} Bädern. Im Culturraffinement sind wir noch alle Barbaren.
               Ja – Theater wollten Sie wissen?\pend
           \settowidth{\longeste}{La martire (Samarra)}\settowidth{\longestz}{Mailand}\settowidth{\longestd}{}\settowidth{\longestv}{}\settowidth{\longestf}{}\addtolength\longeste{1em}
        \addtolength\longestz{1em}
      \pstart\noindent\makebox[\the\longeste][l]{La martire\pwindex{Samaras, Spyros 29.11.1861 – 07.04.1917@\textsc{Samaras, Spyros} (29.11.1861 – 07.04.1917), \emph{Komponist}!Martire1894@\strich\emph{La Martire} {[}1894{]}|pw} (Samarra\pwindex{Samaras, Spyros 29.11.1861 – 07.04.1917@\textsc{Samaras, Spyros} (29.11.1861 – 07.04.1917), \emph{Komponist}|pw})}\makebox[\the\longestz][l]{Mailand\oindex{Mailand@\textbf{Mailand}|pw}}
                  \pend\pstart\noindent\makebox[\the\longeste][l]{Medici\pwindex{\textcolor{red}{\textsuperscript{XXXX1 indx}}!I Medici6. 11. 1893@\strich\emph{I Medici} {[}6. 11. 1893{]}|pw}}\makebox[\the\longestz][l]{}
                  \pend\settowidth{\longeste}{Premiere von}\settowidth{\longestz}{Puppenfeela fata del bambolitalienisch richtig: La fata delle bambole}\settowidth{\longestd}{Rom}\settowidth{\longestv}{}\settowidth{\longestf}{}\addtolength\longeste{1em}
        \addtolength\longestz{1em}
        \addtolength\longestd{1em}
      \pstart\noindent\makebox[\the\longeste][l]{Premiere von}\makebox[\the\longestz][l]{Ennemico del popolo\pwindex{\textcolor{red}{\textsuperscript{XXXX1 indx}}!Volksfeind1882@\strich\emph{Ein Volksfeind} {[}1882{]}|pw}}
                  \makebox[\the\longestd][l]{Rom\oindex{Rom@\textbf{Rom}|pw}}\pend\pstart\noindent\makebox[\the\longeste][l]{\hspace*{2.5em}“\hspace*{1.5em}“\hspace*{1em}}\makebox[\the\longestz][l]{Puppenfee\pwindex{\textcolor{red}{\textsuperscript{XXXX1 indx}}!Puppenfee1888@\strich\emph{Die Puppenfee} {[}1888{]}|pw}\pwindex{\textcolor{red}{\textsuperscript{XXXX1 indx}}!Puppenfee1888@\strich\emph{Die Puppenfee} {[}Vertonung, 1888{]}|pw}{ }\label{K_L00391_1v}\edtext{la fata del bambol\pwindex{\textcolor{red}{\textsuperscript{XXXX1 indx}}!Puppenfee1888@\strich\emph{Die Puppenfee} {[}1888{]}|pw}\pwindex{\textcolor{red}{\textsuperscript{XXXX1 indx}}!Puppenfee1888@\strich\emph{Die Puppenfee} {[}Vertonung, 1888{]}|pw}}{\lemma{\textnormal{\emph{la fata del bambol}}}\Cendnote{\textnormal{italienisch richtig: \emph{La fata delle bambole}\pwindex{\textcolor{red}{\textsuperscript{XXXX1 indx}}!Puppenfee1888@\strich\emph{Die Puppenfee} {[}1888{]}|pwk}\pwindex{\textcolor{red}{\textsuperscript{XXXX1 indx}}!Puppenfee1888@\strich\emph{Die Puppenfee} {[}Vertonung, 1888{]}|pwk}}}}\label{K_L00391_1h}}
                  \makebox[\the\longestd][l]{}\pend\pstart
           Varietés, Operetten etc. überall.\pend
           \pstart Herzlichst Ihr \spacefill\mbox{Richard.}\pend{}\pstart
           \noindent{}der sich auf Sie freut\pend
           \pstart
           \raggedleft{}Neapel\oindex{Neapel@\textbf{Neapel}|pw}{ }23/X 94.\pend
           
         
         \endnumbering\mylabel{h}\end{ledgroupsized}  \newcommand{\dateiname}{L00391}\newcommand{\titel}{Richard Beer-Hofmann an Arthur Schnitzler, 23. 10. 1894}\newcommand{\editorInnen}{Martin Anton Müller und Gerd-Hermann Susen}%% latex-leseansicht-abspann.tex
%% Abspann für die Leseansicht.
%% Der Schalter \ifkorrekturansicht ist bereits durch den Vorspann gesetzt.

%% latex-abspann.tex
%% Gemeinsamer Abspann für Korrekturansicht und Leseansicht.
%% Setzt den Schalter \ifkorrekturansicht voraus (gesetzt in den
%% einbindenden Dateien latex-korrekturansicht-abspann.tex bzw.
%% latex-leseansicht-abspann.tex).
%% ---------------------------------------------------------------

\normalsize

% Das esempio-Environment wird nur in der Leseansicht benötigt
\ifkorrekturansicht\else
\newenvironment{esempio}[3]%
{
    \vspace{1.5ex}
    \rlap{\underline{#1}}
    \par
    \setlength{\parindent}{0cm}
    \nopagebreak
    \leftskip=#2cm
    \rightskip=#3cm
}
{
    \par
}
\fi

\doendnotes{C}
\bigskip
\vfill

\clearpage

\footnotesize

\ifkorrekturansicht
  \lohead{\textsc{register}}
\fi

% theindex-Environment neu definieren ohne reledmac
\makeatletter
\renewenvironment{theindex}{%
  \ifkorrekturansicht
    \section*{\indexname}%
  \else
    \subsubsection*{Index der erwähnten Entitäten}%
  \fi
  \setlength{\parindent}{0pt}%
  \setlength{\parskip}{0pt plus 0.3pt}%
  \let\item\@idxitem
}{%
  \ifkorrekturansicht\clearpage\fi
}
\makeatother

\IfFileExists{\jobname-pw.ind}{\input{\jobname-pw.ind}}{}

% Quellenangabe nur in der Leseansicht
\ifkorrekturansicht\else
% Fallback-Definitionen, falls die .tex-Datei \titel etc. nicht gesetzt hat
\providecommand{\titel}{}
\providecommand{\editorInnen}{}
\providecommand{\dateiname}{\jobname}

\vspace{3cm}

\vfill

\footnotesize
\textsc{Quelle}: \titel. Herausgegeben von {\editorInnen}. In: \emph{Arthur Schnitzler: Briefwechsel mit Autorinnen und Autoren}.
 Digitale Edition, https://schnitzler-briefe.acdh.oeaw.ac.at/{\dateiname}.html (Stand \today)
\fi

\end{document}


      