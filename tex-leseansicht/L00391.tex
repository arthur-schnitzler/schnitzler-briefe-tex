%% latex-korrekturansicht-vorspann.tex
%% Vorspann für die Korrekturansicht.
%% Lädt die gemeinsame Datei latex-vorspann.tex mit gesetztem Schalter.

\newif\ifkorrekturansicht
\korrekturansichttrue

\input{../tex-inputs/latex-vorspann}


\section[Richard Beer-Hofmann an Arthur Schnitzler, 23. 10. 1894]{L00391 Richard Beer-Hofmann an Arthur Schnitzler, 23. 10. 1894}
\nopagebreak\mylabel{L00391v}
\rehead{ }\normalsize\beginnumbering\briefempfaengerindex{Schnitzler, Arthur@\textsc{Schnitzler, Arthur}!zzzBeer-Hofmann, Richard@\emph{von Richard Beer-Hofmann}!1894-10-231@{23. 10. 1894}|(be}
\toendnotes[C]{\smallbreak\pagebreak[2]}\Standort{CUL, Schnitzler, B 8.}
\physDesc{Brief, 1 Blatt, 4 Seiten, 927 Zeichen
\newline{}Handschrift: Bleistift, lateinische Kurrent
\newline{}Schnitzler: mit Bleistift datiert: »23/10 94« und nummeriert »51« }
\buchAbdrucke{\weitereDrucke{Arthur Schnitzler, Richard Beer-Hofmann: \emph{Briefwechsel 1891–1931}. Wien, Zürich: \emph{Europaverlag} 1992, S. 68.} }\toendnotes[C]{\smallbreak}
\pstart
           \noindent{}{\pb}Lieber Arthur! Soeben
               erhalte ich Ihren »Sudermann\pwindex{Sudermann, Hermann 30.09.1857 – 21.11.1928@\textsc{Sudermann, Hermann} (30.09.1857 – 21.11.1928), \emph{Schriftsteller/Schriftstellerin}|pw}«brief, er hat
               sich mit meinem gestrigen gekreuzt, wo ich von »Schmetterlingsschlacht\pwindex{Schmetterlingsschlacht. Komoedie in 4 Akten@\emph{Die Schmetterlingsschlacht. Komödie in 4 Akten}|pw}« sprach. Also ich habe richtig empfunden. Schön wär
               es wenn »Liebelei\pwindex{Liebelei. Schauspiel in drei Akten@\emph{Liebelei. Schauspiel in drei Akten}|pw}« am Burgtheater\orgindex{Burgtheater@Burgtheater|pw} drankäme – sehr {\pb}schön, der Erfolg der Aufführung
               wäre beinahe nebensächlich \strikeout{neb} gegenüber dem Erfolg
               der Annahme. Freilich, Schönthan\pwindex{Schoenthan-Pernwald, Franz von 20.06.1849 – 02.12.1913@\textsc{Schönthan-Pernwald, Franz von} (20.06.1849 – 02.12.1913), \emph{Schriftsteller/Schriftstellerin}|pw} und Rudolf Lothar\pwindex{Lothar, Rudolf 23.2.1865 – 2.10.1943@\textsc{Lothar, Rudolf} (23.2.1865 – 2.10.1943), \emph{Schriftsteller/Schriftstellerin, Journalist/Journalistin, Theaterdirektor/Theaterdirektorin}|pw} und das Buch Hiob\pwindex{Buch Hiob. Schauspiel in einem Akt@\emph{Das Buch Hiob. Schauspiel in einem Akt}|pw}, spielt man auch am Burgtea{\pb}ter\orgindex{Burgtheater@Burgtheater|pw}. Nur \uline{wir} würden eigentlich erstaunt sein daß »Liebelei\pwindex{Liebelei. Schauspiel in drei Akten@\emph{Liebelei. Schauspiel in drei Akten}|pw}« angeno{\geminationm}en
               wird, und finden die Annahme all’ des Andern begreiflich. Nein arrogant sind wir
               nicht. In Pompei\strikeout{j}\oindex{Pompeji@\textbf{Pompeji}, \emph{S.ANS}|pw} war ich heute; ich bin ganz krank \strikeout{nach} vor
               Sehnsucht nach {\pb}wirklichen römischen\oindex{Rom@\textbf{Rom}, \emph{P.PPLC}|pw} Bädern. Im Culturraffinement sind wir
               noch alle Barbaren. Ja – Theater wollten Sie wissen?\pend
           \settowidth{\longeste}{}\settowidth{\longestz}{}\settowidth{\longestd}{}\settowidth{\longestv}{}\settowidth{\longestf}{}\addtolength\longeste{1em}
        \addtolength\longestz{1em}
      \settowidth{\longeste}{}\settowidth{\longestz}{}\settowidth{\longestd}{}\settowidth{\longestv}{}\settowidth{\longestf}{}\addtolength\longeste{1em}
        \addtolength\longestz{1em}
      
\pstart
           Varietés, Operetten etc. überall.\pend
           \pstart Herzlichst Ihr \spacefill\mbox{Richard.}\pend{}
\pstart
           \noindent{}der sich auf Sie freut\pend
           
\pstart
           \raggedleft{}Neapel\oindex{Neapel@\textbf{Neapel}, \emph{P.PPLA}|pw}{ }23/X 94.\pend
           \selectlanguage{ngerman}\endnumbering\briefempfaengerindex{Schnitzler, Arthur@\textsc{Schnitzler, Arthur}!zzzBeer-Hofmann, Richard@\emph{von Richard Beer-Hofmann}!1894-10-231@{23. 10. 1894}|)be}\mylabel{L00391h}  \normalsize

\doendnotes{C}
\bigskip
\vfill

\clearpage

\footnotesize

\lohead{\textsc{register}}

% Definiere theindex-Environment komplett neu ohne reledmac
\makeatletter
\renewenvironment{theindex}{%
  \section*{\indexname}%
  \setlength{\parindent}{0pt}%
  \setlength{\parskip}{0pt plus 0.3pt}%
  \let\item\@idxitem
}{%
  \clearpage
}
\makeatother

\IfFileExists{\jobname-pw.ind}{\input{\jobname-pw.ind}}{}

\end{document}

      