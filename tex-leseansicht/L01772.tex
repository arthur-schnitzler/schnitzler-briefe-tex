%% latex-korrekturansicht-vorspann.tex
%% Vorspann für die Korrekturansicht.
%% Lädt die gemeinsame Datei latex-vorspann.tex mit gesetztem Schalter.

\newif\ifkorrekturansicht
\korrekturansichttrue

\input{../tex-inputs/latex-vorspann}


\section[Lou Andreas-Salomé an Arthur Schnitzler, Juni 1908]{L01772 Lou Andreas-Salomé an Arthur Schnitzler, Juni 1908}
\nopagebreak\mylabel{L01772v}
\rehead{ }\normalsize\beginnumbering\briefempfaengerindex{Schnitzler, Arthur@\textsc{Schnitzler, Arthur}!zzzAndreas-Salome, Lou@\emph{von Lou Andreas-Salomé}!1908-06-301@{Juni 1908}|(be}
\toendnotes[C]{\smallbreak\pagebreak[2]}\Standort{CUL, Schnitzler, B 3.}
\physDesc{Visitenkarte, 189 Zeichen
\newline{}Handschrift: schwarze Tinte, deutsche Kurrent
\newline{}Ordnung: mit Bleistift von unbekannter Hand nummeriert:
                                    »21« }
\pstart
           \centering{}{\pb}\textcolor{gray}{\textbf{Lou Andreas-Salomé}}\pend
           
\pstart{}Lieber \textsc{Arthur Schnitzler}!\pend\vspace{0.5em}
\pstart
           Frl. \textsc{Lulu v. Strauss u Torney}\pwindex{Strauss und Torney, Lulu von 20.09.1873 – 19.06.1956@\textsc{Strauss und Torney, Lulu von} (20.09.1873 – 19.06.1956), \emph{Schriftsteller/Schriftstellerin}|pw} kommt fremd nach Wien\oindex{Wien@\textbf{Wien}, \emph{A.ADM2}|pw}; ich habe ſie gebeten
               zu Ihnen zu gehen, mit den ſchönſten Grü{\pb}ßen an Sie und Ihre Frau\pwindex{Schnitzler, Olga 17.01.1882 – 13.01.1970@\textsc{Schnitzler, Olga} (17.01.1882 – 13.01.1970), \emph{Schauspieler/Schauspielerin, Sänger/Sängerin}|pw}\pend
           
\pstart
           von{\\[\baselineskip]}\spacefill\mbox{Frau Lou.}\pend
           \leftskip=0em{}
\pstart
           \noindent{}\textsc{Göttingen\oindex{Goettingen@\textbf{Göttingen}, \emph{P.PPLA3}|pw}, Juni 1908}\pend
           \selectlanguage{ngerman}\endnumbering\briefempfaengerindex{Schnitzler, Arthur@\textsc{Schnitzler, Arthur}!zzzAndreas-Salome, Lou@\emph{von Lou Andreas-Salomé}!1908-06-011@{Juni 1908}|)be}\mylabel{L01772h}  \normalsize

\doendnotes{C}
\bigskip
\vfill

\clearpage

\footnotesize

\lohead{\textsc{register}}

% Definiere theindex-Environment komplett neu ohne reledmac
\makeatletter
\renewenvironment{theindex}{%
  \section*{\indexname}%
  \setlength{\parindent}{0pt}%
  \setlength{\parskip}{0pt plus 0.3pt}%
  \let\item\@idxitem
}{%
  \clearpage
}
\makeatother

\IfFileExists{\jobname-pw.ind}{\input{\jobname-pw.ind}}{}

\end{document}

      