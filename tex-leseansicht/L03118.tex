%% latex-korrekturansicht-vorspann.tex
%% Vorspann für die Korrekturansicht.
%% Lädt die gemeinsame Datei latex-vorspann.tex mit gesetztem Schalter.

\newif\ifkorrekturansicht
\korrekturansichttrue

\input{../tex-inputs/latex-vorspann}


\section[Felix Salten an Arthur Schnitzler, {[}zwischen 16. 11. 1892 und 3. 12. 1892{]}]{L03118 Felix Salten an Arthur Schnitzler, {[}zwischen 16. 11. 1892 und
               3. 12. 1892{]}}
\nopagebreak\mylabel{L03118v}
\rehead{ }\normalsize\beginnumbering\briefempfaengerindex{Schnitzler, Arthur@\textsc{Schnitzler, Arthur}!zzzSalten, Felix@\emph{von Felix Salten}!1892-12-031@{{[}zwischen 16. 11. 1892 und
                  3. 12. 1892{]}}|(be}
\toendnotes[C]{\smallbreak\pagebreak[2]}\Standort{CUL, Schnitzler, B 89, A 1.}
\physDesc{Brief, 1 Blatt, 2 Seiten, 201 Zeichen
\newline{}Handschrift: Bleistift, lateinische Kurrent
\newline{}Schnitzler: mit Bleistift datiert: »Ende 92« 
\newline{}Ordnung: mit Bleistift von unbekannter Hand nummeriert: »21« }\toendnotes[C]{\smallbreak}
\pstart
           \noindent{}{\pb}Lieber Freund! Ich sende Ihnen die \label{K_L03118-1v}\edtext{Pantomime\pwindex{Schleier der Pierrette. Pantomime in drei Bildern@\emph{Der Schleier der Pierrette. Pantomime in drei Bildern}|pw}}{\lemma{\textnormal{\emph{Pantomime}}}\Cendnote{\textnormal{Am 15. 11. 1892 hatte Schnitzler in Anwesenheit Saltens\pwindex{Salten, Felix 06.09.1869 – 08.10.1945@\textsc{Salten, Felix} (06.09.1869 – 08.10.1945), \emph{Schriftsteller/Schriftstellerin, Journalist/Journalistin, Chefredakteur/Chefredakteurin}|pwk} seine \emph{Pantomime}\pwindex{Schleier der Pierrette. Pantomime in drei Bildern@\emph{Der Schleier der Pierrette. Pantomime in drei Bildern}|pwk} (erst 1910
                   als \emph{Der Schleier der Pierrette}\pwindex{Schleier der Pierrette. Pantomime in drei Bildern@\emph{Der Schleier der Pierrette. Pantomime in drei Bildern}|pwk}
                  publiziert) vorgelesen. Sofern hier dieses Werk\pwindex{Schleier der Pierrette. Pantomime in drei Bildern@\emph{Der Schleier der Pierrette. Pantomime in drei Bildern}|pwkv} gemeint ist, würde das den Tag nach
                  der Lesung als frühesten möglichen Termin für das undatierte Korrespondenzstück
                  festlegen. Da \emph{Sterben}\pwindex{Sterben. Novelle@\emph{Sterben. Novelle}|pwk} bereits vorlag, ist
                  anzunehmen, dass Salten\pwindex{Salten, Felix 06.09.1869 – 08.10.1945@\textsc{Salten, Felix} (06.09.1869 – 08.10.1945), \emph{Schriftsteller/Schriftstellerin, Journalist/Journalistin, Chefredakteur/Chefredakteurin}|pwk} das Manuskript\pwindex{Schleier der Pierrette. Pantomime in drei Bildern@\emph{Der Schleier der Pierrette. Pantomime in drei Bildern}|pwkv} beim Besuch der Lesung der \emph{Pantomime}\pwindex{Schleier der Pierrette. Pantomime in drei Bildern@\emph{Der Schleier der Pierrette. Pantomime in drei Bildern}|pwk} bekommen hatte. Bei dem in Folge
                  angedachten Treffen bei Specht\pwindex{Specht, Richard 07.12.1870 – 18.03.1932@\textsc{Specht, Richard} (07.12.1870 – 18.03.1932), \emph{Schriftsteller/Schriftstellerin, Journalist/Journalistin, Kritiker/Kritikerin}|pwk} dürfte es
                  sich um den 4. 12. 1892 handeln, was das zeitliche Ende einer
                  möglichen Datierung bildet.}}}\label{K_L03118-1}, da ich momentan zu müd und unwol bin, um
               selbst zu Ihnen zu kommen. Ich liege hier, und lese Ihre \label{K_L03118-2v}\edtext{Novelle\pwindex{Sterben. Novelle@\emph{Sterben. Novelle}|pwv}}{\lemma{\textnormal{\emph{Novelle}}}\Cendnote{\textnormal{Am 30. 10. 1892 hatte Schnitzler in Anwesenheit Saltens\pwindex{Salten, Felix 06.09.1869 – 08.10.1945@\textsc{Salten, Felix} (06.09.1869 – 08.10.1945), \emph{Schriftsteller/Schriftstellerin, Journalist/Journalistin, Chefredakteur/Chefredakteurin}|pwk} seine Novelle \emph{Sterben}\pwindex{Sterben. Novelle@\emph{Sterben. Novelle}|pwk}
                  vorgelesen.}}}\label{K_L03118-2}.\pend
           
\pstart
           Auf Wiedersehen {\pb}eventuell
               bei Specht\pwindex{Specht, Richard 07.12.1870 – 18.03.1932@\textsc{Specht, Richard} (07.12.1870 – 18.03.1932), \emph{Schriftsteller/Schriftstellerin, Journalist/Journalistin, Kritiker/Kritikerin}|pw}. {\\[\baselineskip]}Herzlich {\\[\baselineskip]}Ihr {\\[\baselineskip]}\spacefill\mbox{Salten}\pend
           \leftskip=0em{}\selectlanguage{ngerman}\endnumbering\briefempfaengerindex{Schnitzler, Arthur@\textsc{Schnitzler, Arthur}!zzzSalten, Felix@\emph{von Felix Salten}!1892-11-161@{{[}zwischen 16. 11. 1892 und
                  3. 12. 1892{]}}|)be}\mylabel{L03118h}  \normalsize

\doendnotes{C}
\bigskip
\vfill

\clearpage

\footnotesize

\lohead{\textsc{register}}

% Definiere theindex-Environment komplett neu ohne reledmac
\makeatletter
\renewenvironment{theindex}{%
  \section*{\indexname}%
  \setlength{\parindent}{0pt}%
  \setlength{\parskip}{0pt plus 0.3pt}%
  \let\item\@idxitem
}{%
  \clearpage
}
\makeatother

\IfFileExists{\jobname-pw.ind}{\input{\jobname-pw.ind}}{}

\end{document}

      