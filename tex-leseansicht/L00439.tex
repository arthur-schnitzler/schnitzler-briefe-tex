%% latex-leseansicht-vorspann.tex
%% Vorspann für die Leseansicht.
%% Lädt die gemeinsame Datei latex-vorspann.tex mit nicht gesetztem Schalter.

\newif\ifkorrekturansicht
\korrekturansichtfalse

\input{../tex-inputs/latex-vorspann}


\section[Richard Beer-Hofmann an Arthur Schnitzler, {{[}}9. 5. 1895{{]}}]{L00439 Richard Beer-Hofmann an Arthur Schnitzler, {[}9. 5. 1895{]}}
\nopagebreak\mylabel{L00439v}
\rehead{ }\normalsize\beginnumbering\briefempfaengerindex{Schnitzler, Arthur@\textsc{Schnitzler, Arthur}!zzzBeer-Hofmann, Richard@\emph{von Richard Beer-Hofmann}!1895-05-091@{{[}9. 5. 1895{]}}|(be}
\toendnotes[C]{\smallbreak\pagebreak[2]}
\correspDesc{Versand  durch Richard Beer-Hofmann am [9. 5. 1895] in Wien
\newline{}Erhalt  durch Arthur Schnitzler im Zeitraum [9. 5. 1895
                  – 13. 5. 1895?] in Wien}\toendnotes[C]{\smallbreak}
\Standort{CUL, Schnitzler, B 8.}
\physDesc{Brief, 1 Blatt, 1 Seite, 178 Zeichen (Auf der Rückseite von Beer-Hofmann mit Bleistift: »\noindent{}{\pb}D\textsuperscript{r}{ }\introOben{}Josef\introOben{} Heidenthaller\pwindex{Haidenthaller, Josef 23.\,5.\,1863 Rohrbach in Oberösterreich – 14.\,5.\,1934 Bad Hall@\textsc{Haidenthaller, Josef} (23.\,5.\,1863 Rohrbach in Oberösterreich – 14.\,5.\,1934 Bad Hall), \emph{Mediziner}|pw}{ / }Wohnung Johannesgasse 3. u.
                                          5\oindex{Wien@\textbf{Wien}!I., Innere Stadt@\textbf{I., Innere Stadt}!Johannesgasse@\textbf{Johannesgasse}, \emph{Straße}|pw}{ / }Berlin\oindex{Berlin@\textbf{Berlin}, \emph{Hauptstadt}|pw}! D\textsuperscript{r}. F. C. Andreas\pwindex{Andreas, Friedrich Carl 14.\,4.\,1846 Jakarta – 3.\,10.\,1930 Göttingen@\textsc{Andreas, Friedrich Carl} (14.\,4.\,1846 Jakarta – 3.\,10.\,1930 Göttingen), \emph{Orientalist}|pw}.{ / }vor. 10 Tagen.«)
\newline{}Handschrift: Bleistift, lateinische Kurrent
\newline{}Schnitzler: mit Bleistift datiert: »\substVorne{}\textsuperscript{2}\substDazwischen{}9\substHinten{} /5 95« und nummeriert: »58« }\toendnotes[C]{\smallbreak}
\pstart
           \noindent{}{\pb}Wir\pwindex{Andreas-Salomé, Lou 12.\,2.\,1861 Sankt Petersburg – 5.\,2.\,1937 Göttingen@\textsc{Andreas-Salomé, Lou} (12.\,2.\,1861 Sankt Petersburg – 5.\,2.\,1937 Göttingen), \emph{Schriftstellerin}|pwv} sind die Strasse längs
               des Hauses (Stelzer\oindex{Wien@\textbf{Wien}!XXIII., Liesing@\textbf{XXIII., Liesing}!Gasthaus Stelzer@\textbf{Gasthaus Stelzer}, \emph{Lokal}|pw}) (Badgasse\oindex{Wien@\textbf{Wien}!XXIII., Liesing@\textbf{XXIII., Liesing}!Badgasse@\textbf{Badgasse}, \emph{Straße}|pw}) geradeaus in den Wald gegangen und halten uns i{\geminationm}er an der Mauer des Kalksburger Convicts\oindex{Wien@\textbf{Wien}!XXIII., Liesing@\textbf{XXIII., Liesing}!Kollegium Kalksburg@\textbf{Kollegium Kalksburg}, \emph{Schule}|pw} –\pend
           \pstart \spacefill\mbox{Richard}\pend{}
\pstart
           \noindent{}Herrn D\textsuperscript{r} Arthur Schnitzler\pend
           \selectlanguage{ngerman}\endnumbering\briefempfaengerindex{Schnitzler, Arthur@\textsc{Schnitzler, Arthur}!zzzBeer-Hofmann, Richard@\emph{von Richard Beer-Hofmann}!1895-05-091@{{[}9. 5. 1895{]}}|)be}\mylabel{L00439h}  \newcommand{\dateiname}{L00439}\newcommand{\titel}{Richard Beer-Hofmann an Arthur Schnitzler, [9. 5. 1895]}\newcommand{\editorInnen}{Martin Anton Müller und Gerd-Hermann Susen}%% latex-leseansicht-abspann.tex
%% Abspann für die Leseansicht.
%% Der Schalter \ifkorrekturansicht ist bereits durch den Vorspann gesetzt.

%% latex-abspann.tex
%% Gemeinsamer Abspann für Korrekturansicht und Leseansicht.
%% Setzt den Schalter \ifkorrekturansicht voraus (gesetzt in den
%% einbindenden Dateien latex-korrekturansicht-abspann.tex bzw.
%% latex-leseansicht-abspann.tex).
%% ---------------------------------------------------------------

\normalsize

% Das esempio-Environment wird nur in der Leseansicht benötigt
\ifkorrekturansicht\else
\newenvironment{esempio}[3]%
{
    \vspace{1.5ex}
    \rlap{\underline{#1}}
    \par
    \setlength{\parindent}{0cm}
    \nopagebreak
    \leftskip=#2cm
    \rightskip=#3cm
}
{
    \par
}
\fi

\doendnotes{C}
\bigskip
\vfill

\clearpage

\footnotesize

\ifkorrekturansicht
  \lohead{\textsc{register}}
\fi

% theindex-Environment neu definieren ohne reledmac
\makeatletter
\renewenvironment{theindex}{%
  \ifkorrekturansicht
    \section*{\indexname}%
  \else
    \subsubsection*{Index der erwähnten Entitäten}%
  \fi
  \setlength{\parindent}{0pt}%
  \setlength{\parskip}{0pt plus 0.3pt}%
  \let\item\@idxitem
}{%
  \ifkorrekturansicht\clearpage\fi
}
\makeatother

\IfFileExists{\jobname-pw.ind}{\input{\jobname-pw.ind}}{}

% Quellenangabe nur in der Leseansicht
\ifkorrekturansicht\else
% Fallback-Definitionen, falls die .tex-Datei \titel etc. nicht gesetzt hat
\providecommand{\titel}{}
\providecommand{\editorInnen}{}
\providecommand{\dateiname}{\jobname}

\vspace{3cm}

\vfill

\footnotesize
\textsc{Quelle}: \titel. Herausgegeben von {\editorInnen}. In: \emph{Arthur Schnitzler: Briefwechsel mit Autorinnen und Autoren}.
 Digitale Edition, https://schnitzler-briefe.acdh.oeaw.ac.at/{\dateiname}.html (Stand \today)
\fi

\end{document}


