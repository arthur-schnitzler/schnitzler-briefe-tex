%% latex-leseansicht-vorspann.tex
%% Vorspann für die Leseansicht.
%% Lädt die gemeinsame Datei latex-vorspann.tex mit nicht gesetztem Schalter.

\newif\ifkorrekturansicht
\korrekturansichtfalse

\input{../tex-inputs/latex-vorspann}


         
         \newcommand{\erwaehntePersonen}{Personen: Friedrich Carl Andreas, Lou Andreas-Salomé, Josef Haidenthaller}
         \newcommand{\erwaehnteOrte}{Orte: Badgasse, Berlin, Gasthaus Stelzer, Johannesgasse, Kollegium Kalksburg, Wien}
         \newcommand{\erwaehnteWerke}{
               \section[Richard Beer-Hofmann an Arthur Schnitzler, {[}9. 5. 1895{]}]{ Richard Beer-Hofmann an Arthur Schnitzler, {[}9. 5. 1895{]}}\nopagebreak\mylabel{v}\rehead{ }\begin{ledgroupsized}[t]{13cm}\normalsize\beginnumbering \toendnotes[C]{\smallbreak\pagebreak[2]} \Standort{CUL, Schnitzler, B 8.}
\physDesc{Brief, 1 Blatt (Auf der Rückseite von Beer-Hofmann mit Bleistift: »DrJosef HeidenthallerWohnung Johannesgasse 3. u. 5Berlin! Dr. F. C. Andreas.vor. 10 Tagen.«), 1 Seite
\newline{}Handschrift: Bleistift, lateinische Kurrent
\newline{}Schnitzler: mit Bleistift datiert: »\substVorne{}\textsuperscript{2}\substDazwischen{}9\substHinten{} /5 95« und nummeriert: »58« }\toendnotes[C]{\smallbreak}\pstart
           \noindent{}{\pb}Wir\pwindex{Andreas-Salome, Lou 12.02.1861 – 05.02.1937@\textsc{Andreas-Salomé, Lou} (12.02.1861 – 05.02.1937), \emph{Schriftstellerin}|pwv} sind die Strasse längs des
               Hauses (Stelzer\oindex{Gasthaus Stelzer@\textbf{Gasthaus Stelzer}|pw}) (Badgasse\oindex{Badgasse@\textbf{Badgasse}|pw}) geradeaus in den Wald gegangen und halten uns i{\geminationm}er an der Mauer des Kalksburger Convicts\oindex{Kollegium Kalksburg@\textbf{Kollegium Kalksburg}|pw} –\pend
           \pstart \spacefill\mbox{Richard}\pend{}\pstart
           \noindent{}Herrn D\textsuperscript{r} Arthur Schnitzler\pend
           
         
         \endnumbering\mylabel{h}\end{ledgroupsized}  \newcommand{\dateiname}{L00439}\newcommand{\titel}{Richard Beer-Hofmann an Arthur Schnitzler, [9. 5. 1895]}\newcommand{\editorInnen}{Martin Anton Müller und Gerd-Hermann Susen}%% latex-leseansicht-abspann.tex
%% Abspann für die Leseansicht.
%% Der Schalter \ifkorrekturansicht ist bereits durch den Vorspann gesetzt.

%% latex-abspann.tex
%% Gemeinsamer Abspann für Korrekturansicht und Leseansicht.
%% Setzt den Schalter \ifkorrekturansicht voraus (gesetzt in den
%% einbindenden Dateien latex-korrekturansicht-abspann.tex bzw.
%% latex-leseansicht-abspann.tex).
%% ---------------------------------------------------------------

\normalsize

% Das esempio-Environment wird nur in der Leseansicht benötigt
\ifkorrekturansicht\else
\newenvironment{esempio}[3]%
{
    \vspace{1.5ex}
    \rlap{\underline{#1}}
    \par
    \setlength{\parindent}{0cm}
    \nopagebreak
    \leftskip=#2cm
    \rightskip=#3cm
}
{
    \par
}
\fi

\doendnotes{C}
\bigskip
\vfill

\clearpage

\footnotesize

\ifkorrekturansicht
  \lohead{\textsc{register}}
\fi

% theindex-Environment neu definieren ohne reledmac
\makeatletter
\renewenvironment{theindex}{%
  \ifkorrekturansicht
    \section*{\indexname}%
  \else
    \subsubsection*{Index der erwähnten Entitäten}%
  \fi
  \setlength{\parindent}{0pt}%
  \setlength{\parskip}{0pt plus 0.3pt}%
  \let\item\@idxitem
}{%
  \ifkorrekturansicht\clearpage\fi
}
\makeatother

\IfFileExists{\jobname-pw.ind}{\input{\jobname-pw.ind}}{}

% Quellenangabe nur in der Leseansicht
\ifkorrekturansicht\else
% Fallback-Definitionen, falls die .tex-Datei \titel etc. nicht gesetzt hat
\providecommand{\titel}{}
\providecommand{\editorInnen}{}
\providecommand{\dateiname}{\jobname}

\vspace{3cm}

\vfill

\footnotesize
\textsc{Quelle}: \titel. Herausgegeben von {\editorInnen}. In: \emph{Arthur Schnitzler: Briefwechsel mit Autorinnen und Autoren}.
 Digitale Edition, https://schnitzler-briefe.acdh.oeaw.ac.at/{\dateiname}.html (Stand \today)
\fi

\end{document}


      