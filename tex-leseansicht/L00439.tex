%% latex-korrekturansicht-vorspann.tex
%% Vorspann für die Korrekturansicht.
%% Lädt die gemeinsame Datei latex-vorspann.tex mit gesetztem Schalter.

\newif\ifkorrekturansicht
\korrekturansichttrue

\input{../tex-inputs/latex-vorspann}


\section[Richard Beer-Hofmann an Arthur Schnitzler, {[}9. 5. 1895{]}]{L00439 Richard Beer-Hofmann an Arthur Schnitzler, {[}9. 5. 1895{]}}
\nopagebreak\mylabel{L00439v}
\rehead{ }\normalsize\beginnumbering\briefempfaengerindex{Schnitzler, Arthur@\textsc{Schnitzler, Arthur}!zzzBeer-Hofmann, Richard@\emph{von Richard Beer-Hofmann}!1895-05-091@{{[}9. 5. 1895{]}}|(be}
\toendnotes[C]{\smallbreak\pagebreak[2]}\Standort{CUL, Schnitzler, B 8.}
\physDesc{Brief, 1 Blatt, 1 Seite, 178 Zeichen (Auf der Rückseite von Beer-Hofmann mit Bleistift: »\noindent{}{\pb}D\textsuperscript{r}{ }\introOben{}Josef\introOben{} Heidenthaller\pwindex{Haidenthaller, Josef 23.5.1863 – 14.5.1934@\textsc{Haidenthaller, Josef} (23.5.1863 – 14.5.1934), \emph{Mediziner/Medizinerin}|pw}{ / }Wohnung Johannesgasse 3. u.
                                          5\oindex{Johannesgasse@\textbf{Johannesgasse}, \emph{Straße (K.STR)}|pw}{ / }Berlin\oindex{Berlin@\textbf{Berlin}, \emph{P.PPLC}|pw}! D\textsuperscript{r}. F. C. Andreas\pwindex{Andreas, Friedrich Carl 14.04.1846 – 03.10.1930@\textsc{Andreas, Friedrich Carl} (14.04.1846 – 03.10.1930), \emph{Orientalist/Orientalistin}|pw}.{ / }vor. 10 Tagen.«)
\newline{}Handschrift: Bleistift, lateinische Kurrent
\newline{}Schnitzler: mit Bleistift datiert: »\substVorne{}\textsuperscript{2}\substDazwischen{}9\substHinten{} /5 95« und nummeriert: »58« }\toendnotes[C]{\smallbreak}
\pstart
           \noindent{}{\pb}Wir\pwindex{Andreas-Salome, Lou 12.02.1861 – 05.02.1937@\textsc{Andreas-Salomé, Lou} (12.02.1861 – 05.02.1937), \emph{Schriftsteller/Schriftstellerin}|pwv} sind die Strasse längs
               des Hauses (Stelzer\oindex{Gasthaus Stelzer@\textbf{Gasthaus Stelzer}, \emph{Lokal (K.LKL)}|pw}) (Badgasse\oindex{Badgasse@\textbf{Badgasse}, \emph{Straße (K.STR)}|pw}) geradeaus in den Wald gegangen und halten uns i{\geminationm}er an der Mauer des Kalksburger Convicts\oindex{Kollegium Kalksburg@\textbf{Kollegium Kalksburg}, \emph{Schule (K.SCH)}|pw} –\pend
           \pstart \spacefill\mbox{Richard}\pend{}
\pstart
           \noindent{}Herrn D\textsuperscript{r} Arthur Schnitzler\pend
           \selectlanguage{ngerman}\endnumbering\briefempfaengerindex{Schnitzler, Arthur@\textsc{Schnitzler, Arthur}!zzzBeer-Hofmann, Richard@\emph{von Richard Beer-Hofmann}!1895-05-091@{{[}9. 5. 1895{]}}|)be}\mylabel{L00439h}  \normalsize

\doendnotes{C}
\bigskip
\vfill

\clearpage

\footnotesize

\lohead{\textsc{register}}

% Definiere theindex-Environment komplett neu ohne reledmac
\makeatletter
\renewenvironment{theindex}{%
  \section*{\indexname}%
  \setlength{\parindent}{0pt}%
  \setlength{\parskip}{0pt plus 0.3pt}%
  \let\item\@idxitem
}{%
  \clearpage
}
\makeatother

\IfFileExists{\jobname-pw.ind}{\input{\jobname-pw.ind}}{}

\end{document}

      