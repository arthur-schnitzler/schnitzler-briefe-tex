%% latex-leseansicht-vorspann.tex
%% Vorspann für die Leseansicht.
%% Lädt die gemeinsame Datei latex-vorspann.tex mit nicht gesetztem Schalter.

\newif\ifkorrekturansicht
\korrekturansichtfalse

\input{../tex-inputs/latex-vorspann}


\section[Arthur Schnitzler an Max Burckhard, {[}Mitte Februar 1894?{]}]{L00299 Arthur Schnitzler an Max Burckhard, {[}Mitte Februar 1894?{]}}
\nopagebreak\mylabel{L00299v}
\rehead{ }\normalsize\beginnumbering\briefempfaengerindex{Burckhard, Max Eugen@\textsc{Burckhard, Max Eugen}!zzzSchnitzler, Arthur@\emph{von Arthur Schnitzler}!1894-02-141@{{[}Mitte Februar 1894?{]}}|(be}
\toendnotes[C]{\smallbreak\pagebreak[2]}
\correspDesc{Versand  durch Arthur Schnitzler am [Mitte Februar 1894?] in Wien
\newline{}Erhalt  durch Max Burckhard im Zeitraum [14. 2. 1894
                  – 18. 2. 1894?] in Wien}\toendnotes[C]{\smallbreak}
\buchAlsQuelle{\pwindex{Glossy, Karl 7.\,3.\,1848 Wien – 9.\,9.\,1937 ebd.@\textsc{Glossy, Karl} (7.\,3.\,1848 Wien – 9.\,9.\,1937 ebd.), \emph{Schriftsteller, Museumsleiter, Zensurbeirat}!Schnitzlers Einzug ins Burgtheater@\strich\emph{Schnitzlers Einzug ins Burgtheater}|pwk}\pwindex{Neue Freie Presse@\emph{Neue Freie Presse}|pwk}Karl Glossy: \emph{Schnitzlers Einzug ins Burgtheater. Unbekannte Briefe des Dichters.} In: \emph{Neue Freie Presse}, Nr. 24162, 19. 12. 1931, S. 14.}
\buchAbdrucke{\weitereDrucke{1) \pwindex{Glossy, Karl 7.\,3.\,1848 Wien – 9.\,9.\,1937 ebd.@\textsc{Glossy, Karl} (7.\,3.\,1848 Wien – 9.\,9.\,1937 ebd.), \emph{Schriftsteller, Museumsleiter, Zensurbeirat}!Schnitzlers Einzug ins Burgtheater@\strich\emph{Schnitzlers Einzug ins Burgtheater}|pwk}Karl Glossy: \emph{Schnitzlers Einzug ins Burgtheater. Unbekannte Briefe des Dichters.} In: \emph{Wiener Studien und Dokumente}. Zum 85. Geburtstag des Verfassers herausgegeben von seinen Freunden. Wien: \emph{Steyrermühl} 1933, S. 166–168.} \weitereDrucke{2) Hans-Ulrich Lindken: \emph{Arthur Schnitzler. Aspekte und Akzente. Materialien zu Leben
                        und Werk}. Frankfurt am Main, Bern, Göttingen: \emph{Peter Lang} 1984, S. 243–246 (Europäische Hochschulschriften, Reihe 1, Deutsche Sprache und
                        Literatur, 754).} }\toendnotes[C]{\smallbreak}
\pstart
           \noindent{}{\pb}\so{Schnitzler an Burckhard}, \label{K_L00299-1v}\edtext{1894}{\lemma{\textnormal{\emph{1894}}}\Cendnote{\textnormal{Die Datierung folgt der Annahme, dass
                     Schnitzler{ }\emph{Anatol}\pwindex{Schnitzler, Arthur 15.\,5.\,1862 Wien – 21.\,10.\,1931 ebd.@\textsc{Schnitzler, Arthur} (15.\,5.\,1862 Wien – 21.\,10.\,1931 ebd.), \emph{Schriftsteller, Mediziner}!Anatol@\strich\emph{Anatol}|pwk}, unmittelbar nachdem ihm Burckhard\pwindex{Burckhard, Max Eugen 14.\,7.\,1854 Korneuburg – 16.\,3.\,1912 Wien@\textsc{Burckhard, Max Eugen} (14.\,7.\,1854 Korneuburg – 16.\,3.\,1912 Wien), \emph{Schriftsteller, Rechtswissenschaftler, Theaterleiter}|pwk} mitgeteilt hatte, das Buch nicht
                  erhalten zu haben, mit diesem Begleitschreiben neuerlich zukommen ließ.}}}\label{K_L00299-1}:
               »Sehr verehrter Herr Direktor! Die drei Stücke\pwindex{Schnitzler, Arthur 15.\,5.\,1862 Wien – 21.\,10.\,1931 ebd.@\textsc{Schnitzler, Arthur} (15.\,5.\,1862 Wien – 21.\,10.\,1931 ebd.), \emph{Schriftsteller, Mediziner}!Frage an das Schicksal@\strich\emph{Die Frage an das Schicksal}|pwv}\pwindex{Schnitzler, Arthur 15.\,5.\,1862 Wien – 21.\,10.\,1931 ebd.@\textsc{Schnitzler, Arthur} (15.\,5.\,1862 Wien – 21.\,10.\,1931 ebd.), \emph{Schriftsteller, Mediziner}!Abschiedssouper@\strich\emph{Abschiedssouper}|pwv}\pwindex{Schnitzler, Arthur 15.\,5.\,1862 Wien – 21.\,10.\,1931 ebd.@\textsc{Schnitzler, Arthur} (15.\,5.\,1862 Wien – 21.\,10.\,1931 ebd.), \emph{Schriftsteller, Mediziner}!Episode@\strich\emph{Episode}|pwv}, welche ich für
               aufführbar halte, habe ich bezeichnet. Das letzte, ›Abſchiedsſouper\pwindex{Schnitzler, Arthur 15.\,5.\,1862 Wien – 21.\,10.\,1931 ebd.@\textsc{Schnitzler, Arthur} (15.\,5.\,1862 Wien – 21.\,10.\,1931 ebd.), \emph{Schriftsteller, Mediziner}!Abschiedssouper@\strich\emph{Abschiedssouper}|pw}‹, mag allerdings für eine Hofbühne nicht geeignet{ }ſein; die
               beiden anderen werden Sie möglicherweiſe eines Verſuchs wert finden. Beſonders
               geeignet erſchienen{ }ſie mir anläßlich einer Matinée im Repertoire zu erſcheinen. Für
               den Fall aber, daß Sie die anſpruchsloſen Szenen nicht für aufführbar halten, will
               ich wenigſtens hoffen, daß Sie die Lektüre derſelben nicht allzuſehr langweilt. Mit
               ausgezeichneter Hochachtung Ihr{ }ſehr ergebener Dr. Arthur Schnitzler.«\pend
           \selectlanguage{ngerman}\endnumbering\briefempfaengerindex{Burckhard, Max Eugen@\textsc{Burckhard, Max Eugen}!zzzSchnitzler, Arthur@\emph{von Arthur Schnitzler}!1894-02-141@{{[}Mitte Februar 1894?{]}}|)be}\mylabel{L00299h}  \newcommand{\dateiname}{L00299}\newcommand{\titel}{Arthur Schnitzler an Max Burckhard, [Mitte Februar 1894?]}\newcommand{\editorInnen}{Martin Anton Müller und Gerd-Hermann Susen}%% latex-leseansicht-abspann.tex
%% Abspann für die Leseansicht.
%% Der Schalter \ifkorrekturansicht ist bereits durch den Vorspann gesetzt.

%% latex-abspann.tex
%% Gemeinsamer Abspann für Korrekturansicht und Leseansicht.
%% Setzt den Schalter \ifkorrekturansicht voraus (gesetzt in den
%% einbindenden Dateien latex-korrekturansicht-abspann.tex bzw.
%% latex-leseansicht-abspann.tex).
%% ---------------------------------------------------------------

\normalsize

% Das esempio-Environment wird nur in der Leseansicht benötigt
\ifkorrekturansicht\else
\newenvironment{esempio}[3]%
{
    \vspace{1.5ex}
    \rlap{\underline{#1}}
    \par
    \setlength{\parindent}{0cm}
    \nopagebreak
    \leftskip=#2cm
    \rightskip=#3cm
}
{
    \par
}
\fi

\doendnotes{C}
\bigskip
\vfill

\clearpage

\footnotesize

\ifkorrekturansicht
  \lohead{\textsc{register}}
\fi

% theindex-Environment neu definieren ohne reledmac
\makeatletter
\renewenvironment{theindex}{%
  \ifkorrekturansicht
    \section*{\indexname}%
  \else
    \subsubsection*{Index der erwähnten Entitäten}%
  \fi
  \setlength{\parindent}{0pt}%
  \setlength{\parskip}{0pt plus 0.3pt}%
  \let\item\@idxitem
}{%
  \ifkorrekturansicht\clearpage\fi
}
\makeatother

\IfFileExists{\jobname-pw.ind}{\input{\jobname-pw.ind}}{}

% Quellenangabe nur in der Leseansicht
\ifkorrekturansicht\else
% Fallback-Definitionen, falls die .tex-Datei \titel etc. nicht gesetzt hat
\providecommand{\titel}{}
\providecommand{\editorInnen}{}
\providecommand{\dateiname}{\jobname}

\vspace{3cm}

\vfill

\footnotesize
\textsc{Quelle}: \titel. Herausgegeben von {\editorInnen}. In: \emph{Arthur Schnitzler: Briefwechsel mit Autorinnen und Autoren}.
 Digitale Edition, https://schnitzler-briefe.acdh.oeaw.ac.at/{\dateiname}.html (Stand \today)
\fi

\end{document}


