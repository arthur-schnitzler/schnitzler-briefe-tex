%% latex-leseansicht-vorspann.tex
%% Vorspann für die Leseansicht.
%% Lädt die gemeinsame Datei latex-vorspann.tex mit nicht gesetztem Schalter.

\newif\ifkorrekturansicht
\korrekturansichtfalse

\input{../tex-inputs/latex-vorspann}


         
         \newcommand{\erwaehntePersonen}{Personen: Max Eugen Burckhard, Karl Glossy}
         \newcommand{\erwaehnteInstitutionen}{}
         \newcommand{\erwaehnteOrte}{Orte: Wien}
         \newcommand{\erwaehnteWerke}{Werke: Abschiedssouper, Anatol, Die Frage an das Schicksal, Episode, Neue Freie Presse, Schnitzlers Einzug ins Burgtheater}
               \section[Arthur Schnitzler an Max Burckhard, {[}Mitte Februar 1894?{]}]{ Arthur Schnitzler an Max Burckhard, {[}Mitte Februar 1894?{]}}\nopagebreak\mylabel{v}\rehead{ }\begin{ledgroupsized}[t]{13cm}\normalsize\beginnumbering \toendnotes[C]{\smallbreak\pagebreak[2]} \buchAlsQuelle{\pwindex{Glossy, Karl 07.03.1848 – 09.09.1937@\textsc{Glossy, Karl} (07.03.1848 – 09.09.1937), \emph{Schriftsteller, Museumsleiter, Zensurbeirat}!Schnitzlers Einzug ins Burgtheater19. 12. 1931@\strich\emph{Schnitzlers Einzug ins Burgtheater} {[}19. 12. 1931{]}|pwk}\pwindex{Neue Freie Presse1864 – 1939@\emph{Neue Freie Presse} {[}1864 – 1939{]}|pwk}Karl Glossy: \emph{Schnitzlers Einzug ins Burgtheater. Unbekannte Briefe des Dichters.} In: \emph{Neue Freie Presse}, Nr. 24162, 19. 12. 1931, S. 14.}\buchAbdrucke{\weitereDrucke{1) \pwindex{Glossy, Karl 07.03.1848 – 09.09.1937@\textsc{Glossy, Karl} (07.03.1848 – 09.09.1937), \emph{Schriftsteller, Museumsleiter, Zensurbeirat}!Schnitzlers Einzug ins Burgtheater19. 12. 1931@\strich\emph{Schnitzlers Einzug ins Burgtheater} {[}19. 12. 1931{]}|pwk}Karl Glossy: \emph{Schnitzlers Einzug ins Burgtheater. Unbekannte Briefe des Dichters.} In: \emph{Wiener Studien und Dokumente}. Zum 85. Geburtstag des Verfassers hg. von seinen Freunden. Wien: \emph{Steyrermühl} 1933, S. 166–168.} \weitereDrucke{2) Hans-Ulrich Lindken: \emph{Arthur Schnitzler. Aspekte und Akzente. Materialien zu Leben
                        und Werk}. Frankfurt am Main, Bern, Göttingen: \emph{Peter Lang} 1984, S. 243–246 (Europäische Hochschulschriften, Reihe 1, Deutsche Sprache und
                        Literatur, 754).} }\toendnotes[C]{\smallbreak}\pstart
           \noindent{}{\pb}\so{Schnitzler an Burckhard}, \label{K_L00299_1v}\edtext{1894}{\lemma{\textnormal{\emph{1894}}}\Cendnote{\textnormal{Die Datierung folgt der Annahme, dass Schnitzler\pwindex{Schnitzler, Arthur 15.05.1862 – 21.10.1931@\textsc{Schnitzler, Arthur} (15.05.1862 – 21.10.1931), \emph{Schriftsteller, Mediziner}|pwk}{ }\emph{Anatol}\pwindex{Schnitzler, Arthur 15.05.1862 – 21.10.1931@\textsc{Schnitzler, Arthur} (15.05.1862 – 21.10.1931), \emph{Schriftsteller, Mediziner}!Anatol1892-10-29@\strich\emph{Anatol} {[}1892-10-29{]}|pwk}, unmittelbar nachdem ihm Burckhard\pwindex{Burckhard, Max Eugen 14.07.1854 – 16.03.1912@\textsc{Burckhard, Max Eugen} (14.07.1854 – 16.03.1912), \emph{Schriftsteller, Rechtswissenschaftler, Theaterleiter}|pwk} mitgeteilt hatte, das Buch nicht erhalten zu haben,
                  neuerlich mit diesem Begleitschreiben zukommen ließ.}}}\label{K_L00299_1h}: »Sehr verehrter Herr
               Direktor! Die drei Stücke\pwindex{Schnitzler, Arthur 15.05.1862 – 21.10.1931@\textsc{Schnitzler, Arthur} (15.05.1862 – 21.10.1931), \emph{Schriftsteller, Mediziner}!Frage an das Schicksal01. 05. 1890@\strich\emph{Die Frage an das Schicksal} {[}01. 05. 1890{]}|pwv}\pwindex{Schnitzler, Arthur 15.05.1862 – 21.10.1931@\textsc{Schnitzler, Arthur} (15.05.1862 – 21.10.1931), \emph{Schriftsteller, Mediziner}!Abschiedssouper1892@\strich\emph{Abschiedssouper} {[}1892{]}|pwv}\pwindex{Schnitzler, Arthur 15.05.1862 – 21.10.1931@\textsc{Schnitzler, Arthur} (15.05.1862 – 21.10.1931), \emph{Schriftsteller, Mediziner}!Episode8. 09. 1889@\strich\emph{Episode} {[}8. 09. 1889{]}|pwv}, welche ich für aufführbar halte, habe ich bezeichnet. Das letzte,
                  ›Abſchiedsſouper\pwindex{Schnitzler, Arthur 15.05.1862 – 21.10.1931@\textsc{Schnitzler, Arthur} (15.05.1862 – 21.10.1931), \emph{Schriftsteller, Mediziner}!Abschiedssouper1892@\strich\emph{Abschiedssouper} {[}1892{]}|pw}‹, mag allerdings für eine
               Hofbühne nicht geeignet ſein; die beiden anderen werden Sie möglicherweiſe eines
               Verſuchs wert finden. Beſonders geeignet erſchienen ſie mir anläßlich einer Matinée
               im Repertoire zu erſcheinen. Für den Fall aber, daß Sie die anſpruchsloſen Szenen
               nicht für aufführbar halten, will ich wenigſtens hoffen, daß Sie die Lektüre
               derſelben nicht allzuſehr langweilt. Mit ausgezeichneter Hochachtung Ihr ſehr
               ergebener Dr. Arthur Schnitzler.«\pend
           
         
         \endnumbering\mylabel{h}\end{ledgroupsized}  \newcommand{\dateiname}{L00299}\newcommand{\titel}{Arthur Schnitzler an Max Burckhard, [Mitte Februar 1894?]}\newcommand{\editorInnen}{Martin Anton Müller und Gerd-Hermann Susen}%% latex-leseansicht-abspann.tex
%% Abspann für die Leseansicht.
%% Der Schalter \ifkorrekturansicht ist bereits durch den Vorspann gesetzt.

%% latex-abspann.tex
%% Gemeinsamer Abspann für Korrekturansicht und Leseansicht.
%% Setzt den Schalter \ifkorrekturansicht voraus (gesetzt in den
%% einbindenden Dateien latex-korrekturansicht-abspann.tex bzw.
%% latex-leseansicht-abspann.tex).
%% ---------------------------------------------------------------

\normalsize

% Das esempio-Environment wird nur in der Leseansicht benötigt
\ifkorrekturansicht\else
\newenvironment{esempio}[3]%
{
    \vspace{1.5ex}
    \rlap{\underline{#1}}
    \par
    \setlength{\parindent}{0cm}
    \nopagebreak
    \leftskip=#2cm
    \rightskip=#3cm
}
{
    \par
}
\fi

\doendnotes{C}
\bigskip
\vfill

\clearpage

\footnotesize

\ifkorrekturansicht
  \lohead{\textsc{register}}
\fi

% theindex-Environment neu definieren ohne reledmac
\makeatletter
\renewenvironment{theindex}{%
  \ifkorrekturansicht
    \section*{\indexname}%
  \else
    \subsubsection*{Index der erwähnten Entitäten}%
  \fi
  \setlength{\parindent}{0pt}%
  \setlength{\parskip}{0pt plus 0.3pt}%
  \let\item\@idxitem
}{%
  \ifkorrekturansicht\clearpage\fi
}
\makeatother

\IfFileExists{\jobname-pw.ind}{\input{\jobname-pw.ind}}{}

% Quellenangabe nur in der Leseansicht
\ifkorrekturansicht\else
% Fallback-Definitionen, falls die .tex-Datei \titel etc. nicht gesetzt hat
\providecommand{\titel}{}
\providecommand{\editorInnen}{}
\providecommand{\dateiname}{\jobname}

\vspace{3cm}

\vfill

\footnotesize
\textsc{Quelle}: \titel. Herausgegeben von {\editorInnen}. In: \emph{Arthur Schnitzler: Briefwechsel mit Autorinnen und Autoren}.
 Digitale Edition, https://schnitzler-briefe.acdh.oeaw.ac.at/{\dateiname}.html (Stand \today)
\fi

\end{document}


      