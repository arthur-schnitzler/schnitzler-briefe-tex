%% latex-korrekturansicht-vorspann.tex
%% Vorspann für die Korrekturansicht.
%% Lädt die gemeinsame Datei latex-vorspann.tex mit gesetztem Schalter.

\newif\ifkorrekturansicht
\korrekturansichttrue

\input{../tex-inputs/latex-vorspann}


\section[Arthur Schnitzler an Hermann Bahr, 6. 9. 1898]{L00845 Arthur Schnitzler an Hermann Bahr, 6. 9. 1898}
\nopagebreak\mylabel{L00845v}
\rehead{ }\normalsize\beginnumbering\briefempfaengerindex{Bahr, Hermann@\textsc{Bahr, Hermann}!zzzSchnitzler, Arthur@\emph{von Arthur Schnitzler}!1898-09-061@{6. 9. 1898}|(be}
\toendnotes[C]{\smallbreak\pagebreak[2]}\Standort{TMW, HS AM 60158 Ba.}
\physDesc{Briefkarte, 272 Zeichen
\newline{}Handschrift: schwarze Tinte, deutsche Kurrent
\newline{}Ordnung: Lochung }
\buchAbdrucke{\weitereDrucke{1) Arthur Schnitzler: \emph{The Letters of Arthur Schnitzler to Hermann Bahr}. Chapel Hill: \emph{The University of North Carolina Press} 1978, S. 64.} \weitereDrucke{2) Hermann Bahr, Arthur Schnitzler: \emph{Briefwechsel, Aufzeichnungen, Dokumente (1891–1931)}. Göttingen: \emph{Wallstein} 2018, S. 163.} }\toendnotes[C]{\smallbreak}
\pstart
           \noindent{}{\pb}Lieber Hermann, ich war neulich in der Redaction\oindex{Redaktion der »Zeit«@\textbf{Redaktion der »Zeit«}, \emph{Redaktionsgebäude (K.RDK)}|pw} u habe
               dich nicht getroffen. Auf dieſem Weg alſo meine herzlichſte Theilnahme zu dem \label{K_L00845-1v}\edtext{Hinſcheiden deines Vaters\pwindex{Bahr, Alois 11.04.1834 – 05.09.1898@\textsc{Bahr, Alois} (11.04.1834 – 05.09.1898), \emph{Notar/Notarin, Politiker/Politikerin}|pwv}}{\lemma{\textnormal{\emph{Hinſcheiden … Vaters}}}\Cendnote{\textnormal{Alois Bahr\pwindex{Bahr, Alois 11.04.1834 – 05.09.1898@\textsc{Bahr, Alois} (11.04.1834 – 05.09.1898), \emph{Notar/Notarin, Politiker/Politikerin}|pwk} war am 5. 9. 1898
                  in Salzburg\oindex{Salzburg@\textbf{Salzburg}, \emph{A.ADM2}|pwk} gestorben.}}}\label{K_L00845-1}.\pend
           
\pstart
           Wenn du wieder in Wien\oindex{Wien@\textbf{Wien}, \emph{A.ADM2}|pw} biſt, ſehen wir uns
               hoffentlich bald. Mit den herzlichſten Grüßen dein\pend
           \pstart \spacefill\mbox{Arthur Schnitzler}\pend{}
\pstart
           6. 9. 98.\pend
           \selectlanguage{ngerman}\endnumbering\briefempfaengerindex{Bahr, Hermann@\textsc{Bahr, Hermann}!zzzSchnitzler, Arthur@\emph{von Arthur Schnitzler}!1898-09-061@{6. 9. 1898}|)be}\mylabel{L00845h}  \normalsize

\doendnotes{C}
\bigskip
\vfill

\clearpage

\footnotesize

\lohead{\textsc{register}}

% Definiere theindex-Environment komplett neu ohne reledmac
\makeatletter
\renewenvironment{theindex}{%
  \section*{\indexname}%
  \setlength{\parindent}{0pt}%
  \setlength{\parskip}{0pt plus 0.3pt}%
  \let\item\@idxitem
}{%
  \clearpage
}
\makeatother

\IfFileExists{\jobname-pw.ind}{\input{\jobname-pw.ind}}{}

\end{document}

      