%% latex-leseansicht-vorspann.tex
%% Vorspann für die Leseansicht.
%% Lädt die gemeinsame Datei latex-vorspann.tex mit nicht gesetztem Schalter.

\newif\ifkorrekturansicht
\korrekturansichtfalse

\input{../tex-inputs/latex-vorspann}


         
         \renewcommand{\erwaehntePersonen}{Personen: Hermann Bahr, Alois Bahr}
         \renewcommand{\erwaehnteOrte}{Orte: Redaktion der »Zeit«, Salzburg, Wien}
         \renewcommand{\erwaehnteWerke}{}
               \section[Arthur Schnitzler an Hermann Bahr, 6. 9. 1898]{ Arthur Schnitzler an Hermann Bahr, 6. 9. 1898}\nopagebreak\mylabel{v}\rehead{ }\begin{ledgroupsized}[t]{13cm}\normalsize\beginnumbering \toendnotes[C]{\smallbreak\pagebreak[2]} \Standort{TMW, HS AM 60158 Ba.}
\physDesc{Briefkarte, 272 Zeichen
\newline{}Handschrift: schwarze Tinte, deutsche Kurrent
\newline{}Ordnung: Lochung }\buchAbdrucke{\weitereDrucke{1) \emph{6. 9. 1898, Abschrift.} In: Arthur Schnitzler: \emph{The Letters of Arthur Schnitzler to Hermann Bahr}. Edited, annotated, and with an introduction, by Donald G.
                        Daviau. Chapel Hill: \emph{The University of North Carolina Press} 1978, S. 64 (University of North Carolina studies in the Germanic languages
                        and literatures, 89).} \weitereDrucke{2) Hermann Bahr, Arthur Schnitzler: \emph{Briefwechsel, Aufzeichnungen, Dokumente (1891–1931)}. Hg. Kurt Ifkovits und Martin Anton Müller. Göttingen: \emph{Wallstein} 2018, S. 163.} }\toendnotes[C]{\smallbreak}\pstart
           \noindent{}{\pb}Lieber Hermann,
               ich war neulich in der Redaction\oindex{Redaktion der »Zeit«@\textbf{Redaktion der »Zeit«}|pw} u habe
               dich nicht getroffen. Auf dieſem Weg alſo meine herzlichſte Theilnahme zu dem \label{K_L00845_1v}\edtext{Hinſcheiden deines Vaters\pwindex{Bahr, Alois 11.04.1834 – 05.09.1898@\textsc{Bahr, Alois} (11.04.1834 – 05.09.1898), \emph{Notar, Politiker}|pwv}}{\lemma{\textnormal{\emph{Hinſcheiden … Vaters}}}\Cendnote{\textnormal{Alois Bahr\pwindex{Bahr, Alois 11.04.1834 – 05.09.1898@\textsc{Bahr, Alois} (11.04.1834 – 05.09.1898), \emph{Notar, Politiker}|pwk} starb am 5. 9. 1898
                  in Salzburg\oindex{Salzburg@\textbf{Salzburg}|pwk}.}}}\label{K_L00845_1h}.\pend
           \pstart
           Wenn du wieder in Wien\oindex{Wien@\textbf{Wien}|pw} biſt, ſehen wir uns
               hoffentlich bald. Mit den herzlichſten Grüßen dein\pend
           \pstart \spacefill\mbox{Arthur Schnitzler}\pend{}\pstart
           6. 9. 98.\pend
           
         
         \endnumbering\mylabel{h}\end{ledgroupsized}  \newcommand{\dateiname}{L00845}\newcommand{\titel}{Arthur Schnitzler an Hermann Bahr, 6. 9. 1898}\newcommand{\editorInnen}{ Kurt Ifkovits,  Martin Anton Müller}%% latex-leseansicht-abspann.tex
%% Abspann für die Leseansicht.
%% Der Schalter \ifkorrekturansicht ist bereits durch den Vorspann gesetzt.

%% latex-abspann.tex
%% Gemeinsamer Abspann für Korrekturansicht und Leseansicht.
%% Setzt den Schalter \ifkorrekturansicht voraus (gesetzt in den
%% einbindenden Dateien latex-korrekturansicht-abspann.tex bzw.
%% latex-leseansicht-abspann.tex).
%% ---------------------------------------------------------------

\normalsize

% Das esempio-Environment wird nur in der Leseansicht benötigt
\ifkorrekturansicht\else
\newenvironment{esempio}[3]%
{
    \vspace{1.5ex}
    \rlap{\underline{#1}}
    \par
    \setlength{\parindent}{0cm}
    \nopagebreak
    \leftskip=#2cm
    \rightskip=#3cm
}
{
    \par
}
\fi

\doendnotes{C}
\bigskip
\vfill

\clearpage

\footnotesize

\ifkorrekturansicht
  \lohead{\textsc{register}}
\fi

% theindex-Environment neu definieren ohne reledmac
\makeatletter
\renewenvironment{theindex}{%
  \ifkorrekturansicht
    \section*{\indexname}%
  \else
    \subsubsection*{Index der erwähnten Entitäten}%
  \fi
  \setlength{\parindent}{0pt}%
  \setlength{\parskip}{0pt plus 0.3pt}%
  \let\item\@idxitem
}{%
  \ifkorrekturansicht\clearpage\fi
}
\makeatother

\IfFileExists{\jobname-pw.ind}{\input{\jobname-pw.ind}}{}

% Quellenangabe nur in der Leseansicht
\ifkorrekturansicht\else
% Fallback-Definitionen, falls die .tex-Datei \titel etc. nicht gesetzt hat
\providecommand{\titel}{}
\providecommand{\editorInnen}{}
\providecommand{\dateiname}{\jobname}

\vspace{3cm}

\vfill

\footnotesize
\textsc{Quelle}: \titel. Herausgegeben von {\editorInnen}. In: \emph{Arthur Schnitzler: Briefwechsel mit Autorinnen und Autoren}.
 Digitale Edition, https://schnitzler-briefe.acdh.oeaw.ac.at/{\dateiname}.html (Stand \today)
\fi

\end{document}


      