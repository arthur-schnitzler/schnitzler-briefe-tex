%% latex-korrekturansicht-vorspann.tex
%% Vorspann für die Korrekturansicht.
%% Lädt die gemeinsame Datei latex-vorspann.tex mit gesetztem Schalter.

\newif\ifkorrekturansicht
\korrekturansichttrue

\input{../tex-inputs/latex-vorspann}


\section[Elsa Plessner an Arthur Schnitzler, 6. 12. 1912]{L03733 Elsa Plessner an Arthur Schnitzler, 6. 12. 1912}
\nopagebreak\mylabel{L03733v}
\rehead{ }\normalsize\beginnumbering\briefempfaengerindex{Schnitzler, Arthur@\textsc{Schnitzler, Arthur}!zzzPlessner, Elsa@\emph{von Elsa Plessner}!1912-12-061@{6. 12. 1912}|(be}
\toendnotes[C]{\smallbreak\pagebreak[2]}\Standort{DLA, A:Schnitzler, HS.1985.1.419.}
\physDesc{Kartenbrief, 1 Blatt, 2 Seiten, 752 Zeichen
\newline{}Handschrift: , lateinische Kurrent
\newline{}Schnitzler: 1) eine Unterstreichung  2) beschriftet: »Plessner«}\toendnotes[C]{\smallbreak}
\pstart
           \raggedleft{}{\pb}den 6. Dec. 1912\pend
           
\pstart{}Verehrter Herr Doctor!\pend\vspace{0.5em}
\pstart
           Wie wundervoll schön ist »das weite Land\pwindex{weite Land. Tragikomoedie in fuenf Akten@\emph{Das weite Land. Tragikomödie in fünf Akten}|pw}«! Ich
               kannte es nicht und habe es \label{K_L03733-1v}\edtext{heute zum erstenmal gesehen}{\lemma{\textnormal{\emph{heute … gesehen}}}\Cendnote{\textnormal{Am 6. 12. 1912 wurde \emph{Das weite Land}\pwindex{weite Land. Tragikomoedie in fuenf Akten@\emph{Das weite Land. Tragikomödie in fünf Akten}|pwk} zum 22. Mal seit der Premiere am 14. 10. 1911  am \emph{Burgtheater}\orgindex{Burgtheater@Burgtheater|pwk}
                  gegeben. Damit läßt sich der unbezeichnete Versandort des Briefes als Wien\oindex{Wien@\textbf{Wien}, \emph{A.ADM2}|pwk} bestimmen.}}}\label{K_L03733-1}. Jetzt hat mein Liebling in Ihrem Werk »der einsame Weg\pwindex{einsame Weg. Schauspiel in fuenf Akten@\emph{Der einsame Weg. Schauspiel in fünf Akten}|pw}« einen gefährlichen Rivalen gefunden. Aber
               das schönste ist, dass man sich freuen darf auf viele neue Arbeiten, die noch kommen
                  {\pb}werden.\pend
           
\pstart
           Wenn ich nicht irre, so habe ich vergangene Woche als Einzige den
               zähneknirschenden Grimm, den tückischen Humor bemerkt, der den »gemüthlichen Hofrath«
               aus dem »Prof. Bernhardy\pwindex{Professor Bernhardi. Komoedie in fuenf Akten@\emph{Professor Bernhardi. Komödie in fünf Akten}|pw}« geschaffen hat.
               Prototyp »das »liberale« Oesterreich\oindex{Oesterreich@\textbf{Österreich}, \emph{A.PCLI}|pw}« – Ich sah
               vor meinem inneren Auge eine geballte Künstlerfaust – in Glacéhandschuhen.\pend
           
\pstart
           Nichts für ungut. Heimsuchung mit unerbetenen Meinungsäußerungen. Verbindlichstes an
               Frau Gemahlin\pwindex{Schnitzler, Olga 17.01.1882 – 13.01.1970@\textsc{Schnitzler, Olga} (17.01.1882 – 13.01.1970), \emph{Schauspieler/Schauspielerin, Sänger/Sängerin}|pwv}!\pend
           \pstart \spacefill\mbox{Elsa Ginsberg}\pend{}\selectlanguage{ngerman}\endnumbering\briefempfaengerindex{Schnitzler, Arthur@\textsc{Schnitzler, Arthur}!zzzPlessner, Elsa@\emph{von Elsa Plessner}!1912-12-061@{6. 12. 1912}|)be}\mylabel{L03733h}
\begin{anhang}
\end{anhang}\normalsize

\doendnotes{C}
\bigskip
\vfill

\clearpage

\footnotesize

\lohead{\textsc{register}}

% Definiere theindex-Environment komplett neu ohne reledmac
\makeatletter
\renewenvironment{theindex}{%
  \section*{\indexname}%
  \setlength{\parindent}{0pt}%
  \setlength{\parskip}{0pt plus 0.3pt}%
  \let\item\@idxitem
}{%
  \clearpage
}
\makeatother

\IfFileExists{\jobname-pw.ind}{\input{\jobname-pw.ind}}{}

\end{document}

      