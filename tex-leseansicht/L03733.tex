%% latex-leseansicht-vorspann.tex
%% Vorspann für die Leseansicht.
%% Lädt die gemeinsame Datei latex-vorspann.tex mit nicht gesetztem Schalter.

\newif\ifkorrekturansicht
\korrekturansichtfalse

\input{../tex-inputs/latex-vorspann}


\section[Elsa Ginsberg-Plessner an Arthur Schnitzler, 6. 12. 1912]{L03733 Elsa Ginsberg-Plessner an Arthur Schnitzler, 6. 12. 1912}
\nopagebreak\mylabel{L03733v}
\rehead{ }\normalsize\beginnumbering\briefempfaengerindex{Schnitzler, Arthur@\textsc{Schnitzler, Arthur}!zzzPlessner, Elsa@\emph{von Elsa Plessner}!1912-12-061@{6. 12. 1912}|(be}
\toendnotes[C]{\smallbreak\pagebreak[2]}
\correspDesc{Versand  durch Elsa Plessner am 6. 12. 1912 in Wien
\newline{}Erhalt  durch Arthur Schnitzler im Zeitraum [6. 12. 1912
                  – 10. 12. 1912?] in Wien}\toendnotes[C]{\smallbreak}
\Standort{DLA, A:Schnitzler, HS.1985.1.419.}
\physDesc{Kartenbrief, 755 Zeichen
\newline{}Handschrift: schwarze Tinte, lateinische Kurrent
\newline{}Schnitzler: 1) mit rotem Buntstift eine Unterstreichung  2) mit Bleistift Vermerk: »Plessner«}\toendnotes[C]{\smallbreak}
\pstart
           \raggedleft{}{\pb}den 6. Dec. 1912\pend
           
\pstart{}Verehrter Herr Doctor!\pend\vspace{0.5em}
\pstart
           Wie wundervoll schön ist »das weite Land\pwindex{Schnitzler, Arthur 15. 5. 1862 Wien – 21. 10. 1931 ebd.@\textsc{Schnitzler, Arthur} (15. 5. 1862 Wien – 21. 10. 1931 ebd.), \emph{Schriftsteller, Mediziner}!weite Land. Tragikomödie in fünf Akten@\strich\emph{Das weite Land. Tragikomödie in fünf Akten}|pw}«! Ich
               kannte es nicht und habe es \label{K_L03733-1v}\edtext{heute zum erstenmal gesehen}{\lemma{\textnormal{\emph{heute … gesehen}}}\Cendnote{\textnormal{Am 6. 12. 1912 wurde \emph{Das weite Land}\pwindex{Schnitzler, Arthur 15. 5. 1862 Wien – 21. 10. 1931 ebd.@\textsc{Schnitzler, Arthur} (15. 5. 1862 Wien – 21. 10. 1931 ebd.), \emph{Schriftsteller, Mediziner}!weite Land. Tragikomödie in fünf Akten@\strich\emph{Das weite Land. Tragikomödie in fünf Akten}|pwk} zum 22. Mal seit der Premiere\eventindex{Burgtheater@\textbf{Burgtheater}!Premiere von Das weite Land, 14.10.1911 [I.]@Premiere von Das weite Land, 14.10.1911 [I.]|pwkv} am 14. 10. 1911  am \emph{Burgtheater}\orgindex{Burgtheater@Burgtheater|pwk}{ }gegeben\eventindex{Burgtheater@\textbf{Burgtheater}!22. Aufführung von Das weite Land, 6.12.1912@22. Aufführung von Das weite Land, 6.12.1912|pwkv}. Damit dürfte der nicht genannte Versandort des Briefes mit Wien\oindex{Wien@\textbf{Wien}, \emph{Verwaltungsgebiet}|pwk} zu bestimmen sein.}}}\label{K_L03733-1}. Jetzt hat mein Liebling in Ihrem Werk »der einsame Weg\pwindex{Schnitzler, Arthur 15. 5. 1862 Wien – 21. 10. 1931 ebd.@\textsc{Schnitzler, Arthur} (15. 5. 1862 Wien – 21. 10. 1931 ebd.), \emph{Schriftsteller, Mediziner}!einsame Weg. Schauspiel in fünf Akten@\strich\emph{Der einsame Weg. Schauspiel in fünf Akten}|pw}« einen gefährlichen Rivalen gefunden. Aber
               das schönste ist, dass man sich freuen darf auf viele neue Arbeiten, die noch kommen
               {\pb}werden. – –\pend
           
\pstart
           Wenn ich nicht irre, so habe ich \label{K_L03733-2v}\edtext{vergangene Woche\eventindex{Wien@\textbf{Wien}!Lesung von Professor Bernhardi, 28.11.1912@Lesung von Professor Bernhardi, 28.11.1912|pwv} als Einzige den
               zähneknirschenden Grimm, den tückischen Humor bemerkt, der den »gemüthlichen Hofrath«
               aus dem »Prof. Bernhardy\pwindex{Schnitzler, Arthur 15. 5. 1862 Wien – 21. 10. 1931 ebd.@\textsc{Schnitzler, Arthur} (15. 5. 1862 Wien – 21. 10. 1931 ebd.), \emph{Schriftsteller, Mediziner}!Professor Bernhardi. Komödie in fünf Akten@\strich\emph{Professor Bernhardi. Komödie in fünf Akten}|pw}}{\lemma{\textnormal{\emph{vergangene … Bernhardy}}}\Cendnote{\textnormal{Das scheint eine implizite Aussage zu sein, dass sie
                  der Lesung\eventindex{Wien@\textbf{Wien}!Lesung von Professor Bernhardi, 28.11.1912@Lesung von Professor Bernhardi, 28.11.1912|pwkv} von \emph{Professor Bernhardi}\pwindex{Schnitzler, Arthur 15. 5. 1862 Wien – 21. 10. 1931 ebd.@\textsc{Schnitzler, Arthur} (15. 5. 1862 Wien – 21. 10. 1931 ebd.), \emph{Schriftsteller, Mediziner}!Professor Bernhardi. Komödie in fünf Akten@\strich\emph{Professor Bernhardi. Komödie in fünf Akten}|pwk} durch Ferdinand Onno\pwindex{Onno, Ferdinand 19.\,10.\,1881 Czernowitz – 18.\,8.\,1969 Wien@\textsc{Onno, Ferdinand} (19.\,10.\,1881 Czernowitz – 18.\,8.\,1969 Wien), \emph{Schauspieler}|pwk} beigewohnt hat,
                  die dieser am 28. 11. 1912 im Saal des Architektenvereins\oindex{Wien@\textbf{Wien}!I., Innere Stadt@\textbf{I., Innere Stadt}!Österreichischer Ingenieur- und Architektenverein@\textbf{Österreichischer Ingenieur- und Architektenverein}|pwk} in Wien\oindex{Wien@\textbf{Wien}, \emph{Verwaltungsgebiet}|pwk} gehalten
               hatte. Schnitzler selbst hatte nicht teilgenommen.}}}\label{K_L03733-2}« geschaffen hat.
               Prototyp »das ›liberale‹ Oesterreich\oindex{Österreich@\textbf{Österreich}|pw}« – Ich sah
               vor meinem inneren Auge eine geballte Künstlerfaust – in Glacéhandschuhen.\pend
           
\pstart
           Nichts für ungut. Heimsuchung mit unerbetenen Meinungsäußerungen. Verbindlichstes an
               \label{K_L03733-3v}\edtext{Frau Gemahlin\pwindex{Schnitzler, Olga 17.\,1.\,1882 Wien – 13.\,1.\,1970 Lugano@\textsc{Schnitzler, Olga} (17.\,1.\,1882 Wien – 13.\,1.\,1970 Lugano), \emph{Schauspielerin, Sängerin}|pwv}}{\lemma{\textnormal{\emph{Frau Gemahlin}}}\Cendnote{\textnormal{Die beiden
                  sind sich nachweislich zumindest einmal begegnet, vgl. A. S.: \emph{Tagebuch}, 21. 11. 1909.}}}\label{K_L03733-3}!\pend
           \pstart \spacefill\mbox{Elsa Ginsberg}\pend{}\selectlanguage{ngerman}\endnumbering\briefempfaengerindex{Schnitzler, Arthur@\textsc{Schnitzler, Arthur}!zzzPlessner, Elsa@\emph{von Elsa Plessner}!1912-12-061@{6. 12. 1912}|)be}\mylabel{L03733h}  \newcommand{\dateiname}{L03733}\newcommand{\titel}{Elsa Ginsberg-Plessner an Arthur Schnitzler, 6. 12. 1912}\newcommand{\editorInnen}{Selma Jahnke und Martin Anton Müller}%% latex-leseansicht-abspann.tex
%% Abspann für die Leseansicht.
%% Der Schalter \ifkorrekturansicht ist bereits durch den Vorspann gesetzt.

%% latex-abspann.tex
%% Gemeinsamer Abspann für Korrekturansicht und Leseansicht.
%% Setzt den Schalter \ifkorrekturansicht voraus (gesetzt in den
%% einbindenden Dateien latex-korrekturansicht-abspann.tex bzw.
%% latex-leseansicht-abspann.tex).
%% ---------------------------------------------------------------

\normalsize

% Das esempio-Environment wird nur in der Leseansicht benötigt
\ifkorrekturansicht\else
\newenvironment{esempio}[3]%
{
    \vspace{1.5ex}
    \rlap{\underline{#1}}
    \par
    \setlength{\parindent}{0cm}
    \nopagebreak
    \leftskip=#2cm
    \rightskip=#3cm
}
{
    \par
}
\fi

\doendnotes{C}
\bigskip
\vfill

\clearpage

\footnotesize

\ifkorrekturansicht
  \lohead{\textsc{register}}
\fi

% theindex-Environment neu definieren ohne reledmac
\makeatletter
\renewenvironment{theindex}{%
  \ifkorrekturansicht
    \section*{\indexname}%
  \else
    \subsubsection*{Index der erwähnten Entitäten}%
  \fi
  \setlength{\parindent}{0pt}%
  \setlength{\parskip}{0pt plus 0.3pt}%
  \let\item\@idxitem
}{%
  \ifkorrekturansicht\clearpage\fi
}
\makeatother

\IfFileExists{\jobname-pw.ind}{\input{\jobname-pw.ind}}{}

% Quellenangabe nur in der Leseansicht
\ifkorrekturansicht\else
% Fallback-Definitionen, falls die .tex-Datei \titel etc. nicht gesetzt hat
\providecommand{\titel}{}
\providecommand{\editorInnen}{}
\providecommand{\dateiname}{\jobname}

\vspace{3cm}

\vfill

\footnotesize
\textsc{Quelle}: \titel. Herausgegeben von {\editorInnen}. In: \emph{Arthur Schnitzler: Briefwechsel mit Autorinnen und Autoren}.
 Digitale Edition, https://schnitzler-briefe.acdh.oeaw.ac.at/{\dateiname}.html (Stand \today)
\fi

\end{document}


