%% latex-leseansicht-vorspann.tex
%% Vorspann für die Leseansicht.
%% Lädt die gemeinsame Datei latex-vorspann.tex mit nicht gesetztem Schalter.

\newif\ifkorrekturansicht
\korrekturansichtfalse

\input{../tex-inputs/latex-vorspann}


\section[Arthur Schnitzler an Berta Zuckerkandl, 20. 11. 1924]{L03954 Arthur Schnitzler an Berta Zuckerkandl, 20. 11. 1924}
\nopagebreak\mylabel{L03954v}
\rehead{ }\normalsize\beginnumbering\briefempfaengerindex{Zuckerkandl, Berta@\textsc{Zuckerkandl, Berta}!zzzSchnitzler, Arthur@\emph{von Arthur Schnitzler}!1924-11-201@{20. 11. 1924}|(be}
\toendnotes[C]{\smallbreak\pagebreak[2]}
\correspDesc{Versand  durch Arthur Schnitzler am 20. 11. 1924 in Wien
\newline{}Erhalt  durch Berta Zuckerkandl im Zeitraum [20. 11. 1924 – 23. 11. 1924?] in Wien}\toendnotes[C]{\smallbreak}
\Standort{DLA, HS.1985.1.2282.}
\physDesc{Brief, Durchschlag, 1 Blatt, 2 Seiten, 1825 Zeichen
\newline{}Schreibmaschine
\newline{}Handschrift: roter Buntstift, lateinische Kurrent (\noindent{}beschriftet: »\uline{Zuckerkandl}«, sechs Unterstreichungen)}\toendnotes[C]{\smallbreak}
\pstart
           \raggedleft{}{\pb}20. 11. 1924.\pend
           
\pstart{}Liebe und verehrte Frau Hofrätin.\pend\vspace{0.5em}
\pstart
           Ich erhalte einen \label{K_L03954-1v}\edtext{Brief von Mme Genevieve Bianquis\pwindex{Bianquis, Geneviève 19.\,9.\,1886 Rouen – 24.\,3.\,1972 Antony@\textsc{Bianquis, Geneviève} (19.\,9.\,1886 Rouen – 24.\,3.\,1972 Antony), \emph{Übersetzerin, Literaturhistorikerin}|pw}}{\lemma{\textnormal{\emph{Brief … Bianquis}}}\Cendnote{\textnormal{nicht überliefert}}}\label{K_L03954-1}, Paris 6, XIV\oindex{6, Rue Mouton Duvernet@\textbf{6, Rue Mouton Duvernet}, \emph{Wohngebäude}|pw}, von dessen Inhalt ich Ihnen
               gleich Mitteilung machen will. Schon vor dem Krieg stand ich mit Mme.
                  B.\pwindex{Bianquis, Geneviève 19.\,9.\,1886 Rouen – 24.\,3.\,1972 Antony@\textsc{Bianquis, Geneviève} (19.\,9.\,1886 Rouen – 24.\,3.\,1972 Antony), \emph{Übersetzerin, Literaturhistorikerin}|pw}{ }\label{K_L03954-2v}\edtext{in Verbindung}{\lemma{\textnormal{\emph{in Verbindung}}}\Cendnote{\textnormal{Vgl. Arthur Schnitzler
                  an Geneviève Bianquis\pwindex{Bianquis, Geneviève 19.\,9.\,1886 Rouen – 24.\,3.\,1972 Antony@\textsc{Bianquis, Geneviève} (19.\,9.\,1886 Rouen – 24.\,3.\,1972 Antony), \emph{Übersetzerin, Literaturhistorikerin}|pwk},
                     18. 2. 1909, \emph{Deutsches Literaturarchiv Marbach},
                        HS.1985.1.387,1.}}}\label{K_L03954-2} und zwar bezüglich des »Einsamen Wegs\pwindex{Schnitzler, Arthur 15. 5. 1862 Wien – 21. 10. 1931 ebd.@\textsc{Schnitzler, Arthur} (15. 5. 1862 Wien – 21. 10. 1931 ebd.), \emph{Schriftsteller, Mediziner}!einsame Weg. Schauspiel in fünf Akten@\strich\emph{Der einsame Weg. Schauspiel in fünf Akten}|pw}«. Sie schreibt mir nun, dass die Revue de Paris\pwindex{Revue de Paris@\emph{La Revue de Paris}|pw} (M. André Chaumeix\pwindex{Chaumeix, André 6.\,6.\,1874 Clermont-Ferrand – 23.\,2.\,1955 Paris@\textsc{Chaumeix, André} (6.\,6.\,1874 Clermont-Ferrand – 23.\,2.\,1955 Paris), \emph{Journalist, Literaturwissenschaftler}|pw}) geneigt wäre ihre Uebersetzung\textcolor{red}{\textsuperscript{XXXX indx2}} zu
               publizieren und frägt nach meinen Bedingungen. Auch eine Uebersetzung\textcolor{red}{\textsuperscript{XXXX indx2}} von »Komödie der Worte\pwindex{Schnitzler, Arthur 15. 5. 1862 Wien – 21. 10. 1931 ebd.@\textsc{Schnitzler, Arthur} (15. 5. 1862 Wien – 21. 10. 1931 ebd.), \emph{Schriftsteller, Mediziner}!Komödie der Worte. Drei Einakter@\strich\emph{Komödie der Worte. Drei Einakter}|pw}« hat sie begonnen und zwar wie
               es scheint mit Hinblick auf eine eventuelle Veröffentlichung bei Stock\orgindex{Éditions Stock@Éditions Stock|pw}; sie ist auch von dem Abbruch meiner Verhandlungen mit
                  Boutelleau\pwindex{Chardonne, Jacques 2.\,1.\,1884 Barbezieux-Saint-Hilaire – 29.\,5.\,1968 La Frette-sur-Seine@\textsc{Chardonne, Jacques} (2.\,1.\,1884 Barbezieux-Saint-Hilaire – 29.\,5.\,1968 La Frette-sur-Seine), \emph{Schriftsteller, Verleger}|pw} unterrichtet, scheint überhaupt
               gut informiert und akreditiert zu sein. \label{K_L03954-3v}\edtext{Ich schrieb ihr}{\lemma{\textnormal{\emph{Ich schrieb ihr}}}\Cendnote{\textnormal{Arthur Schnitzler an Geneviève Bianquis\pwindex{Bianquis, Geneviève 19.\,9.\,1886 Rouen – 24.\,3.\,1972 Antony@\textsc{Bianquis, Geneviève} (19.\,9.\,1886 Rouen – 24.\,3.\,1972 Antony), \emph{Übersetzerin, Literaturhistorikerin}|pwk}, 20. 11. 1924, \emph{Deutsches Literaturarchiv Marbach},
                     HS.1985.1.387,2.}}}\label{K_L03954-3}, dass Sie, verehrte Frau Hofrätin, sich jetzt in
                  Paris\oindex{Paris@\textbf{Paris}, \emph{Hauptstadt}|pw} aufhielten und ich Sie gebeten hätte
               sich mit ihr, falls Sie Zeit dazu hätten, in Verbindung setzen würden. Sie sind also
               in keiner Weise gebunden, wenn es Ihre Zeit aber erlaubt, so wäre es sehr
               liebenswürdig und wohl auch vorteilhaft für die ganze Angelegenheit, wenn Sie Mme.
                  B.\pwindex{Bianquis, Geneviève 19.\,9.\,1886 Rouen – 24.\,3.\,1972 Antony@\textsc{Bianquis, Geneviève} (19.\,9.\,1886 Rouen – 24.\,3.\,1972 Antony), \emph{Übersetzerin, Literaturhistorikerin}|pw} eine Nachricht geben wollten, ob Sie
               geneigt wären sie zu empfangen.\pend
           
\pstart
           So haben sie auch in Paris\oindex{Paris@\textbf{Paris}, \emph{Hauptstadt}|pw} keine Ruhe von mir,
               verehrte Freundin, aber ich wiederhole, dass Mme B.\pwindex{Bianquis, Geneviève 19.\,9.\,1886 Rouen – 24.\,3.\,1972 Antony@\textsc{Bianquis, Geneviève} (19.\,9.\,1886 Rouen – 24.\,3.\,1972 Antony), \emph{Übersetzerin, Literaturhistorikerin}|pw} keineswegs mit Sicherheit eine Verständigung von Ihnen erwartet. Mich
               würde es natürlich sehr freuen, wenn Sie sich wie für meine anderen Angelegenheiten
               auch für diese, die mir aussichtsvoll erscheint interessieren wollten. Ich hoffe,
               liebe Frau Hofrätin, dass Sie wohlbehalten angelangt sind, sich weiterhin wohl
               befinden und dass Ihre verschiedenen Bemühungen, insbesondere {\pb}aber die im Interesse Ihres Sohns\pwindex{Zuckerkandl, Fritz 30.\,7.\,1895 Wien – 14.\,12.\,1983 Krattigen@\textsc{Zuckerkandl, Fritz} (30.\,7.\,1895 Wien – 14.\,12.\,1983 Krattigen), \emph{Chemiker}|pwv}, von gewünschtem Erfolg begleitet
               sind. Bitte empfehlen Sie mich allerbestens Mme. Clemenceau\pwindex{Clemenceau, Sophie 25.\,5.\,1862 – 24.\,9.\,1937@\textsc{Clemenceau, Sophie} (25.\,5.\,1862 – 24.\,9.\,1937)|pw} und seien sie selbst sehr herzlich begrüsst von\pend
           {\vspace{1\baselineskip}}\pstart Ihrem dankbar ergebenen\pend{}{\vspace{1\baselineskip}}
\pstart
           \noindent{}Frau Hofrätin Bertha Zuckerkandl,{\\}Paris 12, Rue d’Eylau\oindex{12, Avenue d’Eylau@\textbf{12, Avenue d’Eylau}, \emph{Wohngebäude}|pw}\pend
           \selectlanguage{ngerman}\endnumbering\briefempfaengerindex{Zuckerkandl, Berta@\textsc{Zuckerkandl, Berta}!zzzSchnitzler, Arthur@\emph{von Arthur Schnitzler}!1924-11-201@{20. 11. 1924}|)be}\mylabel{L03954h}
\begin{anhang}
\end{anhang}\newcommand{\dateiname}{L03954}\newcommand{\titel}{Arthur Schnitzler an Berta Zuckerkandl, 20. 11. 1924}\newcommand{\editorInnen}{Herausgegeben von Jahnke, SelmaMüller, Martin Anton}%% latex-leseansicht-abspann.tex
%% Abspann für die Leseansicht.
%% Der Schalter \ifkorrekturansicht ist bereits durch den Vorspann gesetzt.

%% latex-abspann.tex
%% Gemeinsamer Abspann für Korrekturansicht und Leseansicht.
%% Setzt den Schalter \ifkorrekturansicht voraus (gesetzt in den
%% einbindenden Dateien latex-korrekturansicht-abspann.tex bzw.
%% latex-leseansicht-abspann.tex).
%% ---------------------------------------------------------------

\normalsize

% Das esempio-Environment wird nur in der Leseansicht benötigt
\ifkorrekturansicht\else
\newenvironment{esempio}[3]%
{
    \vspace{1.5ex}
    \rlap{\underline{#1}}
    \par
    \setlength{\parindent}{0cm}
    \nopagebreak
    \leftskip=#2cm
    \rightskip=#3cm
}
{
    \par
}
\fi

\doendnotes{C}
\bigskip
\vfill

\clearpage

\footnotesize

\ifkorrekturansicht
  \lohead{\textsc{register}}
\fi

% theindex-Environment neu definieren ohne reledmac
\makeatletter
\renewenvironment{theindex}{%
  \ifkorrekturansicht
    \section*{\indexname}%
  \else
    \subsubsection*{Index der erwähnten Entitäten}%
  \fi
  \setlength{\parindent}{0pt}%
  \setlength{\parskip}{0pt plus 0.3pt}%
  \let\item\@idxitem
}{%
  \ifkorrekturansicht\clearpage\fi
}
\makeatother

\IfFileExists{\jobname-pw.ind}{\input{\jobname-pw.ind}}{}

% Quellenangabe nur in der Leseansicht
\ifkorrekturansicht\else
% Fallback-Definitionen, falls die .tex-Datei \titel etc. nicht gesetzt hat
\providecommand{\titel}{}
\providecommand{\editorInnen}{}
\providecommand{\dateiname}{\jobname}

\vspace{3cm}

\vfill

\footnotesize
\textsc{Quelle}: \titel. Herausgegeben von {\editorInnen}. In: \emph{Arthur Schnitzler: Briefwechsel mit Autorinnen und Autoren}.
 Digitale Edition, https://schnitzler-briefe.acdh.oeaw.ac.at/{\dateiname}.html (Stand \today)
\fi

\end{document}


