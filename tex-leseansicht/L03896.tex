%% latex-leseansicht-vorspann.tex
%% Vorspann für die Leseansicht.
%% Lädt die gemeinsame Datei latex-vorspann.tex mit nicht gesetztem Schalter.

\newif\ifkorrekturansicht
\korrekturansichtfalse

\input{../tex-inputs/latex-vorspann}


\section[Theodor Herzl an Arthur Schnitzler, 23. 6. 1895]{L03896 Theodor Herzl an Arthur Schnitzler, 23. 6. 1895}
\nopagebreak\mylabel{L03896v}
\rehead{ }\normalsize\beginnumbering\briefempfaengerindex{Schnitzler, Arthur@\textsc{Schnitzler, Arthur}!zzzHerzl, Theodor@\emph{von Theodor Herzl}!1895-06-232@{23. 6. 1895}|(be}
\toendnotes[C]{\smallbreak\pagebreak[2]}
\correspDesc{Versand  durch Theodor Herzl am 23. 6. 1895 in Paris
\newline{}Erhalt  durch Arthur Schnitzler im Zeitraum [24. 6. 1895
                  – 28. 6. 1895?] in Wien}\toendnotes[C]{\smallbreak}
\Standort{Jerusalem, Central Zionist Archives, H1:2540-7.}
\physDesc{Brief, Fotokopie, 1 Blatt, 1 Seite, 2329 Zeichen
\newline{}Handschrift: schwarze Tinte, lateinische Kurrent
\newline{}Ordnung: mit Bleistift von unbekannter Hand nummeriert: »87« 
\newline{}Zusatz: Das Faksimile der ersten Seite stammt aus dem Katalog 576 von
                                  \emph{J. A. Stargardt}, wo es als Lot. Nr. 1007 um 3000 DM
                                  angeboten und um 15.000 DM verkauft wurde. Die Auktion fand von 24. – 25. 5. 1966 in Marburg\oindex{XXXX Ortsangabe fehlt|pw}
                                  statt. Der gegenwärtige Verbleib ist nicht geklärt. }\Standort{Wien, Österreichische Gesellschaft für Literatur, Abschrift Herzl.}
\physDesc{Brief, maschinenschriftliche Abschrift, 1 Blatt, 1 Seite, 2329 Zeichen
\newline{}maschinell
\newline{}Zusatz: In der Nachlassmappe B 39 hat Heinrich Schnitzler\pwindex{Schnitzler, Heinrich 9.\,8.\,1902 Hinterbrühl – 12.\,7.\,1982 Wien@\textsc{Schnitzler, Heinrich} (9.\,8.\,1902 Hinterbrühl – 12.\,7.\,1982 Wien), \emph{Regisseur, Schauspieler}|pw} vermerkt: »\noindent{}2 Briefe geschenkt ans Wolf-Museum Eisenstadt\orgindex{Landesmuseum Burgenland@Landesmuseum Burgenland|pw}{ }22. VIII. 1937.{ / }1 Brief entnommen{ / }1 Brief geschenkt an Paul
                                          Marx\pwindex{Marx, Paul 21.\,7.\,1879 Wien – 30.\,10.\,1956 ebd.@\textsc{Marx, Paul} (21.\,7.\,1879 Wien – 30.\,10.\,1956 ebd.), \emph{Regisseur, Schauspieler}|pw}{ }15. VIII. 1936.{ / }1 Brief gegeben an Mutter\pwindex{Schnitzler, Olga 17.\,1.\,1882 Wien – 13.\,1.\,1970 Lugano@\textsc{Schnitzler, Olga} (17.\,1.\,1882 Wien – 13.\,1.\,1970 Lugano), \emph{Schauspielerin, Sängerin}|pwv}, 15. VIII. 36.« Das entspricht der Anzahl von fünf Korrespondenzstücken
                                 von Herzl, die nicht im Original überliefert sind. Alle finden sich
                                 in einer Abschrift, die nach Arthur Schnitzlers Tod im Zeitraum
                                    1932 bis 1936 entstanden sein dürfte.
                                 Da Olga Schnitzler\pwindex{Schnitzler, Olga 17.\,1.\,1882 Wien – 13.\,1.\,1970 Lugano@\textsc{Schnitzler, Olga} (17.\,1.\,1882 Wien – 13.\,1.\,1970 Lugano), \emph{Schauspielerin, Sängerin}|pw} in ihrer Darstellung
                              den Brief ausführlich zitiert, könnte es sein, dass sie das Original
                              dieses Briefs besessen hat. }
\buchAbdrucke{\weitereDrucke{1) \pwindex{Kellner, Leon 17.\,4.\,1859 Tarnów – 5.\,12.\,1928 Wien@\textsc{Kellner, Leon} (17.\,4.\,1859 Tarnów – 5.\,12.\,1928 Wien), \emph{Zionist, Literaturhistoriker, Anglist}!Theodor Herzls Lehrjahre (1860–1895). Nach den handschriftlichen Quellen@\strich\emph{Theodor Herzls Lehrjahre (1860–1895). Nach den handschriftlichen Quellen}|pwk}Leon Kellner: \emph{Theodor Herzls Lehrjahre (1860–1895). Nach den
                        handschriftlichen Quellen}. Wien, Berlin: \emph{R. Löwit-Verlag} 1920, S. 157–159.} \weitereDrucke{2) \emph{Die Geburt des Judenstaates.} In: \emph{Jüdische Nachrichten für die österreichischen
                        Alpenländer}, Nr. 20, 3. 7. 1920, S. 4–5.} \weitereDrucke{3) \pwindex{Schnitzler, Olga 17.\,1.\,1882 Wien – 13.\,1.\,1970 Lugano@\textsc{Schnitzler, Olga} (17.\,1.\,1882 Wien – 13.\,1.\,1970 Lugano), \emph{Schauspielerin, Sängerin}!Spiegelbild der Freundschaft@\strich\emph{Spiegelbild der Freundschaft}|pwk}Olga Schnitzler: \emph{Spiegelbild der Freundschaft}. Salzburg: \emph{Residenz-Verlag} 1962, S. 94–95.} \weitereDrucke{4) \emph{Herzl-Briefe}. Herausgegeben und eingeleitet Manfred Georg. Berlin: \emph{Brandusche Verlagsbuchhandlung} [1935], S. 53–54.} \weitereDrucke{5) Theodor Herzl: \emph{Briefe Anfang Mai 1895 – Anfang Dezember 1898}. Bearbeitet von Barbara Schäfer in Zusammenarbeit mit Sofia Gelmann, Chaya Harel, Ines Rubin und Daisy Ticho. Berlin, Frankfurt am Main, Wien: \emph{Propyläen} 1990, S. 56–57 (Briefe und Tagebücher. Herausgegeben von Alex Bein, Hermann Greive, Moshe Schaerf, Julius H. Schoeps und Johannes Wachten, 4).} }\toendnotes[C]{\smallbreak}
\pstart
           \raggedleft{}{\pb}23. 6. 95\pend
           
\pstart{}mein lieber Freund,\pend\vspace{0.5em}
\pstart
           Dank für Ihren Brief. Die Sache liegt in Prag\oindex{Prag@\textbf{Prag}, \emph{Land}|pw},
               eine Entscheidung ist noch nicht da. Das Ganze ist jetzt in den Hintergrund meines
               Bewusstseins getreten.\pend
           
\pstart
           Aber Sie hatten damals Recht, als Sie mit Ihrem klugen Blick sahen, dass ich mit
               dieser einen Eruption mir die Sache nicht vom Herzen und nicht von der Seele geladen
               habe.\pend
           
\pstart
           In den Wochen, seit ich Ihnen nicht geschrieben, ist etwas Anderes, Neues, viel
               Grösseres in mir aufgeschossen, was mir jetzt wie ein Basaltberg vorkommt, vielleicht
               weil ich noch so erschüttert bin und das Entstandene noch so fürchterlich glüht. {\pb}Wochen der ungeheuerlichsten Poduktionsaufregung, in der ich manchmal fürchtete,
               verrückt zu werden.\pend
           
\pstart
           Es sind vorläufig nur die Planskizzen – sie sind schon ein ganzes Buch\pwindex{Herzl, Theodor 2.\,5.\,1860 Budapest – 3.\,7.\,1904 Edlach@\textsc{Herzl, Theodor} (2.\,5.\,1860 Budapest – 3.\,7.\,1904 Edlach), \emph{Schriftsteller, Journalist}!Judenstaat. Versuch einer modernen Lösung der Judenfrage@\strich\emph{Der Judenstaat. Versuch einer modernen Lösung der Judenfrage}|pwv}.\pend
           
\pstart
           Wir werden, wenn wir im Sommer im Salzkammergut\oindex{Salzkammergut@\textbf{Salzkammergut}, \emph{Region}|pw}
               Zusammentreffen, darüber reden.\pend
           
\pstart
           Dieses Werk\pwindex{Herzl, Theodor 2.\,5.\,1860 Budapest – 3.\,7.\,1904 Edlach@\textsc{Herzl, Theodor} (2.\,5.\,1860 Budapest – 3.\,7.\,1904 Edlach), \emph{Schriftsteller, Journalist}!Judenstaat. Versuch einer modernen Lösung der Judenfrage@\strich\emph{Der Judenstaat. Versuch einer modernen Lösung der Judenfrage}|pwv} ist jedenfalls für
               mich und mein ferneres Leben von der grössten Bedeutung – vielleicht auch für andere
               Menschen. Denn was mich annehmen lässt, dass ich etwas Wertvolles entworfen habe, ist
               die Tatsache, dass ich dabei keine Sekunde lang literatenhaft an mich gedacht habe,
               sondern immer an andere Menschen, welche schwer leiden.\pend
           
\pstart
           Noch ein paar Tage Arbeit, und die Sache ist so fertig, dass sie nicht mehr verloren
               gehen kann, auch wenn ich durch Umstände des Lebens an der munitiösen Ausführung
               verhindert werden sollte.\pend
           
\pstart
           Dann verlasse ich Paris\oindex{Paris@\textbf{Paris}, \emph{Hauptstadt}|pw} auf einige Tage, um mich
               zu erholen. Mein Urlaub ist das noch nicht; den trete ich erst Mitte oder Ende Juli
               an.\pend
           
\pstart
           Sie kennen das liebe Gedicht von Heyse\pwindex{Heyse, Paul 15.\,3.\,1830 Berlin – 2.\,4.\,1914 München@\textsc{Heyse, Paul} (15.\,3.\,1830 Berlin – 2.\,4.\,1914 München), \emph{Schriftsteller}|pw}
                  »\label{K_L03896-1v}\edtext{an den Künstler\pwindex{Heyse, Paul 15.\,3.\,1830 Berlin – 2.\,4.\,1914 München@\textsc{Heyse, Paul} (15.\,3.\,1830 Berlin – 2.\,4.\,1914 München), \emph{Schriftsteller}!Weihe der Kunst@\strich\emph{Weihe der Kunst}|pw}}{\lemma{\textnormal{\emph{an den Künstler}}}\Cendnote{\textnormal{»Und bangſt, du möchteſt über Nacht{ / }Hinfahren, eh dies Werk vollbracht:«. Paul Heyse\pwindex{Heyse, Paul 15.\,3.\,1830 Berlin – 2.\,4.\,1914 München@\textsc{Heyse, Paul} (15.\,3.\,1830 Berlin – 2.\,4.\,1914 München), \emph{Schriftsteller}|pwk}: \emph{Weihe der Kunst}\pwindex{Heyse, Paul 15.\,3.\,1830 Berlin – 2.\,4.\,1914 München@\textsc{Heyse, Paul} (15.\,3.\,1830 Berlin – 2.\,4.\,1914 München), \emph{Schriftsteller}!Weihe der Kunst@\strich\emph{Weihe der Kunst}|pwk}. In: \emph{Der Kunstwart}\pwindex{Kunstwart. Rundschau über alle Gebiete des Schönen@\emph{Der Kunstwart. Rundschau über alle Gebiete des Schönen}|pwk}, Jg. 1, H. 1, 5. 10. 1887,
                  S. 10.}}}\label{K_L03896-1}«, das ich oft citiere. Da heisst es\pend
           \stanza{}{\pb}{\dots}{ }Bangend, er könnte über
                     Nacht\pwindex{Heyse, Paul 15.\,3.\,1830 Berlin – 2.\,4.\,1914 München@\textsc{Heyse, Paul} (15.\,3.\,1830 Berlin – 2.\,4.\,1914 München), \emph{Schriftsteller}!Weihe der Kunst@\strich\emph{Weihe der Kunst}|pwv}\newverse{}Hinfahren ehe dies Werk
                     vollbracht.\pwindex{Heyse, Paul 15.\,3.\,1830 Berlin – 2.\,4.\,1914 München@\textsc{Heyse, Paul} (15.\,3.\,1830 Berlin – 2.\,4.\,1914 München), \emph{Schriftsteller}!Weihe der Kunst@\strich\emph{Weihe der Kunst}|pwv}\stanzaend{}
\pstart
           Das ist meine Stimmung.\pend
           
\pstart
           Ich habe den Stoss bisheriger Notizen im \begin{otherlanguage}{french}Comptoir d’Escompte\end{otherlanguage}\orgindex{Comptoir d’Escompte@Comptoir d’Escompte|pw}{ }deponiert, in der Kasse Nr. 6, Fach Nr. 2. Um zu öffnen muss man jeden der drei
               Knöpfe siebenmal nach rechts drücken. Jemand muss das wissen, falls ich »hinfahre über Nacht.\pwindex{Heyse, Paul 15.\,3.\,1830 Berlin – 2.\,4.\,1914 München@\textsc{Heyse, Paul} (15.\,3.\,1830 Berlin – 2.\,4.\,1914 München), \emph{Schriftsteller}!Weihe der Kunst@\strich\emph{Weihe der Kunst}|pwv}«\pend
           
\pstart
           Das sind jetzt Sie.\pend
           
\pstart
           Komme ich Ihnen aufgeregt vor? Ich bin es nicht. Ich war nie in einer so glücklichen
               hohen Stimmung. Ich denke nicht ans Sterben, sondern an ein Leben voll männlicher
               Taten, das alles Niedere, Wüste, Verworrene, das je in mir gewesen sein mag,
               auslöscht, aufhebt und alle mit mir versöhnt, so wie ich mich durch diese Arbeit mit
               allen versöhnt habe.\pend
           
\pstart
           Ich grüsse Sie herzlich{\\[\baselineskip]} Ihr Freund{\\[\baselineskip]}\spacefill\mbox{Herzl}\pend
           \leftskip=0em{}\selectlanguage{ngerman}\endnumbering\briefempfaengerindex{Schnitzler, Arthur@\textsc{Schnitzler, Arthur}!zzzHerzl, Theodor@\emph{von Theodor Herzl}!1895-06-232@{23. 6. 1895}|)be}\mylabel{L03896h}
\begin{anhang}
\end{anhang}\newcommand{\dateiname}{L03896}\newcommand{\titel}{Theodor Herzl an Arthur Schnitzler, 23. 6. 1895}\newcommand{\editorInnen}{Selma Jahnke und Martin Anton Müller}%% latex-leseansicht-abspann.tex
%% Abspann für die Leseansicht.
%% Der Schalter \ifkorrekturansicht ist bereits durch den Vorspann gesetzt.

%% latex-abspann.tex
%% Gemeinsamer Abspann für Korrekturansicht und Leseansicht.
%% Setzt den Schalter \ifkorrekturansicht voraus (gesetzt in den
%% einbindenden Dateien latex-korrekturansicht-abspann.tex bzw.
%% latex-leseansicht-abspann.tex).
%% ---------------------------------------------------------------

\normalsize

% Das esempio-Environment wird nur in der Leseansicht benötigt
\ifkorrekturansicht\else
\newenvironment{esempio}[3]%
{
    \vspace{1.5ex}
    \rlap{\underline{#1}}
    \par
    \setlength{\parindent}{0cm}
    \nopagebreak
    \leftskip=#2cm
    \rightskip=#3cm
}
{
    \par
}
\fi

\doendnotes{C}
\bigskip
\vfill

\clearpage

\footnotesize

\ifkorrekturansicht
  \lohead{\textsc{register}}
\fi

% theindex-Environment neu definieren ohne reledmac
\makeatletter
\renewenvironment{theindex}{%
  \ifkorrekturansicht
    \section*{\indexname}%
  \else
    \subsubsection*{Index der erwähnten Entitäten}%
  \fi
  \setlength{\parindent}{0pt}%
  \setlength{\parskip}{0pt plus 0.3pt}%
  \let\item\@idxitem
}{%
  \ifkorrekturansicht\clearpage\fi
}
\makeatother

\IfFileExists{\jobname-pw.ind}{\input{\jobname-pw.ind}}{}

% Quellenangabe nur in der Leseansicht
\ifkorrekturansicht\else
% Fallback-Definitionen, falls die .tex-Datei \titel etc. nicht gesetzt hat
\providecommand{\titel}{}
\providecommand{\editorInnen}{}
\providecommand{\dateiname}{\jobname}

\vspace{3cm}

\vfill

\footnotesize
\textsc{Quelle}: \titel. Herausgegeben von {\editorInnen}. In: \emph{Arthur Schnitzler: Briefwechsel mit Autorinnen und Autoren}.
 Digitale Edition, https://schnitzler-briefe.acdh.oeaw.ac.at/{\dateiname}.html (Stand \today)
\fi

\end{document}


