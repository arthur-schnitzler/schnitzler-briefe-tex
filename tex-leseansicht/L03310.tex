%% latex-leseansicht-vorspann.tex
%% Vorspann für die Leseansicht.
%% Lädt die gemeinsame Datei latex-vorspann.tex mit nicht gesetztem Schalter.

\newif\ifkorrekturansicht
\korrekturansichtfalse

\input{../tex-inputs/latex-vorspann}

\begin{center}
            \textcolor{red}{ENTWURF, NICHT FERTIG KORRIGIERT}
                      \end{center}
            
         
         \renewcommand{\erwaehntePersonen}{Personen: Oskar Mayer}
         \renewcommand{\erwaehnteOrte}{Orte: Ala, Bad Ischl, Bludenz, Bozen, Innsbruck, Karlsbad, Meran, Schruns, Schweiz, Traunkai, Triest, Venedig, Verona, Wien}
         \renewcommand{\erwaehnteWerke}{}
               \section[Felix Salten an Arthur Schnitzler, 14. 8. 1900]{ Felix Salten an Arthur Schnitzler, 14. 8. 1900}\nopagebreak\mylabel{v}\rehead{ }\begin{ledgroupsized}[t]{13cm}\normalsize\beginnumbering \toendnotes[C]{\smallbreak\pagebreak[2]} \Standort{CUL, Schnitzler, B 89, A 2.}
\physDesc{Brief, 1 Blatt, 3 Seiten
\newline{}Handschrift: schwarze Tinte, lateinische Kurrent\newline{}Ordnung: mit Bleistift von unbekannter Hand nummeriert:
                                    »134« }\toendnotes[C]{\smallbreak}\pstart
           \raggedleft{}{\pb}Ischl, Traunquai 11\oindex{Traunkai@\textbf{Traunkai}|pw}. \pend
           \pstart
           \raggedleft{}14. Aug. 00. \pend
           \pstart
           Lieber Freund, leider mußte ich von Wien\oindex{Wien@\textbf{Wien}|pw} aus zuerst nach Karlsbad\oindex{Karlsbad@\textbf{Karlsbad}|pw}, wie Sie
               wissen, u. bin erst heute hierhergekommen. Ich muss nun wenigstens 7–8 Tage still
               sitzen und arbeiten. Außerdem bin ich auch nicht besonders wol. Es ist für mich
               garnicht dran zu denken, dass ich nach Schruns\oindex{Schruns@\textbf{Schruns}|pw} komme.
               Aber einen Vorschlag: Möchten Sie vom Endpunkt Ihrer Tour aus mit mir eine mehrtägige
               Radparthie machen? Wenn Sie, wie Sie mir schreiben, nach Meran\oindex{Meran@\textbf{Meran}|pw} kommen, dann schlage ich vor, dass wir uns in Bozen\oindex{Bozen@\textbf{Bozen}|pw} treffen, und überlasse dann Ihnen die Bestimmung der
               Route. (Gerne {\pb}würde ich über
               Verona\oindex{Verona@\textbf{Verona}|pw} nach Venedig\oindex{Venedig@\textbf{Venedig}|pw}) Jedenfalls bitte ich Sie, mir gleich Nachricht darüber zu geben und
               mir besonders Ort der Zusammenkunft und Ziel der Radtour anzugeben, möglichst genau,
               weil ich mir danach meine Eisenbahnkarte bestellen muß. Ich habe die Karte bis Bludenz\oindex{Bludenz@\textbf{Bludenz}|pw} bei mir, aber ich muß jedenfalls noch
               andere Karten aus Wien\oindex{Wien@\textbf{Wien}|pw} verschreiben, ich denke:
                  Innsbruck\oindex{Innsbruck@\textbf{Innsbruck}|pw} – Ala\oindex{Ala@\textbf{Ala}|pw}, Triest\oindex{Triest@\textbf{Triest}|pw} – Wien\oindex{Wien@\textbf{Wien}|pw}, oder auch anders. Das hängt dann eben ganz von der Tour
               ab. {\pb}Ich möchte noch sagen,
               dass ich jeden Vorschlag acceptire, (es sei denn Schweiz\oindex{Schweiz@\textbf{Schweiz}|pw}, was mir vielleicht zu theuer wäre) und dass ich voraussichtlich keine
               Abhaltung mehr haben werde. \pend
           \pstart
           Hat mein \label{K_L03310-1v}\edtext{Brief mit Inschluß}{\lemma{\textnormal{\emph{Brief mit Inschluß}}}\Cendnote{\textnormal{siehe Felix Salten an Arthur Schnitzler, 8. 8. 1900}}}\label{K_L03310-1h} an Mayer\pwindex{Mayer, Oskar 1876 – 15.05.1915@\textsc{Mayer, Oskar} (1876 – 15.05.1915), \emph{Schriftsteller, Beamter}|pw} Sie erreicht? \pend
           \pstart
           Bitte, schreiben Sie bald. \pend
           \pstart
           Herzlichst {\\[\baselineskip]}Ihr {\\[\baselineskip]}\spacefill\mbox{Salten}\pend
           \leftskip=0em{}
         
         \endnumbering\mylabel{h}\end{ledgroupsized}\begin{anhang}\end{anhang}\newcommand{\dateiname}{L03310}\newcommand{\titel}{Felix Salten an Arthur Schnitzler, 14. 8. 1900}\newcommand{\editorInnen}{Martin Anton Müller und Laura Untner}%% latex-leseansicht-abspann.tex
%% Abspann für die Leseansicht.
%% Der Schalter \ifkorrekturansicht ist bereits durch den Vorspann gesetzt.

%% latex-abspann.tex
%% Gemeinsamer Abspann für Korrekturansicht und Leseansicht.
%% Setzt den Schalter \ifkorrekturansicht voraus (gesetzt in den
%% einbindenden Dateien latex-korrekturansicht-abspann.tex bzw.
%% latex-leseansicht-abspann.tex).
%% ---------------------------------------------------------------

\normalsize

% Das esempio-Environment wird nur in der Leseansicht benötigt
\ifkorrekturansicht\else
\newenvironment{esempio}[3]%
{
    \vspace{1.5ex}
    \rlap{\underline{#1}}
    \par
    \setlength{\parindent}{0cm}
    \nopagebreak
    \leftskip=#2cm
    \rightskip=#3cm
}
{
    \par
}
\fi

\doendnotes{C}
\bigskip
\vfill

\clearpage

\footnotesize

\ifkorrekturansicht
  \lohead{\textsc{register}}
\fi

% theindex-Environment neu definieren ohne reledmac
\makeatletter
\renewenvironment{theindex}{%
  \ifkorrekturansicht
    \section*{\indexname}%
  \else
    \subsubsection*{Index der erwähnten Entitäten}%
  \fi
  \setlength{\parindent}{0pt}%
  \setlength{\parskip}{0pt plus 0.3pt}%
  \let\item\@idxitem
}{%
  \ifkorrekturansicht\clearpage\fi
}
\makeatother

\IfFileExists{\jobname-pw.ind}{\input{\jobname-pw.ind}}{}

% Quellenangabe nur in der Leseansicht
\ifkorrekturansicht\else
% Fallback-Definitionen, falls die .tex-Datei \titel etc. nicht gesetzt hat
\providecommand{\titel}{}
\providecommand{\editorInnen}{}
\providecommand{\dateiname}{\jobname}

\vspace{3cm}

\vfill

\footnotesize
\textsc{Quelle}: \titel. Herausgegeben von {\editorInnen}. In: \emph{Arthur Schnitzler: Briefwechsel mit Autorinnen und Autoren}.
 Digitale Edition, https://schnitzler-briefe.acdh.oeaw.ac.at/{\dateiname}.html (Stand \today)
\fi

\end{document}


      