%% latex-korrekturansicht-vorspann.tex
%% Vorspann für die Korrekturansicht.
%% Lädt die gemeinsame Datei latex-vorspann.tex mit gesetztem Schalter.

\newif\ifkorrekturansicht
\korrekturansichttrue

\input{../tex-inputs/latex-vorspann}


\section[ Felix Salten an Arthur Schnitzler, 14. 8. 1900]{L03310 Felix Salten an Arthur Schnitzler, 14. 8. 1900}
\nopagebreak\mylabel{L03310v}
\rehead{ }\normalsize\beginnumbering\briefempfaengerindex{Schnitzler, Arthur@\textsc{Schnitzler, Arthur}!zzzSalten, Felix@\emph{von Felix Salten}!1900-08-141@{14. 8. 1900}|(be}
\toendnotes[C]{\smallbreak\pagebreak[2]}\Standort{CUL, Schnitzler, B 89, A 2.}
\physDesc{Brief, 1 Blatt, 3 Seiten, 1286 Zeichen
\newline{}Handschrift: schwarze Tinte, lateinische Kurrent
\newline{}Ordnung: mit Bleistift von unbekannter Hand nummeriert: »134« }\toendnotes[C]{\smallbreak}
\pstart
           \raggedleft{}{\pb}Ischl, Traunquai 11\oindex{Traunkai@\textbf{Traunkai}, \emph{Straße (K.STR)}|pw}. {\\}14. Aug. 00.\pend
           \vspace{0.5em}
\pstart
           Lieber Freund, leider mußte ich von Wien\oindex{Wien@\textbf{Wien}, \emph{A.ADM2}|pw} aus zuerst nach Karlsbad\oindex{Karlsbad@\textbf{Karlsbad}, \emph{P.PPLA}|pw}, wie Sie
               wissen, u. bin erst heute hierhergekommen. Ich muss
               nun wenigstens 7–8 Tage still sitzen und arbeiten. Außerdem bin ich auch nicht
               besonders wol. Es ist für mich garnicht dran zu denken, dass ich nach \label{K_L03310-1v}\edtext{Schruns\oindex{Schruns@\textbf{Schruns}, \emph{A.ADM3}|pw}}{\lemma{\textnormal{\emph{Schruns}}}\Cendnote{\textnormal{Siehe Felix Salten an Arthur Schnitzler, 5. 8. 1900.
               }}}\label{K_L03310-1} komme. Aber einen Vorschlag: Möchten Sie vom Endpunct Ihrer Tour aus mit mir
               eine mehrtägige Radparthie machen? Wenn Sie, wie Sie mir
                  schreiben{[},{]} nach Meran\oindex{Meran@\textbf{Meran}, \emph{P.PPLA3}|pw}
               kommen, dann schlage ich vor, dass wir uns in Bozen\oindex{Bozen@\textbf{Bozen}, \emph{P.PPLA2}|pw} treffen, und überlasse dann Ihnen die Bestimmung der Route. (Gerne {\pb}würde ich über Verona\oindex{Verona@\textbf{Verona}, \emph{P.PPLA2}|pw} nach Venedig\oindex{Venedig@\textbf{Venedig}, \emph{P.PPLA}|pw})
               Jedenfalls bitte ich Sie, mir gleich Nachricht darüber zu geben und mir besonders Ort
               der Zusammenkunft und Ziel der Radtour anzugeben, möglichst genau, weil ich mir
               danach meine Eisenbahnkarte bestellen muß. Ich habe die Karte bis Bludenz\oindex{Bludenz@\textbf{Bludenz}, \emph{P.PPLA2}|pw} bei mir, aber ich muß jedesfalls noch andere Karten
               aus Wien\oindex{Wien@\textbf{Wien}, \emph{A.ADM2}|pw} verschreiben, ich denke: \strikeout{(}Innsbruck\oindex{Innsbruck@\textbf{Innsbruck}, \emph{A.ADM2}|pw} – Ala\oindex{Ala@\textbf{Ala}, \emph{P.PPLA3}|pw}, Triest\oindex{Triest@\textbf{Triest}, \emph{A.ADM3}|pw} – Wien\oindex{Wien@\textbf{Wien}, \emph{A.ADM2}|pw}, \substVorne{}\textsuperscript{)}\substDazwischen{}o\substHinten{}der auch anders. Das hängt dann eben ganz von der Tour ab. {\pb}Ich möchte noch sagen, dass ich
               jeden Vorschlag acceptire, (es sei denn Schweiz\oindex{Schweiz@\textbf{Schweiz}, \emph{A.PCLI}|pw}, was mir vielleicht zu theuer wäre) und dass ich voraussichtlich keine
               Abhaltung mehr haben werde.\pend
           
\pstart
           Hat mein Brief mit
               \label{K_L03310-2v}\edtext{\textcolor{gray}{I}nschluß}{\lemma{\textnormal{\emph{Inschluß}}}\Cendnote{\textnormal{Beilage,
                     siehe Felix Salten an Arthur Schnitzler, 8. 8. 1900.
               }}}\label{K_L03310-2} an Mayer\pwindex{Mayer, Oskar 1876 – 15.05.1915@\textsc{Mayer, Oskar} (1876 – 15.05.1915), \emph{Schriftsteller/Schriftstellerin, Beamter/Beamte}|pw} Sie erreicht?\pend
           
\pstart
           Bitte, schreiben Sie bald.\pend
           
\pstart
           Herzlichst {\\[\baselineskip]}Ihr {\\[\baselineskip]}\spacefill\mbox{Salten}\pend
           \leftskip=0em{}\selectlanguage{ngerman}\endnumbering\briefempfaengerindex{Schnitzler, Arthur@\textsc{Schnitzler, Arthur}!zzzSalten, Felix@\emph{von Felix Salten}!1900-08-141@{14. 8. 1900}|)be}\mylabel{L03310h}  \normalsize

\doendnotes{C}
\bigskip
\vfill

\clearpage

\footnotesize

\lohead{\textsc{register}}

% Definiere theindex-Environment komplett neu ohne reledmac
\makeatletter
\renewenvironment{theindex}{%
  \section*{\indexname}%
  \setlength{\parindent}{0pt}%
  \setlength{\parskip}{0pt plus 0.3pt}%
  \let\item\@idxitem
}{%
  \clearpage
}
\makeatother

\IfFileExists{\jobname-pw.ind}{\input{\jobname-pw.ind}}{}

\end{document}

      