%% latex-leseansicht-vorspann.tex
%% Vorspann für die Leseansicht.
%% Lädt die gemeinsame Datei latex-vorspann.tex mit nicht gesetztem Schalter.

\newif\ifkorrekturansicht
\korrekturansichtfalse

\input{../tex-inputs/latex-vorspann}


\section[ Felix Salten an Arthur Schnitzler, 14. 8. 1900]{L03310 Felix Salten an Arthur Schnitzler,  14. 8. 1900}
\nopagebreak\mylabel{L03310v}
\rehead{ }\normalsize\beginnumbering\briefempfaengerindex{Schnitzler, Arthur@\textsc{Schnitzler, Arthur}!zzzSalten, Felix@\emph{von Felix Salten}!1900-08-141@{14. 8. 1900}|(be}
\toendnotes[C]{\smallbreak\pagebreak[2]}
\correspDesc{Versand  durch Felix Salten am 14. 8. 1900 in Bad Ischl
\newline{}Erhalt  durch Arthur Schnitzler am [17. 8. 1900?] in Schruns}\toendnotes[C]{\smallbreak}
\Standort{CUL, Schnitzler, B 89, A 2.}
\physDesc{Brief, 1 Blatt, 3 Seiten, 1286 Zeichen
\newline{}Handschrift: schwarze Tinte, lateinische Kurrent
\newline{}Ordnung: mit Bleistift von unbekannter Hand nummeriert: »134« }\toendnotes[C]{\smallbreak}
\pstart
           \raggedleft{}{\pb}Ischl, Traunquai 11\oindex{Traunkai@\textbf{Traunkai}, \emph{Straße}|pw}. {\\}14. Aug. 00.\pend
           \vspace{0.5em}
\pstart
           Lieber Freund, leider mußte ich von Wien\oindex{Wien@\textbf{Wien}, \emph{Verwaltungsgebiet}|pw} aus zuerst nach Karlsbad\oindex{Karlsbad@\textbf{Karlsbad}|pw}, wie Sie
               wissen, u. bin erst heute hierhergekommen. Ich muss
               nun wenigstens 7–8 Tage still sitzen und arbeiten. Außerdem bin ich auch nicht
               besonders wol. Es ist für mich garnicht dran zu denken, dass ich nach \label{K_L03310-1v}\edtext{Schruns\oindex{Schruns@\textbf{Schruns}, \emph{Verwaltungsgebiet}|pw}}{\lemma{\textnormal{\emph{Schruns}}}\Cendnote{\textnormal{Siehe XXXX Auszeichnungsfehler: Dokument L03307 nicht gefunden.
               }}}\label{K_L03310-1} komme. Aber einen Vorschlag: Möchten Sie vom Endpunct Ihrer Tour aus mit mir
               eine mehrtägige Radparthie machen? Wenn Sie, wie Sie mir
                  schreiben{[},{]} nach Meran\oindex{Meran@\textbf{Meran}, \emph{Hauptstadt}|pw}
               kommen, dann schlage ich vor, dass wir uns in Bozen\oindex{Bozen@\textbf{Bozen}, \emph{Hauptstadt}|pw} treffen, und überlasse dann Ihnen die Bestimmung der Route. (Gerne {\pb}würde ich über Verona\oindex{Verona@\textbf{Verona}, \emph{Hauptstadt}|pw} nach Venedig\oindex{Venedig@\textbf{Venedig}|pw})
               Jedenfalls bitte ich Sie, mir gleich Nachricht darüber zu geben und mir besonders Ort
               der Zusammenkunft und Ziel der Radtour anzugeben, möglichst genau, weil ich mir
               danach meine Eisenbahnkarte bestellen muß. Ich habe die Karte bis Bludenz\oindex{Bludenz@\textbf{Bludenz}, \emph{Hauptstadt}|pw} bei mir, aber ich muß jedesfalls noch andere Karten
               aus Wien\oindex{Wien@\textbf{Wien}, \emph{Verwaltungsgebiet}|pw} verschreiben, ich denke: \strikeout{(}Innsbruck\oindex{Innsbruck@\textbf{Innsbruck}, \emph{Verwaltungsgebiet}|pw} – Ala\oindex{Ala@\textbf{Ala}, \emph{Hauptstadt}|pw}, Triest\oindex{Triest@\textbf{Triest}, \emph{Verwaltungsgebiet}|pw} – Wien\oindex{Wien@\textbf{Wien}, \emph{Verwaltungsgebiet}|pw}, \substVorne{}\textsuperscript{)}\substDazwischen{}o\substHinten{}der auch anders. Das hängt dann eben ganz von der Tour ab. {\pb}Ich möchte noch sagen, dass ich
               jeden Vorschlag acceptire, (es sei denn Schweiz\oindex{Schweiz@\textbf{Schweiz}|pw}, was mir vielleicht zu theuer wäre) und dass ich voraussichtlich keine
               Abhaltung mehr haben werde.\pend
           
\pstart
           Hat mein Brief mit
               \label{K_L03310-2v}\edtext{\textcolor{gray}{I}nschluß}{\lemma{\textnormal{\emph{Inschluß}}}\Cendnote{\textnormal{Beilage,
                     siehe XXXX Auszeichnungsfehler: Dokument L03309 nicht gefunden.
               }}}\label{K_L03310-2} an Mayer\pwindex{Mayer, Oskar 1876 – 15.\,5.\,1915 München@\textsc{Mayer, Oskar} (1876 – 15.\,5.\,1915 München), \emph{Schriftsteller, Beamter}|pw} Sie erreicht?\pend
           
\pstart
           Bitte, schreiben Sie bald.\pend
           
\pstart
           Herzlichst {\\[\baselineskip]}Ihr {\\[\baselineskip]}\spacefill\mbox{Salten}\pend
           \leftskip=0em{}\selectlanguage{ngerman}\endnumbering\briefempfaengerindex{Schnitzler, Arthur@\textsc{Schnitzler, Arthur}!zzzSalten, Felix@\emph{von Felix Salten}!1900-08-141@{14. 8. 1900}|)be}\mylabel{L03310h}  \newcommand{\dateiname}{L03310}\newcommand{\titel}{Felix Salten an Arthur Schnitzler, 14. 8. 1900}\newcommand{\editorInnen}{Martin Anton Müller und Laura Untner}%% latex-leseansicht-abspann.tex
%% Abspann für die Leseansicht.
%% Der Schalter \ifkorrekturansicht ist bereits durch den Vorspann gesetzt.

%% latex-abspann.tex
%% Gemeinsamer Abspann für Korrekturansicht und Leseansicht.
%% Setzt den Schalter \ifkorrekturansicht voraus (gesetzt in den
%% einbindenden Dateien latex-korrekturansicht-abspann.tex bzw.
%% latex-leseansicht-abspann.tex).
%% ---------------------------------------------------------------

\normalsize

% Das esempio-Environment wird nur in der Leseansicht benötigt
\ifkorrekturansicht\else
\newenvironment{esempio}[3]%
{
    \vspace{1.5ex}
    \rlap{\underline{#1}}
    \par
    \setlength{\parindent}{0cm}
    \nopagebreak
    \leftskip=#2cm
    \rightskip=#3cm
}
{
    \par
}
\fi

\doendnotes{C}
\bigskip
\vfill

\clearpage

\footnotesize

\ifkorrekturansicht
  \lohead{\textsc{register}}
\fi

% theindex-Environment neu definieren ohne reledmac
\makeatletter
\renewenvironment{theindex}{%
  \ifkorrekturansicht
    \section*{\indexname}%
  \else
    \subsubsection*{Index der erwähnten Entitäten}%
  \fi
  \setlength{\parindent}{0pt}%
  \setlength{\parskip}{0pt plus 0.3pt}%
  \let\item\@idxitem
}{%
  \ifkorrekturansicht\clearpage\fi
}
\makeatother

\IfFileExists{\jobname-pw.ind}{\input{\jobname-pw.ind}}{}

% Quellenangabe nur in der Leseansicht
\ifkorrekturansicht\else
% Fallback-Definitionen, falls die .tex-Datei \titel etc. nicht gesetzt hat
\providecommand{\titel}{}
\providecommand{\editorInnen}{}
\providecommand{\dateiname}{\jobname}

\vspace{3cm}

\vfill

\footnotesize
\textsc{Quelle}: \titel. Herausgegeben von {\editorInnen}. In: \emph{Arthur Schnitzler: Briefwechsel mit Autorinnen und Autoren}.
 Digitale Edition, https://schnitzler-briefe.acdh.oeaw.ac.at/{\dateiname}.html (Stand \today)
\fi

\end{document}


