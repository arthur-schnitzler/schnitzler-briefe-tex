%% latex-korrekturansicht-vorspann.tex
%% Vorspann für die Korrekturansicht.
%% Lädt die gemeinsame Datei latex-vorspann.tex mit gesetztem Schalter.

\newif\ifkorrekturansicht
\korrekturansichttrue

\input{../tex-inputs/latex-vorspann}


\section[Arthur Schnitzler an Richard Beer-Hofmann, 23. 6. 1897]{L00690 Arthur Schnitzler an Richard Beer-Hofmann, 23. 6. 1897}
\nopagebreak\mylabel{L00690v}
\rehead{ }\normalsize\beginnumbering\briefempfaengerindex{Beer-Hofmann, Richard@\textsc{Beer-Hofmann, Richard}!zzzSchnitzler, Arthur@\emph{von Arthur Schnitzler}!1897-06-231@{23. 6. 1897}|(be}
\toendnotes[C]{\smallbreak\pagebreak[2]}\Standort{YCGL, MSS 31.}
\physDesc{Brief, 1 Blatt, 4 Seiten, Umschlag, 1084 Zeichen
\newline{}Handschrift: Bleistift, deutsche Kurrent
\newline{}Versand: 1) Stempel: »\nobreak{}\oindex{IX., Alsergrund@\textbf{IX., Alsergrund}, \emph{A.ADM3}|pwk}Wien 9/3, 23. 6. 97, 5–6N\nobreak{}«.   2) Stempel: »\nobreak{}\oindex{Bad Ischl@\textbf{Bad Ischl}, \emph{P.PPL}|pwk}Ischl, 24. 6. 97, 7–8{[}V{]}\nobreak{}«. }
\buchAbdrucke{\weitereDrucke{Arthur Schnitzler, Richard Beer-Hofmann: \emph{Briefwechsel 1891–1931}. Wien, Zürich: \emph{Europaverlag} 1992, S. 110–111.} }\toendnotes[C]{\smallbreak}\pstart{}{\pb}Herrn \textsc{Dr. Rich.
                     Beer-Hofmann}\pend{}\pstart{}\textsc{Ischl\oindex{Bad Ischl@\textbf{Bad Ischl}, \emph{P.PPL}|pw}}\pend{}\pstart{}\textsc{Egelmoos 22}\oindex{Eglmoosgasse@\textbf{Eglmoosgasse}, \emph{Bezirk (A.BZK)}|pw}\pend{}\pstart{}\textsc{\strikeout{N}O.Oe.\oindex{Oberoesterreich@\textbf{Oberösterreich}, \emph{A.ADM1}|pw}}\pend{}{\bigskip}\vspace{1em}
\pstart
           \raggedleft{}{\pb}23. 6. 97. \pend
           \vspace{0.5em}
\pstart
           Lieber Richard. In den letzten Tagen war ich vielfach beſchäftigt
               und beunruhigt; Wohnung ſuchen für »\label{K_L00690-1v}\edtext{ſpäter}{\lemma{\textnormal{\emph{ſpäter}}}\Cendnote{\textnormal{Marie Reinhard\pwindex{Reinhard, Marie 1871-03-13 – 1899-03-18@\textsc{Reinhard, Marie} (1871-03-13 – 1899-03-18), \emph{Gesangspädagoge/Gesangspädagogin}|pwk} und er erwarteten ein
                  gemeinsames Kind.}}}\label{K_L00690-1}«, und die \textsc{inconnue}\pwindex{Gluemer, Marie 03.07.1867 – 16.11.1925@\textsc{Glümer, Marie} (03.07.1867 – 16.11.1925), \emph{Schauspieler/Schauspielerin}|pw} (Sie wiſſen ja wer das iſt) – ich hab Ihnen manchmal ſchreiben wollen, litt
               aber an »Überfülle des Stoffes«. Laſſe mir alles aufs mündliche. Daſs Ihr letzter
               Brief ſehr ſchön {\pb}war, wiſſen Sie ja ſelbſt; es iſt
               recht ſchmachvoll dſs ich mir überlegen mußte, ob ich das ſagen ſoll. Ich mein
               übrigens Ihren vorletzten. Ihr letzter iſt heut geko{\geminationm}en.\pend
           
\pstart
           Alles ſoll beſorgt werden, ſelbſt dasjenige, womit Sie der Vorſehung in die Speichen
               fallen wollen, u. womit ich nicht das Vogel{\pb}futter
               meine.\pend
           
\pstart
           Ich komme \uline{Samſtag}, vielleicht schon Samſtag früh an.
               Bitte, we{\geminationn}’s Ihnen nicht unbequem, beſtellen Sie \uline{mir} (nicht für meine Mama\pwindex{Schnitzler, Louise 1840-07-08 – 1911-09-09@\textsc{Schnitzler, Louise} (1840-07-08 – 1911-09-09)|pwv}, die ſpäter ko{\geminationm}t) das
               Zimmer; iſt’s Ihnen unbequem, ſo ſchreiben Sie dem \textsc{Petter\pwindex{Petter, Leopold 17.11.1850 – 03.07.1917@\textsc{Petter, Leopold} (17.11.1850 – 03.07.1917), \emph{Hotelier/Hotelière}|pw}} eine {\pb}Karte. – Ich ſage nichts näheres über das
               Zimmer, \uline{Sie} haben die ganze Verantwortung.\pend
           
\pstart
           Schwkopf\pwindex{Schwarzkopf, Gustav 07.11.1853 – 13.11.1939@\textsc{Schwarzkopf, Gustav} (07.11.1853 – 13.11.1939), \emph{Schriftsteller/Schriftstellerin}|pw} noch nicht entſchieden, ſchreiben Sie
               ihm zuredend.\pend
           
\pstart
           Ich freue mich ſehr auf Sie, beinah ſehn’ ich mich.\pend
           \pstart Herzlich Ihr \spacefill\mbox{Arthur}\pend{}\selectlanguage{ngerman}\endnumbering\briefempfaengerindex{Beer-Hofmann, Richard@\textsc{Beer-Hofmann, Richard}!zzzSchnitzler, Arthur@\emph{von Arthur Schnitzler}!1897-06-231@{23. 6. 1897}|)be}\mylabel{L00690h}  \normalsize

\doendnotes{C}
\bigskip
\vfill

\clearpage

\footnotesize

\lohead{\textsc{register}}

% Definiere theindex-Environment komplett neu ohne reledmac
\makeatletter
\renewenvironment{theindex}{%
  \section*{\indexname}%
  \setlength{\parindent}{0pt}%
  \setlength{\parskip}{0pt plus 0.3pt}%
  \let\item\@idxitem
}{%
  \clearpage
}
\makeatother

\IfFileExists{\jobname-pw.ind}{\input{\jobname-pw.ind}}{}

\end{document}

      