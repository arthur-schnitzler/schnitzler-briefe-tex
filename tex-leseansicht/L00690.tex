%% latex-leseansicht-vorspann.tex
%% Vorspann für die Leseansicht.
%% Lädt die gemeinsame Datei latex-vorspann.tex mit nicht gesetztem Schalter.

\newif\ifkorrekturansicht
\korrekturansichtfalse

\input{../tex-inputs/latex-vorspann}

\begin{center}
            \textcolor{red}{ENTWURF. ENTZIFFERUNG NOCH NICHT KORREKTURGELESEN}
                      \end{center}
            
               \section[Arthur Schnitzler an Richard Beer-Hofmann, 23. 6. 1897]{ Arthur Schnitzler an Richard Beer-Hofmann,
               23. 6. 1897}\nopagebreak\mylabel{v}\rehead{ }\begin{ledgroupsized}[t]{13cm}\normalsize\beginnumbering\briefempfaengerindex{Beer-Hofmann, Richard@\textsc{Beer-Hofmann, Richard}!zzzSchnitzler, Arthur@\emph{von Arthur Schnitzler}!1897-06-231@{23. 6. 1897}|(be} \toendnotes[C]{\smallbreak\pagebreak[2]} \Standort{YCGL, MSS 31.}
\physDesc{Brief, 1 Blatt, 4 Seiten, Umschlag
\newline{}Handschrift: Bleistift, deutsche Kurrent\newline{}Versand: 1) Stempel: »\nobreak{}\oindex{IX., Alsergrund@\textbf{IX., Alsergrund}|pwk}Wien 9/3, 23. 6. 97, 5–6N\nobreak{}«.  2) Stempel: »\nobreak{}\oindex{Bad Ischl@\textbf{Bad Ischl}|pwk}Ischl, 24. 6. 97, 7–8{[}V{]}\nobreak{}«. }\buchAbdrucke{\weitereDrucke{Arthur Schnitzler, Richard Beer-Hofmann: \emph{Briefwechsel 1891–1931}. Hg. Konstanze Fliedl. Wien, Zürich: \emph{Europaverlag} 1992, S. 110–111.} }\toendnotes[C]{\smallbreak}\pstart{}{\pb}Herrn \textsc{Dr. Rich.
                     Beer-Hofmann}\pend{}\pstart{}\textsc{Ischl\oindex{Bad Ischl@\textbf{Bad Ischl}|pw}}\pend{}\pstart{}\textsc{Egelmoos 22}\oindex{Eglmoosgasse@\textbf{Eglmoosgasse}|pw}\pend{}\pstart{}\textsc{\strikeout{N}O.Oe.\oindex{Oberoesterreich@\textbf{Oberösterreich}|pw}}\pend{}{\bigskip}\pstart
           \raggedleft{}{\pb}23. 6. 97. \pend
           \pstart
           Lieber Richard. In den letzten Tagen war ich vielfach beſchäftigt
               und beunruhigt; Wohnung ſuchen für »\label{K_L00690_1v}\edtext{ſpäter}{\lemma{\textnormal{\emph{ſpäter}}}\Cendnote{\textnormal{Marie Reinhard\pwindex{Reinhard, Marie 13.03.1871 – 18.03.1899@\textsc{Reinhard, Marie} (13.03.1871 – 18.03.1899), \emph{Gesangspädagogin}|pwk} und er erwarteten ein gemeinsames
               Kind.}}}\label{K_L00690_1h}«, und die \textsc{inconnue}\pwindex{Gluemer, Marie 03.07.1867 – 16.11.1925@\textsc{Glümer, Marie} (03.07.1867 – 16.11.1925), \emph{Schauspielerin}|pw} (Sie wiſſen ja wer das iſt) – ich hab Ihnen manchmal
               ſchreiben wollen, litt aber an »Überfülle des Stoffes«. Laſſe mir alles aufs
               mündliche. Daſs Ihr letzter Brief ſehr ſchön {\pb}war, wiſſen Sie ja ſelbſt; es iſt recht ſchmachvoll dſs ich mir überlegen mußte, ob
               ich das ſagen ſoll. Ich mein übrigens Ihren vorletzten. Ihr letzter iſt heut geko{\geminationm}en.\pend
           \pstart
           Alles ſoll beſorgt werden, ſelbſt dasjenige, womit Sie der Vorſehung in die Speichen
               fallen wollen, u. womit ich nicht das Vogel{\pb}futter meine.\pend
           \pstart
           Ich komme \uline{Samſtag}, vielleicht schon Samſtag früh an.
               Bitte, we{\geminationn}’s Ihnen nicht unbequem, beſtellen Sie \uline{mir} (nicht für meine Mama\pwindex{Schnitzler, Louise 08.07.1840 – 09.09.1911@\textsc{Schnitzler, Louise} (08.07.1840 – 09.09.1911)|pwv}, die ſpäter ko{\geminationm}t) das
               Zimmer; iſt’s Ihnen unbequem, ſo ſchreiben Sie dem \textsc{Petter\pwindex{Petter, Leopold 17.11.1850 – 03.07.1917@\textsc{Petter, Leopold} (17.11.1850 – 03.07.1917), \emph{Hotelier}|pw}} eine {\pb}Karte. – Ich ſage nichts näheres
               über das Zimmer, \uline{Sie} haben die ganze
               Verantwortung.\pend
           \pstart
           Schwkopf\pwindex{Schwarzkopf, Gustav 07.11.1853 – 13.11.1939@\textsc{Schwarzkopf, Gustav} (07.11.1853 – 13.11.1939), \emph{Schriftsteller}|pw} noch nicht entſchieden, ſchreiben Sie
               ihm zuredend.\pend
           \pstart
           Ich freue mich ſehr auf Sie, beinah ſehn’ ich mich.\pend
           \pstart Herzlich Ihr \spacefill\mbox{Arthur}\pend{}\endnumbering\briefempfaengerindex{Beer-Hofmann, Richard@\textsc{Beer-Hofmann, Richard}!zzzSchnitzler, Arthur@\emph{von Arthur Schnitzler}!1897-06-231@{23. 6. 1897}|)be}\mylabel{h}\end{ledgroupsized}  \newcommand{\dateiname}{L00690}\newcommand{\titel}{Arthur Schnitzler an Richard Beer-Hofmann, 23. 6. 1897}\newcommand{\editorInnen}{Martin Anton Müller und Gerd-Hermann Susen}%% latex-leseansicht-abspann.tex
%% Abspann für die Leseansicht.
%% Der Schalter \ifkorrekturansicht ist bereits durch den Vorspann gesetzt.

%% latex-abspann.tex
%% Gemeinsamer Abspann für Korrekturansicht und Leseansicht.
%% Setzt den Schalter \ifkorrekturansicht voraus (gesetzt in den
%% einbindenden Dateien latex-korrekturansicht-abspann.tex bzw.
%% latex-leseansicht-abspann.tex).
%% ---------------------------------------------------------------

\normalsize

% Das esempio-Environment wird nur in der Leseansicht benötigt
\ifkorrekturansicht\else
\newenvironment{esempio}[3]%
{
    \vspace{1.5ex}
    \rlap{\underline{#1}}
    \par
    \setlength{\parindent}{0cm}
    \nopagebreak
    \leftskip=#2cm
    \rightskip=#3cm
}
{
    \par
}
\fi

\doendnotes{C}
\bigskip
\vfill

\clearpage

\footnotesize

\ifkorrekturansicht
  \lohead{\textsc{register}}
\fi

% theindex-Environment neu definieren ohne reledmac
\makeatletter
\renewenvironment{theindex}{%
  \ifkorrekturansicht
    \section*{\indexname}%
  \else
    \subsubsection*{Index der erwähnten Entitäten}%
  \fi
  \setlength{\parindent}{0pt}%
  \setlength{\parskip}{0pt plus 0.3pt}%
  \let\item\@idxitem
}{%
  \ifkorrekturansicht\clearpage\fi
}
\makeatother

\IfFileExists{\jobname-pw.ind}{\input{\jobname-pw.ind}}{}

% Quellenangabe nur in der Leseansicht
\ifkorrekturansicht\else
% Fallback-Definitionen, falls die .tex-Datei \titel etc. nicht gesetzt hat
\providecommand{\titel}{}
\providecommand{\editorInnen}{}
\providecommand{\dateiname}{\jobname}

\vspace{3cm}

\vfill

\footnotesize
\textsc{Quelle}: \titel. Herausgegeben von {\editorInnen}. In: \emph{Arthur Schnitzler: Briefwechsel mit Autorinnen und Autoren}.
 Digitale Edition, https://schnitzler-briefe.acdh.oeaw.ac.at/{\dateiname}.html (Stand \today)
\fi

\end{document}


      