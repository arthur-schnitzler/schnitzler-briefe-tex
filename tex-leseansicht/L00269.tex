%% latex-korrekturansicht-vorspann.tex
%% Vorspann für die Korrekturansicht.
%% Lädt die gemeinsame Datei latex-vorspann.tex mit gesetztem Schalter.

\newif\ifkorrekturansicht
\korrekturansichttrue

\input{../tex-inputs/latex-vorspann}


\section[Hugo von Hofmannsthal an Arthur Schnitzler, {[}12. 10. 1893{]}]{L00269 Hugo von Hofmannsthal an Arthur Schnitzler, {[}12. 10. 1893{]}}
\nopagebreak\mylabel{L00269v}
\rehead{ }\normalsize\beginnumbering\briefempfaengerindex{Schnitzler, Arthur@\textsc{Schnitzler, Arthur}!zzzHofmannsthal, Hugo von@\emph{von Hugo von Hofmannsthal}!1893-10-121@{{[}12. 10. 1893{]}}|(be}
\toendnotes[C]{\smallbreak\pagebreak[2]}\Standort{CUL, Schnitzler, B 43.}
\physDesc{Briefkarte, 264 Zeichen
\newline{}Handschrift: Bleistift, lateinische Kurrent
\newline{}Schnitzler: mit Bleistift datiert: »12/X 93« und nummeriert: »58« }
\buchAbdrucke{\weitereDrucke{Hugo von Hofmannsthal, Arthur Schnitzler: \emph{Briefwechsel}. Frankfurt am Main: \emph{S. Fischer} 1964, S. 46.} }\toendnotes[C]{\smallbreak}
\pstart{}{\pb}Sie lieber Arthur!\pend\vspace{0.5em}
\pstart
           Es wäre doch vielleicht nicht absolut verächtlich oder überflüssig, wenn wir einmal
               ein paar Viertelstunden zusammen verbringen {\pb}könnten.\pend
           
\pstart
           Ich halte mir \label{K_L00269-1v}\edtext{Sonntagnachmittag}{\lemma{\textnormal{\emph{Sonntagnachmittag}}}\Cendnote{\textnormal{Das erhoffte Treffen fand tatsächlich am
                     Sonntag, dem 15. 10. 1893 statt.}}}\label{K_L00269-1} frei.\pend
           
\pstart
           Das verpflichtet \strikeout{im Allgemeinen} natürlich zu nichts.
               Aber im Allgemeinen!!\pend
           \pstart \spacefill\mbox{Hugo}\pend{}\selectlanguage{ngerman}\endnumbering\briefempfaengerindex{Schnitzler, Arthur@\textsc{Schnitzler, Arthur}!zzzHofmannsthal, Hugo von@\emph{von Hugo von Hofmannsthal}!1893-10-121@{{[}12. 10. 1893{]}}|)be}\mylabel{L00269h}  \normalsize

\doendnotes{C}
\bigskip
\vfill

\clearpage

\footnotesize

\lohead{\textsc{register}}

% Definiere theindex-Environment komplett neu ohne reledmac
\makeatletter
\renewenvironment{theindex}{%
  \section*{\indexname}%
  \setlength{\parindent}{0pt}%
  \setlength{\parskip}{0pt plus 0.3pt}%
  \let\item\@idxitem
}{%
  \clearpage
}
\makeatother

\IfFileExists{\jobname-pw.ind}{\input{\jobname-pw.ind}}{}

\end{document}

      