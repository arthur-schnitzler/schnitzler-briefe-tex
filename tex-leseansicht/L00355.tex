%% latex-korrekturansicht-vorspann.tex
%% Vorspann für die Korrekturansicht.
%% Lädt die gemeinsame Datei latex-vorspann.tex mit gesetztem Schalter.

\newif\ifkorrekturansicht
\korrekturansichttrue

\input{../tex-inputs/latex-vorspann}


\section[Richard Beer-Hofmann an Arthur Schnitzler, {[}18. 7. 1894{]}]{L00355 Richard Beer-Hofmann an Arthur Schnitzler, {[}18. 7. 1894{]}}
\nopagebreak\mylabel{L00355v}
\rehead{ }\normalsize\beginnumbering\briefempfaengerindex{Schnitzler, Arthur@\textsc{Schnitzler, Arthur}!zzzBeer-Hofmann, Richard@\emph{von Richard Beer-Hofmann}!1894-07-182@{{[}18. 7. 1894{]}}|(be}
\toendnotes[C]{\smallbreak\pagebreak[2]}\Standort{CUL, Schnitzler, B 8.}
\physDesc{Brief, 1 Blatt, 1 Seite, 321 Zeichen
\newline{}Handschrift: Bleistift, lateinische Kurrent
\newline{}Schnitzler: mit Bleistift datiert: »18/7 94« und nummeriert: »34« }
\buchAbdrucke{\weitereDrucke{Hermann Bahr, Arthur Schnitzler: \emph{Briefwechsel, Aufzeichnungen, Dokumente (1891–1931)}. Göttingen: \emph{Wallstein} 2018.} }\toendnotes[C]{\smallbreak}
\pstart
           \noindent{}{\pb}Lieber Arthur! Habe
               den Brief irrthümlich geöffnet \strikeout{B} sie \strikeout{A} antworten wol Bahr\pwindex{Bahr, Hermann 19.07.1863 – 15.01.1934@\textsc{Bahr, Hermann} (19.07.1863 – 15.01.1934), \emph{Schriftsteller/Schriftstellerin, Kritiker/Kritikerin}|pw} dass er Samstag hieher kommen soll? Mit Salzburg\oindex{Salzburg@\textbf{Salzburg}, \emph{A.ADM2}|pw} wird es vorläufig nichts sein: Hugo\pwindex{Hofmannsthal, Hugo von 1874-02-01 – 1929-07-15@\textsc{Hofmannsthal, Hugo von} (1874-02-01 – 1929-07-15), \emph{Schriftsteller/Schriftstellerin}|pw} wird \label{K_L00355-1v}\edtext{auch
               nicht}{\lemma{\textnormal{\emph{auch
               nicht}}}\Cendnote{\textnormal{Hofmannsthal\pwindex{Hofmannsthal, Hugo von 1874-02-01 – 1929-07-15@\textsc{Hofmannsthal, Hugo von} (1874-02-01 – 1929-07-15), \emph{Schriftsteller/Schriftstellerin}|pwk} trauerte um Josefine Wertheimstein\pwindex{Wertheimstein, Josephine von 1820-11-19 – 1894-07-16@\textsc{Wertheimstein, Josephine von} (1820-11-19 – 1894-07-16), \emph{männliche Salonnière/Salonnière}|pwk}, die am
                     16. 7. 1894 gestorben war.}}}\label{K_L00355-1} von Fusch\oindex{Bad Fusch@\textbf{Bad Fusch}, \emph{A.ADM3}|pw} wo er seit ein paar Tagen ist kommen wollen. Verschieben
               wir also die Sache\pend
           
\pstart
           Was ist Nachmittag? Ich bin jedenfalls bis circa ½ 5 zu Hause\pend
           \pstart Herzlich \spacefill\mbox{Richard}\pend{}\selectlanguage{ngerman}\endnumbering\briefempfaengerindex{Schnitzler, Arthur@\textsc{Schnitzler, Arthur}!zzzBeer-Hofmann, Richard@\emph{von Richard Beer-Hofmann}!1894-07-182@{{[}18. 7. 1894{]}}|)be}\mylabel{L00355h}  \normalsize

\doendnotes{C}
\bigskip
\vfill

\clearpage

\footnotesize

\lohead{\textsc{register}}

% Definiere theindex-Environment komplett neu ohne reledmac
\makeatletter
\renewenvironment{theindex}{%
  \section*{\indexname}%
  \setlength{\parindent}{0pt}%
  \setlength{\parskip}{0pt plus 0.3pt}%
  \let\item\@idxitem
}{%
  \clearpage
}
\makeatother

\IfFileExists{\jobname-pw.ind}{\input{\jobname-pw.ind}}{}

\end{document}

      