%% latex-leseansicht-vorspann.tex
%% Vorspann für die Leseansicht.
%% Lädt die gemeinsame Datei latex-vorspann.tex mit nicht gesetztem Schalter.

\newif\ifkorrekturansicht
\korrekturansichtfalse

\input{../tex-inputs/latex-vorspann}


\section[ Felix Salten an Arthur Schnitzler, 19. 9. {[}1903{]}]{L03344 Felix Salten an Arthur Schnitzler,  19. 9. [1903]}
\nopagebreak\mylabel{L03344v}
\rehead{ }\normalsize\beginnumbering\briefempfaengerindex{Schnitzler, Arthur@\textsc{Schnitzler, Arthur}!zzzSalten, Felix@\emph{von Felix Salten}!1903-09-191@{19. 9. [1903]}|(be}
\toendnotes[C]{\smallbreak\pagebreak[2]}
\correspDesc{Versand  durch Felix Salten am 19. 9. [1903] in Wien
\newline{}Erhalt  durch Arthur Schnitzler im Zeitraum [19. 9. 1903
                  – 22. 9. 1903?] in Wien}\toendnotes[C]{\smallbreak}
\Standort{CUL, Schnitzler, B 89, A 2.}
\physDesc{Brief, 1 Blatt, 2 Seiten, 612 Zeichen
\newline{}Handschrift: Bleistift, lateinische Kurrent
\newline{}Ordnung: mit Bleistift von unbekannter Hand nummeriert: »169« }\toendnotes[C]{\smallbreak}
\pstart
           {\pb}\textcolor{gray}{\textbf{DIE}}\pend
           
\pstart
           \textcolor{gray}{\textbf{ZEIT\orgindex{Zeit@Die Zeit|pw}}}\hfill \textcolor{gray}{\textbf{\emph{WIEN}\oindex{Wien@\textbf{Wien}, \emph{Verwaltungsgebiet}|pw}}}{ }19/9.\pend
           
\pstart
           \textcolor{gray}{\textbf{Wien\oindex{Wien@\textbf{Wien}, \emph{Verwaltungsgebiet}|pw}er Tageszeitung}}\hfill \textcolor{gray}{\textbf{\emph{I. Wipplingerstrasse 38\oindex{Wien@\textbf{Wien}!I., Innere Stadt@\textbf{I., Innere Stadt}!Wipplingerstraße@\textbf{Wipplingerstraße}, \emph{Straße}|pw}}}}\pend
           
\pstart
           \textcolor{gray}{\textbf{Herausgeber:}}\pend
           
\pstart
           \textcolor{gray}{\textbf{\textbf{Prof. Dr. I. Singer\pwindex{Singer, Isidor 16.\,1.\,1857 Budapest – 8.\,12.\,1927 Wien@\textsc{Singer, Isidor} (16.\,1.\,1857 Budapest – 8.\,12.\,1927 Wien), \emph{Journalist, Herausgeber, Soziologe}|pw}}}}\pend
           
\pstart
           \textcolor{gray}{\textbf{\textbf{Dr. Heinrich Kanner\pwindex{Kanner, Heinrich 9.\,11.\,1864 Galați – 15.\,2.\,1930 Wien@\textsc{Kanner, Heinrich} (9.\,11.\,1864 Galați – 15.\,2.\,1930 Wien), \emph{Herausgeber, Publizist}|pw}}}}\pend
           
\pstart
           \textcolor{gray}{\textbf{\textbf{Redaction}}}\pend
           
\pstart
           \textcolor{gray}{\textbf{Telegramm-Adresse: \so{Zeit}\orgindex{Zeit@Die Zeit|pw}\so{,}{ }\so{Wien}\oindex{Wien@\textbf{Wien}, \emph{Verwaltungsgebiet}|pw}}}\pend
           
\pstart
           \textcolor{gray}{\textbf{Interurbanes Telephon Nr. 15.988}}\pend
           
\pstart
           \textcolor{gray}{\textbf{= Telephone Nr. 17.040, 17.041 =}}\pend
           \vspace{0.5em}
\pstart
           Lieber, die Sache ist folgende: Die Zt\orgindex{Zeit@Die Zeit|pw} veranstaltet ein \label{K_L03344-1v}\edtext{Preisaus{[}s{]}chreiben}{\lemma{\textnormal{\emph{Preisausschreiben}}}\Cendnote{\textnormal{Das Preisausschreiben wurde am 4. 10. 1903 beworben. Schnitzler
                  fand sich nicht in der Jury. Stattdessen waren in dieser – neben den anderen von
                     Salten\pwindex{Salten, Felix 6.\,9.\,1869 Budapest – 8.\,10.\,1945 Zürich@\textsc{Salten, Felix} (6.\,9.\,1869 Budapest – 8.\,10.\,1945 Zürich), \emph{Schriftsteller, Journalist, Chefredakteur}|pwk} Genannten – Karl Glossy\pwindex{Glossy, Karl 7.\,3.\,1848 Wien – 9.\,9.\,1937 ebd.@\textsc{Glossy, Karl} (7.\,3.\,1848 Wien – 9.\,9.\,1937 ebd.), \emph{Schriftsteller, Museumsleiter, Zensurbeirat}|pwk}, August
                     Sauer\pwindex{Sauer, August 12.\,10.\,1855 Wiener Neustadt – 17.\,9.\,1926 Prag@\textsc{Sauer, August} (12.\,10.\,1855 Wiener Neustadt – 17.\,9.\,1926 Prag)|pwk} und Isidor Singer\pwindex{Singer, Isidor 16.\,1.\,1857 Budapest – 8.\,12.\,1927 Wien@\textsc{Singer, Isidor} (16.\,1.\,1857 Budapest – 8.\,12.\,1927 Wien), \emph{Journalist, Herausgeber, Soziologe}|pwk}
                  vertreten.}}}\label{K_L03344-1} für Feuilleton, 3 Preise zu 800, 400 {\kaufmannsund}{ }\substVorne{}\textsuperscript{3}\substDazwischen{}2\substHinten{}00 Kronen. Noch Geheimnis. Ich soll Sie nun ersuchen, in die Jury
               einzutreten, die dann aus Burckhard\pwindex{Burckhard, Max Eugen 14.\,7.\,1854 Korneuburg – 16.\,3.\,1912 Wien@\textsc{Burckhard, Max Eugen} (14.\,7.\,1854 Korneuburg – 16.\,3.\,1912 Wien), \emph{Schriftsteller, Rechtswissenschaftler, Theaterleiter}|pw}, Muther\pwindex{Muther, Richard 25.\,2.\,1860 Ohrdruf – 28.\,6.\,1909 Międzygórze@\textsc{Muther, Richard} (25.\,2.\,1860 Ohrdruf – 28.\,6.\,1909 Międzygórze), \emph{Kunsthistoriker}|pw}, Saar\pwindex{Saar, Ferdinand von 30.\,9.\,1833 Wien – 24.\,7.\,1906 ebd.@\textsc{Saar, Ferdinand von} (30.\,9.\,1833 Wien – 24.\,7.\,1906 ebd.), \emph{Schriftsteller}|pw}, Ihnen und mir bestehen würde. Arbeit hätten Sie nicht besonders viel
               daran, weil die Feuilleton-Redaction\orgindex{Zeit@Die Zeit|pwv} natürlich die Auslese trifft {\kaufmannsund} den Herren nur jene Arbeiten vorlegt, die zur
               Prämirung in Betracht kommen. Vielleicht sind Sie so liebenswürdig und theilen mir
               rasch mit, {\pb}ob Sie ja oder nein
               dazu sagen, weil die Sache in den nächsten Tagen publicirt werden soll.\pend
           
\pstart
           Aufrichtig {\\[\baselineskip]}Ihr {\\[\baselineskip]}\spacefill\mbox{Salten}\pend
           \leftskip=0em{}\selectlanguage{ngerman}\endnumbering\briefempfaengerindex{Schnitzler, Arthur@\textsc{Schnitzler, Arthur}!zzzSalten, Felix@\emph{von Felix Salten}!1903-09-191@{19. 9. [1903]}|)be}\mylabel{L03344h}  \newcommand{\dateiname}{L03344}\newcommand{\titel}{Felix Salten an Arthur Schnitzler, 19. 9. [1903]}\newcommand{\editorInnen}{Martin Anton Müller und Laura Untner}%% latex-leseansicht-abspann.tex
%% Abspann für die Leseansicht.
%% Der Schalter \ifkorrekturansicht ist bereits durch den Vorspann gesetzt.

%% latex-abspann.tex
%% Gemeinsamer Abspann für Korrekturansicht und Leseansicht.
%% Setzt den Schalter \ifkorrekturansicht voraus (gesetzt in den
%% einbindenden Dateien latex-korrekturansicht-abspann.tex bzw.
%% latex-leseansicht-abspann.tex).
%% ---------------------------------------------------------------

\normalsize

% Das esempio-Environment wird nur in der Leseansicht benötigt
\ifkorrekturansicht\else
\newenvironment{esempio}[3]%
{
    \vspace{1.5ex}
    \rlap{\underline{#1}}
    \par
    \setlength{\parindent}{0cm}
    \nopagebreak
    \leftskip=#2cm
    \rightskip=#3cm
}
{
    \par
}
\fi

\doendnotes{C}
\bigskip
\vfill

\clearpage

\footnotesize

\ifkorrekturansicht
  \lohead{\textsc{register}}
\fi

% theindex-Environment neu definieren ohne reledmac
\makeatletter
\renewenvironment{theindex}{%
  \ifkorrekturansicht
    \section*{\indexname}%
  \else
    \subsubsection*{Index der erwähnten Entitäten}%
  \fi
  \setlength{\parindent}{0pt}%
  \setlength{\parskip}{0pt plus 0.3pt}%
  \let\item\@idxitem
}{%
  \ifkorrekturansicht\clearpage\fi
}
\makeatother

\IfFileExists{\jobname-pw.ind}{\input{\jobname-pw.ind}}{}

% Quellenangabe nur in der Leseansicht
\ifkorrekturansicht\else
% Fallback-Definitionen, falls die .tex-Datei \titel etc. nicht gesetzt hat
\providecommand{\titel}{}
\providecommand{\editorInnen}{}
\providecommand{\dateiname}{\jobname}

\vspace{3cm}

\vfill

\footnotesize
\textsc{Quelle}: \titel. Herausgegeben von {\editorInnen}. In: \emph{Arthur Schnitzler: Briefwechsel mit Autorinnen und Autoren}.
 Digitale Edition, https://schnitzler-briefe.acdh.oeaw.ac.at/{\dateiname}.html (Stand \today)
\fi

\end{document}


