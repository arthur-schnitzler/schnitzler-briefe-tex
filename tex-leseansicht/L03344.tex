%% latex-korrekturansicht-vorspann.tex
%% Vorspann für die Korrekturansicht.
%% Lädt die gemeinsame Datei latex-vorspann.tex mit gesetztem Schalter.

\newif\ifkorrekturansicht
\korrekturansichttrue

\input{../tex-inputs/latex-vorspann}


\section[ Felix Salten an Arthur Schnitzler, 19. 9. {[}1903{]}]{L03344 Felix Salten an Arthur Schnitzler, 19. 9. {[}1903{]}}
\nopagebreak\mylabel{L03344v}
\rehead{ }\normalsize\beginnumbering\briefempfaengerindex{Schnitzler, Arthur@\textsc{Schnitzler, Arthur}!zzzSalten, Felix@\emph{von Felix Salten}!1903-09-191@{19. 9. {[}1903{]}}|(be}
\toendnotes[C]{\smallbreak\pagebreak[2]}\Standort{CUL, Schnitzler, B 89, A 2.}
\physDesc{Brief, 1 Blatt, 2 Seiten, 612 Zeichen
\newline{}Handschrift: Bleistift, lateinische Kurrent
\newline{}Ordnung: mit Bleistift von unbekannter Hand nummeriert: »169« }\toendnotes[C]{\smallbreak}
\pstart
           {\pb}\textcolor{gray}{\textbf{DIE}}\pend
           
\pstart
           \textcolor{gray}{\textbf{ZEIT\orgindex{Zeit@Die Zeit|pw}}}\hfill \textcolor{gray}{\textbf{\emph{WIEN}\oindex{Wien@\textbf{Wien}, \emph{A.ADM2}|pw}}}{ }19/9.\pend
           
\pstart
           \textcolor{gray}{\textbf{Wien\oindex{Wien@\textbf{Wien}, \emph{A.ADM2}|pw}er Tageszeitung}}\hfill \textcolor{gray}{\textbf{\emph{I. Wipplingerstrasse 38\oindex{Wipplingerstrasse@\textbf{Wipplingerstraße}, \emph{Straße (K.STR)}|pw}}}}\pend
           
\pstart
           \textcolor{gray}{\textbf{Herausgeber:}}\pend
           
\pstart
           \textcolor{gray}{\textbf{\textbf{Prof. Dr. I. Singer\pwindex{Singer, Isidor 16.01.1857 – 08.12.1927@\textsc{Singer, Isidor} (16.01.1857 – 08.12.1927), \emph{Journalist/Journalistin, Herausgeber/Herausgeberin, Soziologe/Soziologin}|pw}}}}\pend
           
\pstart
           \textcolor{gray}{\textbf{\textbf{Dr. Heinrich Kanner\pwindex{Kanner, Heinrich 09.11.1864 – 15.02.1930@\textsc{Kanner, Heinrich} (09.11.1864 – 15.02.1930), \emph{Herausgeber/Herausgeberin, Publizist/Publizistin}|pw}}}}\pend
           
\pstart
           \textcolor{gray}{\textbf{\textbf{Redaction}}}\pend
           
\pstart
           \textcolor{gray}{\textbf{Telegramm-Adresse: \so{Zeit}\orgindex{Zeit@Die Zeit|pw}\so{,}{ }\so{Wien}\oindex{Wien@\textbf{Wien}, \emph{A.ADM2}|pw}}}\pend
           
\pstart
           \textcolor{gray}{\textbf{Interurbanes Telephon Nr. 15.988}}\pend
           
\pstart
           \textcolor{gray}{\textbf{= Telephone Nr. 17.040, 17.041 =}}\pend
           \vspace{0.5em}
\pstart
           Lieber, die Sache ist folgende: Die Zt\orgindex{Zeit@Die Zeit|pw} veranstaltet ein \label{K_L03344-1v}\edtext{Preisaus{[}s{]}chreiben}{\lemma{\textnormal{\emph{Preisausschreiben}}}\Cendnote{\textnormal{Das Preisausschreiben wurde am 4. 10. 1903 beworben. Schnitzler
                  fand sich nicht in der Jury. Stattdessen waren in dieser – neben den anderen von
                     Salten\pwindex{Salten, Felix 06.09.1869 – 08.10.1945@\textsc{Salten, Felix} (06.09.1869 – 08.10.1945), \emph{Schriftsteller/Schriftstellerin, Journalist/Journalistin, Chefredakteur/Chefredakteurin}|pwk} Genannten – Karl Glossy\pwindex{Glossy, Karl 07.03.1848 – 09.09.1937@\textsc{Glossy, Karl} (07.03.1848 – 09.09.1937), \emph{Schriftsteller/Schriftstellerin, Museumsleiter/Museumsleiterin, Zensurbeirat/Zensurbeirätin}|pwk}, August
                     Sauer\pwindex{Sauer, August 12.10.1855 – 17.09.1926@\textsc{Sauer, August} (12.10.1855 – 17.09.1926)|pwk} und Isidor Singer\pwindex{Singer, Isidor 16.01.1857 – 08.12.1927@\textsc{Singer, Isidor} (16.01.1857 – 08.12.1927), \emph{Journalist/Journalistin, Herausgeber/Herausgeberin, Soziologe/Soziologin}|pwk}
                  vertreten.}}}\label{K_L03344-1} für Feuilleton, 3 Preise zu 800, 400 {\kaufmannsund}{ }\substVorne{}\textsuperscript{3}\substDazwischen{}2\substHinten{}00 Kronen. Noch Geheimnis. Ich soll Sie nun ersuchen, in die Jury
               einzutreten, die dann aus Burckhard\pwindex{Burckhard, Max Eugen 14.07.1854 – 16.03.1912@\textsc{Burckhard, Max Eugen} (14.07.1854 – 16.03.1912), \emph{Schriftsteller/Schriftstellerin, Rechtswissenschaftler/Rechtswissenschaftlerin, Theaterleiter/Theaterleiterin}|pw}, Muther\pwindex{Muther, Richard 25.02.1860 – 1909-06-28@\textsc{Muther, Richard} (25.02.1860 – 1909-06-28), \emph{Kunsthistoriker/Kunsthistorikerin}|pw}, Saar\pwindex{Saar, Ferdinand von 30.09.1833 – 24.07.1906@\textsc{Saar, Ferdinand von} (30.09.1833 – 24.07.1906), \emph{Schriftsteller/Schriftstellerin}|pw}, Ihnen und mir bestehen würde. Arbeit hätten Sie nicht besonders viel
               daran, weil die Feuilleton-Redaction\orgindex{Zeit@Die Zeit|pwv} natürlich die Auslese trifft {\kaufmannsund} den Herren nur jene Arbeiten vorlegt, die zur
               Prämirung in Betracht kommen. Vielleicht sind Sie so liebenswürdig und theilen mir
               rasch mit, {\pb}ob Sie ja oder nein
               dazu sagen, weil die Sache in den nächsten Tagen publicirt werden soll.\pend
           
\pstart
           Aufrichtig {\\[\baselineskip]}Ihr {\\[\baselineskip]}\spacefill\mbox{Salten}\pend
           \leftskip=0em{}\selectlanguage{ngerman}\endnumbering\briefempfaengerindex{Schnitzler, Arthur@\textsc{Schnitzler, Arthur}!zzzSalten, Felix@\emph{von Felix Salten}!1903-09-191@{19. 9. {[}1903{]}}|)be}\mylabel{L03344h}  \normalsize

\doendnotes{C}
\bigskip
\vfill

\clearpage

\footnotesize

\lohead{\textsc{register}}

% Definiere theindex-Environment komplett neu ohne reledmac
\makeatletter
\renewenvironment{theindex}{%
  \section*{\indexname}%
  \setlength{\parindent}{0pt}%
  \setlength{\parskip}{0pt plus 0.3pt}%
  \let\item\@idxitem
}{%
  \clearpage
}
\makeatother

\IfFileExists{\jobname-pw.ind}{\input{\jobname-pw.ind}}{}

\end{document}

      