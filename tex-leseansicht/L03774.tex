%% latex-korrekturansicht-vorspann.tex
%% Vorspann für die Korrekturansicht.
%% Lädt die gemeinsame Datei latex-vorspann.tex mit gesetztem Schalter.

\newif\ifkorrekturansicht
\korrekturansichttrue

\input{../tex-inputs/latex-vorspann}


\section[Arthur Schnitzler an Stefan Zweig, 27. 11. 1914]{L03774 Arthur Schnitzler an Stefan Zweig, 27. 11. 1914}
\nopagebreak\mylabel{L03774v}
\rehead{ }\normalsize\beginnumbering\briefempfaengerindex{Zweig, Stefan@\textsc{Zweig, Stefan}!zzzSchnitzler, Arthur@\emph{von Arthur Schnitzler}!1914-11-271@{27. 11. 1914}|(be}
\toendnotes[C]{\smallbreak\pagebreak[2]}\Standort{Jerusalem, National Library of Israel, ARC. Ms. Var. 305 1 58 Stefan Zweig Collection.}
\physDesc{Brief, 1 Blatt, 2 Seiten, 4598 Zeichen
\newline{}Schreibmaschine
\newline{}Handschrift: schwarze Tinte (\noindent{}Unterschrift)
\newline{}Beilage: ms., 5 Blatt, 5 Seiten, ms. paginiert. Korrekturen mit Bleistift
                                 in lateinischer Kurrent. Auf S. 1 von unbekannter Hand mit blauem
                                 Stift: »XI. 1914« }\toendnotes[C]{\smallbreak}
\pstart
           {\pb}\textcolor{gray}{\textbf{Dr. Arthur Schnitzler}}\hfill 27. 11. 1914. \pend
           
\pstart
           \textcolor{gray}{\textbf{Wien XVIII. Sternwartestrasse 71\oindex{Sternwartestrasse 71@\textbf{Sternwartestraße 71}, \emph{Wohngebäude (K.WHS)}|pw}}}\pend
           
\pstart\center{}Lieber Herr Doktor.\pend\vspace{0.5em}
\pstart
           Beifolgend die Berichtigung\pwindex{Brief Artur Schnitzlers@\emph{Ein Brief Artur Schnitzlers}|pwv}
               oder Erklärung\pwindex{Brief Artur Schnitzlers@\emph{Ein Brief Artur Schnitzlers}|pwv} oder wie sie
               es nennen wollen. Ich wünschte gern zu wissen, 1., ob Sie im Ganzen damit
               einverstanden sind, 2. ob Sie eine \label{K_L03774-1v}\edtext{Veröffentlichung von Seite 4 an}{\lemma{\textnormal{\emph{Veröffentlichung … an}}}\Cendnote{\textnormal{Wie
                  hier schon antizipiert, wurde dieser Teil nicht in die Veröffentlichung\pwindex{Brief Artur Schnitzlers@\emph{Ein Brief Artur Schnitzlers}|pwkv} im \emph{Journal de Genève}\pwindex{Journal de Geneve@\emph{Journal de Genève}|pwk} (Arthur Schnitzler, Romain Rolland\pwindex{Rolland, Romain 29.01.1866 – 30.12.1944@\textsc{Rolland, Romain} (29.01.1866 – 30.12.1944), \emph{Schriftsteller/Schriftstellerin}|pwk} [Einleitung und Übersetzung]: \emph{Une protestation d’Arthur Schnitzler}\pwindex{Une protestation DArthur Schnitzler@\emph{Une protestation d’Arthur Schnitzler}|pwk}. In:
                        \emph{Journal de Genève}\pwindex{Journal de Geneve@\emph{Journal de Genève}|pwk}, Jg. 85, 21. 12. 1914, 3. Ausgabe, S. [1].)
                  bzw. in der \emph{Neuen Zürcher Zeitung}\pwindex{Neue Zuercher Zeitung@\emph{Neue Zürcher Zeitung}|pwk} aufgenommen (\emph{Ein Brief Artur Schnitzlers}\pwindex{Brief Artur Schnitzlers@\emph{Ein Brief Artur Schnitzlers}|pwk}. In: \emph{Neue Zürcher Zeitung}\pwindex{Neue Zuercher Zeitung@\emph{Neue Zürcher Zeitung}|pwk}, Jg. 135, Nr. 1700,
                        22. 12. 1914, 2. Mittagsblatt, S. 2).}}}\label{K_L03774-1} für
               notwendig und opportun hielten. Haben Sie nichts einzuwenden,so senden Sie
               freundlichst unserer Verabredung gemäss das Ganze mit meinen verehrungsvollen Grüssen
                  anRolland\pwindex{Rolland, Romain 29.01.1866 – 30.12.1944@\textsc{Rolland, Romain} (29.01.1866 – 30.12.1944), \emph{Schriftsteller/Schriftstellerin}|pw}. Was in dieser Angelegenheit
               anderswo und eventuell hier geschehen könnte oder sollte, möchte ich doch gerne
               persönlich oder wenigstens telefonisch mit Ihnen besprechen. Vielleicht schreiben Sie
               mir ein Wort, wann man Sie in den nächsten Tagen anrufen darf. Wie telefoniert man
               denn an den Regierungsrat Winternitz\pwindex{Winternitz, Jakob von 03.03.1843 – 26.01.1921@\textsc{Winternitz, Jakob von} (03.03.1843 – 26.01.1921), \emph{Ministerialbeamter/Ministerialbeamte}|pw}; ich habe
               mich bisher noch nicht an ihn gewandt.\pend
           
\pstart
           {\pb}Zu Ihrer militärischen Verwendung\orgindex{Kriegsarchiv@Kriegsarchiv|pwv} kann man Ihnen gratulieren, glaube ich. Sie
               werden Interessanteres und wahrscheinlich sogar authentischeres erfahren als die
               Leute an der Front. Der Baron Winterstein\pwindex{Winterstein, Alfred von 25.09.1885 – 28.04.1958@\textsc{Winterstein, Alfred von} (25.09.1885 – 28.04.1958), \emph{Schriftsteller/Schriftstellerin, Psychoanalytiker/Psychoanalytikerin, Beamter/Beamte}|pw} hat
               uns \label{K_L03774-2v}\edtext{neulich}{\lemma{\textnormal{\emph{neulich}}}\Cendnote{\textnormal{Vgl. A. S.: \emph{Tagebuch}, 25. 11. 1914.}}}\label{K_L03774-2}
               allerlei Anregendes erzählt; wir hätten sie gern dabei gehabt.\pend
           
\pstart
           Herzlichst grüssend{\\[\baselineskip]}Ihr{\\[\baselineskip]}\spacefill\mbox{{[}hs.:{]} Arthur Schnitzler}\pend
           \leftskip=0em{}
\pstart
           \noindent{}{[}ms.:{]} Beiliegend \label{K_L03774-3v}\edtext{zwei Exemplare\pwindex{Brief Artur Schnitzlers@\emph{Ein Brief Artur Schnitzlers}|pwv}}{\lemma{\textnormal{\emph{zwei Exemplare}}}\Cendnote{\textnormal{Nur ein Exemplar
                     ist überliefert und wird im Folgenden wiedergegeben.}}}\label{K_L03774-3}.\pend
           \selectlanguage{ngerman}\vspace{1em}
\pstart
           \noindent{}{\pb}Wie ich durch Freunde\pwindex{Vengerova, Isabella 01.03.1877 – 07.02.1956@\textsc{Vengerova, Isabella} (01.03.1877 – 07.02.1956), \emph{Musikpädagoge/Musikpädagogin, Pianist/Pianistin}|pwv}\pwindex{Moller, Alice 24.04.1871 – Oktober 1962@\textsc{Moller, Alice} (24.04.1871 – Oktober 1962), \emph{Kassier/Kassierin}|pwv} in Russland\oindex{Russland@\textbf{Russland}, \emph{A.PCLI}|pw}
               auf einem Umweg erfahre, sind in Petersburg\oindex{Sankt Petersburg@\textbf{Sankt Petersburg}, \emph{P.PPLA}|pw}er
               Blättern angebliche Aeusserungen\pwindex{?? [Fiktives Interview aus der Kriegszeit]@\emph{?? [Fiktives Interview aus der Kriegszeit]}|pwv} von mir über Tolstoi\pwindex{Tolstoi, Leo N. von 09.09.1828 – 20.11.1910@\textsc{Tolstoi, Leo N. von} (09.09.1828 – 20.11.1910), \emph{Schriftsteller/Schriftstellerin, Schriftsteller/Schriftstellerin, Krimiautor/Krimiautorin}|pw}, Materlinck\pwindex{Maeterlinck, Maurice 29.08.1862 – 06.05.1949@\textsc{Maeterlinck, Maurice} (29.08.1862 – 06.05.1949), \emph{Schriftsteller/Schriftstellerin}|pw}, Anatole France\pwindex{France, Anatole 16.04.1844 – 12.10.1924@\textsc{France, Anatole} (16.04.1844 – 12.10.1924), \emph{Schriftsteller/Schriftstellerin}|pw}, Shakespeare\pwindex{Shakespeare, William 23.4.1564? – 03.05.1616@\textsc{Shakespeare, William} (23.4.1564? – 03.05.1616), \emph{Schauspieler/Schauspielerin, Dramatiker/Dramatikerin}|pw} von so
               phantastischer Unsinnigkeit veröffentlicht worden, dass sie mir zu normalen Zeiten
               von niemanden, der mich kennt, zugetraut würden, die aber in unserer vom Uebermass
               des Hasses und vom Wahnsinn der Lüge verwirrten Welt immerhin auch sonst
               urteilsfähigen Menschen nicht unglaubhaft erscheinen könnten.\pend
           
\pstart
           Solche Verhetzungsversuche, wie sie weit hinter den Fronten der ehrlich kämpfenden
               Armeen im wohlgedeckten Gelände unverantwortlicher Publizistik von den Marodeuren des
               Patriotismus gefahrlos unternommen werden, scheinen ja eine besondere, und vielleicht
               die widerwärtigste, Eigentümlichkeit dieses Krieges zu bedeuten; auch der
               lächerlichste dieser Ver{\pb}suche, wenn er gelingt, könnte
               späteren Verständigungen zwischen Einzelnen Schwierigkeiten bereiten; daher schiene
               es mir ein Fehler, gerade diesen (\substVorne{}\textsuperscript{eben}\substDazwischen{}etwa\substHinten{} um seiner besonderen Albernheit willen) auf sich beruhen zu lassen.\pend
           
\pstart
           Der Wortlaut der mir zugeschriebenen Aeusserung\introOben{}en\introOben{} ist mir
               noch nicht bekannt; ihr Sinn, und die Tatsache der Veröffentlichung aber steht
               unbezweifelbar fest. Da es unter den gegenwärtigen Verhältnissen lange dauern kann,
               ehe ich in den Besitz des Originalartikels\pwindex{?? [Fiktives Interview aus der Kriegszeit]@\emph{?? [Fiktives Interview aus der Kriegszeit]}|pwv} gelange, muss ich mich auf die Erklärung beschränken, dass
               Aeusserungen der Art, wie sie in jener Publikation offenbar mitgeteilt sind, von
               meiner Seite selbstverständlich niemals gefallen sind; – und – im Vertrauen auf eine
               auch während des Weltkrieges weiterdauernde Giltigkeit internationaler
               journalistischer Anstandsgesetze – erwarte ich von {\pb}der
               Loyalität der jenigen Zeitungen, die jenem erdichteten Bericht Raum gegönnt haben –
               auch von solchen, die (um in der Sprache der Politik zu reden) im Feindesland
               erscheinen – dass sie sich auch zur Aufnahme meiner Richtigstellung verpflichtet
               finden werden.\pend
           
\pstart
           {\pb}Nach Niederschrift dieser Zeilen finde ich in der \label{K_L03774-4v}\edtext{New-Yorkerstaats-Zeitung\pwindex{New Yorker Staats-Zeitung@\emph{New Yorker Staats-Zeitung}|pw} einen Privatbrief\pwindex{Artur Schnitzler ueber den Krieg. Brief an einen Schulfreund in New York@\emph{Artur Schnitzler über den Krieg. Brief an einen Schulfreund in New York}|pwv}}{\lemma{\textnormal{\emph{New-Yorkerstaats-Zeitung einen Privatbrief}}}\Cendnote{\textnormal{A. S.: \emph{»Das Zeitlose ist von kürzester Dauer«}, Artur Schnitzler über den Krieg. Brief an einen Schulfreund in New York, 17. 11. 1914. Dieser Abschnitt wurde nicht
                  veröffentlicht. Schnitzler hatte bereits 
                   eine Richtigstellung dazu publiziert, A. S.: \emph{»Das Zeitlose ist von kürzester Dauer«}, Ein Brief von Artur Schnitzler, 20. 11. 1914.}}}\label{K_L03774-4} abgedruckt, die den ich vor mehreren Wochen an einen in
                  New-York\oindex{New York City@\textbf{New York City}, \emph{P.PPL}|pw} lebenden Freund\pwindex{Deimel, Eugen Maerz 1860 – 10.03.1920@\textsc{Deimel, Eugen} (März 1860 – 10.03.1920), \emph{Journalist/Journalistin}|pwv} gerichtet habe oder vielmehr
               gerichtet haben soll. Denn in dem von der New-Yorker-Staats-Zeitung\pwindex{New Yorker Staats-Zeitung@\emph{New Yorker Staats-Zeitung}|pw} veröffentlichten Brief\pwindex{Artur Schnitzler ueber den Krieg. Brief an einen Schulfreund in New York@\emph{Artur Schnitzler über den Krieg. Brief an einen Schulfreund in New York}|pwv} ist (offenbar in bester redaktioneller Absicht Zur
               Erhöhung einer populären Wirkung auf das deutsch-amerikanische\oindex{Vereinigte Staaten von Amerika [USA]@\textbf{Vereinigte Staaten von Amerika [USA]}, \emph{A.PCLI}|pw} Publikum kaum mehr ein Satz gleichlautend mit dem
               entsprechenden Satz des Originals; manche Sätze sind völlig ausgefallen, andere
               hinzuerfunden, so dass zwischen den beiden Briefen, meinem eigenen und dem in der New-Yorker-Staatszeitung\pwindex{New Yorker Staats-Zeitung@\emph{New Yorker Staats-Zeitung}|pw} abgedruckten, an manchen
               Stellen, auch dem Sinne nach, nur mehr eine ganz enfernte Aehnlichkeit besteht.
               Diesen, an sich gewiss ziemlich gleichgültigen Fall, möchte ich immerhin zum Anlasse
               nehmen, um ganz im Allgemeinen und {\pb}nach allen Seiten hin
               vor raschgläubiger Hinnahme auch solcher Veröffentlichungen zu warnen, die
               durch irgend ein bestechendes äusseres Zeichen der Echtheit als welche wohl die mit
               Anrede, Gruss und Unterschrift versehene Form eines Privatbriefes gelten kann) den
               Charakter absoluter Worttreue vorzutäuschen suchen. Es ist in solcher Zeit nicht
               leicht zu entscheiden, wo man vertrauen und wo man misstrauen soll; nicht nur
               Urteilsfähigkeit, sondern auch Verantwortungsgefühl scheinen manchmal auch dort
               geschwunden, wo wir sie noch vor kurzem als etwas Unverlierbares betrachtet hätten; –
               also seien wir in Glauben und Zweifel, Ihr Freunde und Ihr Feinde gleich vorsichtig
               gegenüber Feind und Freund\substVorne{}\textsuperscript{.}\substDazwischen{}!\substHinten{}\pend
           \pstart \spacefill\mbox{Arthur Schnitzler}\pend{}
\pstart
           Im November 1914.\pend
           \selectlanguage{ngerman}\endnumbering\briefempfaengerindex{Zweig, Stefan@\textsc{Zweig, Stefan}!zzzSchnitzler, Arthur@\emph{von Arthur Schnitzler}!1914-11-271@{27. 11. 1914}|)be}\mylabel{L03774h}
\begin{anhang}
\end{anhang}\normalsize

\doendnotes{C}
\bigskip
\vfill

\clearpage

\footnotesize

\lohead{\textsc{register}}

% Definiere theindex-Environment komplett neu ohne reledmac
\makeatletter
\renewenvironment{theindex}{%
  \section*{\indexname}%
  \setlength{\parindent}{0pt}%
  \setlength{\parskip}{0pt plus 0.3pt}%
  \let\item\@idxitem
}{%
  \clearpage
}
\makeatother

\IfFileExists{\jobname-pw.ind}{\input{\jobname-pw.ind}}{}

\end{document}

      