%% latex-leseansicht-vorspann.tex
%% Vorspann für die Leseansicht.
%% Lädt die gemeinsame Datei latex-vorspann.tex mit nicht gesetztem Schalter.

\newif\ifkorrekturansicht
\korrekturansichtfalse

\input{../tex-inputs/latex-vorspann}


\section[Arthur Schnitzler an Stefan Zweig, 27. 11. 1914]{L03774 Arthur Schnitzler an Stefan Zweig, 27. 11. 1914}
\nopagebreak\mylabel{L03774v}
\rehead{ }\normalsize\beginnumbering\briefempfaengerindex{Zweig, Stefan@\textsc{Zweig, Stefan}!zzzSchnitzler, Arthur@\emph{von Arthur Schnitzler}!1914-11-271@{27. 11. 1914}|(be}
\toendnotes[C]{\smallbreak\pagebreak[2]}
\correspDesc{Versand  durch Arthur Schnitzler am 27. 11. 1914 in Wien
\newline{}Erhalt  durch Stefan Zweig im Zeitraum [27. 11. 1914 – 29. 11. 1914?] in Wien}\toendnotes[C]{\smallbreak}
\Standort{Jerusalem, National Library of Israel, ARC. Ms. Var. 305 1 58 Stefan Zweig Collection.}
\physDesc{Brief, 1 Blatt, 2 Seiten, 4592 Zeichen
\newline{}Schreibmaschine
\newline{}Handschrift: schwarze Tinte (\noindent{}Unterschrift)
\newline{}Beilage: ms., 5 Blatt, 5 Seiten, ms. paginiert. Korrekturen mit Bleistift
                                 in lateinischer Kurrent. Auf S. 1 von unbekannter Hand mit blauem
                                 Stift: »XI. 1914« }\toendnotes[C]{\smallbreak}
\pstart
           {\pb}\textcolor{gray}{\textbf{Dr. Arthur Schnitzler}}\hfill 27. 11. 1914.\pend
           
\pstart
           \textcolor{gray}{\textbf{Wien XVIII. Sternwartestrasse 71\oindex{Wien@\textbf{Wien}!XVIII., Währing@\textbf{XVIII., Währing}!Sternwartestraße 71@\textbf{Sternwartestraße 71}, \emph{Wohngebäude}|pw}}}\pend
           
\pstart\center{}Lieber Herr Doktor.\pend\vspace{0.5em}
\pstart
           Beifolgend die Berichtigung\pwindex{Schnitzler, Arthur 15.\,5.\,1862 Wien – 21.\,10.\,1931 ebd.@\textsc{Schnitzler, Arthur} (15.\,5.\,1862 Wien – 21.\,10.\,1931 ebd.), \emph{Schriftsteller, Mediziner}!Brief Artur Schnitzlers@\strich\emph{Ein Brief Artur Schnitzlers}|pwv}
               oder Erklärung\pwindex{Schnitzler, Arthur 15.\,5.\,1862 Wien – 21.\,10.\,1931 ebd.@\textsc{Schnitzler, Arthur} (15.\,5.\,1862 Wien – 21.\,10.\,1931 ebd.), \emph{Schriftsteller, Mediziner}!Brief Artur Schnitzlers@\strich\emph{Ein Brief Artur Schnitzlers}|pwv} oder wie Sie
               es nennen wollen. Ich wünschte gern zu wissen, 1., ob Sie im Ganzen damit
               einverstanden sind, 2. ob Sie eine \label{K_L03774-1v}\edtext{Veröffentlichung von Seite 4 an}{\lemma{\textnormal{\emph{Veröffentlichung … an}}}\Cendnote{\textnormal{Es
                  handelt sich um die Nachschrift, die mit »Nach Niederschrift dieser
                     Zeilen…« beginnt, siehe Schluss der Beilage. Wie hier antizipiert, wurde dieser Teil
                  nicht in die Veröffentlichung\pwindex{Schnitzler, Arthur 15.\,5.\,1862 Wien – 21.\,10.\,1931 ebd.@\textsc{Schnitzler, Arthur} (15.\,5.\,1862 Wien – 21.\,10.\,1931 ebd.), \emph{Schriftsteller, Mediziner}!Brief Artur Schnitzlers@\strich\emph{Ein Brief Artur Schnitzlers}|pwkv} im \emph{Journal de
                     Genève}\pwindex{Journal de Genève@\emph{Journal de Genève}|pwk} bzw. in der \emph{Neuen Zürcher
                     Zeitung}\pwindex{Neue Zürcher Zeitung@\emph{Neue Zürcher Zeitung}|pwk} aufgenommen: \emph{Une protestation d’Arthur Schnitzler}\pwindex{Schnitzler, Arthur 15.\,5.\,1862 Wien – 21.\,10.\,1931 ebd.@\textsc{Schnitzler, Arthur} (15.\,5.\,1862 Wien – 21.\,10.\,1931 ebd.), \emph{Schriftsteller, Mediziner}!Une protestation d’Arthur Schnitzler@\strich\emph{Une protestation d’Arthur Schnitzler}|pwk}. In:
                        \emph{Journal de Genève}\pwindex{Journal de Genève@\emph{Journal de Genève}|pwk}, Jg. 85, 21.\,12.\,1914, 3. Ausgabe, S. [1]. \emph{Ein Brief Artur Schnitzlers}\pwindex{Schnitzler, Arthur 15.\,5.\,1862 Wien – 21.\,10.\,1931 ebd.@\textsc{Schnitzler, Arthur} (15.\,5.\,1862 Wien – 21.\,10.\,1931 ebd.), \emph{Schriftsteller, Mediziner}!Brief Artur Schnitzlers@\strich\emph{Ein Brief Artur Schnitzlers}|pwk}. In: \emph{Neue Zürcher Zeitung}\pwindex{Neue Zürcher Zeitung@\emph{Neue Zürcher Zeitung}|pwk}, Jg. 135, Nr. 1700,
                        22. 12. 1914, 2. Mittagsblatt, S. 2).}}}\label{K_L03774-1} für
               notwendig und opportun hielten. Haben Sie nichts einzuwenden, so senden Sie
               freundlichst unserer Verabredung gemäss das Ganze mit meinen verehrungsvollen Grüssen
                  \label{K_L03774-2v}\edtext{an Rolland\pwindex{Rolland, Romain 29.\,1.\,1866 Clamecy – 30.\,12.\,1944 Vézelay@\textsc{Rolland, Romain} (29.\,1.\,1866 Clamecy – 30.\,12.\,1944 Vézelay), \emph{Schriftsteller}|pw}}{\lemma{\textnormal{\emph{an Rolland}}}\Cendnote{\textnormal{Die Sendung verzögerte sich noch um
                  einen Korrekturlauf auf den 5. 12. 1914, siehe XXXX Auszeichnungsfehler: Dokument L03779 nicht gefunden.}}}\label{K_L03774-2}. Was in dieser
               Angelegenheit anderswo und eventuell hier geschehen könnte oder sollte, möchte ich
               doch gerne persönlich oder wenigstens telefonisch mit Ihnen besprechen. Vielleicht
               schreiben Sie mir ein Wort, wann man Sie in den nächsten Tagen anrufen darf. Wie
               telefoniert man denn an den \label{K_L03774-3v}\edtext{Regierungsrat Winternitz\pwindex{Winternitz, Jakob von 3.\,3.\,1843 Horažďovice – 26.\,1.\,1921 Wien@\textsc{Winternitz, Jakob von} (3.\,3.\,1843 Horažďovice – 26.\,1.\,1921 Wien), \emph{Ministerialbeamter}|pw}}{\lemma{\textnormal{\emph{Regierungsrat Winternitz}}}\Cendnote{\textnormal{Dieser war Regierungsrat im
                  literarischen Bureau des \emph{Ministeriums des
                     Äußeren}\orgindex{Außenministerium@Außenministerium|pwk} – und der Schwiegervater von Friderike von Winternitz\pwindex{Zweig, Friderike Maria 4.\,12.\,1882 Wien – 18.\,1.\,1971 Stamford@\textsc{Zweig, Friderike Maria} (4.\,12.\,1882 Wien – 18.\,1.\,1971 Stamford), \emph{Schriftstellerin}|pwk}. Diese ließ just in diesem Jahr ihre Ehe mit Felix Adolf von Wintenitz\pwindex{Winternitz, Felix Adolf von 9.\,10.\,1877 Wien – 1950@\textsc{Winternitz, Felix Adolf von} (9.\,10.\,1877 Wien – 1950), \emph{Finanzbeamter}|pwkv}
                  annulieren, um ihre Beziehung mit Zweig\pwindex{Zweig, Stefan 28.\,11.\,1881 Wien – 23.\,2.\,1942 Petrópolis@\textsc{Zweig, Stefan} (28.\,11.\,1881 Wien – 23.\,2.\,1942 Petrópolis), \emph{Schriftsteller}|pwk} zu
                  legalisieren. Die Kontaktaufnahme auf diesem Weg hat dementsprechend eine pikante
                  Note.}}}\label{K_L03774-3}; ich habe mich bisher noch nicht an ihn gewandt.\pend
           
\pstart
           {\pb}Zu Ihrer militärischen Verwendung\orgindex{Kriegsarchiv@Kriegsarchiv|pwv} kann man Ihnen gratulieren, glaube ich. Sie
               werden Interessanteres und wahrscheinlich sogar Authentischeres erfahren als die
               Leute an der Front. Der Baron Winterstein\pwindex{Winterstein, Alfred von 25.\,9.\,1885 Wien – 28.\,4.\,1958 ebd.@\textsc{Winterstein, Alfred von} (25.\,9.\,1885 Wien – 28.\,4.\,1958 ebd.), \emph{Schriftsteller, Psychoanalytiker, Beamter}|pw} hat
               uns \label{K_L03774-4v}\edtext{neulich}{\lemma{\textnormal{\emph{neulich}}}\Cendnote{\textnormal{Vgl. A. S.: \emph{Tagebuch}, 25. 11. 1914.}}}\label{K_L03774-4}
               allerlei Anregendes erzählt; wir hätten Sie gern dabei gehabt.\pend
           
\pstart
           Herzlichst grüssend{\\[\baselineskip]}Ihr{\\[\baselineskip]}\spacefill\mbox{{[}hs.:{]} Arthur Schnitzler}\pend
           \leftskip=0em{}
\pstart
           \noindent{}{[}ms.:{]} Beiliegend \label{K_L03774-5v}\edtext{zwei Exemplare\pwindex{Schnitzler, Arthur 15.\,5.\,1862 Wien – 21.\,10.\,1931 ebd.@\textsc{Schnitzler, Arthur} (15.\,5.\,1862 Wien – 21.\,10.\,1931 ebd.), \emph{Schriftsteller, Mediziner}!Brief Artur Schnitzlers@\strich\emph{Ein Brief Artur Schnitzlers}|pwv}}{\lemma{\textnormal{\emph{zwei Exemplare}}}\Cendnote{\textnormal{Nur ein Exemplar ist überliefert und
                     wird im Folgenden wiedergegeben.}}}\label{K_L03774-5}.\pend
           \selectlanguage{ngerman}\vspace{1em}
\pstart
           \noindent{}{\pb}Wie ich durch Freunde\pwindex{Vengerova, Isabella 1.\,3.\,1877 Minsk – 7.\,2.\,1956 New York City@\textsc{Vengerova, Isabella} (1.\,3.\,1877 Minsk – 7.\,2.\,1956 New York City), \emph{Musikpädagogin, Pianistin}|pwv}\pwindex{Moller, Alice 24.\,4.\,1871 Wien – Oktober 1962@\textsc{Moller, Alice} (24.\,4.\,1871 Wien – Oktober 1962), \emph{Kassierin}|pwv} in Russland\oindex{Russland@\textbf{Russland}|pw}
               auf einem Umweg erfahre, sind in Petersburg\oindex{Sankt Petersburg@\textbf{Sankt Petersburg}|pw}er
               Blättern angebliche Aeusserungen\pwindex{\textcolor{red}{\textsuperscript{XXXX indx1}}!?? [Fiktives Interview aus St. Petersburg, 1914]@\strich\emph{?? [Fiktives Interview aus St. Petersburg, 1914]}|pwv} von mir über Tolstoi\pwindex{Tolstoi, Lew Nikolajewitsch 9.\,9.\,1828 Yasnaya Polyana – 20.\,11.\,1910 Lev Tolstoy@\textsc{Tolstoi, Lew Nikolajewitsch} (9.\,9.\,1828 Yasnaya Polyana – 20.\,11.\,1910 Lev Tolstoy), \emph{Schriftsteller}|pw}, Ma{[}e{]}terlinck\pwindex{Maeterlinck, Maurice 29.\,8.\,1862 Gent – 6.\,5.\,1949 Nizza@\textsc{Maeterlinck, Maurice} (29.\,8.\,1862 Gent – 6.\,5.\,1949 Nizza), \emph{Schriftsteller}|pw}, Anatole France\pwindex{France, Anatole 16.\,4.\,1844 Paris – 12.\,10.\,1924 Saint-Cyr-sur-Loire@\textsc{France, Anatole} (16.\,4.\,1844 Paris – 12.\,10.\,1924 Saint-Cyr-sur-Loire), \emph{Schriftsteller}|pw}, Shakespeare\pwindex{Shakespeare, William 23.\,4.\,1564? Stratford-upon-Avon – 3.\,5.\,1616 ebd.@\textsc{Shakespeare, William} (23.\,4.\,1564? Stratford-upon-Avon – 3.\,5.\,1616 ebd.), \emph{Schauspieler, Dramatiker}|pw} von so
               phantastischer Unsinnigkeit veröffentlicht worden, dass sie mir zu normalen Zeiten
               von niemanden, der mich kennt, zugetraut würden, die aber in unserer vom Uebermass
               des Hasses und vom Wahnsinn der Lüge verwirrten Welt immerhin auch sonst
               urteilsfähigen Menschen nicht unglaubhaft erscheinen könnten.\pend
           
\pstart
           Solche Verhetzungsversuche, wie sie weit hinter den Fronten der ehrlich kämpfenden
               Armeen im wohlgedeckten Gelände unverantwortlicher Publizistik von den Marodeuren des
               Patriotismus gefahrlos unternommen werden, scheinen ja eine besondere, und vielleicht
               die widerwärtigste, Eigentümlichkeit dieses Krieges zu bedeuten; auch der
               lächerlichste dieser Ver{\pb}suche, wenn er gelingt, könnte
               späteren Verständigungen zwischen Einzelnen Schwierigkeiten bereiten; daher schiene
               es mir ein Fehler, gerade diesen (\substVorne{}\textsuperscript{eben}\substDazwischen{}etwa\substHinten{} um seiner besonderen Albernheit willen) auf sich beruhen zu lassen.\pend
           
\pstart
           Der Wortlaut der mir zugeschriebenen Aeusserung\introOben{}en\introOben{} ist mir
               noch nicht bekannt; ihr Sinn, und die Tatsache der Veröffentlichung aber steht
               unbezweifelbar fest. Da es unter den gegenwärtigen Verhältnissen lange dauern kann,
               ehe ich in den Besitz des Originalartikels\pwindex{\textcolor{red}{\textsuperscript{XXXX indx1}}!?? [Fiktives Interview aus St. Petersburg, 1914]@\strich\emph{?? [Fiktives Interview aus St. Petersburg, 1914]}|pwv} gelange, muss ich mich auf die Erklärung beschränken, dass
               Aeusserungen der Art, wie sie in jener Publikation offenbar mitgeteilt sind, von
               meiner Seite selbstverständlich niemals gefallen sind; – und – im Vertrauen auf eine
               auch während des Weltkrieges weiterdauernde Giltigkeit internationaler
               journalistischer Anstandsgesetze – erwarte ich von {\pb}der
               Loyalität derjenigen Zeitungen, die jenem erdichteten Bericht\pwindex{\textcolor{red}{\textsuperscript{XXXX indx1}}!?? [Fiktives Interview aus St. Petersburg, 1914]@\strich\emph{?? [Fiktives Interview aus St. Petersburg, 1914]}|pwv} Raum gegönnt haben – auch von solchen, die (um in
               der Sprache der Politik zu reden) im Feindesland erscheinen – dass sie sich auch zur
               Aufnahme meiner Richtigstellung verpflichtet finden werden.\pend
           {\vspace{1\baselineskip}}
\pstart
           {\pb}Nach Niederschrift dieser Zeilen finde ich in der \label{K_L03774-6v}\edtext{New-Yorkerstaats-Zeitung\pwindex{New Yorker Staats-Zeitung@\emph{New Yorker Staats-Zeitung}|pw} einen Privatbrief\pwindex{Artur Schnitzler über den Krieg. Brief an einen Schulfreund in New York@\emph{Artur Schnitzler über den Krieg. Brief an einen Schulfreund in New York}|pwv}}{\lemma{\textnormal{\emph{New-Yorkerstaats-Zeitung einen Privatbrief}}}\Cendnote{\textnormal{A. S.: \emph{»Das Zeitlose ist von kürzester Dauer«}, Artur Schnitzler über den Krieg. Brief an einen Schulfreund in New York, 17. 11. 1914. Dieser Abschnitt wurde nicht
                  veröffentlicht. Schnitzler hatte bereits
                  eine Richtigstellung dazu publiziert, A. S.: \emph{»Das Zeitlose ist von kürzester Dauer«}, Ein Brief von Artur Schnitzler, 20. 11. 1914.}}}\label{K_L03774-6} abgedruckt, den ich vor mehreren Wochen an einen in New-York\oindex{New York City@\textbf{New York City}|pw} lebenden Freund\pwindex{Deimel, Eugen März 1860 – 10.\,3.\,1920 New York City@\textsc{Deimel, Eugen} (März 1860 – 10.\,3.\,1920 New York City), \emph{Journalist}|pwv} gerichtet habe oder vielmehr
               gerichtet haben soll. Denn in dem von der New-Yorker-Staats-Zeitung\pwindex{New Yorker Staats-Zeitung@\emph{New Yorker Staats-Zeitung}|pw} veröffentlichten Brief\pwindex{Artur Schnitzler über den Krieg. Brief an einen Schulfreund in New York@\emph{Artur Schnitzler über den Krieg. Brief an einen Schulfreund in New York}|pwv} ist (offenbar in bester redaktioneller Absicht zur
               Erhöhung einer populären Wirkung auf das deutsch-amerikanische\oindex{Vereinigte Staaten von Amerika [USA]@\textbf{Vereinigte Staaten von Amerika [USA]}|pw} Publikum) kaum mehr ein Satz gleichlautend mit dem
               entsprechenden Satz des Originals; manche Sätze sind völlig ausgefallen, andere
               hinzuerfunden, so dass zwischen den beiden Briefen\pwindex{Artur Schnitzler über den Krieg. Brief an einen Schulfreund in New York@\emph{Artur Schnitzler über den Krieg. Brief an einen Schulfreund in New York}|pwv}, meinem eigenen und dem in der New-Yorker-Staatszeitung\pwindex{New Yorker Staats-Zeitung@\emph{New Yorker Staats-Zeitung}|pw}\pwindex{Artur Schnitzler über den Krieg. Brief an einen Schulfreund in New York@\emph{Artur Schnitzler über den Krieg. Brief an einen Schulfreund in New York}|pwv} abgedruckten, an manchen
               Stellen, auch dem Sinne nach, nur mehr eine ganz enfernte Aehnlichkeit besteht.\pend
           
\pstart
           Diesen, an sich gewiss ziemlich gleichgültigen Fall, möchte ich immerhin zum Anlasse
               nehmen, um ganz im Allgemeinen und {\pb}nach allen Seiten hin
               vor raschgläubiger Hinnahme auch solcher Veröffentlichungen zu warnen, die
               durch irgend ein bestechendes äusseres Zeichen der Echtheit (als welche wohl die mit
               Anrede, Gruss und Unterschrift versehene Form eines Privatbriefes gelten kann) den
               Charakter absoluter Worttreue vorzutäuschen suchen. Es ist in solcher Zeit nicht
               leicht zu entscheiden, wo man vertrauen und wo man misstrauen soll; nicht nur
               Urteilsfähigkeit, sondern auch Verantwortungsgefühl scheinen manchmal auch dort
               geschwunden, wo wir sie noch vor kurzem als etwas Unverlierbares betrachtet hätten; –
               also seien wir in Glauben und Zweifel, Ihr Freunde und Ihr Feinde, gleich vorsichtig
               gegenüber Feind und Freund\substVorne{}\textsuperscript{.}\substDazwischen{}!\substHinten{}\pend
           \pstart {[}hs.:{]} \spacefill\mbox{Arthur Schnitzler}\pend{}
\pstart
           {[}ms.:{]} Im November 1914.\pend
           \selectlanguage{ngerman}\endnumbering\briefempfaengerindex{Zweig, Stefan@\textsc{Zweig, Stefan}!zzzSchnitzler, Arthur@\emph{von Arthur Schnitzler}!1914-11-271@{27. 11. 1914}|)be}\mylabel{L03774h}  \newcommand{\dateiname}{L03774}\newcommand{\titel}{Arthur Schnitzler an Stefan Zweig, 27. 11. 1914}\newcommand{\editorInnen}{Selma Jahnke und Martin Anton Müller}%% latex-leseansicht-abspann.tex
%% Abspann für die Leseansicht.
%% Der Schalter \ifkorrekturansicht ist bereits durch den Vorspann gesetzt.

%% latex-abspann.tex
%% Gemeinsamer Abspann für Korrekturansicht und Leseansicht.
%% Setzt den Schalter \ifkorrekturansicht voraus (gesetzt in den
%% einbindenden Dateien latex-korrekturansicht-abspann.tex bzw.
%% latex-leseansicht-abspann.tex).
%% ---------------------------------------------------------------

\normalsize

% Das esempio-Environment wird nur in der Leseansicht benötigt
\ifkorrekturansicht\else
\newenvironment{esempio}[3]%
{
    \vspace{1.5ex}
    \rlap{\underline{#1}}
    \par
    \setlength{\parindent}{0cm}
    \nopagebreak
    \leftskip=#2cm
    \rightskip=#3cm
}
{
    \par
}
\fi

\doendnotes{C}
\bigskip
\vfill

\clearpage

\footnotesize

\ifkorrekturansicht
  \lohead{\textsc{register}}
\fi

% theindex-Environment neu definieren ohne reledmac
\makeatletter
\renewenvironment{theindex}{%
  \ifkorrekturansicht
    \section*{\indexname}%
  \else
    \subsubsection*{Index der erwähnten Entitäten}%
  \fi
  \setlength{\parindent}{0pt}%
  \setlength{\parskip}{0pt plus 0.3pt}%
  \let\item\@idxitem
}{%
  \ifkorrekturansicht\clearpage\fi
}
\makeatother

\IfFileExists{\jobname-pw.ind}{\input{\jobname-pw.ind}}{}

% Quellenangabe nur in der Leseansicht
\ifkorrekturansicht\else
% Fallback-Definitionen, falls die .tex-Datei \titel etc. nicht gesetzt hat
\providecommand{\titel}{}
\providecommand{\editorInnen}{}
\providecommand{\dateiname}{\jobname}

\vspace{3cm}

\vfill

\footnotesize
\textsc{Quelle}: \titel. Herausgegeben von {\editorInnen}. In: \emph{Arthur Schnitzler: Briefwechsel mit Autorinnen und Autoren}.
 Digitale Edition, https://schnitzler-briefe.acdh.oeaw.ac.at/{\dateiname}.html (Stand \today)
\fi

\end{document}


