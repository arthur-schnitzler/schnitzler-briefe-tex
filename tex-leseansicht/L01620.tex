%% latex-leseansicht-vorspann.tex
%% Vorspann für die Leseansicht.
%% Lädt die gemeinsame Datei latex-vorspann.tex mit nicht gesetztem Schalter.

\newif\ifkorrekturansicht
\korrekturansichtfalse

\input{../tex-inputs/latex-vorspann}


\section[Felix Salten und Arthur Schnitzler an Hugo von Hofmannsthal, 3. 8. 1906]{L01620 Felix Salten und Arthur Schnitzler an Hugo von Hofmannsthal, 3. 8. 1906}
\nopagebreak\mylabel{L01620v}
\rehead{ }\normalsize\beginnumbering\briefempfaengerindex{Hofmannsthal, Hugo von@\textsc{Hofmannsthal, Hugo von}!zzzSalten, Felix@\emph{von Felix Salten}!1906-08-031@{3. 8. 1906}|(be}\briefempfaengerindex{Hofmannsthal, Hugo von@\textsc{Hofmannsthal, Hugo von}!zzzSchnitzler, Arthur@\emph{von Arthur Schnitzler}!1906-08-031@{3. 8. 1906}|(be}
\toendnotes[C]{\smallbreak\pagebreak[2]}
\correspDesc{Versand  durch Arthur Schnitzler, Felix Salten am 3. 8. 1906 in Helsingør
\newline{}Erhalt  durch Hugo von Hofmannsthal im Zeitraum [4. 8. 1906
                  – 8. 8. 1906?] in St. Gilgen}\toendnotes[C]{\smallbreak}
\Standort{FDH, Hs-25714,2.}
\physDesc{Bildpostkarte, 142 Zeichen
\newline{}Handschrift Felix Salten: schwarze Tinte, lateinische Kurrent
\newline{}Handschrift Arthur Schnitzler: schwarze Tinte, deutsche Kurrent
\newline{}Versand: Stempel: »\nobreak{}\oindex{Helsingør@\textbf{Helsingør}, \emph{Hauptstadt}|pwk}Helsingør, 3. 8. {[}1906{]}\nobreak{}«.  }\toendnotes[C]{\smallbreak}\pstart{}{\pb}Österreich\oindex{Österreich@\textbf{Österreich}|pw}\pend{}\pstart{}Herrn D\textsuperscript{r} Hugo v. Hofmannsthal\pend{}\pstart{}Lueg \textsuperscript{bei}/St Gilgen am
                     Wolfgangsee\oindex{Lueg@\textbf{Lueg}, \emph{Teil eines besiedelten Ortes}|pw}\pend{}{\bigskip}
\pstart
           \noindent{}\centering{}{\pb}\textcolor{gray}{\textbf{Krenborg Slot\oindex{Schloss Kronborg@\textbf{Schloss Kronborg}, \emph{Gebäude}|pw}.}}\pend
           \vspace{1em}
\pstart
           \noindent{}{\pb}Kommender Tage froh gedenkend herzlichst
                  \label{K_L01620-1v}\edtext{\spacefill\mbox{Salten}}{\lemma{\textnormal{\emph{Salten}}}\Cendnote{\textnormal{Saltens\pwindex{Salten, Felix 6.\,9.\,1869 Budapest – 8.\,10.\,1945 Zürich@\textsc{Salten, Felix} (6.\,9.\,1869 Budapest – 8.\,10.\,1945 Zürich), \emph{Schriftsteller, Journalist, Chefredakteur}|pwk} Aufenthalt 
                     in Dänemark\oindex{Dänemark@\textbf{Dänemark}|pwk} ist nur für den 2. 8. 1906 belegt. Die vorliegende Karte
                     an Hofmannsthal\pwindex{Hofmannsthal, Hugo von 1.\,2.\,1874 Wien – 15.\,7.\,1929 Rodaun@\textsc{Hofmannsthal, Hugo von} (1.\,2.\,1874 Wien – 15.\,7.\,1929 Rodaun), \emph{Schriftsteller}|pwk} und jene an Beer-Hofmann\pwindex{Beer-Hofmann, Richard 11.\,7.\,1866 Wien – 26.\,9.\,1945 New York City@\textsc{Beer-Hofmann, Richard} (11.\,7.\,1866 Wien – 26.\,9.\,1945 New York City), \emph{Schriftsteller}|pwk} (XXXX Auszeichnungsfehler: Dokument L01621 nicht gefunden)
                     klären den Sachverhalt: Am 3. 8. 1906 fand noch ein gemeinsamer Besuch 
                     von Schloss Kronborg\oindex{Schloss Kronborg@\textbf{Schloss Kronborg}, \emph{Gebäude}|pwk} statt, bevor sich Salten\pwindex{Salten, Felix 6.\,9.\,1869 Budapest – 8.\,10.\,1945 Zürich@\textsc{Salten, Felix} (6.\,9.\,1869 Budapest – 8.\,10.\,1945 Zürich), \emph{Schriftsteller, Journalist, Chefredakteur}|pwk} verabschiedete.}}}\label{K_L01620-1}\pend
           
\pstart
           {[}hs. Schnitzler:{]} Herzliche Grüße. Ihr \spacefill\mbox{Arthur}\pend
           \selectlanguage{ngerman}\endnumbering\briefempfaengerindex{Hofmannsthal, Hugo von@\textsc{Hofmannsthal, Hugo von}!zzzSalten, Felix@\emph{von Felix Salten}!1906-08-031@{3. 8. 1906}|)be}\briefempfaengerindex{Hofmannsthal, Hugo von@\textsc{Hofmannsthal, Hugo von}!zzzSchnitzler, Arthur@\emph{von Arthur Schnitzler}!1906-08-031@{3. 8. 1906}|)be}\mylabel{L01620h}  \newcommand{\dateiname}{L01620}\newcommand{\titel}{Felix Salten und Arthur Schnitzler an Hugo von Hofmannsthal, 3. 8. 1906}\newcommand{\editorInnen}{Martin Anton Müller und Gerd-Hermann Susen}%% latex-leseansicht-abspann.tex
%% Abspann für die Leseansicht.
%% Der Schalter \ifkorrekturansicht ist bereits durch den Vorspann gesetzt.

%% latex-abspann.tex
%% Gemeinsamer Abspann für Korrekturansicht und Leseansicht.
%% Setzt den Schalter \ifkorrekturansicht voraus (gesetzt in den
%% einbindenden Dateien latex-korrekturansicht-abspann.tex bzw.
%% latex-leseansicht-abspann.tex).
%% ---------------------------------------------------------------

\normalsize

% Das esempio-Environment wird nur in der Leseansicht benötigt
\ifkorrekturansicht\else
\newenvironment{esempio}[3]%
{
    \vspace{1.5ex}
    \rlap{\underline{#1}}
    \par
    \setlength{\parindent}{0cm}
    \nopagebreak
    \leftskip=#2cm
    \rightskip=#3cm
}
{
    \par
}
\fi

\doendnotes{C}
\bigskip
\vfill

\clearpage

\footnotesize

\ifkorrekturansicht
  \lohead{\textsc{register}}
\fi

% theindex-Environment neu definieren ohne reledmac
\makeatletter
\renewenvironment{theindex}{%
  \ifkorrekturansicht
    \section*{\indexname}%
  \else
    \subsubsection*{Index der erwähnten Entitäten}%
  \fi
  \setlength{\parindent}{0pt}%
  \setlength{\parskip}{0pt plus 0.3pt}%
  \let\item\@idxitem
}{%
  \ifkorrekturansicht\clearpage\fi
}
\makeatother

\IfFileExists{\jobname-pw.ind}{\input{\jobname-pw.ind}}{}

% Quellenangabe nur in der Leseansicht
\ifkorrekturansicht\else
% Fallback-Definitionen, falls die .tex-Datei \titel etc. nicht gesetzt hat
\providecommand{\titel}{}
\providecommand{\editorInnen}{}
\providecommand{\dateiname}{\jobname}

\vspace{3cm}

\vfill

\footnotesize
\textsc{Quelle}: \titel. Herausgegeben von {\editorInnen}. In: \emph{Arthur Schnitzler: Briefwechsel mit Autorinnen und Autoren}.
 Digitale Edition, https://schnitzler-briefe.acdh.oeaw.ac.at/{\dateiname}.html (Stand \today)
\fi

\end{document}


