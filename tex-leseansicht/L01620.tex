%% latex-korrekturansicht-vorspann.tex
%% Vorspann für die Korrekturansicht.
%% Lädt die gemeinsame Datei latex-vorspann.tex mit gesetztem Schalter.

\newif\ifkorrekturansicht
\korrekturansichttrue

\input{../tex-inputs/latex-vorspann}


\section[Felix Salten und Arthur Schnitzler an Hugo von Hofmannsthal, 3. 8. 1906]{L01620 Felix Salten und Arthur Schnitzler an Hugo von Hofmannsthal,
               3. 8. 1906}
\nopagebreak\mylabel{L01620v}
\rehead{ }\normalsize\beginnumbering\briefempfaengerindex{Hofmannsthal, Hugo von@\textsc{Hofmannsthal, Hugo von}!zzzSalten, Felix@\emph{von Felix Salten}!1906-08-031@{3. 8. 1906}|(be}\briefempfaengerindex{Hofmannsthal, Hugo von@\textsc{Hofmannsthal, Hugo von}!zzzSchnitzler, Arthur@\emph{von Arthur Schnitzler}!1906-08-031@{3. 8. 1906}|(be}
\toendnotes[C]{\smallbreak\pagebreak[2]}\Standort{FDH, Hs-25714,2.}
\physDesc{Bildpostkarte, 142 Zeichen
\newline{}Handschrift Felix Salten: schwarze Tinte, lateinische Kurrent
\newline{}Handschrift Arthur Schnitzler: schwarze Tinte, deutsche Kurrent
\newline{}Versand: Stempel: »\nobreak{}\oindex{Helsingør@\textbf{Helsingør}, \emph{P.PPLA2}|pwk}Helsingør, 3. 8. {[}1906{]}\nobreak{}«.  }\toendnotes[C]{\smallbreak}\pstart{}{\pb}Österreich\oindex{Oesterreich@\textbf{Österreich}, \emph{A.PCLI}|pw}\pend{}\pstart{}Herrn D\textsuperscript{r} Hugo v. Hofmannsthal\pend{}\pstart{}Lueg \textsuperscript{bei}/St Gilgen am
                     Wolfgangsee\oindex{Lueg@\textbf{Lueg}, \emph{Teil eines besiedelten Ortes (A.BSOX)}|pw}\pend{}{\bigskip}
\pstart
           \noindent{}\centering{}{\pb}\textcolor{gray}{\textbf{Krenborg Slot\oindex{Schloss Kronborg@\textbf{Schloss Kronborg}, \emph{Gebäude (K.GBD)}|pw}.}}\pend
           \vspace{1em}
\pstart
           \noindent{}{\pb}Kommender Tage froh gedenkend herzlichst
                  \label{K_L01620-1v}\edtext{\spacefill\mbox{Salten}}{\lemma{\textnormal{\emph{Salten}}}\Cendnote{\textnormal{Saltens\pwindex{Salten, Felix 06.09.1869 – 08.10.1945@\textsc{Salten, Felix} (06.09.1869 – 08.10.1945), \emph{Schriftsteller/Schriftstellerin, Journalist/Journalistin, Chefredakteur/Chefredakteurin}|pwk} Aufenthalt 
                     in Dänemark\oindex{Daenemark@\textbf{Dänemark}, \emph{A.PCLI}|pwk} ist nur für den 2. 8. 1906 belegt. Die vorliegende Karte
                     an Hofmannsthal\pwindex{Hofmannsthal, Hugo von 1874-02-01 – 1929-07-15@\textsc{Hofmannsthal, Hugo von} (1874-02-01 – 1929-07-15), \emph{Schriftsteller/Schriftstellerin}|pwk} und jene an Beer-Hofmann\pwindex{Beer-Hofmann, Richard 1866-07-11 – 1945-09-26@\textsc{Beer-Hofmann, Richard} (1866-07-11 – 1945-09-26), \emph{Schriftsteller/Schriftstellerin}|pwk} (Felix Salten und Arthur Schnitzler an Richard Beer-Hofmann,
               3. 8. 190[6])
                     klären den Sachverhalt: Am 3. 8. 1906 fand noch ein gemeinsamer Besuch 
                     von Schloss Kronborg\oindex{Schloss Kronborg@\textbf{Schloss Kronborg}, \emph{Gebäude (K.GBD)}|pwk} statt, bevor sich Salten\pwindex{Salten, Felix 06.09.1869 – 08.10.1945@\textsc{Salten, Felix} (06.09.1869 – 08.10.1945), \emph{Schriftsteller/Schriftstellerin, Journalist/Journalistin, Chefredakteur/Chefredakteurin}|pwk} verabschiedete.}}}\label{K_L01620-1}\pend
           
\pstart
           {[}hs. :{]} Herzliche Grüße. Ihr \spacefill\mbox{Arthur}\pend
           \selectlanguage{ngerman}\endnumbering\briefempfaengerindex{Hofmannsthal, Hugo von@\textsc{Hofmannsthal, Hugo von}!zzzSalten, Felix@\emph{von Felix Salten}!1906-08-031@{3. 8. 1906}|)be}\briefempfaengerindex{Hofmannsthal, Hugo von@\textsc{Hofmannsthal, Hugo von}!zzzSchnitzler, Arthur@\emph{von Arthur Schnitzler}!1906-08-031@{3. 8. 1906}|)be}\mylabel{L01620h}  \normalsize

\doendnotes{C}
\bigskip
\vfill

\clearpage

\footnotesize

\lohead{\textsc{register}}

% Definiere theindex-Environment komplett neu ohne reledmac
\makeatletter
\renewenvironment{theindex}{%
  \section*{\indexname}%
  \setlength{\parindent}{0pt}%
  \setlength{\parskip}{0pt plus 0.3pt}%
  \let\item\@idxitem
}{%
  \clearpage
}
\makeatother

\IfFileExists{\jobname-pw.ind}{\input{\jobname-pw.ind}}{}

\end{document}

      