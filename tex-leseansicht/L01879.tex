%% latex-leseansicht-vorspann.tex
%% Vorspann für die Leseansicht.
%% Lädt die gemeinsame Datei latex-vorspann.tex mit nicht gesetztem Schalter.

\newif\ifkorrekturansicht
\korrekturansichtfalse

\input{../tex-inputs/latex-vorspann}


\section[Max Burckhard an Arthur Schnitzler, 14. 10. 1909]{L01879 Max Burckhard an Arthur Schnitzler, 14. 10. 1909}
\nopagebreak\mylabel{L01879v}
\rehead{ }\normalsize\beginnumbering\briefempfaengerindex{Schnitzler, Arthur@\textsc{Schnitzler, Arthur}!zzzBurckhard, Max Eugen@\emph{von Max Eugen Burckhard}!1909-10-141@{14. 10. 1909}|(be}
\toendnotes[C]{\smallbreak\pagebreak[2]}
\correspDesc{Versand  durch Max Burckhard am 14. 10. 1909 in Wien
\newline{}Erhalt  durch Arthur Schnitzler am 14. 10. 1909 in Wien}\toendnotes[C]{\smallbreak}
\Standort{CUL, Schnitzler, B 20.}
\physDesc{Telegramm, 205 Zeichen
\newline{}HandschriftX2 einer Schreibkraft: Bleistift, deutsche Kurrent
\newline{}Versand: »\noindent{}Wien\oindex{Wien@\textbf{Wien}, \emph{Verwaltungsgebiet}|pw} tel \textcolor{gray}{\textbf{Nr.}} 729 \textcolor{gray}{\textbf{Taxw.}} 40 \textcolor{gray}{\textbf{Ch.{\dots}) aufgegeben am}}{ }14\textcolor{gray}{\textbf{/}}10 \textcolor{gray}{\textbf{19}}{[}0{]}9{ }\textcolor{gray}{\textbf{um}}{ }11 \textcolor{gray}{\textbf{Uhr}} 40 \textcolor{gray}{\textbf{M.}}\textcolor{gray}{V}\textcolor{gray}{\textbf{Mittag}}\textcolor{gray}{\textbf{.}}« 
\newline{}Schnitzler: mit Bleistift datiert: »14/X 09« }\toendnotes[C]{\smallbreak}
\pstart
           \noindent{}{\pb}Ich hatte{ }ſie{ }ſchon{ }ſelbſt bitten wollen
               bin nur in der Hetzjagd noch nicht in die Spöttelgaſſe\oindex{Wien@\textbf{Wien}!XVIII., Währing@\textbf{XVIII., Währing}!Edmund-Weiß-Gasse 7@\textbf{Edmund-Weiß-Gasse 7}, \emph{Wohngebäude}|pw} gekommen. ich teile Ihnen die \label{K_L01879-1v}\edtext{directoriale Zuſtimmung}{\lemma{\textnormal{\emph{directoriale Zustimmung}}}\Cendnote{\textnormal{zur Teilnahme an der Generalprobe von \emph{Jene Asra}\pwindex{Burckhard, Max Eugen 14.\,7.\,1854 Korneuburg – 16.\,3.\,1912 Wien@\textsc{Burckhard, Max Eugen} (14.\,7.\,1854 Korneuburg – 16.\,3.\,1912 Wien), \emph{Schriftsteller, Rechtswissenschaftler, Theaterleiter}!Jene Asra,… Komödie in 4 Akten@\strich\emph{Jene Asra,… Komödie in 4 Akten}|pwk} am 15. 10. 1909}}}\label{K_L01879-1} mit und meine herzlichſte Freude mit Gruß \spacefill\mbox{Doktor Burckhard +}\pend
           \selectlanguage{ngerman}\endnumbering\briefempfaengerindex{Schnitzler, Arthur@\textsc{Schnitzler, Arthur}!zzzBurckhard, Max Eugen@\emph{von Max Eugen Burckhard}!1909-10-141@{14. 10. 1909}|)be}\mylabel{L01879h}  \newcommand{\dateiname}{L01879}\newcommand{\titel}{Max Burckhard an Arthur Schnitzler, 14. 10. 1909}\newcommand{\editorInnen}{Martin Anton Müller und Gerd-Hermann Susen}%% latex-leseansicht-abspann.tex
%% Abspann für die Leseansicht.
%% Der Schalter \ifkorrekturansicht ist bereits durch den Vorspann gesetzt.

%% latex-abspann.tex
%% Gemeinsamer Abspann für Korrekturansicht und Leseansicht.
%% Setzt den Schalter \ifkorrekturansicht voraus (gesetzt in den
%% einbindenden Dateien latex-korrekturansicht-abspann.tex bzw.
%% latex-leseansicht-abspann.tex).
%% ---------------------------------------------------------------

\normalsize

% Das esempio-Environment wird nur in der Leseansicht benötigt
\ifkorrekturansicht\else
\newenvironment{esempio}[3]%
{
    \vspace{1.5ex}
    \rlap{\underline{#1}}
    \par
    \setlength{\parindent}{0cm}
    \nopagebreak
    \leftskip=#2cm
    \rightskip=#3cm
}
{
    \par
}
\fi

\doendnotes{C}
\bigskip
\vfill

\clearpage

\footnotesize

\ifkorrekturansicht
  \lohead{\textsc{register}}
\fi

% theindex-Environment neu definieren ohne reledmac
\makeatletter
\renewenvironment{theindex}{%
  \ifkorrekturansicht
    \section*{\indexname}%
  \else
    \subsubsection*{Index der erwähnten Entitäten}%
  \fi
  \setlength{\parindent}{0pt}%
  \setlength{\parskip}{0pt plus 0.3pt}%
  \let\item\@idxitem
}{%
  \ifkorrekturansicht\clearpage\fi
}
\makeatother

\IfFileExists{\jobname-pw.ind}{\input{\jobname-pw.ind}}{}

% Quellenangabe nur in der Leseansicht
\ifkorrekturansicht\else
% Fallback-Definitionen, falls die .tex-Datei \titel etc. nicht gesetzt hat
\providecommand{\titel}{}
\providecommand{\editorInnen}{}
\providecommand{\dateiname}{\jobname}

\vspace{3cm}

\vfill

\footnotesize
\textsc{Quelle}: \titel. Herausgegeben von {\editorInnen}. In: \emph{Arthur Schnitzler: Briefwechsel mit Autorinnen und Autoren}.
 Digitale Edition, https://schnitzler-briefe.acdh.oeaw.ac.at/{\dateiname}.html (Stand \today)
\fi

\end{document}


