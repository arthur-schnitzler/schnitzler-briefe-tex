%% latex-korrekturansicht-vorspann.tex
%% Vorspann für die Korrekturansicht.
%% Lädt die gemeinsame Datei latex-vorspann.tex mit gesetztem Schalter.

\newif\ifkorrekturansicht
\korrekturansichttrue

\input{../tex-inputs/latex-vorspann}


\section[Joseph Victor Widmann an Arthur Schnitzler, 26. 2. 1894]{L00301 Joseph Victor Widmann an Arthur Schnitzler, 26. 2. 1894}
\nopagebreak\mylabel{L00301v}
\rehead{ }\normalsize\beginnumbering\briefempfaengerindex{Schnitzler, Arthur@\textsc{Schnitzler, Arthur}!zzzWidmann, Joseph Victor@\emph{von Joseph Victor Widmann}!1894-02-261@{26. 2. 1894}|(be}
\toendnotes[C]{\smallbreak\pagebreak[2]}\Standort{CUL, Schnitzler, B 113.}
\physDesc{Postkarte, 480 Zeichen
\newline{}Handschrift: schwarze Tinte, deutsche Kurrent
\newline{}Versand: 1) Stempel: »\nobreak{}\oindex{Bern@\textbf{Bern}, \emph{P.PPLC}|pwk}Bern Brf. Exp., 26. II. 94., 1\nobreak{}«.   2) Stempel: »\nobreak{}\oindex{IX., Alsergrund@\textbf{IX., Alsergrund}, \emph{A.ADM3}|pwk}Wien 9/{[}3{]}, 28. 2. 94, 8.V, Bestellt\nobreak{}«. }\toendnotes[C]{\smallbreak}\pstart{}{\pb}\textsc{Herrn D\textsuperscript{r} Arthur Schnitzler}\pend{}\pstart{}Schriftsteller in\pend{}\pstart{}\textsc{Wien} IX\oindex{I., Innere Stadt@\textbf{I., Innere Stadt}, \emph{A.ADM3}|pw}\pend{}\pstart{}\textsc{Frankenstr 1/?}\oindex{Frankgasse 1@\textbf{Frankgasse 1}, \emph{Wohngebäude (K.WHS)}|pw}\pend{}{\bigskip}\vspace{1em}
\pstart
           \raggedleft{}{\pb}Bern\oindex{Bern@\textbf{Bern}, \emph{P.PPLC}|pw}, d.
                  26. Febr. 1894.\pend
           
\pstart{}Sehr geehrter Herr!\pend\vspace{0.5em}
\pstart
           Selbſtverſtändlicher Weiſe habe ich gar nichts dagegen, we{\geminationn} Sie zu meiner Kritik\pwindex{Kunst und Litteratur@\emph{Kunst und Litteratur}|pwv} über den prächtigen Anatol\pwindex{Anatol@\emph{Anatol}|pw} meinen vollen \label{K_L00301-1v}\edtext{Namen
                  ſetzen}{\lemma{\textnormal{\emph{Namen
                  ſetzen}}}\Cendnote{\textnormal{Am Ende der Buchausgabe von \emph{Das Märchen}\pwindex{Maerchen. Schauspiel in drei Aufzuegen@\emph{Das Märchen. Schauspiel in drei Aufzügen}|pwk} (Schauspiel in drei
                     Aufzügen. Dresden, Leipzig: \emph{E. Pierson’s
                        Verlag}\orgindex{E. Pierson s Verlag@E. Pierson’s Verlag|pwk}{ }1894) wurden, als Verlagswerbung, Auszüge aus Kritiken von \emph{Anatol}\pwindex{Anatol@\emph{Anatol}|pwk} gesetzt. Mit seinem nicht erhaltenen Brief dürfte
                     Schnitzler um die Erlaubnis für Widmanns\pwindex{Widmann, Joseph Victor 20.02.1842 – 06.11.1911@\textsc{Widmann, Joseph Victor} (20.02.1842 – 06.11.1911), \emph{Schriftsteller/Schriftstellerin, Journalist/Journalistin}|pwk}{ }Besprechung\pwindex{Kunst und Litteratur@\emph{Kunst und Litteratur}|pwkv} angesucht haben.}}}\label{K_L00301-1}; im Gegentheil, ich
                  beke{\geminationn}e mich ſehr gern dazu.\pend
           
\pstart
           Hoffentlich beko{\geminationm}en Sie dieſe Zeilen, obwohl in Ihrem
               Briefchen juſt Ihre Wohnungsangabe verwiſcht war u. ich ſie daher nur andeutungsweiſe
               auf dieſe Karte ſetzen ko{\geminationn}te.\pend
           
\pstart
           Mit freundl. Gruß{\\[\baselineskip]}\spacefill\mbox{J. V. Widmann}\pend
           \leftskip=0em{}\selectlanguage{ngerman}\endnumbering\briefempfaengerindex{Schnitzler, Arthur@\textsc{Schnitzler, Arthur}!zzzWidmann, Joseph Victor@\emph{von Joseph Victor Widmann}!1894-02-261@{26. 2. 1894}|)be}\mylabel{L00301h}  \normalsize

\doendnotes{C}
\bigskip
\vfill

\clearpage

\footnotesize

\lohead{\textsc{register}}

% Definiere theindex-Environment komplett neu ohne reledmac
\makeatletter
\renewenvironment{theindex}{%
  \section*{\indexname}%
  \setlength{\parindent}{0pt}%
  \setlength{\parskip}{0pt plus 0.3pt}%
  \let\item\@idxitem
}{%
  \clearpage
}
\makeatother

\IfFileExists{\jobname-pw.ind}{\input{\jobname-pw.ind}}{}

\end{document}

      