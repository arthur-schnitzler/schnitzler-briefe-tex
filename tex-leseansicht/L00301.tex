%% latex-leseansicht-vorspann.tex
%% Vorspann für die Leseansicht.
%% Lädt die gemeinsame Datei latex-vorspann.tex mit nicht gesetztem Schalter.

\newif\ifkorrekturansicht
\korrekturansichtfalse

\input{../tex-inputs/latex-vorspann}


         \newcommand{\erwaehnteInstitutionen}{Institutionen: E. Pierson’s Verlag}
         \newcommand{\erwaehnteOrte}{Orte: Bern, Frankgasse, I., Innere Stadt, IX., Alsergrund, Wien}
         \newcommand{\erwaehnteWerke}{Werke: Anatol, Das Märchen. Schauspiel in drei Aufzügen, Kunst und Litteratur}
               \section[Joseph Victor Widmann an Arthur Schnitzler, 26. 2. 1894]{ Joseph Victor Widmann an Arthur Schnitzler,
                    26. 2. 1894}\nopagebreak\mylabel{v}\rehead{ }\begin{ledgroupsized}[t]{13cm}\normalsize\beginnumbering \toendnotes[C]{\smallbreak\pagebreak[2]} \Standort{CUL, Schnitzler, B 113.}
\physDesc{Postkarte
\newline{}Handschrift: schwarze Tinte, deutsche Kurrent\newline{}Versand: 1) Stempel: »\nobreak{}\oindex{Bern@\textbf{Bern}|pwk}Bern Brf. Exp., 26. II. 94., 1\nobreak{}«.   2) Stempel: »\nobreak{}\oindex{IX., Alsergrund@\textbf{IX., Alsergrund}|pwk}Wien 9/{[}3{]}, 28. 2. 94, 8.V, Bestellt\nobreak{}«. }\toendnotes[C]{\smallbreak}\pstart{}{\pb}\textsc{Herrn
                                D\textsuperscript{r} Arthur Schnitzler}\pend{}\pstart{}Schriftsteller in\pend{}\pstart{}\textsc{Wien} IX\oindex{I., Innere Stadt@\textbf{I., Innere Stadt}|pw}\pend{}\pstart{}\textsc{Frankenstr 1/?}\oindex{Frankgasse@\textbf{Frankgasse}|pw}\pend{}{\bigskip}\pstart
           \raggedleft{}{\pb}Bern\oindex{Bern@\textbf{Bern}|pw}, d.
                            26. Febr. 1894.\pend
           \pstart{}Sehr geehrter Herr!\pend\pstart
           Selbſtverſtändlicher Weiſe habe ich gar nichts dagegen, we{\geminationn}
               Sie zu meiner Kritik\pwindex{Kunst und Litteratur12.02.1893 – 12.2.1893@\emph{Kunst und Litteratur} {[}12.02.1893 – 12.2.1893{]}|pwv} über den prächtigen Anatol\pwindex{Schnitzler, Arthur 15.05.1862 – 21.10.1931@\textsc{Schnitzler, Arthur} (15.05.1862 – 21.10.1931), \emph{Schriftsteller, Mediziner}!Anatol1892-10-29@\strich\emph{Anatol} {[}1892-10-29{]}|pw} meinen vollen \label{K_L00301_1v}\edtext{Namen ſetzen}{\lemma{\textnormal{\emph{Namen ſetzen}}}\Cendnote{\textnormal{Am Ende der Buchausgabe von \emph{Das Märchen}\pwindex{Schnitzler, Arthur 15.05.1862 – 21.10.1931@\textsc{Schnitzler, Arthur} (15.05.1862 – 21.10.1931), \emph{Schriftsteller, Mediziner}!Maerchen. Schauspiel in drei Aufzuegen1893-12-01@\strich\emph{Das Märchen. Schauspiel in drei Aufzügen} {[}1893-12-01{]}|pwk} (Schauspiel in drei Aufzügen.
                            Dresden, Leipzig: \emph{E. Pierson’s
                                Verlag}\orgindex{E. Pierson s Verlag@E. Pierson’s Verlag|pwk}{ }1894) wurden, als Verlagswerbung, Auszüge aus
                        Kritiken von \emph{Anatol}\pwindex{Schnitzler, Arthur 15.05.1862 – 21.10.1931@\textsc{Schnitzler, Arthur} (15.05.1862 – 21.10.1931), \emph{Schriftsteller, Mediziner}!Anatol1892-10-29@\strich\emph{Anatol} {[}1892-10-29{]}|pwk} gesetzt. Mit seinem
                        nicht erhaltenen Brief dürfte Schnitzler\pwindex{Schnitzler, Arthur 15.05.1862 – 21.10.1931@\textsc{Schnitzler, Arthur} (15.05.1862 – 21.10.1931), \emph{Schriftsteller, Mediziner}|pwk}
                        um die Erlaubnis für Widmann\pwindex{Widmann, Joseph Victor 20.02.1842 – 06.11.1911@\textsc{Widmann, Joseph Victor} (20.02.1842 – 06.11.1911), \emph{Schriftsteller, Journalist}|pwk}s Besprechung\pwindex{Kunst und Litteratur12.02.1893 – 12.2.1893@\emph{Kunst und Litteratur} {[}12.02.1893 – 12.2.1893{]}|pwkv} angesucht
                        haben.}}}\label{K_L00301_1h}; im Gegentheil, ich beke{\geminationn}e mich
                    ſehr gern dazu.\pend
           \pstart
           Hoffentlich beko{\geminationm}en Sie dieſe Zeilen, obwohl in
                    Ihrem Briefchen juſt Ihre Wohnungsangabe verwiſcht war u. ich ſie daher nur
                    andeutungsweiſe auf dieſe Karte ſetzen ko{\geminationn}te.\pend
           \pstart
           Mit freundl. Gruß{\\[\baselineskip]}\spacefill\mbox{J. V. Widmann}\pend
           \leftskip=0em{}
         
         \endnumbering\mylabel{h}\end{ledgroupsized}  \newcommand{\dateiname}{L00301}\newcommand{\titel}{Joseph Victor Widmann an Arthur Schnitzler, 26. 2. 1894}\newcommand{\editorInnen}{Martin Anton Müller und Gerd-Hermann Susen}%% latex-leseansicht-abspann.tex
%% Abspann für die Leseansicht.
%% Der Schalter \ifkorrekturansicht ist bereits durch den Vorspann gesetzt.

%% latex-abspann.tex
%% Gemeinsamer Abspann für Korrekturansicht und Leseansicht.
%% Setzt den Schalter \ifkorrekturansicht voraus (gesetzt in den
%% einbindenden Dateien latex-korrekturansicht-abspann.tex bzw.
%% latex-leseansicht-abspann.tex).
%% ---------------------------------------------------------------

\normalsize

% Das esempio-Environment wird nur in der Leseansicht benötigt
\ifkorrekturansicht\else
\newenvironment{esempio}[3]%
{
    \vspace{1.5ex}
    \rlap{\underline{#1}}
    \par
    \setlength{\parindent}{0cm}
    \nopagebreak
    \leftskip=#2cm
    \rightskip=#3cm
}
{
    \par
}
\fi

\doendnotes{C}
\bigskip
\vfill

\clearpage

\footnotesize

\ifkorrekturansicht
  \lohead{\textsc{register}}
\fi

% theindex-Environment neu definieren ohne reledmac
\makeatletter
\renewenvironment{theindex}{%
  \ifkorrekturansicht
    \section*{\indexname}%
  \else
    \subsubsection*{Index der erwähnten Entitäten}%
  \fi
  \setlength{\parindent}{0pt}%
  \setlength{\parskip}{0pt plus 0.3pt}%
  \let\item\@idxitem
}{%
  \ifkorrekturansicht\clearpage\fi
}
\makeatother

\IfFileExists{\jobname-pw.ind}{\input{\jobname-pw.ind}}{}

% Quellenangabe nur in der Leseansicht
\ifkorrekturansicht\else
% Fallback-Definitionen, falls die .tex-Datei \titel etc. nicht gesetzt hat
\providecommand{\titel}{}
\providecommand{\editorInnen}{}
\providecommand{\dateiname}{\jobname}

\vspace{3cm}

\vfill

\footnotesize
\textsc{Quelle}: \titel. Herausgegeben von {\editorInnen}. In: \emph{Arthur Schnitzler: Briefwechsel mit Autorinnen und Autoren}.
 Digitale Edition, https://schnitzler-briefe.acdh.oeaw.ac.at/{\dateiname}.html (Stand \today)
\fi

\end{document}


      