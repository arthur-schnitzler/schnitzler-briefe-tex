%% latex-korrekturansicht-vorspann.tex
%% Vorspann für die Korrekturansicht.
%% Lädt die gemeinsame Datei latex-vorspann.tex mit gesetztem Schalter.

\newif\ifkorrekturansicht
\korrekturansichttrue

\input{../tex-inputs/latex-vorspann}


\section[Arthur Schnitzler an Hugo von Hofmannsthal, 15. 7. 1897]{L00702 Arthur Schnitzler an Hugo von Hofmannsthal, 15. 7. 1897}
\nopagebreak\mylabel{L00702v}
\rehead{ }\normalsize\beginnumbering\briefempfaengerindex{Hofmannsthal, Hugo von@\textsc{Hofmannsthal, Hugo von}!zzzSchnitzler, Arthur@\emph{von Arthur Schnitzler}!1897-07-151@{15. 7. 1897}|(be}
\toendnotes[C]{\smallbreak\pagebreak[2]}\Standort{FDH, Hs-30885,61.}
\physDesc{Brief, 1 Blatt, 4 Seiten, 917 Zeichen
\newline{}Handschrift: Bleistift, deutsche Kurrent
\newline{}Ordnung: mit Bleistift von Schnitzler mutmaßlich bei der Durchsicht der
                                 Korrespondenz 1929 das erste Blatt datiert: »15/7 97« }
\buchAbdrucke{\weitereDrucke{Hugo von Hofmannsthal, Arthur Schnitzler: \emph{Briefwechsel}. Frankfurt am Main: \emph{S. Fischer} 1964, S. 91–92.} }\toendnotes[C]{\smallbreak}
\pstart
           \noindent{}{\pb}Mein lieber Hugo, ich ka{\geminationn} keineswegs
                  Anfang Auguſt mit Ihnen zusa{\geminationm}entreffen –
                  \label{K_L00702-1v}\edtext{Sie wiſſen ja}{\lemma{\textnormal{\emph{Sie wiſſen ja}}}\Cendnote{\textnormal{Seine Partnerin Marie Reinhard\pwindex{Reinhard, Marie 1871-03-13 – 1899-03-18@\textsc{Reinhard, Marie} (1871-03-13 – 1899-03-18), \emph{Gesangspädagoge/Gesangspädagogin}|pwk} war schwanger. Das Kind\pwindex{?? [Totgeborener Sohn von Arthur Schnitzler und Marie Reinhard] 1897-09-24 – 1897-09-24@\textsc{?? [Totgeborener Sohn von Arthur Schnitzler und Marie Reinhard]} (1897-09-24 – 1897-09-24)|pwkv} kam tot zur Welt.}}}\label{K_L00702-1}. Dagegen
               unterbreiten Richard\pwindex{Beer-Hofmann, Richard 1866-07-11 – 1945-09-26@\textsc{Beer-Hofmann, Richard} (1866-07-11 – 1945-09-26), \emph{Schriftsteller/Schriftstellerin}|pw} u ich Ihnen einen andern
               Vorschlag. Wir wollen Ihnen weiter, \textsc{resp}. näher entgegen.
               Ich möchte z. B. Freitag den 23. von hier fort, nach Salzburg\oindex{Salzburg@\textbf{Salzburg}, \emph{A.ADM2}|pw}, da{\geminationn}{ }\textsc{per} Rad (we{\geminationn} ſich meines bis
               dahin erholt hat und {\pb}Richard\pwindex{Beer-Hofmann, Richard 1866-07-11 – 1945-09-26@\textsc{Beer-Hofmann, Richard} (1866-07-11 – 1945-09-26), \emph{Schriftsteller/Schriftstellerin}|pw} nicht faul iſt) über Reichenhall\oindex{Bad Reichenhall@\textbf{Bad Reichenhall}, \emph{A.ADM4}|pw}, \textsc{Lofer}\oindex{Lofer@\textbf{Lofer}, \emph{P.PPLA3}|pw} nach \textsc{Zell am See}\oindex{Zell am See@\textbf{Zell am See}, \emph{P.PPLA3}|pw}. Ich \textsc{resp}. wir würden Samſtag{ }Früh in Zell am See\oindex{Zell am See@\textbf{Zell am See}, \emph{P.PPLA3}|pw}{ }{[}ſ{]}ein, dort verbringen wir den Tag miteinander. Und Abend führe
               ich nach Wien\oindex{Wien@\textbf{Wien}, \emph{A.ADM2}|pw}. – Es handelt ſich alſo darum, ob
               Sie auf einen Tag von der \textsc{Fusch}\oindex{Bad Fusch@\textbf{Bad Fusch}, \emph{A.ADM3}|pw} wegkönnen. We{\geminationn}{ }Andrian\pwindex{Andrian-Werburg, Leopold von 09.05.1875 – 19.11.1951@\textsc{Andrian-Werburg, Leopold von} (09.05.1875 – 19.11.1951), \emph{Schriftsteller/Schriftstellerin, Diplomat/Diplomatin}|pw}{ }{\pb}mit Ihnen fahren wollte, ſo käme er mit. Grüßen Sie ihn
               herzlich von mir; es geht ihm hoffentlich wieder beſſer.\pend
           
\pstart
           Jahn\pwindex{Jahn, Otto 1813-06-16 – 1869-09-09@\textsc{Jahn, Otto} (1813-06-16 – 1869-09-09), \emph{Musikwissenschaftler/Musikwissenschaftlerin, Philologe/Philologin, Archäologe/Archäologin}|pw}{ }2. Band\pwindex{W. A. Mozart@\emph{W. A. Mozart}|pwv} beko{\geminationm}en? –\pend
           
\pstart
           – Auf einen ſchönen So{\geminationm}ertag mit Ihnen, we{\geminationn}’s ſchon nicht mehr ſein können, möcht ich nicht gern
               verzichten. Aber Sie ſollen ſich auch nicht die geringſte {\pb}Ungelegenheit machen.\pend
           \pstart Herzlich Ihr\spacefill\mbox{Arthur}\pend{}
\pstart
           \textsc{Ischl\oindex{Bad Ischl@\textbf{Bad Ischl}, \emph{P.PPL}|pw}}{ }15. 7. 97\pend
           \selectlanguage{ngerman}\endnumbering\briefempfaengerindex{Hofmannsthal, Hugo von@\textsc{Hofmannsthal, Hugo von}!zzzSchnitzler, Arthur@\emph{von Arthur Schnitzler}!1897-07-151@{15. 7. 1897}|)be}\mylabel{L00702h}  \normalsize

\doendnotes{C}
\bigskip
\vfill

\clearpage

\footnotesize

\lohead{\textsc{register}}

% Definiere theindex-Environment komplett neu ohne reledmac
\makeatletter
\renewenvironment{theindex}{%
  \section*{\indexname}%
  \setlength{\parindent}{0pt}%
  \setlength{\parskip}{0pt plus 0.3pt}%
  \let\item\@idxitem
}{%
  \clearpage
}
\makeatother

\IfFileExists{\jobname-pw.ind}{\input{\jobname-pw.ind}}{}

\end{document}

      