%% latex-leseansicht-vorspann.tex
%% Vorspann für die Leseansicht.
%% Lädt die gemeinsame Datei latex-vorspann.tex mit nicht gesetztem Schalter.

\newif\ifkorrekturansicht
\korrekturansichtfalse

\input{../tex-inputs/latex-vorspann}


\section[Arthur Schnitzler an Hugo von Hofmannsthal, 15. 7. 1897]{L00702 Arthur Schnitzler an Hugo von Hofmannsthal, 15. 7. 1897}
\nopagebreak\mylabel{L00702v}
\rehead{ }\normalsize\beginnumbering\briefempfaengerindex{Hofmannsthal, Hugo von@\textsc{Hofmannsthal, Hugo von}!zzzSchnitzler, Arthur@\emph{von Arthur Schnitzler}!1897-07-151@{15. 7. 1897}|(be}
\toendnotes[C]{\smallbreak\pagebreak[2]}
\correspDesc{Versand  durch Arthur Schnitzler am 15. 7. 1897 in Bad Ischl
\newline{}Erhalt  durch Hugo von Hofmannsthal im Zeitraum [16. 7. 1897
                  – 20. 7. 1897?] in Bad Fusch}\toendnotes[C]{\smallbreak}
\Standort{FDH, Hs-30885,61.}
\physDesc{Brief, 1 Blatt, 4 Seiten, 917 Zeichen
\newline{}Handschrift: Bleistift, deutsche Kurrent
\newline{}Ordnung: mit Bleistift von Schnitzler mutmaßlich bei der Durchsicht der
                                 Korrespondenz 1929 das erste Blatt datiert: »15/7 97« }
\buchAbdrucke{\weitereDrucke{Hugo von Hofmannsthal, Arthur Schnitzler: \emph{Briefwechsel}. Herausgegeben von Therese Nickl und Heinrich Schnitzler. Frankfurt am Main: \emph{S. Fischer} 1964, S. 91–92.} }\toendnotes[C]{\smallbreak}
\pstart
           \noindent{}{\pb}Mein lieber Hugo, ich ka{\geminationn} keineswegs
                  Anfang Auguſt mit Ihnen zusa{\geminationm}entreffen –
                  \label{K_L00702-1v}\edtext{Sie wiſſen ja}{\lemma{\textnormal{\emph{Sie wissen ja}}}\Cendnote{\textnormal{Seine Lebensgefährtin Marie Reinhard\pwindex{Reinhard, Marie 13.\,3.\,1871 Wien – 18.\,3.\,1899 ebd.@\textsc{Reinhard, Marie} (13.\,3.\,1871 Wien – 18.\,3.\,1899 ebd.), \emph{Gesangspädagogin}|pwk} war schwanger. Die ihm zugesagte Unterkunft für eine heimliche
                      Niederkunft war gerade abgesagt worden, vgl. XXXX Auszeichnungsfehler: Dokument L04121 nicht gefunden.
                      (Das Kind\pwindex{?? [Totgeborener Sohn von Arthur Schnitzler und Marie Reinhard] 24.\,9.\,1897 Endresstraße 68 – 24.\,9.\,1897 ebd.@\textsc{?? [Totgeborener Sohn von Arthur Schnitzler und Marie Reinhard]} (24.\,9.\,1897 Endresstraße 68 – 24.\,9.\,1897 ebd.)|pwkv} überlebte die Geburt nicht.)}}}\label{K_L00702-1}. Dagegen
               unterbreiten Richard\pwindex{Beer-Hofmann, Richard 11.\,7.\,1866 Wien – 26.\,9.\,1945 New York City@\textsc{Beer-Hofmann, Richard} (11.\,7.\,1866 Wien – 26.\,9.\,1945 New York City), \emph{Schriftsteller}|pw} u ich Ihnen einen andern
               Vorschlag. Wir wollen Ihnen weiter, \textsc{resp}. näher entgegen.
               Ich möchte z. B. Freitag den 23. von hier fort, nach Salzburg\oindex{Salzburg@\textbf{Salzburg}, \emph{Verwaltungsgebiet}|pw}, da{\geminationn}{ }\textsc{per} Rad (we{\geminationn}{ }ſich meines bis
               dahin erholt hat und {\pb}Richard\pwindex{Beer-Hofmann, Richard 11.\,7.\,1866 Wien – 26.\,9.\,1945 New York City@\textsc{Beer-Hofmann, Richard} (11.\,7.\,1866 Wien – 26.\,9.\,1945 New York City), \emph{Schriftsteller}|pw} nicht faul iſt) über Reichenhall\oindex{Bad Reichenhall@\textbf{Bad Reichenhall}, \emph{Region}|pw}, \textsc{Lofer}\oindex{Lofer@\textbf{Lofer}, \emph{Hauptstadt}|pw} nach \textsc{Zell am See}\oindex{Zell am See@\textbf{Zell am See}, \emph{Hauptstadt}|pw}. Ich \textsc{resp}. wir würden Samſtag{ }Früh in Zell am See\oindex{Zell am See@\textbf{Zell am See}, \emph{Hauptstadt}|pw}{ }{[}ſ{]}ein, dort verbringen wir den Tag miteinander. Und Abend führe
               ich nach Wien\oindex{Wien@\textbf{Wien}, \emph{Verwaltungsgebiet}|pw}. – Es handelt{ }ſich alſo darum, ob
               Sie auf einen Tag von der \textsc{Fusch}\oindex{Bad Fusch@\textbf{Bad Fusch}|pw} wegkönnen. We{\geminationn}{ }Andrian\pwindex{Andrian-Werburg, Leopold von 9.\,5.\,1875 Berlin – 19.\,11.\,1951 Fribourg@\textsc{Andrian-Werburg, Leopold von} (9.\,5.\,1875 Berlin – 19.\,11.\,1951 Fribourg), \emph{Schriftsteller, Diplomat}|pw}{ }{\pb}mit Ihnen fahren wollte,{ }ſo käme er mit. Grüßen Sie ihn
               herzlich von mir; es geht ihm hoffentlich wieder beſſer.\pend
           
\pstart
           Jahn\pwindex{Jahn, Otto 16.\,6.\,1813 Kiel – 9.\,9.\,1869 Göttingen@\textsc{Jahn, Otto} (16.\,6.\,1813 Kiel – 9.\,9.\,1869 Göttingen), \emph{Musikwissenschaftler, Philologe, Archäologe}|pw}{ }2. Band\pwindex{Jahn, Otto 16.\,6.\,1813 Kiel – 9.\,9.\,1869 Göttingen@\textsc{Jahn, Otto} (16.\,6.\,1813 Kiel – 9.\,9.\,1869 Göttingen), \emph{Musikwissenschaftler, Philologe, Archäologe}!W. A. Mozart@\strich\emph{W. A. Mozart}|pwv} beko{\geminationm}en? –\pend
           
\pstart
           – Auf einen{ }ſchönen So{\geminationm}ertag mit Ihnen, we{\geminationn}’s{ }ſchon nicht mehr{ }ſein können, möcht ich nicht gern
               verzichten. Aber Sie{ }ſollen{ }ſich auch nicht die geringſte {\pb}Ungelegenheit machen.\pend
           \pstart Herzlich Ihr\spacefill\mbox{Arthur}\pend{}
\pstart
           \textsc{Ischl\oindex{Bad Ischl@\textbf{Bad Ischl}|pw}}{ }15. 7. 97\pend
           \selectlanguage{ngerman}\endnumbering\briefempfaengerindex{Hofmannsthal, Hugo von@\textsc{Hofmannsthal, Hugo von}!zzzSchnitzler, Arthur@\emph{von Arthur Schnitzler}!1897-07-151@{15. 7. 1897}|)be}\mylabel{L00702h}  \newcommand{\dateiname}{L00702}\newcommand{\titel}{Arthur Schnitzler an Hugo von Hofmannsthal, 15. 7. 1897}\newcommand{\editorInnen}{Martin Anton Müller und Gerd-Hermann Susen}%% latex-leseansicht-abspann.tex
%% Abspann für die Leseansicht.
%% Der Schalter \ifkorrekturansicht ist bereits durch den Vorspann gesetzt.

%% latex-abspann.tex
%% Gemeinsamer Abspann für Korrekturansicht und Leseansicht.
%% Setzt den Schalter \ifkorrekturansicht voraus (gesetzt in den
%% einbindenden Dateien latex-korrekturansicht-abspann.tex bzw.
%% latex-leseansicht-abspann.tex).
%% ---------------------------------------------------------------

\normalsize

% Das esempio-Environment wird nur in der Leseansicht benötigt
\ifkorrekturansicht\else
\newenvironment{esempio}[3]%
{
    \vspace{1.5ex}
    \rlap{\underline{#1}}
    \par
    \setlength{\parindent}{0cm}
    \nopagebreak
    \leftskip=#2cm
    \rightskip=#3cm
}
{
    \par
}
\fi

\doendnotes{C}
\bigskip
\vfill

\clearpage

\footnotesize

\ifkorrekturansicht
  \lohead{\textsc{register}}
\fi

% theindex-Environment neu definieren ohne reledmac
\makeatletter
\renewenvironment{theindex}{%
  \ifkorrekturansicht
    \section*{\indexname}%
  \else
    \subsubsection*{Index der erwähnten Entitäten}%
  \fi
  \setlength{\parindent}{0pt}%
  \setlength{\parskip}{0pt plus 0.3pt}%
  \let\item\@idxitem
}{%
  \ifkorrekturansicht\clearpage\fi
}
\makeatother

\IfFileExists{\jobname-pw.ind}{\input{\jobname-pw.ind}}{}

% Quellenangabe nur in der Leseansicht
\ifkorrekturansicht\else
% Fallback-Definitionen, falls die .tex-Datei \titel etc. nicht gesetzt hat
\providecommand{\titel}{}
\providecommand{\editorInnen}{}
\providecommand{\dateiname}{\jobname}

\vspace{3cm}

\vfill

\footnotesize
\textsc{Quelle}: \titel. Herausgegeben von {\editorInnen}. In: \emph{Arthur Schnitzler: Briefwechsel mit Autorinnen und Autoren}.
 Digitale Edition, https://schnitzler-briefe.acdh.oeaw.ac.at/{\dateiname}.html (Stand \today)
\fi

\end{document}


