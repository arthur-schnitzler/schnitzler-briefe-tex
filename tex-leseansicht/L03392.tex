%% latex-leseansicht-vorspann.tex
%% Vorspann für die Leseansicht.
%% Lädt die gemeinsame Datei latex-vorspann.tex mit nicht gesetztem Schalter.

\newif\ifkorrekturansicht
\korrekturansichtfalse

\input{../tex-inputs/latex-vorspann}


\section[ Felix Salten an Arthur Schnitzler, 1. 2. 1904]{L03392 Felix Salten an Arthur Schnitzler,  1. 2. 1904}
\nopagebreak\mylabel{L03392v}
\rehead{ }\normalsize\beginnumbering\briefempfaengerindex{Schnitzler, Arthur@\textsc{Schnitzler, Arthur}!zzzSalten, Felix@\emph{von Felix Salten}!1904-02-013@{1. 2. 1904}|(be}
\toendnotes[C]{\smallbreak\pagebreak[2]}
\correspDesc{Versand  durch Felix Salten am 1. 2. 1904 in Wien
\newline{}Erhalt  durch Arthur Schnitzler im Zeitraum [1. 2. 1904
                  – 4. 2. 1904?] in Wien}\toendnotes[C]{\smallbreak}
\Standort{CUL, Schnitzler, B 89, B 1.}
\physDesc{Brief, 1 Blatt, 1 Seite, 239 Zeichen
\newline{}Handschrift: Bleistift, lateinische Kurrent
\newline{}Ordnung: mit Bleistift von unbekannter Hand nummeriert: »184« }\toendnotes[C]{\smallbreak}
\pstart
           \raggedleft{}{\pb}I. II. 04\pend
           \vspace{0.5em}
\pstart
           Lieber, wenns mir halbwegs möglich wird, denn ich habe – da \label{K_L03392-1v}\edtext{Otti\pwindex{Salten, Ottilie 7.\,3.\,1868 Prag – 22.\,6.\,1942 Zürich@\textsc{Salten, Ottilie} (7.\,3.\,1868 Prag – 22.\,6.\,1942 Zürich), \emph{Schauspielerin}|pw} spielt}{\lemma{\textnormal{\emph{Otti spielt}}}\Cendnote{\textnormal{Ottilie Salten\pwindex{Salten, Ottilie 7.\,3.\,1868 Prag – 22.\,6.\,1942 Zürich@\textsc{Salten, Ottilie} (7.\,3.\,1868 Prag – 22.\,6.\,1942 Zürich), \emph{Schauspielerin}|pwk} spielte die Marie Hartner\pwindex{Hawel, Rudolf 19.\,4.\,1860 Wien – 23.\,11.\,1923 ebd.@\textsc{Hawel, Rudolf} (19.\,4.\,1860 Wien – 23.\,11.\,1923 ebd.), \emph{Schriftsteller, Lehrer}!Politiker. Komödie in fünf Aufzügen@\strich\emph{Die Politiker. Komödie in fünf Aufzügen}|pwkv} in Rudolf Hawels\pwindex{Hawel, Rudolf 19.\,4.\,1860 Wien – 23.\,11.\,1923 ebd.@\textsc{Hawel, Rudolf} (19.\,4.\,1860 Wien – 23.\,11.\,1923 ebd.), \emph{Schriftsteller, Lehrer}|pwk} Komödie \emph{Die Politiker}\pwindex{Hawel, Rudolf 19.\,4.\,1860 Wien – 23.\,11.\,1923 ebd.@\textsc{Hawel, Rudolf} (19.\,4.\,1860 Wien – 23.\,11.\,1923 ebd.), \emph{Schriftsteller, Lehrer}!Politiker. Komödie in fünf Aufzügen@\strich\emph{Die Politiker. Komödie in fünf Aufzügen}|pwk} am \emph{Raimund-Theater}\orgindex{Raimund-Theater@Raimund-Theater|pwk}.}}}\label{K_L03392-1} – schon wo anders zugesagt, werd ich also \label{K_L03392-2v}\edtext{morgen{ }Abend (kaum vor 7\textsuperscript{h.}) zu Ihnen kommen}{\lemma{\textnormal{\emph{morgen … kommen}}}\Cendnote{\textnormal{Er kam, 
                  siehe A. S.: \emph{Tagebuch}, 2. 2. 1904.
               }}}\label{K_L03392-2}. Geht’s nicht, dann schreibe ich Ihnen noch heute{ }Nacht pneumatisch.\pend
           
\pstart
           Bestens Ihr {\\[\baselineskip]}\spacefill\mbox{S.}\pend
           \leftskip=0em{}\selectlanguage{ngerman}\endnumbering\briefempfaengerindex{Schnitzler, Arthur@\textsc{Schnitzler, Arthur}!zzzSalten, Felix@\emph{von Felix Salten}!1904-02-013@{1. 2. 1904}|)be}\mylabel{L03392h}  \newcommand{\dateiname}{L03392}\newcommand{\titel}{Felix Salten an Arthur Schnitzler, 1. 2. 1904}\newcommand{\editorInnen}{Martin Anton Müller und Laura Untner}%% latex-leseansicht-abspann.tex
%% Abspann für die Leseansicht.
%% Der Schalter \ifkorrekturansicht ist bereits durch den Vorspann gesetzt.

%% latex-abspann.tex
%% Gemeinsamer Abspann für Korrekturansicht und Leseansicht.
%% Setzt den Schalter \ifkorrekturansicht voraus (gesetzt in den
%% einbindenden Dateien latex-korrekturansicht-abspann.tex bzw.
%% latex-leseansicht-abspann.tex).
%% ---------------------------------------------------------------

\normalsize

% Das esempio-Environment wird nur in der Leseansicht benötigt
\ifkorrekturansicht\else
\newenvironment{esempio}[3]%
{
    \vspace{1.5ex}
    \rlap{\underline{#1}}
    \par
    \setlength{\parindent}{0cm}
    \nopagebreak
    \leftskip=#2cm
    \rightskip=#3cm
}
{
    \par
}
\fi

\doendnotes{C}
\bigskip
\vfill

\clearpage

\footnotesize

\ifkorrekturansicht
  \lohead{\textsc{register}}
\fi

% theindex-Environment neu definieren ohne reledmac
\makeatletter
\renewenvironment{theindex}{%
  \ifkorrekturansicht
    \section*{\indexname}%
  \else
    \subsubsection*{Index der erwähnten Entitäten}%
  \fi
  \setlength{\parindent}{0pt}%
  \setlength{\parskip}{0pt plus 0.3pt}%
  \let\item\@idxitem
}{%
  \ifkorrekturansicht\clearpage\fi
}
\makeatother

\IfFileExists{\jobname-pw.ind}{\input{\jobname-pw.ind}}{}

% Quellenangabe nur in der Leseansicht
\ifkorrekturansicht\else
% Fallback-Definitionen, falls die .tex-Datei \titel etc. nicht gesetzt hat
\providecommand{\titel}{}
\providecommand{\editorInnen}{}
\providecommand{\dateiname}{\jobname}

\vspace{3cm}

\vfill

\footnotesize
\textsc{Quelle}: \titel. Herausgegeben von {\editorInnen}. In: \emph{Arthur Schnitzler: Briefwechsel mit Autorinnen und Autoren}.
 Digitale Edition, https://schnitzler-briefe.acdh.oeaw.ac.at/{\dateiname}.html (Stand \today)
\fi

\end{document}


