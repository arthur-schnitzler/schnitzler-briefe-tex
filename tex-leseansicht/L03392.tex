%% latex-korrekturansicht-vorspann.tex
%% Vorspann für die Korrekturansicht.
%% Lädt die gemeinsame Datei latex-vorspann.tex mit gesetztem Schalter.

\newif\ifkorrekturansicht
\korrekturansichttrue

\input{../tex-inputs/latex-vorspann}


\section[ Felix Salten an Arthur Schnitzler, 1. 2. 1904]{L03392 Felix Salten an Arthur Schnitzler, 1. 2. 1904}
\nopagebreak\mylabel{L03392v}
\rehead{ }\normalsize\beginnumbering\briefempfaengerindex{Schnitzler, Arthur@\textsc{Schnitzler, Arthur}!zzzSalten, Felix@\emph{von Felix Salten}!1904-02-013@{1. 2. 1904}|(be}
\toendnotes[C]{\smallbreak\pagebreak[2]}\Standort{CUL, Schnitzler, B 89, B 1.}
\physDesc{Brief, 1 Blatt, 1 Seite, 239 Zeichen
\newline{}Handschrift: Bleistift, lateinische Kurrent
\newline{}Ordnung: mit Bleistift von unbekannter Hand nummeriert: »184« }\toendnotes[C]{\smallbreak}
\pstart
           \raggedleft{}{\pb}I. II. 04\pend
           \vspace{0.5em}
\pstart
           Lieber, wenns mir halbwegs möglich wird, denn ich habe – da \label{K_L03392-1v}\edtext{Otti\pwindex{Salten, Ottilie 07.03.1868 – 22.06.1942@\textsc{Salten, Ottilie} (07.03.1868 – 22.06.1942), \emph{Schauspieler/Schauspielerin}|pw} spielt}{\lemma{\textnormal{\emph{Otti spielt}}}\Cendnote{\textnormal{Ottilie Salten\pwindex{Salten, Ottilie 07.03.1868 – 22.06.1942@\textsc{Salten, Ottilie} (07.03.1868 – 22.06.1942), \emph{Schauspieler/Schauspielerin}|pwk} spielte die Marie Hartner\pwindex{Politiker. Komoedie in fuenf Aufzuegen@\emph{Die Politiker. Komödie in fünf Aufzügen}|pwkv} in Rudolf Hawels\pwindex{Hawel, Rudolf 1860-04-19 – 1923-11-23@\textsc{Hawel, Rudolf} (1860-04-19 – 1923-11-23), \emph{Schriftsteller/Schriftstellerin, Lehrer/Lehrerin}|pwk} Komödie \emph{Die Politiker}\pwindex{Politiker. Komoedie in fuenf Aufzuegen@\emph{Die Politiker. Komödie in fünf Aufzügen}|pwk} am \emph{Raimund-Theater}\orgindex{Raimund-Theater@Raimund-Theater|pwk}.}}}\label{K_L03392-1} – schon wo anders zugesagt, werd ich also \label{K_L03392-2v}\edtext{morgen{ }Abend (kaum vor 7\textsuperscript{h.}) zu Ihnen kommen}{\lemma{\textnormal{\emph{morgen … kommen}}}\Cendnote{\textnormal{Er kam, 
                  siehe A. S.: \emph{Tagebuch}, 2. 2. 1904.
               }}}\label{K_L03392-2}. Geht’s nicht, dann schreibe ich Ihnen noch heute{ }Nacht pneumatisch.\pend
           
\pstart
           Bestens Ihr {\\[\baselineskip]}\spacefill\mbox{S.}\pend
           \leftskip=0em{}\selectlanguage{ngerman}\endnumbering\briefempfaengerindex{Schnitzler, Arthur@\textsc{Schnitzler, Arthur}!zzzSalten, Felix@\emph{von Felix Salten}!1904-02-013@{1. 2. 1904}|)be}\mylabel{L03392h}  \normalsize

\doendnotes{C}
\bigskip
\vfill

\clearpage

\footnotesize

\lohead{\textsc{register}}

% Definiere theindex-Environment komplett neu ohne reledmac
\makeatletter
\renewenvironment{theindex}{%
  \section*{\indexname}%
  \setlength{\parindent}{0pt}%
  \setlength{\parskip}{0pt plus 0.3pt}%
  \let\item\@idxitem
}{%
  \clearpage
}
\makeatother

\IfFileExists{\jobname-pw.ind}{\input{\jobname-pw.ind}}{}

\end{document}

      