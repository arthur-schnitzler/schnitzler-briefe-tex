%% latex-leseansicht-vorspann.tex
%% Vorspann für die Leseansicht.
%% Lädt die gemeinsame Datei latex-vorspann.tex mit nicht gesetztem Schalter.

\newif\ifkorrekturansicht
\korrekturansichtfalse

\input{../tex-inputs/latex-vorspann}


\section[Berta Zuckerkandl an Arthur Schnitzler, 25. 7. {[}1911{]}]{L04009 Berta Zuckerkandl an Arthur Schnitzler, 25. 7. [1911]}
\nopagebreak\mylabel{L04009v}
\rehead{ }\normalsize\beginnumbering\briefempfaengerindex{Schnitzler, Arthur@\textsc{Schnitzler, Arthur}!zzzZuckerkandl, Berta@\emph{von Berta Zuckerkandl}!1911-07-251@{25. 7. [1911]}|(be}
\toendnotes[C]{\smallbreak\pagebreak[2]}
\correspDesc{Versand  durch Berta Zuckerkandl am 25. 7. [1911] in Wien
\newline{}Erhalt  durch Arthur Schnitzler im Zeitraum [25. 7. 1911
                  – 28. 7. 1911?] in Wien}\toendnotes[C]{\smallbreak}
\Standort{CUL, Schnitzler, B 200.}
\physDesc{Brief, 2 Blätter, 8 Seiten, 2048 Zeichen
\newline{}Handschrift: schwarze Tinte, lateinische Kurrent (\noindent{}Nummerierung des zweiten Bogens: »II.«)
\newline{}Schnitzler: beschriftet: »Zuckerkand\textcolor{gray}{l}« }\toendnotes[C]{\smallbreak}
\pstart
           \raggedleft{}{\pb}\textcolor{gray}{\textbf{\oindex{Hotel Axenstein@\textbf{Hotel Axenstein}, \emph{Hotel}|pw}PARK HOTEL AXENSTEIN }}\pend
           
\pstart
           \raggedleft{}\textcolor{gray}{\textbf{VIERWALDSTÄTTERSEE\oindex{Vierwaldstättersee@\textbf{Vierwaldstättersee}, \emph{See}|pw}}}\pend
           
\pstart
           \raggedleft{}\textcolor{gray}{\textbf{Axenstein\oindex{Axenstein@\textbf{Axenstein}, \emph{Ausflugsziel}|pw},}}{ }25\textsuperscript{ter} Juli\pend
           
\pstart{}Sehr verehrter Herr Doktor!\pend\vspace{0.5em}
\pstart
           Als ich von Paris\oindex{Paris@\textbf{Paris}, \emph{Hauptstadt}|pw} gegen 8\textsuperscript{ten} Juli wegfuhr stand die Medardus\pwindex{Schnitzler, Arthur 15. 5. 1862 Wien – 21. 10. 1931 ebd.@\textsc{Schnitzler, Arthur} (15. 5. 1862 Wien – 21. 10. 1931 ebd.), \emph{Schriftsteller, Mediziner}!junge Medardus. Dramatische Historie in einem Vorspiel und fünf Aufzügen@\strich\emph{Der junge Medardus. Dramatische Historie in einem Vorspiel und fünf Aufzügen}|pw}-Angelegenheit so gut – dass ich schon an Sieg
               glaubte – und Ihnen ein Telegram senden wollte. Es war besser ich {\pb}tat es nicht – denn so ist Ihnen
               wenigstens die Enttäuschung erspart worden – welche mir zu Teil wurde. Denn
                  gestern erhielt ich den Bescheid des Direktor Herz\pwindex{Hertz, Henri 17.\,6.\,1875 Nogent-sur-Seine – 11.\,10.\,1966 16. arrondissement [Paris]@\textsc{Hertz, Henri} (17.\,6.\,1875 Nogent-sur-Seine – 11.\,10.\,1966 16. arrondissement [Paris]), \emph{Schriftsteller, Journalist, Theaterdirektor}|pw} hieher, dass er trotzdem sie \label{K_L04009-1v}\edtext{\begin{otherlanguage}{french}entusiasmés de la pièce\pwindex{Schnitzler, Arthur 15. 5. 1862 Wien – 21. 10. 1931 ebd.@\textsc{Schnitzler, Arthur} (15. 5. 1862 Wien – 21. 10. 1931 ebd.), \emph{Schriftsteller, Mediziner}!junge Medardus. Dramatische Historie in einem Vorspiel und fünf Aufzügen@\strich\emph{Der junge Medardus. Dramatische Historie in einem Vorspiel und fünf Aufzügen}|pwv}\end{otherlanguage}}{\lemma{\textnormal{\emph{entusiasmés de la pièce}}}\Cendnote{\textnormal{französisch \begin{otherlanguage}{french}enthousiasmés de la pièce\end{otherlanguage}: begeistert vom Stück}}}\label{K_L04009-1} wären, dies\substVorne{}\textsuperscript{es}\substDazwischen{}e\substHinten{} nach reiflicher Erwägung {\pb}nicht
               zur Aufführung bringen könnten. Und zwar weil I\textsuperscript{sten} Für
               diese Saison 1911–12 die Porte St.
                  Martin\orgindex{Théâtre de la Porte Saint-Martin@Théâtre de la Porte Saint-Martin|pw} kontraktlich bereits abgeschlossen habe: Ein Lustspiel\pwindex{Flers, Robert de 25.\,11.\,1872 Pont-l'Évêque – 30.\,7.\,1927 Vittel@\textsc{Flers, Robert de} (25.\,11.\,1872 Pont-l'Évêque – 30.\,7.\,1927 Vittel), \emph{Schriftsteller}!Primerose, comédie en 3 actes@\strich\emph{Primerose, comédie en 3 actes}|pwv}\pwindex{Caillavet, Gaston Arman de 13.\,3.\,1869 Paris – 13.\,1.\,1915 Saint-Médard-d'Excideuil@\textsc{Caillavet, Gaston Arman de} (13.\,3.\,1869 Paris – 13.\,1.\,1915 Saint-Médard-d'Excideuil), \emph{Schriftsteller}!Primerose, comédie en 3 actes@\strich\emph{Primerose, comédie en 3 actes}|pwv} von Flers\pwindex{Flers, Robert de 25.\,11.\,1872 Pont-l'Évêque – 30.\,7.\,1927 Vittel@\textsc{Flers, Robert de} (25.\,11.\,1872 Pont-l'Évêque – 30.\,7.\,1927 Vittel), \emph{Schriftsteller}|pw} u Caillavet\pwindex{Caillavet, Gaston Arman de 13.\,3.\,1869 Paris – 13.\,1.\,1915 Saint-Médard-d'Excideuil@\textsc{Caillavet, Gaston Arman de} (13.\,3.\,1869 Paris – 13.\,1.\,1915 Saint-Médard-d'Excideuil), \emph{Schriftsteller}|pw}; \uline{zwei}{ }\label{K_L04009-2v}\edtext{Stücke\pwindex{Bataille, Henri 4.\,4.\,1872 Nîmes – 2.\,3.\,1922 Rueil-Malmaison@\textsc{Bataille, Henri} (4.\,4.\,1872 Nîmes – 2.\,3.\,1922 Rueil-Malmaison), \emph{Schriftsteller}!femme nue, pièce en quatre actes@\strich\emph{La femme nue, pièce en quatre actes}|pw}\pwindex{Bataille, Henri 4.\,4.\,1872 Nîmes – 2.\,3.\,1922 Rueil-Malmaison@\textsc{Bataille, Henri} (4.\,4.\,1872 Nîmes – 2.\,3.\,1922 Rueil-Malmaison), \emph{Schriftsteller}!Flambeaux, pièce en trois actes@\strich\emph{Les Flambeaux, pièce en trois actes}|pw} von Bataille\pwindex{Bataille, Henri 4.\,4.\,1872 Nîmes – 2.\,3.\,1922 Rueil-Malmaison@\textsc{Bataille, Henri} (4.\,4.\,1872 Nîmes – 2.\,3.\,1922 Rueil-Malmaison), \emph{Schriftsteller}|pw}}{\lemma{\textnormal{\emph{Stücke von Bataille}}}\Cendnote{\textnormal{Am 5. 10. 1911 hatte Batailles\pwindex{Bataille, Henri 4.\,4.\,1872 Nîmes – 2.\,3.\,1922 Rueil-Malmaison@\textsc{Bataille, Henri} (4.\,4.\,1872 Nîmes – 2.\,3.\,1922 Rueil-Malmaison), \emph{Schriftsteller}|pwk} Schauspiel \emph{La femme nue}\pwindex{Bataille, Henri 4.\,4.\,1872 Nîmes – 2.\,3.\,1922 Rueil-Malmaison@\textsc{Bataille, Henri} (4.\,4.\,1872 Nîmes – 2.\,3.\,1922 Rueil-Malmaison), \emph{Schriftsteller}!femme nue, pièce en quatre actes@\strich\emph{La femme nue, pièce en quatre actes}|pwk} am Théâtre
                     de la Porte Saint-Martin\oindex{Théâtre de la Porte Saint-Martin@\textbf{Théâtre de la Porte Saint-Martin}, \emph{Theater}|pwk}{ }Premiere\eventindex{Théâtre de la Porte Saint-Martin@\textbf{Théâtre de la Porte Saint-Martin}!Premiere von La femme nue@Premiere von La femme nue|pwkv}, das bereits 1908 am Théâtre de la Renaissance\oindex{Théâtre de la Renaissance@\textbf{Théâtre de la Renaissance}, \emph{Theater}|pwk} seine Uraufführung\eventindex{Théâtre de la Renaissance@\textbf{Théâtre de la Renaissance}!Uraufführung von La femme nue@Uraufführung von La femme nue|pwkv} erlebt hatte.
                  Die nächste Premiere von einem Schauspiel Batailles\pwindex{Bataille, Henri 4.\,4.\,1872 Nîmes – 2.\,3.\,1922 Rueil-Malmaison@\textsc{Bataille, Henri} (4.\,4.\,1872 Nîmes – 2.\,3.\,1922 Rueil-Malmaison), \emph{Schriftsteller}|pwk} am Théâtre de la Porte
                     Saint-Martin\oindex{Théâtre de la Porte Saint-Martin@\textbf{Théâtre de la Porte Saint-Martin}, \emph{Theater}|pwk}, die Uraufführung von \emph{Les Flambeaux}\pwindex{Bataille, Henri 4.\,4.\,1872 Nîmes – 2.\,3.\,1922 Rueil-Malmaison@\textsc{Bataille, Henri} (4.\,4.\,1872 Nîmes – 2.\,3.\,1922 Rueil-Malmaison), \emph{Schriftsteller}!Flambeaux, pièce en trois actes@\strich\emph{Les Flambeaux, pièce en trois actes}|pwk}\eventindex{Théâtre de la Porte Saint-Martin@\textbf{Théâtre de la Porte Saint-Martin}!Uraufführung von Les Flambeaux@Uraufführung von Les Flambeaux|pwk}, gab es erst in der nächsten Spielzeit am 26. 11. 1912.}}}\label{K_L04009-2}.
               Lauter Lustspiele die leicht zu montieren {\pb}sind, gar nichts kosten. Der \label{K_L04009-3v}\edtext{\begin{otherlanguage}{french}Eschek\end{otherlanguage}}{\lemma{\textnormal{\emph{Eschek}}}\Cendnote{\textnormal{französisch l'échec: Misserfolg; jiddisch heisik/hêsek: Schaden.}}}\label{K_L04009-3} des \label{K_L04009-4v}\edtext{Chantekler\pwindex{Rostand, Edmond 1.\,4.\,1868 Marseille – 2.\,12.\,1918 Paris@\textsc{Rostand, Edmond} (1.\,4.\,1868 Marseille – 2.\,12.\,1918 Paris), \emph{Schriftsteller}!Chantecler@\strich\emph{Chantecler}|pw}}{\lemma{\textnormal{\emph{Chantekler}}}\Cendnote{\textnormal{Die Uraufführung\eventindex{Théâtre de la Porte Saint-Martin@\textbf{Théâtre de la Porte Saint-Martin}!Uraufführung von Chantecler@Uraufführung von Chantecler|pwkv} des neuen Stückes \emph{Chantecler}\pwindex{Rostand, Edmond 1.\,4.\,1868 Marseille – 2.\,12.\,1918 Paris@\textsc{Rostand, Edmond} (1.\,4.\,1868 Marseille – 2.\,12.\,1918 Paris), \emph{Schriftsteller}!Chantecler@\strich\emph{Chantecler}|pwk} des Erfolgsautors von \emph{Cyrano de Bergerac}\pwindex{Rostand, Edmond 1.\,4.\,1868 Marseille – 2.\,12.\,1918 Paris@\textsc{Rostand, Edmond} (1.\,4.\,1868 Marseille – 2.\,12.\,1918 Paris), \emph{Schriftsteller}!Cyrano de Bergerac. Comédie héroïque en cinq actes en vers@\strich\emph{Cyrano de Bergerac. Comédie héroïque en cinq actes en vers}|pwk}, Edmond Rostand\pwindex{Rostand, Edmond 1.\,4.\,1868 Marseille – 2.\,12.\,1918 Paris@\textsc{Rostand, Edmond} (1.\,4.\,1868 Marseille – 2.\,12.\,1918 Paris), \emph{Schriftsteller}|pwk} im Vorjahr, war mit Spannung erwartet und mehrfach
                  verschoben worden. Die Inszenierung des Dramas\pwindex{Rostand, Edmond 1.\,4.\,1868 Marseille – 2.\,12.\,1918 Paris@\textsc{Rostand, Edmond} (1.\,4.\,1868 Marseille – 2.\,12.\,1918 Paris), \emph{Schriftsteller}!Chantecler@\strich\emph{Chantecler}|pwkv}, das über 70 Charaktere und noch mehr Kostüme
                  vorsah, war sehr aufwendig.}}}\label{K_L04009-4} hat Herz\pwindex{Hertz, Henri 17.\,6.\,1875 Nogent-sur-Seine – 11.\,10.\,1966 16. arrondissement [Paris]@\textsc{Hertz, Henri} (17.\,6.\,1875 Nogent-sur-Seine – 11.\,10.\,1966 16. arrondissement [Paris]), \emph{Schriftsteller, Journalist, Theaterdirektor}|pw} u Coquelin\pwindex{Coquelin, Jean 1.\,12.\,1865 Paris – 1.\,10.\,1944@\textsc{Coquelin, Jean} (1.\,12.\,1865 Paris – 1.\,10.\,1944), \emph{Schauspieler}|pw} abgeschreckt – das
               grosse Schauspiel zu pflegen, u sie suchen ihr Théater\orgindex{Théâtre de la Porte Saint-Martin@Théâtre de la Porte Saint-Martin|pwv} auf eine kleinere Basis zu stellen. Zufolge dessen
               wurde ein grosser Teil des Personals als überflüssig entlassen. Sie müssten für den
                  {\pb}Medardus\pwindex{Schnitzler, Arthur 15. 5. 1862 Wien – 21. 10. 1931 ebd.@\textsc{Schnitzler, Arthur} (15. 5. 1862 Wien – 21. 10. 1931 ebd.), \emph{Schriftsteller, Mediziner}!junge Medardus. Dramatische Historie in einem Vorspiel und fünf Aufzügen@\strich\emph{Der junge Medardus. Dramatische Historie in einem Vorspiel und fünf Aufzügen}|pw} (der jeden Falls erst Oktober 1912 hätte dran ko{\geminationm}en können) Alles wieder
               anders einrichten. Trotzdem liessen mich die Direktoren\pwindex{Hertz, Henri 17.\,6.\,1875 Nogent-sur-Seine – 11.\,10.\,1966 16. arrondissement [Paris]@\textsc{Hertz, Henri} (17.\,6.\,1875 Nogent-sur-Seine – 11.\,10.\,1966 16. arrondissement [Paris]), \emph{Schriftsteller, Journalist, Theaterdirektor}|pwv}\pwindex{Coquelin, Jean 1.\,12.\,1865 Paris – 1.\,10.\,1944@\textsc{Coquelin, Jean} (1.\,12.\,1865 Paris – 1.\,10.\,1944), \emph{Schauspieler}|pwv} auf diesen Bescheid
               wochenlang warten. Weil wie mir Albert Clemen{\pb}ceau\pwindex{Clemenceau, Albert 23.\,2.\,1861 Nantes – 23.\,7.\,1955 Sceaux@\textsc{Clemenceau, Albert} (23.\,2.\,1861 Nantes – 23.\,7.\,1955 Sceaux), \emph{Politiker, Jurist}|pw} noch am Tag vor meiner Abreise sagte – die grösste Lust besteht Medardus\pwindex{Schnitzler, Arthur 15. 5. 1862 Wien – 21. 10. 1931 ebd.@\textsc{Schnitzler, Arthur} (15. 5. 1862 Wien – 21. 10. 1931 ebd.), \emph{Schriftsteller, Mediziner}!junge Medardus. Dramatische Historie in einem Vorspiel und fünf Aufzügen@\strich\emph{Der junge Medardus. Dramatische Historie in einem Vorspiel und fünf Aufzügen}|pw} zu bringen. In der Auvergne\oindex{Auvergne@\textbf{Auvergne}|pw}
               wo beide Direktoren\pwindex{Coquelin, Jean 1.\,12.\,1865 Paris – 1.\,10.\,1944@\textsc{Coquelin, Jean} (1.\,12.\,1865 Paris – 1.\,10.\,1944), \emph{Schauspieler}|pwv}\pwindex{Hertz, Henri 17.\,6.\,1875 Nogent-sur-Seine – 11.\,10.\,1966 16. arrondissement [Paris]@\textsc{Hertz, Henri} (17.\,6.\,1875 Nogent-sur-Seine – 11.\,10.\,1966 16. arrondissement [Paris]), \emph{Schriftsteller, Journalist, Theaterdirektor}|pwv} ihr Ferien zubringen, scheint nun leider doch der mir mitgeteilte
               Entschluss gereift zu sein.\pend
           
\pstart
           {\pb}Es ist mir so leid Ihnen die Mühe des Scenario\pwindex{Schnitzler, Arthur 15. 5. 1862 Wien – 21. 10. 1931 ebd.@\textsc{Schnitzler, Arthur} (15. 5. 1862 Wien – 21. 10. 1931 ebd.), \emph{Schriftsteller, Mediziner}!junge Medardus. Dramatische Historie in einem Vorspiel und fünf Aufzügen@\strich\emph{Der junge Medardus. Dramatische Historie in einem Vorspiel und fünf Aufzügen}|pwv} verursacht zu haben.
               Vielleicht ist dieselbe aber nicht ganz verloren. D\textsuperscript{r}{ }Frischauer\pwindex{Frischauer, Berthold 9.\,9.\,1851 Brünn – 4.\,2.\,1924 Wien@\textsc{Frischauer, Berthold} (9.\,9.\,1851 Brünn – 4.\,2.\,1924 Wien), \emph{Journalist}|pw} schreibt mir eben – ich möge meine
                  \introOben{}Herz\pwindex{Hertz, Henri 17.\,6.\,1875 Nogent-sur-Seine – 11.\,10.\,1966 16. arrondissement [Paris]@\textsc{Hertz, Henri} (17.\,6.\,1875 Nogent-sur-Seine – 11.\,10.\,1966 16. arrondissement [Paris]), \emph{Schriftsteller, Journalist, Theaterdirektor}|pw}\introOben{} gelieferte Übersetzung desselben zurückverlangen – da eine andere Kombination
               in Paris\oindex{Paris@\textbf{Paris}, \emph{Hauptstadt}|pw} möglich ist. Wenn Sie also Geduld {\pb}haben wollen – so könnten wir abwarten.\pend
           
\pstart
           Wegen des weiten Landes\pwindex{Schnitzler, Arthur 15. 5. 1862 Wien – 21. 10. 1931 ebd.@\textsc{Schnitzler, Arthur} (15. 5. 1862 Wien – 21. 10. 1931 ebd.), \emph{Schriftsteller, Mediziner}!weite Land. Tragikomödie in fünf Akten@\strich\emph{Das weite Land. Tragikomödie in fünf Akten}|pw} dencke ich an Gémier\pwindex{Gémier, Firmin 21.\,2.\,1865 Aubervilliers – 26.\,11.\,1933 Paris@\textsc{Gémier, Firmin} (21.\,2.\,1865 Aubervilliers – 26.\,11.\,1933 Paris), \emph{Theaterleiter, Schauspieler, Drehbuchautor}|pw}. Ich werde wol im
                  Spätherbst wieder in Paris\oindex{Paris@\textbf{Paris}, \emph{Hauptstadt}|pw} sein –
               ich bin bereits (ohne das er weiss weshalb) in Relation mit ihm. Dafür muss ich nun
               aber noch Ihre besondere Autorisation haben –. Ende September also in
                  Wien\oindex{Wien@\textbf{Wien}, \emph{Verwaltungsgebiet}|pw}. Bitte verzeihen Sie – {\pb}\label{T_L04009-1v}\edtext{dass es nicht gelang –. Ihrer Frau Gemahlin\pwindex{Schnitzler, Olga 17.\,1.\,1882 Wien – 13.\,1.\,1970 Lugano@\textsc{Schnitzler, Olga} (17.\,1.\,1882 Wien – 13.\,1.\,1970 Lugano), \emph{Schauspielerin, Sängerin}|pw} u Ihnen herzlichste Grüsse. Ihre
               ergebene \spacefill\mbox{B. Zuckerkandl}}{\lemma{\textnormal{\emph{dass … Zuckerkandl}}}\Cendnote{\textnormal{Der Briefschluss befindet sich am unteren
                  Rand der fünften Seite.}}}\label{T_L04009-1}\pend
           \selectlanguage{ngerman}\endnumbering\briefempfaengerindex{Schnitzler, Arthur@\textsc{Schnitzler, Arthur}!zzzZuckerkandl, Berta@\emph{von Berta Zuckerkandl}!1911-07-251@{25. 7. [1911]}|)be}\mylabel{L04009h}
\begin{anhang}
\end{anhang}\newcommand{\dateiname}{L04009}\newcommand{\titel}{Berta Zuckerkandl an Arthur Schnitzler, 25. 7. [1911]}\newcommand{\editorInnen}{Herausgegeben von Jahnke, SelmaMüller, Martin Anton}%% latex-leseansicht-abspann.tex
%% Abspann für die Leseansicht.
%% Der Schalter \ifkorrekturansicht ist bereits durch den Vorspann gesetzt.

%% latex-abspann.tex
%% Gemeinsamer Abspann für Korrekturansicht und Leseansicht.
%% Setzt den Schalter \ifkorrekturansicht voraus (gesetzt in den
%% einbindenden Dateien latex-korrekturansicht-abspann.tex bzw.
%% latex-leseansicht-abspann.tex).
%% ---------------------------------------------------------------

\normalsize

% Das esempio-Environment wird nur in der Leseansicht benötigt
\ifkorrekturansicht\else
\newenvironment{esempio}[3]%
{
    \vspace{1.5ex}
    \rlap{\underline{#1}}
    \par
    \setlength{\parindent}{0cm}
    \nopagebreak
    \leftskip=#2cm
    \rightskip=#3cm
}
{
    \par
}
\fi

\doendnotes{C}
\bigskip
\vfill

\clearpage

\footnotesize

\ifkorrekturansicht
  \lohead{\textsc{register}}
\fi

% theindex-Environment neu definieren ohne reledmac
\makeatletter
\renewenvironment{theindex}{%
  \ifkorrekturansicht
    \section*{\indexname}%
  \else
    \subsubsection*{Index der erwähnten Entitäten}%
  \fi
  \setlength{\parindent}{0pt}%
  \setlength{\parskip}{0pt plus 0.3pt}%
  \let\item\@idxitem
}{%
  \ifkorrekturansicht\clearpage\fi
}
\makeatother

\IfFileExists{\jobname-pw.ind}{\input{\jobname-pw.ind}}{}

% Quellenangabe nur in der Leseansicht
\ifkorrekturansicht\else
% Fallback-Definitionen, falls die .tex-Datei \titel etc. nicht gesetzt hat
\providecommand{\titel}{}
\providecommand{\editorInnen}{}
\providecommand{\dateiname}{\jobname}

\vspace{3cm}

\vfill

\footnotesize
\textsc{Quelle}: \titel. Herausgegeben von {\editorInnen}. In: \emph{Arthur Schnitzler: Briefwechsel mit Autorinnen und Autoren}.
 Digitale Edition, https://schnitzler-briefe.acdh.oeaw.ac.at/{\dateiname}.html (Stand \today)
\fi

\end{document}


