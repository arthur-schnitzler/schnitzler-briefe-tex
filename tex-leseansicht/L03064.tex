%% latex-korrekturansicht-vorspann.tex
%% Vorspann für die Korrekturansicht.
%% Lädt die gemeinsame Datei latex-vorspann.tex mit gesetztem Schalter.

\newif\ifkorrekturansicht
\korrekturansichttrue

\input{../tex-inputs/latex-vorspann}


\section[ Paul Goldmann an Arthur Schnitzler, 26. 4. {[}1901{]}]{L03064 Paul Goldmann an Arthur Schnitzler, 26. 4. {[}1901{]}}
\nopagebreak\mylabel{L03064v}
\rehead{ }\normalsize\beginnumbering\briefempfaengerindex{Schnitzler, Arthur@\textsc{Schnitzler, Arthur}!zzzGoldmann, Paul@\emph{von Paul Goldmann}!1901-04-261@{26. 4. {[}1901{]}}|(be}
\toendnotes[C]{\smallbreak\pagebreak[2]}\Standort{DLA, A:Schnitzler, HS.NZ85.1.3171.}
\physDesc{Brief, 1 Blatt, 3 Seiten, 1435 Zeichen
\newline{}Handschrift: blaue Tinte, deutsche Kurrent
\newline{}Schnitzler: mit rotem Buntstift zwei Unterstreichungen }\toendnotes[C]{\smallbreak}
\pstart
           \raggedleft{}{\pb}\textcolor{gray}{\textbf{DESSAUERSTRASSE 19}}\oindex{Dessauer Strasse@\textbf{Dessauer Straße}, \emph{Straße (K.STR)}|pw}\pend
           
\pstart
           Berlin\oindex{Berlin@\textbf{Berlin}, \emph{P.PPLC}|pw}, 26. April.\pend
           
\pstart\center{}Mein lieber Freund,\pend\vspace{0.5em}
\pstart
           Dank für den lieben Brief! Dank auch für den »Schleier der \textsc{Beatrice}\pwindex{Schleier der Beatrice. Schauspiel in fuenf Akten@\emph{Der Schleier der Beatrice. Schauspiel in fünf Akten}|pw}« und »\textsc{Bertha Garlan\pwindex{Frau Bertha Garlan. Roman@\emph{Frau Bertha Garlan. Roman}|pw}}«, die ich in \label{K_L03064-1v}\edtext{ſchön gebundenen
                  Exemplaren}{\lemma{\textnormal{\emph{ſchön … Exemplaren}}}\Cendnote{\textnormal{\emph{Der Schleier der Beatrice}\pwindex{Schleier der Beatrice. Schauspiel in fuenf Akten@\emph{Der Schleier der Beatrice. Schauspiel in fünf Akten}|pwk} war am 21. 2. 1901 bei \emph{S.
                     Fischer}\orgindex{S. Fischer Verlag@S. Fischer Verlag|pwk} erschienen, \emph{Frau Bertha
                     Garlan}\pwindex{Frau Bertha Garlan. Roman@\emph{Frau Bertha Garlan. Roman}|pwk} am 13. 4. 1901.}}}\label{K_L03064-1} erhielt! Dank
               endlich für Deine Bemühungen bei \textsc{Bahr\pwindex{Bahr, Hermann 19.07.1863 – 15.01.1934@\textsc{Bahr, Hermann} (19.07.1863 – 15.01.1934), \emph{Schriftsteller/Schriftstellerin, Kritiker/Kritikerin}|pw}} in Sachen des Stückes \label{K_L03064-2v}\edtext{»Gewitter\pwindex{Gewitter. Drama in fuenf Akten@\emph{Gewitter. Drama in fünf Akten}|pwu}«}{\lemma{\textnormal{\emph{»Gewitter«}}}\Cendnote{\textnormal{Unklarer Bezug. Das Fehlen einer unmittelbaren Bezugnahme
                  verwirrt auch, weil das auf ein verlorenes Korrespondenzstück Goldmanns\pwindex{Goldmann, Paul 31.01.1865 – 25.09.1935@\textsc{Goldmann, Paul} (31.01.1865 – 25.09.1935), \emph{Schriftsteller/Schriftstellerin, Journalist/Journalistin}|pwk} verweisen dürfte. Möglicherweise handelte es
                  sich um den Fünfakter \emph{Gewitter}\pwindex{Gewitter. Drama in fuenf Akten@\emph{Gewitter. Drama in fünf Akten}|pwk} von Alexander Ostrowski\pwindex{Ostrovskij, Alexander N. 12.04.1823 – 14.06.1886@\textsc{Ostrovskij, Alexander N.} (12.04.1823 – 14.06.1886), \emph{Schriftsteller/Schriftstellerin}|pwk}, oder ein noch
                  unveröffentlichtes Werk einer unbekannten Person.}}}\label{K_L03064-2}!\pend
           
\pstart
           Ich freue mich, daß Du wieder glücklich \label{K_L03064-3v}\edtext{daheim}{\lemma{\textnormal{\emph{daheim}}}\Cendnote{\textnormal{Schnitzler war am 19. 4. 1901 von seiner
                     Italien\oindex{Italien@\textbf{Italien}, \emph{A.PCLI}|pwk}reise zurückgekehrt.}}}\label{K_L03064-3} biſt.
               Auch \label{K_L03064-4v}\edtext{die andere Nachricht}{\lemma{\textnormal{\emph{die andere Nachricht}}}\Cendnote{\textnormal{Olga\pwindex{Schnitzler, Olga 17.01.1882 – 13.01.1970@\textsc{Schnitzler, Olga} (17.01.1882 – 13.01.1970), \emph{Schauspieler/Schauspielerin, Sänger/Sängerin}|pwk} war mit dem gemeinsamen Kind schwanger.
                  Am 10. 5. 1901
                  musste die Schwangerschaft beendet werden.}}}\label{K_L03064-4} iſt \strikeout{\textcolor{gray}{rec}h\textcolor{gray}{t}} eine erfreuliche. Eine Frau und ein Kind, – das iſt wohl die \strikeout{Löſ\textcolor{gray}{un}} Erklärung für das, was die Natur mit uns vorhat; und demjenigen, der danach
               handelt, ſpendet ſie Glücksgefühle, wie immer, wenn man ihre geheimen Abſichten
               erräth. Das iſt der Weg zum Glück: die geheimen Abſichten der Natur errathen. Ich
               wünſche Dir einen Sohn.\pend
           
\pstart
           {\pb}Daß man mit ſeiner Geliebten nach Italien\oindex{Italien@\textbf{Italien}, \emph{A.PCLI}|pw} gehen muß, iſt ſelbſtverſtändlich. Ich möchte wiſſen,
               was Italien\oindex{Italien@\textbf{Italien}, \emph{A.PCLI}|pw} ſonſt \strikeout{\textcolor{gray}{×}\-\textcolor{gray}{×}} für einen Sinn hat, als den: eine Umgebung für eine Liebe zu ſein. Darum
               beneide ich Dich nicht um Deine \label{K_L03064-5v}\edtext{Rom\oindex{Rom@\textbf{Rom}, \emph{P.PPLC}|pw}fahrt}{\lemma{\textnormal{\emph{Romfahrt}}}\Cendnote{\textnormal{Siehe Paul Goldmann an Arthur Schnitzler, 6. 4. [1901].
               }}}\label{K_L03064-5}. Wohl aber beneide ich Dich um Deine \label{K_L03064-6v}\edtext{Sehnſucht nach \textsc{Olga\pwindex{Schnitzler, Olga 17.01.1882 – 13.01.1970@\textsc{Schnitzler, Olga} (17.01.1882 – 13.01.1970), \emph{Schauspieler/Schauspielerin, Sänger/Sängerin}|pw}}}{\lemma{\textnormal{\emph{Sehnſucht nach Olga}}}\Cendnote{\textnormal{Siehe A. S.: \emph{Tagebuch}, 17. 4. 1901.
               }}}\label{K_L03064-6}. Ich darf mich nach Keiner ſehnen.\pend
           
\pstart
           Der \label{K_L03064-7v}\edtext{Artikel\pwindex{Skikkelser og Tanker. Arthur Schnitzler@\emph{Skikkelser og Tanker. Arthur Schnitzler}|pwv}}{\lemma{\textnormal{\emph{Artikel}}}\Cendnote{\textnormal{Georg Brandes\pwindex{Brandes, Georg 04.02.1842 – 19.02.1927@\textsc{Brandes, Georg} (04.02.1842 – 19.02.1927)|pwk}: \emph{Skikkelser og Tanker. Arthur Schnitzler}\pwindex{Skikkelser og Tanker. Arthur Schnitzler@\emph{Skikkelser og Tanker. Arthur Schnitzler}|pwk}. In: \emph{Politiken}\pwindex{Politiken@\emph{Politiken}|pwk}, Nr. 98, 9. 4. 1901,
                     S. 1. Es gibt ein nicht überliefertes Korrespondenzstück Goldmanns\pwindex{Goldmann, Paul 31.01.1865 – 25.09.1935@\textsc{Goldmann, Paul} (31.01.1865 – 25.09.1935), \emph{Schriftsteller/Schriftstellerin, Journalist/Journalistin}|pwk}, in dem er Schnitzler den Artikel\pwindex{Skikkelser og Tanker. Arthur Schnitzler@\emph{Skikkelser og Tanker. Arthur Schnitzler}|pwkv} übersandte (vgl. Arthur Schnitzler an Georg Brandes, 25. 4. 1901).}}}\label{K_L03064-7} von \textsc{Brandes\pwindex{Brandes, Georg 04.02.1842 – 19.02.1927@\textsc{Brandes, Georg} (04.02.1842 – 19.02.1927)|pw}} über Dich war recht ſchleuderhaft geſchrieben. \textsc{Brandes\pwindex{Brandes, Georg 04.02.1842 – 19.02.1927@\textsc{Brandes, Georg} (04.02.1842 – 19.02.1927)|pw}} war dieſer Tage in Berlin\oindex{Berlin@\textbf{Berlin}, \emph{P.PPLC}|pw} – in merkwürdiger
               Stimmung: gezwungen heiter, manchmal verſtört. Plötzlich iſt er abgereiſt. Ich habe
               ihn ſehr gern. Er hat etwas ſo Feines und Gütiges\substVorne{}\textsuperscript{\textcolor{gray}{!}}\substDazwischen{}.\substHinten{}\pend
           
\pstart
           \label{K_L03064-8v}\edtext{Sommerpläne}{\lemma{\textnormal{\emph{Sommerpläne}}}\Cendnote{\textnormal{Goldmann\pwindex{Goldmann, Paul 31.01.1865 – 25.09.1935@\textsc{Goldmann, Paul} (31.01.1865 – 25.09.1935), \emph{Schriftsteller/Schriftstellerin, Journalist/Journalistin}|pwk}
                  versuchte in mehreren Briefen, Schnitzler und Olga Gussmann\pwindex{Schnitzler, Olga 17.01.1882 – 13.01.1970@\textsc{Schnitzler, Olga} (17.01.1882 – 13.01.1970), \emph{Schauspieler/Schauspielerin, Sänger/Sängerin}|pwk} zu einem Treffen
                  am Wörthersee\oindex{Woerthersee@\textbf{Wörthersee}, \emph{H.LK}|pwk} zu bewegen (Paul Goldmann an Arthur Schnitzler, 7. 5. [1901], Paul Goldmann an Olga Gussmann, 10. 5. [1901], Paul Goldmann an Arthur Schnitzler, 13. 5. [1901] und öfter). Letztlich
                  sahen er und 
                  Schnitzler sich im August 1901 mehrmals in Südtirol\oindex{Suedtirol@\textbf{Südtirol}, \emph{A.ADM2}|pwk}, konkret am 7. 8. 1901 in Welsberg\oindex{Welsberg-Taisten@\textbf{Welsberg-Taisten}, \emph{A.ADM3}|pwk}, am 13. 8. 1901 in Bozen\oindex{Bozen@\textbf{Bozen}, \emph{P.PPLA2}|pwk} und zwischen 18. 8. 1901 und 29. 8. 1901 noch
                  einmal in Welsberg\oindex{Welsberg-Taisten@\textbf{Welsberg-Taisten}, \emph{A.ADM3}|pwk}. Danach reiste Goldmann\pwindex{Goldmann, Paul 31.01.1865 – 25.09.1935@\textsc{Goldmann, Paul} (31.01.1865 – 25.09.1935), \emph{Schriftsteller/Schriftstellerin, Journalist/Journalistin}|pwk} mit Schnitzler nach Wien\oindex{Wien@\textbf{Wien}, \emph{A.ADM2}|pwk}
                  zurück und blieb dort wohl noch ein paar Tage.}}}\label{K_L03064-8}? Wie Du willſt. Mir {\pb}iſt Alles eins. Ich fahre weg oder bleibe auch zu
               Hauſe. Bin auf dem Tiefpunkt aller menſchlichen Verfaſſung angelangt{\dotsfour}\pend
           
\pstart
           Grüße an die Grünethorgaſſe\oindex{Gruenentorgasse@\textbf{Grünentorgasse}, \emph{Straße (K.STR)}|pw}\pwindex{Schnitzler, Olga 17.01.1882 – 13.01.1970@\textsc{Schnitzler, Olga} (17.01.1882 – 13.01.1970), \emph{Schauspieler/Schauspielerin, Sänger/Sängerin}|pwv}\pwindex{Steinrueck, Elisabeth 19.11.1885 – 07.04.1920@\textsc{Steinrück, Elisabeth} (19.11.1885 – 07.04.1920)|pwv}, Grüße an Dich! {\\[\baselineskip]}Von Herzen {\\[\baselineskip]}Dein {\\[\baselineskip]}\spacefill\mbox{Paul Goldmnn}\pend
           \leftskip=0em{}\selectlanguage{ngerman}\endnumbering\briefempfaengerindex{Schnitzler, Arthur@\textsc{Schnitzler, Arthur}!zzzGoldmann, Paul@\emph{von Paul Goldmann}!1901-04-261@{26. 4. {[}1901{]}}|)be}\mylabel{L03064h}  \normalsize

\doendnotes{C}
\bigskip
\vfill

\clearpage

\footnotesize

\lohead{\textsc{register}}

% Definiere theindex-Environment komplett neu ohne reledmac
\makeatletter
\renewenvironment{theindex}{%
  \section*{\indexname}%
  \setlength{\parindent}{0pt}%
  \setlength{\parskip}{0pt plus 0.3pt}%
  \let\item\@idxitem
}{%
  \clearpage
}
\makeatother

\IfFileExists{\jobname-pw.ind}{\input{\jobname-pw.ind}}{}

\end{document}

      