%% latex-leseansicht-vorspann.tex
%% Vorspann für die Leseansicht.
%% Lädt die gemeinsame Datei latex-vorspann.tex mit nicht gesetztem Schalter.

\newif\ifkorrekturansicht
\korrekturansichtfalse

\input{../tex-inputs/latex-vorspann}


\section[ Paul Goldmann an Arthur Schnitzler, 26. 4. {[}1901{]}]{L03064 Paul Goldmann an Arthur Schnitzler,  26. 4. [1901]}
\nopagebreak\mylabel{L03064v}
\rehead{ }\normalsize\beginnumbering\briefempfaengerindex{Schnitzler, Arthur@\textsc{Schnitzler, Arthur}!zzzGoldmann, Paul@\emph{von Paul Goldmann}!1901-04-261@{26. 4. [1901]}|(be}
\toendnotes[C]{\smallbreak\pagebreak[2]}
\correspDesc{Versand  durch Paul Goldmann am 26. 4. [1901] in Berlin
\newline{}Erhalt  durch Arthur Schnitzler im Zeitraum [27. 4. 1901
                  – 1. 5. 1901?] in Wien}\toendnotes[C]{\smallbreak}
\Standort{DLA, A:Schnitzler, HS.NZ85.1.3171.}
\physDesc{Brief, 1 Blatt, 3 Seiten, 1435 Zeichen
\newline{}Handschrift: blaue Tinte, deutsche Kurrent
\newline{}Schnitzler: mit rotem Buntstift zwei Unterstreichungen }\toendnotes[C]{\smallbreak}
\pstart
           \raggedleft{}{\pb}\textcolor{gray}{\textbf{DESSAUERSTRASSE 19}}\oindex{Dessauer Straße@\textbf{Dessauer Straße}, \emph{Straße}|pw}\pend
           
\pstart
           Berlin\oindex{Berlin@\textbf{Berlin}, \emph{Hauptstadt}|pw}, 26. April.\pend
           
\pstart\center{}Mein lieber Freund,\pend\vspace{0.5em}
\pstart
           Dank für den lieben Brief! Dank auch für den »Schleier der \textsc{Beatrice}\pwindex{Schnitzler, Arthur 15.\,5.\,1862 Wien – 21.\,10.\,1931 ebd.@\textsc{Schnitzler, Arthur} (15.\,5.\,1862 Wien – 21.\,10.\,1931 ebd.), \emph{Schriftsteller, Mediziner}!Schleier der Beatrice. Schauspiel in fünf Akten@\strich\emph{Der Schleier der Beatrice. Schauspiel in fünf Akten}|pw}« und »\textsc{Bertha Garlan\pwindex{Schnitzler, Arthur 15.\,5.\,1862 Wien – 21.\,10.\,1931 ebd.@\textsc{Schnitzler, Arthur} (15.\,5.\,1862 Wien – 21.\,10.\,1931 ebd.), \emph{Schriftsteller, Mediziner}!Frau Bertha Garlan. Roman@\strich\emph{Frau Bertha Garlan. Roman}|pw}}«, die ich in \label{K_L03064-1v}\edtext{ſchön gebundenen
                  Exemplaren}{\lemma{\textnormal{\emph{schön … Exemplaren}}}\Cendnote{\textnormal{\emph{Der Schleier der Beatrice}\pwindex{Schnitzler, Arthur 15.\,5.\,1862 Wien – 21.\,10.\,1931 ebd.@\textsc{Schnitzler, Arthur} (15.\,5.\,1862 Wien – 21.\,10.\,1931 ebd.), \emph{Schriftsteller, Mediziner}!Schleier der Beatrice. Schauspiel in fünf Akten@\strich\emph{Der Schleier der Beatrice. Schauspiel in fünf Akten}|pwk} war am 21. 2. 1901 bei \emph{S.
                     Fischer}\orgindex{S. Fischer Verlag@S. Fischer Verlag|pwk} erschienen, \emph{Frau Bertha
                     Garlan}\pwindex{Schnitzler, Arthur 15.\,5.\,1862 Wien – 21.\,10.\,1931 ebd.@\textsc{Schnitzler, Arthur} (15.\,5.\,1862 Wien – 21.\,10.\,1931 ebd.), \emph{Schriftsteller, Mediziner}!Frau Bertha Garlan. Roman@\strich\emph{Frau Bertha Garlan. Roman}|pwk} am 13. 4. 1901.}}}\label{K_L03064-1} erhielt! Dank
               endlich für Deine Bemühungen bei \textsc{Bahr\pwindex{Bahr, Hermann 19.\,7.\,1863 Linz – 15.\,1.\,1934 München@\textsc{Bahr, Hermann} (19.\,7.\,1863 Linz – 15.\,1.\,1934 München), \emph{Schriftsteller, Kritiker}|pw}} in Sachen des Stückes \label{K_L03064-2v}\edtext{»Gewitter\pwindex{Ostrovskij, Alexander N. 12.\,4.\,1823 Moskau – 14.\,6.\,1886 Shchelykovo@\textsc{Ostrovskij, Alexander N.} (12.\,4.\,1823 Moskau – 14.\,6.\,1886 Shchelykovo), \emph{Schriftsteller}!Gewitter. Drama in fünf Akten@\strich\emph{Gewitter. Drama in fünf Akten}|pwu}«}{\lemma{\textnormal{\emph{»Gewitter«}}}\Cendnote{\textnormal{Unklarer Bezug. Das Fehlen einer unmittelbaren Bezugnahme
                  verwirrt auch, weil das auf ein verlorenes Korrespondenzstück Goldmanns\pwindex{Goldmann, Paul 31.\,1.\,1865 Breslau – 25.\,9.\,1935 Wien@\textsc{Goldmann, Paul} (31.\,1.\,1865 Breslau – 25.\,9.\,1935 Wien), \emph{Schriftsteller, Journalist}|pwk} verweisen dürfte. Möglicherweise handelte es
                  sich um den Fünfakter \emph{Gewitter}\pwindex{Ostrovskij, Alexander N. 12.\,4.\,1823 Moskau – 14.\,6.\,1886 Shchelykovo@\textsc{Ostrovskij, Alexander N.} (12.\,4.\,1823 Moskau – 14.\,6.\,1886 Shchelykovo), \emph{Schriftsteller}!Gewitter. Drama in fünf Akten@\strich\emph{Gewitter. Drama in fünf Akten}|pwk} von Alexander Ostrowski\pwindex{Ostrovskij, Alexander N. 12.\,4.\,1823 Moskau – 14.\,6.\,1886 Shchelykovo@\textsc{Ostrovskij, Alexander N.} (12.\,4.\,1823 Moskau – 14.\,6.\,1886 Shchelykovo), \emph{Schriftsteller}|pwk}, oder ein noch
                  unveröffentlichtes Werk einer unbekannten Person.}}}\label{K_L03064-2}!\pend
           
\pstart
           Ich freue mich, daß Du wieder glücklich \label{K_L03064-3v}\edtext{daheim}{\lemma{\textnormal{\emph{daheim}}}\Cendnote{\textnormal{Schnitzler war am 19. 4. 1901 von seiner
                     Italien\oindex{Italien@\textbf{Italien}|pwk}reise zurückgekehrt.}}}\label{K_L03064-3} biſt.
               Auch \label{K_L03064-4v}\edtext{die andere Nachricht}{\lemma{\textnormal{\emph{die andere Nachricht}}}\Cendnote{\textnormal{Olga\pwindex{Schnitzler, Olga 17.\,1.\,1882 Wien – 13.\,1.\,1970 Lugano@\textsc{Schnitzler, Olga} (17.\,1.\,1882 Wien – 13.\,1.\,1970 Lugano), \emph{Schauspielerin, Sängerin}|pwk} war mit dem gemeinsamen Kind schwanger.
                  Am 10. 5. 1901
                  musste die Schwangerschaft beendet werden.}}}\label{K_L03064-4} iſt \strikeout{\textcolor{gray}{rec}h\textcolor{gray}{t}} eine erfreuliche. Eine Frau und ein Kind, – das iſt wohl die \strikeout{Löſ\textcolor{gray}{un}} Erklärung für das, was die Natur mit uns vorhat; und demjenigen, der danach
               handelt,{ }ſpendet{ }ſie Glücksgefühle, wie immer, wenn man ihre geheimen Abſichten
               erräth. Das iſt der Weg zum Glück: die geheimen Abſichten der Natur errathen. Ich
               wünſche Dir einen Sohn.\pend
           
\pstart
           {\pb}Daß man mit{ }ſeiner Geliebten nach Italien\oindex{Italien@\textbf{Italien}|pw} gehen muß, iſt{ }ſelbſtverſtändlich. Ich möchte wiſſen,
               was Italien\oindex{Italien@\textbf{Italien}|pw}{ }ſonſt \strikeout{\textcolor{gray}{×}\-\textcolor{gray}{×}} für einen Sinn hat, als den: eine Umgebung für eine Liebe zu{ }ſein. Darum
               beneide ich Dich nicht um Deine \label{K_L03064-5v}\edtext{Rom\oindex{Rom@\textbf{Rom}, \emph{Hauptstadt}|pw}fahrt}{\lemma{\textnormal{\emph{Romfahrt}}}\Cendnote{\textnormal{Siehe XXXX Auszeichnungsfehler: Dokument L03063 nicht gefunden.
               }}}\label{K_L03064-5}. Wohl aber beneide ich Dich um Deine \label{K_L03064-6v}\edtext{Sehnſucht nach \textsc{Olga\pwindex{Schnitzler, Olga 17.\,1.\,1882 Wien – 13.\,1.\,1970 Lugano@\textsc{Schnitzler, Olga} (17.\,1.\,1882 Wien – 13.\,1.\,1970 Lugano), \emph{Schauspielerin, Sängerin}|pw}}}{\lemma{\textnormal{\emph{Sehnsucht nach Olga}}}\Cendnote{\textnormal{Siehe A. S.: \emph{Tagebuch}, 17. 4. 1901.
               }}}\label{K_L03064-6}. Ich darf mich nach Keiner{ }ſehnen.\pend
           
\pstart
           Der \label{K_L03064-7v}\edtext{Artikel\pwindex{Brandes, Georg 4.\,2.\,1842 Kopenhagen – 19.\,2.\,1927 ebd.@\textsc{Brandes, Georg} (4.\,2.\,1842 Kopenhagen – 19.\,2.\,1927 ebd.)!Skikkelser og Tanker. Arthur Schnitzler@\strich\emph{Skikkelser og Tanker. Arthur Schnitzler}|pwv}}{\lemma{\textnormal{\emph{Artikel}}}\Cendnote{\textnormal{Georg Brandes\pwindex{Brandes, Georg 4.\,2.\,1842 Kopenhagen – 19.\,2.\,1927 ebd.@\textsc{Brandes, Georg} (4.\,2.\,1842 Kopenhagen – 19.\,2.\,1927 ebd.)|pwk}: \emph{Skikkelser og Tanker. Arthur Schnitzler}\pwindex{Brandes, Georg 4.\,2.\,1842 Kopenhagen – 19.\,2.\,1927 ebd.@\textsc{Brandes, Georg} (4.\,2.\,1842 Kopenhagen – 19.\,2.\,1927 ebd.)!Skikkelser og Tanker. Arthur Schnitzler@\strich\emph{Skikkelser og Tanker. Arthur Schnitzler}|pwk}. In: \emph{Politiken}\pwindex{Politiken@\emph{Politiken}|pwk}, Nr. 98, 9. 4. 1901,
                     S. 1. Es gibt ein nicht überliefertes Korrespondenzstück Goldmanns\pwindex{Goldmann, Paul 31.\,1.\,1865 Breslau – 25.\,9.\,1935 Wien@\textsc{Goldmann, Paul} (31.\,1.\,1865 Breslau – 25.\,9.\,1935 Wien), \emph{Schriftsteller, Journalist}|pwk}, in dem er Schnitzler den Artikel\pwindex{Brandes, Georg 4.\,2.\,1842 Kopenhagen – 19.\,2.\,1927 ebd.@\textsc{Brandes, Georg} (4.\,2.\,1842 Kopenhagen – 19.\,2.\,1927 ebd.)!Skikkelser og Tanker. Arthur Schnitzler@\strich\emph{Skikkelser og Tanker. Arthur Schnitzler}|pwkv} übersandte (vgl. XXXX Auszeichnungsfehler: Dokument L01114 nicht gefunden).}}}\label{K_L03064-7} von \textsc{Brandes\pwindex{Brandes, Georg 4.\,2.\,1842 Kopenhagen – 19.\,2.\,1927 ebd.@\textsc{Brandes, Georg} (4.\,2.\,1842 Kopenhagen – 19.\,2.\,1927 ebd.)|pw}} über Dich war recht{ }ſchleuderhaft geſchrieben. \textsc{Brandes\pwindex{Brandes, Georg 4.\,2.\,1842 Kopenhagen – 19.\,2.\,1927 ebd.@\textsc{Brandes, Georg} (4.\,2.\,1842 Kopenhagen – 19.\,2.\,1927 ebd.)|pw}} war dieſer Tage in Berlin\oindex{Berlin@\textbf{Berlin}, \emph{Hauptstadt}|pw} – in merkwürdiger
               Stimmung: gezwungen heiter, manchmal verſtört. Plötzlich iſt er abgereiſt. Ich habe
               ihn{ }ſehr gern. Er hat etwas{ }ſo Feines und Gütiges\substVorne{}\textsuperscript{\textcolor{gray}{!}}\substDazwischen{}.\substHinten{}\pend
           
\pstart
           \label{K_L03064-8v}\edtext{Sommerpläne}{\lemma{\textnormal{\emph{Sommerpläne}}}\Cendnote{\textnormal{Goldmann\pwindex{Goldmann, Paul 31.\,1.\,1865 Breslau – 25.\,9.\,1935 Wien@\textsc{Goldmann, Paul} (31.\,1.\,1865 Breslau – 25.\,9.\,1935 Wien), \emph{Schriftsteller, Journalist}|pwk}
                  versuchte in mehreren Briefen, Schnitzler und Olga Gussmann\pwindex{Schnitzler, Olga 17.\,1.\,1882 Wien – 13.\,1.\,1970 Lugano@\textsc{Schnitzler, Olga} (17.\,1.\,1882 Wien – 13.\,1.\,1970 Lugano), \emph{Schauspielerin, Sängerin}|pwk} zu einem Treffen
                  am Wörthersee\oindex{Wörthersee@\textbf{Wörthersee}, \emph{See}|pwk} zu bewegen (XXXX Auszeichnungsfehler: Dokument L03065 nicht gefunden, XXXX Auszeichnungsfehler: Dokument L03527 nicht gefunden, XXXX Auszeichnungsfehler: Dokument L03066 nicht gefunden und öfter). Letztlich
                  sahen er und 
                  Schnitzler sich im August 1901 mehrmals in Südtirol\oindex{Südtirol@\textbf{Südtirol}, \emph{Verwaltungsgebiet}|pwk}, konkret am 7. 8. 1901 in Welsberg\oindex{Welsberg-Taisten@\textbf{Welsberg-Taisten}, \emph{Verwaltungsgebiet}|pwk}, am 13. 8. 1901 in Bozen\oindex{Bozen@\textbf{Bozen}, \emph{Hauptstadt}|pwk} und zwischen 18. 8. 1901 und 29. 8. 1901 noch
                  einmal in Welsberg\oindex{Welsberg-Taisten@\textbf{Welsberg-Taisten}, \emph{Verwaltungsgebiet}|pwk}. Danach reiste Goldmann\pwindex{Goldmann, Paul 31.\,1.\,1865 Breslau – 25.\,9.\,1935 Wien@\textsc{Goldmann, Paul} (31.\,1.\,1865 Breslau – 25.\,9.\,1935 Wien), \emph{Schriftsteller, Journalist}|pwk} mit Schnitzler nach Wien\oindex{Wien@\textbf{Wien}, \emph{Verwaltungsgebiet}|pwk}
                  zurück und blieb dort wohl noch ein paar Tage.}}}\label{K_L03064-8}? Wie Du willſt. Mir {\pb}iſt Alles eins. Ich fahre weg oder bleibe auch zu
               Hauſe. Bin auf dem Tiefpunkt aller menſchlichen Verfaſſung angelangt{\dotsfour}\pend
           
\pstart
           Grüße an die Grünethorgaſſe\oindex{Wien@\textbf{Wien}!IX., Alsergrund@\textbf{IX., Alsergrund}!Grünentorgasse@\textbf{Grünentorgasse}, \emph{Straße}|pw}\pwindex{Schnitzler, Olga 17.\,1.\,1882 Wien – 13.\,1.\,1970 Lugano@\textsc{Schnitzler, Olga} (17.\,1.\,1882 Wien – 13.\,1.\,1970 Lugano), \emph{Schauspielerin, Sängerin}|pwv}\pwindex{Steinrück, Elisabeth 19.\,11.\,1885 – 7.\,4.\,1920 Partenkirchen@\textsc{Steinrück, Elisabeth} (19.\,11.\,1885 – 7.\,4.\,1920 Partenkirchen)|pwv}, Grüße an Dich! {\\[\baselineskip]}Von Herzen {\\[\baselineskip]}Dein {\\[\baselineskip]}\spacefill\mbox{Paul Goldmnn}\pend
           \leftskip=0em{}\selectlanguage{ngerman}\endnumbering\briefempfaengerindex{Schnitzler, Arthur@\textsc{Schnitzler, Arthur}!zzzGoldmann, Paul@\emph{von Paul Goldmann}!1901-04-261@{26. 4. [1901]}|)be}\mylabel{L03064h}  \newcommand{\dateiname}{L03064}\newcommand{\titel}{Paul Goldmann an Arthur Schnitzler, 26. 4. [1901]}\newcommand{\editorInnen}{Martin Anton Müller und Laura Untner}%% latex-leseansicht-abspann.tex
%% Abspann für die Leseansicht.
%% Der Schalter \ifkorrekturansicht ist bereits durch den Vorspann gesetzt.

%% latex-abspann.tex
%% Gemeinsamer Abspann für Korrekturansicht und Leseansicht.
%% Setzt den Schalter \ifkorrekturansicht voraus (gesetzt in den
%% einbindenden Dateien latex-korrekturansicht-abspann.tex bzw.
%% latex-leseansicht-abspann.tex).
%% ---------------------------------------------------------------

\normalsize

% Das esempio-Environment wird nur in der Leseansicht benötigt
\ifkorrekturansicht\else
\newenvironment{esempio}[3]%
{
    \vspace{1.5ex}
    \rlap{\underline{#1}}
    \par
    \setlength{\parindent}{0cm}
    \nopagebreak
    \leftskip=#2cm
    \rightskip=#3cm
}
{
    \par
}
\fi

\doendnotes{C}
\bigskip
\vfill

\clearpage

\footnotesize

\ifkorrekturansicht
  \lohead{\textsc{register}}
\fi

% theindex-Environment neu definieren ohne reledmac
\makeatletter
\renewenvironment{theindex}{%
  \ifkorrekturansicht
    \section*{\indexname}%
  \else
    \subsubsection*{Index der erwähnten Entitäten}%
  \fi
  \setlength{\parindent}{0pt}%
  \setlength{\parskip}{0pt plus 0.3pt}%
  \let\item\@idxitem
}{%
  \ifkorrekturansicht\clearpage\fi
}
\makeatother

\IfFileExists{\jobname-pw.ind}{\input{\jobname-pw.ind}}{}

% Quellenangabe nur in der Leseansicht
\ifkorrekturansicht\else
% Fallback-Definitionen, falls die .tex-Datei \titel etc. nicht gesetzt hat
\providecommand{\titel}{}
\providecommand{\editorInnen}{}
\providecommand{\dateiname}{\jobname}

\vspace{3cm}

\vfill

\footnotesize
\textsc{Quelle}: \titel. Herausgegeben von {\editorInnen}. In: \emph{Arthur Schnitzler: Briefwechsel mit Autorinnen und Autoren}.
 Digitale Edition, https://schnitzler-briefe.acdh.oeaw.ac.at/{\dateiname}.html (Stand \today)
\fi

\end{document}


