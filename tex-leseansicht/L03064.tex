%% latex-leseansicht-vorspann.tex
%% Vorspann für die Leseansicht.
%% Lädt die gemeinsame Datei latex-vorspann.tex mit nicht gesetztem Schalter.

\newif\ifkorrekturansicht
\korrekturansichtfalse

\input{../tex-inputs/latex-vorspann}


         
         \renewcommand{\erwaehntePersonen}{Personen: Hermann Bahr, Georg Brandes, Alexander N. Ostrovskij, Olga Schnitzler, Elisabeth Steinrück}
         \renewcommand{\erwaehnteInstitutionen}{Institutionen: S. Fischer Verlag}
         \renewcommand{\erwaehnteOrte}{Orte: Berlin, Bozen, Dessauer Straße, Grünentorgasse, Italien, Rom, Südtirol, Welsberg-Taisten, Wien}
         \renewcommand{\erwaehnteWerke}{Werke: Der Schleier der Beatrice. Schauspiel in fünf Akten, Frau Bertha Garlan. Roman, Gewitter. Drama in fünf Akten, Politiken, Skikkelser og Tanker. Arthur Schnitzler}
               \section[ Paul Goldmann an Arthur Schnitzler, 26. 4. {[}1901{]}]{ Paul Goldmann an Arthur Schnitzler, 26. 4. {[}1901{]}}\nopagebreak\mylabel{v}\rehead{ }\begin{ledgroupsized}[t]{13cm}\normalsize\beginnumbering \toendnotes[C]{\smallbreak\pagebreak[2]} \Standort{DLA, A:Schnitzler, HS.NZ85.1.3171.}
\physDesc{Brief, 1 Blatt, 3 Seiten, 1435 Zeichen
\newline{}Handschrift: blaue Tinte, deutsche Kurrent
\newline{}Schnitzler: mit rotem Buntstift zwei Unterstreichungen }\toendnotes[C]{\smallbreak}\pstart
           \noindent{}\raggedleft{}{\pb}\textcolor{gray}{\textbf{DESSAUERSTRASSE 19}}\oindex{Dessauer Strasse@\textbf{Dessauer Straße}|pw}\pend
           \pstart
           Berlin\oindex{Berlin@\textbf{Berlin}|pw}, 26. April.\pend
           \pstart\center{}Mein lieber Freund,\pend\pstart
           Dank für den lieben Brief! Dank auch für den »Schleier der \textsc{Beatrice}\pwindex{Schnitzler, Arthur 15.05.1862 – 21.10.1931@\textsc{Schnitzler, Arthur} (15.05.1862 – 21.10.1931), \emph{Schriftsteller, Mediziner}!Schleier der Beatrice. Schauspiel in fuenf Akten1900-12-01@\strich\emph{Der Schleier der Beatrice. Schauspiel in fünf Akten} {[}1900-12-01{]}|pw}« und »\textsc{Bertha Garlan\pwindex{Schnitzler, Arthur 15.05.1862 – 21.10.1931@\textsc{Schnitzler, Arthur} (15.05.1862 – 21.10.1931), \emph{Schriftsteller, Mediziner}!Frau Bertha Garlan. Roman15.1.1901 – 15.3.1901@\strich\emph{Frau Bertha Garlan. Roman} {[}15.1.1901 – 15.3.1901{]}|pw}}«, die ich in \label{K_L03064-1v}\edtext{ſchön gebundenen
                  Exemplaren}{\lemma{\textnormal{\emph{ſchön … Exemplaren}}}\Cendnote{\textnormal{\emph{Der Schleier der Beatrice}\pwindex{Schnitzler, Arthur 15.05.1862 – 21.10.1931@\textsc{Schnitzler, Arthur} (15.05.1862 – 21.10.1931), \emph{Schriftsteller, Mediziner}!Schleier der Beatrice. Schauspiel in fuenf Akten1900-12-01@\strich\emph{Der Schleier der Beatrice. Schauspiel in fünf Akten} {[}1900-12-01{]}|pwk} war am 21. 2. 1901 bei \emph{S.
                     Fischer}\orgindex{S. Fischer Verlag@S. Fischer Verlag|pwk} erschienen, \emph{Frau Bertha
                     Garlan}\pwindex{Schnitzler, Arthur 15.05.1862 – 21.10.1931@\textsc{Schnitzler, Arthur} (15.05.1862 – 21.10.1931), \emph{Schriftsteller, Mediziner}!Frau Bertha Garlan. Roman15.1.1901 – 15.3.1901@\strich\emph{Frau Bertha Garlan. Roman} {[}15.1.1901 – 15.3.1901{]}|pwk} am 13. 4. 1901.}}}\label{K_L03064-1h} erhielt! Dank
               endlich für Deine Bemühungen bei \textsc{Bahr\pwindex{Bahr, Hermann 19.07.1863 – 15.01.1934@\textsc{Bahr, Hermann} (19.07.1863 – 15.01.1934), \emph{Schriftsteller, Kritiker}|pw}} in Sachen des Stückes \label{K_L03064-2v}\edtext{»Gewitter\pwindex{Ostrovskij, Alexander N. 12.04.1823 – 14.06.1886@\textsc{Ostrovskij, Alexander N.} (12.04.1823 – 14.06.1886), \emph{Schriftsteller}!Gewitter. Drama in fuenf Akten1860@\strich\emph{Gewitter. Drama in fünf Akten} {[}1860{]}|pwu}«}{\lemma{\textnormal{\emph{»Gewitter«}}}\Cendnote{\textnormal{Unklarer Bezug. Das Fehlen einer unmittelbaren Bezugnahme
                  verwirrt auch, weil das auf ein verlorenes Korrespondenzstück Goldmann\pwindex{Goldmann, Paul 31.01.1865 – 25.09.1935@\textsc{Goldmann, Paul} (31.01.1865 – 25.09.1935), \emph{Schriftsteller, Journalist}|pwk}s verweisen dürfte. Möglicherweise handelte es
                  sich um den Fünfakter \emph{Gewitter}\pwindex{Ostrovskij, Alexander N. 12.04.1823 – 14.06.1886@\textsc{Ostrovskij, Alexander N.} (12.04.1823 – 14.06.1886), \emph{Schriftsteller}!Gewitter. Drama in fuenf Akten1860@\strich\emph{Gewitter. Drama in fünf Akten} {[}1860{]}|pwk} von Alexander Ostrowski\pwindex{Ostrovskij, Alexander N. 12.04.1823 – 14.06.1886@\textsc{Ostrovskij, Alexander N.} (12.04.1823 – 14.06.1886), \emph{Schriftsteller}|pwk}, oder ein noch
                  unveröffentlichtes Werk einer unbekannten Person.}}}\label{K_L03064-2h}!\pend
           \pstart
           Ich freue mich, daß Du wieder glücklich \label{K_L03064-3v}\edtext{daheim}{\lemma{\textnormal{\emph{daheim}}}\Cendnote{\textnormal{Schnitzler\pwindex{Schnitzler, Arthur 15.05.1862 – 21.10.1931@\textsc{Schnitzler, Arthur} (15.05.1862 – 21.10.1931), \emph{Schriftsteller, Mediziner}|pwk} war am 19. 4. 1901 von seiner
                     Italien\oindex{Italien@\textbf{Italien}|pwk}reise zurückgekehrt.}}}\label{K_L03064-3h} biſt.
               Auch \label{K_L03064-4v}\edtext{die andere Nachricht}{\lemma{\textnormal{\emph{die andere Nachricht}}}\Cendnote{\textnormal{Olga\pwindex{Schnitzler, Olga 17.01.1882 – 13.01.1970@\textsc{Schnitzler, Olga} (17.01.1882 – 13.01.1970), \emph{Schauspielerin, Sängerin}|pwk} war mit dem gemeinsamen Kind schwanger.
                  Am 10. 5. 1901
                  musste die Schwangerschaft beendet werden.}}}\label{K_L03064-4h} iſt \strikeout{\textcolor{gray}{rec}h\textcolor{gray}{t}} eine erfreuliche. Eine Frau und ein Kind, – das iſt wohl die \strikeout{Löſ\textcolor{gray}{un}} Erklärung für das, was die Natur mit uns vorhat; und demjenigen, der danach
               handelt, ſpendet ſie Glücksgefühle, wie immer, wenn man ihre geheimen Abſichten
               erräth. Das iſt der Weg zum Glück: die geheimen Abſichten der Natur errathen. Ich
               wünſche Dir einen Sohn.\pend
           \pstart
           {\pb}Daß man mit ſeiner Geliebten nach Italien\oindex{Italien@\textbf{Italien}|pw} gehen muß, iſt ſelbſtverſtändlich. Ich möchte wiſſen,
               was Italien\oindex{Italien@\textbf{Italien}|pw} ſonſt \strikeout{\textcolor{gray}{×}\-\textcolor{gray}{×}} für einen Sinn hat, als den: eine Umgebung für eine Liebe zu ſein. Darum
               beneide ich Dich nicht um Deine \label{K_L03064-5v}\edtext{Rom\oindex{Rom@\textbf{Rom}|pw}fahrt}{\lemma{\textnormal{\emph{Romfahrt}}}\Cendnote{\textnormal{siehe Paul Goldmann an Arthur Schnitzler, 6. 4. [1901]}}}\label{K_L03064-5h}. Wohl aber beneide ich Dich um Deine \label{K_L03064-6v}\edtext{Sehnſucht nach \textsc{Olga\pwindex{Schnitzler, Olga 17.01.1882 – 13.01.1970@\textsc{Schnitzler, Olga} (17.01.1882 – 13.01.1970), \emph{Schauspielerin, Sängerin}|pw}}}{\lemma{\textnormal{\emph{Sehnſucht nach Olga}}}\Cendnote{\textnormal{siehe A. S.: \emph{Tagebuch}, 17. 4. 1901}}}\label{K_L03064-6h}. Ich darf mich nach Keiner ſehnen.\pend
           \pstart
           Der \label{K_L03064-7v}\edtext{Artikel\pwindex{Brandes, Georg 04.02.1842 – 19.02.1927@\textsc{Brandes, Georg} (04.02.1842 – 19.02.1927)!Skikkelser og Tanker. Arthur Schnitzler1901-04-09@\strich\emph{Skikkelser og Tanker. Arthur Schnitzler} {[}1901-04-09{]}|pwv}}{\lemma{\textnormal{\emph{Artikel}}}\Cendnote{\textnormal{Georg Brandes\pwindex{Brandes, Georg 04.02.1842 – 19.02.1927@\textsc{Brandes, Georg} (04.02.1842 – 19.02.1927)|pwk}: \emph{Skikkelser og Tanker. Arthur Schnitzler}\pwindex{Brandes, Georg 04.02.1842 – 19.02.1927@\textsc{Brandes, Georg} (04.02.1842 – 19.02.1927)!Skikkelser og Tanker. Arthur Schnitzler1901-04-09@\strich\emph{Skikkelser og Tanker. Arthur Schnitzler} {[}1901-04-09{]}|pwk}. In: \emph{Politiken}\pwindex{?? Werk@Nicht ermittelte Verfasserinnen und Verfasser!Politiken1. 1. 1884@\emph{Politiken} {[}1. 1. 1884{]}|pwk}, Nr. 98, 9. 4. 1901,
                     S. 1. Es gibt ein nicht überliefertes Korrespondenzstück Goldmann\pwindex{Goldmann, Paul 31.01.1865 – 25.09.1935@\textsc{Goldmann, Paul} (31.01.1865 – 25.09.1935), \emph{Schriftsteller, Journalist}|pwk}s, in dem er Schnitzler\pwindex{Schnitzler, Arthur 15.05.1862 – 21.10.1931@\textsc{Schnitzler, Arthur} (15.05.1862 – 21.10.1931), \emph{Schriftsteller, Mediziner}|pwk} den Artikel\pwindex{Brandes, Georg 04.02.1842 – 19.02.1927@\textsc{Brandes, Georg} (04.02.1842 – 19.02.1927)!Skikkelser og Tanker. Arthur Schnitzler1901-04-09@\strich\emph{Skikkelser og Tanker. Arthur Schnitzler} {[}1901-04-09{]}|pwkv} übersandte (vgl. Arthur Schnitzler an Georg Brandes, 25. 4. 1901).}}}\label{K_L03064-7h} von \textsc{Brandes\pwindex{Brandes, Georg 04.02.1842 – 19.02.1927@\textsc{Brandes, Georg} (04.02.1842 – 19.02.1927)|pw}} über Dich war recht ſchleuderhaft geſchrieben. \textsc{Brandes\pwindex{Brandes, Georg 04.02.1842 – 19.02.1927@\textsc{Brandes, Georg} (04.02.1842 – 19.02.1927)|pw}} war dieſer Tage in Berlin\oindex{Berlin@\textbf{Berlin}|pw} – in merkwürdiger
               Stimmung: gezwungen heiter, manchmal verſtört. Plötzlich iſt er abgereiſt. Ich habe
               ihn ſehr gern. Er hat etwas ſo Feines und Gütiges\substVorne{}\textsuperscript{\textcolor{gray}{!}}\substDazwischen{}.\substHinten{}\pend
           \pstart
           \label{K_L03064-8v}\edtext{Sommerpläne}{\lemma{\textnormal{\emph{Sommerpläne}}}\Cendnote{\textnormal{Schnitzler\pwindex{Schnitzler, Arthur 15.05.1862 – 21.10.1931@\textsc{Schnitzler, Arthur} (15.05.1862 – 21.10.1931), \emph{Schriftsteller, Mediziner}|pwk} und Goldmann\pwindex{Goldmann, Paul 31.01.1865 – 25.09.1935@\textsc{Goldmann, Paul} (31.01.1865 – 25.09.1935), \emph{Schriftsteller, Journalist}|pwk} trafen sich im August 1901 jedenfalls mehrmals in Südtirol\oindex{Suedtirol@\textbf{Südtirol}|pwk}, konkret am 7. 8. 1901 in Welsberg\oindex{Welsberg-Taisten@\textbf{Welsberg-Taisten}|pwk}, am 13. 8. 1901 in Bozen\oindex{Bozen@\textbf{Bozen}|pwk} und zwischen 18. 8. 1901 und 29. 8. 1901 noch
                  einmal in Welsberg\oindex{Welsberg-Taisten@\textbf{Welsberg-Taisten}|pwk}. Danach reiste Goldmann\pwindex{Goldmann, Paul 31.01.1865 – 25.09.1935@\textsc{Goldmann, Paul} (31.01.1865 – 25.09.1935), \emph{Schriftsteller, Journalist}|pwk} mit Schnitzler\pwindex{Schnitzler, Arthur 15.05.1862 – 21.10.1931@\textsc{Schnitzler, Arthur} (15.05.1862 – 21.10.1931), \emph{Schriftsteller, Mediziner}|pwk} nach Wien\oindex{Wien@\textbf{Wien}|pwk}
                  zurück und blieb dort wohl noch ein paar Tage.}}}\label{K_L03064-8h}? Wie Du willſt. Mir {\pb}iſt Alles eins. Ich fahre weg oder bleibe auch zu
               Hauſe. Bin auf dem Tiefpunkt aller menſchlichen Verfaſſung angelangt{\dotsfour}\pend
           \pstart
           Grüße an die Grünethorgaſſe\oindex{Gruenentorgasse@\textbf{Grünentorgasse}|pw}\pwindex{Schnitzler, Olga 17.01.1882 – 13.01.1970@\textsc{Schnitzler, Olga} (17.01.1882 – 13.01.1970), \emph{Schauspielerin, Sängerin}|pwv}\pwindex{Steinrueck, Elisabeth 19.11.1885 – 07.04.1920@\textsc{Steinrück, Elisabeth} (19.11.1885 – 07.04.1920)|pwv}, Grüße an Dich! {\\[\baselineskip]}Von Herzen {\\[\baselineskip]}Dein {\\[\baselineskip]}\spacefill\mbox{Paul Goldmnn}\pend
           \leftskip=0em{}
         
         \endnumbering\mylabel{h}\end{ledgroupsized}  \newcommand{\dateiname}{L03064}\newcommand{\titel}{Paul Goldmann an Arthur Schnitzler, 26. 4. [1901]}\newcommand{\editorInnen}{Martin Anton Müller und Laura Untner}%% latex-leseansicht-abspann.tex
%% Abspann für die Leseansicht.
%% Der Schalter \ifkorrekturansicht ist bereits durch den Vorspann gesetzt.

%% latex-abspann.tex
%% Gemeinsamer Abspann für Korrekturansicht und Leseansicht.
%% Setzt den Schalter \ifkorrekturansicht voraus (gesetzt in den
%% einbindenden Dateien latex-korrekturansicht-abspann.tex bzw.
%% latex-leseansicht-abspann.tex).
%% ---------------------------------------------------------------

\normalsize

% Das esempio-Environment wird nur in der Leseansicht benötigt
\ifkorrekturansicht\else
\newenvironment{esempio}[3]%
{
    \vspace{1.5ex}
    \rlap{\underline{#1}}
    \par
    \setlength{\parindent}{0cm}
    \nopagebreak
    \leftskip=#2cm
    \rightskip=#3cm
}
{
    \par
}
\fi

\doendnotes{C}
\bigskip
\vfill

\clearpage

\footnotesize

\ifkorrekturansicht
  \lohead{\textsc{register}}
\fi

% theindex-Environment neu definieren ohne reledmac
\makeatletter
\renewenvironment{theindex}{%
  \ifkorrekturansicht
    \section*{\indexname}%
  \else
    \subsubsection*{Index der erwähnten Entitäten}%
  \fi
  \setlength{\parindent}{0pt}%
  \setlength{\parskip}{0pt plus 0.3pt}%
  \let\item\@idxitem
}{%
  \ifkorrekturansicht\clearpage\fi
}
\makeatother

\IfFileExists{\jobname-pw.ind}{\input{\jobname-pw.ind}}{}

% Quellenangabe nur in der Leseansicht
\ifkorrekturansicht\else
% Fallback-Definitionen, falls die .tex-Datei \titel etc. nicht gesetzt hat
\providecommand{\titel}{}
\providecommand{\editorInnen}{}
\providecommand{\dateiname}{\jobname}

\vspace{3cm}

\vfill

\footnotesize
\textsc{Quelle}: \titel. Herausgegeben von {\editorInnen}. In: \emph{Arthur Schnitzler: Briefwechsel mit Autorinnen und Autoren}.
 Digitale Edition, https://schnitzler-briefe.acdh.oeaw.ac.at/{\dateiname}.html (Stand \today)
\fi

\end{document}


      