%% latex-leseansicht-vorspann.tex
%% Vorspann für die Leseansicht.
%% Lädt die gemeinsame Datei latex-vorspann.tex mit nicht gesetztem Schalter.

\newif\ifkorrekturansicht
\korrekturansichtfalse

\input{../tex-inputs/latex-vorspann}

\begin{center}
            \textcolor{red}{ENTWURF, NICHT FERTIG KORRIGIERT}
                      \end{center}
            
         
         \renewcommand{\erwaehntePersonen}{Personen: Frieda Pollak, Ottilie Salten, Heinrich Schnitzler, Lili Schnitzler}
         \renewcommand{\erwaehnteOrte}{Orte: Böhmen, Olomouc, Sternwartestraße, Wien}
         \renewcommand{\erwaehnteWerke}{Werke: Der Hund von Florenz}
               \section[Felix Salten an Arthur Schnitzler, 18. 3. 1921]{ Felix Salten an Arthur Schnitzler, 18. 3. 1921}\nopagebreak\mylabel{v}\rehead{ }\begin{ledgroupsized}[t]{13cm}\normalsize\beginnumbering \toendnotes[C]{\smallbreak\pagebreak[2]} \Standort{CUL, Schnitzler, B 89, B 2.}
\physDesc{Postkarte, 821 Zeichen
\newline{}Handschrift: schwarze Tinte, lateinische Kurrent
\newline{}Versand: Stempel: »\nobreak{}\oindex{Olomouc@\textbf{Olomouc}|pwk}Olomouc 3 C.S.P., 19. III. 21, 9\textcolor{gray}{–}10N\nobreak{}«.  
\newline{}Schnitzler: mit Bleistift Jahreszahl ergänzt:
                                    »21.« 
\newline{}Ordnung: 1) mit Bleistift von Frieda
                                    Pollak\pwindex{Pollak, Frieda 08.12.1881 – 13.07.1937@\textsc{Pollak, Frieda} (08.12.1881 – 13.07.1937), \emph{Sekretärin}|pw} (?) mit dem Buchstaben »A«
                                 (Abgeschrieben/Abschrift) gekennzeichnet  2) mit Bleistift von unbekannter Hand nummeriert:
                                    »282«}\toendnotes[C]{\smallbreak}\pstart{}{\pb}Herrn\pend{}\pstart{}D\textsuperscript{r} Arthur Schnitzler\pend{}\pstart{}Wien\oindex{Wien@\textbf{Wien}|pw}\pend{}\pstart{}XVIII. Sternwartestraße 71\oindex{Sternwartestrasse@\textbf{Sternwartestraße}|pw}\pend{}{\bigskip}\pstart
           \raggedleft{}{\pb}Olmütz\oindex{Olomouc@\textbf{Olomouc}|pw}, 18. 3.\pend
           \pstart{}Lieber,\pend\pstart
           hoffentlich \label{K_L03569-1v}\edtext{haben Sie von Otti\pwindex{Salten, Ottilie 07.03.1868 – 22.06.1942@\textsc{Salten, Ottilie} (07.03.1868 – 22.06.1942), \emph{Schauspielerin}|pw} schon}{\lemma{\textnormal{\emph{haben Sie von Otti schon}}}\Cendnote{\textnormal{Er bekam das Manuskript
               von \emph{Der Hund von Florenz}\pwindex{Salten, Felix 06.09.1869 – 08.10.1945@\textsc{Salten, Felix} (06.09.1869 – 08.10.1945), \emph{Schriftsteller, Journalist}!Hund von Florenz1923@\strich\emph{Der Hund von Florenz} {[}1923{]}|pwk}
            erst am 21. 3. 1921.}}}\label{K_L03569-1h} das
               Mscpt. meiner Erzählung\pwindex{Salten, Felix 06.09.1869 – 08.10.1945@\textsc{Salten, Felix} (06.09.1869 – 08.10.1945), \emph{Schriftsteller, Journalist}!Hund von Florenz1923@\strich\emph{Der Hund von Florenz} {[}1923{]}|pwv}. Wenn nicht,
               bitte, verlangen Sie’s. Ich hoffe sehr, dass Sie wohl und mehr und mehr ruhig sind
               und dass Ihnen das Arbeiten von der Hand geht! Und ich hoffe, dass Ihnen der Frühling
               so stark hilft, wie er kann. Das viele Umherfahren, das ich jetzt absolvieren muß,
               meist in Bu{\geminationm}el-Zügen, ist zu nicht angenehm, aber das
               Anschauen der milden, böhmischen\oindex{Boehmen@\textbf{Böhmen}|pw} Landschaft, die
               jetzt, bei dem schönen Wetter, wie neu aussieht, beruhigt so angenehm. Auch ist das
               die vierte Stadt, in der ich seit Sonntag lese. Noch vier folgen. Es geht gut. Ich
               bin zwischendurch doch viel allein, was wohltut, denke viel und denke natürlich auch
               sehr viel an Sie! \pend
           \pstart
           Alles Herzliche Ihnen und den Kindern\pwindex{Schnitzler, Heinrich 09.08.1902 – 12.07.1982@\textsc{Schnitzler, Heinrich} (09.08.1902 – 12.07.1982), \emph{Regisseur, Schauspieler}|pwv}\pwindex{Schnitzler, Lili 13.09.1909 – 26.07.1928@\textsc{Schnitzler, Lili} (13.09.1909 – 26.07.1928)|pwv}. {\\[\baselineskip]}Ihr {\\[\baselineskip]}\spacefill\mbox{Felix Salten}\pend
           \leftskip=0em{}
         
         \endnumbering\mylabel{h}\end{ledgroupsized}\begin{anhang}\end{anhang}\newcommand{\dateiname}{L03569}\newcommand{\titel}{Felix Salten an Arthur Schnitzler, 18. 3. 1921}\newcommand{\editorInnen}{Martin Anton Müller und Laura Untner}%% latex-leseansicht-abspann.tex
%% Abspann für die Leseansicht.
%% Der Schalter \ifkorrekturansicht ist bereits durch den Vorspann gesetzt.

%% latex-abspann.tex
%% Gemeinsamer Abspann für Korrekturansicht und Leseansicht.
%% Setzt den Schalter \ifkorrekturansicht voraus (gesetzt in den
%% einbindenden Dateien latex-korrekturansicht-abspann.tex bzw.
%% latex-leseansicht-abspann.tex).
%% ---------------------------------------------------------------

\normalsize

% Das esempio-Environment wird nur in der Leseansicht benötigt
\ifkorrekturansicht\else
\newenvironment{esempio}[3]%
{
    \vspace{1.5ex}
    \rlap{\underline{#1}}
    \par
    \setlength{\parindent}{0cm}
    \nopagebreak
    \leftskip=#2cm
    \rightskip=#3cm
}
{
    \par
}
\fi

\doendnotes{C}
\bigskip
\vfill

\clearpage

\footnotesize

\ifkorrekturansicht
  \lohead{\textsc{register}}
\fi

% theindex-Environment neu definieren ohne reledmac
\makeatletter
\renewenvironment{theindex}{%
  \ifkorrekturansicht
    \section*{\indexname}%
  \else
    \subsubsection*{Index der erwähnten Entitäten}%
  \fi
  \setlength{\parindent}{0pt}%
  \setlength{\parskip}{0pt plus 0.3pt}%
  \let\item\@idxitem
}{%
  \ifkorrekturansicht\clearpage\fi
}
\makeatother

\IfFileExists{\jobname-pw.ind}{\input{\jobname-pw.ind}}{}

% Quellenangabe nur in der Leseansicht
\ifkorrekturansicht\else
% Fallback-Definitionen, falls die .tex-Datei \titel etc. nicht gesetzt hat
\providecommand{\titel}{}
\providecommand{\editorInnen}{}
\providecommand{\dateiname}{\jobname}

\vspace{3cm}

\vfill

\footnotesize
\textsc{Quelle}: \titel. Herausgegeben von {\editorInnen}. In: \emph{Arthur Schnitzler: Briefwechsel mit Autorinnen und Autoren}.
 Digitale Edition, https://schnitzler-briefe.acdh.oeaw.ac.at/{\dateiname}.html (Stand \today)
\fi

\end{document}


      