%% latex-korrekturansicht-vorspann.tex
%% Vorspann für die Korrekturansicht.
%% Lädt die gemeinsame Datei latex-vorspann.tex mit gesetztem Schalter.

\newif\ifkorrekturansicht
\korrekturansichttrue

\input{../tex-inputs/latex-vorspann}


\section[ Paul Goldmann an Arthur Schnitzler, 12. 8. {[}1902{]}]{L03219 Paul Goldmann an Arthur Schnitzler, 12. 8. {[}1902{]}}
\nopagebreak\mylabel{L03219v}
\rehead{ }\normalsize\beginnumbering\briefempfaengerindex{Schnitzler, Arthur@\textsc{Schnitzler, Arthur}!zzzGoldmann, Paul@\emph{von Paul Goldmann}!1902-08-122@{12. 8. {[}1902{]}}|(be}
\toendnotes[C]{\smallbreak\pagebreak[2]}\Standort{DLA, A:Schnitzler, HS.NZ85.1.3172.}
\physDesc{Brief, 1 Blatt, 4 Seiten, 1244 Zeichen
\newline{}Handschrift: schwarze Tinte, deutsche Kurrent
\newline{}Schnitzler: mit Bleistift das Jahr »902« vermerkt }\toendnotes[C]{\smallbreak}
\pstart
           \centering{}{\pb}\textcolor{gray}{\textbf{\textsc{Grand Hôtel {\kaufmannsund}
                           Kurhaus\oindex{Grand Hotel und Kurhaus Muerren@\textbf{Grand Hotel und Kurhaus Mürren}, \emph{Hotel (K.HTL)}|pw}, Mürren\oindex{Muerren@\textbf{Mürren}, \emph{P.PPL}|pw}}}}\pend
           
\pstart
           \centering{}\textcolor{gray}{\textbf{(\begin{otherlanguage}{french}SUISSE\oindex{Schweiz@\textbf{Schweiz}, \emph{A.PCLI}|pw}\end{otherlanguage})}}\pend
           
\pstart
           12. Auguſt\pend
           
\pstart{}Mein lieber Freund,\pend\vspace{0.5em}
\pstart
           Nochmals innigſte Glückwünſche. Nun haſt Du auch einen \label{K_L03219-1v}\edtext{Sohn\pwindex{Schnitzler, Heinrich 09.08.1902 – 12.07.1982@\textsc{Schnitzler, Heinrich} (09.08.1902 – 12.07.1982), \emph{Regisseur/Regisseurin, Schauspieler/Schauspielerin}|pwv}}{\lemma{\textnormal{\emph{Sohn}}}\Cendnote{\textnormal{Heinrich Schnitzler\pwindex{Schnitzler, Heinrich 09.08.1902 – 12.07.1982@\textsc{Schnitzler, Heinrich} (09.08.1902 – 12.07.1982), \emph{Regisseur/Regisseurin, Schauspieler/Schauspielerin}|pwk}, geboren am 9. 8. 1902 in der Hinterbrühl\oindex{Hinterbruehl@\textbf{Hinterbrühl}, \emph{P.PPLA3}|pwk}}}}\label{K_L03219-1}. So kommt Alles. Ich wünſche Deinem Sohn\pwindex{Schnitzler, Heinrich 09.08.1902 – 12.07.1982@\textsc{Schnitzler, Heinrich} (09.08.1902 – 12.07.1982), \emph{Regisseur/Regisseurin, Schauspieler/Schauspielerin}|pwv} all’ das Gute und Liebe, das ich Dir ſelbſt wünſche, –
               und das iſt ſehr viel. Wie wird er heißen? Sieht er ſchon Jemandem ähnlich? Was macht
               die Mutter\pwindex{Schnitzler, Olga 17.01.1882 – 13.01.1970@\textsc{Schnitzler, Olga} (17.01.1882 – 13.01.1970), \emph{Schauspieler/Schauspielerin, Sänger/Sängerin}|pwv}? Sage ihr, bitte,
               in meinem Namen alles Herzliche.\pend
           
\pstart
           Über Deine literariſche {\pb}Produktivität freue ich
               mich nicht weniger. Von dem Junggeſellenſtück\pwindex{einsame Weg. Schauspiel in fuenf Akten@\emph{Der einsame Weg. Schauspiel in fünf Akten}|pwv} verſpreche ich mir ſehr viel. Auf das \label{K_L03219-2v}\edtext{Alt-Wien\oindex{Wien@\textbf{Wien}, \emph{A.ADM2}|pw}er Stück\pwindex{junge Medardus. Dramatische Historie in einem Vorspiel und fuenf Aufzuegen@\emph{Der junge Medardus. Dramatische Historie in einem Vorspiel und fünf Aufzügen}|pwv}}{\lemma{\textnormal{\emph{Alt-Wiener Stück}}}\Cendnote{\textnormal{Schnitzler hatte zwischen 12. 7. 1902 und 1. 8. 1902 die erste
                  Fassung des Stückes ausgearbeitet, das zu \emph{Der junge
                     Medardus}\pwindex{junge Medardus. Dramatische Historie in einem Vorspiel und fuenf Aufzuegen@\emph{Der junge Medardus. Dramatische Historie in einem Vorspiel und fünf Aufzügen}|pwk} wurde. Zum Alt-Wien\oindex{Wien@\textbf{Wien}, \emph{A.ADM2}|pwk}er Stoff siehe auch
                     Paul Goldmann an Arthur Schnitzler, 24. 8. [1898].}}}\label{K_L03219-2} bin ich
               beſonders neugierig; auch da erwarte ich mir etwas \strikeout{beſond\textcolor{gray}{e}} beſonders Feines. Wie haſt Du über die \label{K_L03219-3v}\edtext{»\textsc{Beatrice\pwindex{Schleier der Beatrice. Schauspiel in fuenf Akten@\emph{Der Schleier der Beatrice. Schauspiel in fünf Akten}|pw}}«}{\lemma{\textnormal{\emph{»Beatrice«}}}\Cendnote{\textnormal{Siehe Paul Goldmann an Arthur Schnitzler, 14. 7. [1902].
               }}}\label{K_L03219-3} entſchieden? Im »Schillertheater\oindex{Schiller-Theater@\textbf{Schiller-Theater}, \emph{Theater (K.THE)}|pw}« wird
               ſie aller Wahrſcheinlichkeit nach beſſer geſpielt werden, als im »\label{K_L03219-4v}\edtext{Deutſchen\oindex{Deutsches Theater Berlin@\textbf{Deutsches Theater Berlin}, \emph{Theater (K.THE)}|pwv}}{\lemma{\textnormal{\emph{Deutſchen}}}\Cendnote{\textnormal{Deutsches Theater\oindex{Deutsches Theater Berlin@\textbf{Deutsches Theater Berlin}, \emph{Theater (K.THE)}|pwk}}}}\label{K_L03219-4}«, aber das Schillertheater\orgindex{Schiller-Theater@Schiller-Theater|pw} hat doch nicht
               das große literariſche Publikum und iſt ein Provinztheater in der \strikeout{\textcolor{gray}{H}}{ }Hauptſtadt\oindex{Berlin@\textbf{Berlin}, \emph{P.PPLC}|pwv}.\pend
           
\pstart
           Bitte, ſchreib’ mir bald {\pb}einige Einzelheiten über
               das Ereigniß in der Hinterbrühl\oindex{Hinterbruehl@\textbf{Hinterbrühl}, \emph{P.PPLA3}|pw}, – an meine Berlin\oindex{Berlin@\textbf{Berlin}, \emph{P.PPLC}|pw}er Adreſſe. Ich werde morgen{ }hier\oindex{Muerren@\textbf{Mürren}, \emph{P.PPL}|pwv} von meinem Onkel\pwindex{Mamroth, Fedor 21.02.1851 – 25.06.1907@\textsc{Mamroth, Fedor} (21.02.1851 – 25.06.1907), \emph{Journalist/Journalistin, Kritiker/Kritikerin}|pwv} abgeholt und weiß noch
               nicht, wohin wir wandern werden. Wir ſitzen hier\oindex{Muerren@\textbf{Mürren}, \emph{P.PPL}|pwv} ſeit zwei Tagen im dichten Schneegeſtöber.
               Weihnachtswetter im Auguſt. Hände und Füße ſind mir ſtarr vor Kälte; das iſt der \introOben{}Grund\introOben{}{ }\substVorne{}\textsuperscript{Brief}\substDazwischen{}\strikeout{G\textcolor{gray}{run}}\substHinten{}, weshalb \strikeout{der} dieſer Brief nicht länger
               wird.\pend
           
\pstart
           {\pb}Tauſend Grüße! {\\[\baselineskip]}Dein {\\[\baselineskip]}\spacefill\mbox{Paul Goldmann.}\pend
           \leftskip=0em{}\selectlanguage{ngerman}\endnumbering\briefempfaengerindex{Schnitzler, Arthur@\textsc{Schnitzler, Arthur}!zzzGoldmann, Paul@\emph{von Paul Goldmann}!1902-08-122@{12. 8. {[}1902{]}}|)be}\mylabel{L03219h}  \normalsize

\doendnotes{C}
\bigskip
\vfill

\clearpage

\footnotesize

\lohead{\textsc{register}}

% Definiere theindex-Environment komplett neu ohne reledmac
\makeatletter
\renewenvironment{theindex}{%
  \section*{\indexname}%
  \setlength{\parindent}{0pt}%
  \setlength{\parskip}{0pt plus 0.3pt}%
  \let\item\@idxitem
}{%
  \clearpage
}
\makeatother

\IfFileExists{\jobname-pw.ind}{\input{\jobname-pw.ind}}{}

\end{document}

      