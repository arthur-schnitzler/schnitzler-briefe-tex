%% latex-leseansicht-vorspann.tex
%% Vorspann für die Leseansicht.
%% Lädt die gemeinsame Datei latex-vorspann.tex mit nicht gesetztem Schalter.

\newif\ifkorrekturansicht
\korrekturansichtfalse

\input{../tex-inputs/latex-vorspann}

\begin{center}
            \textcolor{red}{ENTWURF, NICHT FERTIG KORRIGIERT}
                      \end{center}
            
         \renewcommand{\erwaehnteOrte}{Orte: Mürren, Schweiz, Wien}
         \renewcommand{\erwaehnteWerke}{Werke: Der Schleier der Beatrice. Schauspiel in fünf Akten}
               \section[ Paul Goldmann an Arthur Schnitzler, 12. 8. {[}1902{]}]{ Paul Goldmann an Arthur Schnitzler, 12. 8. {[}1902{]}}\nopagebreak\mylabel{v}\rehead{ }\begin{ledgroupsized}[t]{13cm}\normalsize\beginnumbering \toendnotes[C]{\smallbreak\pagebreak[2]} \Standort{DLA, A:Schnitzler, HS.NZ85.1.3172.}
\physDesc{Brief, 1 Blatt, 4 Seiten
\newline{}Handschrift: schwarze Tinte, deutsche Kurrent
\newline{}Schnitzler: mit Bleistift das Jahr »{[}1{]}902«
                                            vermerkt }\toendnotes[C]{\smallbreak}\pstart
           \noindent{}\centering{}{\pb}\textcolor{gray}{\textbf{GRAND HÔTEL & KURHAUS\textcolor{red}{\textsuperscript{\textbf{KEY}}}, MÜRREN\oindex{Muerren@\textbf{Mürren}|pw}}}\pend
           \pstart
           \noindent{}\centering{}\textcolor{gray}{\textbf{(\begin{otherlanguage}{french}SUISSE\oindex{Schweiz@\textbf{Schweiz}|pw}\end{otherlanguage})}}\pend
           \pstart
           12. Auguſt\pend
           \pstart{}Mein lieber Freund,\pend\pstart
           Nochmals innigſte Glückwünſche. Nun haſt Du auch einen Sohn\textcolor{red}{\textsuperscript{\textbf{KEY}}}. So kommt Alles. Ich wünſche Deinem Sohn\textcolor{red}{\textsuperscript{\textbf{KEY}}} all’ das Gute und Liebe, das ich Dir ſelbſt wünſche, – und das
                    iſt ſehr viel. Wie wied er heißen? Sieht er ſchon Jemandem ähnlich? Was macht
                    die Mutter\textcolor{red}{\textsuperscript{\textbf{KEY}}}? Sage ihr, bitte, in meinem Namen alles
                    Herzliche. \pend
           \pstart
           Über Deine literariſche {\pb} Produktivität
                    freue ich mich nicht weniger. Von dem Junggeſellenſtück\textcolor{red}{\textsuperscript{\textbf{KEY}}} verſpreche ich mir ſehr viel. Auf das Alt-Wien\textcolor{red}{\textsuperscript{\textbf{KEY}}}er Stück\textcolor{red}{\textsuperscript{\textbf{KEY}}} bin ich
                    beſonders neugierig; auch da erwarte ich mir etwas \strikeout{beſond\textcolor{gray}{e}} beſonders Feines! Wie haſt Du über die »\textsc{Beatrice\pwindex{Schnitzler, Arthur 15.05.1862 – 21.10.1931@\textsc{Schnitzler, Arthur} (15.05.1862 – 21.10.1931), \emph{Schriftsteller, Mediziner}!Schleier der Beatrice. Schauspiel in fuenf Akten1900-12-01@\strich\emph{Der Schleier der Beatrice. Schauspiel in fünf Akten} {[}1900-12-01{]}|pw}}« entſchieden? Im »Schiller\textcolor{red}{\textsuperscript{\textbf{KEY}}}theater\textcolor{red}{\textsuperscript{\textbf{KEY}}}« wird ſie aller Wahrſcheinlichkeit nach
                    beſſer geſpielt werden, als im »\label{XXXXv}\edtext{Deutſchen\textcolor{red}{\textsuperscript{\textbf{KEY}}}[Kommentar: Volkstheater]}{\lemma{\textnormal{\emph{XXXX Lemmafehler}}}\Cendnote{\textnormal{}}}\label{XXXX}«, aber das Schillertheater\textcolor{red}{\textsuperscript{\textbf{KEY}}} hat doch nicht das große literariſche Publikum und iſt
                    ein Provinztheater in der \strikeout{\textcolor{gray}{H}}Hauptſtadt\textcolor{red}{\textsuperscript{\textbf{KEY}}}. Bitte, ſchreib’ mir bald {\pb} einige Einzelheiten über das Ereigniß in der
                        Hinterbrühl\textcolor{red}{\textsuperscript{\textbf{KEY}}}, – an meine Berlin\textcolor{red}{\textsuperscript{\textbf{KEY}}}er Adieſſe. Ich werde morgen hier von meinem Onkel\textcolor{red}{\textsuperscript{\textbf{KEY}}} abgeholt und weiß noch nicht, wohin wir
                    wandern werden. Wir\textcolor{red}{\textsuperscript{\textbf{KEY}}} ſitzen hier\textcolor{red}{\textsuperscript{\textbf{KEY}}} ſeit zwei Tagen im dichten Schneegeſtöber. Weihnachtswetter im
                        Auguſt. Hände und Füße ſind mir ſtarr vor Kälte; das iſt
                        Grund\strikeout{G\textcolor{gray}{run}} der \strikeout{Brief}, weshalb \strikeout{der} dieſer Brief nicht länger wird. {\pb}{\\[\baselineskip]}Tauſend Grüße!\pend
           \leftskip=0em{}\pstart
           {\\[\baselineskip]}Dein\pend
           \leftskip=0em{}\pstart
           {\\[\baselineskip]}\spacefill\mbox{Paul Goldmann. }\pend
           \leftskip=0em{}
         
         \endnumbering\mylabel{h}\end{ledgroupsized}\begin{anhang}\end{anhang}\newcommand{\dateiname}{L03219}\newcommand{\titel}{Paul Goldmann an Arthur Schnitzler, 12. 8. [1902]}\newcommand{\editorInnen}{Martin Anton Müller und Laura Untner}%% latex-leseansicht-abspann.tex
%% Abspann für die Leseansicht.
%% Der Schalter \ifkorrekturansicht ist bereits durch den Vorspann gesetzt.

%% latex-abspann.tex
%% Gemeinsamer Abspann für Korrekturansicht und Leseansicht.
%% Setzt den Schalter \ifkorrekturansicht voraus (gesetzt in den
%% einbindenden Dateien latex-korrekturansicht-abspann.tex bzw.
%% latex-leseansicht-abspann.tex).
%% ---------------------------------------------------------------

\normalsize

% Das esempio-Environment wird nur in der Leseansicht benötigt
\ifkorrekturansicht\else
\newenvironment{esempio}[3]%
{
    \vspace{1.5ex}
    \rlap{\underline{#1}}
    \par
    \setlength{\parindent}{0cm}
    \nopagebreak
    \leftskip=#2cm
    \rightskip=#3cm
}
{
    \par
}
\fi

\doendnotes{C}
\bigskip
\vfill

\clearpage

\footnotesize

\ifkorrekturansicht
  \lohead{\textsc{register}}
\fi

% theindex-Environment neu definieren ohne reledmac
\makeatletter
\renewenvironment{theindex}{%
  \ifkorrekturansicht
    \section*{\indexname}%
  \else
    \subsubsection*{Index der erwähnten Entitäten}%
  \fi
  \setlength{\parindent}{0pt}%
  \setlength{\parskip}{0pt plus 0.3pt}%
  \let\item\@idxitem
}{%
  \ifkorrekturansicht\clearpage\fi
}
\makeatother

\IfFileExists{\jobname-pw.ind}{\input{\jobname-pw.ind}}{}

% Quellenangabe nur in der Leseansicht
\ifkorrekturansicht\else
% Fallback-Definitionen, falls die .tex-Datei \titel etc. nicht gesetzt hat
\providecommand{\titel}{}
\providecommand{\editorInnen}{}
\providecommand{\dateiname}{\jobname}

\vspace{3cm}

\vfill

\footnotesize
\textsc{Quelle}: \titel. Herausgegeben von {\editorInnen}. In: \emph{Arthur Schnitzler: Briefwechsel mit Autorinnen und Autoren}.
 Digitale Edition, https://schnitzler-briefe.acdh.oeaw.ac.at/{\dateiname}.html (Stand \today)
\fi

\end{document}


      