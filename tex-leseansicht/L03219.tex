%% latex-leseansicht-vorspann.tex
%% Vorspann für die Leseansicht.
%% Lädt die gemeinsame Datei latex-vorspann.tex mit nicht gesetztem Schalter.

\newif\ifkorrekturansicht
\korrekturansichtfalse

\input{../tex-inputs/latex-vorspann}


\section[ Paul Goldmann an Arthur Schnitzler, 12. 8. {[}1902{]}]{L03219 Paul Goldmann an Arthur Schnitzler,  12. 8. [1902]}
\nopagebreak\mylabel{L03219v}
\rehead{ }\normalsize\beginnumbering\briefempfaengerindex{Schnitzler, Arthur@\textsc{Schnitzler, Arthur}!zzzGoldmann, Paul@\emph{von Paul Goldmann}!1902-08-122@{12. 8. [1902]}|(be}
\toendnotes[C]{\smallbreak\pagebreak[2]}
\correspDesc{Versand  durch Paul Goldmann am 12. 8. [1902] in Mürren
\newline{}Erhalt  durch Arthur Schnitzler im Zeitraum [13. 8. 1902
                  – 17. 8. 1902?] in Wien}\toendnotes[C]{\smallbreak}
\Standort{DLA, A:Schnitzler, HS.NZ85.1.3172.}
\physDesc{Brief, 1 Blatt, 4 Seiten, 1244 Zeichen
\newline{}Handschrift: schwarze Tinte, deutsche Kurrent
\newline{}Schnitzler: mit Bleistift das Jahr »902« vermerkt }\toendnotes[C]{\smallbreak}
\pstart
           \centering{}{\pb}\textcolor{gray}{\textbf{\textsc{Grand Hôtel {\kaufmannsund}
                           Kurhaus\oindex{Grand Hotel und Kurhaus Mürren@\textbf{Grand Hotel und Kurhaus Mürren}, \emph{Hotel}|pw}, Mürren\oindex{Mürren@\textbf{Mürren}|pw}}}}\pend
           
\pstart
           \centering{}\textcolor{gray}{\textbf{(\begin{otherlanguage}{french}SUISSE\oindex{Schweiz@\textbf{Schweiz}|pw}\end{otherlanguage})}}\pend
           
\pstart
           12. Auguſt\pend
           
\pstart{}Mein lieber Freund,\pend\vspace{0.5em}
\pstart
           Nochmals innigſte Glückwünſche. Nun haſt Du auch einen \label{K_L03219-1v}\edtext{Sohn\pwindex{Schnitzler, Heinrich 9.\,8.\,1902 Hinterbrühl – 12.\,7.\,1982 Wien@\textsc{Schnitzler, Heinrich} (9.\,8.\,1902 Hinterbrühl – 12.\,7.\,1982 Wien), \emph{Regisseur, Schauspieler}|pwv}}{\lemma{\textnormal{\emph{Sohn}}}\Cendnote{\textnormal{Heinrich Schnitzler\pwindex{Schnitzler, Heinrich 9.\,8.\,1902 Hinterbrühl – 12.\,7.\,1982 Wien@\textsc{Schnitzler, Heinrich} (9.\,8.\,1902 Hinterbrühl – 12.\,7.\,1982 Wien), \emph{Regisseur, Schauspieler}|pwk}, geboren am 9. 8. 1902 in der Hinterbrühl\oindex{Hinterbrühl@\textbf{Hinterbrühl}, \emph{Hauptstadt}|pwk}}}}\label{K_L03219-1}. So kommt Alles. Ich wünſche Deinem Sohn\pwindex{Schnitzler, Heinrich 9.\,8.\,1902 Hinterbrühl – 12.\,7.\,1982 Wien@\textsc{Schnitzler, Heinrich} (9.\,8.\,1902 Hinterbrühl – 12.\,7.\,1982 Wien), \emph{Regisseur, Schauspieler}|pwv} all’ das Gute und Liebe, das ich Dir{ }ſelbſt wünſche, –
               und das iſt{ }ſehr viel. Wie wird er heißen? Sieht er{ }ſchon Jemandem ähnlich? Was macht
               die Mutter\pwindex{Schnitzler, Olga 17.\,1.\,1882 Wien – 13.\,1.\,1970 Lugano@\textsc{Schnitzler, Olga} (17.\,1.\,1882 Wien – 13.\,1.\,1970 Lugano), \emph{Schauspielerin, Sängerin}|pwv}? Sage ihr, bitte,
               in meinem Namen alles Herzliche.\pend
           
\pstart
           Über Deine literariſche {\pb}Produktivität freue ich
               mich nicht weniger. Von dem Junggeſellenſtück\pwindex{Schnitzler, Arthur 15.\,5.\,1862 Wien – 21.\,10.\,1931 ebd.@\textsc{Schnitzler, Arthur} (15.\,5.\,1862 Wien – 21.\,10.\,1931 ebd.), \emph{Schriftsteller, Mediziner}!einsame Weg. Schauspiel in fünf Akten@\strich\emph{Der einsame Weg. Schauspiel in fünf Akten}|pwv} verſpreche ich mir{ }ſehr viel. Auf das \label{K_L03219-2v}\edtext{Alt-Wien\oindex{Wien@\textbf{Wien}, \emph{Verwaltungsgebiet}|pw}er Stück\pwindex{Schnitzler, Arthur 15.\,5.\,1862 Wien – 21.\,10.\,1931 ebd.@\textsc{Schnitzler, Arthur} (15.\,5.\,1862 Wien – 21.\,10.\,1931 ebd.), \emph{Schriftsteller, Mediziner}!junge Medardus. Dramatische Historie in einem Vorspiel und fünf Aufzügen@\strich\emph{Der junge Medardus. Dramatische Historie in einem Vorspiel und fünf Aufzügen}|pwv}}{\lemma{\textnormal{\emph{Alt-Wiener Stück}}}\Cendnote{\textnormal{Schnitzler hatte zwischen 12. 7. 1902 und 1. 8. 1902 die erste
                  Fassung des Stückes ausgearbeitet, das zu \emph{Der junge
                     Medardus}\pwindex{Schnitzler, Arthur 15.\,5.\,1862 Wien – 21.\,10.\,1931 ebd.@\textsc{Schnitzler, Arthur} (15.\,5.\,1862 Wien – 21.\,10.\,1931 ebd.), \emph{Schriftsteller, Mediziner}!junge Medardus. Dramatische Historie in einem Vorspiel und fünf Aufzügen@\strich\emph{Der junge Medardus. Dramatische Historie in einem Vorspiel und fünf Aufzügen}|pwk} wurde. Zum Alt-Wien\oindex{Wien@\textbf{Wien}, \emph{Verwaltungsgebiet}|pwk}er Stoff siehe auch
                     XXXX Auszeichnungsfehler: Dokument L02854 nicht gefunden.}}}\label{K_L03219-2} bin ich
               beſonders neugierig; auch da erwarte ich mir etwas \strikeout{beſond\textcolor{gray}{e}} beſonders Feines. Wie haſt Du über die \label{K_L03219-3v}\edtext{»\textsc{Beatrice\pwindex{Schnitzler, Arthur 15.\,5.\,1862 Wien – 21.\,10.\,1931 ebd.@\textsc{Schnitzler, Arthur} (15.\,5.\,1862 Wien – 21.\,10.\,1931 ebd.), \emph{Schriftsteller, Mediziner}!Schleier der Beatrice. Schauspiel in fünf Akten@\strich\emph{Der Schleier der Beatrice. Schauspiel in fünf Akten}|pw}}«}{\lemma{\textnormal{\emph{»Beatrice«}}}\Cendnote{\textnormal{Siehe XXXX Auszeichnungsfehler: Dokument L03213 nicht gefunden.
               }}}\label{K_L03219-3} entſchieden? Im »Schillertheater\oindex{Schiller-Theater@\textbf{Schiller-Theater}, \emph{Theater}|pw}« wird{ }ſie aller Wahrſcheinlichkeit nach beſſer geſpielt werden, als im »\label{K_L03219-4v}\edtext{Deutſchen\oindex{Deutsches Theater Berlin@\textbf{Deutsches Theater Berlin}, \emph{Theater}|pwv}}{\lemma{\textnormal{\emph{Deutschen}}}\Cendnote{\textnormal{Deutsches Theater\oindex{Deutsches Theater Berlin@\textbf{Deutsches Theater Berlin}, \emph{Theater}|pwk}}}}\label{K_L03219-4}«, aber das Schillertheater\orgindex{Schiller-Theater@Schiller-Theater|pw} hat doch nicht
               das große literariſche Publikum und iſt ein Provinztheater in der \strikeout{\textcolor{gray}{H}}{ }Hauptſtadt\oindex{Berlin@\textbf{Berlin}, \emph{Hauptstadt}|pwv}.\pend
           
\pstart
           Bitte,{ }ſchreib’ mir bald {\pb}einige Einzelheiten über
               das Ereigniß in der Hinterbrühl\oindex{Hinterbrühl@\textbf{Hinterbrühl}, \emph{Hauptstadt}|pw}, – an meine Berlin\oindex{Berlin@\textbf{Berlin}, \emph{Hauptstadt}|pw}er Adreſſe. Ich werde morgen{ }hier\oindex{Mürren@\textbf{Mürren}|pwv} von meinem Onkel\pwindex{Mamroth, Fedor 21.\,2.\,1851 Breslau – 25.\,6.\,1907 Frankfurt am Main@\textsc{Mamroth, Fedor} (21.\,2.\,1851 Breslau – 25.\,6.\,1907 Frankfurt am Main), \emph{Journalist, Kritiker}|pwv} abgeholt und weiß noch
               nicht, wohin wir wandern werden. Wir{ }ſitzen hier\oindex{Mürren@\textbf{Mürren}|pwv}{ }ſeit zwei Tagen im dichten Schneegeſtöber.
               Weihnachtswetter im Auguſt. Hände und Füße{ }ſind mir{ }ſtarr vor Kälte; das iſt der \introOben{}Grund\introOben{}{ }\substVorne{}\textsuperscript{Brief}\substDazwischen{}\strikeout{G\textcolor{gray}{run}}\substHinten{}, weshalb \strikeout{der} dieſer Brief nicht länger
               wird.\pend
           
\pstart
           {\pb}Tauſend Grüße! {\\[\baselineskip]}Dein {\\[\baselineskip]}\spacefill\mbox{Paul Goldmann.}\pend
           \leftskip=0em{}\selectlanguage{ngerman}\endnumbering\briefempfaengerindex{Schnitzler, Arthur@\textsc{Schnitzler, Arthur}!zzzGoldmann, Paul@\emph{von Paul Goldmann}!1902-08-122@{12. 8. [1902]}|)be}\mylabel{L03219h}  \newcommand{\dateiname}{L03219}\newcommand{\titel}{Paul Goldmann an Arthur Schnitzler, 12. 8. [1902]}\newcommand{\editorInnen}{Martin Anton Müller und Laura Untner}%% latex-leseansicht-abspann.tex
%% Abspann für die Leseansicht.
%% Der Schalter \ifkorrekturansicht ist bereits durch den Vorspann gesetzt.

%% latex-abspann.tex
%% Gemeinsamer Abspann für Korrekturansicht und Leseansicht.
%% Setzt den Schalter \ifkorrekturansicht voraus (gesetzt in den
%% einbindenden Dateien latex-korrekturansicht-abspann.tex bzw.
%% latex-leseansicht-abspann.tex).
%% ---------------------------------------------------------------

\normalsize

% Das esempio-Environment wird nur in der Leseansicht benötigt
\ifkorrekturansicht\else
\newenvironment{esempio}[3]%
{
    \vspace{1.5ex}
    \rlap{\underline{#1}}
    \par
    \setlength{\parindent}{0cm}
    \nopagebreak
    \leftskip=#2cm
    \rightskip=#3cm
}
{
    \par
}
\fi

\doendnotes{C}
\bigskip
\vfill

\clearpage

\footnotesize

\ifkorrekturansicht
  \lohead{\textsc{register}}
\fi

% theindex-Environment neu definieren ohne reledmac
\makeatletter
\renewenvironment{theindex}{%
  \ifkorrekturansicht
    \section*{\indexname}%
  \else
    \subsubsection*{Index der erwähnten Entitäten}%
  \fi
  \setlength{\parindent}{0pt}%
  \setlength{\parskip}{0pt plus 0.3pt}%
  \let\item\@idxitem
}{%
  \ifkorrekturansicht\clearpage\fi
}
\makeatother

\IfFileExists{\jobname-pw.ind}{\input{\jobname-pw.ind}}{}

% Quellenangabe nur in der Leseansicht
\ifkorrekturansicht\else
% Fallback-Definitionen, falls die .tex-Datei \titel etc. nicht gesetzt hat
\providecommand{\titel}{}
\providecommand{\editorInnen}{}
\providecommand{\dateiname}{\jobname}

\vspace{3cm}

\vfill

\footnotesize
\textsc{Quelle}: \titel. Herausgegeben von {\editorInnen}. In: \emph{Arthur Schnitzler: Briefwechsel mit Autorinnen und Autoren}.
 Digitale Edition, https://schnitzler-briefe.acdh.oeaw.ac.at/{\dateiname}.html (Stand \today)
\fi

\end{document}


