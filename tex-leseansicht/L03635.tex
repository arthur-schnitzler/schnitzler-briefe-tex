%% latex-korrekturansicht-vorspann.tex
%% Vorspann für die Korrekturansicht.
%% Lädt die gemeinsame Datei latex-vorspann.tex mit gesetztem Schalter.

\newif\ifkorrekturansicht
\korrekturansichttrue

\input{../tex-inputs/latex-vorspann}


\section[Stefan Zweig an Arthur Schnitzler, {[}24. 6. 1911?{]}]{L03635 Stefan Zweig an Arthur Schnitzler, {[}24. 6. 1911?{]}}
\nopagebreak\mylabel{L03635v}
\rehead{ }\normalsize\beginnumbering\briefempfaengerindex{Schnitzler, Arthur@\textsc{Schnitzler, Arthur}!zzzZweig, Stefan@\emph{von Stefan Zweig}!1911-06-241@{{[}24. 6. 1911?{]}}|(be}
\toendnotes[C]{\smallbreak\pagebreak[2]}\Standort{CUL, Schnitzler, B 118.}
\physDesc{Brief, 1 Blatt, 1 Seite, 413 Zeichen
\newline{}Handschrift: lila Tinte, lateinische Kurrent}
\buchAbdrucke{\weitereDrucke{Stefan Zweig: \emph{Briefwechsel mit Hermann Bahr, Sigmund Freud, Rainer Maria
                        Rilke und Arthur Schnitzler}. Frankfurt am Main: \emph{S. Fischer} 1987, S. 365.} }\toendnotes[C]{\smallbreak}
\pstart
           {\pb}\textcolor{gray}{\textbf{SZ}}\hfill \textcolor{gray}{\textbf{VIII. KOCHGASSE 8\oindex{Kochgasse 8@\textbf{Kochgasse 8}, \emph{Wohngebäude (K.WHS)}|pw}}}\pend
           
\pstart
           \raggedleft{}\textcolor{gray}{\textbf{WIEN\oindex{Wien@\textbf{Wien}, \emph{A.ADM2}|pw},}}{ }\label{K_L03635-1v}\edtext{Samstag}{\lemma{\textnormal{\emph{Samstag}}}\Cendnote{\textnormal{Der Brief ist nicht genauer datiert. Für Sonntag, den 25. 6. 1911 vermerkte Schnitzler die Lektüre von Zweigs\pwindex{Zweig, Stefan 28.11.1881 – 23.02.1942@\textsc{Zweig, Stefan} (28.11.1881 – 23.02.1942), \emph{Schriftsteller/Schriftstellerin}|pwk} neuem Schauspiel \emph{Das Haus am Meer}\pwindex{Haus am Meer. Ein Schauspiel in zwei Teilen (drei Aufzuegen)@\emph{Das Haus am Meer. Ein Schauspiel in zwei Teilen (drei Aufzügen)}|pwk} im \emph{Tagebuch}\pwindex{Tagebuch@\emph{Tagebuch}|pwk}. Der Brief, der als Beilage eine Werkabschrift\pwindex{Haus am Meer. Ein Schauspiel in zwei Teilen (drei Aufzuegen)@\emph{Das Haus am Meer. Ein Schauspiel in zwei Teilen (drei Aufzügen)}|pwkv} enthielt, dürfte
                  also am Vortag versandt worden sein.}}}\label{K_L03635-1}\pend
           
\pstart{}Lieber, verehrter Herr Doktor,\pend\vspace{0.5em}
\pstart
           hier sende ich Ihnen mein neues Stück\pwindex{Haus am Meer. Ein Schauspiel in zwei Teilen (drei Aufzuegen)@\emph{Das Haus am Meer. Ein Schauspiel in zwei Teilen (drei Aufzügen)}|pwv} und danke Ihnen innig im voraus für die Mühe der Lectüre. Ehe es die leidige
               Wanderschaft zu den Theatern antritt, möge es nun bei Ihnen eine gute Stunde haben
               und Ihnen den \strikeout{\textcolor{gray}{I}} ge\substVorne{}\textsuperscript{\textcolor{gray}{×}\-\textcolor{gray}{×}\-\textcolor{gray}{×}\textcolor{gray}{g}}\substDazwischen{}treuen\substHinten{} Gruss überbringen\pend
           
\pstart
           Ihres aufrichtigen{\\[\baselineskip]}\spacefill\mbox{Stefan Zweig}\pend
           \leftskip=0em{}{\vspace{1\baselineskip}}
\pstart
           \noindent{}Ich lasse es, sobald Sie es \label{K_L03635-3v}\edtext{gelesen haben{[},{]} von Ihnen abholen, damit Sie nicht die
                  Plage der Rücksendung}{\lemma{\textnormal{\emph{gelesen … Rücksendung}}}\Cendnote{\textnormal{Schnitzler und Zweig\pwindex{Zweig, Stefan 28.11.1881 – 23.02.1942@\textsc{Zweig, Stefan} (28.11.1881 – 23.02.1942), \emph{Schriftsteller/Schriftstellerin}|pwk}
                     sprachen sich am Abend nach der Lektüre (25. 6. 1911), die Übergabe dürfte 
                  also zu diesem Zeitpunkt stattgefunden oder zumindest vereinbart worden sein.}}}\label{K_L03635-3} haben.\pend
           \selectlanguage{ngerman}\endnumbering\briefempfaengerindex{Schnitzler, Arthur@\textsc{Schnitzler, Arthur}!zzzZweig, Stefan@\emph{von Stefan Zweig}!1911-06-241@{{[}24. 6. 1911?{]}}|)be}\mylabel{L03635h}  \normalsize

\doendnotes{C}
\bigskip
\vfill

\clearpage

\footnotesize

\lohead{\textsc{register}}

% Definiere theindex-Environment komplett neu ohne reledmac
\makeatletter
\renewenvironment{theindex}{%
  \section*{\indexname}%
  \setlength{\parindent}{0pt}%
  \setlength{\parskip}{0pt plus 0.3pt}%
  \let\item\@idxitem
}{%
  \clearpage
}
\makeatother

\IfFileExists{\jobname-pw.ind}{\input{\jobname-pw.ind}}{}

\end{document}

      