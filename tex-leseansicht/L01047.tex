%% latex-leseansicht-vorspann.tex
%% Vorspann für die Leseansicht.
%% Lädt die gemeinsame Datei latex-vorspann.tex mit nicht gesetztem Schalter.

\newif\ifkorrekturansicht
\korrekturansichtfalse

\input{../tex-inputs/latex-vorspann}


\section[Arthur Schnitzler an Richard Beer-Hofmann, 21. 6. 1900]{L01047 Arthur Schnitzler an Richard Beer-Hofmann, 21. 6. 1900}
\nopagebreak\mylabel{L01047v}
\rehead{ }\normalsize\beginnumbering\briefempfaengerindex{Beer-Hofmann, Richard@\textsc{Beer-Hofmann, Richard}!zzzSchnitzler, Arthur@\emph{von Arthur Schnitzler}!1900-06-211@{21. 6. 1900}|(be}
\toendnotes[C]{\smallbreak\pagebreak[2]}
\correspDesc{Versand  durch Arthur Schnitzler am 21. 6. 1900 in Wien
\newline{}Erhalt  durch Richard Beer-Hofmann am 23. 6. 1900 in Altaussee}\toendnotes[C]{\smallbreak}
\Standort{YCGL, MSS 31.}
\physDesc{Bildpostkarte, 182 Zeichen
\newline{}Handschrift: Bleistift, deutsche Kurrent
\newline{}Versand: 1) Stempel: »\nobreak{}\oindex{Wien@\textbf{Wien}, \emph{Verwaltungsgebiet}|pwk}Wien 54, 22. 6. 00, 7–8V\nobreak{}«.   2) Stempel: »\nobreak{}\oindex{Altaussee@\textbf{Altaussee}, \emph{Verwaltungsgebiet}|pwk}Alt-Aussee, 23 6 00\nobreak{}«. }
\buchAbdrucke{\weitereDrucke{Arthur Schnitzler, Richard Beer-Hofmann: \emph{Briefwechsel 1891–1931}. Herausgegeben von Konstanze Fliedl. Wien, Zürich: \emph{Europaverlag} 1992, S. 146.} }\toendnotes[C]{\smallbreak}\pstart{}{\pb}Herrn \textsc{Dr. Richard
                     Beer-Hofmann}\pend{}\pstart{}\textsc{Altaussee\oindex{Altaussee@\textbf{Altaussee}, \emph{Verwaltungsgebiet}|pw}}\pend{}{\bigskip}
\pstart
           \noindent{}\centering{}{\pb}\textcolor{gray}{\textbf{WIEN\oindex{Wien@\textbf{Wien}, \emph{Verwaltungsgebiet}|pw}, K. u.  k. kunsthistorisches Museum\oindex{Wien@\textbf{Wien}!I., Innere Stadt@\textbf{I., Innere Stadt}!Kunsthistorisches Museum@\textbf{Kunsthistorisches Museum}, \emph{Museum}|pw},
                  Stiegenhaus.}}\pend
           \vspace{1em}
\pstart
           {\pb}Ein \label{K_L01047-1v}\edtext{Citat\pwindex{\textcolor{red}{\textsuperscript{XXXX indx1}}!Theseus besiegt den Kentaur@\strich\emph{Theseus besiegt den Kentaur}|pwv}}{\lemma{\textnormal{\emph{Citat}}}\Cendnote{\textnormal{Auf dem Bild ist \emph{Theseus besiegt den Kentaur}\pwindex{\textcolor{red}{\textsuperscript{XXXX indx1}}!Theseus besiegt den Kentaur@\strich\emph{Theseus besiegt den Kentaur}|pwk} zu sehen. In \emph{Frau Bertha Garlan}\pwindex{Schnitzler, Arthur 15.\,5.\,1862 Wien – 21.\,10.\,1931 ebd.@\textsc{Schnitzler, Arthur} (15.\,5.\,1862 Wien – 21.\,10.\,1931 ebd.), \emph{Schriftsteller, Mediziner}!Frau Bertha Garlan. Roman@\strich\emph{Frau Bertha Garlan. Roman}|pwk} erwähnt Schnitzler aber das heute in London\oindex{London@\textbf{London}, \emph{Hauptstadt}|pwk} befindliche \emph{Theseus und der Minotaurus}\pwindex{Schnitzler, Arthur 15.\,5.\,1862 Wien – 21.\,10.\,1931 ebd.@\textsc{Schnitzler, Arthur} (15.\,5.\,1862 Wien – 21.\,10.\,1931 ebd.), \emph{Schriftsteller, Mediziner}!Frau Bertha Garlan. Roman@\strich\emph{Frau Bertha Garlan. Roman}|pwk}.}}}\label{K_L01047-1} aus der neuen Novelle\pwindex{Schnitzler, Arthur 15.\,5.\,1862 Wien – 21.\,10.\,1931 ebd.@\textsc{Schnitzler, Arthur} (15.\,5.\,1862 Wien – 21.\,10.\,1931 ebd.), \emph{Schriftsteller, Mediziner}!Frau Bertha Garlan. Roman@\strich\emph{Frau Bertha Garlan. Roman}|pwv}. –\pend
           
\pstart
           21. 6. 900\pend
           \vspace{0.5em}
\pstart
           Im \label{K_L01047-2v}\edtext{Vorſtadtbeiſel}{\lemma{\textnormal{\emph{Vorstadtbeisel}}}\Cendnote{\textnormal{österreichisch Beisl: einfaches
                  Lokal}}}\label{K_L01047-2} mit den geſchmackloſen Tapeten und den kleinen Beamten am
               Nebentiſch.\pend
           
\pstart
           Herzlichſt{\\[\baselineskip]}Ihr \spacefill\mbox{Arth.}\pend
           \leftskip=0em{}\selectlanguage{ngerman}\endnumbering\briefempfaengerindex{Beer-Hofmann, Richard@\textsc{Beer-Hofmann, Richard}!zzzSchnitzler, Arthur@\emph{von Arthur Schnitzler}!1900-06-211@{21. 6. 1900}|)be}\mylabel{L01047h}  \newcommand{\dateiname}{L01047}\newcommand{\titel}{Arthur Schnitzler an Richard Beer-Hofmann, 21. 6. 1900}\newcommand{\editorInnen}{Martin Anton Müller und Gerd-Hermann Susen}%% latex-leseansicht-abspann.tex
%% Abspann für die Leseansicht.
%% Der Schalter \ifkorrekturansicht ist bereits durch den Vorspann gesetzt.

%% latex-abspann.tex
%% Gemeinsamer Abspann für Korrekturansicht und Leseansicht.
%% Setzt den Schalter \ifkorrekturansicht voraus (gesetzt in den
%% einbindenden Dateien latex-korrekturansicht-abspann.tex bzw.
%% latex-leseansicht-abspann.tex).
%% ---------------------------------------------------------------

\normalsize

% Das esempio-Environment wird nur in der Leseansicht benötigt
\ifkorrekturansicht\else
\newenvironment{esempio}[3]%
{
    \vspace{1.5ex}
    \rlap{\underline{#1}}
    \par
    \setlength{\parindent}{0cm}
    \nopagebreak
    \leftskip=#2cm
    \rightskip=#3cm
}
{
    \par
}
\fi

\doendnotes{C}
\bigskip
\vfill

\clearpage

\footnotesize

\ifkorrekturansicht
  \lohead{\textsc{register}}
\fi

% theindex-Environment neu definieren ohne reledmac
\makeatletter
\renewenvironment{theindex}{%
  \ifkorrekturansicht
    \section*{\indexname}%
  \else
    \subsubsection*{Index der erwähnten Entitäten}%
  \fi
  \setlength{\parindent}{0pt}%
  \setlength{\parskip}{0pt plus 0.3pt}%
  \let\item\@idxitem
}{%
  \ifkorrekturansicht\clearpage\fi
}
\makeatother

\IfFileExists{\jobname-pw.ind}{\input{\jobname-pw.ind}}{}

% Quellenangabe nur in der Leseansicht
\ifkorrekturansicht\else
% Fallback-Definitionen, falls die .tex-Datei \titel etc. nicht gesetzt hat
\providecommand{\titel}{}
\providecommand{\editorInnen}{}
\providecommand{\dateiname}{\jobname}

\vspace{3cm}

\vfill

\footnotesize
\textsc{Quelle}: \titel. Herausgegeben von {\editorInnen}. In: \emph{Arthur Schnitzler: Briefwechsel mit Autorinnen und Autoren}.
 Digitale Edition, https://schnitzler-briefe.acdh.oeaw.ac.at/{\dateiname}.html (Stand \today)
\fi

\end{document}


