%% latex-korrekturansicht-vorspann.tex
%% Vorspann für die Korrekturansicht.
%% Lädt die gemeinsame Datei latex-vorspann.tex mit gesetztem Schalter.

\newif\ifkorrekturansicht
\korrekturansichttrue

\input{../tex-inputs/latex-vorspann}


\section[Arthur Schnitzler an Richard Beer-Hofmann, 21. 6. 1900]{L01047 Arthur Schnitzler an Richard Beer-Hofmann, 21. 6. 1900}
\nopagebreak\mylabel{L01047v}
\rehead{ }\normalsize\beginnumbering\briefempfaengerindex{Beer-Hofmann, Richard@\textsc{Beer-Hofmann, Richard}!zzzSchnitzler, Arthur@\emph{von Arthur Schnitzler}!1900-06-211@{21. 6. 1900}|(be}
\toendnotes[C]{\smallbreak\pagebreak[2]}\Standort{YCGL, MSS 31.}
\physDesc{Bildpostkarte, 182 Zeichen
\newline{}Handschrift: Bleistift, deutsche Kurrent
\newline{}Versand: 1) Stempel: »\nobreak{}Wien 54, 22. 6. 00, 7–8V\nobreak{}«.   2) Stempel: »\nobreak{}\oindex{Altaussee@\textbf{Altaussee}, \emph{A.ADM3}|pwk}Alt-Aussee, 23 6 00\nobreak{}«. }
\buchAbdrucke{\weitereDrucke{Arthur Schnitzler, Richard Beer-Hofmann: \emph{Briefwechsel 1891–1931}. Wien, Zürich: \emph{Europaverlag} 1992, S. 146.} }\toendnotes[C]{\smallbreak}\pstart{}{\pb}Herrn \textsc{Dr. Richard
                     Beer-Hofmann}\pend{}\pstart{}\textsc{Altaussee\oindex{Altaussee@\textbf{Altaussee}, \emph{A.ADM3}|pw}}\pend{}{\bigskip}
\pstart
           \noindent{}\centering{}{\pb}\textcolor{gray}{\textbf{WIEN\oindex{Wien@\textbf{Wien}, \emph{A.ADM2}|pw}, K. u.  k. kunsthistorisches Museum\oindex{Kunsthistorisches Museum@\textbf{Kunsthistorisches Museum}, \emph{Museum (K.MUS)}|pw},
                  Stiegenhaus.}}\pend
           \vspace{1em}
\pstart
           {\pb}Ein \label{K_L01047-1v}\edtext{Citat\pwindex{Theseus besiegt den Kentaur@\emph{Theseus besiegt den Kentaur}|pwv}}{\lemma{\textnormal{\emph{Citat}}}\Cendnote{\textnormal{Auf dem Bild ist \emph{Theseus besiegt den Kentaur}\pwindex{Theseus besiegt den Kentaur@\emph{Theseus besiegt den Kentaur}|pwk} zu sehen. In \emph{Frau Bertha Garlan}\pwindex{Frau Bertha Garlan. Roman@\emph{Frau Bertha Garlan. Roman}|pwk} erwähnt Schnitzler aber das heute in London\oindex{London@\textbf{London}, \emph{P.PPLC}|pwk} befindliche \emph{Theseus und der Minotaurus}\pwindex{Frau Bertha Garlan. Roman@\emph{Frau Bertha Garlan. Roman}|pwk}.}}}\label{K_L01047-1} aus der neuen Novelle\pwindex{Frau Bertha Garlan. Roman@\emph{Frau Bertha Garlan. Roman}|pwv}. –\pend
           
\pstart
           21. 6. 900\pend
           \vspace{0.5em}
\pstart
           Im \label{K_L01047-2v}\edtext{Vorſtadtbeiſel}{\lemma{\textnormal{\emph{Vorſtadtbeiſel}}}\Cendnote{\textnormal{österreichisch Beisl: einfaches
                  Lokal}}}\label{K_L01047-2} mit den geſchmackloſen Tapeten und den kleinen Beamten am
               Nebentiſch.\pend
           
\pstart
           Herzlichſt{\\[\baselineskip]}Ihr \spacefill\mbox{Arth.}\pend
           \leftskip=0em{}\selectlanguage{ngerman}\endnumbering\briefempfaengerindex{Beer-Hofmann, Richard@\textsc{Beer-Hofmann, Richard}!zzzSchnitzler, Arthur@\emph{von Arthur Schnitzler}!1900-06-211@{21. 6. 1900}|)be}\mylabel{L01047h}  \normalsize

\doendnotes{C}
\bigskip
\vfill

\clearpage

\footnotesize

\lohead{\textsc{register}}

% Definiere theindex-Environment komplett neu ohne reledmac
\makeatletter
\renewenvironment{theindex}{%
  \section*{\indexname}%
  \setlength{\parindent}{0pt}%
  \setlength{\parskip}{0pt plus 0.3pt}%
  \let\item\@idxitem
}{%
  \clearpage
}
\makeatother

\IfFileExists{\jobname-pw.ind}{\input{\jobname-pw.ind}}{}

\end{document}

      