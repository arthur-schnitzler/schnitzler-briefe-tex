%% latex-leseansicht-vorspann.tex
%% Vorspann für die Leseansicht.
%% Lädt die gemeinsame Datei latex-vorspann.tex mit nicht gesetztem Schalter.

\newif\ifkorrekturansicht
\korrekturansichtfalse

\input{../tex-inputs/latex-vorspann}


         
         \renewcommand{\erwaehntePersonen}{Personen:  ?? [Berliner Korrespondent des Figaro], Paul Bourget, Albert Carré, Benjamin Constant, Paul Ginisty, Jules Huret, Otto Jahn,  Josef II., Wolfgang Amadeus Mozart, Leopold Sonnemann, Jean Thorel, Leo Van-Jung, Franz Walsegg-Stuppach}
         \renewcommand{\erwaehnteInstitutionen}{Institutionen: Frankfurter Zeitung, Le Figaro}
         \renewcommand{\erwaehnteOrte}{Orte: Paris, Wien, rue Feydeau}
         \renewcommand{\erwaehnteWerke}{Werke: Adolphe. Anecdote trouvée dans les papiers d’un inconnu, Amourette. Pièce en trois actes. Adaptée de Arthur Schnitzler, Courrier des Théatres [Freiwild in Berlin und Liebelei], Courrier des Théatres [Mon cher Huret; Thorel zur Liebelei-Übersetzung], Frankfurter Zeitung, Freiwild. Schauspiel in 3 Akten, Le Figaro, Liebelei. Schauspiel in drei Akten, Requiem d-Moll KV 626, W. A. Mozart}
               \section[ Paul Goldmann an Arthur Schnitzler, 13. 11. {[}1896{]}]{ Paul Goldmann an Arthur Schnitzler, 13. 11. {[}1896{]}}\nopagebreak\mylabel{v}\rehead{ }\begin{ledgroupsized}[t]{13cm}\normalsize\beginnumbering \toendnotes[C]{\smallbreak\pagebreak[2]} \Standort{DLA, A:Schnitzler, HS.NZ85.1.3166.}
\physDesc{Brief, 3 Blätter, 10 Seiten, 4309 Zeichen
\newline{}Handschrift: blaue Tinte, deutsche Kurrent
\newline{}Beilage: zwei beschnittene und zusammengeklebte Zeitungsausschnitte auf
                                 der ersten Seite, der eine aus der Kopfzeile bestehend 
\newline{}Schnitzler: 1) mit Bleistift das Jahr »96« vermerkt  2) mit rotem Buntstift eine Unterstreichung}\toendnotes[C]{\smallbreak}\pstart
           \noindent{}{\pb}\textcolor{gray}{\textbf{\textbf{Frankfurter Zeitung\orgindex{Frankfurter Zeitung@Frankfurter Zeitung|pw}}}}\pend
           \pstart
           \textcolor{gray}{\textbf{(\begin{otherlanguage}{french}Gazette de Francfort\end{otherlanguage}\orgindex{Frankfurter Zeitung@Frankfurter Zeitung|pw}).}}\pend
           \pstart
           \textcolor{gray}{\textbf{\textbf{\begin{otherlanguage}{french}Fondateur M.\end{otherlanguage}{ }L. Sonnemann\pwindex{Sonnemann, Leopold 1831-10-29 – 1909-10-30@\textsc{Sonnemann, Leopold} (1831-10-29 – 1909-10-30), \emph{Journalist, Herausgeber}|pw}.}}}\pend
           \pstart
           \begin{otherlanguage}{french}\textcolor{gray}{\textbf{Journal\pwindex{?? Werk@Nicht ermittelte Verfasserinnen und Verfasser!Frankfurter Zeitung1856 – 1943@\emph{Frankfurter Zeitung} {[}1856 – 1943{]}|pwv} politique,
                        financier,}}\end{otherlanguage}\pend
           \pstart
           \begin{otherlanguage}{french}\textcolor{gray}{\textbf{commercial et littéraire.}}\end{otherlanguage}\pend
           \pstart
           \begin{otherlanguage}{french}\textcolor{gray}{\textbf{\textbf{Paraissant trois fois par jour.}}}\end{otherlanguage}\pend
           \pstart
           \begin{otherlanguage}{french}\textcolor{gray}{\textbf{\textbf{Bureau à Paris\oindex{Paris@\textbf{Paris}|pw}}}}\end{otherlanguage}\pend
           \pstart
           \begin{otherlanguage}{french}\textcolor{gray}{\textbf{\textbf{24. Rue Feydeau\oindex{rue Feydeau@\textbf{rue Feydeau}|pw}.}}}\end{otherlanguage}\pend
           {\bigskip}\pstart
           \noindent{}\centering{}\label{K_L02790-88v}\edtext{\begin{otherlanguage}{french}\textcolor{gray}{\textbf{LE FIGARO\orgindex{Le Figaro@Le Figaro|pw}{ }MARDI 10 NOVEMBRE}}\end{otherlanguage}}{\lemma{\textnormal{\emph{Le … Mardi 10 Novembre}}}\Cendnote{\textnormal{französisch: \emph{Le Figaro}\orgindex{Le Figaro@Le Figaro|pwk}{ }Dienstag, 10. November}}}\label{K_L02790-88h}\pend
           \leftskip=3em{}\pstart
           \noindent{}\label{K_L02790-77v}\edtext{\begin{otherlanguage}{french}\textcolor{gray}{\textbf{Mon cher Huret\pwindex{Huret, Jules 1863-04-08 – 1915-02-14@\textsc{Huret, Jules} (1863-04-08 – 1915-02-14), \emph{Journalist, Publizist, Schriftsteller}|pw},
                     }}\end{otherlanguage}}{\lemma{\textnormal{\emph{Mon cher Huret,
                     }}}\Cendnote{\textnormal{französisch: Mein lieber Huret\pwindex{Huret, Jules 1863-04-08 – 1915-02-14@\textsc{Huret, Jules} (1863-04-08 – 1915-02-14), \emph{Journalist, Publizist, Schriftsteller}|pwk},}}}\label{K_L02790-77h}\pend
           \leftskip=0em{}\pstart
           \noindent{}\label{K_L02790-16v}\edtext{\begin{otherlanguage}{french}\textcolor{gray}{\textbf{Pour compléter vos \label{K_L02790-111v}\edtext{renseignements\pwindex{?? [Berliner Korrespondent des Figaro] @\textsc{?? [Berliner Korrespondent des Figaro]}!Courrier des Theatres [Freiwild in Berlin und Liebelei]1896-11-07@\strich\emph{Courrier des Théatres [Freiwild in Berlin und Liebelei]} {[}1896-11-07{]}|pwv}}{\lemma{\textnormal{\emph{renseignements}}}\Cendnote{\textnormal{Jules Huret\pwindex{Huret, Jules 1863-04-08 – 1915-02-14@\textsc{Huret, Jules} (1863-04-08 – 1915-02-14), \emph{Journalist, Publizist, Schriftsteller}|pwk} leitete die
                           Theaterrubrik des \emph{Figaro}\pwindex{Le Figaro1826-01-15@\emph{Le Figaro} {[}1826-01-15{]}|pwk}. Das Telegramm\pwindex{?? [Berliner Korrespondent des Figaro] @\textsc{?? [Berliner Korrespondent des Figaro]}!Courrier des Theatres [Freiwild in Berlin und Liebelei]1896-11-07@\strich\emph{Courrier des Théatres [Freiwild in Berlin und Liebelei]} {[}1896-11-07{]}|pwkv} des Berliner
                              Korrespondenten\pwindex{?? [Berliner Korrespondent des Figaro] @\textsc{?? [Berliner Korrespondent des Figaro]}|pwkv} wurde abgedruckt: \emph{Le Figaro}\pwindex{Le Figaro1826-01-15@\emph{Le Figaro} {[}1826-01-15{]}|pwk}, Jg. 42, Nr. 312, 7. 11. 1896, S. 4.}}}\label{K_L02790-111h} sur
                        Arthur Schnitzler, laissez-moi vous dire que je viens de terminer la traduction\pwindex{Thorel, Jean 1859-09-11 – 1916-08-20@\textsc{Thorel, Jean} (1859-09-11 – 1916-08-20), \emph{Übersetzer, Schriftsteller}!Amourette. Piece en trois actes. Adaptee de Arthur Schnitzler1897@\strich\emph{Amourette. Pièce en trois actes. Adaptée de Arthur Schnitzler} {[}Übersetzung, 1897{]}|pwv} en français
                        de cette \emph{\label{T_L02790-56v}\edtext{Liebelei}{\lemma{\textnormal{\emph{Liebelei}}}\Cendnote{\textnormal{im gedruckten Text steht: »Liebelci\pwindex{Schnitzler, Arthur 15.05.1862 – 21.10.1931@\textsc{Schnitzler, Arthur} (15.05.1862 – 21.10.1931), \emph{Schriftsteller, Mediziner}!Liebelei. Schauspiel in drei Akten1895-10-09@\strich\emph{Liebelei. Schauspiel in drei Akten} {[}1895-10-09{]}|pwv}«}}}\label{T_L02790-56h}\pwindex{Schnitzler, Arthur 15.05.1862 – 21.10.1931@\textsc{Schnitzler, Arthur} (15.05.1862 – 21.10.1931), \emph{Schriftsteller, Mediziner}!Liebelei. Schauspiel in drei Akten1895-10-09@\strich\emph{Liebelei. Schauspiel in drei Akten} {[}1895-10-09{]}|pw}} dont vous rappelez le grand succès, l’hiver dernier, à Vienne\oindex{Wien@\textbf{Wien}|pw}.}}\end{otherlanguage}}{\lemma{\textnormal{\emph{Pour … Vienne.}}}\Cendnote{\textnormal{französisch: Um Ihre Auskünfte\pwindex{?? [Berliner Korrespondent des Figaro] @\textsc{?? [Berliner Korrespondent des Figaro]}!Courrier des Theatres [Freiwild in Berlin und Liebelei]1896-11-07@\strich\emph{Courrier des Théatres [Freiwild in Berlin und Liebelei]} {[}1896-11-07{]}|pwkv} über Arthur Schnitzler\pwindex{Schnitzler, Arthur 15.05.1862 – 21.10.1931@\textsc{Schnitzler, Arthur} (15.05.1862 – 21.10.1931), \emph{Schriftsteller, Mediziner}|pwk} zu vervollständigen, möchte ich
                     kundtun, dass ich gerade die französische Übersetzung\pwindex{Thorel, Jean 1859-09-11 – 1916-08-20@\textsc{Thorel, Jean} (1859-09-11 – 1916-08-20), \emph{Übersetzer, Schriftsteller}!Amourette. Piece en trois actes. Adaptee de Arthur Schnitzler1897@\strich\emph{Amourette. Pièce en trois actes. Adaptée de Arthur Schnitzler} {[}Übersetzung, 1897{]}|pwkv} der \emph{Liebelei}\pwindex{Schnitzler, Arthur 15.05.1862 – 21.10.1931@\textsc{Schnitzler, Arthur} (15.05.1862 – 21.10.1931), \emph{Schriftsteller, Mediziner}!Liebelei. Schauspiel in drei Akten1895-10-09@\strich\emph{Liebelei. Schauspiel in drei Akten} {[}1895-10-09{]}|pwk} abgeschlossen habe, an deren großen Erfolg in Wien\oindex{Wien@\textbf{Wien}|pwk} im letzten Winter Sie sich
                     erinnern.}}}\label{K_L02790-16h}\pend
           \pstart
           \label{K_L02790-12v}\edtext{\begin{otherlanguage}{french}\textcolor{gray}{\textbf{Déjà \label{K_L02790-14v}\edtext{deux de nos
                           directeurs de théâtre\pwindex{Ginisty, Paul 1855-04-04 – 1932-03-05@\textsc{Ginisty, Paul} (1855-04-04 – 1932-03-05), \emph{Schriftsteller, Theaterleiter}|pwv}\pwindex{Carre, Albert 22.06.1852 – 11.12.1938@\textsc{Carré, Albert} (22.06.1852 – 11.12.1938), \emph{Schriftsteller, Theaterleiter, Schauspieler}|pwv}}{\lemma{\textnormal{\emph{deux … théâtre}}}\Cendnote{\textnormal{vgl. Paul Goldmann an Arthur Schnitzler, 2. [1.? 1897]}}}\label{K_L02790-14h} m’ont promis{\dots} de lire cette traduction\pwindex{Thorel, Jean 1859-09-11 – 1916-08-20@\textsc{Thorel, Jean} (1859-09-11 – 1916-08-20), \emph{Übersetzer, Schriftsteller}!Amourette. Piece en trois actes. Adaptee de Arthur Schnitzler1897@\strich\emph{Amourette. Pièce en trois actes. Adaptée de Arthur Schnitzler} {[}Übersetzung, 1897{]}|pwv}. Ai-je
                        besoin d’ajouter qu’ils se proposent même de faire cette lecture »avec le
                        plus vif intérêt«.}}\end{otherlanguage}}{\lemma{\textnormal{\emph{Déjà … intérêt«.}}}\Cendnote{\textnormal{französisch: Zwei unserer
                        Theaterdirektoren\pwindex{Ginisty, Paul 1855-04-04 – 1932-03-05@\textsc{Ginisty, Paul} (1855-04-04 – 1932-03-05), \emph{Schriftsteller, Theaterleiter}|pwkv}\pwindex{Carre, Albert 22.06.1852 – 11.12.1938@\textsc{Carré, Albert} (22.06.1852 – 11.12.1938), \emph{Schriftsteller, Theaterleiter, Schauspieler}|pwkv} haben mir bereits versprochen, die Übersetzung\pwindex{Thorel, Jean 1859-09-11 – 1916-08-20@\textsc{Thorel, Jean} (1859-09-11 – 1916-08-20), \emph{Übersetzer, Schriftsteller}!Amourette. Piece en trois actes. Adaptee de Arthur Schnitzler1897@\strich\emph{Amourette. Pièce en trois actes. Adaptée de Arthur Schnitzler} {[}Übersetzung, 1897{]}|pwkv} zu lesen. Muss ich noch
                     hinzufügen, dass sie diese Lektüre »mit dem lebhaftesten Interesse«
                     unternehmen?}}}\label{K_L02790-12h}\pend
           \pstart
           \raggedleft{}\label{K_L02790-889v}\edtext{\begin{otherlanguage}{french}\textcolor{gray}{\textbf{Votre bien dévoué,}}\end{otherlanguage}}{\lemma{\textnormal{\emph{Votre bien dévoué,}}}\Cendnote{\textnormal{französisch: Ihr sehr
                     ergeber}}}\label{K_L02790-889h}\pend
           \pstart
           \noindent{}\raggedleft{}\textcolor{gray}{\textbf{Jean THOREL\pwindex{Thorel, Jean 1859-09-11 – 1916-08-20@\textsc{Thorel, Jean} (1859-09-11 – 1916-08-20), \emph{Übersetzer, Schriftsteller}|pw}.}}\pend
           {\bigskip}\pstart
           \raggedleft{}\textsc{Paris\oindex{Paris@\textbf{Paris}|pw}}, 13. November.\pend
           \pstart\center{}Mein lieber Freund,\pend\pstart
           Oben ſiehſt Du einen \label{K_L02790-1v}\edtext{Ausſchnitt\pwindex{Thorel, Jean 1859-09-11 – 1916-08-20@\textsc{Thorel, Jean} (1859-09-11 – 1916-08-20), \emph{Übersetzer, Schriftsteller}!Courrier des Theatres [Mon cher Huret; Thorel zur
                  Liebelei-Uebersetzung]1896-11-10@\strich\emph{Courrier des Théatres [Mon cher Huret; Thorel zur Liebelei-Übersetzung]} {[}1896-11-10{]}|pwv} aus dem »\textsc{Figaro\pwindex{Le Figaro1826-01-15@\emph{Le Figaro} {[}1826-01-15{]}|pw}}«}{\lemma{\textnormal{\emph{Ausſchnitt … »Figaro«}}}\Cendnote{\textnormal{Jean Thorel\pwindex{Thorel, Jean 1859-09-11 – 1916-08-20@\textsc{Thorel, Jean} (1859-09-11 – 1916-08-20), \emph{Übersetzer, Schriftsteller}|pwk}: \emph{[Mon cher Huret]}\pwindex{Thorel, Jean 1859-09-11 – 1916-08-20@\textsc{Thorel, Jean} (1859-09-11 – 1916-08-20), \emph{Übersetzer, Schriftsteller}!Courrier des Theatres [Mon cher Huret; Thorel zur
                  Liebelei-Uebersetzung]1896-11-10@\strich\emph{Courrier des Théatres [Mon cher Huret; Thorel zur Liebelei-Übersetzung]} {[}1896-11-10{]}|pwk}. In: \emph{Le Figaro}\pwindex{Le Figaro1826-01-15@\emph{Le Figaro} {[}1826-01-15{]}|pwk}, Jg. 42, Nr. 315, 10. 11. 1896, S. 4.}}}\label{K_L02790-1h}. Die Überſetzung\pwindex{Thorel, Jean 1859-09-11 – 1916-08-20@\textsc{Thorel, Jean} (1859-09-11 – 1916-08-20), \emph{Übersetzer, Schriftsteller}!Amourette. Piece en trois actes. Adaptee de Arthur Schnitzler1897@\strich\emph{Amourette. Pièce en trois actes. Adaptée de Arthur Schnitzler} {[}Übersetzung, 1897{]}|pwv} von \textsc{Thorel\pwindex{Thorel, Jean 1859-09-11 – 1916-08-20@\textsc{Thorel, Jean} (1859-09-11 – 1916-08-20), \emph{Übersetzer, Schriftsteller}|pw}} iſt \strikeout{\textcolor{gray}{×}} – unter uns geſagt – leider recht ſchlecht, noch ſchlechter, als ich geglaubt.
               Er hat ſich gar keine Mühe gegeben, \strikeout{d\textcolor{gray}{ie}} das natürliche und lebendige Deutſch des Dialog\pwindex{Schnitzler, Arthur 15.05.1862 – 21.10.1931@\textsc{Schnitzler, Arthur} (15.05.1862 – 21.10.1931), \emph{Schriftsteller, Mediziner}!Liebelei. Schauspiel in drei Akten1895-10-09@\strich\emph{Liebelei. Schauspiel in drei Akten} {[}1895-10-09{]}|pwv}es in natürliches und lebendiges Franzöſiſch
               umzuſetzen. Ich tröſte mich damit, daß es ein Anderer noch ſchlechter gemacht hätte.
                  {\pb}Auch rechne ich auf die dem Stücke\pwindex{Schnitzler, Arthur 15.05.1862 – 21.10.1931@\textsc{Schnitzler, Arthur} (15.05.1862 – 21.10.1931), \emph{Schriftsteller, Mediziner}!Liebelei. Schauspiel in drei Akten1895-10-09@\strich\emph{Liebelei. Schauspiel in drei Akten} {[}1895-10-09{]}|pwv} innewohnende Poeſie, die ſich beim
               beſten Willen nicht umbringen läßt{\dotsfive}\pend
           \pstart
           Mit Deinem lieben Briefe habe ich mich ſehr gefreut. Ich begreife Deine Stimmung, und
               da Du Dir gewiß über die Gründe klar biſt, wird auch dieſes zweite Stück\pwindex{Schnitzler, Arthur 15.05.1862 – 21.10.1931@\textsc{Schnitzler, Arthur} (15.05.1862 – 21.10.1931), \emph{Schriftsteller, Mediziner}!Freiwild. Schauspiel in 3 Akten1896@\strich\emph{Freiwild. Schauspiel in 3 Akten} {[}1896{]}|pwv} für Deine Entwickelung nützlich ſein.
                  \label{K_L02790-3v}\edtext{Das Stück\pwindex{Schnitzler, Arthur 15.05.1862 – 21.10.1931@\textsc{Schnitzler, Arthur} (15.05.1862 – 21.10.1931), \emph{Schriftsteller, Mediziner}!Freiwild. Schauspiel in 3 Akten1896@\strich\emph{Freiwild. Schauspiel in 3 Akten} {[}1896{]}|pwv} iſt Dir unſympathiſch}{\lemma{\textnormal{\emph{Das … unſympathiſch}}}\Cendnote{\textnormal{siehe A. S.: \emph{Tagebuch}, 5. 11. 1896}}}\label{K_L02790-3h}, weil es nicht Deiner Natur und Deiner Schaffensart entſpricht. Es iſt nicht
               aus dem Leben herausgewachſen, ſondern aus einer Idee, zu der hinterdrein die Figuren
               geſucht wurden. Beſonders {\pb}ſieht man das an dem Helden\pwindex{Schnitzler, Arthur 15.05.1862 – 21.10.1931@\textsc{Schnitzler, Arthur} (15.05.1862 – 21.10.1931), \emph{Schriftsteller, Mediziner}!Freiwild. Schauspiel in 3 Akten1896@\strich\emph{Freiwild. Schauspiel in 3 Akten} {[}1896{]}|pwv}. Den haſt Du nie
               geſehen. Du haſt ihn Dir künſtlich zuſammenzimmern müſſen, damit er zu Deiner Idee
               paßt. Darum biſt Du ſo unſicher bei ſeiner Geſtaltung geweſen, darum iſt er Dir ſo
               ſchwer gefallen, darum iſt er auch heut nicht recht gelungen. Und der Hauptfehler
               war: Es war ein Tendenzſtück, und Du haſt Dir das nicht eingeſtehen wollen und haſt
               es nicht als Tendenzſtück ſchreiben wollen. Es war ein Tendenzſtück, das ſo ausſehen
               ſollte, als ſei es natürlich {\pb}und erlebt. Das iſt
               unmöglich. Die \label{K_L02790-777v}\edtext{\textsc{procédés}}{\lemma{\textnormal{\emph{procédés}}}\Cendnote{\textnormal{französisch: das Prozedere}}}\label{K_L02790-777h}
               Deiner Kunſt, die Natürliches und Erlebtes ausdrücken will und kann, waren hier im
               Zwieſpalt mit den Anforderungen des \textsc{Sujets}. Gerade die
               Unparteilichkeit halte ich für einen Fehler des Stück\pwindex{Schnitzler, Arthur 15.05.1862 – 21.10.1931@\textsc{Schnitzler, Arthur} (15.05.1862 – 21.10.1931), \emph{Schriftsteller, Mediziner}!Freiwild. Schauspiel in 3 Akten1896@\strich\emph{Freiwild. Schauspiel in 3 Akten} {[}1896{]}|pwv}es. Es mußte parteilich ſein. Es mußte ein Stück\pwindex{Schnitzler, Arthur 15.05.1862 – 21.10.1931@\textsc{Schnitzler, Arthur} (15.05.1862 – 21.10.1931), \emph{Schriftsteller, Mediziner}!Freiwild. Schauspiel in 3 Akten1896@\strich\emph{Freiwild. Schauspiel in 3 Akten} {[}1896{]}|pwv} werden gegen das Duell.
               Für dieſes Stück\pwindex{Schnitzler, Arthur 15.05.1862 – 21.10.1931@\textsc{Schnitzler, Arthur} (15.05.1862 – 21.10.1931), \emph{Schriftsteller, Mediziner}!Freiwild. Schauspiel in 3 Akten1896@\strich\emph{Freiwild. Schauspiel in 3 Akten} {[}1896{]}|pwv} mußteſt Du
               Deine bisherige Productions-Art beiſeite laſſen und \introOben{}Du\introOben{}
               mußteſt es mit Haß und Leidenſchaft ſchreiben, \strikeout{g} ganz
               ohne Rückſicht darauf, ob es unwahrſcheinlich und {\pb}ungerecht wurde. Ich meine, Du ſollſt fürs Erſte von allen Stoffen dieſer Art, von
               allen »großen Zeitfragen« \textsc{etc.} laſſen. Ich möchte Dir jetzt
               gerade einen \strikeout{\textcolor{gray}{×}\-\textcolor{gray}{×}\-\textcolor{gray}{×}\-\textcolor{gray}{×}\-\textcolor{gray}{×}\-\textcolor{gray}{×}\-\textcolor{gray}{×}\-\textcolor{gray}{×}} Wanderzug in die Vergangenheit und in die reine Poeſie empfehlen. \uline{Das hiſtoriſche Wien\oindex{Wien@\textbf{Wien}|pw}er
                  Stück!} Jetzt mußt Du es ſchreiben, und ich bin überzeugt, es wird Dir
               köſtlich gelingen. Nimm’ Dir zwei oder drei Jahre Zeit und ruhe Dich ein wenig auf
               den zwei ſtarken Erfolgen aus, durch welche Du mit einem {\pb}Male in die allererſte Reihe unter den deutſchen
               Bühnen-Dichtern gerückt biſt. Ich möchte Dir einen ſchönen Stoff vorſchlagen: \uline{\textsc{Mozart\pwindex{Mozart, Wolfgang Amadeus 27.01.1756 – 05.12.1791@\textsc{Mozart, Wolfgang Amadeus} (27.01.1756 – 05.12.1791), \emph{Komponist}|pw}}}, ein Wien\oindex{Wien@\textbf{Wien}|pw}er Volksſtück mit \textsc{Mozart\pwindex{Mozart, Wolfgang Amadeus 27.01.1756 – 05.12.1791@\textsc{Mozart, Wolfgang Amadeus} (27.01.1756 – 05.12.1791), \emph{Komponist}|pw}}’ſcher Muſik. Ich hatte neulich Gelegenheit, \textsc{Otto Jahn\pwindex{Jahn, Otto 1813-06-16 – 1869-09-09@\textsc{Jahn, Otto} (1813-06-16 – 1869-09-09), \emph{Wissenschaftler, Philologe, Archäologe}|pw}s}{ }\textsc{Mozart\pwindex{Mozart, Wolfgang Amadeus 27.01.1756 – 05.12.1791@\textsc{Mozart, Wolfgang Amadeus} (27.01.1756 – 05.12.1791), \emph{Komponist}|pw}}-Biographie\pwindex{Jahn, Otto 1813-06-16 – 1869-09-09@\textsc{Jahn, Otto} (1813-06-16 – 1869-09-09), \emph{Wissenschaftler, Philologe, Archäologe}!W. A. Mozart1856 – 1859@\strich\emph{W. A. Mozart} {[}1856 – 1859{]}|pwv} einzuſehen. Natürlich hatte ich keine Zeit, die beiden
               dicken Bände\pwindex{Jahn, Otto 1813-06-16 – 1869-09-09@\textsc{Jahn, Otto} (1813-06-16 – 1869-09-09), \emph{Wissenschaftler, Philologe, Archäologe}!W. A. Mozart1856 – 1859@\strich\emph{W. A. Mozart} {[}1856 – 1859{]}|pwv} ganz zu leſen.
               Aber aus dem, was ich geleſen, habe ich den Eindruck gewonnen, daß es ganz einfach
               eine der beſten Biographien iſt, die es gibt. Lies’ das Werk\pwindex{Jahn, Otto 1813-06-16 – 1869-09-09@\textsc{Jahn, Otto} (1813-06-16 – 1869-09-09), \emph{Wissenschaftler, Philologe, Archäologe}!W. A. Mozart1856 – 1859@\strich\emph{W. A. Mozart} {[}1856 – 1859{]}|pwv}. Du wirſt \textsc{Mozart\pwindex{Mozart, Wolfgang Amadeus 27.01.1756 – 05.12.1791@\textsc{Mozart, Wolfgang Amadeus} (27.01.1756 – 05.12.1791), \emph{Komponist}|pw}}{ }{\pb}lieb gewinnen, er wird Dir nahe treten als Wien\oindex{Wien@\textbf{Wien}|pw}er\strikeout{,}{ }\strikeout{als} und als Künſtler. Es iſt ein erſchütterndes
               Ringen in dieſem Leben, das nach dem Dramatiker ruft. Es laſſen ſich ſchöne Dinge
               ſagen über Kunſt und Dummheit und Infamie der Kritik und des Publicums – Dinge, die
               wir oft erlebt haben. Und am Schluß ein großartiges, ergreifendes Sterben, in welches
               das Übernatürliche hineingreift durch die ſo unendlich ſeltſame Geſchichte mit dem
                  \textsc{Requiem\pwindex{Mozart, Wolfgang Amadeus 27.01.1756 – 05.12.1791@\textsc{Mozart, Wolfgang Amadeus} (27.01.1756 – 05.12.1791), \emph{Komponist}!Requiem d-Moll KV 6261791-12-10@\strich\emph{Requiem d-Moll KV 626} {[}1791-12-10{]}|pwv}}. Alles, was Du vom Tode weißt, {\pb}kannſt Du da
               ſagen, und das Publicum \strikeout{dürfte \textcolor{gray}{an}} müßte im Unklaren darüber bleiben, ob der \label{K_L02790-123v}\edtext{geheimnißvolle Mann\pwindex{Walsegg-Stuppach, Franz 1763-01-16 – 1827-11-11@\textsc{Walsegg-Stuppach, Franz} (1763-01-16 – 1827-11-11), \emph{Grundbesitzer}|pw}, der das \textsc{Requiem\pwindex{Mozart, Wolfgang Amadeus 27.01.1756 – 05.12.1791@\textsc{Mozart, Wolfgang Amadeus} (27.01.1756 – 05.12.1791), \emph{Komponist}!Requiem d-Moll KV 6261791-12-10@\strich\emph{Requiem d-Moll KV 626} {[}1791-12-10{]}|pwv}} beſtellt}{\lemma{\textnormal{\emph{geheimnißvolle … beſtellt}}}\Cendnote{\textnormal{Das \emph{Requiem d-Moll}\pwindex{Mozart, Wolfgang Amadeus 27.01.1756 – 05.12.1791@\textsc{Mozart, Wolfgang Amadeus} (27.01.1756 – 05.12.1791), \emph{Komponist}!Requiem d-Moll KV 6261791-12-10@\strich\emph{Requiem d-Moll KV 626} {[}1791-12-10{]}|pwk} (KV 626) wurde von Franz von Walsegg\pwindex{Walsegg-Stuppach, Franz 1763-01-16 – 1827-11-11@\textsc{Walsegg-Stuppach, Franz} (1763-01-16 – 1827-11-11), \emph{Grundbesitzer}|pwk} über Mittelsmänner beauftragt. Dass Mozart\pwindex{Mozart, Wolfgang Amadeus 27.01.1756 – 05.12.1791@\textsc{Mozart, Wolfgang Amadeus} (27.01.1756 – 05.12.1791), \emph{Komponist}|pwk} während der Komposition einer
                  Seelenmesse starb, wurde als Hinweis genommen, bei dem zu dieser Zeit noch
                  verborgenen Auftraggeber hätte es sich um ein übernatürliches Wesen
                  gehandelt.}}}\label{K_L02790-123h}, nicht wirklich aus dem Übernatürlichen herkommt. Und \strikeout{d} um das Alles herum das alte liebe Wien\oindex{Wien@\textbf{Wien}|pw} und ſogar, bitte, der Kaiſer \textsc{Josef\pwindex{Josef II. 13.03.1741 – 20.02.1790@\textsc{Josef II.} (13.03.1741 – 20.02.1790), \emph{Kaiser}|pw}} (der ſich allerdings in der Sache ſehr dumm benommen hat).\pend
           \pstart
           Dieſer Tage ſende ich Dir auch \strikeout{ein} das erſte
               franzöſiſche Buch\pwindex{Constant, Benjamin 23.10.1767 – 08.12.1830@\textsc{Constant, Benjamin} (23.10.1767 – 08.12.1830), \emph{Schriftsteller, Politiker}!Adolphe. Anecdote trouvee dans les papiers Dun inconnu1816@\strich\emph{Adolphe. Anecdote trouvée dans les papiers d’un inconnu} {[}1816{]}|pwv}, das ich
               ſeit Langem mit Genuß geleſen habe (dieſer Satz iſt {\pb}grammatikaliſch ſehr falſch). Es ſtammt natürlich aus dem Jahre 1820 und iſt ganz einfach der größte pſychologiſche Roman, den es gibt:
                  \label{K_L02790-32v}\edtext{»\textsc{Adolphe\pwindex{Constant, Benjamin 23.10.1767 – 08.12.1830@\textsc{Constant, Benjamin} (23.10.1767 – 08.12.1830), \emph{Schriftsteller, Politiker}!Adolphe. Anecdote trouvee dans les papiers Dun inconnu1816@\strich\emph{Adolphe. Anecdote trouvée dans les papiers d’un inconnu} {[}1816{]}|pw}}« von \textsc{Benjamin Constant\pwindex{Constant, Benjamin 23.10.1767 – 08.12.1830@\textsc{Constant, Benjamin} (23.10.1767 – 08.12.1830), \emph{Schriftsteller, Politiker}|pw}}}{\lemma{\textnormal{\emph{»Adolphe« … Constant}}}\Cendnote{\textnormal{Eine zeitnahe Rezeption durch Schnitzler\pwindex{Schnitzler, Arthur 15.05.1862 – 21.10.1931@\textsc{Schnitzler, Arthur} (15.05.1862 – 21.10.1931), \emph{Schriftsteller, Mediziner}|pwk} ist nicht belegt. Er beendete die
                  Lektüre von \emph{Adolphe}\pwindex{Constant, Benjamin 23.10.1767 – 08.12.1830@\textsc{Constant, Benjamin} (23.10.1767 – 08.12.1830), \emph{Schriftsteller, Politiker}!Adolphe. Anecdote trouvee dans les papiers Dun inconnu1816@\strich\emph{Adolphe. Anecdote trouvée dans les papiers d’un inconnu} {[}1816{]}|pwk} am 7. 2. 1906.}}}\label{K_L02790-32h}.
               Freilich ein Buch\pwindex{Constant, Benjamin 23.10.1767 – 08.12.1830@\textsc{Constant, Benjamin} (23.10.1767 – 08.12.1830), \emph{Schriftsteller, Politiker}!Adolphe. Anecdote trouvee dans les papiers Dun inconnu1816@\strich\emph{Adolphe. Anecdote trouvée dans les papiers d’un inconnu} {[}1816{]}|pwv} ohne Wärme,
               aber wie aus Erz gegoſſen, – nicht ein Wort zu viel, nicht eines zu wenig – die
               unerbittlichſte Analyſe\pwindex{Constant, Benjamin 23.10.1767 – 08.12.1830@\textsc{Constant, Benjamin} (23.10.1767 – 08.12.1830), \emph{Schriftsteller, Politiker}!Adolphe. Anecdote trouvee dans les papiers Dun inconnu1816@\strich\emph{Adolphe. Anecdote trouvée dans les papiers d’un inconnu} {[}1816{]}|pwv} eines
               ſchwachen Characters, die je ausgeführt worden. Und wenn man bedenkt, daß \strikeout{\textcolor{gray}{m}} wir \strikeout{hinterher}{ }{\pb}\textsc{Paul Bourget\pwindex{Bourget, Paul 02.09.1852 – 25.12.1935@\textsc{Bourget, Paul} (02.09.1852 – 25.12.1935), \emph{Schriftsteller}|pw}} bewundert haben, nachdem es einen »\textsc{Adolphe\pwindex{Constant, Benjamin 23.10.1767 – 08.12.1830@\textsc{Constant, Benjamin} (23.10.1767 – 08.12.1830), \emph{Schriftsteller, Politiker}!Adolphe. Anecdote trouvee dans les papiers Dun inconnu1816@\strich\emph{Adolphe. Anecdote trouvée dans les papiers d’un inconnu} {[}1816{]}|pwv}}« gegeben hat!\pend
           \pstart
           Grüß’ Dich Gott, mein lieber Freund!\pend
           \pstart
           Schreib’ mir bald!\pend
           \pstart
           In Treue {\\[\baselineskip]}Dein {\\[\baselineskip]}\spacefill\mbox{Paul Goldmann.}\pend
           \leftskip=0em{}\pstart
           \noindent{}Wenn Du den \strikeout{\textsc{L\textcolor{gray}{eo}}}{ }\label{K_L02790-24v}\edtext{\textsc{Leo Fanjung}\pwindex{Van-Jung, Leo 15.10.1866 – 02.07.1939@\textsc{Van-Jung, Leo} (15.10.1866 – 02.07.1939), \emph{Gesangspädagoge, Mathematiker}|pw} ſiehſt}{\lemma{\textnormal{\emph{Leo Fanjung ſiehſt}}}\Cendnote{\textnormal{Das nächste belegte
                     Zusammentreffen von Schnitzler\pwindex{Schnitzler, Arthur 15.05.1862 – 21.10.1931@\textsc{Schnitzler, Arthur} (15.05.1862 – 21.10.1931), \emph{Schriftsteller, Mediziner}|pwk} und Leo Van-Jung\pwindex{Van-Jung, Leo 15.10.1866 – 02.07.1939@\textsc{Van-Jung, Leo} (15.10.1866 – 02.07.1939), \emph{Gesangspädagoge, Mathematiker}|pwk} fand am 22. 11. 1896
                     statt.}}}\label{K_L02790-24h}, ſo grüß’ ihn, bitte.\pend
           
         
         \endnumbering\mylabel{h}\end{ledgroupsized}  \newcommand{\dateiname}{L02790}\newcommand{\titel}{Paul Goldmann an Arthur Schnitzler, 13. 11. [1896]}\newcommand{\editorInnen}{Martin Anton Müller und Laura Untner}%% latex-leseansicht-abspann.tex
%% Abspann für die Leseansicht.
%% Der Schalter \ifkorrekturansicht ist bereits durch den Vorspann gesetzt.

%% latex-abspann.tex
%% Gemeinsamer Abspann für Korrekturansicht und Leseansicht.
%% Setzt den Schalter \ifkorrekturansicht voraus (gesetzt in den
%% einbindenden Dateien latex-korrekturansicht-abspann.tex bzw.
%% latex-leseansicht-abspann.tex).
%% ---------------------------------------------------------------

\normalsize

% Das esempio-Environment wird nur in der Leseansicht benötigt
\ifkorrekturansicht\else
\newenvironment{esempio}[3]%
{
    \vspace{1.5ex}
    \rlap{\underline{#1}}
    \par
    \setlength{\parindent}{0cm}
    \nopagebreak
    \leftskip=#2cm
    \rightskip=#3cm
}
{
    \par
}
\fi

\doendnotes{C}
\bigskip
\vfill

\clearpage

\footnotesize

\ifkorrekturansicht
  \lohead{\textsc{register}}
\fi

% theindex-Environment neu definieren ohne reledmac
\makeatletter
\renewenvironment{theindex}{%
  \ifkorrekturansicht
    \section*{\indexname}%
  \else
    \subsubsection*{Index der erwähnten Entitäten}%
  \fi
  \setlength{\parindent}{0pt}%
  \setlength{\parskip}{0pt plus 0.3pt}%
  \let\item\@idxitem
}{%
  \ifkorrekturansicht\clearpage\fi
}
\makeatother

\IfFileExists{\jobname-pw.ind}{\input{\jobname-pw.ind}}{}

% Quellenangabe nur in der Leseansicht
\ifkorrekturansicht\else
% Fallback-Definitionen, falls die .tex-Datei \titel etc. nicht gesetzt hat
\providecommand{\titel}{}
\providecommand{\editorInnen}{}
\providecommand{\dateiname}{\jobname}

\vspace{3cm}

\vfill

\footnotesize
\textsc{Quelle}: \titel. Herausgegeben von {\editorInnen}. In: \emph{Arthur Schnitzler: Briefwechsel mit Autorinnen und Autoren}.
 Digitale Edition, https://schnitzler-briefe.acdh.oeaw.ac.at/{\dateiname}.html (Stand \today)
\fi

\end{document}


      