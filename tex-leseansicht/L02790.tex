%% latex-leseansicht-vorspann.tex
%% Vorspann für die Leseansicht.
%% Lädt die gemeinsame Datei latex-vorspann.tex mit nicht gesetztem Schalter.

\newif\ifkorrekturansicht
\korrekturansichtfalse

\input{../tex-inputs/latex-vorspann}


\section[ Paul Goldmann an Arthur Schnitzler, 13. 11. {[}1896{]}]{L02790 Paul Goldmann an Arthur Schnitzler,  13. 11. [1896]}
\nopagebreak\mylabel{L02790v}
\rehead{ }\normalsize\beginnumbering\briefempfaengerindex{Schnitzler, Arthur@\textsc{Schnitzler, Arthur}!zzzGoldmann, Paul@\emph{von Paul Goldmann}!1896-11-131@{13. 11. [1896]}|(be}
\toendnotes[C]{\smallbreak\pagebreak[2]}
\correspDesc{Versand  durch Paul Goldmann am 13. 11. [1896] in Paris
\newline{}Erhalt  durch Arthur Schnitzler im Zeitraum [14. 11. 1896 – 18. 11. 1896?] in Wien}\toendnotes[C]{\smallbreak}
\Standort{DLA, A:Schnitzler, HS.NZ85.1.3166.}
\physDesc{Brief, 3 Blätter, 10 Seiten, 4309 Zeichen
\newline{}Handschrift: blaue Tinte, deutsche Kurrent
\newline{}Beilage: zwei beschnittene und zusammengeklebte Zeitungsausschnitte auf
                                 der ersten Seite, der eine aus der Kopfzeile bestehend 
\newline{}Schnitzler: 1) mit Bleistift das Jahr »96« vermerkt  2) mit rotem Buntstift eine Unterstreichung}\toendnotes[C]{\smallbreak}
\pstart
           {\pb}\textcolor{gray}{\textbf{\textbf{Frankfurter Zeitung\orgindex{Frankfurter Zeitung@Frankfurter Zeitung|pw}}}}\pend
           
\pstart
           \textcolor{gray}{\textbf{(\begin{otherlanguage}{french}Gazette de Francfort\end{otherlanguage}\orgindex{Frankfurter Zeitung@Frankfurter Zeitung|pw}).}}\pend
           
\pstart
           \textcolor{gray}{\textbf{\textbf{\begin{otherlanguage}{french}Fondateur M.\end{otherlanguage}{ }L. Sonnemann\pwindex{Sonnemann, Leopold 29.\,10.\,1831 Höchberg – 30.\,10.\,1909 Frankfurt am Main@\textsc{Sonnemann, Leopold} (29.\,10.\,1831 Höchberg – 30.\,10.\,1909 Frankfurt am Main), \emph{Journalist, Herausgeber}|pw}.}}}\pend
           
\pstart
           \begin{otherlanguage}{french}\textcolor{gray}{\textbf{Journal\pwindex{Frankfurter Zeitung@\emph{Frankfurter Zeitung}|pwv} politique,
                        financier,}}\end{otherlanguage}\pend
           
\pstart
           \begin{otherlanguage}{french}\textcolor{gray}{\textbf{commercial et littéraire.}}\end{otherlanguage}\pend
           
\pstart
           \begin{otherlanguage}{french}\textcolor{gray}{\textbf{\textbf{Paraissant trois fois par jour.}}}\end{otherlanguage}\pend
           
\pstart
           \begin{otherlanguage}{french}\textcolor{gray}{\textbf{\textbf{Bureau à Paris\oindex{Paris@\textbf{Paris}, \emph{Hauptstadt}|pw}}}}\end{otherlanguage}\pend
           
\pstart
           \begin{otherlanguage}{french}\textcolor{gray}{\textbf{\textbf{24. Rue Feydeau\oindex{rue Feydeau@\textbf{rue Feydeau}, \emph{Straße}|pw}.}}}\end{otherlanguage}\pend
           {\vspace{1\baselineskip}}
\pstart
           \centering{}\label{K_L02790-1v}\edtext{\begin{otherlanguage}{french}\textcolor{gray}{\textbf{LE FIGARO\orgindex{Le Figaro@Le Figaro|pw}{ }MARDI 10 NOVEMBRE}}\end{otherlanguage}}{\lemma{\textnormal{\emph{Le … Mardi 10 Novembre}}}\Cendnote{\textnormal{französisch: \emph{Le Figaro}\orgindex{Le Figaro@Le Figaro|pwk}{ }Dienstag, 10. November}}}\label{K_L02790-1}\pend
           
\pstart
           \begin{otherlanguage}{french}\textcolor{gray}{\textbf{\label{K_L02790-2v}\edtext{Mon cher}{\lemma{\textnormal{\emph{Mon cher}}}\Cendnote{\textnormal{französisch: mein lieber,}}}\label{K_L02790-2}{ }Huret\pwindex{Huret, Jules 8.\,4.\,1863 Boulogne-sur-Mer – 14.\,2.\,1915 Paris@\textsc{Huret, Jules} (8.\,4.\,1863 Boulogne-sur-Mer – 14.\,2.\,1915 Paris), \emph{Schriftsteller, Journalist, Publizist}|pw},}}\end{otherlanguage}\pend
           
\pstart
           \label{K_L02790-3v}\edtext{\begin{otherlanguage}{french}\textcolor{gray}{\textbf{Pour compléter vos \label{K_L02790-4v}\edtext{renseignements\pwindex{?? [Berliner Korrespondent des Figaro] @\textsc{?? [Berliner Korrespondent des Figaro]}!Courrier des Théâtres [Freiwild in Berlin und Liebelei]@\strich\emph{Courrier des Théâtres [Freiwild in Berlin und Liebelei]}|pwv}}{\lemma{\textnormal{\emph{renseignements}}}\Cendnote{\textnormal{Jules Huret\pwindex{Huret, Jules 8.\,4.\,1863 Boulogne-sur-Mer – 14.\,2.\,1915 Paris@\textsc{Huret, Jules} (8.\,4.\,1863 Boulogne-sur-Mer – 14.\,2.\,1915 Paris), \emph{Schriftsteller, Journalist, Publizist}|pwk} leitete die
                           Theaterrubrik des \emph{Figaro}\pwindex{Le Figaro@\emph{Le Figaro}|pwk}. Das Telegramm\pwindex{?? [Berliner Korrespondent des Figaro] @\textsc{?? [Berliner Korrespondent des Figaro]}!Courrier des Théâtres [Freiwild in Berlin und Liebelei]@\strich\emph{Courrier des Théâtres [Freiwild in Berlin und Liebelei]}|pwkv} des Berliner
                              Korrespondenten\pwindex{?? [Berliner Korrespondent des Figaro] @\textsc{?? [Berliner Korrespondent des Figaro]}|pwkv} wurde abgedruckt: \emph{Le Figaro}\pwindex{Le Figaro@\emph{Le Figaro}|pwk}, Jg. 42, Nr. 312, 7. 11. 1896, S. 4.}}}\label{K_L02790-4} sur
                        Arthur Schnitzler, laissez-moi vous dire que je viens de terminer la traduction\pwindex{Schnitzler, Arthur 15.\,5.\,1862 Wien – 21.\,10.\,1931 ebd.@\textsc{Schnitzler, Arthur} (15.\,5.\,1862 Wien – 21.\,10.\,1931 ebd.), \emph{Schriftsteller, Mediziner}!Amourette. Pièce en trois actes. Adaptée de Arthur Schnitzler@\strich\emph{Amourette. Pièce en trois actes. Adaptée de Arthur Schnitzler}|pwv} en français
                        de cette \emph{\label{T_L02790-1v}\edtext{Liebelei}{\lemma{\textnormal{\emph{Liebelei}}}\Cendnote{\textnormal{Im gedruckten Text steht: »Liebelci\pwindex{Schnitzler, Arthur 15.\,5.\,1862 Wien – 21.\,10.\,1931 ebd.@\textsc{Schnitzler, Arthur} (15.\,5.\,1862 Wien – 21.\,10.\,1931 ebd.), \emph{Schriftsteller, Mediziner}!Liebelei. Schauspiel in drei Akten@\strich\emph{Liebelei. Schauspiel in drei Akten}|pwv}.
                                 «}}}\label{T_L02790-1}\pwindex{Schnitzler, Arthur 15.\,5.\,1862 Wien – 21.\,10.\,1931 ebd.@\textsc{Schnitzler, Arthur} (15.\,5.\,1862 Wien – 21.\,10.\,1931 ebd.), \emph{Schriftsteller, Mediziner}!Liebelei. Schauspiel in drei Akten@\strich\emph{Liebelei. Schauspiel in drei Akten}|pw}} dont vous rappelez le grand succès, l’hiver dernier, à Vienne\oindex{Wien@\textbf{Wien}, \emph{Verwaltungsgebiet}|pw}.}}\end{otherlanguage}}{\lemma{\textnormal{\emph{Pour … Vienne.}}}\Cendnote{\textnormal{französisch: Um Ihre Auskünfte\pwindex{?? [Berliner Korrespondent des Figaro] @\textsc{?? [Berliner Korrespondent des Figaro]}!Courrier des Théâtres [Freiwild in Berlin und Liebelei]@\strich\emph{Courrier des Théâtres [Freiwild in Berlin und Liebelei]}|pwkv} über Arthur Schnitzler zu vervollständigen, möchte ich
                     kundtun, dass ich gerade die französische Übersetzung\pwindex{Schnitzler, Arthur 15.\,5.\,1862 Wien – 21.\,10.\,1931 ebd.@\textsc{Schnitzler, Arthur} (15.\,5.\,1862 Wien – 21.\,10.\,1931 ebd.), \emph{Schriftsteller, Mediziner}!Amourette. Pièce en trois actes. Adaptée de Arthur Schnitzler@\strich\emph{Amourette. Pièce en trois actes. Adaptée de Arthur Schnitzler}|pwkv} von \emph{Liebelei}\pwindex{Schnitzler, Arthur 15.\,5.\,1862 Wien – 21.\,10.\,1931 ebd.@\textsc{Schnitzler, Arthur} (15.\,5.\,1862 Wien – 21.\,10.\,1931 ebd.), \emph{Schriftsteller, Mediziner}!Liebelei. Schauspiel in drei Akten@\strich\emph{Liebelei. Schauspiel in drei Akten}|pwk} abgeschlossen habe, an deren großen Erfolg in Wien\oindex{Wien@\textbf{Wien}, \emph{Verwaltungsgebiet}|pwk} im letzten Winter Sie sich
                     erinnern.}}}\label{K_L02790-3}\pend
           
\pstart
           \label{K_L02790-5v}\edtext{\begin{otherlanguage}{french}\textcolor{gray}{\textbf{Déjà \label{K_L02790-6v}\edtext{deux de nos
                           directeurs de théâtre\pwindex{Ginisty, Paul 4.\,4.\,1855 Paris – 5.\,3.\,1932 ebd.@\textsc{Ginisty, Paul} (4.\,4.\,1855 Paris – 5.\,3.\,1932 ebd.), \emph{Schriftsteller, Theaterleiter}|pwv}\pwindex{Carré, Albert 22.\,6.\,1852 Straßburg – 11.\,12.\,1938 Paris@\textsc{Carré, Albert} (22.\,6.\,1852 Straßburg – 11.\,12.\,1938 Paris), \emph{Schriftsteller, Theaterleiter, Schauspieler}|pwv}}{\lemma{\textnormal{\emph{deux … théâtre}}}\Cendnote{\textnormal{Vgl. XXXX Auszeichnungsfehler: Dokument L02792 nicht gefunden.
                        }}}\label{K_L02790-6} m’ont promis{\dots} de lire cette traduction\pwindex{Schnitzler, Arthur 15.\,5.\,1862 Wien – 21.\,10.\,1931 ebd.@\textsc{Schnitzler, Arthur} (15.\,5.\,1862 Wien – 21.\,10.\,1931 ebd.), \emph{Schriftsteller, Mediziner}!Amourette. Pièce en trois actes. Adaptée de Arthur Schnitzler@\strich\emph{Amourette. Pièce en trois actes. Adaptée de Arthur Schnitzler}|pwv}. Ai-je
                        besoin d’ajouter qu’ils se proposent même de faire cette lecture »avec le
                        plus vif intérêt«.}}\end{otherlanguage}}{\lemma{\textnormal{\emph{Déjà … intérêt«.}}}\Cendnote{\textnormal{französisch: Zwei unserer
                        Theaterdirektoren\pwindex{Ginisty, Paul 4.\,4.\,1855 Paris – 5.\,3.\,1932 ebd.@\textsc{Ginisty, Paul} (4.\,4.\,1855 Paris – 5.\,3.\,1932 ebd.), \emph{Schriftsteller, Theaterleiter}|pwkv}\pwindex{Carré, Albert 22.\,6.\,1852 Straßburg – 11.\,12.\,1938 Paris@\textsc{Carré, Albert} (22.\,6.\,1852 Straßburg – 11.\,12.\,1938 Paris), \emph{Schriftsteller, Theaterleiter, Schauspieler}|pwkv} haben mir bereits versprochen, die Übersetzung\pwindex{Schnitzler, Arthur 15.\,5.\,1862 Wien – 21.\,10.\,1931 ebd.@\textsc{Schnitzler, Arthur} (15.\,5.\,1862 Wien – 21.\,10.\,1931 ebd.), \emph{Schriftsteller, Mediziner}!Amourette. Pièce en trois actes. Adaptée de Arthur Schnitzler@\strich\emph{Amourette. Pièce en trois actes. Adaptée de Arthur Schnitzler}|pwkv} zu lesen. Muss ich noch
                     hinzufügen, dass sie diese Lektüre »mit dem lebhaftesten Interesse«
                     unternehmen?}}}\label{K_L02790-5}\pend
           
\pstart
           \raggedleft{}\label{K_L02790-7v}\edtext{\begin{otherlanguage}{french}\textcolor{gray}{\textbf{Votre bien dévoué,}}\end{otherlanguage}}{\lemma{\textnormal{\emph{Votre bien dévoué,}}}\Cendnote{\textnormal{französisch: Ihr sehr
                     ergebener}}}\label{K_L02790-7}\pend
           
\pstart
           \raggedleft{}\textcolor{gray}{\textbf{Jean THOREL\pwindex{Thorel, Jean 11.\,9.\,1859 Éragny – 20.\,8.\,1916 Enghien-les-Bains@\textsc{Thorel, Jean} (11.\,9.\,1859 Éragny – 20.\,8.\,1916 Enghien-les-Bains), \emph{Übersetzer, Dramatiker}|pw}.}}\pend
           {\vspace{1\baselineskip}}
\pstart
           \raggedleft{}\textsc{Paris\oindex{Paris@\textbf{Paris}, \emph{Hauptstadt}|pw}}, 13. November.\pend
           
\pstart\center{}Mein lieber Freund,\pend\vspace{0.5em}
\pstart
           Oben{ }ſiehſt Du einen \label{K_L02790-8v}\edtext{Ausſchnitt\pwindex{Thorel, Jean 11.\,9.\,1859 Éragny – 20.\,8.\,1916 Enghien-les-Bains@\textsc{Thorel, Jean} (11.\,9.\,1859 Éragny – 20.\,8.\,1916 Enghien-les-Bains), \emph{Übersetzer, Dramatiker}!Courrier des Théâtres [Mon cher Huret; Thorel zur Liebelei-Übersetzung]@\strich\emph{Courrier des Théâtres [Mon cher Huret; Thorel zur Liebelei-Übersetzung]}|pwv} aus dem »\textsc{Figaro\pwindex{Le Figaro@\emph{Le Figaro}|pw}}«}{\lemma{\textnormal{\emph{Ausschnitt … »Figaro«}}}\Cendnote{\textnormal{Jean Thorel\pwindex{Thorel, Jean 11.\,9.\,1859 Éragny – 20.\,8.\,1916 Enghien-les-Bains@\textsc{Thorel, Jean} (11.\,9.\,1859 Éragny – 20.\,8.\,1916 Enghien-les-Bains), \emph{Übersetzer, Dramatiker}|pwk}: \emph{[Mon cher Huret]}\pwindex{Thorel, Jean 11.\,9.\,1859 Éragny – 20.\,8.\,1916 Enghien-les-Bains@\textsc{Thorel, Jean} (11.\,9.\,1859 Éragny – 20.\,8.\,1916 Enghien-les-Bains), \emph{Übersetzer, Dramatiker}!Courrier des Théâtres [Mon cher Huret; Thorel zur Liebelei-Übersetzung]@\strich\emph{Courrier des Théâtres [Mon cher Huret; Thorel zur Liebelei-Übersetzung]}|pwk}. In: \emph{Le Figaro}\pwindex{Le Figaro@\emph{Le Figaro}|pwk}, Jg. 42, Nr. 315, 10. 11. 1896, S. 4.}}}\label{K_L02790-8}. Die Überſetzung\pwindex{Schnitzler, Arthur 15.\,5.\,1862 Wien – 21.\,10.\,1931 ebd.@\textsc{Schnitzler, Arthur} (15.\,5.\,1862 Wien – 21.\,10.\,1931 ebd.), \emph{Schriftsteller, Mediziner}!Amourette. Pièce en trois actes. Adaptée de Arthur Schnitzler@\strich\emph{Amourette. Pièce en trois actes. Adaptée de Arthur Schnitzler}|pwv} von \textsc{Thorel\pwindex{Thorel, Jean 11.\,9.\,1859 Éragny – 20.\,8.\,1916 Enghien-les-Bains@\textsc{Thorel, Jean} (11.\,9.\,1859 Éragny – 20.\,8.\,1916 Enghien-les-Bains), \emph{Übersetzer, Dramatiker}|pw}} iſt \strikeout{\textcolor{gray}{×}} – unter uns geſagt – leider recht{ }ſchlecht, noch{ }ſchlechter, als ich geglaubt.
               Er hat{ }ſich gar keine Mühe gegeben, \strikeout{d\textcolor{gray}{ie}} das natürliche und lebendige Deutſch des Dialog\pwindex{Schnitzler, Arthur 15.\,5.\,1862 Wien – 21.\,10.\,1931 ebd.@\textsc{Schnitzler, Arthur} (15.\,5.\,1862 Wien – 21.\,10.\,1931 ebd.), \emph{Schriftsteller, Mediziner}!Liebelei. Schauspiel in drei Akten@\strich\emph{Liebelei. Schauspiel in drei Akten}|pwv}es in natürliches und lebendiges Franzöſiſch
               umzuſetzen. Ich tröſte mich damit, daß es ein Anderer noch{ }ſchlechter gemacht hätte.
                  {\pb}Auch rechne ich auf die dem Stücke\pwindex{Schnitzler, Arthur 15.\,5.\,1862 Wien – 21.\,10.\,1931 ebd.@\textsc{Schnitzler, Arthur} (15.\,5.\,1862 Wien – 21.\,10.\,1931 ebd.), \emph{Schriftsteller, Mediziner}!Liebelei. Schauspiel in drei Akten@\strich\emph{Liebelei. Schauspiel in drei Akten}|pwv} innewohnende Poeſie, die{ }ſich beim
               beſten Willen nicht umbringen läßt{\dotsfive}\pend
           
\pstart
           Mit Deinem lieben Briefe habe ich mich{ }ſehr gefreut. Ich begreife Deine Stimmung, und
               da Du Dir gewiß über die Gründe klar biſt, wird auch dieſes zweite Stück\pwindex{Schnitzler, Arthur 15.\,5.\,1862 Wien – 21.\,10.\,1931 ebd.@\textsc{Schnitzler, Arthur} (15.\,5.\,1862 Wien – 21.\,10.\,1931 ebd.), \emph{Schriftsteller, Mediziner}!Freiwild. Schauspiel in 3 Akten@\strich\emph{Freiwild. Schauspiel in 3 Akten}|pwv} für Deine Entwickelung nützlich{ }ſein.
                  \label{K_L02790-9v}\edtext{Das Stück\pwindex{Schnitzler, Arthur 15.\,5.\,1862 Wien – 21.\,10.\,1931 ebd.@\textsc{Schnitzler, Arthur} (15.\,5.\,1862 Wien – 21.\,10.\,1931 ebd.), \emph{Schriftsteller, Mediziner}!Freiwild. Schauspiel in 3 Akten@\strich\emph{Freiwild. Schauspiel in 3 Akten}|pwv} iſt Dir unſympathiſch}{\lemma{\textnormal{\emph{Das … unsympathisch}}}\Cendnote{\textnormal{Siehe A. S.: \emph{Tagebuch}, 5. 11. 1896.
               }}}\label{K_L02790-9}, weil es nicht Deiner Natur und Deiner Schaffensart entſpricht. Es iſt nicht
               aus dem Leben herausgewachſen,{ }ſondern aus einer Idee, zu der hinterdrein die Figuren
               geſucht wurden. Beſonders {\pb}ſieht man das an dem Helden\pwindex{Schnitzler, Arthur 15.\,5.\,1862 Wien – 21.\,10.\,1931 ebd.@\textsc{Schnitzler, Arthur} (15.\,5.\,1862 Wien – 21.\,10.\,1931 ebd.), \emph{Schriftsteller, Mediziner}!Freiwild. Schauspiel in 3 Akten@\strich\emph{Freiwild. Schauspiel in 3 Akten}|pwv}. Den haſt Du nie
               geſehen. Du haſt ihn Dir künſtlich zuſammenzimmern müſſen, damit er zu Deiner Idee
               paßt. Darum biſt Du{ }ſo unſicher bei{ }ſeiner Geſtaltung geweſen, darum iſt er Dir{ }ſo{ }ſchwer gefallen, darum iſt er auch heut nicht recht gelungen. Und der Hauptfehler
               war: Es war ein Tendenzſtück, und Du haſt Dir das nicht eingeſtehen wollen und haſt
               es nicht als Tendenzſtück{ }ſchreiben wollen. Es war ein Tendenzſtück, das{ }ſo ausſehen{ }ſollte, als{ }ſei es natürlich {\pb}und erlebt. Das iſt
               unmöglich. Die \label{K_L02790-10v}\edtext{\textsc{procédés}}{\lemma{\textnormal{\emph{procédés}}}\Cendnote{\textnormal{französisch: das Prozedere}}}\label{K_L02790-10}
               Deiner Kunſt, die Natürliches und Erlebtes ausdrücken will und kann, waren hier im
               Zwieſpalt mit den Anforderungen des \textsc{Sujets}. Gerade die
               Unparteilichkeit halte ich für einen Fehler des Stück\pwindex{Schnitzler, Arthur 15.\,5.\,1862 Wien – 21.\,10.\,1931 ebd.@\textsc{Schnitzler, Arthur} (15.\,5.\,1862 Wien – 21.\,10.\,1931 ebd.), \emph{Schriftsteller, Mediziner}!Freiwild. Schauspiel in 3 Akten@\strich\emph{Freiwild. Schauspiel in 3 Akten}|pwv}es. Es mußte parteilich{ }ſein. Es mußte ein Stück\pwindex{Schnitzler, Arthur 15.\,5.\,1862 Wien – 21.\,10.\,1931 ebd.@\textsc{Schnitzler, Arthur} (15.\,5.\,1862 Wien – 21.\,10.\,1931 ebd.), \emph{Schriftsteller, Mediziner}!Freiwild. Schauspiel in 3 Akten@\strich\emph{Freiwild. Schauspiel in 3 Akten}|pwv} werden gegen das Duell.
               Für dieſes Stück\pwindex{Schnitzler, Arthur 15.\,5.\,1862 Wien – 21.\,10.\,1931 ebd.@\textsc{Schnitzler, Arthur} (15.\,5.\,1862 Wien – 21.\,10.\,1931 ebd.), \emph{Schriftsteller, Mediziner}!Freiwild. Schauspiel in 3 Akten@\strich\emph{Freiwild. Schauspiel in 3 Akten}|pwv} mußteſt Du
               Deine bisherige Productions-Art beiſeite laſſen und \introOben{}Du\introOben{}
               mußteſt es mit Haß und Leidenſchaft{ }ſchreiben, \strikeout{g} ganz
               ohne Rückſicht darauf, ob es unwahrſcheinlich und {\pb}ungerecht wurde. Ich meine, Du{ }ſollſt fürs Erſte von allen Stoffen dieſer Art, von
               allen »großen Zeitfragen« \textsc{etc.} laſſen. Ich möchte Dir jetzt
               gerade einen \strikeout{\textcolor{gray}{×}\-\textcolor{gray}{×}\-\textcolor{gray}{×}\-\textcolor{gray}{×}\-\textcolor{gray}{×}\-\textcolor{gray}{×}\-\textcolor{gray}{×}\-\textcolor{gray}{×}} Wanderzug in die Vergangenheit und in die reine Poeſie empfehlen. \uline{Das hiſtoriſche Wien\oindex{Wien@\textbf{Wien}, \emph{Verwaltungsgebiet}|pw}er
                  Stück!} Jetzt mußt Du es{ }ſchreiben, und ich bin überzeugt, es wird Dir
               köſtlich gelingen. Nimm’ Dir zwei oder drei Jahre Zeit und ruhe Dich ein wenig auf
               den zwei{ }ſtarken Erfolgen aus, durch welche Du mit einem {\pb}Male in die allererſte Reihe unter den deutſchen
               Bühnen-Dichtern gerückt biſt. Ich möchte Dir einen{ }ſchönen Stoff vorſchlagen: \uline{\textsc{Mozart\pwindex{Mozart, Wolfgang Amadeus 27.\,1.\,1756 Salzburg – 5.\,12.\,1791 Wien@\textsc{Mozart, Wolfgang Amadeus} (27.\,1.\,1756 Salzburg – 5.\,12.\,1791 Wien), \emph{Komponist}|pw}}}, ein Wien\oindex{Wien@\textbf{Wien}, \emph{Verwaltungsgebiet}|pw}er Volksſtück mit \textsc{Mozart\pwindex{Mozart, Wolfgang Amadeus 27.\,1.\,1756 Salzburg – 5.\,12.\,1791 Wien@\textsc{Mozart, Wolfgang Amadeus} (27.\,1.\,1756 Salzburg – 5.\,12.\,1791 Wien), \emph{Komponist}|pw}}’ſcher Muſik. Ich hatte neulich Gelegenheit, \textsc{Otto Jahns\pwindex{Jahn, Otto 16.\,6.\,1813 Kiel – 9.\,9.\,1869 Göttingen@\textsc{Jahn, Otto} (16.\,6.\,1813 Kiel – 9.\,9.\,1869 Göttingen), \emph{Musikwissenschaftler, Philologe, Archäologe}|pw}}{ }\textsc{Mozart\pwindex{Mozart, Wolfgang Amadeus 27.\,1.\,1756 Salzburg – 5.\,12.\,1791 Wien@\textsc{Mozart, Wolfgang Amadeus} (27.\,1.\,1756 Salzburg – 5.\,12.\,1791 Wien), \emph{Komponist}|pw}}-Biographie\pwindex{Jahn, Otto 16.\,6.\,1813 Kiel – 9.\,9.\,1869 Göttingen@\textsc{Jahn, Otto} (16.\,6.\,1813 Kiel – 9.\,9.\,1869 Göttingen), \emph{Musikwissenschaftler, Philologe, Archäologe}!W. A. Mozart@\strich\emph{W. A. Mozart}|pwv} einzuſehen. Natürlich hatte ich keine Zeit, die beiden
               dicken Bände\pwindex{Jahn, Otto 16.\,6.\,1813 Kiel – 9.\,9.\,1869 Göttingen@\textsc{Jahn, Otto} (16.\,6.\,1813 Kiel – 9.\,9.\,1869 Göttingen), \emph{Musikwissenschaftler, Philologe, Archäologe}!W. A. Mozart@\strich\emph{W. A. Mozart}|pwv} ganz zu leſen.
               Aber aus dem, was ich geleſen, habe ich den Eindruck gewonnen, daß es ganz einfach
               eine der beſten Biographien iſt, die es gibt. Lies’ das Werk\pwindex{Jahn, Otto 16.\,6.\,1813 Kiel – 9.\,9.\,1869 Göttingen@\textsc{Jahn, Otto} (16.\,6.\,1813 Kiel – 9.\,9.\,1869 Göttingen), \emph{Musikwissenschaftler, Philologe, Archäologe}!W. A. Mozart@\strich\emph{W. A. Mozart}|pwv}. Du wirſt \textsc{Mozart\pwindex{Mozart, Wolfgang Amadeus 27.\,1.\,1756 Salzburg – 5.\,12.\,1791 Wien@\textsc{Mozart, Wolfgang Amadeus} (27.\,1.\,1756 Salzburg – 5.\,12.\,1791 Wien), \emph{Komponist}|pw}}{ }{\pb}lieb gewinnen, er wird Dir nahe treten als Wien\oindex{Wien@\textbf{Wien}, \emph{Verwaltungsgebiet}|pw}er\strikeout{,}{ }\strikeout{als} und als Künſtler. Es iſt ein erſchütterndes
               Ringen in dieſem Leben, das nach dem Dramatiker ruft. Es laſſen{ }ſich{ }ſchöne Dinge{ }ſagen über Kunſt und Dummheit und Infamie der Kritik und des Publicums – Dinge, die
               wir oft erlebt haben. Und am Schluß ein großartiges, ergreifendes Sterben, in welches
               das Übernatürliche hineingreift durch die{ }ſo unendlich{ }ſeltſame Geſchichte mit dem
                  \textsc{Requiem\pwindex{Mozart, Wolfgang Amadeus 27.\,1.\,1756 Salzburg – 5.\,12.\,1791 Wien@\textsc{Mozart, Wolfgang Amadeus} (27.\,1.\,1756 Salzburg – 5.\,12.\,1791 Wien), \emph{Komponist}!Requiem d-Moll KV 626@\strich\emph{Requiem d-Moll KV 626}|pwv}}. Alles, was Du vom Tode weißt, {\pb}kannſt Du da{ }ſagen, und das Publicum \strikeout{dürfte \textcolor{gray}{an}} müßte im Unklaren darüber bleiben, ob der \label{K_L02790-11v}\edtext{geheimnißvolle Mann\pwindex{Walsegg-Stuppach, Franz 16.\,1.\,1763 Stuppach – 11.\,11.\,1827 ebd.@\textsc{Walsegg-Stuppach, Franz} (16.\,1.\,1763 Stuppach – 11.\,11.\,1827 ebd.), \emph{Grundbesitzer}|pw}, der das \textsc{Requiem\pwindex{Mozart, Wolfgang Amadeus 27.\,1.\,1756 Salzburg – 5.\,12.\,1791 Wien@\textsc{Mozart, Wolfgang Amadeus} (27.\,1.\,1756 Salzburg – 5.\,12.\,1791 Wien), \emph{Komponist}!Requiem d-Moll KV 626@\strich\emph{Requiem d-Moll KV 626}|pwv}} beſtellt}{\lemma{\textnormal{\emph{geheimnißvolle … bestellt}}}\Cendnote{\textnormal{Das \emph{Requiem d-Moll}\pwindex{Mozart, Wolfgang Amadeus 27.\,1.\,1756 Salzburg – 5.\,12.\,1791 Wien@\textsc{Mozart, Wolfgang Amadeus} (27.\,1.\,1756 Salzburg – 5.\,12.\,1791 Wien), \emph{Komponist}!Requiem d-Moll KV 626@\strich\emph{Requiem d-Moll KV 626}|pwk} (KV 626) wurde von Franz von Walsegg\pwindex{Walsegg-Stuppach, Franz 16.\,1.\,1763 Stuppach – 11.\,11.\,1827 ebd.@\textsc{Walsegg-Stuppach, Franz} (16.\,1.\,1763 Stuppach – 11.\,11.\,1827 ebd.), \emph{Grundbesitzer}|pwk} über Mittelsmänner beauftragt. Dass Mozart\pwindex{Mozart, Wolfgang Amadeus 27.\,1.\,1756 Salzburg – 5.\,12.\,1791 Wien@\textsc{Mozart, Wolfgang Amadeus} (27.\,1.\,1756 Salzburg – 5.\,12.\,1791 Wien), \emph{Komponist}|pwk} während der Komposition einer
                  Seelenmesse starb, wurde als Hinweis genommen, bei dem zu dieser Zeit noch
                  verborgenen Auftraggeber hätte es sich um ein übernatürliches Wesen
                  gehandelt.}}}\label{K_L02790-11}, nicht wirklich aus dem Übernatürlichen herkommt. Und \strikeout{d} um das Alles herum das alte liebe Wien\oindex{Wien@\textbf{Wien}, \emph{Verwaltungsgebiet}|pw} und{ }ſogar, bitte, der Kaiſer \textsc{Josef\pwindex{Josef II. 13.\,3.\,1741 Wien – 20.\,2.\,1790 ebd.@\textsc{Josef II.} (13.\,3.\,1741 Wien – 20.\,2.\,1790 ebd.), \emph{Kaiser}|pw}} (der{ }ſich allerdings in der Sache{ }ſehr dumm benommen hat).\pend
           
\pstart
           Dieſer Tage{ }ſende ich Dir auch \strikeout{ein} das erſte
               franzöſiſche Buch\pwindex{Constant, Benjamin 23.\,10.\,1767 Lausanne – 8.\,12.\,1830 Paris@\textsc{Constant, Benjamin} (23.\,10.\,1767 Lausanne – 8.\,12.\,1830 Paris), \emph{Schriftsteller, Politiker}!Adolphe. Anecdote trouvée dans les papiers d’un inconnu@\strich\emph{Adolphe. Anecdote trouvée dans les papiers d’un inconnu}|pwv}, das ich{ }ſeit Langem mit Genuß geleſen habe (dieſer Satz iſt {\pb}grammatikaliſch{ }ſehr falſch). Es{ }ſtammt natürlich aus dem Jahre 1820 und iſt ganz einfach der größte pſychologiſche Roman, den es gibt:
                  \label{K_L02790-12v}\edtext{»\textsc{Adolphe\pwindex{Constant, Benjamin 23.\,10.\,1767 Lausanne – 8.\,12.\,1830 Paris@\textsc{Constant, Benjamin} (23.\,10.\,1767 Lausanne – 8.\,12.\,1830 Paris), \emph{Schriftsteller, Politiker}!Adolphe. Anecdote trouvée dans les papiers d’un inconnu@\strich\emph{Adolphe. Anecdote trouvée dans les papiers d’un inconnu}|pw}}« von \textsc{Benjamin Constant\pwindex{Constant, Benjamin 23.\,10.\,1767 Lausanne – 8.\,12.\,1830 Paris@\textsc{Constant, Benjamin} (23.\,10.\,1767 Lausanne – 8.\,12.\,1830 Paris), \emph{Schriftsteller, Politiker}|pw}}}{\lemma{\textnormal{\emph{»Adolphe« … Constant}}}\Cendnote{\textnormal{Eine zeitnahe Rezeption durch Schnitzler ist nicht belegt. Er beendete die
                  Lektüre von \emph{Adolphe}\pwindex{Constant, Benjamin 23.\,10.\,1767 Lausanne – 8.\,12.\,1830 Paris@\textsc{Constant, Benjamin} (23.\,10.\,1767 Lausanne – 8.\,12.\,1830 Paris), \emph{Schriftsteller, Politiker}!Adolphe. Anecdote trouvée dans les papiers d’un inconnu@\strich\emph{Adolphe. Anecdote trouvée dans les papiers d’un inconnu}|pwk} am 7. 2. 1906.}}}\label{K_L02790-12}.
               Freilich ein Buch\pwindex{Constant, Benjamin 23.\,10.\,1767 Lausanne – 8.\,12.\,1830 Paris@\textsc{Constant, Benjamin} (23.\,10.\,1767 Lausanne – 8.\,12.\,1830 Paris), \emph{Schriftsteller, Politiker}!Adolphe. Anecdote trouvée dans les papiers d’un inconnu@\strich\emph{Adolphe. Anecdote trouvée dans les papiers d’un inconnu}|pwv} ohne Wärme,
               aber wie aus Erz gegoſſen, – nicht ein Wort zu viel, nicht eines zu wenig – die
               unerbittlichſte Analyſe\pwindex{Constant, Benjamin 23.\,10.\,1767 Lausanne – 8.\,12.\,1830 Paris@\textsc{Constant, Benjamin} (23.\,10.\,1767 Lausanne – 8.\,12.\,1830 Paris), \emph{Schriftsteller, Politiker}!Adolphe. Anecdote trouvée dans les papiers d’un inconnu@\strich\emph{Adolphe. Anecdote trouvée dans les papiers d’un inconnu}|pwv} eines{ }ſchwachen Characters, die je ausgeführt worden. Und wenn man bedenkt, daß \strikeout{\textcolor{gray}{m}} wir \strikeout{hinterher}{ }{\pb}\textsc{Paul Bourget\pwindex{Bourget, Paul 2.\,9.\,1852 Amiens – 25.\,12.\,1935 Paris@\textsc{Bourget, Paul} (2.\,9.\,1852 Amiens – 25.\,12.\,1935 Paris), \emph{Schriftsteller}|pw}} bewundert haben, nachdem es einen »\textsc{Adolphe\pwindex{Constant, Benjamin 23.\,10.\,1767 Lausanne – 8.\,12.\,1830 Paris@\textsc{Constant, Benjamin} (23.\,10.\,1767 Lausanne – 8.\,12.\,1830 Paris), \emph{Schriftsteller, Politiker}!Adolphe. Anecdote trouvée dans les papiers d’un inconnu@\strich\emph{Adolphe. Anecdote trouvée dans les papiers d’un inconnu}|pwv}}« gegeben hat!\pend
           
\pstart
           Grüß’ Dich Gott, mein lieber Freund!\pend
           
\pstart
           Schreib’ mir bald!\pend
           
\pstart
           In Treue {\\[\baselineskip]}Dein {\\[\baselineskip]}\spacefill\mbox{Paul Goldmann.}\pend
           \leftskip=0em{}
\pstart
           \noindent{}Wenn Du den \strikeout{\textsc{L\textcolor{gray}{eo}}}{ }\label{K_L02790-13v}\edtext{\textsc{Leo Fanjung}\pwindex{Van-Jung, Leo 15.\,10.\,1866 Odessa – 2.\,7.\,1939 Riga@\textsc{Van-Jung, Leo} (15.\,10.\,1866 Odessa – 2.\,7.\,1939 Riga), \emph{Gesangspädagoge, Mathematiker}|pw}{ }ſiehſt}{\lemma{\textnormal{\emph{Leo Fanjung siehst}}}\Cendnote{\textnormal{Das nächste belegte
                     Zusammentreffen von Schnitzler und Leo Van-Jung\pwindex{Van-Jung, Leo 15.\,10.\,1866 Odessa – 2.\,7.\,1939 Riga@\textsc{Van-Jung, Leo} (15.\,10.\,1866 Odessa – 2.\,7.\,1939 Riga), \emph{Gesangspädagoge, Mathematiker}|pwk} fand am 22. 11. 1896
                     statt.}}}\label{K_L02790-13},{ }ſo grüß’ ihn, bitte.\pend
           \selectlanguage{ngerman}\endnumbering\briefempfaengerindex{Schnitzler, Arthur@\textsc{Schnitzler, Arthur}!zzzGoldmann, Paul@\emph{von Paul Goldmann}!1896-11-131@{13. 11. [1896]}|)be}\mylabel{L02790h}  \newcommand{\dateiname}{L02790}\newcommand{\titel}{Paul Goldmann an Arthur Schnitzler, 13. 11. [1896]}\newcommand{\editorInnen}{Martin Anton Müller und Laura Untner}%% latex-leseansicht-abspann.tex
%% Abspann für die Leseansicht.
%% Der Schalter \ifkorrekturansicht ist bereits durch den Vorspann gesetzt.

%% latex-abspann.tex
%% Gemeinsamer Abspann für Korrekturansicht und Leseansicht.
%% Setzt den Schalter \ifkorrekturansicht voraus (gesetzt in den
%% einbindenden Dateien latex-korrekturansicht-abspann.tex bzw.
%% latex-leseansicht-abspann.tex).
%% ---------------------------------------------------------------

\normalsize

% Das esempio-Environment wird nur in der Leseansicht benötigt
\ifkorrekturansicht\else
\newenvironment{esempio}[3]%
{
    \vspace{1.5ex}
    \rlap{\underline{#1}}
    \par
    \setlength{\parindent}{0cm}
    \nopagebreak
    \leftskip=#2cm
    \rightskip=#3cm
}
{
    \par
}
\fi

\doendnotes{C}
\bigskip
\vfill

\clearpage

\footnotesize

\ifkorrekturansicht
  \lohead{\textsc{register}}
\fi

% theindex-Environment neu definieren ohne reledmac
\makeatletter
\renewenvironment{theindex}{%
  \ifkorrekturansicht
    \section*{\indexname}%
  \else
    \subsubsection*{Index der erwähnten Entitäten}%
  \fi
  \setlength{\parindent}{0pt}%
  \setlength{\parskip}{0pt plus 0.3pt}%
  \let\item\@idxitem
}{%
  \ifkorrekturansicht\clearpage\fi
}
\makeatother

\IfFileExists{\jobname-pw.ind}{\input{\jobname-pw.ind}}{}

% Quellenangabe nur in der Leseansicht
\ifkorrekturansicht\else
% Fallback-Definitionen, falls die .tex-Datei \titel etc. nicht gesetzt hat
\providecommand{\titel}{}
\providecommand{\editorInnen}{}
\providecommand{\dateiname}{\jobname}

\vspace{3cm}

\vfill

\footnotesize
\textsc{Quelle}: \titel. Herausgegeben von {\editorInnen}. In: \emph{Arthur Schnitzler: Briefwechsel mit Autorinnen und Autoren}.
 Digitale Edition, https://schnitzler-briefe.acdh.oeaw.ac.at/{\dateiname}.html (Stand \today)
\fi

\end{document}


