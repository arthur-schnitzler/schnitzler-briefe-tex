%% latex-korrekturansicht-vorspann.tex
%% Vorspann für die Korrekturansicht.
%% Lädt die gemeinsame Datei latex-vorspann.tex mit gesetztem Schalter.

\newif\ifkorrekturansicht
\korrekturansichttrue

\input{../tex-inputs/latex-vorspann}


\section[ Felix Salten an Arthur Schnitzler, {[}17. 8. 1899{]}]{L03297 Felix Salten an Arthur Schnitzler, {[}17. 8. 1899{]}}
\nopagebreak\mylabel{L03297v}
\rehead{ }\normalsize\beginnumbering\briefempfaengerindex{Schnitzler, Arthur@\textsc{Schnitzler, Arthur}!zzzSalten, Felix@\emph{von Felix Salten}!1899-08-172@{{[}17. 8. 1899{]}}|(be}
\toendnotes[C]{\smallbreak\pagebreak[2]}\Standort{CUL, Schnitzler, B 89, A 2.}
\physDesc{Karte, 1172 Zeichen
\newline{}Handschrift: Bleistift, lateinische Kurrent
\newline{}Schnitzler: mit Bleistift datiert: »17/8 99.« 
\newline{}Ordnung: mit Bleistift von unbekannter Hand nummeriert: »121« }\toendnotes[C]{\smallbreak}
\pstart
           \noindent{}{\pb}Lieber Freund, den Gedanken an eine Radtour scheinen Sie selbst
               aufgegeben zu haben, – nun, ich hätte auch nur sehr schwer abkommen können, und es
               ist mir ganz recht. Sind Sie dafür im September oder
               halben October vielleicht für \label{K_L03297-1v}\edtext{Ragusa\oindex{Dubrovnik@\textbf{Dubrovnik}, \emph{P.PPLA}|pw}}{\lemma{\textnormal{\emph{Ragusa}}}\Cendnote{\textnormal{Dazu kam es nicht.}}}\label{K_L03297-1} zu haben? Ich
               möchte gerne auf acht Tage dahin gehen. In der nächsten Woche ko{\geminationm}e ich vermutlich auf einen od. zwei Tage nach \label{K_L03297-2v}\edtext{Ischl\oindex{Bad Ischl@\textbf{Bad Ischl}, \emph{P.PPL}|pw}}{\lemma{\textnormal{\emph{Ischl}}}\Cendnote{\textnormal{Salten\pwindex{Salten, Felix 06.09.1869 – 08.10.1945@\textsc{Salten, Felix} (06.09.1869 – 08.10.1945), \emph{Schriftsteller/Schriftstellerin, Journalist/Journalistin, Chefredakteur/Chefredakteurin}|pwk} kam am 22. 8. 1899 in Ischl\oindex{Bad Ischl@\textbf{Bad Ischl}, \emph{P.PPL}|pwk} an.}}}\label{K_L03297-2}. Ich zeige Ihnen das
               jedenfalls noch genau an. Haben Sie heute das Feuilleton\pwindex{Decadence-Romane@\emph{Decadence-Romane}|pwv} von Franz Servaes\pwindex{Servaes, Franz 17.06.1862 – 14.07.1947@\textsc{Servaes, Franz} (17.06.1862 – 14.07.1947), \emph{Journalist/Journalistin, Kritiker/Kritikerin}|pw} gelesen? »\label{K_L03297-3v}\edtext{Decadence Romane\pwindex{Decadence-Romane@\emph{Decadence-Romane}|pw}}{\lemma{\textnormal{\emph{Decadence Romane}}}\Cendnote{\textnormal{Franz Servaes\pwindex{Servaes, Franz 17.06.1862 – 14.07.1947@\textsc{Servaes, Franz} (17.06.1862 – 14.07.1947), \emph{Journalist/Journalistin, Kritiker/Kritikerin}|pwk}: \emph{Decadence-Romane}\pwindex{Decadence-Romane@\emph{Decadence-Romane}|pwk}. In: \emph{Neue Freie Presse}\pwindex{Neue Freie Presse@\emph{Neue Freie Presse}|pwk}, Nr. 12.566, 17. 8. 1899, Morgenblatt, S. 1–3.}}}\label{K_L03297-3}« – – Die Neue freie Presse\orgindex{Neue Freie Presse@Neue Freie Presse|pw} brauchte für den alternden Karl v. Thaler\pwindex{Thaler, Karl von 30.09.1836 – 20.10.1919@\textsc{Thaler, Karl von} (30.09.1836 – 20.10.1919), \emph{Schriftsteller/Schriftstellerin, Journalist/Journalistin, Redakteur/Redakteurin}|pw} einen Ersatz und hat ihn in Servaes\pwindex{Servaes, Franz 17.06.1862 – 14.07.1947@\textsc{Servaes, Franz} (17.06.1862 – 14.07.1947), \emph{Journalist/Journalistin, Kritiker/Kritikerin}|pw} gefunden, nur dass mir Servaes\pwindex{Servaes, Franz 17.06.1862 – 14.07.1947@\textsc{Servaes, Franz} (17.06.1862 – 14.07.1947), \emph{Journalist/Journalistin, Kritiker/Kritikerin}|pw} mit seinem Orientirtsein noch eckel{\pb}hafter ist. Wo befindet sich
                  \label{K_L03297-4v}\edtext{Beer-Hofmann\pwindex{Beer-Hofmann, Richard 1866-07-11 – 1945-09-26@\textsc{Beer-Hofmann, Richard} (1866-07-11 – 1945-09-26), \emph{Schriftsteller/Schriftstellerin}|pw}}{\lemma{\textnormal{\emph{Beer-Hofmann}}}\Cendnote{\textnormal{Beer-Hofmann\pwindex{Beer-Hofmann, Richard 1866-07-11 – 1945-09-26@\textsc{Beer-Hofmann, Richard} (1866-07-11 – 1945-09-26), \emph{Schriftsteller/Schriftstellerin}|pwk} reiste nach der gemeinsamen
                  Wanderung mit Schnitzler und Jakob Wassermann\pwindex{Wassermann, Jakob 10.03.1873 – 01.01.1934@\textsc{Wassermann, Jakob} (10.03.1873 – 01.01.1934), \emph{Schriftsteller/Schriftstellerin}|pwk} wieder nach Seeboden\oindex{Seeboden@\textbf{Seeboden}, \emph{A.ADM3}|pwk}.}}}\label{K_L03297-4} jetzt?\pend
           
\pstart
           Otti\pwindex{Salten, Ottilie 07.03.1868 – 22.06.1942@\textsc{Salten, Ottilie} (07.03.1868 – 22.06.1942), \emph{Schauspieler/Schauspielerin}|pw} ist in Ischl\oindex{Bad Ischl@\textbf{Bad Ischl}, \emph{P.PPL}|pw}. Wahrscheinlich haben Sie sie schon \label{K_L03297-5v}\edtext{gesehen}{\lemma{\textnormal{\emph{gesehen}}}\Cendnote{\textnormal{Schnitzler war seit 15. 8. 1899 in Ischl\oindex{Bad Ischl@\textbf{Bad Ischl}, \emph{P.PPL}|pwk}. Eine Begegnung mit Ottilie Metzl\pwindex{Salten, Ottilie 07.03.1868 – 22.06.1942@\textsc{Salten, Ottilie} (07.03.1868 – 22.06.1942), \emph{Schauspieler/Schauspielerin}|pwk} ist in seinem  \emph{Tagebuch}\pwindex{Tagebuch@\emph{Tagebuch}|pwk} nur gemeinsam mit Salten\pwindex{Salten, Felix 06.09.1869 – 08.10.1945@\textsc{Salten, Felix} (06.09.1869 – 08.10.1945), \emph{Schriftsteller/Schriftstellerin, Journalist/Journalistin, Chefredakteur/Chefredakteurin}|pwk} am 24. 8. 1899 festgehalten.}}}\label{K_L03297-5}. Sie hat noch
                  \label{K_L03297-6v}\edtext{kein Engagement}{\lemma{\textnormal{\emph{kein Engagement}}}\Cendnote{\textnormal{Vgl. Felix Salten an Arthur Schnitzler, 28. 4. 1899.
               }}}\label{K_L03297-6}, ist aber im Ganzen ruhiger. Ich bin die ganze Zeit
               schlecht aufgelegt, aber ich arbeite viel. \label{K_L03297-7v}\edtext{»Die Grundlagen des
                  Jahrhunderts\pwindex{Grundlagen des Neunzehnten Jahrhunderts. 2 Bde.@\emph{Die Grundlagen des Neunzehnten Jahrhunderts. 2 Bde.}|pw}« von Chamberlain\pwindex{Chamberlain, Houston Stewart 09.09.1855 – 09.01.1927@\textsc{Chamberlain, Houston Stewart} (09.09.1855 – 09.01.1927), \emph{Schriftsteller/Schriftstellerin}|pw}}{\lemma{\textnormal{\emph{»Die … Chamberlain}}}\Cendnote{\textnormal{Houston Stewart Chamberlain\pwindex{Chamberlain, Houston Stewart 09.09.1855 – 09.01.1927@\textsc{Chamberlain, Houston Stewart} (09.09.1855 – 09.01.1927), \emph{Schriftsteller/Schriftstellerin}|pwk}: \emph{Die Grundlagen des Neunzehnten Jahrhunderts. 2
                        Bde.}\pwindex{Grundlagen des Neunzehnten Jahrhunderts. 2 Bde.@\emph{Die Grundlagen des Neunzehnten Jahrhunderts. 2 Bde.}|pwk}{ }München\oindex{Muenchen@\textbf{München}, \emph{P.PPLA}|pwk}: \emph{Verlagsanstalt F. Bruckmann A.-G.}\orgindex{Bruckmann Verlag@Bruckmann Verlag|pwk}{ }1899. Eine Lektüre durch Schnitzler ist
                  nicht nachweisbar.}}}\label{K_L03297-7} ist ein sehr interessantes Buch. Ich gebe es Ihnen, wenn
               Sie zurückkommen. Ich schreibe augenblicklich darüber eine Anzahl von \label{K_L03297-8v}\edtext{Entgegnungen\pwindex{fremde Volk«. I.@\emph{»Das fremde Volk«. I.}|pwv}\pwindex{fremde Volk«. II.@\emph{»Das fremde Volk«. II.}|pwv}\pwindex{fremde Volk«. III.@\emph{»Das fremde Volk«. III.}|pwv}}{\lemma{\textnormal{\emph{Entgegnungen}}}\Cendnote{\textnormal{Die Reihe »Das fremde Volk«\pwindex{fremde Volk«@\emph{»Das fremde Volk«}|pwkv} erschien in den
                  Nummern 35–37 des dritten Jahrgangs von Theodor
                     Herzls\pwindex{Herzl, Theodor 1860-05-02 – 1904-07-03@\textsc{Herzl, Theodor} (1860-05-02 – 1904-07-03), \emph{Schriftsteller/Schriftstellerin, Journalist/Journalistin}|pwk} zionistischer Zeitschrift \emph{Die
                     Welt}\pwindex{Welt (Wien)@\emph{Die Welt (Wien)}|pwk}: F. S.\pwindex{Salten, Felix 06.09.1869 – 08.10.1945@\textsc{Salten, Felix} (06.09.1869 – 08.10.1945), \emph{Schriftsteller/Schriftstellerin, Journalist/Journalistin, Chefredakteur/Chefredakteurin}|pwk} [ = Felix Salten\pwindex{Salten, Felix 06.09.1869 – 08.10.1945@\textsc{Salten, Felix} (06.09.1869 – 08.10.1945), \emph{Schriftsteller/Schriftstellerin, Journalist/Journalistin, Chefredakteur/Chefredakteurin}|pwk}]: \emph{»Das fremde Volk«. I}\pwindex{fremde Volk«. I.@\emph{»Das fremde Volk«. I.}|pwk}. In: \emph{Die Welt}\pwindex{Welt (Wien)@\emph{Die Welt (Wien)}|pwk}, Jg. 3, Nr. 35, 1. 9. 1899, S. 6–7; F. S.\pwindex{Salten, Felix 06.09.1869 – 08.10.1945@\textsc{Salten, Felix} (06.09.1869 – 08.10.1945), \emph{Schriftsteller/Schriftstellerin, Journalist/Journalistin, Chefredakteur/Chefredakteurin}|pwk}: \emph{»Das fremde Volk«. II}\pwindex{fremde Volk«. II.@\emph{»Das fremde Volk«. II.}|pwk}. In: Nr. 36, 8. 9. 1899, S. 13–14 und F. S.\pwindex{Salten, Felix 06.09.1869 – 08.10.1945@\textsc{Salten, Felix} (06.09.1869 – 08.10.1945), \emph{Schriftsteller/Schriftstellerin, Journalist/Journalistin, Chefredakteur/Chefredakteurin}|pwk}: \emph{»Das fremde Volk«. III}\pwindex{fremde Volk«. III.@\emph{»Das fremde Volk«. III.}|pwk}. In: Nr. 37, 15. 9. 1899, S. 13–14.}}}\label{K_L03297-8} für »Die Welt\pwindex{Welt (Wien)@\emph{Die Welt (Wien)}|pw}«.\pend
           
\pstart
           Senden Sie mir bald wieder eine Zeile. – Die Zeitungen bringe ich Ihnen selbst
               mit.\pend
           
\pstart
           Herzlichst Ihr {\\[\baselineskip]}\spacefill\mbox{Salten}\pend
           \leftskip=0em{}\selectlanguage{ngerman}\endnumbering\briefempfaengerindex{Schnitzler, Arthur@\textsc{Schnitzler, Arthur}!zzzSalten, Felix@\emph{von Felix Salten}!1899-08-172@{{[}17. 8. 1899{]}}|)be}\mylabel{L03297h}  \normalsize

\doendnotes{C}
\bigskip
\vfill

\clearpage

\footnotesize

\lohead{\textsc{register}}

% Definiere theindex-Environment komplett neu ohne reledmac
\makeatletter
\renewenvironment{theindex}{%
  \section*{\indexname}%
  \setlength{\parindent}{0pt}%
  \setlength{\parskip}{0pt plus 0.3pt}%
  \let\item\@idxitem
}{%
  \clearpage
}
\makeatother

\IfFileExists{\jobname-pw.ind}{\input{\jobname-pw.ind}}{}

\end{document}

      