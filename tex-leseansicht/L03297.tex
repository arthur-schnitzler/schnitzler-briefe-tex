%% latex-leseansicht-vorspann.tex
%% Vorspann für die Leseansicht.
%% Lädt die gemeinsame Datei latex-vorspann.tex mit nicht gesetztem Schalter.

\newif\ifkorrekturansicht
\korrekturansichtfalse

\input{../tex-inputs/latex-vorspann}


\section[ Felix Salten an Arthur Schnitzler, {[}17. 8. 1899{]}]{L03297 Felix Salten an Arthur Schnitzler,  [17. 8. 1899]}
\nopagebreak\mylabel{L03297v}
\rehead{ }\normalsize\beginnumbering\briefempfaengerindex{Schnitzler, Arthur@\textsc{Schnitzler, Arthur}!zzzSalten, Felix@\emph{von Felix Salten}!1899-08-172@{{[}17. 8. 1899{]}}|(be}
\toendnotes[C]{\smallbreak\pagebreak[2]}
\correspDesc{Versand  durch Felix Salten am [17. 8. 1899] in Wien?
\newline{}Erhalt  durch Arthur Schnitzler im Zeitraum [18. 8. 1899
                  – 22. 8. 1899?] in Bad Ischl}\toendnotes[C]{\smallbreak}
\Standort{CUL, Schnitzler, B 89, A 2.}
\physDesc{Karte, 1172 Zeichen
\newline{}Handschrift: Bleistift, lateinische Kurrent
\newline{}Schnitzler: mit Bleistift datiert: »17/8 99.« 
\newline{}Ordnung: mit Bleistift von unbekannter Hand nummeriert: »121« }\toendnotes[C]{\smallbreak}
\pstart
           \noindent{}{\pb}Lieber Freund, den Gedanken an eine Radtour scheinen Sie selbst
               aufgegeben zu haben, – nun, ich hätte auch nur sehr schwer abkommen können, und es
               ist mir ganz recht. Sind Sie dafür im September oder
               halben October vielleicht für \label{K_L03297-1v}\edtext{Ragusa\oindex{Dubrovnik@\textbf{Dubrovnik}|pw}}{\lemma{\textnormal{\emph{Ragusa}}}\Cendnote{\textnormal{Dazu kam es nicht.}}}\label{K_L03297-1} zu haben? Ich
               möchte gerne auf acht Tage dahin gehen. In der nächsten Woche ko{\geminationm}e ich vermutlich auf einen od. zwei Tage nach \label{K_L03297-2v}\edtext{Ischl\oindex{Bad Ischl@\textbf{Bad Ischl}|pw}}{\lemma{\textnormal{\emph{Ischl}}}\Cendnote{\textnormal{Salten\pwindex{Salten, Felix 6.\,9.\,1869 Budapest – 8.\,10.\,1945 Zürich@\textsc{Salten, Felix} (6.\,9.\,1869 Budapest – 8.\,10.\,1945 Zürich), \emph{Schriftsteller, Journalist, Chefredakteur}|pwk} kam am 22. 8. 1899 in Ischl\oindex{Bad Ischl@\textbf{Bad Ischl}|pwk} an.}}}\label{K_L03297-2}. Ich zeige Ihnen das
               jedenfalls noch genau an. Haben Sie heute das Feuilleton\pwindex{Servaes, Franz 17.\,6.\,1862 Köln – 14.\,7.\,1947 Wien@\textsc{Servaes, Franz} (17.\,6.\,1862 Köln – 14.\,7.\,1947 Wien), \emph{Journalist, Kritiker}!Decadence-Romane@\strich\emph{Decadence-Romane}|pwv} von Franz Servaes\pwindex{Servaes, Franz 17.\,6.\,1862 Köln – 14.\,7.\,1947 Wien@\textsc{Servaes, Franz} (17.\,6.\,1862 Köln – 14.\,7.\,1947 Wien), \emph{Journalist, Kritiker}|pw} gelesen? »\label{K_L03297-3v}\edtext{Decadence Romane\pwindex{Servaes, Franz 17.\,6.\,1862 Köln – 14.\,7.\,1947 Wien@\textsc{Servaes, Franz} (17.\,6.\,1862 Köln – 14.\,7.\,1947 Wien), \emph{Journalist, Kritiker}!Decadence-Romane@\strich\emph{Decadence-Romane}|pw}}{\lemma{\textnormal{\emph{Decadence Romane}}}\Cendnote{\textnormal{Franz Servaes\pwindex{Servaes, Franz 17.\,6.\,1862 Köln – 14.\,7.\,1947 Wien@\textsc{Servaes, Franz} (17.\,6.\,1862 Köln – 14.\,7.\,1947 Wien), \emph{Journalist, Kritiker}|pwk}: \emph{Decadence-Romane}\pwindex{Servaes, Franz 17.\,6.\,1862 Köln – 14.\,7.\,1947 Wien@\textsc{Servaes, Franz} (17.\,6.\,1862 Köln – 14.\,7.\,1947 Wien), \emph{Journalist, Kritiker}!Decadence-Romane@\strich\emph{Decadence-Romane}|pwk}. In: \emph{Neue Freie Presse}\pwindex{Neue Freie Presse@\emph{Neue Freie Presse}|pwk}, Nr. 12.566, 17. 8. 1899, Morgenblatt, S. 1–3.}}}\label{K_L03297-3}« – – Die Neue freie Presse\orgindex{Neue Freie Presse@Neue Freie Presse|pw} brauchte für den alternden Karl v. Thaler\pwindex{Thaler, Karl von 30.\,9.\,1836 Wien – 20.\,10.\,1919 ebd.@\textsc{Thaler, Karl von} (30.\,9.\,1836 Wien – 20.\,10.\,1919 ebd.), \emph{Schriftsteller, Journalist, Redakteur}|pw} einen Ersatz und hat ihn in Servaes\pwindex{Servaes, Franz 17.\,6.\,1862 Köln – 14.\,7.\,1947 Wien@\textsc{Servaes, Franz} (17.\,6.\,1862 Köln – 14.\,7.\,1947 Wien), \emph{Journalist, Kritiker}|pw} gefunden, nur dass mir Servaes\pwindex{Servaes, Franz 17.\,6.\,1862 Köln – 14.\,7.\,1947 Wien@\textsc{Servaes, Franz} (17.\,6.\,1862 Köln – 14.\,7.\,1947 Wien), \emph{Journalist, Kritiker}|pw} mit seinem Orientirtsein noch eckel{\pb}hafter ist. Wo befindet sich
                  \label{K_L03297-4v}\edtext{Beer-Hofmann\pwindex{Beer-Hofmann, Richard 11.\,7.\,1866 Wien – 26.\,9.\,1945 New York City@\textsc{Beer-Hofmann, Richard} (11.\,7.\,1866 Wien – 26.\,9.\,1945 New York City), \emph{Schriftsteller}|pw}}{\lemma{\textnormal{\emph{Beer-Hofmann}}}\Cendnote{\textnormal{Beer-Hofmann\pwindex{Beer-Hofmann, Richard 11.\,7.\,1866 Wien – 26.\,9.\,1945 New York City@\textsc{Beer-Hofmann, Richard} (11.\,7.\,1866 Wien – 26.\,9.\,1945 New York City), \emph{Schriftsteller}|pwk} reiste nach der gemeinsamen
                  Wanderung mit Schnitzler und Jakob Wassermann\pwindex{Wassermann, Jakob 10.\,3.\,1873 Fürth – 1.\,1.\,1934 Altaussee@\textsc{Wassermann, Jakob} (10.\,3.\,1873 Fürth – 1.\,1.\,1934 Altaussee), \emph{Schriftsteller}|pwk} wieder nach Seeboden\oindex{Seeboden@\textbf{Seeboden}, \emph{Verwaltungsgebiet}|pwk}.}}}\label{K_L03297-4} jetzt?\pend
           
\pstart
           Otti\pwindex{Salten, Ottilie 7.\,3.\,1868 Prag – 22.\,6.\,1942 Zürich@\textsc{Salten, Ottilie} (7.\,3.\,1868 Prag – 22.\,6.\,1942 Zürich), \emph{Schauspielerin}|pw} ist in Ischl\oindex{Bad Ischl@\textbf{Bad Ischl}|pw}. Wahrscheinlich haben Sie sie schon \label{K_L03297-5v}\edtext{gesehen}{\lemma{\textnormal{\emph{gesehen}}}\Cendnote{\textnormal{Schnitzler war seit 15. 8. 1899 in Ischl\oindex{Bad Ischl@\textbf{Bad Ischl}|pwk}. Eine Begegnung mit Ottilie Metzl\pwindex{Salten, Ottilie 7.\,3.\,1868 Prag – 22.\,6.\,1942 Zürich@\textsc{Salten, Ottilie} (7.\,3.\,1868 Prag – 22.\,6.\,1942 Zürich), \emph{Schauspielerin}|pwk} ist in seinem  \emph{Tagebuch}\pwindex{Schnitzler, Arthur 15.\,5.\,1862 Wien – 21.\,10.\,1931 ebd.@\textsc{Schnitzler, Arthur} (15.\,5.\,1862 Wien – 21.\,10.\,1931 ebd.), \emph{Schriftsteller, Mediziner}!Tagebuch@\strich\emph{Tagebuch}|pwk} nur gemeinsam mit Salten\pwindex{Salten, Felix 6.\,9.\,1869 Budapest – 8.\,10.\,1945 Zürich@\textsc{Salten, Felix} (6.\,9.\,1869 Budapest – 8.\,10.\,1945 Zürich), \emph{Schriftsteller, Journalist, Chefredakteur}|pwk} am 24. 8. 1899 festgehalten.}}}\label{K_L03297-5}. Sie hat noch
                  \label{K_L03297-6v}\edtext{kein Engagement}{\lemma{\textnormal{\emph{kein Engagement}}}\Cendnote{\textnormal{Vgl. XXXX Auszeichnungsfehler: Dokument L03288 nicht gefunden.
               }}}\label{K_L03297-6}, ist aber im Ganzen ruhiger. Ich bin die ganze Zeit
               schlecht aufgelegt, aber ich arbeite viel. \label{K_L03297-7v}\edtext{»Die Grundlagen des
                  Jahrhunderts\pwindex{Chamberlain, Houston Stewart 9.\,9.\,1855 Portsmouth – 9.\,1.\,1927 Bayreuth@\textsc{Chamberlain, Houston Stewart} (9.\,9.\,1855 Portsmouth – 9.\,1.\,1927 Bayreuth), \emph{Schriftsteller}!Grundlagen des Neunzehnten Jahrhunderts. 2 Bde.@\strich\emph{Die Grundlagen des Neunzehnten Jahrhunderts. 2 Bde.}|pw}« von Chamberlain\pwindex{Chamberlain, Houston Stewart 9.\,9.\,1855 Portsmouth – 9.\,1.\,1927 Bayreuth@\textsc{Chamberlain, Houston Stewart} (9.\,9.\,1855 Portsmouth – 9.\,1.\,1927 Bayreuth), \emph{Schriftsteller}|pw}}{\lemma{\textnormal{\emph{»Die … Chamberlain}}}\Cendnote{\textnormal{Houston Stewart Chamberlain\pwindex{Chamberlain, Houston Stewart 9.\,9.\,1855 Portsmouth – 9.\,1.\,1927 Bayreuth@\textsc{Chamberlain, Houston Stewart} (9.\,9.\,1855 Portsmouth – 9.\,1.\,1927 Bayreuth), \emph{Schriftsteller}|pwk}: \emph{Die Grundlagen des Neunzehnten Jahrhunderts. 2
                        Bde.}\pwindex{Chamberlain, Houston Stewart 9.\,9.\,1855 Portsmouth – 9.\,1.\,1927 Bayreuth@\textsc{Chamberlain, Houston Stewart} (9.\,9.\,1855 Portsmouth – 9.\,1.\,1927 Bayreuth), \emph{Schriftsteller}!Grundlagen des Neunzehnten Jahrhunderts. 2 Bde.@\strich\emph{Die Grundlagen des Neunzehnten Jahrhunderts. 2 Bde.}|pwk}{ }München\oindex{München@\textbf{München}|pwk}: \emph{Verlagsanstalt F. Bruckmann A.-G.}\orgindex{Bruckmann Verlag@Bruckmann Verlag|pwk}{ }1899. Eine Lektüre durch Schnitzler ist
                  nicht nachweisbar.}}}\label{K_L03297-7} ist ein sehr interessantes Buch. Ich gebe es Ihnen, wenn
               Sie zurückkommen. Ich schreibe augenblicklich darüber eine Anzahl von \label{K_L03297-8v}\edtext{Entgegnungen\pwindex{Salten, Felix 6.\,9.\,1869 Budapest – 8.\,10.\,1945 Zürich@\textsc{Salten, Felix} (6.\,9.\,1869 Budapest – 8.\,10.\,1945 Zürich), \emph{Schriftsteller, Journalist, Chefredakteur}!fremde Volk«. I.@\strich\emph{»Das fremde Volk«. I.}|pwv}\pwindex{Salten, Felix 6.\,9.\,1869 Budapest – 8.\,10.\,1945 Zürich@\textsc{Salten, Felix} (6.\,9.\,1869 Budapest – 8.\,10.\,1945 Zürich), \emph{Schriftsteller, Journalist, Chefredakteur}!fremde Volk«. II.@\strich\emph{»Das fremde Volk«. II.}|pwv}\pwindex{Salten, Felix 6.\,9.\,1869 Budapest – 8.\,10.\,1945 Zürich@\textsc{Salten, Felix} (6.\,9.\,1869 Budapest – 8.\,10.\,1945 Zürich), \emph{Schriftsteller, Journalist, Chefredakteur}!fremde Volk«. III.@\strich\emph{»Das fremde Volk«. III.}|pwv}}{\lemma{\textnormal{\emph{Entgegnungen}}}\Cendnote{\textnormal{Die Reihe »Das fremde Volk«\pwindex{Salten, Felix 6.\,9.\,1869 Budapest – 8.\,10.\,1945 Zürich@\textsc{Salten, Felix} (6.\,9.\,1869 Budapest – 8.\,10.\,1945 Zürich), \emph{Schriftsteller, Journalist, Chefredakteur}!fremde Volk«@\strich\emph{»Das fremde Volk«}|pwkv} erschien in den
                  Nummern 35–37 des dritten Jahrgangs von Theodor
                     Herzls\pwindex{Herzl, Theodor 2.\,5.\,1860 Budapest – 3.\,7.\,1904 Edlach@\textsc{Herzl, Theodor} (2.\,5.\,1860 Budapest – 3.\,7.\,1904 Edlach), \emph{Schriftsteller, Journalist}|pwk} zionistischer Zeitschrift \emph{Die
                     Welt}\pwindex{Welt (Wien)@\emph{Die Welt (Wien)}|pwk}: F. S.\pwindex{Salten, Felix 6.\,9.\,1869 Budapest – 8.\,10.\,1945 Zürich@\textsc{Salten, Felix} (6.\,9.\,1869 Budapest – 8.\,10.\,1945 Zürich), \emph{Schriftsteller, Journalist, Chefredakteur}|pwk} [ = Felix Salten\pwindex{Salten, Felix 6.\,9.\,1869 Budapest – 8.\,10.\,1945 Zürich@\textsc{Salten, Felix} (6.\,9.\,1869 Budapest – 8.\,10.\,1945 Zürich), \emph{Schriftsteller, Journalist, Chefredakteur}|pwk}]: \emph{»Das fremde Volk«. I}\pwindex{Salten, Felix 6.\,9.\,1869 Budapest – 8.\,10.\,1945 Zürich@\textsc{Salten, Felix} (6.\,9.\,1869 Budapest – 8.\,10.\,1945 Zürich), \emph{Schriftsteller, Journalist, Chefredakteur}!fremde Volk«. I.@\strich\emph{»Das fremde Volk«. I.}|pwk}. In: \emph{Die Welt}\pwindex{Welt (Wien)@\emph{Die Welt (Wien)}|pwk}, Jg. 3, Nr. 35, 1. 9. 1899, S. 6–7; F. S.\pwindex{Salten, Felix 6.\,9.\,1869 Budapest – 8.\,10.\,1945 Zürich@\textsc{Salten, Felix} (6.\,9.\,1869 Budapest – 8.\,10.\,1945 Zürich), \emph{Schriftsteller, Journalist, Chefredakteur}|pwk}: \emph{»Das fremde Volk«. II}\pwindex{Salten, Felix 6.\,9.\,1869 Budapest – 8.\,10.\,1945 Zürich@\textsc{Salten, Felix} (6.\,9.\,1869 Budapest – 8.\,10.\,1945 Zürich), \emph{Schriftsteller, Journalist, Chefredakteur}!fremde Volk«. II.@\strich\emph{»Das fremde Volk«. II.}|pwk}. In: Nr. 36, 8. 9. 1899, S. 13–14 und F. S.\pwindex{Salten, Felix 6.\,9.\,1869 Budapest – 8.\,10.\,1945 Zürich@\textsc{Salten, Felix} (6.\,9.\,1869 Budapest – 8.\,10.\,1945 Zürich), \emph{Schriftsteller, Journalist, Chefredakteur}|pwk}: \emph{»Das fremde Volk«. III}\pwindex{Salten, Felix 6.\,9.\,1869 Budapest – 8.\,10.\,1945 Zürich@\textsc{Salten, Felix} (6.\,9.\,1869 Budapest – 8.\,10.\,1945 Zürich), \emph{Schriftsteller, Journalist, Chefredakteur}!fremde Volk«. III.@\strich\emph{»Das fremde Volk«. III.}|pwk}. In: Nr. 37, 15. 9. 1899, S. 13–14.}}}\label{K_L03297-8} für »Die Welt\pwindex{Welt (Wien)@\emph{Die Welt (Wien)}|pw}«.\pend
           
\pstart
           Senden Sie mir bald wieder eine Zeile. – Die Zeitungen bringe ich Ihnen selbst
               mit.\pend
           
\pstart
           Herzlichst Ihr {\\[\baselineskip]}\spacefill\mbox{Salten}\pend
           \leftskip=0em{}\selectlanguage{ngerman}\endnumbering\briefempfaengerindex{Schnitzler, Arthur@\textsc{Schnitzler, Arthur}!zzzSalten, Felix@\emph{von Felix Salten}!1899-08-172@{{[}17. 8. 1899{]}}|)be}\mylabel{L03297h}  \newcommand{\dateiname}{L03297}\newcommand{\titel}{Felix Salten an Arthur Schnitzler, [17. 8. 1899]}\newcommand{\editorInnen}{Martin Anton Müller und Laura Untner}%% latex-leseansicht-abspann.tex
%% Abspann für die Leseansicht.
%% Der Schalter \ifkorrekturansicht ist bereits durch den Vorspann gesetzt.

%% latex-abspann.tex
%% Gemeinsamer Abspann für Korrekturansicht und Leseansicht.
%% Setzt den Schalter \ifkorrekturansicht voraus (gesetzt in den
%% einbindenden Dateien latex-korrekturansicht-abspann.tex bzw.
%% latex-leseansicht-abspann.tex).
%% ---------------------------------------------------------------

\normalsize

% Das esempio-Environment wird nur in der Leseansicht benötigt
\ifkorrekturansicht\else
\newenvironment{esempio}[3]%
{
    \vspace{1.5ex}
    \rlap{\underline{#1}}
    \par
    \setlength{\parindent}{0cm}
    \nopagebreak
    \leftskip=#2cm
    \rightskip=#3cm
}
{
    \par
}
\fi

\doendnotes{C}
\bigskip
\vfill

\clearpage

\footnotesize

\ifkorrekturansicht
  \lohead{\textsc{register}}
\fi

% theindex-Environment neu definieren ohne reledmac
\makeatletter
\renewenvironment{theindex}{%
  \ifkorrekturansicht
    \section*{\indexname}%
  \else
    \subsubsection*{Index der erwähnten Entitäten}%
  \fi
  \setlength{\parindent}{0pt}%
  \setlength{\parskip}{0pt plus 0.3pt}%
  \let\item\@idxitem
}{%
  \ifkorrekturansicht\clearpage\fi
}
\makeatother

\IfFileExists{\jobname-pw.ind}{\input{\jobname-pw.ind}}{}

% Quellenangabe nur in der Leseansicht
\ifkorrekturansicht\else
% Fallback-Definitionen, falls die .tex-Datei \titel etc. nicht gesetzt hat
\providecommand{\titel}{}
\providecommand{\editorInnen}{}
\providecommand{\dateiname}{\jobname}

\vspace{3cm}

\vfill

\footnotesize
\textsc{Quelle}: \titel. Herausgegeben von {\editorInnen}. In: \emph{Arthur Schnitzler: Briefwechsel mit Autorinnen und Autoren}.
 Digitale Edition, https://schnitzler-briefe.acdh.oeaw.ac.at/{\dateiname}.html (Stand \today)
\fi

\end{document}


