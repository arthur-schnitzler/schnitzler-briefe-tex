%% latex-leseansicht-vorspann.tex
%% Vorspann für die Leseansicht.
%% Lädt die gemeinsame Datei latex-vorspann.tex mit nicht gesetztem Schalter.

\newif\ifkorrekturansicht
\korrekturansichtfalse

\input{../tex-inputs/latex-vorspann}


         
         \renewcommand{\erwaehntePersonen}{Personen: Richard Beer-Hofmann, Houston Stewart Chamberlain, Theodor Herzl, Felix Salten, Ottilie Salten, Franz Servaes, Karl von Thaler, Jakob Wassermann}
         \renewcommand{\erwaehnteInstitutionen}{Institutionen: Bruckmann Verlag, Neue Freie Presse}
         \renewcommand{\erwaehnteOrte}{Orte: Bad Ischl, Dubrovnik, München, Seeboden, Wien}
         \renewcommand{\erwaehnteWerke}{Werke: Decadence-Romane, Die Grundlagen des Neunzehnten Jahrhunderts. 2 Bde., Die Welt (Wien), Neue Freie Presse, Tagebuch, »Das fremde Volk«, »Das fremde Volk«. I., »Das fremde Volk«. II., »Das fremde Volk«. III.}
               \section[ Felix Salten an Arthur Schnitzler, {[}17. 8. 1899{]}]{ Felix Salten an Arthur Schnitzler, {[}17. 8. 1899{]}}\nopagebreak\mylabel{v}\rehead{ }\begin{ledgroupsized}[t]{13cm}\normalsize\beginnumbering \toendnotes[C]{\smallbreak\pagebreak[2]} \Standort{CUL, Schnitzler, B 89, A 2.}
\physDesc{Karte, 1172 Zeichen
\newline{}Handschrift: Bleistift, lateinische Kurrent
\newline{}Schnitzler: mit Bleistift datiert: »17/8 99.« 
\newline{}Ordnung: mit Bleistift von unbekannter Hand nummeriert: »121« }\toendnotes[C]{\smallbreak}\pstart
           \noindent{}{\pb}Lieber Freund, den Gedanken an eine Radtour scheinen Sie selbst
               aufgegeben zu haben, – nun, ich hätte auch nur sehr schwer abkommen können, und es
               ist mir ganz recht. Sind Sie dafür im September oder
               halben October vielleicht für \label{K_L03297-1v}\edtext{Ragusa\oindex{Dubrovnik@\textbf{Dubrovnik}|pw}}{\lemma{\textnormal{\emph{Ragusa}}}\Cendnote{\textnormal{nicht geschehen}}}\label{K_L03297-1h} zu haben? Ich
               möchte gerne auf acht Tage dahin gehen. In der nächsten Woche ko{\geminationm}e ich vermutlich auf einen od. zwei Tage nach \label{K_L03297-2v}\edtext{Ischl\oindex{Bad Ischl@\textbf{Bad Ischl}|pw}}{\lemma{\textnormal{\emph{Ischl}}}\Cendnote{\textnormal{Salten\pwindex{Salten, Felix 06.09.1869 – 08.10.1945@\textsc{Salten, Felix} (06.09.1869 – 08.10.1945), \emph{Schriftsteller, Journalist}|pwk} kam am 22. 8. 1899 in Ischl\oindex{Bad Ischl@\textbf{Bad Ischl}|pwk} an.}}}\label{K_L03297-2h}. Ich zeige Ihnen das
               jedenfalls noch genau an. Haben Sie heute das Feuilleton\pwindex{Servaes, Franz 17.06.1862 – 14.07.1947@\textsc{Servaes, Franz} (17.06.1862 – 14.07.1947), \emph{Journalist, Kritiker}!Decadence-Romane1899-08-17@\strich\emph{Decadence-Romane} {[}1899-08-17{]}|pwv} von Franz Servaes\pwindex{Servaes, Franz 17.06.1862 – 14.07.1947@\textsc{Servaes, Franz} (17.06.1862 – 14.07.1947), \emph{Journalist, Kritiker}|pw} gelesen? »\label{K_L03297-3v}\edtext{Decadence Romane\pwindex{Servaes, Franz 17.06.1862 – 14.07.1947@\textsc{Servaes, Franz} (17.06.1862 – 14.07.1947), \emph{Journalist, Kritiker}!Decadence-Romane1899-08-17@\strich\emph{Decadence-Romane} {[}1899-08-17{]}|pw}}{\lemma{\textnormal{\emph{Decadence Romane}}}\Cendnote{\textnormal{Franz Servaes\pwindex{Servaes, Franz 17.06.1862 – 14.07.1947@\textsc{Servaes, Franz} (17.06.1862 – 14.07.1947), \emph{Journalist, Kritiker}|pwk}: \emph{Decadence-Romane}\pwindex{Servaes, Franz 17.06.1862 – 14.07.1947@\textsc{Servaes, Franz} (17.06.1862 – 14.07.1947), \emph{Journalist, Kritiker}!Decadence-Romane1899-08-17@\strich\emph{Decadence-Romane} {[}1899-08-17{]}|pwk}. In: \emph{Neue Freie Presse}\pwindex{Neue Freie Presse1864 – 1939@\emph{Neue Freie Presse} {[}1864 – 1939{]}|pwk}, Nr. 12.566, 17. 8. 1899, Morgenblatt, S. 1–3.}}}\label{K_L03297-3h}« – – Die Neue freie Presse\orgindex{Neue Freie Presse@Neue Freie Presse|pw} brauchte für den alternden Karl v. Thaler\pwindex{Thaler, Karl von 30.09.1836 – 20.10.1919@\textsc{Thaler, Karl von} (30.09.1836 – 20.10.1919), \emph{Schriftsteller, Journalist, Redakteur}|pw} einen Ersatz und hat ihn in Servaes\pwindex{Servaes, Franz 17.06.1862 – 14.07.1947@\textsc{Servaes, Franz} (17.06.1862 – 14.07.1947), \emph{Journalist, Kritiker}|pw} gefunden, nur dass mir Servaes\pwindex{Servaes, Franz 17.06.1862 – 14.07.1947@\textsc{Servaes, Franz} (17.06.1862 – 14.07.1947), \emph{Journalist, Kritiker}|pw} mit seinem Orientirtsein noch eckel{\pb}hafter ist. Wo befindet sich
                  \label{K_L03297-4v}\edtext{Beer-Hofmann\pwindex{Beer-Hofmann, Richard 1866-07-11 – 1945-09-26@\textsc{Beer-Hofmann, Richard} (1866-07-11 – 1945-09-26), \emph{Schriftsteller}|pw}}{\lemma{\textnormal{\emph{Beer-Hofmann}}}\Cendnote{\textnormal{Beer-Hofmann\pwindex{Beer-Hofmann, Richard 1866-07-11 – 1945-09-26@\textsc{Beer-Hofmann, Richard} (1866-07-11 – 1945-09-26), \emph{Schriftsteller}|pwk} reiste nach der gemeinsamen
                  Wanderung mit Schnitzler\pwindex{Schnitzler, Arthur 15.05.1862 – 21.10.1931@\textsc{Schnitzler, Arthur} (15.05.1862 – 21.10.1931), \emph{Schriftsteller, Mediziner}|pwk} und Jakob Wassermann\pwindex{Wassermann, Jakob 10.03.1873 – 01.01.1934@\textsc{Wassermann, Jakob} (10.03.1873 – 01.01.1934), \emph{Schriftsteller}|pwk} wieder nach Seeboden\oindex{Seeboden@\textbf{Seeboden}|pwk}.}}}\label{K_L03297-4h} jetzt?\pend
           \pstart
           Otti\pwindex{Salten, Ottilie 07.03.1868 – 22.06.1942@\textsc{Salten, Ottilie} (07.03.1868 – 22.06.1942), \emph{Schauspielerin}|pw} ist in Ischl\oindex{Bad Ischl@\textbf{Bad Ischl}|pw}. Wahrscheinlich haben Sie sie schon \label{K_L03297-5v}\edtext{gesehen}{\lemma{\textnormal{\emph{gesehen}}}\Cendnote{\textnormal{Schnitzler\pwindex{Schnitzler, Arthur 15.05.1862 – 21.10.1931@\textsc{Schnitzler, Arthur} (15.05.1862 – 21.10.1931), \emph{Schriftsteller, Mediziner}|pwk} war seit 15. 8. 1899 in Ischl\oindex{Bad Ischl@\textbf{Bad Ischl}|pwk}. Eine Begegnung mit Ottilie Metzl\pwindex{Salten, Ottilie 07.03.1868 – 22.06.1942@\textsc{Salten, Ottilie} (07.03.1868 – 22.06.1942), \emph{Schauspielerin}|pwk} ist in seinem  \emph{Tagebuch}\pwindex{\textcolor{red}{\textsuperscript{XXXX1 indx}}!Tagebuch1981 – 2000@\strich\emph{Tagebuch} {[}Hrsg., 1981 – 2000{]}|pwk} nur gemeinsam mit Salten\pwindex{Salten, Felix 06.09.1869 – 08.10.1945@\textsc{Salten, Felix} (06.09.1869 – 08.10.1945), \emph{Schriftsteller, Journalist}|pwk} am 24. 8. 1899 festgehalten.}}}\label{K_L03297-5h}. Sie hat noch
                  \label{K_L03297-6v}\edtext{kein Engagement}{\lemma{\textnormal{\emph{kein Engagement}}}\Cendnote{\textnormal{vgl. Felix Salten an Arthur Schnitzler, 28. 4. 1899}}}\label{K_L03297-6h}, ist aber im Ganzen ruhiger. Ich bin die ganze Zeit
               schlecht aufgelegt, aber ich arbeite viel. \label{K_L03297-7v}\edtext{»Die Grundlagen des
                  Jahrhunderts\pwindex{Chamberlain, Houston Stewart 09.09.1855 – 09.01.1927@\textsc{Chamberlain, Houston Stewart} (09.09.1855 – 09.01.1927), \emph{Schriftsteller}!Grundlagen des Neunzehnten Jahrhunderts. 2 Bde.1899@\strich\emph{Die Grundlagen des Neunzehnten Jahrhunderts. 2 Bde.} {[}1899{]}|pw}« von Chamberlain\pwindex{Chamberlain, Houston Stewart 09.09.1855 – 09.01.1927@\textsc{Chamberlain, Houston Stewart} (09.09.1855 – 09.01.1927), \emph{Schriftsteller}|pw}}{\lemma{\textnormal{\emph{»Die … Chamberlain}}}\Cendnote{\textnormal{Houston Stewart Chamberlain\pwindex{Chamberlain, Houston Stewart 09.09.1855 – 09.01.1927@\textsc{Chamberlain, Houston Stewart} (09.09.1855 – 09.01.1927), \emph{Schriftsteller}|pwk}: \emph{Die Grundlagen des Neunzehnten Jahrhunderts. 2
                        Bde.}\pwindex{Chamberlain, Houston Stewart 09.09.1855 – 09.01.1927@\textsc{Chamberlain, Houston Stewart} (09.09.1855 – 09.01.1927), \emph{Schriftsteller}!Grundlagen des Neunzehnten Jahrhunderts. 2 Bde.1899@\strich\emph{Die Grundlagen des Neunzehnten Jahrhunderts. 2 Bde.} {[}1899{]}|pwk}{ }München\oindex{Muenchen@\textbf{München}|pwk}: \emph{Verlagsanstalt F. Bruckmann A.-G.}\orgindex{Bruckmann Verlag@Bruckmann Verlag|pwk}{ }1899. Eine Lektüre durch Schnitzler\pwindex{Schnitzler, Arthur 15.05.1862 – 21.10.1931@\textsc{Schnitzler, Arthur} (15.05.1862 – 21.10.1931), \emph{Schriftsteller, Mediziner}|pwk} ist
                  nicht nachweisbar.}}}\label{K_L03297-7h} ist ein sehr interessantes Buch. Ich gebe es Ihnen, wenn
               Sie zurückkommen. Ich schreibe augenblicklich darüber eine Anzahl von \label{K_L03297-8v}\edtext{Entgegnungen\pwindex{fremde Volk«. I.1899-09-01@\emph{»Das fremde Volk«. I.} {[}1899-09-01{]}|pwv}\pwindex{fremde Volk«. II.1899-09-08@\emph{»Das fremde Volk«. II.} {[}1899-09-08{]}|pwv}\pwindex{fremde Volk«. III.1899-09-15@\emph{»Das fremde Volk«. III.} {[}1899-09-15{]}|pwv}}{\lemma{\textnormal{\emph{Entgegnungen}}}\Cendnote{\textnormal{Die Reihe »Das fremde Volk«\pwindex{fremde Volk«1899-09-01 – 1899-09-15@\emph{»Das fremde Volk«} {[}1899-09-01 – 1899-09-15{]}|pwkv} erschien in den
                  Nummern 35–37 des dritten Jahrgangs von Theodor
                     Herzl\pwindex{Herzl, Theodor 1860-05-02 – 1904-07-03@\textsc{Herzl, Theodor} (1860-05-02 – 1904-07-03), \emph{Schriftsteller, Journalist}|pwk}s zionistischer Zeitschrift \emph{Die
                     Welt}\pwindex{Welt (Wien)1897 – 1914@\emph{Die Welt (Wien)} {[}1897 – 1914{]}|pwk}: F. S.\pwindex{Salten, Felix 06.09.1869 – 08.10.1945@\textsc{Salten, Felix} (06.09.1869 – 08.10.1945), \emph{Schriftsteller, Journalist}|pwk} [ = Felix Salten\pwindex{Salten, Felix 06.09.1869 – 08.10.1945@\textsc{Salten, Felix} (06.09.1869 – 08.10.1945), \emph{Schriftsteller, Journalist}|pwk}]: \emph{»Das fremde Volk«. I}\pwindex{fremde Volk«. I.1899-09-01@\emph{»Das fremde Volk«. I.} {[}1899-09-01{]}|pwk}. In: \emph{Die Welt}\pwindex{Welt (Wien)1897 – 1914@\emph{Die Welt (Wien)} {[}1897 – 1914{]}|pwk}, Jg. 3, Nr. 35, 1. 9. 1899, S. 6–7; F. S.\pwindex{Salten, Felix 06.09.1869 – 08.10.1945@\textsc{Salten, Felix} (06.09.1869 – 08.10.1945), \emph{Schriftsteller, Journalist}|pwk}: \emph{»Das fremde Volk«. II}\pwindex{fremde Volk«. II.1899-09-08@\emph{»Das fremde Volk«. II.} {[}1899-09-08{]}|pwk}. In: \emph{Die Welt}\pwindex{Welt (Wien)1897 – 1914@\emph{Die Welt (Wien)} {[}1897 – 1914{]}|pwk}, Jg. 3, Nr. 36, 8. 9. 1899, S. 13–14 und F. S.\pwindex{Salten, Felix 06.09.1869 – 08.10.1945@\textsc{Salten, Felix} (06.09.1869 – 08.10.1945), \emph{Schriftsteller, Journalist}|pwk}: \emph{»Das fremde Volk«. III}\pwindex{fremde Volk«. III.1899-09-15@\emph{»Das fremde Volk«. III.} {[}1899-09-15{]}|pwk}. In: \emph{Die Welt}\pwindex{Welt (Wien)1897 – 1914@\emph{Die Welt (Wien)} {[}1897 – 1914{]}|pwk}, Jg. 3, Nr. 37, 15. 9. 1899, S. 13–14.}}}\label{K_L03297-8h} für »Die Welt\pwindex{Welt (Wien)1897 – 1914@\emph{Die Welt (Wien)} {[}1897 – 1914{]}|pw}«.\pend
           \pstart
           Senden Sie mir bald wieder eine Zeile. – Die Zeitungen bringe ich Ihnen selbst
               mit.\pend
           \pstart
           Herzlichst Ihr {\\[\baselineskip]}\spacefill\mbox{Salten}\pend
           \leftskip=0em{}
         
         \endnumbering\mylabel{h}\end{ledgroupsized}  \newcommand{\dateiname}{L03297}\newcommand{\titel}{Felix Salten an Arthur Schnitzler, [17. 8. 1899]}\newcommand{\editorInnen}{Martin Anton Müller und Laura Untner}%% latex-leseansicht-abspann.tex
%% Abspann für die Leseansicht.
%% Der Schalter \ifkorrekturansicht ist bereits durch den Vorspann gesetzt.

%% latex-abspann.tex
%% Gemeinsamer Abspann für Korrekturansicht und Leseansicht.
%% Setzt den Schalter \ifkorrekturansicht voraus (gesetzt in den
%% einbindenden Dateien latex-korrekturansicht-abspann.tex bzw.
%% latex-leseansicht-abspann.tex).
%% ---------------------------------------------------------------

\normalsize

% Das esempio-Environment wird nur in der Leseansicht benötigt
\ifkorrekturansicht\else
\newenvironment{esempio}[3]%
{
    \vspace{1.5ex}
    \rlap{\underline{#1}}
    \par
    \setlength{\parindent}{0cm}
    \nopagebreak
    \leftskip=#2cm
    \rightskip=#3cm
}
{
    \par
}
\fi

\doendnotes{C}
\bigskip
\vfill

\clearpage

\footnotesize

\ifkorrekturansicht
  \lohead{\textsc{register}}
\fi

% theindex-Environment neu definieren ohne reledmac
\makeatletter
\renewenvironment{theindex}{%
  \ifkorrekturansicht
    \section*{\indexname}%
  \else
    \subsubsection*{Index der erwähnten Entitäten}%
  \fi
  \setlength{\parindent}{0pt}%
  \setlength{\parskip}{0pt plus 0.3pt}%
  \let\item\@idxitem
}{%
  \ifkorrekturansicht\clearpage\fi
}
\makeatother

\IfFileExists{\jobname-pw.ind}{\input{\jobname-pw.ind}}{}

% Quellenangabe nur in der Leseansicht
\ifkorrekturansicht\else
% Fallback-Definitionen, falls die .tex-Datei \titel etc. nicht gesetzt hat
\providecommand{\titel}{}
\providecommand{\editorInnen}{}
\providecommand{\dateiname}{\jobname}

\vspace{3cm}

\vfill

\footnotesize
\textsc{Quelle}: \titel. Herausgegeben von {\editorInnen}. In: \emph{Arthur Schnitzler: Briefwechsel mit Autorinnen und Autoren}.
 Digitale Edition, https://schnitzler-briefe.acdh.oeaw.ac.at/{\dateiname}.html (Stand \today)
\fi

\end{document}


      