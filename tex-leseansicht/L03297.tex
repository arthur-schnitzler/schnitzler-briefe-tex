%% latex-leseansicht-vorspann.tex
%% Vorspann für die Leseansicht.
%% Lädt die gemeinsame Datei latex-vorspann.tex mit nicht gesetztem Schalter.

\newif\ifkorrekturansicht
\korrekturansichtfalse

\input{../tex-inputs/latex-vorspann}

\begin{center}
            \textcolor{red}{ENTWURF, NICHT FERTIG KORRIGIERT}
                      \end{center}
            
         
         \renewcommand{\erwaehntePersonen}{Personen: Richard Beer-Hofmann, Houston Stewart Chamberlain, Ottilie Salten, Franz Servaes, Karl von Thaler}
         \renewcommand{\erwaehnteInstitutionen}{Institutionen: Neue Freie Presse}
         \renewcommand{\erwaehnteOrte}{Orte: Bad Ischl, Dubrovnik}
         \renewcommand{\erwaehnteWerke}{Werke: Decadence-Romane, Die Grundlagen des neunzehnten Jahrhunderts, Neue Freie Presse}
               \section[Felix Salten an Arthur Schnitzler, {[}17. 8. 1899{]}]{ Felix Salten an Arthur Schnitzler, {[}17. 8. 1899{]}}\nopagebreak\mylabel{v}\rehead{ }\begin{ledgroupsized}[t]{13cm}\normalsize\beginnumbering \toendnotes[C]{\smallbreak\pagebreak[2]} \Standort{CUL, Schnitzler, B 89, A 2.}
\physDesc{Karte, 1184 Zeichen
\newline{}Handschrift: Bleistift, lateinische Kurrent
\newline{}Schnitzler: mit Bleistift datiert: »17/8 99.« 
\newline{}Ordnung: mit Bleistift von unbekannter Hand nummeriert:
                                    »121« }\toendnotes[C]{\smallbreak}\pstart
           \noindent{}{\pb}Lieber Freund, den Gedanken an eine Radtour scheinen Sie selbst
               aufgegeben zu haben, – nun, ich hätte auch nur sehr schwer abkommen können, und es
               ist mir ganz recht. Sind Sie dafür im September oder halben Oktober vielleicht für
                  Ragusa\oindex{Dubrovnik@\textbf{Dubrovnik}|pw} zu haben? Ich möchte gerne auf acht
               Tage dahingehen. In der nächsten Woche komme ich vermutlich auf einen od. zwei Tage
               nach Ischl\oindex{Bad Ischl@\textbf{Bad Ischl}|pw}. Ich zeige Ihnen das jedenfalls noch
               genau an. Haben Sie heute das Feuilleton\pwindex{Servaes, Franz 17.06.1862 – 14.07.1947@\textsc{Servaes, Franz} (17.06.1862 – 14.07.1947), \emph{Journalist, Kritiker}!Decadence-Romane1899-08-17@\strich\emph{Decadence-Romane} {[}1899-08-17{]}|pwv} von Franz Servaes\pwindex{Servaes, Franz 17.06.1862 – 14.07.1947@\textsc{Servaes, Franz} (17.06.1862 – 14.07.1947), \emph{Journalist, Kritiker}|pw} gelesen?
                  »\label{K_L03297-1v}\edtext{Decadence Romane\pwindex{Servaes, Franz 17.06.1862 – 14.07.1947@\textsc{Servaes, Franz} (17.06.1862 – 14.07.1947), \emph{Journalist, Kritiker}!Decadence-Romane1899-08-17@\strich\emph{Decadence-Romane} {[}1899-08-17{]}|pw}}{\lemma{\textnormal{\emph{Decadence Romane}}}\Cendnote{\textnormal{Franz Servaes\pwindex{Servaes, Franz 17.06.1862 – 14.07.1947@\textsc{Servaes, Franz} (17.06.1862 – 14.07.1947), \emph{Journalist, Kritiker}|pwk}: \emph{Decadence-Romane}\pwindex{Servaes, Franz 17.06.1862 – 14.07.1947@\textsc{Servaes, Franz} (17.06.1862 – 14.07.1947), \emph{Journalist, Kritiker}!Decadence-Romane1899-08-17@\strich\emph{Decadence-Romane} {[}1899-08-17{]}|pwk}. In: \emph{Neue Freie Presse}\pwindex{Neue Freie Presse1864 – 1939@\emph{Neue Freie Presse} {[}1864 – 1939{]}|pwk}, Nr. 12.566, 17. 8. 1899, Morgenblatt,
                     S. 1–3.}}}\label{K_L03297-1h}« – - Die Neue freie
                  Presse\orgindex{Neue Freie Presse@Neue Freie Presse|pw} brauchte für den alternden Karl v.
                  Thaler\pwindex{Thaler, Karl von 30.09.1836 – 20.10.1919@\textsc{Thaler, Karl von} (30.09.1836 – 20.10.1919), \emph{Schriftsteller/Schriftstellerin, Journalist/Journalistin}|pw} einen Ersatz und hat ihn in Servaes\pwindex{Servaes, Franz 17.06.1862 – 14.07.1947@\textsc{Servaes, Franz} (17.06.1862 – 14.07.1947), \emph{Journalist, Kritiker}|pw} gefunden, nur dass mir Servaes\pwindex{Servaes, Franz 17.06.1862 – 14.07.1947@\textsc{Servaes, Franz} (17.06.1862 – 14.07.1947), \emph{Journalist, Kritiker}|pw} mit seinem Orientirtsein noch eckel{\pb}hafter ist. Wo befindet sich Beer-Hofmann\pwindex{Beer-Hofmann, Richard 1866-07-11 – 1945-09-26@\textsc{Beer-Hofmann, Richard} (1866-07-11 – 1945-09-26), \emph{Schriftsteller}|pw} jetzt? \pend
           \pstart
           Otti\pwindex{Salten, Ottilie 07.03.1868 – 22.06.1942@\textsc{Salten, Ottilie} (07.03.1868 – 22.06.1942), \emph{Schauspielerin}|pw} ist in Ischl\oindex{Bad Ischl@\textbf{Bad Ischl}|pw}. Wahrscheinlich haben Sie sie schon gesehen. Sie hat noch kein
               Engagement, ist aber im Ganzen ruhiger. Ich bin die ganze Zeit schlecht aufgelegt,
               aber ich arbeite viel. »Die Grundlagen des
                  Jahrhunderts\pwindex{Chamberlain, Houston Stewart 09.09.1855 – 09.01.1927@\textsc{Chamberlain, Houston Stewart} (09.09.1855 – 09.01.1927), \emph{Schriftsteller}!Grundlagen des neunzehnten Jahrhunderts1899@\strich\emph{Die Grundlagen des neunzehnten Jahrhunderts} {[}1899{]}|pw}« von Chamberlain\pwindex{Chamberlain, Houston Stewart 09.09.1855 – 09.01.1927@\textsc{Chamberlain, Houston Stewart} (09.09.1855 – 09.01.1927), \emph{Schriftsteller}|pw} ist ein
               sehr interessantes Buch. Ich gebe es Ihnen, wenn Sie zurückkommen. Ich schreibe
               augenblicklich darüber eine Anzahl von Entgegnungen, für »Die Welt«.\pend
           \pstart
           Senden Sie mir bald wieder eine Zeile. – Die Zeitungen bringe ich Ihnen selbst mit. \pend
           \pstart
           Herzlichst Ihr {\\[\baselineskip]}\spacefill\mbox{Salten}\pend
           \leftskip=0em{}
         
         \endnumbering\mylabel{h}\end{ledgroupsized}\begin{anhang}\end{anhang}\newcommand{\dateiname}{L03297}\newcommand{\titel}{Felix Salten an Arthur Schnitzler, [17. 8. 1899]}\newcommand{\editorInnen}{Martin Anton Müller und Laura Untner}%% latex-leseansicht-abspann.tex
%% Abspann für die Leseansicht.
%% Der Schalter \ifkorrekturansicht ist bereits durch den Vorspann gesetzt.

%% latex-abspann.tex
%% Gemeinsamer Abspann für Korrekturansicht und Leseansicht.
%% Setzt den Schalter \ifkorrekturansicht voraus (gesetzt in den
%% einbindenden Dateien latex-korrekturansicht-abspann.tex bzw.
%% latex-leseansicht-abspann.tex).
%% ---------------------------------------------------------------

\normalsize

% Das esempio-Environment wird nur in der Leseansicht benötigt
\ifkorrekturansicht\else
\newenvironment{esempio}[3]%
{
    \vspace{1.5ex}
    \rlap{\underline{#1}}
    \par
    \setlength{\parindent}{0cm}
    \nopagebreak
    \leftskip=#2cm
    \rightskip=#3cm
}
{
    \par
}
\fi

\doendnotes{C}
\bigskip
\vfill

\clearpage

\footnotesize

\ifkorrekturansicht
  \lohead{\textsc{register}}
\fi

% theindex-Environment neu definieren ohne reledmac
\makeatletter
\renewenvironment{theindex}{%
  \ifkorrekturansicht
    \section*{\indexname}%
  \else
    \subsubsection*{Index der erwähnten Entitäten}%
  \fi
  \setlength{\parindent}{0pt}%
  \setlength{\parskip}{0pt plus 0.3pt}%
  \let\item\@idxitem
}{%
  \ifkorrekturansicht\clearpage\fi
}
\makeatother

\IfFileExists{\jobname-pw.ind}{\input{\jobname-pw.ind}}{}

% Quellenangabe nur in der Leseansicht
\ifkorrekturansicht\else
% Fallback-Definitionen, falls die .tex-Datei \titel etc. nicht gesetzt hat
\providecommand{\titel}{}
\providecommand{\editorInnen}{}
\providecommand{\dateiname}{\jobname}

\vspace{3cm}

\vfill

\footnotesize
\textsc{Quelle}: \titel. Herausgegeben von {\editorInnen}. In: \emph{Arthur Schnitzler: Briefwechsel mit Autorinnen und Autoren}.
 Digitale Edition, https://schnitzler-briefe.acdh.oeaw.ac.at/{\dateiname}.html (Stand \today)
\fi

\end{document}


      