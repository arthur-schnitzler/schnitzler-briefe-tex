%% latex-leseansicht-vorspann.tex
%% Vorspann für die Leseansicht.
%% Lädt die gemeinsame Datei latex-vorspann.tex mit nicht gesetztem Schalter.

\newif\ifkorrekturansicht
\korrekturansichtfalse

\input{../tex-inputs/latex-vorspann}


         
         \newcommand{\erwaehntePersonen}{Personen: Hermann Bahr, Emerich von Bukovics}
         \newcommand{\erwaehnteOrte}{Orte: Wien}
         \newcommand{\erwaehnteWerke}{Werke: Der grüne Kakadu. Groteske in einem Akt, Die Frau mit dem Dolche, Die letzten Masken, Fastnacht, Literatur, Marionetten. Drei Einakter}
               \section[Arthur Schnitzler an Hermann Bahr, {[}14. 3.? 1901{]}]{ Arthur Schnitzler an Hermann Bahr, {[}14. 3.? 1901{]}}\nopagebreak\mylabel{v}\rehead{ }\begin{ledgroupsized}[t]{13cm}\normalsize\beginnumbering \toendnotes[C]{\smallbreak\pagebreak[2]} \Standort{TMW, HS AM 23339 Ba.}
\physDesc{Brief, 1 Blatt, 3 Seiten
\newline{}Handschrift: Bleistift, deutsche Kurrent\newline{}Ordnung: Lochung }\buchAbdrucke{\weitereDrucke{1) \emph{[September 1901?].} In: Arthur Schnitzler: \emph{The Letters of Arthur Schnitzler to Hermann Bahr}. Edited, annotated, and with an introduction, by Donald G.
                        Daviau. Chapel Hill: \emph{The University of North Carolina Press} 1978, S. 69–70 (University of North Carolina studies in the Germanic languages
                        and literatures, 89).} \weitereDrucke{2) Hermann Bahr, Arthur Schnitzler: \emph{Briefwechsel, Aufzeichnungen, Dokumente (1891–1931)}. Hg. Kurt Ifkovits und Martin Anton Müller. Göttingen: \emph{Wallstein} 2018, S. 202.} }\toendnotes[C]{\smallbreak}\pstart
           \noindent{}{\pb}mein lieber Hermann, es handelt ſich um nichts wichtiges; vielleicht
                  ka{\geminationn} ich also Dienſtg Vormittg zu dir –
               ohne dich im geringſten zu binden. Eines ka{\geminationn} ich dir
               vielleicht gleich hier ſagen, wobei ich dich bitte, gelegentlich zu \textsc{Bukovis}\pwindex{Bukovics, Emerich von 28.02.1844 – 04.07.1905@\textsc{Bukovics, Emerich von} (28.02.1844 – 04.07.1905), \emph{Journalist, Theaterleiter}|pw} davon zu reden.\pend
           \pstart
           \label{K_L01103_1v}\edtext{Mein Einakterabend wird beſtehen}{\lemma{\textnormal{\emph{Mein … beſtehen}}}\Cendnote{\textnormal{Zur Vorgeschichte, die sich Ende
                     Februar ereignete, vgl. den Brief Schnitzler\pwindex{Schnitzler, Arthur 15.05.1862 – 21.10.1931@\textsc{Schnitzler, Arthur} (15.05.1862 – 21.10.1931), \emph{Schriftsteller, Mediziner}|pwk}s an Emerich von
                     Bukovics\pwindex{Bukovics, Emerich von 28.02.1844 – 04.07.1905@\textsc{Bukovics, Emerich von} (28.02.1844 – 04.07.1905), \emph{Journalist, Theaterleiter}|pwk}, 11. 12. 1901, in \emph{Briefwechsel} Bahr/Schnitzler 219–220}}}\label{K_L01103_1h} aus »Literatur\pwindex{Schnitzler, Arthur 15.05.1862 – 21.10.1931@\textsc{Schnitzler, Arthur} (15.05.1862 – 21.10.1931), \emph{Schriftsteller, Mediziner}!Literatur1901@\strich\emph{Literatur} {[}1901{]}|pw}«, einem \label{K_L01103_2v}\edtext{andern\pwindex{Schnitzler, Arthur 15.05.1862 – 21.10.1931@\textsc{Schnitzler, Arthur} (15.05.1862 – 21.10.1931), \emph{Schriftsteller, Mediziner}!Frau mit dem Dolche1901@\strich\emph{Die Frau mit dem Dolche} {[}1901{]}|pwv}, der halb fertig iſt
               ziemlich phantaſtiſch}{\lemma{\textnormal{\emph{andern, … phantaſtiſch}}}\Cendnote{\textnormal{Durch
                  »phantastisch« scheint auf \emph{Die Frau mit dem
                     Dolche}\pwindex{Schnitzler, Arthur 15.05.1862 – 21.10.1931@\textsc{Schnitzler, Arthur} (15.05.1862 – 21.10.1931), \emph{Schriftsteller, Mediziner}!Frau mit dem Dolche1901@\strich\emph{Die Frau mit dem Dolche} {[}1901{]}|pwk} Bezug genommen zu sein, wobei die Niederschrift erst zwischen
                     Mai und August datierbar ist.}}}\label{K_L01103_2h} und {\pb}einem \label{K_L01103_3v}\edtext{dritten\pwindex{Schnitzler, Arthur 15.05.1862 – 21.10.1931@\textsc{Schnitzler, Arthur} (15.05.1862 – 21.10.1931), \emph{Schriftsteller, Mediziner}!letzten Masken1901@\strich\emph{Die letzten Masken} {[}1901{]}|pwuv}}{\lemma{\textnormal{\emph{dritten}}}\Cendnote{\textnormal{\label{LKommKL038-3v}Vermutlich \emph{Die letzten Masken}\pwindex{Schnitzler, Arthur 15.05.1862 – 21.10.1931@\textsc{Schnitzler, Arthur} (15.05.1862 – 21.10.1931), \emph{Schriftsteller, Mediziner}!letzten Masken1901@\strich\emph{Die letzten Masken} {[}1901{]}|pwk}. Seit 12. 3. 1901 lag der Stoff
                     als Novelle abgeschlossen vor, und am »24. 4.?« (\emph{Cambridge University Library}, Schnitzler,
                        A 80) versuchte Schnitzler\pwindex{Schnitzler, Arthur 15.05.1862 – 21.10.1931@\textsc{Schnitzler, Arthur} (15.05.1862 – 21.10.1931), \emph{Schriftsteller, Mediziner}|pwk},
                     ihn dramatisch zu bearbeiten.\label{LKommKL038-3h}}}}\label{K_L01103_3h} – den ich noch nicht begonnen habe. –\pend
           \pstart
           Dagegen ſoll Marionetten\pwindex{Schnitzler, Arthur 15.05.1862 – 21.10.1931@\textsc{Schnitzler, Arthur} (15.05.1862 – 21.10.1931), \emph{Schriftsteller, Mediziner}!Marionetten. Drei Einakter1906@\strich\emph{Marionetten. Drei Einakter} {[}1906{]}|pw} (das hier beſti{\geminationm}t gut wirken wird, in guter Darſtellung) da es doch als
               ſagen wir Literaturſatire nur einen kleinen Kreis intereſſiren kann) lieber an dem
               Abend gegeben werden, wo der Kakadu\pwindex{Schnitzler, Arthur 15.05.1862 – 21.10.1931@\textsc{Schnitzler, Arthur} (15.05.1862 – 21.10.1931), \emph{Schriftsteller, Mediziner}!gruene Kakadu. Groteske in einem Akt1. 3. 1899@\strich\emph{Der grüne Kakadu. Groteske in einem Akt} {[}1. 3. 1899{]}|pw} aufgeführt
               wird. Alſo irgend was von einem andern (man {\pb}ſprach mir von »\textsc{Fast}nacht\pwindex{\textcolor{red}{\textsuperscript{XXXX1 indx}}!Fastnacht1900@\strich\emph{Fastnacht} {[}1900{]}|pw}«) dann Kakadu\pwindex{Schnitzler, Arthur 15.05.1862 – 21.10.1931@\textsc{Schnitzler, Arthur} (15.05.1862 – 21.10.1931), \emph{Schriftsteller, Mediziner}!gruene Kakadu. Groteske in einem Akt1. 3. 1899@\strich\emph{Der grüne Kakadu. Groteske in einem Akt} {[}1. 3. 1899{]}|pw}, am Schluſs \textsc{Marionetten}\pwindex{Schnitzler, Arthur 15.05.1862 – 21.10.1931@\textsc{Schnitzler, Arthur} (15.05.1862 – 21.10.1931), \emph{Schriftsteller, Mediziner}!Marionetten. Drei Einakter1906@\strich\emph{Marionetten. Drei Einakter} {[}1906{]}|pw}.\pend
           \pstart
           Nun, darüber und \introOben{}über\introOben{} einiges andere nächſtens.\pend
           \pstart
           Viele herzliche Grüße{\\[\baselineskip]}dein{\\[\baselineskip]}\spacefill\mbox{ArthurSch}\pend
           \leftskip=0em{}
         
         \endnumbering\mylabel{h}\end{ledgroupsized}  \newcommand{\dateiname}{L01103}\newcommand{\titel}{Arthur Schnitzler an Hermann Bahr, [14. 3.? 1901]}\newcommand{\editorInnen}{ Kurt Ifkovits,  Martin Anton Müller}%% latex-leseansicht-abspann.tex
%% Abspann für die Leseansicht.
%% Der Schalter \ifkorrekturansicht ist bereits durch den Vorspann gesetzt.

%% latex-abspann.tex
%% Gemeinsamer Abspann für Korrekturansicht und Leseansicht.
%% Setzt den Schalter \ifkorrekturansicht voraus (gesetzt in den
%% einbindenden Dateien latex-korrekturansicht-abspann.tex bzw.
%% latex-leseansicht-abspann.tex).
%% ---------------------------------------------------------------

\normalsize

% Das esempio-Environment wird nur in der Leseansicht benötigt
\ifkorrekturansicht\else
\newenvironment{esempio}[3]%
{
    \vspace{1.5ex}
    \rlap{\underline{#1}}
    \par
    \setlength{\parindent}{0cm}
    \nopagebreak
    \leftskip=#2cm
    \rightskip=#3cm
}
{
    \par
}
\fi

\doendnotes{C}
\bigskip
\vfill

\clearpage

\footnotesize

\ifkorrekturansicht
  \lohead{\textsc{register}}
\fi

% theindex-Environment neu definieren ohne reledmac
\makeatletter
\renewenvironment{theindex}{%
  \ifkorrekturansicht
    \section*{\indexname}%
  \else
    \subsubsection*{Index der erwähnten Entitäten}%
  \fi
  \setlength{\parindent}{0pt}%
  \setlength{\parskip}{0pt plus 0.3pt}%
  \let\item\@idxitem
}{%
  \ifkorrekturansicht\clearpage\fi
}
\makeatother

\IfFileExists{\jobname-pw.ind}{\input{\jobname-pw.ind}}{}

% Quellenangabe nur in der Leseansicht
\ifkorrekturansicht\else
% Fallback-Definitionen, falls die .tex-Datei \titel etc. nicht gesetzt hat
\providecommand{\titel}{}
\providecommand{\editorInnen}{}
\providecommand{\dateiname}{\jobname}

\vspace{3cm}

\vfill

\footnotesize
\textsc{Quelle}: \titel. Herausgegeben von {\editorInnen}. In: \emph{Arthur Schnitzler: Briefwechsel mit Autorinnen und Autoren}.
 Digitale Edition, https://schnitzler-briefe.acdh.oeaw.ac.at/{\dateiname}.html (Stand \today)
\fi

\end{document}


      