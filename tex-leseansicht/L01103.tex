%% latex-korrekturansicht-vorspann.tex
%% Vorspann für die Korrekturansicht.
%% Lädt die gemeinsame Datei latex-vorspann.tex mit gesetztem Schalter.

\newif\ifkorrekturansicht
\korrekturansichttrue

\input{../tex-inputs/latex-vorspann}


\section[Arthur Schnitzler an Hermann Bahr, {[}14. 3.? 1901{]}]{L01103 Arthur Schnitzler an Hermann Bahr, {[}14. 3.? 1901{]}}
\nopagebreak\mylabel{L01103v}
\rehead{ }\normalsize\beginnumbering\briefempfaengerindex{Bahr, Hermann@\textsc{Bahr, Hermann}!zzzSchnitzler, Arthur@\emph{von Arthur Schnitzler}!1901-03-141@{{[}14. 3.? 1901{]}}|(be}
\toendnotes[C]{\smallbreak\pagebreak[2]}\Standort{TMW, HS AM 23339 Ba.}
\physDesc{Brief, 1 Blatt, 3 Seiten, 807 Zeichen
\newline{}Handschrift: Bleistift, deutsche Kurrent
\newline{}Ordnung: Lochung }
\buchAbdrucke{\weitereDrucke{1) Arthur Schnitzler: \emph{The Letters of Arthur Schnitzler to Hermann Bahr}. Chapel Hill: \emph{The University of North Carolina Press} 1978, S. 69–70.} \weitereDrucke{2) Hermann Bahr, Arthur Schnitzler: \emph{Briefwechsel, Aufzeichnungen, Dokumente (1891–1931)}. Göttingen: \emph{Wallstein} 2018, S. 202.} }\toendnotes[C]{\smallbreak}
\pstart
           \noindent{}{\pb}mein lieber Hermann, es handelt ſich um nichts wichtiges; vielleicht
                  ka{\geminationn} ich also Dienſtg Vormittg zu dir –
               ohne dich im geringſten zu binden. Eines ka{\geminationn} ich dir
               vielleicht gleich hier ſagen, wobei ich dich bitte, gelegentlich zu \textsc{Bukovis}\pwindex{Bukovics, Emerich von 28.02.1844 – 04.07.1905@\textsc{Bukovics, Emerich von} (28.02.1844 – 04.07.1905), \emph{Journalist/Journalistin, Theaterleiter/Theaterleiterin}|pw} davon zu reden.\pend
           
\pstart
           \label{K_L01103-1v}\edtext{Mein Einakterabend wird beſtehen}{\lemma{\textnormal{\emph{Mein … beſtehen}}}\Cendnote{\textnormal{Zur Vorgeschichte, die sich Ende
                     Februar ereignete, vgl. Hermann Bahr, Arthur Schnitzler: \emph{Briefwechsel, Aufzeichnungen, Dokumente (1891–1931)}, Arthur Schnitzler an Emerich von Bukovics, 11. 12. 1901.
               }}}\label{K_L01103-1} aus »Literatur\pwindex{Literatur@\emph{Literatur}|pw}«, einem \label{K_L01103-2v}\edtext{andern\pwindex{Frau mit dem Dolche@\emph{Die Frau mit dem Dolche}|pwv}, der halb fertig iſt
               ziemlich phantaſtiſch}{\lemma{\textnormal{\emph{andern, … phantaſtiſch}}}\Cendnote{\textnormal{Durch
                  »phantastisch« scheint auf \emph{Die Frau mit dem
                     Dolche}\pwindex{Frau mit dem Dolche@\emph{Die Frau mit dem Dolche}|pwk} Bezug genommen zu sein, wobei die Niederschrift erst zwischen
                     Mai und August datierbar ist.}}}\label{K_L01103-2} und {\pb}einem \label{K_L01103-3v}\edtext{dritten\pwindex{letzten Masken@\emph{Die letzten Masken}|pwuv}}{\lemma{\textnormal{\emph{dritten}}}\Cendnote{\textnormal{\label{LKommKL038-3v}Vermutlich \emph{Die letzten Masken}\pwindex{letzten Masken@\emph{Die letzten Masken}|pwk}. Seit 12. 3. 1901 lag der
                     Stoff als Novelle abgeschlossen vor, und am »24. 4.?« (\emph{Cambridge University Library}, Schnitzler,
                     A 80) versuchte Schnitzler, ihn
                     dramatisch zu bearbeiten.\label{LKommKL038-3h}}}}\label{K_L01103-3} – den ich noch nicht begonnen habe. –\pend
           
\pstart
           Dagegen ſoll Marionetten\pwindex{Marionetten. Drei Einakter@\emph{Marionetten. Drei Einakter}|pw} (das hier beſti{\geminationm}t gut wirken wird, in guter Darſtellung) da es doch als
               ſagen wir Literaturſatire nur einen kleinen Kreis intereſſiren kann) lieber an dem
               Abend gegeben werden, wo der Kakadu\pwindex{gruene Kakadu. Groteske in einem Akt@\emph{Der grüne Kakadu. Groteske in einem Akt}|pw} aufgeführt
               wird. Alſo irgend was von einem andern (man {\pb}ſprach mir von »\textsc{Fast}nacht\pwindex{Fastnacht@\emph{Fastnacht}|pw}«) dann Kakadu\pwindex{gruene Kakadu. Groteske in einem Akt@\emph{Der grüne Kakadu. Groteske in einem Akt}|pw}, am Schluſs \textsc{Marionetten}\pwindex{Marionetten. Drei Einakter@\emph{Marionetten. Drei Einakter}|pw}.\pend
           
\pstart
           Nun, darüber und \introOben{}über\introOben{} einiges andere nächſtens.\pend
           
\pstart
           Viele herzliche Grüße{\\[\baselineskip]}dein{\\[\baselineskip]}\spacefill\mbox{ArthurSch}\pend
           \leftskip=0em{}\selectlanguage{ngerman}\endnumbering\briefempfaengerindex{Bahr, Hermann@\textsc{Bahr, Hermann}!zzzSchnitzler, Arthur@\emph{von Arthur Schnitzler}!1901-03-141@{{[}14. 3.? 1901{]}}|)be}\mylabel{L01103h}  \normalsize

\doendnotes{C}
\bigskip
\vfill

\clearpage

\footnotesize

\lohead{\textsc{register}}

% Definiere theindex-Environment komplett neu ohne reledmac
\makeatletter
\renewenvironment{theindex}{%
  \section*{\indexname}%
  \setlength{\parindent}{0pt}%
  \setlength{\parskip}{0pt plus 0.3pt}%
  \let\item\@idxitem
}{%
  \clearpage
}
\makeatother

\IfFileExists{\jobname-pw.ind}{\input{\jobname-pw.ind}}{}

\end{document}

      