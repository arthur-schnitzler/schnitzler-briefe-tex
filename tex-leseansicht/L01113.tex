%% latex-korrekturansicht-vorspann.tex
%% Vorspann für die Korrekturansicht.
%% Lädt die gemeinsame Datei latex-vorspann.tex mit gesetztem Schalter.

\newif\ifkorrekturansicht
\korrekturansichttrue

\input{../tex-inputs/latex-vorspann}


\section[Arthur Schnitzler: Widmungsexemplar Frau Bertha Garlan für Hugo von Hofmannsthal, {[}24.?{]} 4. 1901]{L01113 Arthur Schnitzler: Widmungsexemplar Frau Bertha Garlan für Hugo von
               Hofmannsthal, {[}24.?{]} 4. 1901}
\nopagebreak\mylabel{L01113v}
\rehead{ }\normalsize\beginnumbering\briefempfaengerindex{Hofmannsthal, Hugo von@\textsc{Hofmannsthal, Hugo von}!zzzSchnitzler, Arthur@\emph{von Arthur Schnitzler}!1901-04-242@{{[}24.?{]} 4. 1901}|(be}
\toendnotes[C]{\smallbreak\pagebreak[2]}\Standort{FDH, FDH 3224.}
\physDesc{Widmung am Vorsatzblatt, 46 Zeichen
\newline{}Handschrift: schwarze Tinte, deutsche Kurrent}
\buchAbdrucke{\weitereDrucke{Hugo von Hofmannsthal: \emph{Bibliothek}. Frankfurt am Main: \emph{S. Fischer} 2011, S. 604.} }\toendnotes[C]{\smallbreak}
\pstart
           \noindent{}\centering{}{\pb}Meinem lieben Hugo\pend
           \pstart \spacefill\mbox{Arthur Sch}\pend{}
\pstart
           \noindent{}Wien\oindex{Wien@\textbf{Wien}, \emph{A.ADM2}|pw}, \label{K_L01113-1v}\edtext{April 901}{\lemma{\textnormal{\emph{April 901}}}\Cendnote{\textnormal{Die Datierung erfolgt in Abgleich zum
                        Widmungsexemplar für Bahr\pwindex{Bahr, Hermann 19.07.1863 – 15.01.1934@\textsc{Bahr, Hermann} (19.07.1863 – 15.01.1934), \emph{Schriftsteller/Schriftstellerin, Kritiker/Kritikerin}|pwk}, für das
                        sich dieser am 25. 4. 1901 bedankt hat.}}}\label{K_L01113-1}.\pend
           {\vspace{1\baselineskip}}\selectlanguage{ngerman}\vspace{1em}{\vspace{1\baselineskip}}
\pstart
           \centering{}{\pb}\textcolor{gray}{\textbf{Frau Bertha Garlan\pwindex{Frau Bertha Garlan. Roman@\emph{Frau Bertha Garlan. Roman}|pw}}}\pend
           
\pstart
           \centering{}\textcolor{gray}{\textbf{\so{Roman}}}\pend
           
\pstart
           \centering{}\textcolor{gray}{\textbf{von}}\pend
           
\pstart
           \centering{}\textcolor{gray}{\textbf{Arthur Schnitzler}}\pend
           
\pstart
           \centering{}\textcolor{gray}{\textbf{Zweite Auflage}}\pend
           {\vspace{1\baselineskip}}
\pstart
           \centering{}\textcolor{gray}{\textbf{\textbf{Berlin}\oindex{Berlin@\textbf{Berlin}, \emph{P.PPLC}|pw}}}\pend
           
\pstart
           \centering{}\textcolor{gray}{\textbf{\so{S. Fiſcher, Verlag}\orgindex{S. Fischer Verlag@S. Fischer Verlag|pw}}}\pend
           
\pstart
           \centering{}\textcolor{gray}{\textbf{1901}}\pend
           \selectlanguage{ngerman}\endnumbering\briefempfaengerindex{Hofmannsthal, Hugo von@\textsc{Hofmannsthal, Hugo von}!zzzSchnitzler, Arthur@\emph{von Arthur Schnitzler}!1901-04-242@{{[}24.?{]} 4. 1901}|)be}\mylabel{L01113h}  \normalsize

\doendnotes{C}
\bigskip
\vfill

\clearpage

\footnotesize

\lohead{\textsc{register}}

% Definiere theindex-Environment komplett neu ohne reledmac
\makeatletter
\renewenvironment{theindex}{%
  \section*{\indexname}%
  \setlength{\parindent}{0pt}%
  \setlength{\parskip}{0pt plus 0.3pt}%
  \let\item\@idxitem
}{%
  \clearpage
}
\makeatother

\IfFileExists{\jobname-pw.ind}{\input{\jobname-pw.ind}}{}

\end{document}

      