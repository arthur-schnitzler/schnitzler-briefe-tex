%% latex-leseansicht-vorspann.tex
%% Vorspann für die Leseansicht.
%% Lädt die gemeinsame Datei latex-vorspann.tex mit nicht gesetztem Schalter.

\newif\ifkorrekturansicht
\korrekturansichtfalse

\input{../tex-inputs/latex-vorspann}


\section[Arthur Schnitzler an Hugo von Hofmannsthal, 21. 12. 1912]{L02110 Arthur Schnitzler an Hugo von Hofmannsthal, 21. 12. 1912}
\nopagebreak\mylabel{L02110v}
\rehead{ }\normalsize\beginnumbering\briefempfaengerindex{Hofmannsthal, Hugo von@\textsc{Hofmannsthal, Hugo von}!zzzSchnitzler, Arthur@\emph{von Arthur Schnitzler}!1912-12-211@{21. 12. 1912}|(be}
\toendnotes[C]{\smallbreak\pagebreak[2]}
\correspDesc{Versand  durch Arthur Schnitzler am 21. 12. 1912 in Wien
\newline{}Erhalt  durch Hugo von Hofmannsthal im Zeitraum [21. 12. 1912 – 25. 12. 1912?] in Wien}\toendnotes[C]{\smallbreak}
\Standort{FDH, Hs-30885,146.}
\physDesc{Briefkarte, 500 Zeichen
\newline{}Handschrift: schwarze Tinte, deutsche Kurrent}
\buchAbdrucke{\weitereDrucke{Hugo von Hofmannsthal, Arthur Schnitzler: \emph{Briefwechsel}. Herausgegeben von Therese Nickl und Heinrich Schnitzler. Frankfurt am Main: \emph{S. Fischer} 1964, S. 271.} }\toendnotes[C]{\smallbreak}
\pstart
           {\pb}\textcolor{gray}{\textbf{Dr. Arthur Schnitzler}}\hfill Wien\oindex{Wien@\textbf{Wien}, \emph{Verwaltungsgebiet}|pw}, 21. 12. 912\pend
           
\pstart
           \textcolor{gray}{\textbf{Wien XVIII. Sternwartestrasse 71\oindex{Wien@\textbf{Wien}!XVIII., Währing@\textbf{XVIII., Währing}!Sternwartestraße 71@\textbf{Sternwartestraße 71}, \emph{Wohngebäude}|pw}}}\pend
           \vspace{0.5em}
\pstart
           lieber Hugo, eben mit dem V. Th.\oindex{Wien@\textbf{Wien}!VII., Neubau@\textbf{VII., Neubau}!Volkstheater@\textbf{Volkstheater}, \emph{Theater}|pw}{ }\textsc{telephonirt};{ }ſie haben mit der \textsc{Roland}\pwindex{Roland, Ida 18.\,2.\,1881 Wien – 27.\,3.\,1951 Nyon@\textsc{Roland, Ida} (18.\,2.\,1881 Wien – 27.\,3.\,1951 Nyon), \emph{Schauspielerin}|pw} noch nicht abgeschloſſen,{ }ſchienen über die Ausſicht \textsc{Terwin}\pwindex{Terwin, Johanna 18.\,3.\,1884 Kaiserslautern – 4.\,1.\,1962 Zürich@\textsc{Terwin, Johanna} (18.\,3.\,1884 Kaiserslautern – 4.\,1.\,1962 Zürich), \emph{Schauspielerin}|pw} poſitiv erfreut. Würde rathen, daſs{ }ſich die T.\pwindex{Terwin, Johanna 18.\,3.\,1884 Kaiserslautern – 4.\,1.\,1962 Zürich@\textsc{Terwin, Johanna} (18.\,3.\,1884 Kaiserslautern – 4.\,1.\,1962 Zürich), \emph{Schauspielerin}|pw} ganz direct mit dem V. Th.\oindex{Wien@\textbf{Wien}!VII., Neubau@\textbf{VII., Neubau}!Volkstheater@\textbf{Volkstheater}, \emph{Theater}|pw} in
               Verbindung{ }ſetzt; u. zw.{ }ſo geſchwind wie möglich. –\pend
           
\pstart
           Mit \textsc{Thimig}\pwindex{Thimig, Hugo 16.\,6.\,1854 Dresden – 24.\,9.\,1944 Wien@\textsc{Thimig, Hugo} (16.\,6.\,1854 Dresden – 24.\,9.\,1944 Wien), \emph{Theaterleiter, Schauspieler}|pw} heute nur zwei Worte {\pb}auf der Probe\pwindex{\textcolor{red}{\textsuperscript{XXXX indx1}}!Märchen vom Wolf@\strich\emph{Das Märchen vom Wolf}|pwv}; er habe mir einiges intereſſante zu{ }ſagen, werde mich nächſtens beſuchen. (Er war auch \label{K_L02110-1v}\edtext{vor ein paar Wochen}{\lemma{\textnormal{\emph{vor ein paar Wochen}}}\Cendnote{\textnormal{Siehe A. S.: \emph{Tagebuch}, 22. 10. 1912.
               }}}\label{K_L02110-1} bei mir) Bei dieſer Gelegenheit gedenke ich den Jederma{\geminationn}\pwindex{Hofmannsthal, Hugo von 1.\,2.\,1874 Wien – 15.\,7.\,1929 Rodaun@\textsc{Hofmannsthal, Hugo von} (1.\,2.\,1874 Wien – 15.\,7.\,1929 Rodaun), \emph{Schriftsteller}!Jedermann. Das Spiel vom Sterben des reichen Mannes@\strich\emph{Jedermann. Das Spiel vom Sterben des reichen Mannes}|pw} anzuſchneiden.\pend
           
\pstart
           Herzlichſt Ihr{\\[\baselineskip]}\spacefill\mbox{A.}\pend
           \leftskip=0em{}\selectlanguage{ngerman}\endnumbering\briefempfaengerindex{Hofmannsthal, Hugo von@\textsc{Hofmannsthal, Hugo von}!zzzSchnitzler, Arthur@\emph{von Arthur Schnitzler}!1912-12-211@{21. 12. 1912}|)be}\mylabel{L02110h}  \newcommand{\dateiname}{L02110}\newcommand{\titel}{Arthur Schnitzler an Hugo von Hofmannsthal, 21. 12. 1912}\newcommand{\editorInnen}{Martin Anton Müller und Gerd-Hermann Susen}%% latex-leseansicht-abspann.tex
%% Abspann für die Leseansicht.
%% Der Schalter \ifkorrekturansicht ist bereits durch den Vorspann gesetzt.

%% latex-abspann.tex
%% Gemeinsamer Abspann für Korrekturansicht und Leseansicht.
%% Setzt den Schalter \ifkorrekturansicht voraus (gesetzt in den
%% einbindenden Dateien latex-korrekturansicht-abspann.tex bzw.
%% latex-leseansicht-abspann.tex).
%% ---------------------------------------------------------------

\normalsize

% Das esempio-Environment wird nur in der Leseansicht benötigt
\ifkorrekturansicht\else
\newenvironment{esempio}[3]%
{
    \vspace{1.5ex}
    \rlap{\underline{#1}}
    \par
    \setlength{\parindent}{0cm}
    \nopagebreak
    \leftskip=#2cm
    \rightskip=#3cm
}
{
    \par
}
\fi

\doendnotes{C}
\bigskip
\vfill

\clearpage

\footnotesize

\ifkorrekturansicht
  \lohead{\textsc{register}}
\fi

% theindex-Environment neu definieren ohne reledmac
\makeatletter
\renewenvironment{theindex}{%
  \ifkorrekturansicht
    \section*{\indexname}%
  \else
    \subsubsection*{Index der erwähnten Entitäten}%
  \fi
  \setlength{\parindent}{0pt}%
  \setlength{\parskip}{0pt plus 0.3pt}%
  \let\item\@idxitem
}{%
  \ifkorrekturansicht\clearpage\fi
}
\makeatother

\IfFileExists{\jobname-pw.ind}{\input{\jobname-pw.ind}}{}

% Quellenangabe nur in der Leseansicht
\ifkorrekturansicht\else
% Fallback-Definitionen, falls die .tex-Datei \titel etc. nicht gesetzt hat
\providecommand{\titel}{}
\providecommand{\editorInnen}{}
\providecommand{\dateiname}{\jobname}

\vspace{3cm}

\vfill

\footnotesize
\textsc{Quelle}: \titel. Herausgegeben von {\editorInnen}. In: \emph{Arthur Schnitzler: Briefwechsel mit Autorinnen und Autoren}.
 Digitale Edition, https://schnitzler-briefe.acdh.oeaw.ac.at/{\dateiname}.html (Stand \today)
\fi

\end{document}


