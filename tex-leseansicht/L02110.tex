%% latex-korrekturansicht-vorspann.tex
%% Vorspann für die Korrekturansicht.
%% Lädt die gemeinsame Datei latex-vorspann.tex mit gesetztem Schalter.

\newif\ifkorrekturansicht
\korrekturansichttrue

\input{../tex-inputs/latex-vorspann}


\section[Arthur Schnitzler an Hugo von Hofmannsthal, 21. 12. 1912]{L02110 Arthur Schnitzler an Hugo von Hofmannsthal, 21. 12. 1912}
\nopagebreak\mylabel{L02110v}
\rehead{ }\normalsize\beginnumbering\briefempfaengerindex{Hofmannsthal, Hugo von@\textsc{Hofmannsthal, Hugo von}!zzzSchnitzler, Arthur@\emph{von Arthur Schnitzler}!1912-12-211@{21. 12. 1912}|(be}
\toendnotes[C]{\smallbreak\pagebreak[2]}\Standort{FDH, Hs-30885,146.}
\physDesc{Briefkarte, 500 Zeichen
\newline{}Handschrift: schwarze Tinte, deutsche Kurrent}
\buchAbdrucke{\weitereDrucke{Hugo von Hofmannsthal, Arthur Schnitzler: \emph{Briefwechsel}. Frankfurt am Main: \emph{S. Fischer} 1964, S. 271.} }\toendnotes[C]{\smallbreak}
\pstart
           {\pb}\textcolor{gray}{\textbf{Dr. Arthur Schnitzler}}\hfill Wien\oindex{Wien@\textbf{Wien}, \emph{A.ADM2}|pw}, 21. 12. 912\pend
           
\pstart
           \textcolor{gray}{\textbf{Wien XVIII. Sternwartestrasse 71\oindex{Sternwartestrasse 71@\textbf{Sternwartestraße 71}, \emph{Wohngebäude (K.WHS)}|pw}}}\pend
           \vspace{0.5em}
\pstart
           lieber Hugo, eben mit dem V. Th.\oindex{Volkstheater@\textbf{Volkstheater}, \emph{Theater (K.THE)}|pw}{ }\textsc{telephonirt}; ſie haben mit der \textsc{Roland}\pwindex{Roland, Ida 18.02.1881 – 27.03.1951@\textsc{Roland, Ida} (18.02.1881 – 27.03.1951), \emph{Schauspieler/Schauspielerin}|pw} noch nicht abgeschloſſen, ſchienen über die Ausſicht \textsc{Terwin}\pwindex{Terwin, Johanna 18.03.1884 – 04.01.1962@\textsc{Terwin, Johanna} (18.03.1884 – 04.01.1962), \emph{Schauspieler/Schauspielerin}|pw} poſitiv erfreut. Würde rathen, daſs ſich die T.\pwindex{Terwin, Johanna 18.03.1884 – 04.01.1962@\textsc{Terwin, Johanna} (18.03.1884 – 04.01.1962), \emph{Schauspieler/Schauspielerin}|pw} ganz direct mit dem V. Th.\oindex{Volkstheater@\textbf{Volkstheater}, \emph{Theater (K.THE)}|pw} in
               Verbindung ſetzt; u. zw. ſo geſchwind wie möglich. –\pend
           
\pstart
           Mit \textsc{Thimig}\pwindex{Thimig, Hugo 16.06.1854 – 24.09.1944@\textsc{Thimig, Hugo} (16.06.1854 – 24.09.1944), \emph{Theaterleiter/Theaterleiterin, Schauspieler/Schauspielerin}|pw} heute nur zwei Worte {\pb}auf der Probe\pwindex{Maerchen vom Wolf@\emph{Das Märchen vom Wolf}|pwv}; er habe mir einiges intereſſante zu
               ſagen, werde mich nächſtens beſuchen. (Er war auch \label{K_L02110-1v}\edtext{vor ein paar Wochen}{\lemma{\textnormal{\emph{vor ein paar Wochen}}}\Cendnote{\textnormal{Siehe A. S.: \emph{Tagebuch}, 22. 10. 1912.
               }}}\label{K_L02110-1} bei mir) Bei dieſer Gelegenheit gedenke ich den Jederma{\geminationn}\pwindex{Jedermann. Das Spiel vom Sterben des reichen Mannes@\emph{Jedermann. Das Spiel vom Sterben des reichen Mannes}|pw} anzuſchneiden.\pend
           
\pstart
           Herzlichſt Ihr{\\[\baselineskip]}\spacefill\mbox{A.}\pend
           \leftskip=0em{}\selectlanguage{ngerman}\endnumbering\briefempfaengerindex{Hofmannsthal, Hugo von@\textsc{Hofmannsthal, Hugo von}!zzzSchnitzler, Arthur@\emph{von Arthur Schnitzler}!1912-12-211@{21. 12. 1912}|)be}\mylabel{L02110h}  \normalsize

\doendnotes{C}
\bigskip
\vfill

\clearpage

\footnotesize

\lohead{\textsc{register}}

% Definiere theindex-Environment komplett neu ohne reledmac
\makeatletter
\renewenvironment{theindex}{%
  \section*{\indexname}%
  \setlength{\parindent}{0pt}%
  \setlength{\parskip}{0pt plus 0.3pt}%
  \let\item\@idxitem
}{%
  \clearpage
}
\makeatother

\IfFileExists{\jobname-pw.ind}{\input{\jobname-pw.ind}}{}

\end{document}

      