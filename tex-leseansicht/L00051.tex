\input{../tex-inputs/latex-pdf-vorspann}
\begin{center}
            \textcolor{red}{ENTWURF. ENTZIFFERUNG NOCH NICHT KORREKTURGELESEN}
                      \end{center}
            
               \section[Arthur Schnitzler an Richard Beer-Hofmann, 12. 12. 1891]{ Arthur Schnitzler an Richard Beer-Hofmann, 12. 12. 1891}\nopagebreak\mylabel{v}\rehead{ }\begin{ledgroupsized}[t]{13cm}\normalsize\beginnumbering\briefempfaengerindex{Beer-Hofmann, Richard@\textsc{Beer-Hofmann, Richard}!zzzSchnitzler, Arthur@\emph{von Arthur Schnitzler}!1891-12-121@{12. 12. 1891}|(be} \toendnotes[C]{\smallbreak\pagebreak[2]} \Standort{YCGL, MSS 31.}
\physDesc{Postkarte
\newline{}Handschrift: Bleistift, deutsche Kurrent\newline{}Versand: 1) Stempel: »\nobreak{}Wien, 12/12 91, 5.A\nobreak{}«.  2) Stempel: »\nobreak{}Wien 3/2, 12.  12. 91, 6–8 N, Bestellt\nobreak{}«. }\toendnotes[C]{\smallbreak}\pstart{}{\pb}\textsc{Hrn Dr Rich Beer Hofmann}\pend{}\pstart{}\textsc{Wien\oindex{Wien@\textbf{Wien}|pw}}\pend{}\pstart{}\textsc{III. Seidlgasse 30\oindex{Seidlgasse@\textbf{Seidlgasse}|pw}}\pend{}{\bigskip}\pstart
           \noindent{}{\pb}Lieber Richard, \label{K_L00051_1v}\edtext{So{\geminationn}tag}{\lemma{\textnormal{\emph{Sotag}}}\Cendnote{\textnormal{am 13. 12. 1891}}}\label{K_L00051_1h} vor
                  4 bei mir\pend
           \pstart Herzlichſt Ihr \spacefill\mbox{Arth}\pend{}\endnumbering\briefempfaengerindex{Beer-Hofmann, Richard@\textsc{Beer-Hofmann, Richard}!zzzSchnitzler, Arthur@\emph{von Arthur Schnitzler}!1891-12-121@{12. 12. 1891}|)be}\mylabel{h}\end{ledgroupsized}  \newcommand{\dateiname}{L00051}\newcommand{\titel}{Arthur Schnitzler an Richard Beer-Hofmann, 12. 12. 1891}\newcommand{\editorInnen}{Martin Anton Müller und Gerd-Hermann Susen}\input{../tex-inputs/latex-pdf-abspann}
      