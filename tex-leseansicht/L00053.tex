%% latex-korrekturansicht-vorspann.tex
%% Vorspann für die Korrekturansicht.
%% Lädt die gemeinsame Datei latex-vorspann.tex mit gesetztem Schalter.

\newif\ifkorrekturansicht
\korrekturansichttrue

\input{../tex-inputs/latex-vorspann}


\section[Wilhelm Bölsche an Arthur Schnitzler, 15. 12. 1891]{L00053 Wilhelm Bölsche an Arthur Schnitzler, 15. 12. 1891}
\nopagebreak\mylabel{L00053v}
\rehead{ }\normalsize\beginnumbering\briefempfaengerindex{Schnitzler, Arthur@\textsc{Schnitzler, Arthur}!zzzBoelsche, Wilhelm@\emph{von Wilhelm Bölsche}!1891-12-152@{15. 12. 1891}|(be}
\toendnotes[C]{\smallbreak\pagebreak[2]}\Standort{DLA, A:Schnitzler, HS.NZ85.1.2577,3.}
\physDesc{Brief, 1 Blatt, 1 Seite, 407 Zeichen
\newline{}Handschrift: schwarze Tinte, deutsche Kurrent
\newline{}Schnitzler: mit rotem Buntstift nummeriert: »4« }
\buchAbdrucke{\weitereDrucke{Wilhelm Bölsche: \emph{Briefwechsel. Mit Autoren der Freien Bühne}. Berlin: \emph{Weidler} 2010, S. 673.} }\toendnotes[C]{\smallbreak}
\pstart
           \centering{}{\pb}\textcolor{gray}{\textbf{\textsc{Freie Bühne\pwindex{Freie Buehne fuer modernes Leben@\emph{Freie Bühne für modernes Leben}|pw}}}}\pend
           
\pstart
           \centering{}\textcolor{gray}{\textbf{\textsc{für modernes Leben.}}}\pend
           
\pstart
           \centering{}\textcolor{gray}{\textbf{\textsc{Herausgegeben von \textbf{Otto Brahm}\pwindex{Brahm, Otto 05.02.1856 – 28.11.1912@\textsc{Brahm, Otto} (05.02.1856 – 28.11.1912), \emph{Theaterleiter/Theaterleiterin, Regisseur/Regisseurin}|pw}.}}}\pend
           
\pstart
           \textcolor{gray}{\textbf{Verlag und Expedition: S. Fischer\orgindex{S. Fischer Verlag@S. Fischer Verlag|pw}.}}\pend
           
\pstart
           \textcolor{gray}{\textbf{Sprechstunden: Mittwoch und Freitag 12–2 Uhr.}}\pend
           
\pstart
           \textcolor{gray}{\textbf{Alle für die Redaction bestimmten Sendungen (Beiträge,
                     Recensions-Exempl.) bitten wir \textbf{ohne Angabe eines
                        Personennamens} an die Redaction der Wochenschrift »\so{Freie Bühne}\pwindex{Freie Buehne fuer modernes Leben@\emph{Freie Bühne für modernes Leben}|pw}« Berlin W. Link-Strasse 25\oindex{Linkstrasse@\textbf{Linkstraße}, \emph{Straße (K.STR)}|pw} zu
                     addressiren.}}\pend
           
\pstart
           \textcolor{gray}{\textbf{Wir ersuchen unsere geehrten Mitarbeiter, jedes Manuscript
                     auf der ersten Seite mit ihrer genauen Adresse zu versehen.}}\pend
           
\pstart
           \raggedleft{}\introOben{}Friedrichshagen\oindex{Friedrichshagen@\textbf{Friedrichshagen}, \emph{P.PPLX}|pw} bei\introOben{}{ }\textcolor{gray}{\textbf{\textsc{Berlin\oindex{Berlin@\textbf{Berlin}, \emph{P.PPLC}|pw}}, den}}{ }15. XII. \textcolor{gray}{\textbf{189}}1.\pend
           
\pstart
           \raggedleft{}\textcolor{gray}{\textbf{\strikeout{W. Link-Straße 25}\oindex{Linkstrasse@\textbf{Linkstraße}, \emph{Straße (K.STR)}|pw}.}}\pend
           
\pstart
           \raggedleft{}Wilhelmſtr. 72\oindex{Peter-Hille-Strasse@\textbf{Peter-Hille-Straße}, \emph{Straße (K.STR)}|pw}.\pend
           
\pstart{}Hochgeehrter Herr Doktor!\pend\vspace{0.5em}
\pstart
           Vom 1. Jan. ab wird die Freie Bühne\pwindex{Freie Buehne fuer den Entwickelungskampf der Zeit@\emph{Freie Bühne für den Entwickelungskampf der Zeit}|pw}{ }\label{K_L00053-1v}\edtext{Monatsſchrift}{\lemma{\textnormal{\emph{Monatsſchrift}}}\Cendnote{\textnormal{In den Jahren 1890 und 1891 war die
                     \emph{Freie Bühne}\pwindex{Freie Buehne fuer den Entwickelungskampf der Zeit@\emph{Freie Bühne für den Entwickelungskampf der Zeit}|pwk} wöchentlich erschienen.}}}\label{K_L00053-1} unter \uline{meiner ausſchließlichen} Leitung. Ich freue mich, daß
               Ihre Novelle\pwindex{Sohn. Aus den Papieren eines Arztes@\emph{Der Sohn. Aus den Papieren eines Arztes}|pwv}, ſo lange zum
               Warten verurteilt, nun an gewichtiger Stelle grade das neue Quartal im erſten
               Monatsheft eröffnen kann. Und ich füge die Bitte bei um freundliche weitere
               Teilnahme.\pend
           
\pstart
           Mit vorzüglicher Hochachtung{\\[\baselineskip]}\spacefill\mbox{Wilhelm Bölsche.}\pend
           \leftskip=0em{}\selectlanguage{ngerman}\endnumbering\briefempfaengerindex{Schnitzler, Arthur@\textsc{Schnitzler, Arthur}!zzzBoelsche, Wilhelm@\emph{von Wilhelm Bölsche}!1891-12-152@{15. 12. 1891}|)be}\mylabel{L00053h}  \normalsize

\doendnotes{C}
\bigskip
\vfill

\clearpage

\footnotesize

\lohead{\textsc{register}}

% Definiere theindex-Environment komplett neu ohne reledmac
\makeatletter
\renewenvironment{theindex}{%
  \section*{\indexname}%
  \setlength{\parindent}{0pt}%
  \setlength{\parskip}{0pt plus 0.3pt}%
  \let\item\@idxitem
}{%
  \clearpage
}
\makeatother

\IfFileExists{\jobname-pw.ind}{\input{\jobname-pw.ind}}{}

\end{document}

      