%% latex-leseansicht-vorspann.tex
%% Vorspann für die Leseansicht.
%% Lädt die gemeinsame Datei latex-vorspann.tex mit nicht gesetztem Schalter.

\newif\ifkorrekturansicht
\korrekturansichtfalse

\input{../tex-inputs/latex-vorspann}


\section[Wilhelm Bölsche an Arthur Schnitzler, 15. 12. 1891]{L00053 Wilhelm Bölsche an Arthur Schnitzler, 15. 12. 1891}
\nopagebreak\mylabel{L00053v}
\rehead{ }\normalsize\beginnumbering\briefempfaengerindex{Schnitzler, Arthur@\textsc{Schnitzler, Arthur}!zzzBölsche, Wilhelm@\emph{von Wilhelm Bölsche}!1891-12-152@{15. 12. 1891}|(be}
\toendnotes[C]{\smallbreak\pagebreak[2]}
\correspDesc{Versand  durch Wilhelm Bölsche am 15. 12. 1891 in Berlin
\newline{}Erhalt  durch Arthur Schnitzler im Zeitraum [16. 12. 1891 – 20. 12. 1891?] in Wien}\toendnotes[C]{\smallbreak}
\Standort{DLA, A:Schnitzler, HS.NZ85.1.2577,3.}
\physDesc{Brief, 1 Blatt, 1 Seite, 407 Zeichen
\newline{}Handschrift: schwarze Tinte, deutsche Kurrent
\newline{}Schnitzler: mit rotem Buntstift nummeriert: »4« }
\buchAbdrucke{\weitereDrucke{Wilhelm Bölsche: \emph{Briefwechsel. Mit Autoren der Freien Bühne}. Herausgegeben von Gerd-Hermann Susen. Berlin: \emph{Weidler} 2010, S. 673 (Werke und Briefe. Wissenschaftliche Ausgabe, Briefe I).} }\toendnotes[C]{\smallbreak}
\pstart
           \centering{}{\pb}\textcolor{gray}{\textbf{\textsc{Freie Bühne\pwindex{Freie Bühne für modernes Leben@\emph{Freie Bühne für modernes Leben}|pw}}}}\pend
           
\pstart
           \centering{}\textcolor{gray}{\textbf{\textsc{für modernes Leben.}}}\pend
           
\pstart
           \centering{}\textcolor{gray}{\textbf{\textsc{Herausgegeben von \textbf{Otto Brahm}\pwindex{Brahm, Otto 5.\,2.\,1856 Hamburg – 28.\,11.\,1912 Berlin@\textsc{Brahm, Otto} (5.\,2.\,1856 Hamburg – 28.\,11.\,1912 Berlin), \emph{Theaterleiter, Regisseur}|pw}.}}}\pend
           
\pstart
           \textcolor{gray}{\textbf{Verlag und Expedition: S. Fischer\orgindex{S. Fischer Verlag@S. Fischer Verlag|pw}.}}\pend
           
\pstart
           \textcolor{gray}{\textbf{Sprechstunden: Mittwoch und Freitag 12–2 Uhr.}}\pend
           
\pstart
           \textcolor{gray}{\textbf{Alle für die Redaction bestimmten Sendungen (Beiträge,
                     Recensions-Exempl.) bitten wir \textbf{ohne Angabe eines
                        Personennamens} an die Redaction der Wochenschrift »\so{Freie Bühne}\pwindex{Freie Bühne für modernes Leben@\emph{Freie Bühne für modernes Leben}|pw}« Berlin W. Link-Strasse 25\oindex{Linkstraße@\textbf{Linkstraße}, \emph{Straße}|pw} zu
                     addressiren.}}\pend
           
\pstart
           \textcolor{gray}{\textbf{Wir ersuchen unsere geehrten Mitarbeiter, jedes Manuscript
                     auf der ersten Seite mit ihrer genauen Adresse zu versehen.}}\pend
           
\pstart
           \raggedleft{}\introOben{}Friedrichshagen\oindex{Friedrichshagen@\textbf{Friedrichshagen}, \emph{Ehemaliger Ort}|pw} bei\introOben{}{ }\textcolor{gray}{\textbf{\textsc{Berlin\oindex{Berlin@\textbf{Berlin}, \emph{Hauptstadt}|pw}}, den}}{ }15. XII. \textcolor{gray}{\textbf{189}}1.\pend
           
\pstart
           \raggedleft{}\textcolor{gray}{\textbf{\strikeout{W. Link-Straße 25}\oindex{Linkstraße@\textbf{Linkstraße}, \emph{Straße}|pw}.}}\pend
           
\pstart
           \raggedleft{}Wilhelmſtr. 72\oindex{Peter-Hille-Straße@\textbf{Peter-Hille-Straße}, \emph{Straße}|pw}.\pend
           
\pstart{}Hochgeehrter Herr Doktor!\pend\vspace{0.5em}
\pstart
           Vom 1. Jan. ab wird die Freie Bühne\pwindex{Freie Bühne für den Entwickelungskampf der Zeit@\emph{Freie Bühne für den Entwickelungskampf der Zeit}|pw}{ }\label{K_L00053-1v}\edtext{Monatsſchrift}{\lemma{\textnormal{\emph{Monatsschrift}}}\Cendnote{\textnormal{In den Jahren 1890 und 1891 war die
                     \emph{Freie Bühne}\pwindex{Freie Bühne für den Entwickelungskampf der Zeit@\emph{Freie Bühne für den Entwickelungskampf der Zeit}|pwk} wöchentlich erschienen.}}}\label{K_L00053-1} unter \uline{meiner ausſchließlichen} Leitung. Ich freue mich, daß
               Ihre Novelle\pwindex{Schnitzler, Arthur 15.\,5.\,1862 Wien – 21.\,10.\,1931 ebd.@\textsc{Schnitzler, Arthur} (15.\,5.\,1862 Wien – 21.\,10.\,1931 ebd.), \emph{Schriftsteller, Mediziner}!Sohn. Aus den Papieren eines Arztes@\strich\emph{Der Sohn. Aus den Papieren eines Arztes}|pwv},{ }ſo lange zum
               Warten verurteilt, nun an gewichtiger Stelle grade das neue Quartal im erſten
               Monatsheft eröffnen kann. Und ich füge die Bitte bei um freundliche weitere
               Teilnahme.\pend
           
\pstart
           Mit vorzüglicher Hochachtung{\\[\baselineskip]}\spacefill\mbox{Wilhelm Bölsche.}\pend
           \leftskip=0em{}\selectlanguage{ngerman}\endnumbering\briefempfaengerindex{Schnitzler, Arthur@\textsc{Schnitzler, Arthur}!zzzBölsche, Wilhelm@\emph{von Wilhelm Bölsche}!1891-12-152@{15. 12. 1891}|)be}\mylabel{L00053h}  \newcommand{\dateiname}{L00053}\newcommand{\titel}{Wilhelm Bölsche an Arthur Schnitzler, 15. 12. 1891}\newcommand{\editorInnen}{Martin Anton Müller und Gerd-Hermann Susen}%% latex-leseansicht-abspann.tex
%% Abspann für die Leseansicht.
%% Der Schalter \ifkorrekturansicht ist bereits durch den Vorspann gesetzt.

%% latex-abspann.tex
%% Gemeinsamer Abspann für Korrekturansicht und Leseansicht.
%% Setzt den Schalter \ifkorrekturansicht voraus (gesetzt in den
%% einbindenden Dateien latex-korrekturansicht-abspann.tex bzw.
%% latex-leseansicht-abspann.tex).
%% ---------------------------------------------------------------

\normalsize

% Das esempio-Environment wird nur in der Leseansicht benötigt
\ifkorrekturansicht\else
\newenvironment{esempio}[3]%
{
    \vspace{1.5ex}
    \rlap{\underline{#1}}
    \par
    \setlength{\parindent}{0cm}
    \nopagebreak
    \leftskip=#2cm
    \rightskip=#3cm
}
{
    \par
}
\fi

\doendnotes{C}
\bigskip
\vfill

\clearpage

\footnotesize

\ifkorrekturansicht
  \lohead{\textsc{register}}
\fi

% theindex-Environment neu definieren ohne reledmac
\makeatletter
\renewenvironment{theindex}{%
  \ifkorrekturansicht
    \section*{\indexname}%
  \else
    \subsubsection*{Index der erwähnten Entitäten}%
  \fi
  \setlength{\parindent}{0pt}%
  \setlength{\parskip}{0pt plus 0.3pt}%
  \let\item\@idxitem
}{%
  \ifkorrekturansicht\clearpage\fi
}
\makeatother

\IfFileExists{\jobname-pw.ind}{\input{\jobname-pw.ind}}{}

% Quellenangabe nur in der Leseansicht
\ifkorrekturansicht\else
% Fallback-Definitionen, falls die .tex-Datei \titel etc. nicht gesetzt hat
\providecommand{\titel}{}
\providecommand{\editorInnen}{}
\providecommand{\dateiname}{\jobname}

\vspace{3cm}

\vfill

\footnotesize
\textsc{Quelle}: \titel. Herausgegeben von {\editorInnen}. In: \emph{Arthur Schnitzler: Briefwechsel mit Autorinnen und Autoren}.
 Digitale Edition, https://schnitzler-briefe.acdh.oeaw.ac.at/{\dateiname}.html (Stand \today)
\fi

\end{document}


