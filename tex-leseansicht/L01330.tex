%% latex-korrekturansicht-vorspann.tex
%% Vorspann für die Korrekturansicht.
%% Lädt die gemeinsame Datei latex-vorspann.tex mit gesetztem Schalter.

\newif\ifkorrekturansicht
\korrekturansichttrue

\input{../tex-inputs/latex-vorspann}


\section[Hugo von Hofmannsthal an Arthur Schnitzler, {[}zwischen 14. und 23. 12. 1903?{]}]{L01330 Hugo von Hofmannsthal an Arthur Schnitzler, {[}zwischen 14. und
               23. 12. 1903?{]}}
\nopagebreak\mylabel{L01330v}
\rehead{ }\normalsize\beginnumbering\briefempfaengerindex{Schnitzler, Arthur@\textsc{Schnitzler, Arthur}!zzzHofmannsthal, Hugo von@\emph{von Hugo von Hofmannsthal}!1903-12-231@{{[}zwischen 14. und
                  23. 12. 1903?{]}}|(be}
\toendnotes[C]{\smallbreak\pagebreak[2]}\buchAlsQuelle{Hugo von Hofmannsthal, Arthur Schnitzler: \emph{Briefwechsel}. Frankfurt am Main: \emph{S. Fischer} 1964, S. 178.}\toendnotes[C]{\smallbreak}
\pstart
           \noindent{}{\pb}Haben Sie \label{K_L01330-1v}\edtext{in Brünn\oindex{Bruenn@\textbf{Brünn}, \emph{P.PPLA}|pw} gelesen}{\lemma{\textnormal{\emph{in Brünn gelesen}}}\Cendnote{\textnormal{Schnitzler hatte am 19. 10. 1903 für die \emph{Neue akademische Vereinigung}\orgindex{Neue akademische Vereinigung@Neue akademische Vereinigung|pwk} im kleinen Festsaal des Deutschen Hauses\oindex{Deutsches Haus [Bruenn]@\textbf{Deutsches Haus [Brünn]}, \emph{Theater (K.THE)}|pwk} gelesen. Das nicht überlieferte
                  Original des Telegramms ist im Erstdruck auf »Ende 1903« 
                  datiert, was auf Schnitzler zurückgehen dürfte. Hofmannsthals\pwindex{Hofmannsthal, Hugo von 1874-02-01 – 1929-07-15@\textsc{Hofmannsthal, Hugo von} (1874-02-01 – 1929-07-15), \emph{Schriftsteller/Schriftstellerin}|pwk}
                  Karte vom 24. [12?.] 1903
                     nimmt auf die (nicht überlieferte) Antwort auf das Telegramm Bezug und ist dementsprechend danach zu datieren. Die Form des Telegramms impliziert
                     eine gewissen Dringlichkeit, wodurch nur der Zeitraum unmittelbar vor der Karte in Betracht
                     kommt. Schnitzlers Brief vom 10. 12. 1903 thematisiert eine längere Funkstille und
                  sie vereinbaren in Folge ein Treffen für den 14. 12. 1903. Wenngleich auch frühere Tage denkbar sind, dürfte diese
                  Kommunikation erst nach diesem Treffen gelaufen sein.}}}\label{K_L01330-1}\hspace*{1em}Wieviel Minuten\hspace*{1em}Welches Honorar\hspace*{1em}Ist es angenehmes Lokal\pend
           \pstart \spacefill\mbox{Hugo}\pend{}\selectlanguage{ngerman}\endnumbering\briefempfaengerindex{Schnitzler, Arthur@\textsc{Schnitzler, Arthur}!zzzHofmannsthal, Hugo von@\emph{von Hugo von Hofmannsthal}!1903-12-141@{{[}zwischen 14. und
                  23. 12. 1903?{]}}|)be}\mylabel{L01330h}  \normalsize

\doendnotes{C}
\bigskip
\vfill

\clearpage

\footnotesize

\lohead{\textsc{register}}

% Definiere theindex-Environment komplett neu ohne reledmac
\makeatletter
\renewenvironment{theindex}{%
  \section*{\indexname}%
  \setlength{\parindent}{0pt}%
  \setlength{\parskip}{0pt plus 0.3pt}%
  \let\item\@idxitem
}{%
  \clearpage
}
\makeatother

\IfFileExists{\jobname-pw.ind}{\input{\jobname-pw.ind}}{}

\end{document}

      