%% latex-leseansicht-vorspann.tex
%% Vorspann für die Leseansicht.
%% Lädt die gemeinsame Datei latex-vorspann.tex mit nicht gesetztem Schalter.

\newif\ifkorrekturansicht
\korrekturansichtfalse

\input{../tex-inputs/latex-vorspann}


\section[Hugo von Hofmannsthal an Arthur Schnitzler, {[}zwischen 14. und 23. 12. 1903?{]}]{L01330 Hugo von Hofmannsthal an Arthur Schnitzler, {[}zwischen 14. und 23. 12. 1903?{]}}
\nopagebreak\mylabel{L01330v}
\rehead{ }\normalsize\beginnumbering\briefempfaengerindex{Schnitzler, Arthur@\textsc{Schnitzler, Arthur}!zzzHofmannsthal, Hugo von@\emph{von Hugo von Hofmannsthal}!1903-12-231@{{[}zwischen 14. und 23. 12. 1903?{]}}|(be}
\toendnotes[C]{\smallbreak\pagebreak[2]}
\correspDesc{Versand  durch Hugo von Hofmannsthal im Zeitraum [zwischen 14. und
                  23. 12. 1903?] in Rodaun
\newline{}Erhalt  durch Arthur Schnitzler im Zeitraum [zwischen
                  14. und 23. 12. 1903?] in Wien}\toendnotes[C]{\smallbreak}
\buchAlsQuelle{Hugo von Hofmannsthal, Arthur Schnitzler: \emph{Briefwechsel}. Herausgegeben von Therese Nickl und Heinrich Schnitzler. Frankfurt am Main: \emph{S. Fischer} 1964, S. 178.}\toendnotes[C]{\smallbreak}
\pstart
           \noindent{}{\pb}Haben Sie \label{K_L01330-1v}\edtext{in Brünn\oindex{Brünn@\textbf{Brünn}|pw} gelesen}{\lemma{\textnormal{\emph{in Brünn gelesen}}}\Cendnote{\textnormal{Schnitzler hatte am 19. 10. 1903 für die \emph{Neue akademische Vereinigung}\orgindex{Neue akademische Vereinigung@Neue akademische Vereinigung|pwk} im kleinen Festsaal des Deutschen Hauses\oindex{Deutsches Haus [Brünn]@\textbf{Deutsches Haus [Brünn]}, \emph{Theater}|pwk} gelesen. Das nicht überlieferte
                  Original des Telegramms ist im Erstdruck auf »Ende 1903« 
                  datiert, was auf Schnitzler zurückgehen dürfte. Hofmannsthals\pwindex{Hofmannsthal, Hugo von 1.\,2.\,1874 Wien – 15.\,7.\,1929 Rodaun@\textsc{Hofmannsthal, Hugo von} (1.\,2.\,1874 Wien – 15.\,7.\,1929 Rodaun), \emph{Schriftsteller}|pwk}
                  Karte vom XXXX Auszeichnungsfehler: Dokument L01331 nicht gefunden
                     nimmt auf die (nicht überlieferte) Antwort auf das Telegramm Bezug und ist dementsprechend danach zu datieren. Die Form des Telegramms impliziert
                     eine gewissen Dringlichkeit, wodurch nur der Zeitraum unmittelbar vor der Karte in Betracht
                     kommt. Schnitzlers Brief vom XXXX Auszeichnungsfehler: Dokument L01348 nicht gefunden thematisiert eine längere Funkstille und
                  sie vereinbaren in Folge ein Treffen für den 14. 12. 1903. Wenngleich auch frühere Tage denkbar sind, dürfte diese
                  Kommunikation erst nach diesem Treffen gelaufen sein.}}}\label{K_L01330-1}\hspace*{1em}Wieviel Minuten\hspace*{1em}Welches Honorar\hspace*{1em}Ist es angenehmes Lokal\pend
           \pstart \spacefill\mbox{Hugo}\pend{}\selectlanguage{ngerman}\endnumbering\briefempfaengerindex{Schnitzler, Arthur@\textsc{Schnitzler, Arthur}!zzzHofmannsthal, Hugo von@\emph{von Hugo von Hofmannsthal}!1903-12-141@{{[}zwischen 14. und 23. 12. 1903?{]}}|)be}\mylabel{L01330h}  \newcommand{\dateiname}{L01330}\newcommand{\titel}{Hugo von Hofmannsthal an Arthur Schnitzler, [zwischen 14. und 23. 12. 1903?]}\newcommand{\editorInnen}{Martin Anton Müller und Gerd-Hermann Susen}%% latex-leseansicht-abspann.tex
%% Abspann für die Leseansicht.
%% Der Schalter \ifkorrekturansicht ist bereits durch den Vorspann gesetzt.

%% latex-abspann.tex
%% Gemeinsamer Abspann für Korrekturansicht und Leseansicht.
%% Setzt den Schalter \ifkorrekturansicht voraus (gesetzt in den
%% einbindenden Dateien latex-korrekturansicht-abspann.tex bzw.
%% latex-leseansicht-abspann.tex).
%% ---------------------------------------------------------------

\normalsize

% Das esempio-Environment wird nur in der Leseansicht benötigt
\ifkorrekturansicht\else
\newenvironment{esempio}[3]%
{
    \vspace{1.5ex}
    \rlap{\underline{#1}}
    \par
    \setlength{\parindent}{0cm}
    \nopagebreak
    \leftskip=#2cm
    \rightskip=#3cm
}
{
    \par
}
\fi

\doendnotes{C}
\bigskip
\vfill

\clearpage

\footnotesize

\ifkorrekturansicht
  \lohead{\textsc{register}}
\fi

% theindex-Environment neu definieren ohne reledmac
\makeatletter
\renewenvironment{theindex}{%
  \ifkorrekturansicht
    \section*{\indexname}%
  \else
    \subsubsection*{Index der erwähnten Entitäten}%
  \fi
  \setlength{\parindent}{0pt}%
  \setlength{\parskip}{0pt plus 0.3pt}%
  \let\item\@idxitem
}{%
  \ifkorrekturansicht\clearpage\fi
}
\makeatother

\IfFileExists{\jobname-pw.ind}{\input{\jobname-pw.ind}}{}

% Quellenangabe nur in der Leseansicht
\ifkorrekturansicht\else
% Fallback-Definitionen, falls die .tex-Datei \titel etc. nicht gesetzt hat
\providecommand{\titel}{}
\providecommand{\editorInnen}{}
\providecommand{\dateiname}{\jobname}

\vspace{3cm}

\vfill

\footnotesize
\textsc{Quelle}: \titel. Herausgegeben von {\editorInnen}. In: \emph{Arthur Schnitzler: Briefwechsel mit Autorinnen und Autoren}.
 Digitale Edition, https://schnitzler-briefe.acdh.oeaw.ac.at/{\dateiname}.html (Stand \today)
\fi

\end{document}


