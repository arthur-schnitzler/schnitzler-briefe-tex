%% latex-korrekturansicht-vorspann.tex
%% Vorspann für die Korrekturansicht.
%% Lädt die gemeinsame Datei latex-vorspann.tex mit gesetztem Schalter.

\newif\ifkorrekturansicht
\korrekturansichttrue

\input{../tex-inputs/latex-vorspann}


\section[Arthur Schnitzler an Stefan Zweig, 7. 10. 1910]{L03795 Arthur Schnitzler an Stefan Zweig, 7. 10. 1910}
\nopagebreak\mylabel{L03795v}
\rehead{ }\normalsize\beginnumbering\briefempfaengerindex{Zweig, Stefan@\textsc{Zweig, Stefan}!zzzSchnitzler, Arthur@\emph{von Arthur Schnitzler}!1910-10-071@{7. 10. 1910}|(be}
\toendnotes[C]{\smallbreak\pagebreak[2]}\Standort{Jerusalem, National Library of Israel, ARC. Ms. Var. 305 1 58 Stefan Zweig Collection.}
\physDesc{Brief, 1 Blatt, 1 Seite, 192 Zeichen
\newline{}Handschrift: schwarze Tinte, deutsche Kurrent}\toendnotes[C]{\smallbreak}
\pstart
           {\pb}\textcolor{gray}{\textbf{Dr. Arthur Schnitzler}}\hfill 7. X. 910\pend
           
\pstart
           \textcolor{gray}{\textbf{Wien XVIII.\oindex{XVIII., Waehring@\textbf{XVIII., Währing}, \emph{A.ADM3}|pw}}}{ }\substVorne{}\textsuperscript{\textcolor{gray}{\textbf{Spoettelgasse 7}}}\substDazwischen{}Sternwartestr 71\oindex{Sternwartestrasse 71@\textbf{Sternwartestraße 71}, \emph{Wohngebäude (K.WHS)}|pw}\substHinten{}\pend
           
\pstart{}lieber Herr Doctor,\pend\vspace{0.5em}
\pstart
           am \label{K_L03795-1v}\edtext{Dinſtag{ }Abend\eventindex{Sternwartestrasse 71@\textbf{Sternwartestraße 71}!Private Lesung von Die ungleichen Schalen, 11.10.1910@Private Lesung von Die ungleichen Schalen, 11.10.1910|pwv}}{\lemma{\textnormal{\emph{Dinſtag Abend}}}\Cendnote{\textnormal{Siehe A. S.: \emph{Kulturveranstaltungen}, 11. 10. 1910. }}}\label{K_L03795-1}, \introOben{}vor\introOben{} 7 Uhr
               will uns \textsc{Wasserma{\geminationn}}\pwindex{Wassermann, Jakob 10.03.1873 – 01.01.1934@\textsc{Wassermann, Jakob} (10.03.1873 – 01.01.1934), \emph{Schriftsteller/Schriftstellerin}|pw} ſeine neuen Einakter\pwindex{ungleichen Schalen@\emph{Die ungleichen Schalen}|pwv}
               vorleſen; ich bitte Sie, in ſeinem, und unſerm Namen herzlichſt dabei zu ſein.\pend
           
\pstart
           Ihr{\\[\baselineskip]}\spacefill\mbox{A. S.}\pend
           \leftskip=0em{}\selectlanguage{ngerman}\endnumbering\briefempfaengerindex{Zweig, Stefan@\textsc{Zweig, Stefan}!zzzSchnitzler, Arthur@\emph{von Arthur Schnitzler}!1910-10-071@{7. 10. 1910}|)be}\mylabel{L03795h}  \normalsize

\doendnotes{C}
\bigskip
\vfill

\clearpage

\footnotesize

\lohead{\textsc{register}}

% Definiere theindex-Environment komplett neu ohne reledmac
\makeatletter
\renewenvironment{theindex}{%
  \section*{\indexname}%
  \setlength{\parindent}{0pt}%
  \setlength{\parskip}{0pt plus 0.3pt}%
  \let\item\@idxitem
}{%
  \clearpage
}
\makeatother

\IfFileExists{\jobname-pw.ind}{\input{\jobname-pw.ind}}{}

\end{document}

      