%% latex-korrekturansicht-vorspann.tex
%% Vorspann für die Korrekturansicht.
%% Lädt die gemeinsame Datei latex-vorspann.tex mit gesetztem Schalter.

\newif\ifkorrekturansicht
\korrekturansichttrue

\input{../tex-inputs/latex-vorspann}


\section[Felix Salten an Arthur Schnitzler, 6. 7. 1893]{L03123 Felix Salten an Arthur Schnitzler, 6. 7. 1893}
\nopagebreak\mylabel{L03123v}
\rehead{ }\normalsize\beginnumbering\briefempfaengerindex{Schnitzler, Arthur@\textsc{Schnitzler, Arthur}!zzzSalten, Felix@\emph{von Felix Salten}!1893-07-061@{6. 7. 1893}|(be}
\toendnotes[C]{\smallbreak\pagebreak[2]}\Standort{CUL, Schnitzler, B 89, A 1.}
\physDesc{Brief, 1 Blatt, 4 Seiten, 2615 Zeichen
\newline{}Handschrift: schwarze Tinte, lateinische Kurrent
\newline{}Ordnung: mit Bleistift von unbekannter Hand nummeriert: »26« }\toendnotes[C]{\smallbreak}
\pstart
           \raggedleft{}{\pb}Wien\oindex{Wien@\textbf{Wien}, \emph{A.ADM2}|pw}, 6. Juli 93.\pend
           \vspace{0.5em}
\pstart
           Lieber Arthur! Ebenso leer, ebenso verstimmt und
               verärgert wie Sie, bin auch ich die ganze Zeit über, und es ist nur der Unterschied,
               dass Sie \label{K_L03123-1v}\edtext{in Ischl\oindex{Bad Ischl@\textbf{Bad Ischl}, \emph{P.PPL}|pw}}{\lemma{\textnormal{\emph{in Ischl}}}\Cendnote{\textnormal{Schnitzler war zwischen 2. 7. 1893 und 15. 7. 1893 in Ischl\oindex{Bad Ischl@\textbf{Bad Ischl}, \emph{P.PPL}|pwk}, von wo aus er mehrere Radausflüge
                  unternahm.}}}\label{K_L03123-1} sind und Bicycle fahren können, während ich in Wien\oindex{Wien@\textbf{Wien}, \emph{A.ADM2}|pw} braten muss und im eckelhaften Bureau\orgindex{»Phoenix« Versicherung@»Phönix« Versicherung|pwv} arbeiten, das ich gerne bald ganz
               verlaßen möchte. Es war auch eine verfehlte Sache, dass ich mich hier einsperren und
               mir einreden ließ, ich hätte Beruf zum Beamten einer \label{K_L03123-2v}\edtext{Assecuranz\orgindex{»Phoenix« Versicherung@»Phönix« Versicherung|pwv}}{\lemma{\textnormal{\emph{Assecuranz}}}\Cendnote{\textnormal{Salten\pwindex{Salten, Felix 06.09.1869 – 08.10.1945@\textsc{Salten, Felix} (06.09.1869 – 08.10.1945), \emph{Schriftsteller/Schriftstellerin, Journalist/Journalistin, Chefredakteur/Chefredakteurin}|pwk} arbeitete für die
                  Versicherungsgesellschaft \emph{Phönix}\orgindex{»Phoenix« Versicherung@»Phönix« Versicherung|pwk}.}}}\label{K_L03123-2}, so
               plötzlich. Und ich glaube noch immer, dass es gehen müsste, sich mit
               schriftstellerischer {\pb}Arbeit 50 fl
               per Monat zu verdienen. Dass ich es bisher nicht gethan, beweist wenig genug, denn
               ich war faul und habe Nichts gearbeitet. Von morgen
               ab, bin ich ganz allein, und sind Sie mir bisher schon sehr abgegangen, so werden Sie
               es dann noch mehr.\pend
           
\pstart
           Es wäre jedenfalls nicht schlecht und würde mich freuen, wenn diese \label{K_L03123-3v}\edtext{Aufführung\pwindex{Abschiedssouper@\emph{Abschiedssouper}|pwv}}{\lemma{\textnormal{\emph{Aufführung}}}\Cendnote{\textnormal{Siehe Arthur Schnitzler an Felix Salten, 5. 7. 1893.
               }}}\label{K_L03123-3} zu
               Stande käme; was Wild\pwindex{Wild, Ignaz 13.05.1849 – 19.10.1909@\textsc{Wild, Ignaz} (13.05.1849 – 19.10.1909), \emph{Theaterleiter/Theaterleiterin, Schauspieler/Schauspielerin}|pw} für Gründe hat, ist ja ziemlich egal, für
               Sie wäre es von Nutzen. Verständigen Sie dann auch Paul Horn\pwindex{Horn, Paul 13.02.1867 – 18.01.1936@\textsc{Horn, Paul} (13.02.1867 – 18.01.1936), \emph{Fabrikant/Fabrikantin}|pw}. Er ist in Aussee\oindex{Bad Aussee@\textbf{Bad Aussee}, \emph{P.PPLA3}|pw} und Specht\pwindex{Specht, Richard 07.12.1870 – 18.03.1932@\textsc{Specht, Richard} (07.12.1870 – 18.03.1932), \emph{Schriftsteller/Schriftstellerin, Journalist/Journalistin, Kritiker/Kritikerin}|pw} von Samstag an bei ihm.\pend
           
\pstart
           Das Buch\pwindex{Decadence. Novelletten@\emph{Decadence. Novelletten}|pwv} vom kleinen Rosner\pwindex{Rosner, Karl Peter 05.02.1873 – 06.05.1951@\textsc{Rosner, Karl Peter} (05.02.1873 – 06.05.1951), \emph{Schriftsteller/Schriftstellerin}|pw} ist erschienen, und heisst \label{K_L03123-4v}\edtext{»\uline{Decadence\pwindex{Decadence. Novelletten@\emph{Decadence. Novelletten}|pw}}«}{\lemma{\textnormal{\emph{»Decadence«}}}\Cendnote{\textnormal{Karl Rosner\pwindex{Rosner, Karl Peter 05.02.1873 – 06.05.1951@\textsc{Rosner, Karl Peter} (05.02.1873 – 06.05.1951), \emph{Schriftsteller/Schriftstellerin}|pwk}: \emph{Decadence. Novelletten}\pwindex{Decadence. Novelletten@\emph{Decadence. Novelletten}|pwk}. Leipzig\oindex{Leipzig@\textbf{Leipzig}, \emph{P.PPLA3}|pwk}: \emph{W. Friedrich}\orgindex{Verlag Wilhelm Friedrich@Verlag Wilhelm Friedrich|pwk}{ }1893.}}}\label{K_L03123-4}. Es ist ganz so, wie die Novelle\pwindex{Strassenliebe@\emph{Straßenliebe}|pwv}, die {\pb}wir \label{K_L03123-5v}\edtext{voriges Jahr}{\lemma{\textnormal{\emph{voriges Jahr}}}\Cendnote{\textnormal{Siehe A. S.: \emph{Tagebuch}, 29. 7. 1892.
               }}}\label{K_L03123-5} auf der Rohrerhütte\oindex{Rohrerhuette@\textbf{Rohrerhütte}, \emph{Gastgewerbegebäude (K.GGW)}|pw} von ihm gehört. Wenn
               diese jungen Sachen prätentiös und aufdringlich im Druck vorliegen, dann sieht man
               erst recht, wie dumm und zuwider diese ganze \label{K_L03123-6v}\edtext{Psychopathia-Sexualis\pwindex{Psychopathia sexualis@\emph{Psychopathia sexualis}|pwv}}{\lemma{\textnormal{\emph{Psychopathia-Sexualis}}}\Cendnote{\textnormal{Richard von Krafft-Ebing\pwindex{Krafft-Ebing, Richard von 14.08.1840 – 22.12.1902@\textsc{Krafft-Ebing, Richard von} (14.08.1840 – 22.12.1902), \emph{Sexualforscher/Sexualforscherin, Psychiater/Psychiaterin}|pwk}: \emph{Psychopathia sexualis. Eine klinisch-forensische
                        Studie}\pwindex{Psychopathia sexualis@\emph{Psychopathia sexualis}|pwk}. Stuttgart: \emph{Ferdinand
                        Enke}{ }1886. 1893 erschien bereits die »Achte, verbesserte und theilweise
                  vermehrte Auflage« des populären Standardwerks der Sexualwissenschaft.}}}\label{K_L03123-6}-pose
               ist, und wie recht die Leute haben, wenn sie auf diese Pubertäts-Geilheiten
               schimpfen.\pend
           
\pstart
           Dass Sie nicht arbeiten, hat, wie ich meine, nicht viel zu bedeuten, ich glaube fest a\substVorne{}\textsuperscript{\textcolor{gray}{m}}\substDazwischen{}n\substHinten{} eine starke Arbeitsperiode von Ihnen für die nächste Zeit – von mir glaube
               ich noch immer dasselbe.\pend
           
\pstart
           Was für ein Leben, sag ich Ihnen! Von 9 bis 5 Uhr oder 6. im Bureau\orgindex{»Phoenix« Versicherung@»Phönix« Versicherung|pwv}, dann hinaus in die staubige Luft,
               im grellen Lärm des vergehenden Tages, und die wachen Sommernächte in der Stadt\oindex{Wien@\textbf{Wien}, \emph{A.ADM2}|pwv}, eckelhaft; – müd vom Bureau\orgindex{»Phoenix« Versicherung@»Phönix« Versicherung|pwv}, schlecht aufgelegt und
               genzenlos nervös. In meinen literarischen Sachen von {\pb}allen Seiten behindert,
               ich kann keinen Weg machen, – nichts. Wer weiss, bekomme ich Urlaub, – wenn das so
               fortgeht, halte ich’s einfach nicht aus.\pend
           
\pstart
           Von Loris\pwindex{Hofmannsthal, Hugo von 1874-02-01 – 1929-07-15@\textsc{Hofmannsthal, Hugo von} (1874-02-01 – 1929-07-15), \emph{Schriftsteller/Schriftstellerin}|pw} habe ich heute einen lieben Brief erhalten. Er verlangt dringend, dass wir im
               Winter Theater spielen. Sie wissen ja, im Sommer reden wir immer von den großen
               Dingen, die wir machen wollen, und im Winter von den gemeinschaftlichten Soupers im
               Freien. Die alte Sache. Nicht einmal nachtmahlen können wir wenn wir’s uns
                  vornehmen.\hspace*{2.5em} Was macht denn \label{K_L03123-7v}\edtext{Beer Hofmann\pwindex{Beer-Hofmann, Richard 1866-07-11 – 1945-09-26@\textsc{Beer-Hofmann, Richard} (1866-07-11 – 1945-09-26), \emph{Schriftsteller/Schriftstellerin}|pw}}{\lemma{\textnormal{\emph{Beer Hofmann}}}\Cendnote{\textnormal{Richard Beer-Hofmann\pwindex{Beer-Hofmann, Richard 1866-07-11 – 1945-09-26@\textsc{Beer-Hofmann, Richard} (1866-07-11 – 1945-09-26), \emph{Schriftsteller/Schriftstellerin}|pwk} war ebenfalls in Ischl\oindex{Bad Ischl@\textbf{Bad Ischl}, \emph{P.PPL}|pwk}.}}}\label{K_L03123-7}? Arbeitet er etwas?\pend
           
\pstart
           Leben Sie wol, ich danke Ihnen bestens für Ihren Brief. Auf baldiges Wiedersehen, und
               möchten wir bald gescheidter sein, viel gescheidter als Sie im »Märchen\pwindex{Maerchen. Schauspiel in drei Aufzuegen@\emph{Das Märchen. Schauspiel in drei Aufzügen}|pw}« und ich im »Begräbnis\pwindex{Begraebnis@\emph{Begräbnis}|pw}«.\pend
           
\pstart
           Herzlichst Ihr{\\[\baselineskip]}\spacefill\mbox{Salten}\pend
           \leftskip=0em{}\selectlanguage{ngerman}\endnumbering\briefempfaengerindex{Schnitzler, Arthur@\textsc{Schnitzler, Arthur}!zzzSalten, Felix@\emph{von Felix Salten}!1893-07-061@{6. 7. 1893}|)be}\mylabel{L03123h}  \normalsize

\doendnotes{C}
\bigskip
\vfill

\clearpage

\footnotesize

\lohead{\textsc{register}}

% Definiere theindex-Environment komplett neu ohne reledmac
\makeatletter
\renewenvironment{theindex}{%
  \section*{\indexname}%
  \setlength{\parindent}{0pt}%
  \setlength{\parskip}{0pt plus 0.3pt}%
  \let\item\@idxitem
}{%
  \clearpage
}
\makeatother

\IfFileExists{\jobname-pw.ind}{\input{\jobname-pw.ind}}{}

\end{document}

      