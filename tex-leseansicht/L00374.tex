%% latex-leseansicht-vorspann.tex
%% Vorspann für die Leseansicht.
%% Lädt die gemeinsame Datei latex-vorspann.tex mit nicht gesetztem Schalter.

\newif\ifkorrekturansicht
\korrekturansichtfalse

\input{../tex-inputs/latex-vorspann}


\section[Arthur Schnitzler an Richard Beer-Hofmann, 29. 9. 1894]{L00374 Arthur Schnitzler an Richard Beer-Hofmann, 29. 9. 1894}
\nopagebreak\mylabel{L00374v}
\rehead{ }\normalsize\beginnumbering\briefempfaengerindex{Beer-Hofmann, Richard@\textsc{Beer-Hofmann, Richard}!zzzSchnitzler, Arthur@\emph{von Arthur Schnitzler}!1894-09-291@{29. 9. 1894}|(be}
\toendnotes[C]{\smallbreak\pagebreak[2]}
\correspDesc{Versand  durch Arthur Schnitzler am 29. 9. 1894 in Wien
\newline{}Erhalt  durch Richard Beer-Hofmann im Zeitraum [30. 9. 1894
                  – 4. 10. 1894?] in Florenz}\toendnotes[C]{\smallbreak}
\Standort{CUL, Schnitzler, B 8.1, S. 23–24.}
\physDesc{Brief, maschinenschriftliche Abschrift, 1 Blatt, 1 Seite, 1150 Zeichen
\newline{}Schreibmaschine
\newline{}Ordnung: von unbekannter Hand nummeriert: »42« }
\buchAbdrucke{\weitereDrucke{Arthur Schnitzler, Richard Beer-Hofmann: \emph{Briefwechsel 1891–1931}. Herausgegeben von Konstanze Fliedl. Wien, Zürich: \emph{Europaverlag} 1992, S. 60–61.} }\toendnotes[C]{\smallbreak}
\pstart
           \raggedleft{}{\pb}Wien\oindex{Wien@\textbf{Wien}, \emph{Verwaltungsgebiet}|pw}, 29. 9. 94.\pend
           \vspace{0.5em}
\pstart
           Lieber Richard,{ }\uline{zwei} (due) Karten hab ich Ihnen nach Pallanza\oindex{Pallanza@\textbf{Pallanza}|pw} geschrieben – das ist doch mehr als
               Mau? – Sie sind offenbar verloren gegangen.\pend
           
\pstart
           (Wer, – ich? (Leon\pwindex{Léon, Victor 4.\,1.\,1858 Senica – 23.\,2.\,1940 Wien@\textsc{Léon, Victor} (4.\,1.\,1858 Senica – 23.\,2.\,1940 Wien), \emph{Schriftsteller, Dramaturg}|pw} und Waldberg\pwindex{Waldberg, Heinrich von 2.\,3.\,1860 Iași – 20.\,10.\,1942 Konzentrationslager Theresienstadt@\textsc{Waldberg, Heinrich von} (2.\,3.\,1860 Iași – 20.\,10.\,1942 Konzentrationslager Theresienstadt), \emph{Schriftsteller}|pw}, Blumenthal\pwindex{Blumenthal, Oskar 13.\,3.\,1852 Berlin – 24.\,4.\,1917 ebd.@\textsc{Blumenthal, Oskar} (13.\,3.\,1852 Berlin – 24.\,4.\,1917 ebd.), \emph{Schriftsteller, Journalist, Theaterleiter}|pw} und
                  Kadelburg\pwindex{Kadelburg, Gustav 26.\,7.\,1851 Budapest – 11.\,9.\,1925 Berlin@\textsc{Kadelburg, Gustav} (26.\,7.\,1851 Budapest – 11.\,9.\,1925 Berlin), \emph{Schriftsteller, Schauspieler}|pw}, Brociner\pwindex{Brociner, Marco 20.\,10.\,1852 Iași – 12.\,4.\,1942 Wien@\textsc{Brociner, Marco} (20.\,10.\,1852 Iași – 12.\,4.\,1942 Wien), \emph{Schriftsteller, Journalist, Kritiker}|pw} und Gerhard\pwindex{Geiringer, Leopold 27.\,6.\,1851 Wien – 29.\,5.\,1900 ebd.@\textsc{Geiringer, Leopold} (27.\,6.\,1851 Wien – 29.\,5.\,1900 ebd.), \emph{Schriftsteller, Dramaturg}|pw})). –\pend
           
\pstart
           Gestern Eröffnung Josefstadt\oindex{Wien@\textbf{Wien}!VIII., Josefstadt@\textbf{VIII., Josefstadt}!Theater in der Josefstadt@\textbf{Theater in der Josefstadt}, \emph{Theater}|pw}; mit Dank des Herrn
                  Léon\pwindex{Léon, Victor 4.\,1.\,1858 Senica – 23.\,2.\,1940 Wien@\textsc{Léon, Victor} (4.\,1.\,1858 Senica – 23.\,2.\,1940 Wien), \emph{Schriftsteller, Dramaturg}|pw} im Frack, mit gekränkter Miene. Sehr
               amüsant, abgesehn vom 1.
               Akt\pwindex{\textcolor{red}{\textsuperscript{XXXX indx1}}!Tata-Toto. Vaudeville in drei Akten@\strich\emph{Tata-Toto. Vaudeville in drei Akten}|pwv}\pwindex{\textcolor{red}{\textsuperscript{XXXX indx1}}!Tata-Toto. Vaudeville in drei Akten@\strich\emph{Tata-Toto. Vaudeville in drei Akten}|pwv}. –\pend
           
\pstart
           Mein Stück\pwindex{Schnitzler, Arthur 15.\,5.\,1862 Wien – 21.\,10.\,1931 ebd.@\textsc{Schnitzler, Arthur} (15.\,5.\,1862 Wien – 21.\,10.\,1931 ebd.), \emph{Schriftsteller, Mediziner}!Liebelei. Schauspiel in drei Akten@\strich\emph{Liebelei. Schauspiel in drei Akten}|pwv} – zwei Akte bis auf
               letzte Feile (exclus.) vollendet. Wohl in acht Tagen fertig, – bühnenfertig in etwa
               4 Wochen, bühnenwirksam – wann? –\pend
           
\pstart
           Wie fühlen Sie sich? »Fliesst die Arbeit munter fort?« –\pend
           
\pstart
           {\pb}»Zeit\pwindex{Zeit. Wiener Wochenschrift@\emph{Die Zeit. Wiener Wochenschrift}|pw}« soll besorgt werden. – Bitte schreiben Sie häufiger – die
               Gemäldegalerie, die so hoffnungsvoll begonnen, hat rasch geendet. –\pend
           
\pstart
           Herzlich der Ihre{\\[\baselineskip]}\spacefill\mbox{\strikeout{Richard} entschuldigen – Arthur.}\pend
           \leftskip=0em{}
\pstart
           \noindent{}»Aeh, Kamerad, und was machen Weiber?« (Carricaturen\pwindex{Wiener Caricaturen@\emph{Wiener Caricaturen}|pw}, Floh\pwindex{\textcolor{red}{\textsuperscript{XXXX indx1}}!Floh@\strich\emph{Der Floh}|pw}\pwindex{\textcolor{red}{\textsuperscript{XXXX indx1}}!Floh@\strich\emph{Der Floh}|pw}\pwindex{\textcolor{red}{\textsuperscript{XXXX indx1}}!Floh@\strich\emph{Der Floh}|pw}\pwindex{\textcolor{red}{\textsuperscript{XXXX indx1}}!Floh@\strich\emph{Der Floh}|pw}\pwindex{\textcolor{red}{\textsuperscript{XXXX indx1}}!Floh@\strich\emph{Der Floh}|pw}\pwindex{\textcolor{red}{\textsuperscript{XXXX indx1}}!Floh@\strich\emph{Der Floh}|pw}\pwindex{\textcolor{red}{\textsuperscript{XXXX indx1}}!Floh@\strich\emph{Der Floh}|pw}\pwindex{\textcolor{red}{\textsuperscript{XXXX indx1}}!Floh@\strich\emph{Der Floh}|pw}, Bombe\pwindex{Bombe@\emph{Die Bombe}|pw}, Wiener
                     Witzblatt\pwindex{Wiener Witzblatt@\emph{Wiener Witzblatt}|pw}).\pend
           \stanza{}Und jene schöne, die vor Zeiten Euch\newverse{}Das Wasser auf den Nachttisch Abends stellte –\newverse{}Mit der Madonna holdem Lächeln – denkt\newverse{}Ihr dieses guten Mädchens manchmal noch, –\newverse{}Das sicher manches gegen die Empfängnis,\newverse{}Doch gegen das Beflecktsein gar nichts hatte –?\stanzaend{}
\pstart
           Der Obige, was ich leider nicht auf jenes Mädchen beziehn kann.\pend
           
\pstart
           \spacefill\mbox{A.}\pend
           
\pstart
           (nach Florenz\oindex{Florenz@\textbf{Florenz}|pw} a posta ferma)\pend
           \selectlanguage{ngerman}\endnumbering\briefempfaengerindex{Beer-Hofmann, Richard@\textsc{Beer-Hofmann, Richard}!zzzSchnitzler, Arthur@\emph{von Arthur Schnitzler}!1894-09-291@{29. 9. 1894}|)be}\mylabel{L00374h}  \newcommand{\dateiname}{L00374}\newcommand{\titel}{Arthur Schnitzler an Richard Beer-Hofmann, 29. 9. 1894}\newcommand{\editorInnen}{Martin Anton Müller und Gerd-Hermann Susen}%% latex-leseansicht-abspann.tex
%% Abspann für die Leseansicht.
%% Der Schalter \ifkorrekturansicht ist bereits durch den Vorspann gesetzt.

%% latex-abspann.tex
%% Gemeinsamer Abspann für Korrekturansicht und Leseansicht.
%% Setzt den Schalter \ifkorrekturansicht voraus (gesetzt in den
%% einbindenden Dateien latex-korrekturansicht-abspann.tex bzw.
%% latex-leseansicht-abspann.tex).
%% ---------------------------------------------------------------

\normalsize

% Das esempio-Environment wird nur in der Leseansicht benötigt
\ifkorrekturansicht\else
\newenvironment{esempio}[3]%
{
    \vspace{1.5ex}
    \rlap{\underline{#1}}
    \par
    \setlength{\parindent}{0cm}
    \nopagebreak
    \leftskip=#2cm
    \rightskip=#3cm
}
{
    \par
}
\fi

\doendnotes{C}
\bigskip
\vfill

\clearpage

\footnotesize

\ifkorrekturansicht
  \lohead{\textsc{register}}
\fi

% theindex-Environment neu definieren ohne reledmac
\makeatletter
\renewenvironment{theindex}{%
  \ifkorrekturansicht
    \section*{\indexname}%
  \else
    \subsubsection*{Index der erwähnten Entitäten}%
  \fi
  \setlength{\parindent}{0pt}%
  \setlength{\parskip}{0pt plus 0.3pt}%
  \let\item\@idxitem
}{%
  \ifkorrekturansicht\clearpage\fi
}
\makeatother

\IfFileExists{\jobname-pw.ind}{\input{\jobname-pw.ind}}{}

% Quellenangabe nur in der Leseansicht
\ifkorrekturansicht\else
% Fallback-Definitionen, falls die .tex-Datei \titel etc. nicht gesetzt hat
\providecommand{\titel}{}
\providecommand{\editorInnen}{}
\providecommand{\dateiname}{\jobname}

\vspace{3cm}

\vfill

\footnotesize
\textsc{Quelle}: \titel. Herausgegeben von {\editorInnen}. In: \emph{Arthur Schnitzler: Briefwechsel mit Autorinnen und Autoren}.
 Digitale Edition, https://schnitzler-briefe.acdh.oeaw.ac.at/{\dateiname}.html (Stand \today)
\fi

\end{document}


