%% latex-korrekturansicht-vorspann.tex
%% Vorspann für die Korrekturansicht.
%% Lädt die gemeinsame Datei latex-vorspann.tex mit gesetztem Schalter.

\newif\ifkorrekturansicht
\korrekturansichttrue

\input{../tex-inputs/latex-vorspann}


\section[Arthur Schnitzler an Richard Beer-Hofmann, 29. 9. 1894]{L00374 Arthur Schnitzler an Richard Beer-Hofmann, 29. 9. 1894}
\nopagebreak\mylabel{L00374v}
\rehead{ }\normalsize\beginnumbering\briefempfaengerindex{Beer-Hofmann, Richard@\textsc{Beer-Hofmann, Richard}!zzzSchnitzler, Arthur@\emph{von Arthur Schnitzler}!1894-09-291@{29. 9. 1894}|(be}
\toendnotes[C]{\smallbreak\pagebreak[2]}\Standort{CUL, Schnitzler, B 8.1, S. 23–24.}
\physDesc{Brief, maschinenschriftliche Abschrift1 Blatt, 1 Seite, 1150 Zeichen
\newline{}Schreibmaschine
\newline{}Ordnung: von unbekannter Hand nummeriert: »42« }
\buchAbdrucke{\weitereDrucke{Arthur Schnitzler, Richard Beer-Hofmann: \emph{Briefwechsel 1891–1931}. Wien, Zürich: \emph{Europaverlag} 1992, S. 60–61.} }\toendnotes[C]{\smallbreak}
\pstart
           \raggedleft{}{\pb}Wien\oindex{Wien@\textbf{Wien}, \emph{A.ADM2}|pw}, 29. 9. 94.\pend
           \vspace{0.5em}
\pstart
           Lieber Richard,{ }\uline{zwei} (due) Karten hab ich Ihnen nach Pallanza\oindex{Pallanza@\textbf{Pallanza}, \emph{P.PPL}|pw} geschrieben – das ist doch mehr als
               Mau? – Sie sind offenbar verloren gegangen.\pend
           
\pstart
           (Wer, – ich? (Leon\pwindex{Leon, Victor 4.1.1858 – 23.2.1940@\textsc{Léon, Victor} (4.1.1858 – 23.2.1940), \emph{Schriftsteller/Schriftstellerin, Dramaturg/Dramaturgin}|pw} und Waldberg\pwindex{Waldberg, Heinrich von 02.03.1860 – 20.10.1942@\textsc{Waldberg, Heinrich von} (02.03.1860 – 20.10.1942), \emph{Schriftsteller/Schriftstellerin}|pw}, Blumenthal\pwindex{Blumenthal, Oskar 13.03.1852 – 24.04.1917@\textsc{Blumenthal, Oskar} (13.03.1852 – 24.04.1917), \emph{Schriftsteller/Schriftstellerin, Journalist/Journalistin, Theaterleiter/Theaterleiterin}|pw} und
                  Kadelburg\pwindex{Kadelburg, Gustav 26.07.1851 – 11.09.1925@\textsc{Kadelburg, Gustav} (26.07.1851 – 11.09.1925), \emph{Schriftsteller/Schriftstellerin, Schauspieler/Schauspielerin}|pw}, Brociner\pwindex{Brociner, Marco 20.10.1852 – 12.04.1942@\textsc{Brociner, Marco} (20.10.1852 – 12.04.1942), \emph{Schriftsteller/Schriftstellerin, Journalist/Journalistin, Kritiker/Kritikerin}|pw} und Gerhard\pwindex{Geiringer, Leopold 27.06.1851 – 29.05.1900@\textsc{Geiringer, Leopold} (27.06.1851 – 29.05.1900), \emph{Schriftsteller/Schriftstellerin, Dramaturg/Dramaturgin}|pw})). –\pend
           
\pstart
           Gestern Eröffnung Josefstadt\oindex{Theater in der Josefstadt@\textbf{Theater in der Josefstadt}, \emph{Theater (K.THE)}|pw}; mit Dank des Herrn
                  Léon\pwindex{Leon, Victor 4.1.1858 – 23.2.1940@\textsc{Léon, Victor} (4.1.1858 – 23.2.1940), \emph{Schriftsteller/Schriftstellerin, Dramaturg/Dramaturgin}|pw} im Frack, mit gekränkter Miene. Sehr
               amüsant, abgesehn vom 1.
               Akt\pwindex{Tata-Toto. Vaudeville in drei Akten@\emph{Tata-Toto. Vaudeville in drei Akten}|pwv}. –\pend
           
\pstart
           Mein Stück\pwindex{Liebelei. Schauspiel in drei Akten@\emph{Liebelei. Schauspiel in drei Akten}|pwv} – zwei Akte bis auf
               letzte Feile (exclus.) vollendet. Wohl in acht Tagen fertig, – bühnenfertig in etwa
               4 Wochen, bühnenwirksam – wann? –\pend
           
\pstart
           Wie fühlen Sie sich? »Fliesst die Arbeit munter fort?« –\pend
           
\pstart
           {\pb}»Zeit\pwindex{Zeit. Wiener Wochenschrift@\emph{Die Zeit. Wiener Wochenschrift}|pw}« soll besorgt werden. – Bitte schreiben Sie häufiger – die
               Gemäldegalerie, die so hoffnungsvoll begonnen, hat rasch geendet. –\pend
           
\pstart
           Herzlich der Ihre{\\[\baselineskip]}\spacefill\mbox{\strikeout{Richard} entschuldigen – Arthur.}\pend
           \leftskip=0em{}
\pstart
           \noindent{}»Aeh, Kamerad, und was machen Weiber?« (Carricaturen\pwindex{Wiener Caricaturen@\emph{Wiener Caricaturen}|pw}, Floh\pwindex{Floh@\emph{Der Floh}|pw}, Bombe\pwindex{Bombe@\emph{Die Bombe}|pw}, Wiener
                     Witzblatt\pwindex{Wiener Witzblatt@\emph{Wiener Witzblatt}|pw}).\pend
           \stanza{}Und jene schöne, die vor Zeiten EuchDas Wasser auf den Nachttisch Abends stellte –Mit der Madonna holdem Lächeln – denktIhr dieses guten Mädchens manchmal noch, –Das sicher manches gegen die Empfängnis,Doch gegen das Beflecktsein gar nichts hatte –?\stanzaend{}
\pstart
           Der Obige, was ich leider nicht auf jenes Mädchen beziehn kann.\pend
           
\pstart
           \spacefill\mbox{A.}\pend
           
\pstart
           (nach Florenz\oindex{Florenz@\textbf{Florenz}, \emph{P.PPLA}|pw} a posta ferma)\pend
           \selectlanguage{ngerman}\endnumbering\briefempfaengerindex{Beer-Hofmann, Richard@\textsc{Beer-Hofmann, Richard}!zzzSchnitzler, Arthur@\emph{von Arthur Schnitzler}!1894-09-291@{29. 9. 1894}|)be}\mylabel{L00374h}  \normalsize

\doendnotes{C}
\bigskip
\vfill

\clearpage

\footnotesize

\lohead{\textsc{register}}

% Definiere theindex-Environment komplett neu ohne reledmac
\makeatletter
\renewenvironment{theindex}{%
  \section*{\indexname}%
  \setlength{\parindent}{0pt}%
  \setlength{\parskip}{0pt plus 0.3pt}%
  \let\item\@idxitem
}{%
  \clearpage
}
\makeatother

\IfFileExists{\jobname-pw.ind}{\input{\jobname-pw.ind}}{}

\end{document}

      