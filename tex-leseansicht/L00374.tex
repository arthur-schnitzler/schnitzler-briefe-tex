%% latex-leseansicht-vorspann.tex
%% Vorspann für die Leseansicht.
%% Lädt die gemeinsame Datei latex-vorspann.tex mit nicht gesetztem Schalter.

\newif\ifkorrekturansicht
\korrekturansichtfalse

\input{../tex-inputs/latex-vorspann}


         
         \renewcommand{\erwaehntePersonen}{Personen: Richard Beer-Hofmann, Oskar Blumenthal, Marco Brociner, Leopold Geiringer, Gustav Kadelburg, Victor Léon, Heinrich von Waldberg}
         \renewcommand{\erwaehnteOrte}{Orte: Florenz, Pallanza, Theater in der Josefstadt, Wien}
         \renewcommand{\erwaehnteWerke}{Werke: Der Floh, Die Bombe, Die Zeit. Wiener Wochenschrift, Liebelei. Schauspiel in drei Akten, Tata-Toto, Wiener Caricaturen, Wiener Witzblatt}
               \section[Arthur Schnitzler an Richard Beer-Hofmann, 29. 9. 1894]{ Arthur Schnitzler an Richard Beer-Hofmann, 29. 9. 1894}\nopagebreak\mylabel{v}\rehead{ }\begin{ledgroupsized}[t]{13cm}\normalsize\beginnumbering \toendnotes[C]{\smallbreak\pagebreak[2]} \Standort{CUL, Schnitzler, B 8.1, S. 23–24.}
\physDesc{Brief, Maschinenschriftliche Abschrift, 1 Blatt, 1 Seite
\newline{}Schreibmaschine\newline{}Ordnung: von unbekannter Hand nummeriert: »42« }\buchAbdrucke{\weitereDrucke{Arthur Schnitzler, Richard Beer-Hofmann: \emph{Briefwechsel 1891–1931}. Hg. Konstanze Fliedl. Wien, Zürich: \emph{Europaverlag} 1992, S. 60–61.} }\toendnotes[C]{\smallbreak}\pstart
           \raggedleft{}{\pb}Wien\oindex{Wien@\textbf{Wien}|pw}, 29. 9. 94.\pend
           \pstart
           Lieber Richard, \uline{zwei} (due) Karten hab ich Ihnen nach Pallanza\oindex{Pallanza@\textbf{Pallanza}|pw} geschrieben – das ist doch mehr als Mau? – Sie sind
               offenbar verloren gegangen.\pend
           \pstart
           (Wer, – ich? (Leon\pwindex{Leon, Victor 4.1.1858 – 23.2.1940@\textsc{Léon, Victor} (4.1.1858 – 23.2.1940), \emph{Schriftsteller, Dramaturg}|pw} und Waldberg\pwindex{Waldberg, Heinrich von 02.03.1860 – 20.10.1942@\textsc{Waldberg, Heinrich von} (02.03.1860 – 20.10.1942), \emph{Schriftsteller}|pw}, Blumenthal\pwindex{Blumenthal, Oskar 13.03.1852 – 24.04.1917@\textsc{Blumenthal, Oskar} (13.03.1852 – 24.04.1917), \emph{Schriftsteller, Journalist, Theaterleiter}|pw} und
                  Kadelburg\pwindex{Kadelburg, Gustav 26.07.1851 – 11.09.1925@\textsc{Kadelburg, Gustav} (26.07.1851 – 11.09.1925), \emph{Schriftsteller, Schauspieler}|pw}, Brociner\pwindex{Brociner, Marco 20.10.1852 – 12.04.1942@\textsc{Brociner, Marco} (20.10.1852 – 12.04.1942), \emph{Schriftsteller, Journalist, Kritiker}|pw} und Gerhard\pwindex{Geiringer, Leopold 27.06.1851 – 29.05.1900@\textsc{Geiringer, Leopold} (27.06.1851 – 29.05.1900), \emph{Schriftsteller, Dramaturg}|pw})). –\pend
           \pstart
           Gestern Eröffnung Josefstadt\oindex{Theater in der Josefstadt@\textbf{Theater in der Josefstadt}|pw}; mit Dank des Herrn Léon\pwindex{Leon, Victor 4.1.1858 – 23.2.1940@\textsc{Léon, Victor} (4.1.1858 – 23.2.1940), \emph{Schriftsteller, Dramaturg}|pw} im Frack, mit gekränkter Miene. Sehr amüsant,
               abgesehn vom 1. Akt\pwindex{\textcolor{red}{\textsuperscript{XXXX1 indx}}!Tata-Toto1892@\strich\emph{Tata-Toto} {[}1892{]}|pwv}\pwindex{\textcolor{red}{\textsuperscript{XXXX1 indx}}!Tata-Toto1892@\strich\emph{Tata-Toto} {[}1892{]}|pwv}. –\pend
           \pstart
           Mein Stück\pwindex{Schnitzler, Arthur 15.05.1862 – 21.10.1931@\textsc{Schnitzler, Arthur} (15.05.1862 – 21.10.1931), \emph{Schriftsteller, Mediziner}!Liebelei. Schauspiel in drei Akten1895-10-09@\strich\emph{Liebelei. Schauspiel in drei Akten} {[}1895-10-09{]}|pwv} – zwei Akte bis auf
               letzte Feile (exclus.) vollendet. Wohl in acht Tagen fertig, – bühnenfertig in etwa
               4 Wochen, bühnenwirksam – wann? –\pend
           \pstart
           Wie fühlen Sie sich? »Fliesst die Arbeit munter fort?« –\pend
           \pstart
           {\pb}»Zeit\pwindex{Zeit. Wiener Wochenschrift1894 – 1904@\emph{Die Zeit. Wiener Wochenschrift} {[}1894 – 1904{]}|pw}«
               soll besorgt werden. – Bitte schreiben Sie häufiger – die Gemäldegalerie, die so
               hoffnungsvoll begonnen, hat rasch geendet. –\pend
           \pstart
           Herzlich der Ihre{\\[\baselineskip]}\spacefill\mbox{\strikeout{Richard} entschuldigen – Arthur.}\pend
           \leftskip=0em{}\pstart
           \noindent{}»Aeh, Kamerad, und was machen Weiber?« (Carricaturen\pwindex{?? Werk@Nicht ermittelte Verfasserinnen und Verfasser!Wiener Caricaturen1881 – 1925@\emph{Wiener Caricaturen} {[}1881 – 1925{]}|pw}, Floh\pwindex{\textcolor{red}{\textsuperscript{XXXX1 indx}}!Floh1869 – 1919@\strich\emph{Der Floh} {[}1869 – 1919{]}|pw}\pwindex{\textcolor{red}{\textsuperscript{XXXX1 indx}}!Floh1869 – 1919@\strich\emph{Der Floh} {[}1869 – 1919{]}|pw}\pwindex{\textcolor{red}{\textsuperscript{XXXX1 indx}}!Floh1869 – 1919@\strich\emph{Der Floh} {[}1869 – 1919{]}|pw}\pwindex{\textcolor{red}{\textsuperscript{XXXX1 indx}}!Floh1869 – 1919@\strich\emph{Der Floh} {[}1869 – 1919{]}|pw}\pwindex{\textcolor{red}{\textsuperscript{XXXX1 indx}}!Floh1869 – 1919@\strich\emph{Der Floh} {[}1869 – 1919{]}|pw}\pwindex{\textcolor{red}{\textsuperscript{XXXX1 indx}}!Floh1869 – 1919@\strich\emph{Der Floh} {[}1869 – 1919{]}|pw}\pwindex{\textcolor{red}{\textsuperscript{XXXX1 indx}}!Floh1869 – 1919@\strich\emph{Der Floh} {[}1869 – 1919{]}|pw}\pwindex{\textcolor{red}{\textsuperscript{XXXX1 indx}}!Floh1869 – 1919@\strich\emph{Der Floh} {[}1869 – 1919{]}|pw}, Bombe\pwindex{?? Werk@Nicht ermittelte Verfasserinnen und Verfasser!Bombe1871 – 1925@\emph{Die Bombe} {[}1871 – 1925{]}|pw}, Wiener
                  Witzblatt\pwindex{?? Werk@Nicht ermittelte Verfasserinnen und Verfasser!Wiener WitzblattNone@\emph{Wiener Witzblatt} {[}None{]}|pw}).\pend
           \stanza{}Und jene schöne, die vor Zeiten Euch\newverse{}Das Wasser auf den Nachttisch Abends stellte –\newverse{}Mit der Madonna holdem Lächeln – denkt\newverse{}Ihr dieses guten Mädchens manchmal noch, –\newverse{}Das sicher manches gegen die Empfängnis,\newverse{}Doch gegen das Beflecktsein gar nichts hatte –?\stanzaend{}\pstart
           Der Obige, was ich leider nicht auf jenes Mädchen beziehn kann.\pend
           \pstart
           \spacefill\mbox{A.}\pend
           \pstart
           (nach Florenz\oindex{Florenz@\textbf{Florenz}|pw} a posta ferma)\pend
           
         
         \endnumbering\mylabel{h}\end{ledgroupsized}  \newcommand{\dateiname}{L00374}\newcommand{\titel}{Arthur Schnitzler an Richard Beer-Hofmann, 29. 9. 1894}\newcommand{\editorInnen}{Martin Anton Müller und Gerd-Hermann Susen}%% latex-leseansicht-abspann.tex
%% Abspann für die Leseansicht.
%% Der Schalter \ifkorrekturansicht ist bereits durch den Vorspann gesetzt.

%% latex-abspann.tex
%% Gemeinsamer Abspann für Korrekturansicht und Leseansicht.
%% Setzt den Schalter \ifkorrekturansicht voraus (gesetzt in den
%% einbindenden Dateien latex-korrekturansicht-abspann.tex bzw.
%% latex-leseansicht-abspann.tex).
%% ---------------------------------------------------------------

\normalsize

% Das esempio-Environment wird nur in der Leseansicht benötigt
\ifkorrekturansicht\else
\newenvironment{esempio}[3]%
{
    \vspace{1.5ex}
    \rlap{\underline{#1}}
    \par
    \setlength{\parindent}{0cm}
    \nopagebreak
    \leftskip=#2cm
    \rightskip=#3cm
}
{
    \par
}
\fi

\doendnotes{C}
\bigskip
\vfill

\clearpage

\footnotesize

\ifkorrekturansicht
  \lohead{\textsc{register}}
\fi

% theindex-Environment neu definieren ohne reledmac
\makeatletter
\renewenvironment{theindex}{%
  \ifkorrekturansicht
    \section*{\indexname}%
  \else
    \subsubsection*{Index der erwähnten Entitäten}%
  \fi
  \setlength{\parindent}{0pt}%
  \setlength{\parskip}{0pt plus 0.3pt}%
  \let\item\@idxitem
}{%
  \ifkorrekturansicht\clearpage\fi
}
\makeatother

\IfFileExists{\jobname-pw.ind}{\input{\jobname-pw.ind}}{}

% Quellenangabe nur in der Leseansicht
\ifkorrekturansicht\else
% Fallback-Definitionen, falls die .tex-Datei \titel etc. nicht gesetzt hat
\providecommand{\titel}{}
\providecommand{\editorInnen}{}
\providecommand{\dateiname}{\jobname}

\vspace{3cm}

\vfill

\footnotesize
\textsc{Quelle}: \titel. Herausgegeben von {\editorInnen}. In: \emph{Arthur Schnitzler: Briefwechsel mit Autorinnen und Autoren}.
 Digitale Edition, https://schnitzler-briefe.acdh.oeaw.ac.at/{\dateiname}.html (Stand \today)
\fi

\end{document}


      