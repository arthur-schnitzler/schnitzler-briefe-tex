%% latex-korrekturansicht-vorspann.tex
%% Vorspann für die Korrekturansicht.
%% Lädt die gemeinsame Datei latex-vorspann.tex mit gesetztem Schalter.

\newif\ifkorrekturansicht
\korrekturansichttrue

\input{../tex-inputs/latex-vorspann}


\section[ Felix Salten an Arthur Schnitzler, 21. 9. 1929]{L03587 Felix Salten an Arthur Schnitzler, 21. 9. 1929}
\nopagebreak\mylabel{L03587v}
\rehead{ }\normalsize\beginnumbering\briefempfaengerindex{Schnitzler, Arthur@\textsc{Schnitzler, Arthur}!zzzSalten, Felix@\emph{von Felix Salten}!1929-09-211@{21. 9. 1929}|(be}
\toendnotes[C]{\smallbreak\pagebreak[2]}\Standort{CUL, Schnitzler, B 89, B 2.}
\physDesc{Brief, 1 Blatt, 1 Seite, 844 Zeichen
\newline{}Handschrift: schwarze Tinte, lateinische Kurrent
\newline{}Schnitzler: mit rotem Buntstift Vermerke: »F. S.\pwindex{Salten, Felix 06.09.1869 – 08.10.1945@\textsc{Salten, Felix} (06.09.1869 – 08.10.1945), \emph{Schriftsteller/Schriftstellerin, Journalist/Journalistin, Chefredakteur/Chefredakteurin}|pw}« und eine Unterstreichung 
\newline{}Ordnung: mit Bleistift von unbekannter Hand nummeriert: »300« }\toendnotes[C]{\smallbreak}
\pstart
           \raggedleft{}{\pb}Grundlsee\oindex{Grundlsee [Gemeinde]@\textbf{Grundlsee [Gemeinde]}, \emph{Besiedelter Ort (A.BSO)}|pw}, 21. 9. 29\pend
           \vspace{0.5em}
\pstart
           Lieber, für Ihr \label{K_L03587-1v}\edtext{Telegramm vom Genfersee\oindex{Genfer See@\textbf{Genfer See}, \emph{H.LK}|pw}}{\lemma{\textnormal{\emph{Telegramm vom Genfersee}}}\Cendnote{\textnormal{Anlässlich von Saltens\pwindex{Salten, Felix 06.09.1869 – 08.10.1945@\textsc{Salten, Felix} (06.09.1869 – 08.10.1945), \emph{Schriftsteller/Schriftstellerin, Journalist/Journalistin, Chefredakteur/Chefredakteurin}|pwk} 60. Geburtstag am 6. 9. 1929, siehe A. S.: \emph{Tagebuch}, 5. 9. 1929. Er
                  war also Beer-Hofmanns\pwindex{Beer-Hofmann, Richard 1866-07-11 – 1945-09-26@\textsc{Beer-Hofmann, Richard} (1866-07-11 – 1945-09-26), \emph{Schriftsteller/Schriftstellerin}|pwk} Vorhaben gefolgt, vgl. Richard Beer-Hofmann an Arthur Schnitzler, 28. 8. 1929.
               }}}\label{K_L03587-1} danke ich Ihnen herzlich! Ebenso für Ihre \label{K_L03587-2v}\edtext{Karte aus Marienbad\oindex{Marienbad@\textbf{Marienbad}, \emph{P.PPL}|pw}}{\lemma{\textnormal{\emph{Karte aus Marienbad}}}\Cendnote{\textnormal{Schnitzler war zwischen 12. 9. 1929 und 21. 9. 1929 in Marienbad\oindex{Marienbad@\textbf{Marienbad}, \emph{P.PPL}|pwk}.}}}\label{K_L03587-2}, die mich sehr gefreut hat.
               Ganz besonders aber muß ich Ihnen für Ihr sozusagen \label{K_L03587-3v}\edtext{öffentlich geäussertes Wort\pwindex{Mein lieber Felix Salten]@\emph{[Mein lieber Felix Salten]}|pwv}}{\lemma{\textnormal{\emph{öffentlich … Wort}}}\Cendnote{\textnormal{Siehe Arthur Schnitzler an Felix Salten, 29. 7. 1929 und A. S.: \emph{»Das Zeitlose ist von kürzester Dauer«}, [Mein lieber Felix Salten!], [November 1929].
               }}}\label{K_L03587-3} sein. Der Zsolnay Verlag\orgindex{Paul Zsolnay Verlag@Paul Zsolnay Verlag|pw} überraschte mich
               damit und ich darf wohl sagen, dass ich nicht viele derartig angenehme Überraschungen
               erlebt habe. Einer der mir wertvollsten und mich am meisten wärmenden Aussprüche ist
               der Ihre! Ach ja – doch wozu stotternd und stammelnd an Dinge rühren, die sich so
               schwer aussprechen lassen. Sie können sich ja ungefähr denken, was man empfindet,
               wenn man so alt werden durfte. Und wenn Sie auch nicht genau alles denken oder
               wissen, was gerade mich bewegt, – ich kann’s doch nicht in Worte bringen. Jedenfalls
               haben Sie innigsten Dank! Sehr herzlich und hoffentlich auf sehr bald!\pend
           
\pstart
           Ihr {\\[\baselineskip]}\spacefill\mbox{Felix Salten}\pend
           \leftskip=0em{}\selectlanguage{ngerman}\endnumbering\briefempfaengerindex{Schnitzler, Arthur@\textsc{Schnitzler, Arthur}!zzzSalten, Felix@\emph{von Felix Salten}!1929-09-211@{21. 9. 1929}|)be}\mylabel{L03587h}  \normalsize

\doendnotes{C}
\bigskip
\vfill

\clearpage

\footnotesize

\lohead{\textsc{register}}

% Definiere theindex-Environment komplett neu ohne reledmac
\makeatletter
\renewenvironment{theindex}{%
  \section*{\indexname}%
  \setlength{\parindent}{0pt}%
  \setlength{\parskip}{0pt plus 0.3pt}%
  \let\item\@idxitem
}{%
  \clearpage
}
\makeatother

\IfFileExists{\jobname-pw.ind}{\input{\jobname-pw.ind}}{}

\end{document}

      