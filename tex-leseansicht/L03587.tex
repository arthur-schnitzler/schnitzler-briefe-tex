%% latex-leseansicht-vorspann.tex
%% Vorspann für die Leseansicht.
%% Lädt die gemeinsame Datei latex-vorspann.tex mit nicht gesetztem Schalter.

\newif\ifkorrekturansicht
\korrekturansichtfalse

\input{../tex-inputs/latex-vorspann}


\section[ Felix Salten an Arthur Schnitzler, 21. 9. 1929]{L03587 Felix Salten an Arthur Schnitzler,  21. 9. 1929}
\nopagebreak\mylabel{L03587v}
\rehead{ }\normalsize\beginnumbering\briefempfaengerindex{Schnitzler, Arthur@\textsc{Schnitzler, Arthur}!zzzSalten, Felix@\emph{von Felix Salten}!1929-09-211@{21. 9. 1929}|(be}
\toendnotes[C]{\smallbreak\pagebreak[2]}
\correspDesc{Versand  durch Felix Salten am 21. 9. 1929 in Grundlsee
\newline{}Erhalt  durch Arthur Schnitzler im Zeitraum [22. 9. 1929
                  – 26. 9. 1929?] in Wien}\toendnotes[C]{\smallbreak}
\Standort{CUL, Schnitzler, B 89, B 2.}
\physDesc{Brief, 1 Blatt, 1 Seite, 844 Zeichen
\newline{}Handschrift: schwarze Tinte, lateinische Kurrent
\newline{}Schnitzler: mit rotem Buntstift Vermerke: »F. S.\pwindex{Salten, Felix 6.\,9.\,1869 Budapest – 8.\,10.\,1945 Zürich@\textsc{Salten, Felix} (6.\,9.\,1869 Budapest – 8.\,10.\,1945 Zürich), \emph{Schriftsteller, Journalist, Chefredakteur}|pw}« und eine Unterstreichung 
\newline{}Ordnung: mit Bleistift von unbekannter Hand nummeriert: »300« }\toendnotes[C]{\smallbreak}
\pstart
           \raggedleft{}{\pb}Grundlsee\oindex{Grundlsee [Gemeinde]@\textbf{Grundlsee [Gemeinde]}|pw}, 21. 9. 29\pend
           \vspace{0.5em}
\pstart
           Lieber, für Ihr \label{K_L03587-1v}\edtext{Telegramm vom Genfersee\oindex{Genfer See@\textbf{Genfer See}, \emph{See}|pw}}{\lemma{\textnormal{\emph{Telegramm vom Genfersee}}}\Cendnote{\textnormal{Anlässlich von Saltens\pwindex{Salten, Felix 6.\,9.\,1869 Budapest – 8.\,10.\,1945 Zürich@\textsc{Salten, Felix} (6.\,9.\,1869 Budapest – 8.\,10.\,1945 Zürich), \emph{Schriftsteller, Journalist, Chefredakteur}|pwk} 60. Geburtstag am 6. 9. 1929, siehe A. S.: \emph{Tagebuch}, 5. 9. 1929. Er
                  war also Beer-Hofmanns\pwindex{Beer-Hofmann, Richard 11.\,7.\,1866 Wien – 26.\,9.\,1945 New York City@\textsc{Beer-Hofmann, Richard} (11.\,7.\,1866 Wien – 26.\,9.\,1945 New York City), \emph{Schriftsteller}|pwk} Vorhaben gefolgt, vgl. XXXX Auszeichnungsfehler: Dokument L02521 nicht gefunden.
               }}}\label{K_L03587-1} danke ich Ihnen herzlich! Ebenso für Ihre \label{K_L03587-2v}\edtext{Karte aus Marienbad\oindex{Marienbad@\textbf{Marienbad}|pw}}{\lemma{\textnormal{\emph{Karte aus Marienbad}}}\Cendnote{\textnormal{Schnitzler war zwischen 12. 9. 1929 und 21. 9. 1929 in Marienbad\oindex{Marienbad@\textbf{Marienbad}|pwk}.}}}\label{K_L03587-2}, die mich sehr gefreut hat.
               Ganz besonders aber muß ich Ihnen für Ihr sozusagen \label{K_L03587-3v}\edtext{öffentlich geäussertes Wort\pwindex{Schnitzler, Arthur 15.\,5.\,1862 Wien – 21.\,10.\,1931 ebd.@\textsc{Schnitzler, Arthur} (15.\,5.\,1862 Wien – 21.\,10.\,1931 ebd.), \emph{Schriftsteller, Mediziner}!Mein lieber Felix Salten]@\strich\emph{[Mein lieber Felix Salten]}|pwv}}{\lemma{\textnormal{\emph{öffentlich … Wort}}}\Cendnote{\textnormal{Siehe XXXX Auszeichnungsfehler: Dokument L02950 nicht gefunden und A. S.: \emph{»Das Zeitlose ist von kürzester Dauer«}, [Mein lieber Felix Salten!], [November 1929].
               }}}\label{K_L03587-3} sein. Der Zsolnay Verlag\orgindex{Paul Zsolnay Verlag@Paul Zsolnay Verlag|pw} überraschte mich
               damit und ich darf wohl sagen, dass ich nicht viele derartig angenehme Überraschungen
               erlebt habe. Einer der mir wertvollsten und mich am meisten wärmenden Aussprüche ist
               der Ihre! Ach ja – doch wozu stotternd und stammelnd an Dinge rühren, die sich so
               schwer aussprechen lassen. Sie können sich ja ungefähr denken, was man empfindet,
               wenn man so alt werden durfte. Und wenn Sie auch nicht genau alles denken oder
               wissen, was gerade mich bewegt, – ich kann’s doch nicht in Worte bringen. Jedenfalls
               haben Sie innigsten Dank! Sehr herzlich und hoffentlich auf sehr bald!\pend
           
\pstart
           Ihr {\\[\baselineskip]}\spacefill\mbox{Felix Salten}\pend
           \leftskip=0em{}\selectlanguage{ngerman}\endnumbering\briefempfaengerindex{Schnitzler, Arthur@\textsc{Schnitzler, Arthur}!zzzSalten, Felix@\emph{von Felix Salten}!1929-09-211@{21. 9. 1929}|)be}\mylabel{L03587h}  \newcommand{\dateiname}{L03587}\newcommand{\titel}{Felix Salten an Arthur Schnitzler, 21. 9. 1929}\newcommand{\editorInnen}{Martin Anton Müller und Laura Untner}%% latex-leseansicht-abspann.tex
%% Abspann für die Leseansicht.
%% Der Schalter \ifkorrekturansicht ist bereits durch den Vorspann gesetzt.

%% latex-abspann.tex
%% Gemeinsamer Abspann für Korrekturansicht und Leseansicht.
%% Setzt den Schalter \ifkorrekturansicht voraus (gesetzt in den
%% einbindenden Dateien latex-korrekturansicht-abspann.tex bzw.
%% latex-leseansicht-abspann.tex).
%% ---------------------------------------------------------------

\normalsize

% Das esempio-Environment wird nur in der Leseansicht benötigt
\ifkorrekturansicht\else
\newenvironment{esempio}[3]%
{
    \vspace{1.5ex}
    \rlap{\underline{#1}}
    \par
    \setlength{\parindent}{0cm}
    \nopagebreak
    \leftskip=#2cm
    \rightskip=#3cm
}
{
    \par
}
\fi

\doendnotes{C}
\bigskip
\vfill

\clearpage

\footnotesize

\ifkorrekturansicht
  \lohead{\textsc{register}}
\fi

% theindex-Environment neu definieren ohne reledmac
\makeatletter
\renewenvironment{theindex}{%
  \ifkorrekturansicht
    \section*{\indexname}%
  \else
    \subsubsection*{Index der erwähnten Entitäten}%
  \fi
  \setlength{\parindent}{0pt}%
  \setlength{\parskip}{0pt plus 0.3pt}%
  \let\item\@idxitem
}{%
  \ifkorrekturansicht\clearpage\fi
}
\makeatother

\IfFileExists{\jobname-pw.ind}{\input{\jobname-pw.ind}}{}

% Quellenangabe nur in der Leseansicht
\ifkorrekturansicht\else
% Fallback-Definitionen, falls die .tex-Datei \titel etc. nicht gesetzt hat
\providecommand{\titel}{}
\providecommand{\editorInnen}{}
\providecommand{\dateiname}{\jobname}

\vspace{3cm}

\vfill

\footnotesize
\textsc{Quelle}: \titel. Herausgegeben von {\editorInnen}. In: \emph{Arthur Schnitzler: Briefwechsel mit Autorinnen und Autoren}.
 Digitale Edition, https://schnitzler-briefe.acdh.oeaw.ac.at/{\dateiname}.html (Stand \today)
\fi

\end{document}


