%% latex-korrekturansicht-vorspann.tex
%% Vorspann für die Korrekturansicht.
%% Lädt die gemeinsame Datei latex-vorspann.tex mit gesetztem Schalter.

\newif\ifkorrekturansicht
\korrekturansichttrue

\input{../tex-inputs/latex-vorspann}


\section[Arthur Schnitzler an Hermann Bahr, 1. 7. 1901]{L01138 Arthur Schnitzler an Hermann Bahr, 1. 7. 1901}
\nopagebreak\mylabel{L01138v}
\rehead{ }\normalsize\beginnumbering\briefempfaengerindex{Bahr, Hermann@\textsc{Bahr, Hermann}!zzzSchnitzler, Arthur@\emph{von Arthur Schnitzler}!1901-07-011@{1. 7. 1901}|(be}
\toendnotes[C]{\smallbreak\pagebreak[2]}\Standort{TMW, HS AM 23390 Ba.}
\physDesc{Brief, 1 Blatt, 3 Seiten, 641 Zeichen
\newline{}Handschrift: Bleistift, deutsche Kurrent
\newline{}Ordnung: Lochung }
\buchAbdrucke{\weitereDrucke{1) Arthur Schnitzler: \emph{The Letters of Arthur Schnitzler to Hermann Bahr}. Chapel Hill: \emph{The University of North Carolina Press} 1978, S. 103.} \weitereDrucke{2) Hermann Bahr, Arthur Schnitzler: \emph{Briefwechsel, Aufzeichnungen, Dokumente (1891–1931)}. Göttingen: \emph{Wallstein} 2018, S. 212.} }\toendnotes[C]{\smallbreak}
\pstart{}{\pb}lieber
                  Hermann\pend\vspace{0.5em}
\pstart
           es drängt mich, dir zu deinem Collegen Poetzl\pwindex{Poetzl, Eduard 17.03.1851 – 20.08.1914@\textsc{Pötzl, Eduard} (17.03.1851 – 20.08.1914), \emph{Schriftsteller/Schriftstellerin, Journalist/Journalistin}|pw}
               wärmſtens zu gratuliren. Das ſind einmal mannhafte, echt \label{K_L01138-1v}\edtext{t\damage{e}utſche Worte\pwindex{Luesternheit (Predigt in der Wueste)@\emph{Lüsternheit (Predigt in der Wüste)}|pwv}}{\lemma{\textnormal{\emph{teutſche Worte}}}\Cendnote{\textnormal{Ed. Pötzl\pwindex{Poetzl, Eduard 17.03.1851 – 20.08.1914@\textsc{Pötzl, Eduard} (17.03.1851 – 20.08.1914), \emph{Schriftsteller/Schriftstellerin, Journalist/Journalistin}|pwk}: \emph{Lüsternheit. (Predigt in der Wüste).}\pwindex{Luesternheit (Predigt in der Wueste)@\emph{Lüsternheit (Predigt in der Wüste)}|pwk} In: \emph{Neues Wiener Tagblatt}\pwindex{Neues Wiener Tagblatt@\emph{Neues Wiener Tagblatt}|pwk}, Jg. 35, Nr. 176,
                        29. 6. 1901, S. 1–2. Schon der Titel macht es als 
                  Replik auf Bahrs\pwindex{Bahr, Hermann 19.07.1863 – 15.01.1934@\textsc{Bahr, Hermann} (19.07.1863 – 15.01.1934), \emph{Schriftsteller/Schriftstellerin, Kritiker/Kritikerin}|pwk}{ }\emph{Erotisch}\pwindex{Erotisch@\emph{Erotisch}|pwk} deutlich.}}}\label{K_L01138-1}! Das Herz geht einem auf,
               wenn man ſie lieſt. »Es iſt beſſer,
                  das gute zu heucheln als es durch offenkundige Frevel {\pb}aller Art von der
                  Tagesordnung gänzlich abſetzen\pwindex{Luesternheit (Predigt in der Wueste)@\emph{Lüsternheit (Predigt in der Wüste)}|pwv}.« – »Es iſt immer noch moraliſcher im Geheimen zu ſündigen als auf
                  oeffentlichem Markte mit dem Laſter Arm in Arm zu gehen –\pwindex{Luesternheit (Predigt in der Wueste)@\emph{Lüsternheit (Predigt in der Wüste)}|pwv}« »Die Geſa{\geminationm}theit darf
                  die Tugend nicht verachten, ſondern muſs ſie heilig halten und auf ihren Schild
                  erheben\pwindex{Luesternheit (Predigt in der Wueste)@\emph{Lüsternheit (Predigt in der Wüste)}|pwv}« –\pend
           
\pstart
           {\pb}– So ehrlich iſt die
               Heuchelei ſelten geweſen!\pend
           
\pstart
           Leb wohl und ſei herzlich gegrüßt.{\\[\baselineskip]}Dein{\\[\baselineskip]}\spacefill\mbox{Arth Sch}\pend
           \leftskip=0em{}
\pstart
           St Anton\oindex{St. Anton am Arlberg@\textbf{St. Anton am Arlberg}, \emph{A.ADM3}|pw}{ }1. 7. 109.\pend
           \selectlanguage{ngerman}\endnumbering\briefempfaengerindex{Bahr, Hermann@\textsc{Bahr, Hermann}!zzzSchnitzler, Arthur@\emph{von Arthur Schnitzler}!1901-07-011@{1. 7. 1901}|)be}\mylabel{L01138h}  \normalsize

\doendnotes{C}
\bigskip
\vfill

\clearpage

\footnotesize

\lohead{\textsc{register}}

% Definiere theindex-Environment komplett neu ohne reledmac
\makeatletter
\renewenvironment{theindex}{%
  \section*{\indexname}%
  \setlength{\parindent}{0pt}%
  \setlength{\parskip}{0pt plus 0.3pt}%
  \let\item\@idxitem
}{%
  \clearpage
}
\makeatother

\IfFileExists{\jobname-pw.ind}{\input{\jobname-pw.ind}}{}

\end{document}

      