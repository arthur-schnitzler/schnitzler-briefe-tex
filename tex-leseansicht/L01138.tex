%% latex-leseansicht-vorspann.tex
%% Vorspann für die Leseansicht.
%% Lädt die gemeinsame Datei latex-vorspann.tex mit nicht gesetztem Schalter.

\newif\ifkorrekturansicht
\korrekturansichtfalse

\input{../tex-inputs/latex-vorspann}


         
         \newcommand{\erwaehntePersonen}{Personen: Hermann Bahr, Eduard Pötzl}
         \newcommand{\erwaehnteInstitutionen}{}
         \newcommand{\erwaehnteOrte}{Orte: St. Anton am Arlberg, Wien}
         \newcommand{\erwaehnteWerke}{Werke: Erotisch, Lüsternheit (Predigt in der Wüste), Neues Wiener Tagblatt}
               \section[Arthur Schnitzler an Hermann Bahr, 1. 7. 1901]{ Arthur Schnitzler an Hermann Bahr, 1. 7. 1901}\nopagebreak\mylabel{v}\rehead{ }\begin{ledgroupsized}[t]{13cm}\normalsize\beginnumbering \toendnotes[C]{\smallbreak\pagebreak[2]} \Standort{TMW, HS AM 23390 Ba.}
\physDesc{Brief, 1 Blatt, 3 Seiten
\newline{}Handschrift: Bleistift, deutsche Kurrent\newline{}Ordnung: Lochung }\buchAbdrucke{\weitereDrucke{1) \emph{1. 7. 1909.} In: Arthur Schnitzler: \emph{The Letters of Arthur Schnitzler to Hermann Bahr}. Edited, annotated, and with an introduction, by Donald G.
                        Daviau. Chapel Hill: \emph{The University of North Carolina Press} 1978, S. 103 (University of North Carolina studies in the Germanic languages
                        and literatures, 89).} \weitereDrucke{2) Hermann Bahr, Arthur Schnitzler: \emph{Briefwechsel, Aufzeichnungen, Dokumente (1891–1931)}. Hg. Kurt Ifkovits und Martin Anton Müller. Göttingen: \emph{Wallstein} 2018, S. 212.} }\toendnotes[C]{\smallbreak}\pstart{}{\pb}lieber
                  Hermann\pend\pstart
           es drängt mich, dir zu deinem Collegen Poetzl\pwindex{Poetzl, Eduard 17.03.1851 – 20.08.1914@\textsc{Pötzl, Eduard} (17.03.1851 – 20.08.1914), \emph{Schriftsteller, Journalist}|pw}
               wärmſtens zu gratuliren. Das ſind einmal mannhafte, echt \label{K_L01138_1v}\edtext{t\damage{e}utſche Worte\pwindex{Poetzl, Eduard 17.03.1851 – 20.08.1914@\textsc{Pötzl, Eduard} (17.03.1851 – 20.08.1914), \emph{Schriftsteller, Journalist}!Luesternheit (Predigt in der Wueste)29. 06. 1901@\strich\emph{Lüsternheit (Predigt in der Wüste)} {[}29. 06. 1901{]}|pwv}}{\lemma{\textnormal{\emph{teutſche Worte}}}\Cendnote{\textnormal{Ed. Pötzl\pwindex{Poetzl, Eduard 17.03.1851 – 20.08.1914@\textsc{Pötzl, Eduard} (17.03.1851 – 20.08.1914), \emph{Schriftsteller, Journalist}|pwk}: \emph{Lüsternheit. (Predigt in der Wüste).}\pwindex{Poetzl, Eduard 17.03.1851 – 20.08.1914@\textsc{Pötzl, Eduard} (17.03.1851 – 20.08.1914), \emph{Schriftsteller, Journalist}!Luesternheit (Predigt in der Wueste)29. 06. 1901@\strich\emph{Lüsternheit (Predigt in der Wüste)} {[}29. 06. 1901{]}|pwk} In: \emph{Neues Wiener Tagblatt}\pwindex{?? Werk@Nicht ermittelte Verfasserinnen und Verfasser!Neues Wiener Tagblatt1867 – 1945@\emph{Neues Wiener Tagblatt} {[}1867 – 1945{]}|pwk}, Jg. 35, Nr. 176, 29. 6. 1901,
                     S. 1–2, ist eine schon im Titel erkennbare Replik auf Bahrs\pwindex{Bahr, Hermann 19.07.1863 – 15.01.1934@\textsc{Bahr, Hermann} (19.07.1863 – 15.01.1934), \emph{Schriftsteller, Kritiker}|pwk}{ }\emph{Erotisch}\pwindex{Bahr, Hermann 19.07.1863 – 15.01.1934@\textsc{Bahr, Hermann} (19.07.1863 – 15.01.1934), \emph{Schriftsteller, Kritiker}!Erotisch22. 06. 1901@\strich\emph{Erotisch} {[}22. 06. 1901{]}|pwk}.}}}\label{K_L01138_1h}! Das Herz geht einem auf, wenn
               man ſie lieſt. »Es iſt beſſer, das
                  gute zu heucheln als es durch offenkundige Frevel {\pb}aller Art von der
                  Tagesordnung gänzlich abſetzen\pwindex{Poetzl, Eduard 17.03.1851 – 20.08.1914@\textsc{Pötzl, Eduard} (17.03.1851 – 20.08.1914), \emph{Schriftsteller, Journalist}!Luesternheit (Predigt in der Wueste)29. 06. 1901@\strich\emph{Lüsternheit (Predigt in der Wüste)} {[}29. 06. 1901{]}|pwv}.« – »Es iſt immer noch moraliſcher im Geheimen zu ſündigen als auf
                  oeffentlichem Markte mit dem Laſter Arm in Arm zu gehen –\pwindex{Poetzl, Eduard 17.03.1851 – 20.08.1914@\textsc{Pötzl, Eduard} (17.03.1851 – 20.08.1914), \emph{Schriftsteller, Journalist}!Luesternheit (Predigt in der Wueste)29. 06. 1901@\strich\emph{Lüsternheit (Predigt in der Wüste)} {[}29. 06. 1901{]}|pwv}« »Die Geſa{\geminationm}theit darf
                  die Tugend nicht verachten, ſondern muſs ſie heilig halten und auf ihren Schild
                  erheben\pwindex{Poetzl, Eduard 17.03.1851 – 20.08.1914@\textsc{Pötzl, Eduard} (17.03.1851 – 20.08.1914), \emph{Schriftsteller, Journalist}!Luesternheit (Predigt in der Wueste)29. 06. 1901@\strich\emph{Lüsternheit (Predigt in der Wüste)} {[}29. 06. 1901{]}|pwv}« –\pend
           \pstart
           {\pb}– So ehrlich iſt die
               Heuchelei ſelten geweſen!\pend
           \pstart
           Leb wohl und ſei herzlich gegrüßt.{\\[\baselineskip]}Dein{\\[\baselineskip]}\spacefill\mbox{Arth Sch}\pend
           \leftskip=0em{}\pstart
           St Anton\oindex{St. Anton am Arlberg@\textbf{St. Anton am Arlberg}|pw}{ }1. 7. 109.\pend
           
         
         \endnumbering\mylabel{h}\end{ledgroupsized}  \newcommand{\dateiname}{L01138}\newcommand{\titel}{Arthur Schnitzler an Hermann Bahr, 1. 7. 1901}\newcommand{\editorInnen}{ Kurt Ifkovits,  Martin Anton Müller}%% latex-leseansicht-abspann.tex
%% Abspann für die Leseansicht.
%% Der Schalter \ifkorrekturansicht ist bereits durch den Vorspann gesetzt.

%% latex-abspann.tex
%% Gemeinsamer Abspann für Korrekturansicht und Leseansicht.
%% Setzt den Schalter \ifkorrekturansicht voraus (gesetzt in den
%% einbindenden Dateien latex-korrekturansicht-abspann.tex bzw.
%% latex-leseansicht-abspann.tex).
%% ---------------------------------------------------------------

\normalsize

% Das esempio-Environment wird nur in der Leseansicht benötigt
\ifkorrekturansicht\else
\newenvironment{esempio}[3]%
{
    \vspace{1.5ex}
    \rlap{\underline{#1}}
    \par
    \setlength{\parindent}{0cm}
    \nopagebreak
    \leftskip=#2cm
    \rightskip=#3cm
}
{
    \par
}
\fi

\doendnotes{C}
\bigskip
\vfill

\clearpage

\footnotesize

\ifkorrekturansicht
  \lohead{\textsc{register}}
\fi

% theindex-Environment neu definieren ohne reledmac
\makeatletter
\renewenvironment{theindex}{%
  \ifkorrekturansicht
    \section*{\indexname}%
  \else
    \subsubsection*{Index der erwähnten Entitäten}%
  \fi
  \setlength{\parindent}{0pt}%
  \setlength{\parskip}{0pt plus 0.3pt}%
  \let\item\@idxitem
}{%
  \ifkorrekturansicht\clearpage\fi
}
\makeatother

\IfFileExists{\jobname-pw.ind}{\input{\jobname-pw.ind}}{}

% Quellenangabe nur in der Leseansicht
\ifkorrekturansicht\else
% Fallback-Definitionen, falls die .tex-Datei \titel etc. nicht gesetzt hat
\providecommand{\titel}{}
\providecommand{\editorInnen}{}
\providecommand{\dateiname}{\jobname}

\vspace{3cm}

\vfill

\footnotesize
\textsc{Quelle}: \titel. Herausgegeben von {\editorInnen}. In: \emph{Arthur Schnitzler: Briefwechsel mit Autorinnen und Autoren}.
 Digitale Edition, https://schnitzler-briefe.acdh.oeaw.ac.at/{\dateiname}.html (Stand \today)
\fi

\end{document}


      