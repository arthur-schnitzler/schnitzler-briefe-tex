%% latex-leseansicht-vorspann.tex
%% Vorspann für die Leseansicht.
%% Lädt die gemeinsame Datei latex-vorspann.tex mit nicht gesetztem Schalter.

\newif\ifkorrekturansicht
\korrekturansichtfalse

\input{../tex-inputs/latex-vorspann}


\section[Edith Brandes an Arthur Schnitzler, 26. 6. 1901]{L01135 Edith Brandes an Arthur Schnitzler, 26. 6. 1901}
\nopagebreak\mylabel{L01135v}
\rehead{ }\normalsize\beginnumbering\briefempfaengerindex{Schnitzler, Arthur@\textsc{Schnitzler, Arthur}!zzzPhilipp, Edith@\emph{von Edith Philipp}!1901-06-262@{26. 6. 1901}|(be}
\toendnotes[C]{\smallbreak\pagebreak[2]}
\correspDesc{Versand  durch Edith Brandes am 26. 6. 1901 in Kopenhagen
\newline{}Weiterleitung  in Wien
\newline{}Erhalt  durch Arthur Schnitzler im Zeitraum [26. 6. 1901
                  – 30. 6. 1901?] \textbf{Ort fehlend} }\toendnotes[C]{\smallbreak}
\Standort{CUL, Schnitzler, B 17.}
\physDesc{Brief, 1 Blatt, 1 Seite, 373 Zeichen
\newline{}Handschrift: blaue Tinte, lateinische Kurrent
\newline{}Ordnung: mit Bleistift von unbekannter Hand nummeriert:
                                    »26« }\Standort{DLA, A:Schnitzler, HS.NZ85.1.2595.}
\physDesc{maschinenschriftliche Abschrift, 1 Blatt, 1 Seite, 373 Zeichen
\newline{}Schreibmaschine}
\buchAbdrucke{\weitereDrucke{Georg Brandes, Arthur Schnitzler: \emph{Ein Briefwechsel}. Herausgegeben von Kurt Bergel. Bern: \emph{Francke} 1956, S. 89.} }\toendnotes[C]{\smallbreak}
\pstart
           \raggedleft{}{\pb}Mittwoch, 26-6-1901\pend
           
\pstart{}Verehrter Herr Schnitzler!\pend\vspace{0.5em}
\pstart
           Ich kenne Sie ein wenig durch die Freundschaft die mein Vater\pwindex{Brandes, Georg 4.\,2.\,1842 Kopenhagen – 19.\,2.\,1927 ebd.@\textsc{Brandes, Georg} (4.\,2.\,1842 Kopenhagen – 19.\,2.\,1927 ebd.)|pwv} für Sie hegt; ich habe ausserdem alle
               Ihre Schriften gelesen. Recht sehr würden Sie mich verpflichten, wollten Sie mir für
               mein Album, worin eine Menge grosser Männer geschrieben haben ein Paar Zeilen
               senden.\pend
           
\pstart
           Ihre grosse Bewunderin{\\[\baselineskip]}\spacefill\mbox{Edith Brandes}\pend
           \leftskip=0em{}
\pstart
           \noindent{}Havnegade 55. Kopenhagen.\oindex{Havnegade@\textbf{Havnegade}, \emph{Straße}|pw}\pend
           \selectlanguage{ngerman}\endnumbering\briefempfaengerindex{Schnitzler, Arthur@\textsc{Schnitzler, Arthur}!zzzPhilipp, Edith@\emph{von Edith Philipp}!1901-06-262@{26. 6. 1901}|)be}\mylabel{L01135h}  \newcommand{\dateiname}{L01135}\newcommand{\titel}{Edith Brandes an Arthur Schnitzler, 26. 6. 1901}\newcommand{\editorInnen}{Martin Anton Müller und Gerd-Hermann Susen}%% latex-leseansicht-abspann.tex
%% Abspann für die Leseansicht.
%% Der Schalter \ifkorrekturansicht ist bereits durch den Vorspann gesetzt.

%% latex-abspann.tex
%% Gemeinsamer Abspann für Korrekturansicht und Leseansicht.
%% Setzt den Schalter \ifkorrekturansicht voraus (gesetzt in den
%% einbindenden Dateien latex-korrekturansicht-abspann.tex bzw.
%% latex-leseansicht-abspann.tex).
%% ---------------------------------------------------------------

\normalsize

% Das esempio-Environment wird nur in der Leseansicht benötigt
\ifkorrekturansicht\else
\newenvironment{esempio}[3]%
{
    \vspace{1.5ex}
    \rlap{\underline{#1}}
    \par
    \setlength{\parindent}{0cm}
    \nopagebreak
    \leftskip=#2cm
    \rightskip=#3cm
}
{
    \par
}
\fi

\doendnotes{C}
\bigskip
\vfill

\clearpage

\footnotesize

\ifkorrekturansicht
  \lohead{\textsc{register}}
\fi

% theindex-Environment neu definieren ohne reledmac
\makeatletter
\renewenvironment{theindex}{%
  \ifkorrekturansicht
    \section*{\indexname}%
  \else
    \subsubsection*{Index der erwähnten Entitäten}%
  \fi
  \setlength{\parindent}{0pt}%
  \setlength{\parskip}{0pt plus 0.3pt}%
  \let\item\@idxitem
}{%
  \ifkorrekturansicht\clearpage\fi
}
\makeatother

\IfFileExists{\jobname-pw.ind}{\input{\jobname-pw.ind}}{}

% Quellenangabe nur in der Leseansicht
\ifkorrekturansicht\else
% Fallback-Definitionen, falls die .tex-Datei \titel etc. nicht gesetzt hat
\providecommand{\titel}{}
\providecommand{\editorInnen}{}
\providecommand{\dateiname}{\jobname}

\vspace{3cm}

\vfill

\footnotesize
\textsc{Quelle}: \titel. Herausgegeben von {\editorInnen}. In: \emph{Arthur Schnitzler: Briefwechsel mit Autorinnen und Autoren}.
 Digitale Edition, https://schnitzler-briefe.acdh.oeaw.ac.at/{\dateiname}.html (Stand \today)
\fi

\end{document}


