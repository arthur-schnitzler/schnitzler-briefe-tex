%% latex-korrekturansicht-vorspann.tex
%% Vorspann für die Korrekturansicht.
%% Lädt die gemeinsame Datei latex-vorspann.tex mit gesetztem Schalter.

\newif\ifkorrekturansicht
\korrekturansichttrue

\input{../tex-inputs/latex-vorspann}


\section[ Paul Goldmann an Arthur Schnitzler, 5. 6. {[}1908{]}]{L03484 Paul Goldmann an Arthur Schnitzler, 5. 6. {[}1908{]}}
\nopagebreak\mylabel{L03484v}
\rehead{ }\normalsize\beginnumbering\briefempfaengerindex{Schnitzler, Arthur@\textsc{Schnitzler, Arthur}!zzzGoldmann, Paul@\emph{von Paul Goldmann}!1908-06-052@{5. 6. {[}1908{]}}|(be}
\toendnotes[C]{\smallbreak\pagebreak[2]}\Standort{DLA, A:Schnitzler, HS.NZ85.1.3176.}
\physDesc{Brief, 1 Blatt, 3 Seiten, 758 Zeichen
\newline{}Handschrift Schreibkraft: blaue Tinte, lateinische Kurrent
\newline{}Handschrift Paul Goldmann: blaue Tinte, deutsche Kurrent (\noindent{}Schlussformel und Unterschrift)
\newline{}Schnitzler: mit Bleistift »Goldma{\geminationn}\pwindex{Goldmann, Paul 31.01.1865 – 25.09.1935@\textsc{Goldmann, Paul} (31.01.1865 – 25.09.1935), \emph{Schriftsteller/Schriftstellerin, Journalist/Journalistin}|pw}« vermerkt }\toendnotes[C]{\smallbreak}
\pstart
           \raggedleft{}{\pb}\textcolor{gray}{\textbf{DESSAUERSTRASSE 19\oindex{Dessauer Strasse@\textbf{Dessauer Straße}, \emph{Straße (K.STR)}|pw}}}, d. 5. 6.\pend
           
\pstart{}Lieber Freund,\pend\vspace{0.5em}
\pstart
           Mit der Uebersendung Deines \label{K_L03484-1v}\edtext{Romans\pwindex{Weg ins Freie. Roman@\emph{Der Weg ins Freie. Roman}|pwv}}{\lemma{\textnormal{\emph{Romans}}}\Cendnote{\textnormal{Die Datierung auf das Jahr 1908 gelingt implizit: Goldmann\pwindex{Goldmann, Paul 31.01.1865 – 25.09.1935@\textsc{Goldmann, Paul} (31.01.1865 – 25.09.1935), \emph{Schriftsteller/Schriftstellerin, Journalist/Journalistin}|pwk} wohnte ab dem Frühjahr 1900 und
                  höchstens bis Anfang 1909 in der Dessauerstraße\oindex{Dessauer Strasse@\textbf{Dessauer Straße}, \emph{Straße (K.STR)}|pwk} (ab 1909 wird er
                  in Berlin\oindex{Berlin@\textbf{Berlin}, \emph{P.PPLC}|pwk}er Adressbüchern als am Schöneberger Ufer\oindex{Schoeneberger Ufer@\textbf{Schöneberger Ufer}, \emph{Straße (K.STR)}|pwk} wohnhaft verzeichnet). In
                  dieser Zeit erschienen nur zwei Romane: \emph{Frau
                     Bertha Garlan}\pwindex{Frau Bertha Garlan. Roman@\emph{Frau Bertha Garlan. Roman}|pwk} (1901) und \emph{Der
                     Weg ins Freie}\pwindex{Weg ins Freie. Roman@\emph{Der Weg ins Freie. Roman}|pwk} (1908). Nur für den zweiten Titel
                  schrieben Salten\pwindex{Salten, Felix 06.09.1869 – 08.10.1945@\textsc{Salten, Felix} (06.09.1869 – 08.10.1945), \emph{Schriftsteller/Schriftstellerin, Journalist/Journalistin, Chefredakteur/Chefredakteurin}|pwk} und Auernheimer\pwindex{Auernheimer, Raoul 15.04.1876 – 06.01.1948@\textsc{Auernheimer, Raoul} (15.04.1876 – 06.01.1948), \emph{Schriftsteller/Schriftstellerin, Journalist/Journalistin, Kritiker/Kritikerin}|pwk} Rezensionen.}}}\label{K_L03484-1} hast Du mir eine große
               Freude gemacht. Ich werde sofort die Lektüre beginnen und danke Dir einstweilen
               herzlichst für Buch\pwindex{Weg ins Freie. Roman@\emph{Der Weg ins Freie. Roman}|pwv} und
               Widmung.\pend
           
\pstart
           Von allen Seiten hoere ich hier\oindex{Berlin@\textbf{Berlin}, \emph{P.PPLC}|pwv}
               in den waermsten Aus{\pb}drücken von Deinem neuen
                  Werke\pwindex{Weg ins Freie. Roman@\emph{Der Weg ins Freie. Roman}|pwv} sprechen. Die Feuilletons\pwindex{Schnitzlers Wiener Roman@\emph{Schnitzlers Wiener Roman}|pwv}\pwindex{Weg ins Freie@\emph{Der Weg ins Freie}|pwv} von
                  \label{K_L03484-2v}\edtext{Salten\pwindex{Salten, Felix 06.09.1869 – 08.10.1945@\textsc{Salten, Felix} (06.09.1869 – 08.10.1945), \emph{Schriftsteller/Schriftstellerin, Journalist/Journalistin, Chefredakteur/Chefredakteurin}|pw}}{\lemma{\textnormal{\emph{Salten}}}\Cendnote{\textnormal{Felix Salten\pwindex{Salten, Felix 06.09.1869 – 08.10.1945@\textsc{Salten, Felix} (06.09.1869 – 08.10.1945), \emph{Schriftsteller/Schriftstellerin, Journalist/Journalistin, Chefredakteur/Chefredakteurin}|pwk}: \emph{Schnitzlers Wiener Roman}\pwindex{Schnitzlers Wiener Roman@\emph{Schnitzlers Wiener Roman}|pwk}. In: \emph{Die Zeit}\pwindex{Zeit@\emph{Die Zeit}|pwk}, Jg. 7, Nr. 2042, 30. 5. 1908, Morgenblatt, S. 1–2.}}}\label{K_L03484-2} und \label{K_L03484-3v}\edtext{Auernheimer\pwindex{Auernheimer, Raoul 15.04.1876 – 06.01.1948@\textsc{Auernheimer, Raoul} (15.04.1876 – 06.01.1948), \emph{Schriftsteller/Schriftstellerin, Journalist/Journalistin, Kritiker/Kritikerin}|pw}}{\lemma{\textnormal{\emph{Auernheimer}}}\Cendnote{\textnormal{Raoul Auernheimer\pwindex{Auernheimer, Raoul 15.04.1876 – 06.01.1948@\textsc{Auernheimer, Raoul} (15.04.1876 – 06.01.1948), \emph{Schriftsteller/Schriftstellerin, Journalist/Journalistin, Kritiker/Kritikerin}|pwk}: \emph{Der Weg ins Freie}\pwindex{Weg ins Freie@\emph{Der Weg ins Freie}|pwk}. In: \emph{Neue Freie Presse}\pwindex{Neue Freie Presse@\emph{Neue Freie Presse}|pwk}, Nr. 15.728, 3. 6. 1908, Morgenblatt, S. 1–3.}}}\label{K_L03484-3} haben das Buch\pwindex{Weg ins Freie. Roman@\emph{Der Weg ins Freie. Roman}|pwv} in Wien\oindex{Wien@\textbf{Wien}, \emph{A.ADM2}|pw} aufs beste eingeführt. Du scheinst also diesmal auf einen
               großen Erfolg rechnen zu dürfen und ich wünsche und hoffe, daß diese
               Erfolgs-Aussichten sich glänzend erfüllen moegen.\pend
           
\pstart
           Hoffentlich geht es Dir und {\pb}Deiner Frau\pwindex{Schnitzler, Olga 17.01.1882 – 13.01.1970@\textsc{Schnitzler, Olga} (17.01.1882 – 13.01.1970), \emph{Schauspieler/Schauspielerin, Sänger/Sängerin}|pwv} gut. Ich vermute, daß Ihr\pwindex{Schnitzler, Olga 17.01.1882 – 13.01.1970@\textsc{Schnitzler, Olga} (17.01.1882 – 13.01.1970), \emph{Schauspieler/Schauspielerin, Sänger/Sängerin}|pwv} von Eurer \label{K_L03484-4v}\edtext{Reise}{\lemma{\textnormal{\emph{Reise}}}\Cendnote{\textnormal{Siehe Paul Goldmann an Arthur Schnitzler, 8. 5. 1908.
               }}}\label{K_L03484-4} schon zurück seid, und denke mir, daß sie sehr interessant gewesen sein
               muß.\pend
           
\pstart
           Ich wünsche Euch\pwindex{Schnitzler, Olga 17.01.1882 – 13.01.1970@\textsc{Schnitzler, Olga} (17.01.1882 – 13.01.1970), \emph{Schauspieler/Schauspielerin, Sänger/Sängerin}|pwv} frohe
                  \label{K_L03484-5v}\edtext{Feiertage}{\lemma{\textnormal{\emph{Feiertage}}}\Cendnote{\textnormal{Am 7. 6. 1908 war Pfingstsonntag, tags darauf
                  Pfingstmontag.}}}\label{K_L03484-5} und bin mit vielen herzlichen Grüßen an Euch Beide\pwindex{Schnitzler, Olga 17.01.1882 – 13.01.1970@\textsc{Schnitzler, Olga} (17.01.1882 – 13.01.1970), \emph{Schauspieler/Schauspielerin, Sänger/Sängerin}|pwv}\pend
           
\pstart
           {[}hs. :{]} Dein {\\[\baselineskip]}\spacefill\mbox{Paul Goldmann.}\pend
           \leftskip=0em{}\selectlanguage{ngerman}\endnumbering\briefempfaengerindex{Schnitzler, Arthur@\textsc{Schnitzler, Arthur}!zzzGoldmann, Paul@\emph{von Paul Goldmann}!1908-06-052@{5. 6. {[}1908{]}}|)be}\mylabel{L03484h}  \normalsize

\doendnotes{C}
\bigskip
\vfill

\clearpage

\footnotesize

\lohead{\textsc{register}}

% Definiere theindex-Environment komplett neu ohne reledmac
\makeatletter
\renewenvironment{theindex}{%
  \section*{\indexname}%
  \setlength{\parindent}{0pt}%
  \setlength{\parskip}{0pt plus 0.3pt}%
  \let\item\@idxitem
}{%
  \clearpage
}
\makeatother

\IfFileExists{\jobname-pw.ind}{\input{\jobname-pw.ind}}{}

\end{document}

      