%% latex-korrekturansicht-vorspann.tex
%% Vorspann für die Korrekturansicht.
%% Lädt die gemeinsame Datei latex-vorspann.tex mit gesetztem Schalter.

\newif\ifkorrekturansicht
\korrekturansichttrue

\input{../tex-inputs/latex-vorspann}


\section[ Felix Salten an Arthur Schnitzler, 21. 4. 1922]{L03579 Felix Salten an Arthur Schnitzler, 21. 4. 1922}
\nopagebreak\mylabel{L03579v}
\rehead{ }\normalsize\beginnumbering\briefempfaengerindex{Schnitzler, Arthur@\textsc{Schnitzler, Arthur}!zzzSalten, Felix@\emph{von Felix Salten}!1922-04-211@{21. 4. 1922}|(be}
\toendnotes[C]{\smallbreak\pagebreak[2]}\Standort{CUL, Schnitzler, B 89, B 2.}
\physDesc{Bildpostkarte, 80 Zeichen
\newline{}Handschrift: schwarze Tinte, lateinische Kurrent
\newline{}Versand: Stempel: »\nobreak{}\oindex{Pompeji@\textbf{Pompeji}, \emph{S.ANS}|pwk}Pompei (Napoli), 21. 4. 22\nobreak{}«.  
\newline{}Ordnung: 1) mit Bleistift von Frieda Pollak\pwindex{Pollak, Frieda 08.12.1881 – 13.07.1937@\textsc{Pollak, Frieda} (08.12.1881 – 13.07.1937), \emph{Sekretär/Sekretärin}|pw} (?) mit
                                 dem Buchstaben »A« (Abgeschrieben/Abschrift)
                                 gekennzeichnet  2) mit Bleistift von unbekannter Hand nummeriert: »\substVorne{}\textsuperscript{292}\substDazwischen{}291a\substHinten{}«}\pstart{}{\pb}Austria\oindex{Oesterreich@\textbf{Österreich}, \emph{A.PCLI}|pw}\pend{}\pstart{}Herrn D\textsuperscript{r} Arthur Schnitzler\pend{}\pstart{}Wien\oindex{Wien@\textbf{Wien}, \emph{A.ADM2}|pw}\pend{}\pstart{}XVIII. Sternwartestraße 71\oindex{Sternwartestrasse 71@\textbf{Sternwartestraße 71}, \emph{Wohngebäude (K.WHS)}|pw}\pend{}{\bigskip}
\pstart
           \noindent{}\centering{}{\pb}\textcolor{gray}{\textbf{Pompei\oindex{Pompeji@\textbf{Pompeji}, \emph{S.ANS}|pw}. \begin{otherlanguage}{italian}Arco di trionfo del Tempio di Giove\oindex{Jupitertempel@\textbf{Jupitertempel}, \emph{Museum (K.MUS)}|pw}\end{otherlanguage}.}}\pend
           
\pstart
           \centering{}\textcolor{gray}{\textbf{Der Triumphbogen des Jupitertempels.}}\oindex{Jupitertempel@\textbf{Jupitertempel}, \emph{Museum (K.MUS)}|pw}\pend
           
\pstart
           \centering{}\textcolor{gray}{\textbf{\begin{otherlanguage}{french}L’arc de triomphe du temple de
                        Jupiter\end{otherlanguage}.}}\oindex{Jupitertempel@\textbf{Jupitertempel}, \emph{Museum (K.MUS)}|pw}\pend
           
\pstart
           \centering{}\textcolor{gray}{\textbf{\begin{otherlanguage}{english}Temple of Jupiter. Triumphal arch\end{otherlanguage}.}}\oindex{Jupitertempel@\textbf{Jupitertempel}, \emph{Museum (K.MUS)}|pw}\pend
           \vspace{1em}
\pstart
           {\pb}Herzlichst {\\[\baselineskip]}Ihr
                  {\\[\baselineskip]}\spacefill\mbox{F. S.}\pend
           \leftskip=0em{}\selectlanguage{ngerman}\endnumbering\briefempfaengerindex{Schnitzler, Arthur@\textsc{Schnitzler, Arthur}!zzzSalten, Felix@\emph{von Felix Salten}!1922-04-211@{21. 4. 1922}|)be}\mylabel{L03579h}  \normalsize

\doendnotes{C}
\bigskip
\vfill

\clearpage

\footnotesize

\lohead{\textsc{register}}

% Definiere theindex-Environment komplett neu ohne reledmac
\makeatletter
\renewenvironment{theindex}{%
  \section*{\indexname}%
  \setlength{\parindent}{0pt}%
  \setlength{\parskip}{0pt plus 0.3pt}%
  \let\item\@idxitem
}{%
  \clearpage
}
\makeatother

\IfFileExists{\jobname-pw.ind}{\input{\jobname-pw.ind}}{}

\end{document}

      