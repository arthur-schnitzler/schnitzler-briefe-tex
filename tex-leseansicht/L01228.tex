%% latex-korrekturansicht-vorspann.tex
%% Vorspann für die Korrekturansicht.
%% Lädt die gemeinsame Datei latex-vorspann.tex mit gesetztem Schalter.

\newif\ifkorrekturansicht
\korrekturansichttrue

\input{../tex-inputs/latex-vorspann}


\section[Hugo von Hofmannsthal und Arthur Schnitzler an Richard Beer-Hofmann, 3. 7. 1902]{L01228 Hugo von Hofmannsthal und Arthur Schnitzler an Richard Beer-Hofmann,
               3. 7. 1902}
\nopagebreak\mylabel{L01228v}
\rehead{ }\normalsize\beginnumbering\briefempfaengerindex{Beer-Hofmann, Richard@\textsc{Beer-Hofmann, Richard}!zzzSchnitzler, Arthur@\emph{von Arthur Schnitzler}!1902-07-032@{3. 7. 1902}|(be}\briefempfaengerindex{Beer-Hofmann, Richard@\textsc{Beer-Hofmann, Richard}!zzzHofmannsthal, Hugo von@\emph{von Hugo von Hofmannsthal}!1902-07-032@{3. 7. 1902}|(be}
\toendnotes[C]{\smallbreak\pagebreak[2]}\Standort{YCGL, MSS 31.}
\physDesc{Bildpostkarte, 119 Zeichen
\newline{}Handschrift Hugo von Hofmannsthal: Bleistift, deutsche Kurrent
\newline{}Handschrift Arthur Schnitzler: Bleistift, deutsche Kurrent
\newline{}Versand: 1) Stempel: »\nobreak{}\oindex{Matrei am Brenner@\textbf{Matrei am Brenner}, \emph{A.ADM3}|pwk}Deutsch-\textcolor{gray}{Matrei}, 3/{[}7 1902{]}\nobreak{}«.   2) Stempel: »\nobreak{}\oindex{Rodaun@\textbf{Rodaun}, \emph{A.ADM4}|pwk}\textcolor{gray}{Rod}aun, 4. 7. 02, 9–12V\nobreak{}«. 
\newline{}Ordnung: mit Bleistift von unbekannter Hand datiert: »3. 7.« }\toendnotes[C]{\smallbreak}\pstart{}{\pb}Hrn \strikeout{\textsc{Gustav Schwarzkopf}\pwindex{Schwarzkopf, Gustav 07.11.1853 – 13.11.1939@\textsc{Schwarzkopf, Gustav} (07.11.1853 – 13.11.1939), \emph{Schriftsteller/Schriftstellerin}|pw}}\pend{}\pstart{}\textsc{Dr Richard Beer-Hofmann}\pend{}\pstart{}\textsc{Rodaun bei Wien}\oindex{Rodaun@\textbf{Rodaun}, \emph{A.ADM4}|pw}\pend{}\pstart{}\textsc{Liesingerstr 2}\oindex{Liesingerstrasse@\textbf{Liesingerstraße}, \emph{Straße (K.STR)}|pw}. \pend{}{\bigskip}
\pstart
           \noindent{}\centering{}{\pb}\textcolor{gray}{\textbf{MATREI\oindex{Matrei am Brenner@\textbf{Matrei am Brenner}, \emph{A.ADM3}|pw}.}}\pend
           \vspace{1em}
\pstart
           {\pb}3. 7. 902.\pend
           \vspace{0.5em}
\pstart
           Herzlichen Gruß!\pend
           \pstart Ihr \spacefill\mbox{Arthur}\pend{}\selectlanguage{ngerman}\vspace{1em}
\pstart
           \noindent{}{[}hs. :{]} \label{K_L01228-1v}\edtext{Jajaſibär}{\lemma{\textnormal{\emph{Jajaſibär}}}\Cendnote{\textnormal{Eventuell: »Ja, ja, Sie Bär«. »Bär« bzw. in der Mehrzahl »die Bären« war eine im Freundeskreis
                  geläufige Art, auf Beer-Hofmann\pwindex{Beer-Hofmann, Richard 1866-07-11 – 1945-09-26@\textsc{Beer-Hofmann, Richard} (1866-07-11 – 1945-09-26), \emph{Schriftsteller/Schriftstellerin}|pwk} und seine Familie Bezug zu nehmen.}}}\label{K_L01228-1}!\pend
           \selectlanguage{ngerman}\endnumbering\briefempfaengerindex{Beer-Hofmann, Richard@\textsc{Beer-Hofmann, Richard}!zzzSchnitzler, Arthur@\emph{von Arthur Schnitzler}!1902-07-032@{3. 7. 1902}|)be}\briefempfaengerindex{Beer-Hofmann, Richard@\textsc{Beer-Hofmann, Richard}!zzzHofmannsthal, Hugo von@\emph{von Hugo von Hofmannsthal}!1902-07-032@{3. 7. 1902}|)be}\mylabel{L01228h}  \normalsize

\doendnotes{C}
\bigskip
\vfill

\clearpage

\footnotesize

\lohead{\textsc{register}}

% Definiere theindex-Environment komplett neu ohne reledmac
\makeatletter
\renewenvironment{theindex}{%
  \section*{\indexname}%
  \setlength{\parindent}{0pt}%
  \setlength{\parskip}{0pt plus 0.3pt}%
  \let\item\@idxitem
}{%
  \clearpage
}
\makeatother

\IfFileExists{\jobname-pw.ind}{\input{\jobname-pw.ind}}{}

\end{document}

      