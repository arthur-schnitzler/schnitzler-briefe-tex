%% latex-leseansicht-vorspann.tex
%% Vorspann für die Leseansicht.
%% Lädt die gemeinsame Datei latex-vorspann.tex mit nicht gesetztem Schalter.

\newif\ifkorrekturansicht
\korrekturansichtfalse

\input{../tex-inputs/latex-vorspann}


\section[Hugo von Hofmannsthal und Arthur Schnitzler an Richard Beer-Hofmann, 3. 7. 1902]{L01228 Hugo von Hofmannsthal und Arthur Schnitzler an Richard Beer-Hofmann, 3. 7. 1902}
\nopagebreak\mylabel{L01228v}
\rehead{ }\normalsize\beginnumbering\briefempfaengerindex{Beer-Hofmann, Richard@\textsc{Beer-Hofmann, Richard}!zzzSchnitzler, Arthur@\emph{von Arthur Schnitzler}!1902-07-032@{3. 7. 1902}|(be}\briefempfaengerindex{Beer-Hofmann, Richard@\textsc{Beer-Hofmann, Richard}!zzzHofmannsthal, Hugo von@\emph{von Hugo von Hofmannsthal}!1902-07-032@{3. 7. 1902}|(be}
\toendnotes[C]{\smallbreak\pagebreak[2]}
\correspDesc{Versand  durch Hugo von Hofmannsthal, Arthur Schnitzler am 3. 7. 1902 in Matrei am Brenner
\newline{}Erhalt  durch Richard Beer-Hofmann am 4. 7. 1902 in Rodaun}\toendnotes[C]{\smallbreak}
\Standort{YCGL, MSS 31.}
\physDesc{Bildpostkarte, 119 Zeichen
\newline{}Handschrift Hugo von Hofmannsthal: Bleistift, deutsche Kurrent
\newline{}Handschrift Arthur Schnitzler: Bleistift, deutsche Kurrent
\newline{}Versand: 1) Stempel: »\nobreak{}\oindex{Matrei am Brenner@\textbf{Matrei am Brenner}, \emph{Verwaltungsgebiet}|pwk}Deutsch-\textcolor{gray}{Matrei}, 3/{[}7 1902{]}\nobreak{}«.   2) Stempel: »\nobreak{}\oindex{Wien@\textbf{Wien}!XXIII., Liesing@\textbf{XXIII., Liesing}!Rodaun@\textbf{Rodaun}, \emph{Region}|pwk}\textcolor{gray}{Rod}aun, 4. 7. 02, 9–12V\nobreak{}«. 
\newline{}Ordnung: mit Bleistift von unbekannter Hand datiert: »3. 7.« }\toendnotes[C]{\smallbreak}\pstart{}{\pb}Hrn \strikeout{\textsc{Gustav Schwarzkopf}\pwindex{Schwarzkopf, Gustav 7.\,11.\,1853 Wien – 13.\,11.\,1939 ebd.@\textsc{Schwarzkopf, Gustav} (7.\,11.\,1853 Wien – 13.\,11.\,1939 ebd.), \emph{Schriftsteller}|pw}}\pend{}\pstart{}\textsc{Dr Richard Beer-Hofmann}\pend{}\pstart{}\textsc{Rodaun bei Wien}\oindex{Wien@\textbf{Wien}!XXIII., Liesing@\textbf{XXIII., Liesing}!Rodaun@\textbf{Rodaun}, \emph{Region}|pw}\pend{}\pstart{}\textsc{Liesingerstr 2}\oindex{Liesingerstraße@\textbf{Liesingerstraße}, \emph{Straße}|pw}. \pend{}{\bigskip}
\pstart
           \noindent{}\centering{}{\pb}\textcolor{gray}{\textbf{MATREI\oindex{Matrei am Brenner@\textbf{Matrei am Brenner}, \emph{Verwaltungsgebiet}|pw}.}}\pend
           \vspace{1em}
\pstart
           {\pb}3. 7. 902.\pend
           \vspace{0.5em}
\pstart
           Herzlichen Gruß!\pend
           \pstart Ihr \spacefill\mbox{Arthur}\pend{}\selectlanguage{ngerman}\vspace{1em}
\pstart
           \noindent{}{[}hs. Hofmannsthal:{]} \label{K_L01228-1v}\edtext{Jajaſibär}{\lemma{\textnormal{\emph{Jajasibär}}}\Cendnote{\textnormal{Eventuell: »Ja, ja, Sie Bär«. »Bär« bzw. in der Mehrzahl »die Bären« war eine im Freundeskreis
                  geläufige Art, auf Beer-Hofmann\pwindex{Beer-Hofmann, Richard 11.\,7.\,1866 Wien – 26.\,9.\,1945 New York City@\textsc{Beer-Hofmann, Richard} (11.\,7.\,1866 Wien – 26.\,9.\,1945 New York City), \emph{Schriftsteller}|pwk} und seine Familie Bezug zu nehmen.}}}\label{K_L01228-1}!\pend
           \selectlanguage{ngerman}\endnumbering\briefempfaengerindex{Beer-Hofmann, Richard@\textsc{Beer-Hofmann, Richard}!zzzSchnitzler, Arthur@\emph{von Arthur Schnitzler}!1902-07-032@{3. 7. 1902}|)be}\briefempfaengerindex{Beer-Hofmann, Richard@\textsc{Beer-Hofmann, Richard}!zzzHofmannsthal, Hugo von@\emph{von Hugo von Hofmannsthal}!1902-07-032@{3. 7. 1902}|)be}\mylabel{L01228h}  \newcommand{\dateiname}{L01228}\newcommand{\titel}{Hugo von Hofmannsthal und Arthur Schnitzler an Richard Beer-Hofmann, 3. 7. 1902}\newcommand{\editorInnen}{Herausgegeben von Martin Anton Müller}%% latex-leseansicht-abspann.tex
%% Abspann für die Leseansicht.
%% Der Schalter \ifkorrekturansicht ist bereits durch den Vorspann gesetzt.

%% latex-abspann.tex
%% Gemeinsamer Abspann für Korrekturansicht und Leseansicht.
%% Setzt den Schalter \ifkorrekturansicht voraus (gesetzt in den
%% einbindenden Dateien latex-korrekturansicht-abspann.tex bzw.
%% latex-leseansicht-abspann.tex).
%% ---------------------------------------------------------------

\normalsize

% Das esempio-Environment wird nur in der Leseansicht benötigt
\ifkorrekturansicht\else
\newenvironment{esempio}[3]%
{
    \vspace{1.5ex}
    \rlap{\underline{#1}}
    \par
    \setlength{\parindent}{0cm}
    \nopagebreak
    \leftskip=#2cm
    \rightskip=#3cm
}
{
    \par
}
\fi

\doendnotes{C}
\bigskip
\vfill

\clearpage

\footnotesize

\ifkorrekturansicht
  \lohead{\textsc{register}}
\fi

% theindex-Environment neu definieren ohne reledmac
\makeatletter
\renewenvironment{theindex}{%
  \ifkorrekturansicht
    \section*{\indexname}%
  \else
    \subsubsection*{Index der erwähnten Entitäten}%
  \fi
  \setlength{\parindent}{0pt}%
  \setlength{\parskip}{0pt plus 0.3pt}%
  \let\item\@idxitem
}{%
  \ifkorrekturansicht\clearpage\fi
}
\makeatother

\IfFileExists{\jobname-pw.ind}{\input{\jobname-pw.ind}}{}

% Quellenangabe nur in der Leseansicht
\ifkorrekturansicht\else
% Fallback-Definitionen, falls die .tex-Datei \titel etc. nicht gesetzt hat
\providecommand{\titel}{}
\providecommand{\editorInnen}{}
\providecommand{\dateiname}{\jobname}

\vspace{3cm}

\vfill

\footnotesize
\textsc{Quelle}: \titel. Herausgegeben von {\editorInnen}. In: \emph{Arthur Schnitzler: Briefwechsel mit Autorinnen und Autoren}.
 Digitale Edition, https://schnitzler-briefe.acdh.oeaw.ac.at/{\dateiname}.html (Stand \today)
\fi

\end{document}


