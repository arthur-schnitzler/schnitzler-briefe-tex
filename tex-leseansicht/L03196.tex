%% latex-korrekturansicht-vorspann.tex
%% Vorspann für die Korrekturansicht.
%% Lädt die gemeinsame Datei latex-vorspann.tex mit gesetztem Schalter.

\newif\ifkorrekturansicht
\korrekturansichttrue

\input{../tex-inputs/latex-vorspann}


\section[ Paul Goldmann an Arthur Schnitzler, 2. 2. {[}1902{]}]{L03196 Paul Goldmann an Arthur Schnitzler, 2. 2. {[}1902{]}}
\nopagebreak\mylabel{L03196v}
\rehead{ }\normalsize\beginnumbering\briefempfaengerindex{Schnitzler, Arthur@\textsc{Schnitzler, Arthur}!zzzGoldmann, Paul@\emph{von Paul Goldmann}!1902-02-021@{2. 2. {[}1902{]}}|(be}
\toendnotes[C]{\smallbreak\pagebreak[2]}\Standort{DLA, A:Schnitzler, HS.NZ85.1.3172.}
\physDesc{Brief, 1 Blatt, 4 Seiten, 1878 Zeichen
\newline{}Handschrift: blaue Tinte, deutsche Kurrent
\newline{}Beilage: ein handschriftlicher Brief, schwarze Tinte, deutsche
                                 Kurrentschrift, beschnitten und eingeklebt 
\newline{}Schnitzler: 1) mit Bleistift das Jahr »902« vermerkt  2) mit rotem Buntstift fünf Unterstreichungen}\toendnotes[C]{\smallbreak}
\pstart
           \raggedleft{}{\pb}\textcolor{gray}{\textbf{DESSAUERSTRASSE 19}}\oindex{Dessauer Strasse@\textbf{Dessauer Straße}, \emph{Straße (K.STR)}|pw}\pend
           
\pstart
           Berlin\oindex{Berlin@\textbf{Berlin}, \emph{P.PPLC}|pw}, 2. Februar.\pend
           
\pstart{}Mein lieber Freund,\pend\vspace{0.5em}
\pstart
           Die Regelung der \label{K_L03196-1v}\edtext{Landaufenthalts-Frage}{\lemma{\textnormal{\emph{Landaufenthalts-Frage}}}\Cendnote{\textnormal{Siehe Paul Goldmann an Arthur Schnitzler, 14. 1. [1902].
               }}}\label{K_L03196-1} freut mich ſehr. »Kurhaus in Mödling\oindex{Hotel Kursalon Moedling@\textbf{Hotel Kursalon Mödling}, \emph{Hotel (K.HTL)}|pw}«
               klingt vielverſprechend. Ich wünſchte, ich könnte auch hin. Ich bin ſchwer
               überarbeitet und leide\strikeout{t} ſeit einer Woche
               ununterbrochen an Kopfſchmerzen.\pend
           
\pstart
           \substVorne{}\textsuperscript{St}\substDazwischen{}D\substHinten{}ie Vorſtellungen von »\label{K_L03196-2v}\edtext{Lebendige Stunden\pwindex{Lebendige Stunden. Vier Einakter@\emph{Lebendige Stunden. Vier Einakter}|pw}}{\lemma{\textnormal{\emph{Lebendige Stunden}}}\Cendnote{\textnormal{im Deutschen Theater Berlin\oindex{Deutsches Theater Berlin@\textbf{Deutsches Theater Berlin}, \emph{Theater (K.THE)}|pwk}}}}\label{K_L03196-2}« ſollen ſtets ausverkauft ſein. Ich freue mich ſehr darüber, daß Dir Deine
               Arbeit auch Geld bringt. Du kannſt es brauchen. Wie hat ſich \textsc{Schlenther\pwindex{Schlenther, Paul 20.08.1854 – 30.04.1916@\textsc{Schlenther, Paul} (20.08.1854 – 30.04.1916), \emph{Schriftsteller/Schriftstellerin, Kritiker/Kritikerin, Theaterleiter/Theaterleiterin}|pw}} verhalten?\pend
           
\pstart
           \textsc{Sudermanns\pwindex{Sudermann, Hermann 30.09.1857 – 21.11.1928@\textsc{Sudermann, Hermann} (30.09.1857 – 21.11.1928), \emph{Schriftsteller/Schriftstellerin}|pw}} neues Stück\pwindex{Es lebe das Leben@\emph{Es lebe das Leben}|pwv} iſt elend. {\pb}\introOben{}In der Art von \textsc{Philippi\pwindex{Philippi, Felix 05.08.1851 – 23.11.1921@\textsc{Philippi, Felix} (05.08.1851 – 23.11.1921), \emph{Schriftsteller/Schriftstellerin}|pw}}. Nur macht es \textsc{Philippi\pwindex{Philippi, Felix 05.08.1851 – 23.11.1921@\textsc{Philippi, Felix} (05.08.1851 – 23.11.1921), \emph{Schriftsteller/Schriftstellerin}|pw}} beſſer.\introOben{} Ich konnte nur ganz kurz darüber \label{K_L03196-3v}\edtext{telegraphiren\pwindex{Theater- und Kunstnachrichten. [Burgtheater.] [Es lebe das Leben]@\emph{Theater- und Kunstnachrichten. [Burgtheater.] [Es lebe das Leben]}|pwv}}{\lemma{\textnormal{\emph{telegraphiren}}}\Cendnote{\textnormal{[Paul Goldmann\pwindex{Goldmann, Paul 31.01.1865 – 25.09.1935@\textsc{Goldmann, Paul} (31.01.1865 – 25.09.1935), \emph{Schriftsteller/Schriftstellerin, Journalist/Journalistin}|pwk}]: \emph{Theater- und Kunstnachrichten. [Burgtheater]}\pwindex{Theater- und Kunstnachrichten. [Burgtheater.] [Es lebe das Leben]@\emph{Theater- und Kunstnachrichten. [Burgtheater.] [Es lebe das Leben]}|pwk}. In: \emph{Neue Freie Presse}\pwindex{Neue Freie Presse@\emph{Neue Freie Presse}|pwk}, Nr. 13.455, 8. 2. 1902, Morgenblatt, S. 7.}}}\label{K_L03196-3}, weil
                  \strikeout{\textcolor{gray}{d}} die Vorſtellung erſt nach elf aus war, und ein Feuilleton darüber
               zu ſchreiben, wurde mir telegraphiſch unterſagt. Herrn \textsc{Wittmanns\pwindex{Wittmann, Hugo 16.10.1839 – 06.02.1923@\textsc{Wittmann, Hugo} (16.10.1839 – 06.02.1923), \emph{Schriftsteller/Schriftstellerin, Journalist/Journalistin}|pw}} kritiſcher \label{K_L03196-4v}\edtext{Würdigung\pwindex{Burgtheater. (Zum erstenmale: »Es lebe das Leben«, Drama in fuenf Acten von Hermann Sudermann.)@\emph{Burgtheater. (Zum erstenmale: »Es lebe das Leben«, Drama in fünf Acten von Hermann Sudermann.)}|pwv}}{\lemma{\textnormal{\emph{Würdigung}}}\Cendnote{\textnormal{W.\pwindex{Wittmann, Hugo 16.10.1839 – 06.02.1923@\textsc{Wittmann, Hugo} (16.10.1839 – 06.02.1923), \emph{Schriftsteller/Schriftstellerin, Journalist/Journalistin}|pwkv} [ = Hugo Wittmann\pwindex{Wittmann, Hugo 16.10.1839 – 06.02.1923@\textsc{Wittmann, Hugo} (16.10.1839 – 06.02.1923), \emph{Schriftsteller/Schriftstellerin, Journalist/Journalistin}|pwk}]: \emph{Burgtheater. (Zum erstenmale: »Es lebe das Leben«, Drama in
                        fünf Acten von Hermann Sudermann)}\pwindex{Burgtheater. (Zum erstenmale: »Es lebe das Leben«, Drama in fuenf Acten von Hermann Sudermann.)@\emph{Burgtheater. (Zum erstenmale: »Es lebe das Leben«, Drama in fünf Acten von Hermann Sudermann.)}|pwk}. In: \emph{Neue Freie Presse}\pwindex{Neue Freie Presse@\emph{Neue Freie Presse}|pwk}, Nr. 13.456, 9. 2. 1902, Morgenblatt, S. 1–3.}}}\label{K_L03196-4} darf ein armer
               Reporter, wie ich bin, nicht vorgreifen.\pend
           
\pstart
           Dank für die B\substVorne{}\textsuperscript{uch}\substDazwischen{}ücher\substHinten{}empfehlungen. Ich leſe nach wie vor mit Genuß die \textsc{Shakespeare\pwindex{Shakespeare, William 23.4.1564? – 03.05.1616@\textsc{Shakespeare, William} (23.4.1564? – 03.05.1616), \emph{Schauspieler/Schauspielerin, Dramatiker/Dramatikerin}|pw}}-Biographie\pwindex{William Shakespeare@\emph{William Shakespeare}|pw} von \textsc{Brandes\pwindex{Brandes, Georg 04.02.1842 – 19.02.1927@\textsc{Brandes, Georg} (04.02.1842 – 19.02.1927)|pw}}.\pend
           
\pstart
           \textsc{Brandes\pwindex{Brandes, Georg 04.02.1842 – 19.02.1927@\textsc{Brandes, Georg} (04.02.1842 – 19.02.1927)|pw}} iſt hier\oindex{Berlin@\textbf{Berlin}, \emph{P.PPLC}|pwv}, läßt ſich aber
               bei mir nicht ſehen. Übermorgen feiert \introOben{}er\introOben{} ſeinen 60. Geburtstag. Vergiß nicht, ihm zu \label{K_L03196-5v}\edtext{gratuliren}{\lemma{\textnormal{\emph{gratuliren}}}\Cendnote{\textnormal{kein entsprechendes Korrespondenzstück
               überliefert}}}\label{K_L03196-5}.\pend
           
\pstart
           {\pb}Mit \textsc{Singer\pwindex{Singer, Isidor 16.01.1857 – 08.12.1927@\textsc{Singer, Isidor} (16.01.1857 – 08.12.1927), \emph{Journalist/Journalistin, Herausgeber/Herausgeberin, Soziologe/Soziologin}|pw}} ſprich’, bitte, einſtweilen nicht. \textsc{Kanner\pwindex{Kanner, Heinrich 09.11.1864 – 15.02.1930@\textsc{Kanner, Heinrich} (09.11.1864 – 15.02.1930), \emph{Herausgeber/Herausgeberin, Publizist/Publizistin}|pw}} ſoll bald wieder hier\oindex{Berlin@\textbf{Berlin}, \emph{P.PPLC}|pwv}herkommen, und ich werde verſuchen, ihn \label{K_L03196-6v}\edtext{zur Rede zu ſtellen}{\lemma{\textnormal{\emph{zur Rede zu ſtellen}}}\Cendnote{\textnormal{Siehe Paul Goldmann an Arthur Schnitzler, 25. 1. [1902].
               }}}\label{K_L03196-6}.\pend
           
\pstart
           An \textsc{Mauthners\pwindex{Mauthner, Fritz 1849-11-20 – 1923-06-29@\textsc{Mauthner, Fritz} (1849-11-20 – 1923-06-29), \emph{Schriftsteller/Schriftstellerin, Journalist/Journalistin, Philosoph/Philosophin}|pw}} Stelle ſoll mein \label{K_L03196-7v}\edtext{Onkel\pwindex{Mamroth, Fedor 21.02.1851 – 25.06.1907@\textsc{Mamroth, Fedor} (21.02.1851 – 25.06.1907), \emph{Journalist/Journalistin, Kritiker/Kritikerin}|pwv} zum Berliner Tageblatt\orgindex{Berliner Tageblatt@Berliner Tageblatt|pw}}{\lemma{\textnormal{\emph{Onkel … Tageblatt}}}\Cendnote{\textnormal{nicht belegbar}}}\label{K_L03196-7} kommen. An mich
               denkt ſelbſtverſtändlich Niemand. Ich bin nicht literariſch.\pend
           
\pstart
           Anbei der \label{K_L03196-8v}\edtext{Brief von \textsc{Herzl\pwindex{Herzl, Theodor 1860-05-02 – 1904-07-03@\textsc{Herzl, Theodor} (1860-05-02 – 1904-07-03), \emph{Schriftsteller/Schriftstellerin, Journalist/Journalistin}|pw}}}{\lemma{\textnormal{\emph{Brief von Herzl}}}\Cendnote{\textnormal{Siehe Paul Goldmann an Arthur Schnitzler, 25. 1. [1902].
               }}}\label{K_L03196-8}. Sende ihn mir, bitte, gelegentlich zurück.\pend
           
\pstart
           \strikeout{»Sie\pwindex{Rottenberg, Theodore 1875-09-07 – 1945-04-05@\textsc{Rottenberg, Theodore} (1875-09-07 – 1945-04-05)|pw}}{ }\label{K_L03196-9v}\edtext{»Sie\pwindex{Rottenberg, Theodore 1875-09-07 – 1945-04-05@\textsc{Rottenberg, Theodore} (1875-09-07 – 1945-04-05)|pwv}«}{\lemma{\textnormal{\emph{»Sie«}}}\Cendnote{\textnormal{mit großer
                  Wahrscheinlichkeit Theodore Rottenberg\pwindex{Rottenberg, Theodore 1875-09-07 – 1945-04-05@\textsc{Rottenberg, Theodore} (1875-09-07 – 1945-04-05)|pwk}, mit
                  der Goldmann\pwindex{Goldmann, Paul 31.01.1865 – 25.09.1935@\textsc{Goldmann, Paul} (31.01.1865 – 25.09.1935), \emph{Schriftsteller/Schriftstellerin, Journalist/Journalistin}|pwk} seit 1899 intim war, siehe Paul Goldmann an Arthur Schnitzler, 8. 10. [1899].}}}\label{K_L03196-9} (aus Frankfurt\oindex{Frankfurt am Main@\textbf{Frankfurt am Main}, \emph{P.PPLA3}|pw}) ſchreibt Folgendes\substVorne{}\textsuperscript{,}\substDazwischen{}:\substHinten{}\pend
           {\vspace{1\baselineskip}}
\pstart
           {\pb}{[}hs. :{]} Dein \textsc{Schnitzler}-Feuilleton\pwindex{Berliner Theater. (»Lebendige Stunden« von Arthur Schnitzler.)@\emph{Berliner Theater. (»Lebendige Stunden« von Arthur Schnitzler.)}|pwv}, womit er doch wohl
               einverſtanden ſein wird, iſt fein, fein, mein Liebſter. Nur die \label{K_L03196-10v}\edtext{Epiſoden-Sache}{\lemma{\textnormal{\emph{Epiſoden-Sache}}}\Cendnote{\textnormal{Siehe Paul Goldmann an Arthur Schnitzler, 25. 1. [1902].
               }}}\label{K_L03196-10} mißfällt mir. Es \uline{giebt} Männer {\kaufmannsund} viele tauſend Frauen, die von der Liebe leben. Bei \textsc{Schnitzler} wird Kunſt {\kaufmannsund} Liebe
               ſicherlich i{\geminationm}er eins bleiben; halb Frauenpoſe {\kaufmannsund} halb Öſterreich\oindex{Oesterreich@\textbf{Österreich}, \emph{A.PCLI}|pw}er iſt er nun einmal. Die wahre, erhabene \label{K_L03196-11v}\edtext{{[}»{]}deutſch\oindex{Deutschland@\textbf{Deutschland}, \emph{A.PCLI}|pwv}e Männlichkeit«}{\lemma{\textnormal{\emph{»deutſche Männlichkeit«}}}\Cendnote{\textnormal{Bezug auf die erwähnte
                     »Epiſoden-Sache«: Schnitzler habe sich vom Thema der Liebe zu lösen und »das starke Werk seiner Mannesjahre\pwindex{Berliner Theater. (»Lebendige Stunden« von Arthur Schnitzler.)@\emph{Berliner Theater. (»Lebendige Stunden« von Arthur Schnitzler.)}|pwkv}« zu
                  schreiben. Paul Goldmann\pwindex{Goldmann, Paul 31.01.1865 – 25.09.1935@\textsc{Goldmann, Paul} (31.01.1865 – 25.09.1935), \emph{Schriftsteller/Schriftstellerin, Journalist/Journalistin}|pwk}: \emph{Berliner Theater. (»Lebendige Stunden« von Arthur
                        Schnitzler)}\pwindex{Berliner Theater. (»Lebendige Stunden« von Arthur Schnitzler.)@\emph{Berliner Theater. (»Lebendige Stunden« von Arthur Schnitzler.)}|pwk}. In: \emph{Neue Freie
                        Presse}\pwindex{Neue Freie Presse@\emph{Neue Freie Presse}|pwk}, Nr. 13.438, 22. 1. 1902,
                     Morgenblatt, S. 1–4, hier: S. 4.}}}\label{K_L03196-11} kann ich mir von ihm eben ſo
               wenig denken wie von M. \textsc{Donnay}\pwindex{Donnay, Maurice 12.10.1859 – 31.03.1945@\textsc{Donnay, Maurice} (12.10.1859 – 31.03.1945), \emph{Schriftsteller/Schriftstellerin}|pw} z. B.\pend
           {\vspace{1\baselineskip}}
\pstart
           {[}hs. :{]} Viele treue Grüße, mein lieber Freund, Dir und den Mädels\pwindex{Schnitzler, Olga 17.01.1882 – 13.01.1970@\textsc{Schnitzler, Olga} (17.01.1882 – 13.01.1970), \emph{Schauspieler/Schauspielerin, Sänger/Sängerin}|pwv}\pwindex{Steinrueck, Elisabeth 19.11.1885 – 07.04.1920@\textsc{Steinrück, Elisabeth} (19.11.1885 – 07.04.1920)|pwv}. {\\[\baselineskip]}Dein {\\[\baselineskip]}\spacefill\mbox{Paul Goldmann}\pend
           \leftskip=0em{}\selectlanguage{ngerman}\endnumbering\briefempfaengerindex{Schnitzler, Arthur@\textsc{Schnitzler, Arthur}!zzzGoldmann, Paul@\emph{von Paul Goldmann}!1902-02-021@{2. 2. {[}1902{]}}|)be}\mylabel{L03196h}  \normalsize

\doendnotes{C}
\bigskip
\vfill

\clearpage

\footnotesize

\lohead{\textsc{register}}

% Definiere theindex-Environment komplett neu ohne reledmac
\makeatletter
\renewenvironment{theindex}{%
  \section*{\indexname}%
  \setlength{\parindent}{0pt}%
  \setlength{\parskip}{0pt plus 0.3pt}%
  \let\item\@idxitem
}{%
  \clearpage
}
\makeatother

\IfFileExists{\jobname-pw.ind}{\input{\jobname-pw.ind}}{}

\end{document}

      