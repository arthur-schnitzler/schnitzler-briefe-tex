%% latex-leseansicht-vorspann.tex
%% Vorspann für die Leseansicht.
%% Lädt die gemeinsame Datei latex-vorspann.tex mit nicht gesetztem Schalter.

\newif\ifkorrekturansicht
\korrekturansichtfalse

\input{../tex-inputs/latex-vorspann}

\begin{center}
            \textcolor{red}{ENTWURF, NICHT FERTIG KORRIGIERT}
                      \end{center}
            
         
         \renewcommand{\erwaehntePersonen}{Personen: Georg Brandes, Maurice Donnay, Paul Goldmann, Theodor Herzl, Heinrich Kanner, Fedor Mamroth, Fritz Mauthner, Felix Philippi, Theodore Rottenberg, Paul Schlenther, Olga Schnitzler, William Shakespeare, Isidor Singer, Elisabeth Steinrück, Hermann Sudermann, Hugo Wittmann}
         \renewcommand{\erwaehnteInstitutionen}{Institutionen: Berliner Tageblatt}
         \renewcommand{\erwaehnteOrte}{Orte: Berlin, Dessauer Straße, Deutsches Theater Berlin, Deutschland, Frankfurt am Main, Kurhaus Mödling, Wien, Österreich}
         \renewcommand{\erwaehnteWerke}{Werke: Berliner Theater. (»Lebendige Stunden« von Arthur Schnitzler.), Burgtheater. (Zum erstenmale: »Es lebe das Leben«, Drama in fünf Acten von Hermann Sudermann.), Es lebe das Leben, Lebendige Stunden. Vier Einakter, Neue Freie Presse, Theater- und Kunstnachrichten. [Burgtheater.] [Es lebe das Leben], William Shakespeare}
               \section[ Paul Goldmann an Arthur Schnitzler, 2. 2. {[}1902{]}]{ Paul Goldmann an Arthur Schnitzler, 2. 2. {[}1902{]}}\nopagebreak\mylabel{v}\rehead{ }\begin{ledgroupsized}[t]{13cm}\normalsize\beginnumbering \toendnotes[C]{\smallbreak\pagebreak[2]} \Standort{DLA, A:Schnitzler, HS.NZ85.1.3172.}
\physDesc{Brief, 1 Blatt, 4 Seiten, 1877 Zeichen
\newline{}Handschrift: blaue Tinte, deutsche Kurrent
\newline{}Beilage: ein handschriftlicher Brief, schwarze Tinte, deutsche
                                 Kurrentschrift, beschnitten und eingeklebt 
\newline{}Schnitzler: 1) mit Bleistift das Jahr »902« vermerkt  2) mit rotem Buntstift fünf Unterstreichungen}\toendnotes[C]{\smallbreak}\pstart
           \noindent{}\raggedleft{}{\pb}\textcolor{gray}{\textbf{DESSAUERSTRASSE 19}}\oindex{Dessauer Strasse@\textbf{Dessauer Straße}|pw}\pend
           \pstart
           Berlin\oindex{Berlin@\textbf{Berlin}|pw}, 2. Februar.\pend
           \pstart{}Mein lieber Freund,\pend\pstart
           Die Regelung der \label{K_L03196-1v}\edtext{Landaufenthalts-Frage}{\lemma{\textnormal{\emph{Landaufenthalts-Frage}}}\Cendnote{\textnormal{siehe Paul Goldmann an Arthur Schnitzler, 14. 1. [1902]}}}\label{K_L03196-1h} freut mich ſehr. »Kurhaus in Mödling\oindex{Kurhaus Moedling@\textbf{Kurhaus Mödling}|pw}«
               klingt vielverſprechend. Ich wünſchte, ich könnte auch hin. Ich bin ſchwer
               überarbeitet und leide\strikeout{t} ſeit einer Woche
               ununterbrochen an Kopfſchmerzen.\pend
           \pstart
           \substVorne{}\textsuperscript{\textcolor{gray}{V}}\substDazwischen{}D\substHinten{}ie Vorſtellungen von »\label{K_L03196-2v}\edtext{Lebendige Stunden\pwindex{Schnitzler, Arthur 15.05.1862 – 21.10.1931@\textsc{Schnitzler, Arthur} (15.05.1862 – 21.10.1931), \emph{Schriftsteller, Mediziner}!Lebendige Stunden. Vier Einakter1901-12-23@\strich\emph{Lebendige Stunden. Vier Einakter} {[}1901-12-23{]}|pw}}{\lemma{\textnormal{\emph{Lebendige Stunden}}}\Cendnote{\textnormal{im Deutschen Theater Berlin\oindex{Deutsches Theater Berlin@\textbf{Deutsches Theater Berlin}|pwk}}}}\label{K_L03196-2h}« ſollen ſtets ausverkauft ſein. Ich freue mich ſehr darüber, daß Dir Deine
               Arbeit auch Geld bringt. Du kannſt es brauchen. Wie hat ſich \textsc{Schlenther\pwindex{Schlenther, Paul 20.08.1854 – 30.04.1916@\textsc{Schlenther, Paul} (20.08.1854 – 30.04.1916), \emph{Schriftsteller, Kritiker, Theaterleiter}|pw}} verhalten?\pend
           \pstart
           \textsc{Sudermann\pwindex{Sudermann, Hermann 30.09.1857 – 21.11.1928@\textsc{Sudermann, Hermann} (30.09.1857 – 21.11.1928), \emph{Schriftsteller}|pw}s} neues Stück\pwindex{Sudermann, Hermann 30.09.1857 – 21.11.1928@\textsc{Sudermann, Hermann} (30.09.1857 – 21.11.1928), \emph{Schriftsteller}!Es lebe das Leben1902-02-01@\strich\emph{Es lebe das Leben} {[}1902-02-01{]}|pwv} iſt elend. {\pb}\introOben{}In der Art von \textsc{Philippi\pwindex{Philippi, Felix 05.08.1851 – 23.11.1921@\textsc{Philippi, Felix} (05.08.1851 – 23.11.1921), \emph{Schriftsteller}|pw}}. Nur macht es \textsc{Philippi\pwindex{Philippi, Felix 05.08.1851 – 23.11.1921@\textsc{Philippi, Felix} (05.08.1851 – 23.11.1921), \emph{Schriftsteller}|pw}} beſſer.\introOben{} Ich konnte nur ganz kurz darüber \label{K_L03196-3v}\edtext{telegraphiren\pwindex{Theater- und Kunstnachrichten. [Burgtheater.] [Es lebe das Leben]1902-02-08@\emph{Theater- und Kunstnachrichten. [Burgtheater.] [Es lebe das Leben]} {[}1902-02-08{]}|pwv}}{\lemma{\textnormal{\emph{telegraphiren}}}\Cendnote{\textnormal{[Paul Goldmann\pwindex{Goldmann, Paul 31.01.1865 – 25.09.1935@\textsc{Goldmann, Paul} (31.01.1865 – 25.09.1935), \emph{Schriftsteller, Journalist}|pwk}]: \emph{Theater- und Kunstnachrichten.
                        [Burgtheater.]}\pwindex{Theater- und Kunstnachrichten. [Burgtheater.] [Es lebe das Leben]1902-02-08@\emph{Theater- und Kunstnachrichten. [Burgtheater.] [Es lebe das Leben]} {[}1902-02-08{]}|pwk}. In: \emph{Neue Freie
                        Presse}\pwindex{Neue Freie Presse1864 – 1939@\emph{Neue Freie Presse} {[}1864 – 1939{]}|pwk}, Nr. 13.455, 8. 2. 1902,
                     Morgenblatt, S. 7.}}}\label{K_L03196-3h}, weil \strikeout{\textcolor{gray}{d}} die Vorſtellung erſt nach elf aus war, und ein Feuilleton darüber
               zu ſchreiben, wurde mir telegraphiſch unterſagt. Herrn \textsc{Wittmann\pwindex{Wittmann, Hugo 16.10.1839 – 06.02.1923@\textsc{Wittmann, Hugo} (16.10.1839 – 06.02.1923), \emph{Schriftsteller, Journalist}|pw}}s kritiſcher \label{K_L03196-4v}\edtext{Würdigung\pwindex{Burgtheater. (Zum erstenmale: »Es lebe das Leben«, Drama in fuenf Acten von Hermann Sudermann.)1902-02-09@\emph{Burgtheater. (Zum erstenmale: »Es lebe das Leben«, Drama in fünf Acten von Hermann Sudermann.)} {[}1902-02-09{]}|pwv}}{\lemma{\textnormal{\emph{Würdigung}}}\Cendnote{\textnormal{W.\pwindex{Wittmann, Hugo 16.10.1839 – 06.02.1923@\textsc{Wittmann, Hugo} (16.10.1839 – 06.02.1923), \emph{Schriftsteller, Journalist}|pwkv} [ = Hugo Wittmann\pwindex{Wittmann, Hugo 16.10.1839 – 06.02.1923@\textsc{Wittmann, Hugo} (16.10.1839 – 06.02.1923), \emph{Schriftsteller, Journalist}|pwk}]: \emph{Burgtheater. (Zum erstenmale: »Es lebe das Leben«, Drama in fünf Acten von
                        Hermann Sudermann.)}\pwindex{Burgtheater. (Zum erstenmale: »Es lebe das Leben«, Drama in fuenf Acten von Hermann Sudermann.)1902-02-09@\emph{Burgtheater. (Zum erstenmale: »Es lebe das Leben«, Drama in fünf Acten von Hermann Sudermann.)} {[}1902-02-09{]}|pwk}. In: \emph{Neue Freie
                        Presse}\pwindex{Neue Freie Presse1864 – 1939@\emph{Neue Freie Presse} {[}1864 – 1939{]}|pwk}, Nr. 13.456, 9. 2. 1902,
                     Morgenblatt, S. 1–3.}}}\label{K_L03196-4h} darf ein armer Reporter wie ich bin, nicht
               vorgreifen.\pend
           \pstart
           Dank für die Bücherempfehlungen. Ich leſe nach wie vor mit Genuß die \textsc{Shakespeare\pwindex{Shakespeare, William 23.4.1564? – 03.05.1616@\textsc{Shakespeare, William} (23.4.1564? – 03.05.1616), \emph{Schauspieler, Dramatiker}|pw}}-Biographie\pwindex{Brandes, Georg 04.02.1842 – 19.02.1927@\textsc{Brandes, Georg} (04.02.1842 – 19.02.1927)!William Shakespeare1895 – 1896@\strich\emph{William Shakespeare} {[}1895 – 1896{]}|pw} von \textsc{Brandes\pwindex{Brandes, Georg 04.02.1842 – 19.02.1927@\textsc{Brandes, Georg} (04.02.1842 – 19.02.1927)|pw}}.\pend
           \pstart
           \textsc{Brandes\pwindex{Brandes, Georg 04.02.1842 – 19.02.1927@\textsc{Brandes, Georg} (04.02.1842 – 19.02.1927)|pw}} iſt hier\oindex{Berlin@\textbf{Berlin}|pwv}, läßt ſich aber
               bei mir nicht ſehen. Übermorgen feiert \introOben{}er\introOben{} ſeinen 60. Geburtstag. Vergiß nicht, ihm zu \label{K_L03196-5v}\edtext{gratuliren}{\lemma{\textnormal{\emph{gratuliren}}}\Cendnote{\textnormal{kein entsprechendes Korrespondenzstück
               überliefert}}}\label{K_L03196-5h}.\pend
           \pstart
           {\pb}Mit \textsc{Singer\pwindex{Singer, Isidor 16.01.1857 – 08.12.1927@\textsc{Singer, Isidor} (16.01.1857 – 08.12.1927), \emph{Journalist, Herausgeber, Soziologe}|pw}} ſprich’, bitte, einſtweilen nicht. \textsc{Kanner\pwindex{Kanner, Heinrich 09.11.1864 – 15.02.1930@\textsc{Kanner, Heinrich} (09.11.1864 – 15.02.1930), \emph{Herausgeber, Publizist}|pw}} ſoll bald wieder hier\oindex{Berlin@\textbf{Berlin}|pwv}herkommen, und ich werde verſuchen, ihn \label{K_L03196-6v}\edtext{zur Rede zu ſtellen}{\lemma{\textnormal{\emph{zur Rede zu ſtellen}}}\Cendnote{\textnormal{siehe Paul Goldmann an Arthur Schnitzler, 25. 1. [1902]}}}\label{K_L03196-6h}.\pend
           \pstart
           An \textsc{Mauthner\pwindex{Mauthner, Fritz 1849-11-20 – 1923-06-29@\textsc{Mauthner, Fritz} (1849-11-20 – 1923-06-29), \emph{Schriftsteller, Journalist, Philosoph}|pw}s} Stelle ſoll mein \label{K_L03196-7v}\edtext{Onkel\pwindex{Mamroth, Fedor 21.02.1851 – 25.06.1907@\textsc{Mamroth, Fedor} (21.02.1851 – 25.06.1907), \emph{Journalist, Kritiker}|pwv} zum Berliner Tageblatt\orgindex{Berliner Tageblatt@Berliner Tageblatt|pw}}{\lemma{\textnormal{\emph{Onkel … Tageblatt}}}\Cendnote{\textnormal{nicht belegbar}}}\label{K_L03196-7h} kommen. An mich
               denkt ſelbſtverſtändlich Niemand. Ich bin nicht literariſch.\pend
           \pstart
           Anbei der \label{K_L03196-8v}\edtext{Brief von \textsc{Herzl\pwindex{Herzl, Theodor 1860-05-02 – 1904-07-03@\textsc{Herzl, Theodor} (1860-05-02 – 1904-07-03), \emph{Schriftsteller, Journalist}|pw}}}{\lemma{\textnormal{\emph{Brief von Herzl}}}\Cendnote{\textnormal{siehe Paul Goldmann an Arthur Schnitzler, 25. 1. [1902]}}}\label{K_L03196-8h}. Sende ihn mir, bitte, gelegentlich zurück.\pend
           \pstart
           \strikeout{»Sie\pwindex{Rottenberg, Theodore 1875-09-07 – 1945-04-05@\textsc{Rottenberg, Theodore} (1875-09-07 – 1945-04-05)|pw}}{ }\label{K_L03196-9v}\edtext{»Sie\pwindex{Rottenberg, Theodore 1875-09-07 – 1945-04-05@\textsc{Rottenberg, Theodore} (1875-09-07 – 1945-04-05)|pwv}«}{\lemma{\textnormal{\emph{»Sie«}}}\Cendnote{\textnormal{mit großer
                  Wahrscheinlichkeit Theodore Rottenberg\pwindex{Rottenberg, Theodore 1875-09-07 – 1945-04-05@\textsc{Rottenberg, Theodore} (1875-09-07 – 1945-04-05)|pwk}, mit
                  der Goldmann\pwindex{Goldmann, Paul 31.01.1865 – 25.09.1935@\textsc{Goldmann, Paul} (31.01.1865 – 25.09.1935), \emph{Schriftsteller, Journalist}|pwk} seit 1899 intim war, siehe Paul Goldmann an Arthur Schnitzler, 8. 10. [1899]}}}\label{K_L03196-9h} (aus Frankfurt\oindex{Frankfurt am Main@\textbf{Frankfurt am Main}|pw}) ſchreibt Folgendes\substVorne{}\textsuperscript{,}\substDazwischen{}:\substHinten{}\pend
           {\bigskip}\pstart
           \noindent{}{\pb}{[}hs. Rottenberg:{]} Dein \textsc{Schnitzler}-Feuilleton\pwindex{Goldmann, Paul 31.01.1865 – 25.09.1935@\textsc{Goldmann, Paul} (31.01.1865 – 25.09.1935), \emph{Schriftsteller, Journalist}!Berliner Theater. (»Lebendige Stunden« von Arthur Schnitzler.)1902-01-22@\strich\emph{Berliner Theater. (»Lebendige Stunden« von Arthur Schnitzler.)} {[}1902-01-22{]}|pwv}, womit er doch wohl
               einverſtanden ſein wird, iſt fein, fein, mein Liebſter. Nur die \label{K_L03196-10v}\edtext{Epiſoden-Sache}{\lemma{\textnormal{\emph{Epiſoden-Sache}}}\Cendnote{\textnormal{siehe Paul Goldmann an Arthur Schnitzler, 25. 1. [1902]}}}\label{K_L03196-10h} mißfällt mir. Es \uline{giebt} Männer {\kaufmannsund} viele tauſend Frauen, die von der Liebe leben. Bei \textsc{Schnitzler} wird Kunſt {\kaufmannsund} Liebe
               ſicherlich i{\geminationm}er eins bleiben; halb Frauenpoſe {\kaufmannsund} halb Öſterreich\oindex{Oesterreich@\textbf{Österreich}|pw}er iſt er nun einmal. Die wahre, erhabene \label{K_L03196-11v}\edtext{{[}»{]}deutſch\oindex{Deutschland@\textbf{Deutschland}|pwv}e Männlichkeit«}{\lemma{\textnormal{\emph{»deutſche Männlichkeit«}}}\Cendnote{\textnormal{Bezug auf die erwähnte
                     »Epiſoden-Sache«, denn Schnitzler\pwindex{Schnitzler, Arthur 15.05.1862 – 21.10.1931@\textsc{Schnitzler, Arthur} (15.05.1862 – 21.10.1931), \emph{Schriftsteller, Mediziner}|pwk} habe sich vom Thema der Liebe loszulösen und »das starke Werk seiner Mannesjahre\pwindex{Goldmann, Paul 31.01.1865 – 25.09.1935@\textsc{Goldmann, Paul} (31.01.1865 – 25.09.1935), \emph{Schriftsteller, Journalist}!Berliner Theater. (»Lebendige Stunden« von Arthur Schnitzler.)1902-01-22@\strich\emph{Berliner Theater. (»Lebendige Stunden« von Arthur Schnitzler.)} {[}1902-01-22{]}|pwkv}« zu
                  schreiben (Paul Goldmann\pwindex{Goldmann, Paul 31.01.1865 – 25.09.1935@\textsc{Goldmann, Paul} (31.01.1865 – 25.09.1935), \emph{Schriftsteller, Journalist}|pwk}: \emph{Berliner Theater. (»Lebendige Stunden« von Arthur
                        Schnitzler.)}\pwindex{Goldmann, Paul 31.01.1865 – 25.09.1935@\textsc{Goldmann, Paul} (31.01.1865 – 25.09.1935), \emph{Schriftsteller, Journalist}!Berliner Theater. (»Lebendige Stunden« von Arthur Schnitzler.)1902-01-22@\strich\emph{Berliner Theater. (»Lebendige Stunden« von Arthur Schnitzler.)} {[}1902-01-22{]}|pwk}. In: \emph{Neue Freie
                        Presse}\pwindex{Neue Freie Presse1864 – 1939@\emph{Neue Freie Presse} {[}1864 – 1939{]}|pwk}, Nr. 13.438, 22. 1. 1902,
                     Morgenblatt, S. 1–4, hier: S. 4)}}}\label{K_L03196-11h} kann ich mir von ihm aber eben
               ſo wenig denken wie von M. \textsc{Donnay}\pwindex{Donnay, Maurice 12.10.1859 – 31.03.1945@\textsc{Donnay, Maurice} (12.10.1859 – 31.03.1945), \emph{Schriftsteller}|pw} z. B.\pend
           {\bigskip}\pstart
           {[}hs. Goldmann:{]} Viele treue Grüße, mein lieber Freund, Dir und den Mädels\pwindex{Schnitzler, Olga 17.01.1882 – 13.01.1970@\textsc{Schnitzler, Olga} (17.01.1882 – 13.01.1970), \emph{Schauspielerin, Sängerin}|pwv}\pwindex{Steinrueck, Elisabeth 19.11.1885 – 07.04.1920@\textsc{Steinrück, Elisabeth} (19.11.1885 – 07.04.1920)|pwv}. {\\[\baselineskip]}Dein {\\[\baselineskip]}\spacefill\mbox{Paul Goldmnn}\pend
           \leftskip=0em{}
         
         \endnumbering\mylabel{h}\end{ledgroupsized}\begin{anhang}\end{anhang}\newcommand{\dateiname}{L03196}\newcommand{\titel}{Paul Goldmann an Arthur Schnitzler, 2. 2. [1902]}\newcommand{\editorInnen}{Martin Anton Müller und Laura Untner}%% latex-leseansicht-abspann.tex
%% Abspann für die Leseansicht.
%% Der Schalter \ifkorrekturansicht ist bereits durch den Vorspann gesetzt.

%% latex-abspann.tex
%% Gemeinsamer Abspann für Korrekturansicht und Leseansicht.
%% Setzt den Schalter \ifkorrekturansicht voraus (gesetzt in den
%% einbindenden Dateien latex-korrekturansicht-abspann.tex bzw.
%% latex-leseansicht-abspann.tex).
%% ---------------------------------------------------------------

\normalsize

% Das esempio-Environment wird nur in der Leseansicht benötigt
\ifkorrekturansicht\else
\newenvironment{esempio}[3]%
{
    \vspace{1.5ex}
    \rlap{\underline{#1}}
    \par
    \setlength{\parindent}{0cm}
    \nopagebreak
    \leftskip=#2cm
    \rightskip=#3cm
}
{
    \par
}
\fi

\doendnotes{C}
\bigskip
\vfill

\clearpage

\footnotesize

\ifkorrekturansicht
  \lohead{\textsc{register}}
\fi

% theindex-Environment neu definieren ohne reledmac
\makeatletter
\renewenvironment{theindex}{%
  \ifkorrekturansicht
    \section*{\indexname}%
  \else
    \subsubsection*{Index der erwähnten Entitäten}%
  \fi
  \setlength{\parindent}{0pt}%
  \setlength{\parskip}{0pt plus 0.3pt}%
  \let\item\@idxitem
}{%
  \ifkorrekturansicht\clearpage\fi
}
\makeatother

\IfFileExists{\jobname-pw.ind}{\input{\jobname-pw.ind}}{}

% Quellenangabe nur in der Leseansicht
\ifkorrekturansicht\else
% Fallback-Definitionen, falls die .tex-Datei \titel etc. nicht gesetzt hat
\providecommand{\titel}{}
\providecommand{\editorInnen}{}
\providecommand{\dateiname}{\jobname}

\vspace{3cm}

\vfill

\footnotesize
\textsc{Quelle}: \titel. Herausgegeben von {\editorInnen}. In: \emph{Arthur Schnitzler: Briefwechsel mit Autorinnen und Autoren}.
 Digitale Edition, https://schnitzler-briefe.acdh.oeaw.ac.at/{\dateiname}.html (Stand \today)
\fi

\end{document}


      