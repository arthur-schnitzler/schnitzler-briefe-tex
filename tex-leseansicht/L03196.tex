%% latex-leseansicht-vorspann.tex
%% Vorspann für die Leseansicht.
%% Lädt die gemeinsame Datei latex-vorspann.tex mit nicht gesetztem Schalter.

\newif\ifkorrekturansicht
\korrekturansichtfalse

\input{../tex-inputs/latex-vorspann}


\section[ Paul Goldmann an Arthur Schnitzler, 2. 2. [1902]]{L03196 Paul Goldmann an Arthur Schnitzler,  2. 2. [1902]}
\nopagebreak\mylabel{L03196v}
\rehead{ }\normalsize\beginnumbering\briefempfaengerindex{Schnitzler, Arthur@\textsc{Schnitzler, Arthur}!zzzGoldmann, Paul@\emph{von Paul Goldmann}!1902-02-021@{2. 2. [1902]}|(be}
\toendnotes[C]{\smallbreak\pagebreak[2]}
\correspDesc{Versand  durch Paul Goldmann am 2. 2. [1902] in Berlin
\newline{}Erhalt  durch Arthur Schnitzler im Zeitraum [3. 2. 1902
                  – 7. 2. 1902?] in Wien}\toendnotes[C]{\smallbreak}
\Standort{DLA, A:Schnitzler, HS.NZ85.1.3172.}
\physDesc{Brief, 1 Blatt, 4 Seiten, 1878 Zeichen
\newline{}Handschrift: blaue Tinte, deutsche Kurrent
\newline{}Beilage: ein handschriftlicher Brief, schwarze Tinte, deutsche
                                 Kurrentschrift, beschnitten und eingeklebt 
\newline{}Schnitzler: 1) mit Bleistift das Jahr »902« vermerkt  2) mit rotem Buntstift fünf Unterstreichungen}\toendnotes[C]{\smallbreak}
\pstart
           \raggedleft{}{\pb}\textcolor{gray}{\textbf{DESSAUERSTRASSE 19}}\oindex{Dessauer Straße@\textbf{Dessauer Straße}, \emph{Straße}|pw}\pend
           
\pstart
           Berlin\oindex{Berlin@\textbf{Berlin}, \emph{Hauptstadt}|pw}, 2. Februar.\pend
           
\pstart{}Mein lieber Freund,\pend\vspace{0.5em}
\pstart
           Die Regelung der \label{K_L03196-1v}\edtext{Landaufenthalts-Frage}{\lemma{\textnormal{\emph{Landaufenthalts-Frage}}}\Cendnote{\textnormal{Siehe XXXX Auszeichnungsfehler: Dokument L03192 nicht gefunden.
               }}}\label{K_L03196-1} freut mich{ }ſehr. »Kurhaus in Mödling\oindex{Hotel Kursalon Mödling@\textbf{Hotel Kursalon Mödling}, \emph{Hotel}|pw}«
               klingt vielverſprechend. Ich wünſchte, ich könnte auch hin. Ich bin{ }ſchwer
               überarbeitet und leide\strikeout{t}{ }ſeit einer Woche
               ununterbrochen an Kopfſchmerzen.\pend
           
\pstart
           \substVorne{}\textsuperscript{St}\substDazwischen{}D\substHinten{}ie Vorſtellungen von »\label{K_L03196-2v}\edtext{Lebendige Stunden\pwindex{Schnitzler, Arthur 15.\,5.\,1862 Wien – 21.\,10.\,1931 ebd.@\textsc{Schnitzler, Arthur} (15.\,5.\,1862 Wien – 21.\,10.\,1931 ebd.), \emph{Schriftsteller, Mediziner}!Lebendige Stunden. Vier Einakter@\strich\emph{Lebendige Stunden. Vier Einakter}|pw}}{\lemma{\textnormal{\emph{Lebendige Stunden}}}\Cendnote{\textnormal{im Deutschen Theater Berlin\oindex{Deutsches Theater Berlin@\textbf{Deutsches Theater Berlin}, \emph{Theater}|pwk}}}}\label{K_L03196-2}«{ }ſollen{ }ſtets ausverkauft{ }ſein. Ich freue mich{ }ſehr darüber, daß Dir Deine
               Arbeit auch Geld bringt. Du kannſt es brauchen. Wie hat{ }ſich \textsc{Schlenther\pwindex{Schlenther, Paul 20.\,8.\,1854 Chernyakhovsk – 30.\,4.\,1916 Berlin@\textsc{Schlenther, Paul} (20.\,8.\,1854 Chernyakhovsk – 30.\,4.\,1916 Berlin), \emph{Schriftsteller, Kritiker, Theaterleiter}|pw}} verhalten?\pend
           
\pstart
           \textsc{Sudermanns\pwindex{Sudermann, Hermann 30.\,9.\,1857 Macikai – 21.\,11.\,1928 Berlin@\textsc{Sudermann, Hermann} (30.\,9.\,1857 Macikai – 21.\,11.\,1928 Berlin), \emph{Schriftsteller}|pw}} neues Stück\pwindex{Sudermann, Hermann 30.\,9.\,1857 Macikai – 21.\,11.\,1928 Berlin@\textsc{Sudermann, Hermann} (30.\,9.\,1857 Macikai – 21.\,11.\,1928 Berlin), \emph{Schriftsteller}!Es lebe das Leben@\strich\emph{Es lebe das Leben}|pwv} iſt elend. {\pb}\introOben{}In der Art von \textsc{Philippi\pwindex{Philippi, Felix 5.\,8.\,1851 Berlin – 23.\,11.\,1921 ebd.@\textsc{Philippi, Felix} (5.\,8.\,1851 Berlin – 23.\,11.\,1921 ebd.), \emph{Schriftsteller}|pw}}. Nur macht es \textsc{Philippi\pwindex{Philippi, Felix 5.\,8.\,1851 Berlin – 23.\,11.\,1921 ebd.@\textsc{Philippi, Felix} (5.\,8.\,1851 Berlin – 23.\,11.\,1921 ebd.), \emph{Schriftsteller}|pw}} beſſer.\introOben{} Ich konnte nur ganz kurz darüber \label{K_L03196-3v}\edtext{telegraphiren\pwindex{Theater- und Kunstnachrichten. [Burgtheater.] [Es lebe das Leben]@\emph{Theater- und Kunstnachrichten. [Burgtheater.] [Es lebe das Leben]}|pwv}}{\lemma{\textnormal{\emph{telegraphiren}}}\Cendnote{\textnormal{[Paul Goldmann\pwindex{Goldmann, Paul 31.\,1.\,1865 Breslau – 25.\,9.\,1935 Wien@\textsc{Goldmann, Paul} (31.\,1.\,1865 Breslau – 25.\,9.\,1935 Wien), \emph{Schriftsteller, Journalist}|pwk}]: \emph{Theater- und Kunstnachrichten. [Burgtheater]}\pwindex{Theater- und Kunstnachrichten. [Burgtheater.] [Es lebe das Leben]@\emph{Theater- und Kunstnachrichten. [Burgtheater.] [Es lebe das Leben]}|pwk}. In: \emph{Neue Freie Presse}\pwindex{Neue Freie Presse@\emph{Neue Freie Presse}|pwk}, Nr. 13.455, 8. 2. 1902, Morgenblatt, S. 7.}}}\label{K_L03196-3}, weil
                  \strikeout{\textcolor{gray}{d}} die Vorſtellung erſt nach elf aus war, und ein Feuilleton darüber
               zu{ }ſchreiben, wurde mir telegraphiſch unterſagt. Herrn \textsc{Wittmanns\pwindex{Wittmann, Hugo 16.\,10.\,1839 Ulm – 6.\,2.\,1923 Wien@\textsc{Wittmann, Hugo} (16.\,10.\,1839 Ulm – 6.\,2.\,1923 Wien), \emph{Schriftsteller, Journalist}|pw}} kritiſcher \label{K_L03196-4v}\edtext{Würdigung\pwindex{Wittmann, Hugo 16.\,10.\,1839 Ulm – 6.\,2.\,1923 Wien@\textsc{Wittmann, Hugo} (16.\,10.\,1839 Ulm – 6.\,2.\,1923 Wien), \emph{Schriftsteller, Journalist}!Burgtheater. (Zum erstenmale: »Es lebe das Leben«, Drama in fünf Acten von Hermann Sudermann.)@\strich\emph{Burgtheater. (Zum erstenmale: »Es lebe das Leben«, Drama in fünf Acten von Hermann Sudermann.)}|pwv}}{\lemma{\textnormal{\emph{Würdigung}}}\Cendnote{\textnormal{W.\pwindex{Wittmann, Hugo 16.\,10.\,1839 Ulm – 6.\,2.\,1923 Wien@\textsc{Wittmann, Hugo} (16.\,10.\,1839 Ulm – 6.\,2.\,1923 Wien), \emph{Schriftsteller, Journalist}|pwkv} [ = Hugo Wittmann\pwindex{Wittmann, Hugo 16.\,10.\,1839 Ulm – 6.\,2.\,1923 Wien@\textsc{Wittmann, Hugo} (16.\,10.\,1839 Ulm – 6.\,2.\,1923 Wien), \emph{Schriftsteller, Journalist}|pwk}]: \emph{Burgtheater. (Zum erstenmale: »Es lebe das Leben«, Drama in
                        fünf Acten von Hermann Sudermann)}\pwindex{Wittmann, Hugo 16.\,10.\,1839 Ulm – 6.\,2.\,1923 Wien@\textsc{Wittmann, Hugo} (16.\,10.\,1839 Ulm – 6.\,2.\,1923 Wien), \emph{Schriftsteller, Journalist}!Burgtheater. (Zum erstenmale: »Es lebe das Leben«, Drama in fünf Acten von Hermann Sudermann.)@\strich\emph{Burgtheater. (Zum erstenmale: »Es lebe das Leben«, Drama in fünf Acten von Hermann Sudermann.)}|pwk}. In: \emph{Neue Freie Presse}\pwindex{Neue Freie Presse@\emph{Neue Freie Presse}|pwk}, Nr. 13.456, 9. 2. 1902, Morgenblatt, S. 1–3.}}}\label{K_L03196-4} darf ein armer
               Reporter, wie ich bin, nicht vorgreifen.\pend
           
\pstart
           Dank für die B\substVorne{}\textsuperscript{uch}\substDazwischen{}ücher\substHinten{}empfehlungen. Ich leſe nach wie vor mit Genuß die \textsc{Shakespeare\pwindex{Shakespeare, William 23.\,4.\,1564? Stratford-upon-Avon – 3.\,5.\,1616 ebd.@\textsc{Shakespeare, William} (23.\,4.\,1564? Stratford-upon-Avon – 3.\,5.\,1616 ebd.), \emph{Schauspieler, Dramatiker}|pw}}-Biographie\pwindex{Brandes, Georg 4.\,2.\,1842 Kopenhagen – 19.\,2.\,1927 ebd.@\textsc{Brandes, Georg} (4.\,2.\,1842 Kopenhagen – 19.\,2.\,1927 ebd.)!William Shakespeare@\strich\emph{William Shakespeare}|pw} von \textsc{Brandes\pwindex{Brandes, Georg 4.\,2.\,1842 Kopenhagen – 19.\,2.\,1927 ebd.@\textsc{Brandes, Georg} (4.\,2.\,1842 Kopenhagen – 19.\,2.\,1927 ebd.)|pw}}.\pend
           
\pstart
           \textsc{Brandes\pwindex{Brandes, Georg 4.\,2.\,1842 Kopenhagen – 19.\,2.\,1927 ebd.@\textsc{Brandes, Georg} (4.\,2.\,1842 Kopenhagen – 19.\,2.\,1927 ebd.)|pw}} iſt hier\oindex{Berlin@\textbf{Berlin}, \emph{Hauptstadt}|pwv}, läßt{ }ſich aber
               bei mir nicht{ }ſehen. Übermorgen feiert \introOben{}er\introOben{}{ }ſeinen 60. Geburtstag. Vergiß nicht, ihm zu \label{K_L03196-5v}\edtext{gratuliren}{\lemma{\textnormal{\emph{gratuliren}}}\Cendnote{\textnormal{kein entsprechendes Korrespondenzstück
               überliefert}}}\label{K_L03196-5}.\pend
           
\pstart
           {\pb}Mit \textsc{Singer\pwindex{Singer, Isidor 16.\,1.\,1857 Budapest – 8.\,12.\,1927 Wien@\textsc{Singer, Isidor} (16.\,1.\,1857 Budapest – 8.\,12.\,1927 Wien), \emph{Journalist, Herausgeber, Soziologe}|pw}}{ }ſprich’, bitte, einſtweilen nicht. \textsc{Kanner\pwindex{Kanner, Heinrich 9.\,11.\,1864 Galați – 15.\,2.\,1930 Wien@\textsc{Kanner, Heinrich} (9.\,11.\,1864 Galați – 15.\,2.\,1930 Wien), \emph{Herausgeber, Publizist}|pw}}{ }ſoll bald wieder hier\oindex{Berlin@\textbf{Berlin}, \emph{Hauptstadt}|pwv}herkommen, und ich werde verſuchen, ihn \label{K_L03196-6v}\edtext{zur Rede zu{ }ſtellen}{\lemma{\textnormal{\emph{zur Rede zu stellen}}}\Cendnote{\textnormal{Siehe XXXX Auszeichnungsfehler: Dokument L03195 nicht gefunden.
               }}}\label{K_L03196-6}.\pend
           
\pstart
           An \textsc{Mauthners\pwindex{Mauthner, Fritz 20.\,11.\,1849 Hořice – 29.\,6.\,1923 Meersburg@\textsc{Mauthner, Fritz} (20.\,11.\,1849 Hořice – 29.\,6.\,1923 Meersburg), \emph{Schriftsteller, Journalist, Philosoph}|pw}} Stelle{ }ſoll mein \label{K_L03196-7v}\edtext{Onkel\pwindex{Mamroth, Fedor 21.\,2.\,1851 Breslau – 25.\,6.\,1907 Frankfurt am Main@\textsc{Mamroth, Fedor} (21.\,2.\,1851 Breslau – 25.\,6.\,1907 Frankfurt am Main), \emph{Journalist, Kritiker}|pwv} zum Berliner Tageblatt\orgindex{Berliner Tageblatt@Berliner Tageblatt|pw}}{\lemma{\textnormal{\emph{Onkel … Tageblatt}}}\Cendnote{\textnormal{nicht belegbar}}}\label{K_L03196-7} kommen. An mich
               denkt{ }ſelbſtverſtändlich Niemand. Ich bin nicht literariſch.\pend
           
\pstart
           Anbei der \label{K_L03196-8v}\edtext{Brief von \textsc{Herzl\pwindex{Herzl, Theodor 2.\,5.\,1860 Budapest – 3.\,7.\,1904 Edlach@\textsc{Herzl, Theodor} (2.\,5.\,1860 Budapest – 3.\,7.\,1904 Edlach), \emph{Schriftsteller, Journalist}|pw}}}{\lemma{\textnormal{\emph{Brief von Herzl}}}\Cendnote{\textnormal{Siehe XXXX Auszeichnungsfehler: Dokument L03195 nicht gefunden.
               }}}\label{K_L03196-8}. Sende ihn mir, bitte, gelegentlich zurück.\pend
           
\pstart
           \strikeout{»Sie\pwindex{Rottenberg, Theodore 7.\,9.\,1875 – 5.\,4.\,1945 Limburg an der Lahn@\textsc{Rottenberg, Theodore} (7.\,9.\,1875 – 5.\,4.\,1945 Limburg an der Lahn)|pw}}{ }\label{K_L03196-9v}\edtext{»Sie\pwindex{Rottenberg, Theodore 7.\,9.\,1875 – 5.\,4.\,1945 Limburg an der Lahn@\textsc{Rottenberg, Theodore} (7.\,9.\,1875 – 5.\,4.\,1945 Limburg an der Lahn)|pwv}«}{\lemma{\textnormal{\emph{»Sie«}}}\Cendnote{\textnormal{mit großer
                  Wahrscheinlichkeit Theodore Rottenberg\pwindex{Rottenberg, Theodore 7.\,9.\,1875 – 5.\,4.\,1945 Limburg an der Lahn@\textsc{Rottenberg, Theodore} (7.\,9.\,1875 – 5.\,4.\,1945 Limburg an der Lahn)|pwk}, mit
                  der Goldmann\pwindex{Goldmann, Paul 31.\,1.\,1865 Breslau – 25.\,9.\,1935 Wien@\textsc{Goldmann, Paul} (31.\,1.\,1865 Breslau – 25.\,9.\,1935 Wien), \emph{Schriftsteller, Journalist}|pwk} seit 1899 intim war, siehe XXXX Auszeichnungsfehler: Dokument L02889 nicht gefunden.}}}\label{K_L03196-9} (aus Frankfurt\oindex{Frankfurt am Main@\textbf{Frankfurt am Main}, \emph{Hauptstadt}|pw}){ }ſchreibt Folgendes\substVorne{}\textsuperscript{,}\substDazwischen{}:\substHinten{}\pend
           {\vspace{1\baselineskip}}
\pstart
           {\pb}{[}hs. Rottenberg:{]} Dein \textsc{Schnitzler}-Feuilleton\pwindex{Goldmann, Paul 31.\,1.\,1865 Breslau – 25.\,9.\,1935 Wien@\textsc{Goldmann, Paul} (31.\,1.\,1865 Breslau – 25.\,9.\,1935 Wien), \emph{Schriftsteller, Journalist}!Berliner Theater. (»Lebendige Stunden« von Arthur Schnitzler.)@\strich\emph{Berliner Theater. (»Lebendige Stunden« von Arthur Schnitzler.)}|pwv}, womit er doch wohl
               einverſtanden{ }ſein wird, iſt fein, fein, mein Liebſter. Nur die \label{K_L03196-10v}\edtext{Epiſoden-Sache}{\lemma{\textnormal{\emph{Episoden-Sache}}}\Cendnote{\textnormal{Siehe XXXX Auszeichnungsfehler: Dokument L03195 nicht gefunden.
               }}}\label{K_L03196-10} mißfällt mir. Es \uline{giebt} Männer {\kaufmannsund} viele tauſend Frauen, die von der Liebe leben. Bei \textsc{Schnitzler} wird Kunſt {\kaufmannsund} Liebe{ }ſicherlich i{\geminationm}er eins bleiben; halb Frauenpoſe {\kaufmannsund} halb Öſterreich\oindex{Österreich@\textbf{Österreich}|pw}er iſt er nun einmal. Die wahre, erhabene \label{K_L03196-11v}\edtext{{[}»{]}deutſch\oindex{Deutschland@\textbf{Deutschland}|pwv}e Männlichkeit«}{\lemma{\textnormal{\emph{»deutsche Männlichkeit«}}}\Cendnote{\textnormal{Bezug auf die erwähnte
                     »Epiſoden-Sache«: Schnitzler habe sich vom Thema der Liebe zu lösen und »das starke Werk seiner Mannesjahre\pwindex{Goldmann, Paul 31.\,1.\,1865 Breslau – 25.\,9.\,1935 Wien@\textsc{Goldmann, Paul} (31.\,1.\,1865 Breslau – 25.\,9.\,1935 Wien), \emph{Schriftsteller, Journalist}!Berliner Theater. (»Lebendige Stunden« von Arthur Schnitzler.)@\strich\emph{Berliner Theater. (»Lebendige Stunden« von Arthur Schnitzler.)}|pwkv}« zu
                  schreiben. Paul Goldmann\pwindex{Goldmann, Paul 31.\,1.\,1865 Breslau – 25.\,9.\,1935 Wien@\textsc{Goldmann, Paul} (31.\,1.\,1865 Breslau – 25.\,9.\,1935 Wien), \emph{Schriftsteller, Journalist}|pwk}: \emph{Berliner Theater. (»Lebendige Stunden« von Arthur
                        Schnitzler)}\pwindex{Goldmann, Paul 31.\,1.\,1865 Breslau – 25.\,9.\,1935 Wien@\textsc{Goldmann, Paul} (31.\,1.\,1865 Breslau – 25.\,9.\,1935 Wien), \emph{Schriftsteller, Journalist}!Berliner Theater. (»Lebendige Stunden« von Arthur Schnitzler.)@\strich\emph{Berliner Theater. (»Lebendige Stunden« von Arthur Schnitzler.)}|pwk}. In: \emph{Neue Freie
                        Presse}\pwindex{Neue Freie Presse@\emph{Neue Freie Presse}|pwk}, Nr. 13.438, 22. 1. 1902,
                     Morgenblatt, S. 1–4, hier: S. 4.}}}\label{K_L03196-11} kann ich mir von ihm eben{ }ſo
               wenig denken wie von M. \textsc{Donnay}\pwindex{Donnay, Maurice 12.\,10.\,1859 Paris – 31.\,3.\,1945 ebd.@\textsc{Donnay, Maurice} (12.\,10.\,1859 Paris – 31.\,3.\,1945 ebd.), \emph{Schriftsteller}|pw} z. B.\pend
           {\vspace{1\baselineskip}}
\pstart
           {[}hs. Goldmann:{]} Viele treue Grüße, mein lieber Freund, Dir und den Mädels\pwindex{Schnitzler, Olga 17.\,1.\,1882 Wien – 13.\,1.\,1970 Lugano@\textsc{Schnitzler, Olga} (17.\,1.\,1882 Wien – 13.\,1.\,1970 Lugano), \emph{Schauspielerin, Sängerin}|pwv}\pwindex{Steinrück, Elisabeth 19.\,11.\,1885 – 7.\,4.\,1920 Partenkirchen@\textsc{Steinrück, Elisabeth} (19.\,11.\,1885 – 7.\,4.\,1920 Partenkirchen)|pwv}. {\\[\baselineskip]}Dein {\\[\baselineskip]}\spacefill\mbox{Paul Goldmann}\pend
           \leftskip=0em{}\selectlanguage{ngerman}\endnumbering\briefempfaengerindex{Schnitzler, Arthur@\textsc{Schnitzler, Arthur}!zzzGoldmann, Paul@\emph{von Paul Goldmann}!1902-02-021@{2. 2. [1902]}|)be}\mylabel{L03196h}  \newcommand{\dateiname}{L03196}\newcommand{\titel}{Paul Goldmann an Arthur Schnitzler, 2. 2. [1902]}\newcommand{\editorInnen}{Martin Anton Müller und Laura Untner}%% latex-leseansicht-abspann.tex
%% Abspann für die Leseansicht.
%% Der Schalter \ifkorrekturansicht ist bereits durch den Vorspann gesetzt.

%% latex-abspann.tex
%% Gemeinsamer Abspann für Korrekturansicht und Leseansicht.
%% Setzt den Schalter \ifkorrekturansicht voraus (gesetzt in den
%% einbindenden Dateien latex-korrekturansicht-abspann.tex bzw.
%% latex-leseansicht-abspann.tex).
%% ---------------------------------------------------------------

\normalsize

% Das esempio-Environment wird nur in der Leseansicht benötigt
\ifkorrekturansicht\else
\newenvironment{esempio}[3]%
{
    \vspace{1.5ex}
    \rlap{\underline{#1}}
    \par
    \setlength{\parindent}{0cm}
    \nopagebreak
    \leftskip=#2cm
    \rightskip=#3cm
}
{
    \par
}
\fi

\doendnotes{C}
\bigskip
\vfill

\clearpage

\footnotesize

\ifkorrekturansicht
  \lohead{\textsc{register}}
\fi

% theindex-Environment neu definieren ohne reledmac
\makeatletter
\renewenvironment{theindex}{%
  \ifkorrekturansicht
    \section*{\indexname}%
  \else
    \subsubsection*{Index der erwähnten Entitäten}%
  \fi
  \setlength{\parindent}{0pt}%
  \setlength{\parskip}{0pt plus 0.3pt}%
  \let\item\@idxitem
}{%
  \ifkorrekturansicht\clearpage\fi
}
\makeatother

\IfFileExists{\jobname-pw.ind}{\input{\jobname-pw.ind}}{}

% Quellenangabe nur in der Leseansicht
\ifkorrekturansicht\else
% Fallback-Definitionen, falls die .tex-Datei \titel etc. nicht gesetzt hat
\providecommand{\titel}{}
\providecommand{\editorInnen}{}
\providecommand{\dateiname}{\jobname}

\vspace{3cm}

\vfill

\footnotesize
\textsc{Quelle}: \titel. Herausgegeben von {\editorInnen}. In: \emph{Arthur Schnitzler: Briefwechsel mit Autorinnen und Autoren}.
 Digitale Edition, https://schnitzler-briefe.acdh.oeaw.ac.at/{\dateiname}.html (Stand \today)
\fi

\end{document}


