%% latex-korrekturansicht-vorspann.tex
%% Vorspann für die Korrekturansicht.
%% Lädt die gemeinsame Datei latex-vorspann.tex mit gesetztem Schalter.

\newif\ifkorrekturansicht
\korrekturansichttrue

\input{../tex-inputs/latex-vorspann}


\section[Hermann Bahr an Arthur Schnitzler, 7. 9. 1931]{L02547 Hermann Bahr an Arthur Schnitzler, 7. 9. 1931}
\nopagebreak\mylabel{L02547v}
\rehead{ }\normalsize\beginnumbering\briefempfaengerindex{Schnitzler, Arthur@\textsc{Schnitzler, Arthur}!zzzBahr, Hermann@\emph{von Hermann Bahr}!1931-09-071@{7. 9. 1931}|(be}
\toendnotes[C]{\smallbreak\pagebreak[2]}\Standort{CUL, Schnitzler, B 5b.}
\physDesc{Brief, 1 Blatt, 2 Seiten, 805 Zeichen
\newline{}Handschrift: schwarze Tinte, deutsche Kurrent
\newline{}Schnitzler: mit rotem Buntstift mehrere Unterstreichungen 
\newline{}Ordnung: mit Bleistift von unbekannter Hand nummeriert:
                                    »188« }
\buchAbdrucke{\weitereDrucke{Hermann Bahr, Arthur Schnitzler: \emph{Briefwechsel, Aufzeichnungen, Dokumente (1891–1931)}. Göttingen: \emph{Wallstein} 2018, S. 599.} }\toendnotes[C]{\smallbreak}
\pstart
           \raggedleft{}{\pb}7. September 31\pend
           
\pstart{}Lieber Arthur!\pend\vspace{0.5em}
\pstart
           Die Verfilmung\pwindex{Konzert@\emph{Das Konzert}|pwv} von »Konzert\pwindex{Konzert. Lustspiel in drei Akten@\emph{Das Konzert. Lustspiel in drei Akten}|pw}« iſt, ſo weit ich mich erinnern kann,
               zunächſt vor Jahren ſchon von S. Fiſcher\pwindex{Fischer, Samuel 24.12.1859 – 15.10.1934@\textsc{Fischer, Samuel} (24.12.1859 – 15.10.1934), \emph{Verleger/Verlegerin}|pw}
               vermittelt worden; ob von ihm auch für Tonfilm weiß ich nicht. Aber der Verlag »Ahn – und Simrock\orgindex{Ahn und Simrock@Ahn {\kaufmannsund}  Simrock|pw}«, Berlin N.W. 7 Dorotheenſtraße 11\oindex{Dorotheenstrasse@\textbf{Dorotheenstraße}, \emph{Straße (K.STR)}|pw}; Eingang Prinz Louis Ferdinandſtraße 1\oindex{Prinz Louis Ferdinandstrasse@\textbf{Prinz Louis Ferdinandstraße}, \emph{Straße (K.STR)}|pw} wird Dir, wenn Du Dich auf mich berufſt,
               darüber genau berichten und Dich beraten können.\pend
           
\pstart
           Was mein »Befinden«, nach dem Du Dich freundlich erkundigſt, betrifft, ſo kann ich
               nur ſagen, daß ich mich eigentlich überhaupt nicht {\pb}mehr {[}befinde{]}: meine Sehkraft ſchwindet, das Augenlicht verſagt
               von Tag zu Tag immer mehr und zum »Ausgleich« (Öſterreicher\oindex{Oesterreich@\textbf{Österreich}, \emph{A.PCLI}|pw} gleichen immer aus) bin ich taub und werde täglich tauber. Ich
               kann mich nur noch mit Hörrohr verſtändigen.\pend
           
\pstart
           Aber immer aufrecht!{\\[\baselineskip]}Herzlichſt{\\[\baselineskip]}Dein getreuer{\\[\baselineskip]}\spacefill\mbox{HermannBahr}\pend
           \leftskip=0em{}\selectlanguage{ngerman}\endnumbering\briefempfaengerindex{Schnitzler, Arthur@\textsc{Schnitzler, Arthur}!zzzBahr, Hermann@\emph{von Hermann Bahr}!1931-09-071@{7. 9. 1931}|)be}\mylabel{L02547h}  \normalsize

\doendnotes{C}
\bigskip
\vfill

\clearpage

\footnotesize

\lohead{\textsc{register}}

% Definiere theindex-Environment komplett neu ohne reledmac
\makeatletter
\renewenvironment{theindex}{%
  \section*{\indexname}%
  \setlength{\parindent}{0pt}%
  \setlength{\parskip}{0pt plus 0.3pt}%
  \let\item\@idxitem
}{%
  \clearpage
}
\makeatother

\IfFileExists{\jobname-pw.ind}{\input{\jobname-pw.ind}}{}

\end{document}

      