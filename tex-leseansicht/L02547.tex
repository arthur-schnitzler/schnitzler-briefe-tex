%% latex-leseansicht-vorspann.tex
%% Vorspann für die Leseansicht.
%% Lädt die gemeinsame Datei latex-vorspann.tex mit nicht gesetztem Schalter.

\newif\ifkorrekturansicht
\korrekturansichtfalse

\input{../tex-inputs/latex-vorspann}

\begin{center}
            \textcolor{red}{ENTWURF. ENTZIFFERUNG NOCH NICHT KORREKTURGELESEN}
                      \end{center}
            
               \section[Hermann Bahr an Arthur Schnitzler, 7. 9. 1931]{ Hermann Bahr an Arthur Schnitzler, 7. 9. 1931}\nopagebreak\mylabel{v}\rehead{ }\begin{ledgroupsized}[t]{13cm}\normalsize\beginnumbering\briefempfaengerindex{Schnitzler, Arthur@\textsc{Schnitzler, Arthur}!zzzBahr, Hermann@\emph{von Hermann Bahr}!1931-09-071@{7. 9. 1931}|(be} \toendnotes[C]{\smallbreak\pagebreak[2]} \Standort{CUL, Schnitzler, B 5b.}
\physDesc{Brief, 1 Blatt (Briefpapier mit Trauerrand), 2 Seiten
\newline{}Handschrift: schwarze Tinte, deutsche Kurrent
\newline{}Schnitzler: mit rotem Buntstift mehrere Unterstreichungen \newline{}Ordnung: mit Bleistift von unbekannter Hand nummeriert: »188« }\buchAbdrucke{\weitereDrucke{Hermann Bahr, Arthur Schnitzler: \emph{Briefwechsel, Aufzeichnungen, Dokumente (1891–1931)}. Hg. Kurt Ifkovits und Martin Anton Müller. Göttingen: \emph{Wallstein} 2018, S. 599.} }\toendnotes[C]{\smallbreak}\pstart
           \raggedleft{}{\pb}7. September 31\pend
           \pstart{}Lieber Arthur!\pend\pstart
           Die Verfilmung\pwindex{\textcolor{red}{\textsuperscript{XXXX1 indx}}!Konzert1931@\strich\emph{Das Konzert} {[}1931{]}|pwv} von »Konzert\pwindex{Bahr, Hermann 19.07.1863 – 15.01.1934@\textsc{Bahr, Hermann} (19.07.1863 – 15.01.1934), \emph{Schriftsteller, Kritiker}!Konzert. Lustspiel in drei Akten1909@\strich\emph{Das Konzert. Lustspiel in drei Akten} {[}1909{]}|pw}« iſt, ſo weit ich mich erinnern kann,
               zunächſt vor Jahren ſchon von S. Fiſcher\pwindex{Fischer, Samuel 24.12.1859 – 15.10.1934@\textsc{Fischer, Samuel} (24.12.1859 – 15.10.1934), \emph{Verleger}|pw}
               vermittelt worden; ob von ihm auch für Tonfilm weiß ich nicht. Aber der Verlag »Ahn – und Simrock\orgindex{Ahn und Simrock@Ahn {\kaufmannsund}  Simrock|pw}«, Berlin N.W. 7 Dorotheenſtraße 11\oindex{Dorotheenstrasse@\textbf{Dorotheenstraße}|pw}; Eingang Prinz
                  Louis Ferdinandſtraße 1\oindex{Prinz Louis Ferdinandstrasse@\textbf{Prinz Louis Ferdinandstraße}|pw} wird Dir, wenn Du Dich auf mich berufſt, darüber
               genau berichten und Dich beraten können.\pend
           \pstart
           Was mein »Befinden«, nach dem Du Dich freundlich erkundigſt, betrifft, ſo kann ich
               nur ſagen, daß ich mich eigentlich überhaupt nicht {\pb}mehr {[}befinde{]}: meine Sehkraft ſchwindet, das Augenlicht verſagt
               von Tag zu Tag immer mehr und zum »Ausgleich« (Öſterreicher\oindex{Oesterreich@\textbf{Österreich}|pw} gleichen immer aus) bin ich taub und werde täglich tauber. Ich
               kann mich nur noch mit Hörrohr verſtändigen.\pend
           \pstart
           Aber immer aufrecht!{\\[\baselineskip]}Herzlichſt{\\[\baselineskip]}Dein getreuer{\\[\baselineskip]}\spacefill\mbox{HermannBahr}\pend
           \leftskip=0em{}\endnumbering\briefempfaengerindex{Schnitzler, Arthur@\textsc{Schnitzler, Arthur}!zzzBahr, Hermann@\emph{von Hermann Bahr}!1931-09-071@{7. 9. 1931}|)be}\mylabel{h}\end{ledgroupsized}  \newcommand{\dateiname}{L02547}\newcommand{\titel}{Hermann Bahr an Arthur Schnitzler, 7. 9. 1931}\newcommand{\editorInnen}{ Kurt Ifkovits,  Martin Anton Müller}%% latex-leseansicht-abspann.tex
%% Abspann für die Leseansicht.
%% Der Schalter \ifkorrekturansicht ist bereits durch den Vorspann gesetzt.

%% latex-abspann.tex
%% Gemeinsamer Abspann für Korrekturansicht und Leseansicht.
%% Setzt den Schalter \ifkorrekturansicht voraus (gesetzt in den
%% einbindenden Dateien latex-korrekturansicht-abspann.tex bzw.
%% latex-leseansicht-abspann.tex).
%% ---------------------------------------------------------------

\normalsize

% Das esempio-Environment wird nur in der Leseansicht benötigt
\ifkorrekturansicht\else
\newenvironment{esempio}[3]%
{
    \vspace{1.5ex}
    \rlap{\underline{#1}}
    \par
    \setlength{\parindent}{0cm}
    \nopagebreak
    \leftskip=#2cm
    \rightskip=#3cm
}
{
    \par
}
\fi

\doendnotes{C}
\bigskip
\vfill

\clearpage

\footnotesize

\ifkorrekturansicht
  \lohead{\textsc{register}}
\fi

% theindex-Environment neu definieren ohne reledmac
\makeatletter
\renewenvironment{theindex}{%
  \ifkorrekturansicht
    \section*{\indexname}%
  \else
    \subsubsection*{Index der erwähnten Entitäten}%
  \fi
  \setlength{\parindent}{0pt}%
  \setlength{\parskip}{0pt plus 0.3pt}%
  \let\item\@idxitem
}{%
  \ifkorrekturansicht\clearpage\fi
}
\makeatother

\IfFileExists{\jobname-pw.ind}{\input{\jobname-pw.ind}}{}

% Quellenangabe nur in der Leseansicht
\ifkorrekturansicht\else
% Fallback-Definitionen, falls die .tex-Datei \titel etc. nicht gesetzt hat
\providecommand{\titel}{}
\providecommand{\editorInnen}{}
\providecommand{\dateiname}{\jobname}

\vspace{3cm}

\vfill

\footnotesize
\textsc{Quelle}: \titel. Herausgegeben von {\editorInnen}. In: \emph{Arthur Schnitzler: Briefwechsel mit Autorinnen und Autoren}.
 Digitale Edition, https://schnitzler-briefe.acdh.oeaw.ac.at/{\dateiname}.html (Stand \today)
\fi

\end{document}


      