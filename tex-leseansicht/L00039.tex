%% latex-korrekturansicht-vorspann.tex
%% Vorspann für die Korrekturansicht.
%% Lädt die gemeinsame Datei latex-vorspann.tex mit gesetztem Schalter.

\newif\ifkorrekturansicht
\korrekturansichttrue

\input{../tex-inputs/latex-vorspann}


\section[Arthur Schnitzler an Hugo von Hofmannsthal, 11. 9. 1891]{L00039 Arthur Schnitzler an Hugo von Hofmannsthal, 11. 9. 1891}
\nopagebreak\mylabel{L00039v}
\rehead{ }\normalsize\beginnumbering\briefempfaengerindex{Hofmannsthal, Hugo von@\textsc{Hofmannsthal, Hugo von}!zzzSchnitzler, Arthur@\emph{von Arthur Schnitzler}!1891-09-111@{11. 9. 1891}|(be}
\toendnotes[C]{\smallbreak\pagebreak[2]}\Standort{FDH, Hs-30885,15.}
\physDesc{Brief, 1 Blatt, 3 Seiten, 681 Zeichen
\newline{}Handschrift: schwarze Tinte, deutsche Kurrent
\newline{}Hofmannsthal: mit Bleistift auf der ersten Seite von Schnitzler mutmaßlich bei der
                                 Durchsicht der Korrespondenz 1929 das
                                 Datum falsch ergänzt: »11/7 91« }
\buchAbdrucke{\weitereDrucke{Hugo von Hofmannsthal, Arthur Schnitzler: \emph{Briefwechsel}. Frankfurt am Main: \emph{S. Fischer} 1964, S. 13.} }\toendnotes[C]{\smallbreak}
\pstart\center{}{\pb}Lieber Freund,\pend\vspace{0.5em}
\pstart
           der Anfang von Reichtum\pwindex{Reichtum. Erzaehlung@\emph{Reichtum. Erzählung}|pw} iſt abſcheulich – Sie
               kennen ja die Moderne Rundſchau\pwindex{Moderne Rundschau@\emph{Moderne Rundschau}|pw}! – plötzlich
               wurde das Ding geſetzt, obwohl es ausgemacht war, daß die erſten Kapitel vorher
               verändert werden müſſten. Jedenfalls änder’ ich für den \label{K_L00039-1v}\edtext{Separatabdruck\pwindex{Reichtum. Erzaehlung@\emph{Reichtum. Erzählung}|pwv}}{\lemma{\textnormal{\emph{Separatabdruck}}}\Cendnote{\textnormal{\emph{Reichtum}\pwindex{Reichtum. Erzaehlung@\emph{Reichtum. Erzählung}|pwk}. Erzählung von Arthur Schnitzler. Separat-Abdruck aus der »\emph{Modernen Rundschau}\pwindex{Moderne Rundschau@\emph{Moderne Rundschau}|pwk}«. Druck von \emph{Carl Steinhardt { }{\kaufmannsund} Cie.}\orgindex{Carl Steinhardt und Co.@Carl Steinhardt {\kaufmannsund}  Co.|pwk}{ }{[}1891{]}.}}}\label{K_L00039-1}. Die Fortſetzung iſt beſſer. Vorläufig {\pb}werd ich in den weiteſten Kreiſen verachtet. –\pend
           
\pstart
           Wann kommen Sie? Durch wen hab ich Sie grüßen laſſen? \textsc{Salten}\pwindex{Salten, Felix 06.09.1869 – 08.10.1945@\textsc{Salten, Felix} (06.09.1869 – 08.10.1945), \emph{Schriftsteller/Schriftstellerin, Journalist/Journalistin, Chefredakteur/Chefredakteurin}|pw} iſt in Miskolcz\oindex{Miskolc@\textbf{Miskolc}, \emph{P.PPLA}|pw}, das wiſſen Sie wohl. Von
                  \textsc{Beer-Hofma{\geminationn}}\pwindex{Beer-Hofmann, Richard 1866-07-11 – 1945-09-26@\textsc{Beer-Hofmann, Richard} (1866-07-11 – 1945-09-26), \emph{Schriftsteller/Schriftstellerin}|pw} hab ich keine Nachricht. Das Mährchen\pwindex{Maerchen. Schauspiel in drei Aufzuegen@\emph{Das Märchen. Schauspiel in drei Aufzügen}|pw}
               reich ich der Burg\oindex{Burgtheater@\textbf{Burgtheater}, \emph{S.THTR}|pw} ein, laſs es vorher als \label{K_L00039-2v}\edtext{Manuscript}{\lemma{\textnormal{\emph{Manuscript}}}\Cendnote{\textnormal{Arthur Schnitzler: \emph{Das Märchen. Schauspiel in drei Aufzügen}\pwindex{Maerchen. Schauspiel in drei Aufzuegen@\emph{Das Märchen. Schauspiel in drei Aufzügen}|pwk}. Wien: \emph{Carl Steinhardt}\orgindex{Carl Steinhardt und Co.@Carl Steinhardt {\kaufmannsund}  Co.|pwk}{ }1891.}}}\label{K_L00039-2} drucken. {\pb}Bringen Sie was mit? Bringen
               Sie was mit! –\pend
           
\pstart
           Leben Sie wohl, ich freu mich ſehr Sie bald wiederzuſehen. Ganz der Ihre{\\[\baselineskip]}\spacefill\mbox{Arth Sch}\pend
           \leftskip=0em{}
\pstart
           Wien\oindex{Wien@\textbf{Wien}, \emph{A.ADM2}|pw}{ }11. Sept. 91.\pend
           \selectlanguage{ngerman}\endnumbering\briefempfaengerindex{Hofmannsthal, Hugo von@\textsc{Hofmannsthal, Hugo von}!zzzSchnitzler, Arthur@\emph{von Arthur Schnitzler}!1891-09-111@{11. 9. 1891}|)be}\mylabel{L00039h}  \normalsize

\doendnotes{C}
\bigskip
\vfill

\clearpage

\footnotesize

\lohead{\textsc{register}}

% Definiere theindex-Environment komplett neu ohne reledmac
\makeatletter
\renewenvironment{theindex}{%
  \section*{\indexname}%
  \setlength{\parindent}{0pt}%
  \setlength{\parskip}{0pt plus 0.3pt}%
  \let\item\@idxitem
}{%
  \clearpage
}
\makeatother

\IfFileExists{\jobname-pw.ind}{\input{\jobname-pw.ind}}{}

\end{document}

      