%% latex-leseansicht-vorspann.tex
%% Vorspann für die Leseansicht.
%% Lädt die gemeinsame Datei latex-vorspann.tex mit nicht gesetztem Schalter.

\newif\ifkorrekturansicht
\korrekturansichtfalse

\input{../tex-inputs/latex-vorspann}


\section[Arthur Schnitzler an Hugo von Hofmannsthal, 11. 9. 1891]{L00039 Arthur Schnitzler an Hugo von Hofmannsthal, 11. 9. 1891}
\nopagebreak\mylabel{L00039v}
\rehead{ }\normalsize\beginnumbering\briefempfaengerindex{Hofmannsthal, Hugo von@\textsc{Hofmannsthal, Hugo von}!zzzSchnitzler, Arthur@\emph{von Arthur Schnitzler}!1891-09-111@{11. 9. 1891}|(be}
\toendnotes[C]{\smallbreak\pagebreak[2]}
\correspDesc{Versand  durch Arthur Schnitzler am 11. 9. 1891 in Wien
\newline{}Erhalt  durch Hugo von Hofmannsthal im Zeitraum [11. 9. 1891
                  – 15. 9. 1891?] in Wien}\toendnotes[C]{\smallbreak}
\Standort{FDH, Hs-30885,15.}
\physDesc{Brief, 1 Blatt, 3 Seiten, 681 Zeichen
\newline{}Handschrift: schwarze Tinte, deutsche Kurrent
\newline{}Hofmannsthal: mit Bleistift auf der ersten Seite von Schnitzler mutmaßlich bei der
                                 Durchsicht der Korrespondenz 1929 das
                                 Datum falsch ergänzt: »11/7 91« }
\buchAbdrucke{\weitereDrucke{Hugo von Hofmannsthal, Arthur Schnitzler: \emph{Briefwechsel}. Herausgegeben von Therese Nickl und Heinrich Schnitzler. Frankfurt am Main: \emph{S. Fischer} 1964, S. 13.} }\toendnotes[C]{\smallbreak}
\pstart\center{}{\pb}Lieber Freund,\pend\vspace{0.5em}
\pstart
           der Anfang von Reichtum\pwindex{Schnitzler, Arthur 15.\,5.\,1862 Wien – 21.\,10.\,1931 ebd.@\textsc{Schnitzler, Arthur} (15.\,5.\,1862 Wien – 21.\,10.\,1931 ebd.), \emph{Schriftsteller, Mediziner}!Reichtum. Erzählung@\strich\emph{Reichtum. Erzählung}|pw} iſt abſcheulich – Sie
               kennen ja die Moderne Rundſchau\pwindex{Moderne Rundschau@\emph{Moderne Rundschau}|pw}! – plötzlich
               wurde das Ding geſetzt, obwohl es ausgemacht war, daß die erſten Kapitel vorher
               verändert werden müſſten. Jedenfalls änder’ ich für den \label{K_L00039-1v}\edtext{Separatabdruck\pwindex{Schnitzler, Arthur 15.\,5.\,1862 Wien – 21.\,10.\,1931 ebd.@\textsc{Schnitzler, Arthur} (15.\,5.\,1862 Wien – 21.\,10.\,1931 ebd.), \emph{Schriftsteller, Mediziner}!Reichtum. Erzählung@\strich\emph{Reichtum. Erzählung}|pwv}}{\lemma{\textnormal{\emph{Separatabdruck}}}\Cendnote{\textnormal{\emph{Reichtum}\pwindex{Schnitzler, Arthur 15.\,5.\,1862 Wien – 21.\,10.\,1931 ebd.@\textsc{Schnitzler, Arthur} (15.\,5.\,1862 Wien – 21.\,10.\,1931 ebd.), \emph{Schriftsteller, Mediziner}!Reichtum. Erzählung@\strich\emph{Reichtum. Erzählung}|pwk}. Erzählung von Arthur Schnitzler. Separat-Abdruck aus der »\emph{Modernen Rundschau}\pwindex{Moderne Rundschau@\emph{Moderne Rundschau}|pwk}«. Druck von \emph{Carl Steinhardt { }{\kaufmannsund} Cie.}\orgindex{Carl Steinhardt und Co.@Carl Steinhardt {\kaufmannsund}  Co.|pwk}{ }{[}1891{]}.}}}\label{K_L00039-1}. Die Fortſetzung iſt beſſer. Vorläufig {\pb}werd ich in den weiteſten Kreiſen verachtet. –\pend
           
\pstart
           Wann kommen Sie? Durch wen hab ich Sie grüßen laſſen? \textsc{Salten}\pwindex{Salten, Felix 6.\,9.\,1869 Budapest – 8.\,10.\,1945 Zürich@\textsc{Salten, Felix} (6.\,9.\,1869 Budapest – 8.\,10.\,1945 Zürich), \emph{Schriftsteller, Journalist, Chefredakteur}|pw} iſt in Miskolcz\oindex{Miskolc@\textbf{Miskolc}|pw}, das wiſſen Sie wohl. Von
                  \textsc{Beer-Hofma{\geminationn}}\pwindex{Beer-Hofmann, Richard 11.\,7.\,1866 Wien – 26.\,9.\,1945 New York City@\textsc{Beer-Hofmann, Richard} (11.\,7.\,1866 Wien – 26.\,9.\,1945 New York City), \emph{Schriftsteller}|pw} hab ich keine Nachricht. Das Mährchen\pwindex{Schnitzler, Arthur 15.\,5.\,1862 Wien – 21.\,10.\,1931 ebd.@\textsc{Schnitzler, Arthur} (15.\,5.\,1862 Wien – 21.\,10.\,1931 ebd.), \emph{Schriftsteller, Mediziner}!Märchen. Schauspiel in drei Aufzügen@\strich\emph{Das Märchen. Schauspiel in drei Aufzügen}|pw}
               reich ich der Burg\oindex{Wien@\textbf{Wien}!I., Innere Stadt@\textbf{I., Innere Stadt}!Burgtheater@\textbf{Burgtheater}, \emph{Theater}|pw} ein, laſs es vorher als \label{K_L00039-2v}\edtext{Manuscript}{\lemma{\textnormal{\emph{Manuscript}}}\Cendnote{\textnormal{Arthur Schnitzler: \emph{Das Märchen. Schauspiel in drei Aufzügen}\pwindex{Schnitzler, Arthur 15.\,5.\,1862 Wien – 21.\,10.\,1931 ebd.@\textsc{Schnitzler, Arthur} (15.\,5.\,1862 Wien – 21.\,10.\,1931 ebd.), \emph{Schriftsteller, Mediziner}!Märchen. Schauspiel in drei Aufzügen@\strich\emph{Das Märchen. Schauspiel in drei Aufzügen}|pwk}. Wien: \emph{Carl Steinhardt}\orgindex{Carl Steinhardt und Co.@Carl Steinhardt {\kaufmannsund}  Co.|pwk}{ }1891.}}}\label{K_L00039-2} drucken. {\pb}Bringen Sie was mit? Bringen
               Sie was mit! –\pend
           
\pstart
           Leben Sie wohl, ich freu mich{ }ſehr Sie bald wiederzuſehen. Ganz der Ihre{\\[\baselineskip]}\spacefill\mbox{Arth Sch}\pend
           \leftskip=0em{}
\pstart
           Wien\oindex{Wien@\textbf{Wien}, \emph{Verwaltungsgebiet}|pw}{ }11. Sept. 91.\pend
           \selectlanguage{ngerman}\endnumbering\briefempfaengerindex{Hofmannsthal, Hugo von@\textsc{Hofmannsthal, Hugo von}!zzzSchnitzler, Arthur@\emph{von Arthur Schnitzler}!1891-09-111@{11. 9. 1891}|)be}\mylabel{L00039h}  \newcommand{\dateiname}{L00039}\newcommand{\titel}{Arthur Schnitzler an Hugo von Hofmannsthal, 11. 9. 1891}\newcommand{\editorInnen}{Martin Anton Müller und Gerd-Hermann Susen}%% latex-leseansicht-abspann.tex
%% Abspann für die Leseansicht.
%% Der Schalter \ifkorrekturansicht ist bereits durch den Vorspann gesetzt.

%% latex-abspann.tex
%% Gemeinsamer Abspann für Korrekturansicht und Leseansicht.
%% Setzt den Schalter \ifkorrekturansicht voraus (gesetzt in den
%% einbindenden Dateien latex-korrekturansicht-abspann.tex bzw.
%% latex-leseansicht-abspann.tex).
%% ---------------------------------------------------------------

\normalsize

% Das esempio-Environment wird nur in der Leseansicht benötigt
\ifkorrekturansicht\else
\newenvironment{esempio}[3]%
{
    \vspace{1.5ex}
    \rlap{\underline{#1}}
    \par
    \setlength{\parindent}{0cm}
    \nopagebreak
    \leftskip=#2cm
    \rightskip=#3cm
}
{
    \par
}
\fi

\doendnotes{C}
\bigskip
\vfill

\clearpage

\footnotesize

\ifkorrekturansicht
  \lohead{\textsc{register}}
\fi

% theindex-Environment neu definieren ohne reledmac
\makeatletter
\renewenvironment{theindex}{%
  \ifkorrekturansicht
    \section*{\indexname}%
  \else
    \subsubsection*{Index der erwähnten Entitäten}%
  \fi
  \setlength{\parindent}{0pt}%
  \setlength{\parskip}{0pt plus 0.3pt}%
  \let\item\@idxitem
}{%
  \ifkorrekturansicht\clearpage\fi
}
\makeatother

\IfFileExists{\jobname-pw.ind}{\input{\jobname-pw.ind}}{}

% Quellenangabe nur in der Leseansicht
\ifkorrekturansicht\else
% Fallback-Definitionen, falls die .tex-Datei \titel etc. nicht gesetzt hat
\providecommand{\titel}{}
\providecommand{\editorInnen}{}
\providecommand{\dateiname}{\jobname}

\vspace{3cm}

\vfill

\footnotesize
\textsc{Quelle}: \titel. Herausgegeben von {\editorInnen}. In: \emph{Arthur Schnitzler: Briefwechsel mit Autorinnen und Autoren}.
 Digitale Edition, https://schnitzler-briefe.acdh.oeaw.ac.at/{\dateiname}.html (Stand \today)
\fi

\end{document}


