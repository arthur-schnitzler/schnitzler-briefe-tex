%% latex-leseansicht-vorspann.tex
%% Vorspann für die Leseansicht.
%% Lädt die gemeinsame Datei latex-vorspann.tex mit nicht gesetztem Schalter.

\newif\ifkorrekturansicht
\korrekturansichtfalse

\input{../tex-inputs/latex-vorspann}


\section[Stefan Zweig an Arthur Schnitzler, {[}28. 5. 1913?{]}]{L03643 Stefan Zweig an Arthur Schnitzler, {[}28. 5. 1913?{]}}
\nopagebreak\mylabel{L03643v}
\rehead{ }\normalsize\beginnumbering\briefempfaengerindex{Schnitzler, Arthur@\textsc{Schnitzler, Arthur}!zzzZweig, Stefan@\emph{von Stefan Zweig}!1913-05-281@{{[}28. 5. 1913?{]}}|(be}
\toendnotes[C]{\smallbreak\pagebreak[2]}
\correspDesc{Versand  durch Stefan Zweig am [28. 5. 1913?] in Wien
\newline{}Erhalt  durch Arthur Schnitzler im Zeitraum [28. 5. 1913
                  – 31. 5. 1913?] in Wien}\toendnotes[C]{\smallbreak}
\Standort{CUL, Schnitzler, B 118.}
\physDesc{Bildpostkarte, 412 Zeichen
\newline{}Handschrift: blaue Tinte, lateinische Kurrent
\newline{}Versand: Stempel: »\nobreak{}\oindex{VIII., Josefstadt@\textbf{VIII., Josefstadt}, \emph{Verwaltungsgebiet}|pwk}8/\textcolor{gray}{×} Wien, 28. {[}5. 1913{]}\nobreak{}«.  
\newline{}Schnitzler: mit rotem Buntstift eine Unterstreichung }
\buchAbdrucke{\weitereDrucke{1) Stefan Zweig: \emph{Briefwechsel mit Hermann Bahr, Sigmund Freud, Rainer Maria
                        Rilke und Arthur Schnitzler}. Herausgegeben von Jeffrey B. Berlin, Hans-Ulrich Lindken und Donald A. Prater. Frankfurt am Main: \emph{S. Fischer} 1987, S. 379.} \weitereDrucke{2) Hermann Bahr, Arthur Schnitzler: \emph{Briefwechsel, Aufzeichnungen, Dokumente (1891–1931)}. Herausgegeben von Kurt Ifkovits und Martin Anton Müller. Göttingen: \emph{Wallstein} 2018, S. 487.} }\toendnotes[C]{\smallbreak}\pstart{}{\pb}D\textsuperscript{r} Artur
                  Schnitzler\pend{}\pstart{}Wien – Cottage\oindex{Wien@\textbf{Wien}!XVIII., Währing@\textbf{XVIII., Währing}!Währinger Cottage@\textbf{Währinger Cottage}, \emph{Teil eines besiedelten Ortes}|pw}\pend{}\pstart{}\label{K_L03643-1v}\edtext{Sternwartestrasse 72}{\lemma{\textnormal{\emph{Sternwartestrasse 72}}}\Cendnote{\textnormal{Zweig\pwindex{Zweig, Stefan 28.\,11.\,1881 Wien – 23.\,2.\,1942 Petrópolis@\textsc{Zweig, Stefan} (28.\,11.\,1881 Wien – 23.\,2.\,1942 Petrópolis), \emph{Schriftsteller}|pwk} wechselt bei der Adressierung
                        seiner Schreiben an Schnitzler immer
                        wieder zwischen der falschen Hausnummer »72« und der
                        richtigen »71«.}}}\label{K_L03643-1}\oindex{Wien@\textbf{Wien}!XVIII., Währing@\textbf{XVIII., Währing}!Sternwartestraße 71@\textbf{Sternwartestraße 71}, \emph{Wohngebäude}|pw}\pend{}{\bigskip}
\pstart
           \noindent{}\centering{}{\pb}\textcolor{gray}{\textbf{Wien\oindex{Wien@\textbf{Wien}, \emph{Verwaltungsgebiet}|pw}.}}\pend
           
\pstart
           \centering{}\textcolor{gray}{\textbf{Kirche Maria am Gestade\oindex{Wien@\textbf{Wien}!I., Innere Stadt@\textbf{I., Innere Stadt}!Maria am Gestade@\textbf{Maria am Gestade}, \emph{Kirche}|pw}}}\pend
           
\pstart
           \centering{}\textcolor{gray}{\textbf{Erwin Pendl\pwindex{Pendl, Erwin August 18.\,10.\,1875 Wien – 4.\,8.\,1945 ebd.@\textsc{Pendl, Erwin August} (18.\,10.\,1875 Wien – 4.\,8.\,1945 ebd.), \emph{Schriftsteller, Maler}|pw} pinx.}}\pend
           \vspace{1em}
\pstart{}{\pb}Verehrter Herr Doktor,\pend\vspace{0.5em}
\pstart
           vielen Dank für Ihre \label{K_L03643-2v}\edtext{guten Worte}{\lemma{\textnormal{\emph{guten Worte}}}\Cendnote{\textnormal{XXXX Auszeichnungsfehler: Dokument L03780 nicht gefunden.}}}\label{K_L03643-2}. Meine \label{K_L03643-3v}\edtext{Bahr\pwindex{Bahr, Hermann 19.\,7.\,1863 Linz – 15.\,1.\,1934 München@\textsc{Bahr, Hermann} (19.\,7.\,1863 Linz – 15.\,1.\,1934 München), \emph{Schriftsteller, Kritiker}|pw}-Rede\pwindex{Zweig, Stefan 28.\,11.\,1881 Wien – 23.\,2.\,1942 Petrópolis@\textsc{Zweig, Stefan} (28.\,11.\,1881 Wien – 23.\,2.\,1942 Petrópolis), \emph{Schriftsteller}!Hermann Bahr, der Fünfzigjährige. (Eine Rede im Akademischen Verband für Literatur)@\strich\emph{Hermann Bahr, der Fünfzigjährige. (Eine Rede im Akademischen Verband für Literatur)}|pw}\eventindex{Elektrotechnisches Institut der Technischen Universität@\textbf{Elektrotechnisches Institut der Technischen Universität}!Hermann-Bahr-Feier, 26.5.1913@Hermann-Bahr-Feier, 26.5.1913|pwv}}{\lemma{\textnormal{\emph{Bahr-Rede}}}\Cendnote{\textnormal{Stefan Zweig\pwindex{Zweig, Stefan 28.\,11.\,1881 Wien – 23.\,2.\,1942 Petrópolis@\textsc{Zweig, Stefan} (28.\,11.\,1881 Wien – 23.\,2.\,1942 Petrópolis), \emph{Schriftsteller}|pwk} hatte am
                     26. 5. 1913 aus Anlass von Hermann Bahrs\pwindex{Bahr, Hermann 19.\,7.\,1863 Linz – 15.\,1.\,1934 München@\textsc{Bahr, Hermann} (19.\,7.\,1863 Linz – 15.\,1.\,1934 München), \emph{Schriftsteller, Kritiker}|pwk} 50. Geburtstag am 19. 5. 1913 eine Rede im \emph{Akademischen Verband für Literatur}\orgindex{Akademischer Verband für Literatur und Musik in Wien@Akademischer Verband für Literatur und Musik in Wien|pwk}\eventindex{Elektrotechnisches Institut der Technischen Universität@\textbf{Elektrotechnisches Institut der Technischen Universität}!Hermann-Bahr-Feier, 26.5.1913@Hermann-Bahr-Feier, 26.5.1913|pwkv}
                  gehalten.}}}\label{K_L03643-3}{ }\label{K_L03643-4v}\edtext{in der N. F. P.\pwindex{Neue Freie Presse@\emph{Neue Freie Presse}|pw}}{\lemma{\textnormal{\emph{in der N. F. P.}}}\Cendnote{\textnormal{Stefan Zweig\pwindex{Zweig, Stefan 28.\,11.\,1881 Wien – 23.\,2.\,1942 Petrópolis@\textsc{Zweig, Stefan} (28.\,11.\,1881 Wien – 23.\,2.\,1942 Petrópolis), \emph{Schriftsteller}|pwk}: \emph{Hermann Bahr, der Fünfzigjährige. (Eine Rede im Akademischen
                        Verband für Literatur)}\pwindex{Zweig, Stefan 28.\,11.\,1881 Wien – 23.\,2.\,1942 Petrópolis@\textsc{Zweig, Stefan} (28.\,11.\,1881 Wien – 23.\,2.\,1942 Petrópolis), \emph{Schriftsteller}!Hermann Bahr, der Fünfzigjährige. (Eine Rede im Akademischen Verband für Literatur)@\strich\emph{Hermann Bahr, der Fünfzigjährige. (Eine Rede im Akademischen Verband für Literatur)}|pwk}. In: \emph{Neue Freie
                        Presse}\orgindex{Neue Freie Presse@Neue Freie Presse|pwk}, Nr. 17.513, 27.\,5.\,1913, Morgenblatt,
                     S. 1–3.}}}\label{K_L03643-4} war stark frisiert und geschoren, ich hoffe, dass sie in
               Wirklichkeit intensiver war und mehr von seinem Rytmus hatte. Ich würde mich sehr
               freuen, Sie im Juli sehen zu dürfen und wünsche Ihnen inzwischen für
               \label{K_L03643-5v}\edtext{Ihre Fahrt}{\lemma{\textnormal{\emph{Ihre Fahrt}}}\Cendnote{\textnormal{Vgl. XXXX Auszeichnungsfehler: Dokument L03638 nicht gefunden.}}}\label{K_L03643-5} alles Schöne.\pend
           \pstart Ihr ergebener \spacefill\mbox{Stefan Zweig}\pend{}\selectlanguage{ngerman}\endnumbering\briefempfaengerindex{Schnitzler, Arthur@\textsc{Schnitzler, Arthur}!zzzZweig, Stefan@\emph{von Stefan Zweig}!1913-05-281@{{[}28. 5. 1913?{]}}|)be}\mylabel{L03643h}  \newcommand{\dateiname}{L03643}\newcommand{\titel}{Stefan Zweig an Arthur Schnitzler, [28. 5. 1913?]}\newcommand{\editorInnen}{Selma Jahnke und Martin Anton Müller}%% latex-leseansicht-abspann.tex
%% Abspann für die Leseansicht.
%% Der Schalter \ifkorrekturansicht ist bereits durch den Vorspann gesetzt.

%% latex-abspann.tex
%% Gemeinsamer Abspann für Korrekturansicht und Leseansicht.
%% Setzt den Schalter \ifkorrekturansicht voraus (gesetzt in den
%% einbindenden Dateien latex-korrekturansicht-abspann.tex bzw.
%% latex-leseansicht-abspann.tex).
%% ---------------------------------------------------------------

\normalsize

% Das esempio-Environment wird nur in der Leseansicht benötigt
\ifkorrekturansicht\else
\newenvironment{esempio}[3]%
{
    \vspace{1.5ex}
    \rlap{\underline{#1}}
    \par
    \setlength{\parindent}{0cm}
    \nopagebreak
    \leftskip=#2cm
    \rightskip=#3cm
}
{
    \par
}
\fi

\doendnotes{C}
\bigskip
\vfill

\clearpage

\footnotesize

\ifkorrekturansicht
  \lohead{\textsc{register}}
\fi

% theindex-Environment neu definieren ohne reledmac
\makeatletter
\renewenvironment{theindex}{%
  \ifkorrekturansicht
    \section*{\indexname}%
  \else
    \subsubsection*{Index der erwähnten Entitäten}%
  \fi
  \setlength{\parindent}{0pt}%
  \setlength{\parskip}{0pt plus 0.3pt}%
  \let\item\@idxitem
}{%
  \ifkorrekturansicht\clearpage\fi
}
\makeatother

\IfFileExists{\jobname-pw.ind}{\input{\jobname-pw.ind}}{}

% Quellenangabe nur in der Leseansicht
\ifkorrekturansicht\else
% Fallback-Definitionen, falls die .tex-Datei \titel etc. nicht gesetzt hat
\providecommand{\titel}{}
\providecommand{\editorInnen}{}
\providecommand{\dateiname}{\jobname}

\vspace{3cm}

\vfill

\footnotesize
\textsc{Quelle}: \titel. Herausgegeben von {\editorInnen}. In: \emph{Arthur Schnitzler: Briefwechsel mit Autorinnen und Autoren}.
 Digitale Edition, https://schnitzler-briefe.acdh.oeaw.ac.at/{\dateiname}.html (Stand \today)
\fi

\end{document}


