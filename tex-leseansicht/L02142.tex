%% latex-korrekturansicht-vorspann.tex
%% Vorspann für die Korrekturansicht.
%% Lädt die gemeinsame Datei latex-vorspann.tex mit gesetztem Schalter.

\newif\ifkorrekturansicht
\korrekturansichttrue

\input{../tex-inputs/latex-vorspann}


\section[Arthur Schnitzler an Hermann Bahr, 19. 7. 1913]{L02142 Arthur Schnitzler an Hermann Bahr, 19. 7. 1913}
\nopagebreak\mylabel{L02142v}
\rehead{ }\pwindex{XXXX Abgedrucktes Werk, Nummer nicht vorhanden|pwt}\normalsize\beginnumbering\briefempfaengerindex{Bahr, Hermann@\textsc{Bahr, Hermann}!zzzSchnitzler, Arthur@\emph{von Arthur Schnitzler}!1913-07-192@{{[}19.{]} 7. 1913}|(be}
\toendnotes[C]{\smallbreak\pagebreak[2]}\Standort{TMW, BT O 1576.}
\physDesc{Bild, 1 Blatt, 1 Seite, 53 Zeichen (Emma Löwenstamm\pwindex{Loewenstamm, Emma 01.07.1879 – 1941-01-09@\textsc{Löwenstamm, Emma} (01.07.1879 – 1941-01-09), \emph{Maler/Malerin, Radierer/Radiererin}|pw}: Brustbild von
                                    Arthur Schnitzler\pwindex{Arthur Schnitzler @\emph{Arthur Schnitzler}|pwv}, Radierung )
\newline{}Handschrift: schwarze Tinte, deutsche Kurrent}
\buchAbdrucke{\weitereDrucke{Hermann Bahr, Arthur Schnitzler: \emph{Briefwechsel, Aufzeichnungen, Dokumente (1891–1931)}. Göttingen: \emph{Wallstein} 2018, S. 489.} }
\pstart
           \noindent{}{\pb}Meinem lieben
               Hermann{\\}zum 19. Juli 1913\pend
           
\pstart
           herzlichſt{\\[\baselineskip]}\spacefill\mbox{Arthur}\pend
           \leftskip=0em{}\selectlanguage{ngerman}\endnumbering\briefempfaengerindex{Bahr, Hermann@\textsc{Bahr, Hermann}!zzzSchnitzler, Arthur@\emph{von Arthur Schnitzler}!1913-07-192@{{[}19.{]} 7. 1913}|)be}\mylabel{L02142h}  \normalsize

\doendnotes{C}
\bigskip
\vfill

\clearpage

\footnotesize

\lohead{\textsc{register}}

% Definiere theindex-Environment komplett neu ohne reledmac
\makeatletter
\renewenvironment{theindex}{%
  \section*{\indexname}%
  \setlength{\parindent}{0pt}%
  \setlength{\parskip}{0pt plus 0.3pt}%
  \let\item\@idxitem
}{%
  \clearpage
}
\makeatother

\IfFileExists{\jobname-pw.ind}{\input{\jobname-pw.ind}}{}

\end{document}

      