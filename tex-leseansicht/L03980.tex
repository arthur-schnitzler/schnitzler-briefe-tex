%% latex-leseansicht-vorspann.tex
%% Vorspann für die Leseansicht.
%% Lädt die gemeinsame Datei latex-vorspann.tex mit nicht gesetztem Schalter.

\newif\ifkorrekturansicht
\korrekturansichtfalse

\input{../tex-inputs/latex-vorspann}


\section[Arthur Schnitzler an Berta Zuckerkandl, 3. 6. 1931]{L03980 Arthur Schnitzler an Berta Zuckerkandl, 3. 6. 1931}
\nopagebreak\mylabel{L03980v}
\rehead{ }\normalsize\beginnumbering\briefempfaengerindex{Zuckerkandl, Berta@\textsc{Zuckerkandl, Berta}!zzzSchnitzler, Arthur@\emph{von Arthur Schnitzler}!1931-06-031@{3. 6. 1931}|(be}
\toendnotes[C]{\smallbreak\pagebreak[2]}
\correspDesc{Versand  durch Arthur Schnitzler am 3. 6. 1931 in Wien
\newline{}Erhalt  durch Berta Zuckerkandl im Zeitraum [4. 6. 1931
                  – 8. 6. 1931?] in Paris}\toendnotes[C]{\smallbreak}
\Standort{DLA, HS.1985.1.2282.}
\physDesc{Brief, Durchschlag, 1 Blatt, 1 Seite, 711 Zeichen
\newline{}Schreibmaschine
\newline{}Handschrift: roter Buntstift, lateinische Kurrent (\noindent{}beschriftet: »\uline{Zuckerkandl}« und »\uline{Paris}«, drei Unterstreichungen)}\toendnotes[C]{\smallbreak}
\pstart
           \raggedleft{}{\pb}3. 6. 1931.\pend
           
\pstart{}Liebe und verehrte Freundin.\pend\vspace{0.5em}
\pstart
           Dieser Tage kommt, wie Sie wissen, die »Komödie der
                  Worte\pwindex{Schnitzler, Arthur 15. 5. 1862 Wien – 21. 10. 1931 ebd.@\textsc{Schnitzler, Arthur} (15. 5. 1862 Wien – 21. 10. 1931 ebd.), \emph{Schriftsteller, Mediziner}!Komödie der Worte. Drei Einakter@\strich\emph{Komödie der Worte. Drei Einakter}|pw}« in Paris\oindex{Paris@\textbf{Paris}, \emph{Hauptstadt}|pw}{ }\label{K_L03980-1v}\edtext{zur deutschen Aufführung\eventindex{Théâtre des Champs-Élysées@\textbf{Théâtre des Champs-Élysées}!Premiere von Komödie der Worte auf Deutsch in Paris@Premiere von Komödie der Worte auf Deutsch in Paris|pwv}}{\lemma{\textnormal{\emph{zur deutschen Aufführung}}}\Cendnote{\textnormal{Die schweizer\oindex{Schweiz@\textbf{Schweiz}|pwk}
                  Regisseurin Georgette Boner\pwindex{Boner, Georgette 4.\,2.\,1903 Mailand – 26.\,11.\,1998 Zürich@\textsc{Boner, Georgette} (4.\,2.\,1903 Mailand – 26.\,11.\,1998 Zürich), \emph{Regisseurin, Malerin, Literaturhistorikerin}|pwk}, die sich mit
                  einer Arbeit über Arthur Schnitzlers
                     Frauengestalten\pwindex{\textcolor{red}{\textsuperscript{XXXX indx1}}|pwk} promoviert hatte, inszenierte mit der Theatergruppe \emph{Studio Allemand}\orgindex{Studio Allemand@Studio Allemand|pwk} den Einakterzyklus \emph{Komödie der Worte}\pwindex{Schnitzler, Arthur 15. 5. 1862 Wien – 21. 10. 1931 ebd.@\textsc{Schnitzler, Arthur} (15. 5. 1862 Wien – 21. 10. 1931 ebd.), \emph{Schriftsteller, Mediziner}!Komödie der Worte. Drei Einakter@\strich\emph{Komödie der Worte. Drei Einakter}|pwk}, der zwischen dem 12. 6.\eventindex{Théâtre des Champs-Élysées@\textbf{Théâtre des Champs-Élysées}!Premiere von Komödie der Worte auf Deutsch in Paris@Premiere von Komödie der Worte auf Deutsch in Paris|pwkv}
                  und dem 17. 06. 1931 am Théâtre des Champs-Élysées\oindex{Théâtre des Champs-Élysées@\textbf{Théâtre des Champs-Élysées}|pwk} aufgeführt wurde.}}}\label{K_L03980-1}. Ich habe mir erlaubt
               sowohl Ihnen als Mme. Clemenceau\pwindex{Clemenceau, Sophie 25.\,5.\,1862 – 24.\,9.\,1937@\textsc{Clemenceau, Sophie} (25.\,5.\,1862 – 24.\,9.\,1937)|pw} Billetts
               zusenden zu lassen und es wird mich freuen, wenn Sie Zeit haben sollten der Generalprobe\eventindex{Paris@\textbf{Paris}!Generalprobe von Komödie der Worte auf Deutsch in Paris@Generalprobe von Komödie der Worte auf Deutsch in Paris|pwv} beizuwohnen und
               mir vielleicht ein Wort darüber zu berichten. Indess hat Frau Clauser\pwindex{Clauser, Suzanne 16.\,5.\,1898 Wien – 11.\,9.\,1981 Paris@\textsc{Clauser, Suzanne} (16.\,5.\,1898 Wien – 11.\,9.\,1981 Paris), \emph{Schriftstellerin, Übersetzerin}|pw} auch eine ausgezeichnete Uebersetzung\pwindex{Schnitzler, Arthur 15. 5. 1862 Wien – 21. 10. 1931 ebd.@\textsc{Schnitzler, Arthur} (15. 5. 1862 Wien – 21. 10. 1931 ebd.), \emph{Schriftsteller, Mediziner}!?? [französische Übersetzung von Das Bacchusfest]@\strich\emph{?? [französische Übersetzung von Das Bacchusfest]}|pw}\pwindex{Schnitzler, Arthur 15. 5. 1862 Wien – 21. 10. 1931 ebd.@\textsc{Schnitzler, Arthur} (15. 5. 1862 Wien – 21. 10. 1931 ebd.), \emph{Schriftsteller, Mediziner}!?? [französische Übersetzung von Große Szene]@\strich\emph{?? [französische Übersetzung von Große Szene]}|pw}\pwindex{Schnitzler, Arthur 15. 5. 1862 Wien – 21. 10. 1931 ebd.@\textsc{Schnitzler, Arthur} (15. 5. 1862 Wien – 21. 10. 1931 ebd.), \emph{Schriftsteller, Mediziner}!?? [französische Übersetzung von Stunde des Erkennens]@\strich\emph{?? [französische Übersetzung von Stunde des Erkennens]}|pw} der drei Einakter\pwindex{Schnitzler, Arthur 15. 5. 1862 Wien – 21. 10. 1931 ebd.@\textsc{Schnitzler, Arthur} (15. 5. 1862 Wien – 21. 10. 1931 ebd.), \emph{Schriftsteller, Mediziner}!Bacchusfest@\strich\emph{Das Bacchusfest}|pwv}\pwindex{Schnitzler, Arthur 15. 5. 1862 Wien – 21. 10. 1931 ebd.@\textsc{Schnitzler, Arthur} (15. 5. 1862 Wien – 21. 10. 1931 ebd.), \emph{Schriftsteller, Mediziner}!Stunde des Erkennens@\strich\emph{Stunde des Erkennens}|pwv}\pwindex{Schnitzler, Arthur 15. 5. 1862 Wien – 21. 10. 1931 ebd.@\textsc{Schnitzler, Arthur} (15. 5. 1862 Wien – 21. 10. 1931 ebd.), \emph{Schriftsteller, Mediziner}!Große Szene@\strich\emph{Große Szene}|pwv} ins Französische\oindex{Frankreich@\textbf{Frankreich}|pw} fertig gestellt{[},{]}
               wir warten also damit, ebenso wie mit dem »Bernhardi\pwindex{Schnitzler, Arthur 15. 5. 1862 Wien – 21. 10. 1931 ebd.@\textsc{Schnitzler, Arthur} (15. 5. 1862 Wien – 21. 10. 1931 ebd.), \emph{Schriftsteller, Mediziner}!Professor Bernhardi. Komödie in fünf Akten@\strich\emph{Professor Bernhardi. Komödie in fünf Akten}|pw}\pwindex{Schnitzler, Arthur 15. 5. 1862 Wien – 21. 10. 1931 ebd.@\textsc{Schnitzler, Arthur} (15. 5. 1862 Wien – 21. 10. 1931 ebd.), \emph{Schriftsteller, Mediziner}!?? [französische Übersetzung von Professor Bernhardi]@\strich\emph{?? [französische Übersetzung von Professor Bernhardi]}|pw}« und mit dem »Weiten Land\pwindex{Schnitzler, Arthur 15. 5. 1862 Wien – 21. 10. 1931 ebd.@\textsc{Schnitzler, Arthur} (15. 5. 1862 Wien – 21. 10. 1931 ebd.), \emph{Schriftsteller, Mediziner}!weite Land. Tragikomödie in fünf Akten@\strich\emph{Das weite Land. Tragikomödie in fünf Akten}|pw}«
               vor den Pforten der Pariser\oindex{Paris@\textbf{Paris}, \emph{Hauptstadt}|pw} Theater auf
               Einlass.\pend
           
\pstart
           Für heute nur dies und viele herzliche{\\[\baselineskip]} Grüsse. Hoffentlich habe ich bald
               die Freude Sie{\\[\baselineskip]} wiederzusehen.\pend
           \leftskip=0em{}{\vspace{1\baselineskip}}
\pstart
           \noindent{}Frau Hofrätin Bertha Zuckerkandl,{\\}Paris\oindex{Paris@\textbf{Paris}, \emph{Hauptstadt}|pw}.\pend
           \selectlanguage{ngerman}\endnumbering\briefempfaengerindex{Zuckerkandl, Berta@\textsc{Zuckerkandl, Berta}!zzzSchnitzler, Arthur@\emph{von Arthur Schnitzler}!1931-06-031@{3. 6. 1931}|)be}\mylabel{L03980h}
\begin{anhang}
\end{anhang}\newcommand{\dateiname}{L03980}\newcommand{\titel}{Arthur Schnitzler an Berta Zuckerkandl, 3. 6. 1931}\newcommand{\editorInnen}{Herausgegeben von Jahnke, SelmaMüller, Martin Anton}%% latex-leseansicht-abspann.tex
%% Abspann für die Leseansicht.
%% Der Schalter \ifkorrekturansicht ist bereits durch den Vorspann gesetzt.

%% latex-abspann.tex
%% Gemeinsamer Abspann für Korrekturansicht und Leseansicht.
%% Setzt den Schalter \ifkorrekturansicht voraus (gesetzt in den
%% einbindenden Dateien latex-korrekturansicht-abspann.tex bzw.
%% latex-leseansicht-abspann.tex).
%% ---------------------------------------------------------------

\normalsize

% Das esempio-Environment wird nur in der Leseansicht benötigt
\ifkorrekturansicht\else
\newenvironment{esempio}[3]%
{
    \vspace{1.5ex}
    \rlap{\underline{#1}}
    \par
    \setlength{\parindent}{0cm}
    \nopagebreak
    \leftskip=#2cm
    \rightskip=#3cm
}
{
    \par
}
\fi

\doendnotes{C}
\bigskip
\vfill

\clearpage

\footnotesize

\ifkorrekturansicht
  \lohead{\textsc{register}}
\fi

% theindex-Environment neu definieren ohne reledmac
\makeatletter
\renewenvironment{theindex}{%
  \ifkorrekturansicht
    \section*{\indexname}%
  \else
    \subsubsection*{Index der erwähnten Entitäten}%
  \fi
  \setlength{\parindent}{0pt}%
  \setlength{\parskip}{0pt plus 0.3pt}%
  \let\item\@idxitem
}{%
  \ifkorrekturansicht\clearpage\fi
}
\makeatother

\IfFileExists{\jobname-pw.ind}{\input{\jobname-pw.ind}}{}

% Quellenangabe nur in der Leseansicht
\ifkorrekturansicht\else
% Fallback-Definitionen, falls die .tex-Datei \titel etc. nicht gesetzt hat
\providecommand{\titel}{}
\providecommand{\editorInnen}{}
\providecommand{\dateiname}{\jobname}

\vspace{3cm}

\vfill

\footnotesize
\textsc{Quelle}: \titel. Herausgegeben von {\editorInnen}. In: \emph{Arthur Schnitzler: Briefwechsel mit Autorinnen und Autoren}.
 Digitale Edition, https://schnitzler-briefe.acdh.oeaw.ac.at/{\dateiname}.html (Stand \today)
\fi

\end{document}


