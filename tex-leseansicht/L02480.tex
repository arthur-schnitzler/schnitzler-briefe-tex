%% latex-leseansicht-vorspann.tex
%% Vorspann für die Leseansicht.
%% Lädt die gemeinsame Datei latex-vorspann.tex mit nicht gesetztem Schalter.

\newif\ifkorrekturansicht
\korrekturansichtfalse

\input{../tex-inputs/latex-vorspann}


\section[Emil Ludwig an Arthur Schnitzler, 3. 2. 1927]{L02480 Emil Ludwig an Arthur Schnitzler, 3. 2. 1927}
\nopagebreak\mylabel{L02480v}
\rehead{ }\normalsize\beginnumbering\briefempfaengerindex{Schnitzler, Arthur@\textsc{Schnitzler, Arthur}!zzzLudwig, Emil@\emph{von Emil Ludwig}!1927-02-031@{3. 2. 1927}|(be}
\toendnotes[C]{\smallbreak\pagebreak[2]}
\correspDesc{Versand  durch Emil Ludwig am 3. 2. 1927 in Ascona
\newline{}Umleitung  in Berlin
\newline{}Erhalt  durch Arthur Schnitzler im Zeitraum [4. 2. 1927
                  – 8. 2. 1927?] in Wien}\toendnotes[C]{\smallbreak}
\Standort{CUL, Schnitzler, B 62.}
\physDesc{Bildpostkarte, 170 Zeichen
\newline{}Handschrift: schwarze Tinte, lateinische Kurrent
\newline{}Versand: Stempel: »\nobreak{}\oindex{Ascona@\textbf{Ascona}|pwk}A\textcolor{gray}{s}cona, 3. II. 27, 18\nobreak{}«.  }\toendnotes[C]{\smallbreak}\pstart{}{\pb}Herrn\pend{}\pstart{}Dr. Arthur Schnitzler\pend{}\pstart{}Adr. S. Fischer Verlag\orgindex{S. Fischer Verlag@S. Fischer Verlag|pw}\pend{}\pstart{}90, Bülowstr. 90\oindex{Bülowstraße@\textbf{Bülowstraße}, \emph{Straße}|pw}\pend{}\pstart{}Berlin W. 57\oindex{Berlin@\textbf{Berlin}, \emph{Hauptstadt}|pw}\pend{}{\bigskip}
\pstart
           \noindent{}\centering{}{\pb}\textcolor{gray}{\textbf{{[}Statue eines Bogenschützens mit Blick auf
                     eine Brissago-Insel\oindex{Isole di Brissago@\textbf{Isole di Brissago}, \emph{Insel}|pw}{]}}}\pend
           \vspace{1em}
\pstart
           \noindent{}{\pb}Dankbar für hochinteressantes Diagramm\pwindex{Schnitzler, Arthur 15.\,5.\,1862 Wien – 21.\,10.\,1931 ebd.@\textsc{Schnitzler, Arthur} (15.\,5.\,1862 Wien – 21.\,10.\,1931 ebd.), \emph{Schriftsteller, Mediziner}!Geist im Wort und der Geist in der Tat@\strich\emph{Der Geist im Wort und der Geist in der Tat}|pwv},\pend
           
\pstart
           grüsst Sie Ihr verehrungsvoll ergebener{\\[\baselineskip]}\spacefill\mbox{Ludwig}\pend
           \leftskip=0em{}
\pstart
           Ascona\oindex{Ascona@\textbf{Ascona}|pw}. 3. 2.\pend
           \selectlanguage{ngerman}\endnumbering\briefempfaengerindex{Schnitzler, Arthur@\textsc{Schnitzler, Arthur}!zzzLudwig, Emil@\emph{von Emil Ludwig}!1927-02-031@{3. 2. 1927}|)be}\mylabel{L02480h}  \newcommand{\dateiname}{L02480}\newcommand{\titel}{Emil Ludwig an Arthur Schnitzler, 3. 2. 1927}\newcommand{\editorInnen}{Martin Anton Müller und Gerd-Hermann Susen}%% latex-leseansicht-abspann.tex
%% Abspann für die Leseansicht.
%% Der Schalter \ifkorrekturansicht ist bereits durch den Vorspann gesetzt.

%% latex-abspann.tex
%% Gemeinsamer Abspann für Korrekturansicht und Leseansicht.
%% Setzt den Schalter \ifkorrekturansicht voraus (gesetzt in den
%% einbindenden Dateien latex-korrekturansicht-abspann.tex bzw.
%% latex-leseansicht-abspann.tex).
%% ---------------------------------------------------------------

\normalsize

% Das esempio-Environment wird nur in der Leseansicht benötigt
\ifkorrekturansicht\else
\newenvironment{esempio}[3]%
{
    \vspace{1.5ex}
    \rlap{\underline{#1}}
    \par
    \setlength{\parindent}{0cm}
    \nopagebreak
    \leftskip=#2cm
    \rightskip=#3cm
}
{
    \par
}
\fi

\doendnotes{C}
\bigskip
\vfill

\clearpage

\footnotesize

\ifkorrekturansicht
  \lohead{\textsc{register}}
\fi

% theindex-Environment neu definieren ohne reledmac
\makeatletter
\renewenvironment{theindex}{%
  \ifkorrekturansicht
    \section*{\indexname}%
  \else
    \subsubsection*{Index der erwähnten Entitäten}%
  \fi
  \setlength{\parindent}{0pt}%
  \setlength{\parskip}{0pt plus 0.3pt}%
  \let\item\@idxitem
}{%
  \ifkorrekturansicht\clearpage\fi
}
\makeatother

\IfFileExists{\jobname-pw.ind}{\input{\jobname-pw.ind}}{}

% Quellenangabe nur in der Leseansicht
\ifkorrekturansicht\else
% Fallback-Definitionen, falls die .tex-Datei \titel etc. nicht gesetzt hat
\providecommand{\titel}{}
\providecommand{\editorInnen}{}
\providecommand{\dateiname}{\jobname}

\vspace{3cm}

\vfill

\footnotesize
\textsc{Quelle}: \titel. Herausgegeben von {\editorInnen}. In: \emph{Arthur Schnitzler: Briefwechsel mit Autorinnen und Autoren}.
 Digitale Edition, https://schnitzler-briefe.acdh.oeaw.ac.at/{\dateiname}.html (Stand \today)
\fi

\end{document}


