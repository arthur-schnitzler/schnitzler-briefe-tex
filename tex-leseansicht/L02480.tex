%% latex-korrekturansicht-vorspann.tex
%% Vorspann für die Korrekturansicht.
%% Lädt die gemeinsame Datei latex-vorspann.tex mit gesetztem Schalter.

\newif\ifkorrekturansicht
\korrekturansichttrue

\input{../tex-inputs/latex-vorspann}


\section[Emil Ludwig an Arthur Schnitzler, 3. 2. 1927]{L02480 Emil Ludwig an Arthur Schnitzler, 3. 2. 1927}
\nopagebreak\mylabel{L02480v}
\rehead{ }\normalsize\beginnumbering\briefempfaengerindex{Schnitzler, Arthur@\textsc{Schnitzler, Arthur}!zzzLudwig, Emil@\emph{von Emil Ludwig}!1927-02-031@{3. 2. 1927}|(be}
\toendnotes[C]{\smallbreak\pagebreak[2]}\Standort{CUL, Schnitzler, B 62.}
\physDesc{Bildpostkarte, 170 Zeichen
\newline{}Handschrift: schwarze Tinte, lateinische Kurrent
\newline{}Versand: Stempel: »\nobreak{}\oindex{Ascona@\textbf{Ascona}, \emph{P.PPL}|pwk}A\textcolor{gray}{s}cona, 3. II. 27, 18\nobreak{}«.  }\toendnotes[C]{\smallbreak}\pstart{}{\pb}Herrn\pend{}\pstart{}Dr. Arthur Schnitzler\pend{}\pstart{}Adr. S. Fischer Verlag\orgindex{S. Fischer Verlag@S. Fischer Verlag|pw}\pend{}\pstart{}90, Bülowstr. 90\oindex{Buelowstrasse@\textbf{Bülowstraße}, \emph{Straße (K.STR)}|pw}\pend{}\pstart{}Berlin W. 57\oindex{Berlin@\textbf{Berlin}, \emph{P.PPLC}|pw}\pend{}{\bigskip}
\pstart
           \noindent{}\centering{}{\pb}\textcolor{gray}{\textbf{{[}Statue eines Bogenschützens mit Blick auf
                     eine Brissago-Insel\oindex{Isole di Brissago@\textbf{Isole di Brissago}, \emph{T.ISLS}|pw}{]}}}\pend
           \vspace{1em}
\pstart
           \noindent{}{\pb}Dankbar für hochinteressantes Diagramm\pwindex{Geist im Wort und der Geist in der Tat@\emph{Der Geist im Wort und der Geist in der Tat}|pwv}, \pend
           
\pstart
           grüsst Sie Ihr verehrungsvoll ergebener{\\[\baselineskip]}\spacefill\mbox{Ludwig}\pend
           \leftskip=0em{}
\pstart
           Ascona\oindex{Ascona@\textbf{Ascona}, \emph{P.PPL}|pw}. 3. 2.\pend
           \selectlanguage{ngerman}\endnumbering\briefempfaengerindex{Schnitzler, Arthur@\textsc{Schnitzler, Arthur}!zzzLudwig, Emil@\emph{von Emil Ludwig}!1927-02-031@{3. 2. 1927}|)be}\mylabel{L02480h}  \normalsize

\doendnotes{C}
\bigskip
\vfill

\clearpage

\footnotesize

\lohead{\textsc{register}}

% Definiere theindex-Environment komplett neu ohne reledmac
\makeatletter
\renewenvironment{theindex}{%
  \section*{\indexname}%
  \setlength{\parindent}{0pt}%
  \setlength{\parskip}{0pt plus 0.3pt}%
  \let\item\@idxitem
}{%
  \clearpage
}
\makeatother

\IfFileExists{\jobname-pw.ind}{\input{\jobname-pw.ind}}{}

\end{document}

      