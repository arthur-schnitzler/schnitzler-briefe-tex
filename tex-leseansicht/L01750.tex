%% latex-leseansicht-vorspann.tex
%% Vorspann für die Leseansicht.
%% Lädt die gemeinsame Datei latex-vorspann.tex mit nicht gesetztem Schalter.

\newif\ifkorrekturansicht
\korrekturansichtfalse

\input{../tex-inputs/latex-vorspann}

\begin{center}
            \textcolor{red}{ENTWURF. ENTZIFFERUNG NOCH NICHT KORREKTURGELESEN}
                      \end{center}
            
               \section[Arthur Schnitzler an Hermann Bahr, 13. 1. 1908]{ Arthur Schnitzler an Hermann Bahr, 13. 1. 1908}\nopagebreak\mylabel{v}\rehead{ }\begin{ledgroupsized}[t]{13cm}\normalsize\beginnumbering\briefempfaengerindex{Bahr, Hermann@\textsc{Bahr, Hermann}!zzzSchnitzler, Arthur@\emph{von Arthur Schnitzler}!1908-01-131@{13. 1. 1908}|(be} \toendnotes[C]{\smallbreak\pagebreak[2]} \Standort{TMW, HS AM 60171 Ba.}
\physDesc{Briefkarte
\newline{}Handschrift: schwarze Tinte, lateinische Kurrent\newline{}Ordnung: Lochung }\buchAbdrucke{\weitereDrucke{1) \emph{13. 1. 1908, Abschrift.} In: Arthur Schnitzler: \emph{The Letters of Arthur Schnitzler to Hermann Bahr}. Edited, annotated, and with an introduction, by Donald G.
                        Daviau. Chapel Hill: \emph{The University of North Carolina Press} 1978, S. 101 (University of North Carolina studies in the Germanic languages
                        and literatures, 89).} \weitereDrucke{2) Hermann Bahr, Arthur Schnitzler: \emph{Briefwechsel, Aufzeichnungen, Dokumente (1891–1931)}. Hg. Kurt Ifkovits und Martin Anton Müller. Göttingen: \emph{Wallstein} 2018, S. 401.} }\toendnotes[C]{\smallbreak}\pstart
           {\pb}13\damage{. 1}. 908\pend
           \pstart
           \textcolor{gray}{\textbf{Dr. Arthur Schnitzler}}\pend
           \pstart
           \textcolor{gray}{\textbf{Wien XVIII. Spoettelgasse 7\oindex{Edmund-Weiss-Gasse@\textbf{Edmund-Weiß-Gasse}|pw}.}}\pend
           \pstart
           mein lieber Hermann, erst heut dank ich dir für deinen guten Brief
               vom 23. v. M. Mit \label{K_L01750_1v}\edtext{Hebbelth\oindex{Hebbel-Theater@\textbf{Hebbel-Theater}|pw} hab ich abgeschlossen}{\lemma{\textnormal{\emph{Hebbelth … abgeschlossen}}}\Cendnote{\textnormal{vgl. Arthur Schnitzler an Hermann Bahr, 16. 12. 1907; Hermann Bahr an Arthur Schnitzler, 18. 12. 1907; Arthur Schnitzler an Hermann Bahr, 20. 12. 1907}}}\label{K_L01750_1h} – doch hör ich von Valentin\pwindex{Vallentin, Richard 03.02.1874 – 14.01.1908@\textsc{Vallentin, Richard} (03.02.1874 – 14.01.1908), \emph{Regisseur, Schauspieler}|pw}s \label{K_L01750_2v}\edtext{Gesundheitszustand}{\lemma{\textnormal{\emph{Gesundheitszustand}}}\Cendnote{\textnormal{Richard Vallentin\pwindex{Vallentin, Richard 03.02.1874 – 14.01.1908@\textsc{Vallentin, Richard} (03.02.1874 – 14.01.1908), \emph{Regisseur, Schauspieler}|pwk} starb am
                     14. 1. 1908.}}}\label{K_L01750_2h} ungünstiges. (Und über das Theater selbst\substVorne{}\textsuperscript{, }\substDazwischen{} (\substHinten{}unter uns) nichts sehr hoffnungsreiches.) Meine Frau\pwindex{Schnitzler, Olga 17.01.1882 – 13.01.1970@\textsc{Schnitzler, Olga} (17.01.1882 – 13.01.1970), \emph{Schauspielerin, Sängerin}|pwv} liegt noch, die Contumaz dauert etwa noch 10–14 Tage.
               Schreib mir {\pb}ein Wort,
                  \label{K_L01750_3v}\edtext{wa{\geminationn} du
               nach Berlin\oindex{Berlin@\textbf{Berlin}|pw} fährst}{\lemma{\textnormal{\emph{wa du
               nach Berlin fährst}}}\Cendnote{\textnormal{Bahr\pwindex{Bahr, Hermann 19.07.1863 – 15.01.1934@\textsc{Bahr, Hermann} (19.07.1863 – 15.01.1934), \emph{Schriftsteller, Kritiker}|pwk} begann am 18. 1. 1908 den vierten (und letzten) zweimonatigen Aufenthalt bei Max Reinhardt\pwindex{Reinhardt, Max 09.09.1873 – 30.10.1943@\textsc{Reinhardt, Max} (09.09.1873 – 30.10.1943), \emph{Theaterleiter, Regisseur, Schauspieler}|pwk} in Berlin\oindex{Berlin@\textbf{Berlin}|pwk}.}}}\label{K_L01750_3h}. Wie gern spräch ich dich bald wieder. Herzliche
               Grüße.\pend
           \pstart
           Dein{\\[\baselineskip]}\spacefill\mbox{Arthur}\pend
           \leftskip=0em{}\endnumbering\briefempfaengerindex{Bahr, Hermann@\textsc{Bahr, Hermann}!zzzSchnitzler, Arthur@\emph{von Arthur Schnitzler}!1908-01-131@{13. 1. 1908}|)be}\mylabel{h}\end{ledgroupsized}  \newcommand{\dateiname}{L01750}\newcommand{\titel}{Arthur Schnitzler an Hermann Bahr, 13. 1. 1908}\newcommand{\editorInnen}{ Kurt Ifkovits,  Martin Anton Müller}%% latex-leseansicht-abspann.tex
%% Abspann für die Leseansicht.
%% Der Schalter \ifkorrekturansicht ist bereits durch den Vorspann gesetzt.

%% latex-abspann.tex
%% Gemeinsamer Abspann für Korrekturansicht und Leseansicht.
%% Setzt den Schalter \ifkorrekturansicht voraus (gesetzt in den
%% einbindenden Dateien latex-korrekturansicht-abspann.tex bzw.
%% latex-leseansicht-abspann.tex).
%% ---------------------------------------------------------------

\normalsize

% Das esempio-Environment wird nur in der Leseansicht benötigt
\ifkorrekturansicht\else
\newenvironment{esempio}[3]%
{
    \vspace{1.5ex}
    \rlap{\underline{#1}}
    \par
    \setlength{\parindent}{0cm}
    \nopagebreak
    \leftskip=#2cm
    \rightskip=#3cm
}
{
    \par
}
\fi

\doendnotes{C}
\bigskip
\vfill

\clearpage

\footnotesize

\ifkorrekturansicht
  \lohead{\textsc{register}}
\fi

% theindex-Environment neu definieren ohne reledmac
\makeatletter
\renewenvironment{theindex}{%
  \ifkorrekturansicht
    \section*{\indexname}%
  \else
    \subsubsection*{Index der erwähnten Entitäten}%
  \fi
  \setlength{\parindent}{0pt}%
  \setlength{\parskip}{0pt plus 0.3pt}%
  \let\item\@idxitem
}{%
  \ifkorrekturansicht\clearpage\fi
}
\makeatother

\IfFileExists{\jobname-pw.ind}{\input{\jobname-pw.ind}}{}

% Quellenangabe nur in der Leseansicht
\ifkorrekturansicht\else
% Fallback-Definitionen, falls die .tex-Datei \titel etc. nicht gesetzt hat
\providecommand{\titel}{}
\providecommand{\editorInnen}{}
\providecommand{\dateiname}{\jobname}

\vspace{3cm}

\vfill

\footnotesize
\textsc{Quelle}: \titel. Herausgegeben von {\editorInnen}. In: \emph{Arthur Schnitzler: Briefwechsel mit Autorinnen und Autoren}.
 Digitale Edition, https://schnitzler-briefe.acdh.oeaw.ac.at/{\dateiname}.html (Stand \today)
\fi

\end{document}


      