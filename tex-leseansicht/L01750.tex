%% latex-leseansicht-vorspann.tex
%% Vorspann für die Leseansicht.
%% Lädt die gemeinsame Datei latex-vorspann.tex mit nicht gesetztem Schalter.

\newif\ifkorrekturansicht
\korrekturansichtfalse

\input{../tex-inputs/latex-vorspann}


\section[Arthur Schnitzler an Hermann Bahr, 13. 1. 1908]{L01750 Arthur Schnitzler an Hermann Bahr, 13. 1. 1908}
\nopagebreak\mylabel{L01750v}
\rehead{ }\normalsize\beginnumbering\briefempfaengerindex{Bahr, Hermann@\textsc{Bahr, Hermann}!zzzSchnitzler, Arthur@\emph{von Arthur Schnitzler}!1908-01-131@{13. 1. 1908}|(be}
\toendnotes[C]{\smallbreak\pagebreak[2]}
\correspDesc{Versand  durch Arthur Schnitzler am 13. 1. 1908 in Wien
\newline{}Erhalt  durch Hermann Bahr im Zeitraum [13. 1. 1908
                  – 17. 1. 1908?] in Wien}\toendnotes[C]{\smallbreak}
\Standort{TMW, HS AM 60171 Ba.}
\physDesc{Briefkarte, 425 Zeichen
\newline{}Handschrift: schwarze Tinte, lateinische Kurrent
\newline{}Ordnung: Lochung }
\buchAbdrucke{\weitereDrucke{1) \emph{13. 1. 1908, Abschrift.} In: Arthur Schnitzler: \emph{The Letters of Arthur Schnitzler to Hermann Bahr}. Edited, annotated, and with an introduction, by Donald G. Daviau. Chapel Hill: \emph{The University of North Carolina Press} 1978, S. 101 (University of North Carolina studies in the Germanic languages
                        and literatures, 89).} \weitereDrucke{2) Hermann Bahr, Arthur Schnitzler: \emph{Briefwechsel, Aufzeichnungen, Dokumente (1891–1931)}. Herausgegeben von Kurt Ifkovits und Martin Anton Müller. Göttingen: \emph{Wallstein} 2018, S. 401.} }\toendnotes[C]{\smallbreak}
\pstart
           {\pb}13\damage{. 1}. 908\pend
           
\pstart
           \textcolor{gray}{\textbf{Dr. Arthur Schnitzler}}\pend
           
\pstart
           \textcolor{gray}{\textbf{Wien XVIII. Spoettelgasse 7\oindex{Wien@\textbf{Wien}!XVIII., Währing@\textbf{XVIII., Währing}!Edmund-Weiß-Gasse 7@\textbf{Edmund-Weiß-Gasse 7}, \emph{Wohngebäude}|pw}.}}\pend
           \vspace{0.5em}
\pstart
           mein lieber Hermann, erst heut dank ich dir für deinen guten Brief
               vom 23. v. M. Mit \label{K_L01750-1v}\edtext{Hebbelth\oindex{Hebbel-Theater@\textbf{Hebbel-Theater}, \emph{Theater}|pw} hab ich abgeschlossen}{\lemma{\textnormal{\emph{Hebbelth … abgeschlossen}}}\Cendnote{\textnormal{Vgl. XXXX Auszeichnungsfehler: Dokument L01741 nicht gefunden, XXXX Auszeichnungsfehler: Dokument L01742 nicht gefunden und XXXX Auszeichnungsfehler: Dokument L01743 nicht gefunden.
               }}}\label{K_L01750-1} – doch hör ich von Valentins\pwindex{Vallentin, Richard 3.\,2.\,1874 Luzern – 14.\,1.\,1908 Berlin@\textsc{Vallentin, Richard} (3.\,2.\,1874 Luzern – 14.\,1.\,1908 Berlin), \emph{Regisseur, Schauspieler}|pw}{ }\label{K_L01750-2v}\edtext{Gesundheitszustand}{\lemma{\textnormal{\emph{Gesundheitszustand}}}\Cendnote{\textnormal{Richard Vallentin\pwindex{Vallentin, Richard 3.\,2.\,1874 Luzern – 14.\,1.\,1908 Berlin@\textsc{Vallentin, Richard} (3.\,2.\,1874 Luzern – 14.\,1.\,1908 Berlin), \emph{Regisseur, Schauspieler}|pwk} starb am
                     14. 1. 1908.}}}\label{K_L01750-2} ungünstiges. (Und über das Theater selbst\substVorne{}\textsuperscript{,{ }}\substDazwischen{}{ }(\substHinten{}unter uns) nichts sehr hoffnungsreiches.) Meine Frau\pwindex{Schnitzler, Olga 17.\,1.\,1882 Wien – 13.\,1.\,1970 Lugano@\textsc{Schnitzler, Olga} (17.\,1.\,1882 Wien – 13.\,1.\,1970 Lugano), \emph{Schauspielerin, Sängerin}|pwv} liegt noch, die Contumaz dauert etwa
               noch 10–14 Tage. Schreib mir {\pb}ein Wort, \label{K_L01750-3v}\edtext{wa{\geminationn} du nach
                  Berlin\oindex{Berlin@\textbf{Berlin}, \emph{Hauptstadt}|pw} fährst}{\lemma{\textnormal{\emph{wann … fährst}}}\Cendnote{\textnormal{Bahr\pwindex{Bahr, Hermann 19.\,7.\,1863 Linz – 15.\,1.\,1934 München@\textsc{Bahr, Hermann} (19.\,7.\,1863 Linz – 15.\,1.\,1934 München), \emph{Schriftsteller, Kritiker}|pwk} begann am 18. 1. 1908 den vierten (und letzten) zweimonatigen Aufenthalt bei Max Reinhardt\pwindex{Reinhardt, Max 9.\,9.\,1873 Baden bei Wien – 30.\,10.\,1943 New York City@\textsc{Reinhardt, Max} (9.\,9.\,1873 Baden bei Wien – 30.\,10.\,1943 New York City), \emph{Theaterleiter, Regisseur, Schauspieler}|pwk} in Berlin\oindex{Berlin@\textbf{Berlin}, \emph{Hauptstadt}|pwk}.}}}\label{K_L01750-3}. Wie gern spräch ich dich bald wieder.
               Herzliche Grüße.\pend
           
\pstart
           Dein{\\[\baselineskip]}\spacefill\mbox{Arthur}\pend
           \leftskip=0em{}\selectlanguage{ngerman}\endnumbering\briefempfaengerindex{Bahr, Hermann@\textsc{Bahr, Hermann}!zzzSchnitzler, Arthur@\emph{von Arthur Schnitzler}!1908-01-131@{13. 1. 1908}|)be}\mylabel{L01750h}  \newcommand{\dateiname}{L01750}\newcommand{\titel}{Arthur Schnitzler an Hermann Bahr, 13. 1. 1908}\newcommand{\editorInnen}{Herausgegeben von Martin Anton Müller}%% latex-leseansicht-abspann.tex
%% Abspann für die Leseansicht.
%% Der Schalter \ifkorrekturansicht ist bereits durch den Vorspann gesetzt.

%% latex-abspann.tex
%% Gemeinsamer Abspann für Korrekturansicht und Leseansicht.
%% Setzt den Schalter \ifkorrekturansicht voraus (gesetzt in den
%% einbindenden Dateien latex-korrekturansicht-abspann.tex bzw.
%% latex-leseansicht-abspann.tex).
%% ---------------------------------------------------------------

\normalsize

% Das esempio-Environment wird nur in der Leseansicht benötigt
\ifkorrekturansicht\else
\newenvironment{esempio}[3]%
{
    \vspace{1.5ex}
    \rlap{\underline{#1}}
    \par
    \setlength{\parindent}{0cm}
    \nopagebreak
    \leftskip=#2cm
    \rightskip=#3cm
}
{
    \par
}
\fi

\doendnotes{C}
\bigskip
\vfill

\clearpage

\footnotesize

\ifkorrekturansicht
  \lohead{\textsc{register}}
\fi

% theindex-Environment neu definieren ohne reledmac
\makeatletter
\renewenvironment{theindex}{%
  \ifkorrekturansicht
    \section*{\indexname}%
  \else
    \subsubsection*{Index der erwähnten Entitäten}%
  \fi
  \setlength{\parindent}{0pt}%
  \setlength{\parskip}{0pt plus 0.3pt}%
  \let\item\@idxitem
}{%
  \ifkorrekturansicht\clearpage\fi
}
\makeatother

\IfFileExists{\jobname-pw.ind}{\input{\jobname-pw.ind}}{}

% Quellenangabe nur in der Leseansicht
\ifkorrekturansicht\else
% Fallback-Definitionen, falls die .tex-Datei \titel etc. nicht gesetzt hat
\providecommand{\titel}{}
\providecommand{\editorInnen}{}
\providecommand{\dateiname}{\jobname}

\vspace{3cm}

\vfill

\footnotesize
\textsc{Quelle}: \titel. Herausgegeben von {\editorInnen}. In: \emph{Arthur Schnitzler: Briefwechsel mit Autorinnen und Autoren}.
 Digitale Edition, https://schnitzler-briefe.acdh.oeaw.ac.at/{\dateiname}.html (Stand \today)
\fi

\end{document}


