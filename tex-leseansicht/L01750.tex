%% latex-korrekturansicht-vorspann.tex
%% Vorspann für die Korrekturansicht.
%% Lädt die gemeinsame Datei latex-vorspann.tex mit gesetztem Schalter.

\newif\ifkorrekturansicht
\korrekturansichttrue

\input{../tex-inputs/latex-vorspann}


\section[Arthur Schnitzler an Hermann Bahr, 13. 1. 1908]{L01750 Arthur Schnitzler an Hermann Bahr, 13. 1. 1908}
\nopagebreak\mylabel{L01750v}
\rehead{ }\normalsize\beginnumbering\briefempfaengerindex{Bahr, Hermann@\textsc{Bahr, Hermann}!zzzSchnitzler, Arthur@\emph{von Arthur Schnitzler}!1908-01-131@{13. 1. 1908}|(be}
\toendnotes[C]{\smallbreak\pagebreak[2]}\Standort{TMW, HS AM 60171 Ba.}
\physDesc{Briefkarte, 425 Zeichen
\newline{}Handschrift: schwarze Tinte, lateinische Kurrent
\newline{}Ordnung: Lochung }
\buchAbdrucke{\weitereDrucke{1) Arthur Schnitzler: \emph{The Letters of Arthur Schnitzler to Hermann Bahr}. Chapel Hill: \emph{The University of North Carolina Press} 1978, S. 101.} \weitereDrucke{2) Hermann Bahr, Arthur Schnitzler: \emph{Briefwechsel, Aufzeichnungen, Dokumente (1891–1931)}. Göttingen: \emph{Wallstein} 2018, S. 401.} }\toendnotes[C]{\smallbreak}
\pstart
           {\pb}13\damage{. 1}. 908\pend
           
\pstart
           \textcolor{gray}{\textbf{Dr. Arthur Schnitzler}}\pend
           
\pstart
           \textcolor{gray}{\textbf{Wien XVIII. Spoettelgasse 7\oindex{Edmund-Weiss-Gasse 7@\textbf{Edmund-Weiß-Gasse 7}, \emph{Wohngebäude (K.WHS)}|pw}.}}\pend
           \vspace{0.5em}
\pstart
           mein lieber Hermann, erst heut dank ich dir für deinen guten Brief
               vom 23. v. M. Mit \label{K_L01750-1v}\edtext{Hebbelth\oindex{Hebbel-Theater@\textbf{Hebbel-Theater}, \emph{Theater (K.THE)}|pw} hab ich abgeschlossen}{\lemma{\textnormal{\emph{Hebbelth … abgeschlossen}}}\Cendnote{\textnormal{Vgl. Arthur Schnitzler an Hermann Bahr, 16. 12. 1907, Hermann Bahr an Arthur Schnitzler, 18. 12. 1907 und Arthur Schnitzler an Hermann Bahr, 20. 12. 1907.
               }}}\label{K_L01750-1} – doch hör ich von Valentins\pwindex{Vallentin, Richard 03.02.1874 – 14.01.1908@\textsc{Vallentin, Richard} (03.02.1874 – 14.01.1908), \emph{Regisseur/Regisseurin, Schauspieler/Schauspielerin}|pw}{ }\label{K_L01750-2v}\edtext{Gesundheitszustand}{\lemma{\textnormal{\emph{Gesundheitszustand}}}\Cendnote{\textnormal{Richard Vallentin\pwindex{Vallentin, Richard 03.02.1874 – 14.01.1908@\textsc{Vallentin, Richard} (03.02.1874 – 14.01.1908), \emph{Regisseur/Regisseurin, Schauspieler/Schauspielerin}|pwk} starb am
                     14. 1. 1908.}}}\label{K_L01750-2} ungünstiges. (Und über das Theater selbst\substVorne{}\textsuperscript{,{ }}\substDazwischen{}{ }(\substHinten{}unter uns) nichts sehr hoffnungsreiches.) Meine Frau\pwindex{Schnitzler, Olga 17.01.1882 – 13.01.1970@\textsc{Schnitzler, Olga} (17.01.1882 – 13.01.1970), \emph{Schauspieler/Schauspielerin, Sänger/Sängerin}|pwv} liegt noch, die Contumaz dauert etwa
               noch 10–14 Tage. Schreib mir {\pb}ein Wort, \label{K_L01750-3v}\edtext{wa{\geminationn} du nach
                  Berlin\oindex{Berlin@\textbf{Berlin}, \emph{P.PPLC}|pw} fährst}{\lemma{\textnormal{\emph{wann … fährst}}}\Cendnote{\textnormal{Bahr\pwindex{Bahr, Hermann 19.07.1863 – 15.01.1934@\textsc{Bahr, Hermann} (19.07.1863 – 15.01.1934), \emph{Schriftsteller/Schriftstellerin, Kritiker/Kritikerin}|pwk} begann am 18. 1. 1908 den vierten (und letzten) zweimonatigen Aufenthalt bei Max Reinhardt\pwindex{Reinhardt, Max 09.09.1873 – 30.10.1943@\textsc{Reinhardt, Max} (09.09.1873 – 30.10.1943), \emph{Theaterleiter/Theaterleiterin, Regisseur/Regisseurin, Schauspieler/Schauspielerin}|pwk} in Berlin\oindex{Berlin@\textbf{Berlin}, \emph{P.PPLC}|pwk}.}}}\label{K_L01750-3}. Wie gern spräch ich dich bald wieder.
               Herzliche Grüße.\pend
           
\pstart
           Dein{\\[\baselineskip]}\spacefill\mbox{Arthur}\pend
           \leftskip=0em{}\selectlanguage{ngerman}\endnumbering\briefempfaengerindex{Bahr, Hermann@\textsc{Bahr, Hermann}!zzzSchnitzler, Arthur@\emph{von Arthur Schnitzler}!1908-01-131@{13. 1. 1908}|)be}\mylabel{L01750h}  \normalsize

\doendnotes{C}
\bigskip
\vfill

\clearpage

\footnotesize

\lohead{\textsc{register}}

% Definiere theindex-Environment komplett neu ohne reledmac
\makeatletter
\renewenvironment{theindex}{%
  \section*{\indexname}%
  \setlength{\parindent}{0pt}%
  \setlength{\parskip}{0pt plus 0.3pt}%
  \let\item\@idxitem
}{%
  \clearpage
}
\makeatother

\IfFileExists{\jobname-pw.ind}{\input{\jobname-pw.ind}}{}

\end{document}

      