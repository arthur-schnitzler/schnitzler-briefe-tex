%% latex-korrekturansicht-vorspann.tex
%% Vorspann für die Korrekturansicht.
%% Lädt die gemeinsame Datei latex-vorspann.tex mit gesetztem Schalter.

\newif\ifkorrekturansicht
\korrekturansichttrue

\input{../tex-inputs/latex-vorspann}


\section[Peter Altenberg an Arthur Schnitzler, 1. 12. 191{[}2{]}]{L02106 Peter Altenberg an Arthur Schnitzler, 1. 12. 191{[}2{]}}
\nopagebreak\mylabel{L02106v}
\rehead{ }\normalsize\beginnumbering\briefempfaengerindex{Schnitzler, Arthur@\textsc{Schnitzler, Arthur}!zzzAltenberg, Peter@\emph{von Peter Altenberg}!1912-12-011@{1. 12. 191{[}2{]}}|(be}
\toendnotes[C]{\smallbreak\pagebreak[2]}\Standort{CUL, Schnitzler, B 2.}
\physDesc{Telegramm, 111 Zeichen
\newline{}Handschrift einer Schreibkraft: blauer Buntstift, lateinische Kurrent
\newline{}Versand: »\noindent{}Semmering\oindex{Semmering@\textbf{Semmering}, \emph{A.ADM3}|pw}{ }\textcolor{gray}{\textbf{Nr.}} 9{ }\textcolor{gray}{\textbf{Taxw.}} 26{ }\textcolor{gray}{\textbf{(W{\dots} Ch{\dots}) aufgegeben am}}{ }1/12\textcolor{gray}{\textbf{191{\dots} um}} 9 \textcolor{gray}{\textbf{Uhr}} 40 \textcolor{gray}{\textbf{Mittag.}}« 
\newline{}Schnitzler: mit Bleistift datiert: »1/12 912« 
\newline{}Ordnung: 1) mit Bleistift von unbekannter Hand nummeriert:
                                    »11«  2) Adresszeile teilweise beschnitten}
\buchAbdrucke{\weitereDrucke{1) \emph{Studies in Arthur Schnitzler. Centennial Commemorative
                        Volume}. Chapel Hill: \emph{University of North Carolina Press} 1963, S. 22.} \weitereDrucke{2) Arthur Schnitzler: \emph{Das Wort. Tragikomödie in fünf Akten. Fragment}. Frankfurt am Main: \emph{S. Fischer Verlag} 1966, S. 10.} }
\pstart
           \noindent{}{\pb}Unter heissen tränen meinen dank kann
               nicht schreiben es wird nicht mehr lang dauern ihr unglücklicher\pend
           \pstart \spacefill\mbox{Altenberg}\pend{}\selectlanguage{ngerman}\endnumbering\briefempfaengerindex{Schnitzler, Arthur@\textsc{Schnitzler, Arthur}!zzzAltenberg, Peter@\emph{von Peter Altenberg}!1912-12-011@{1. 12. 191{[}2{]}}|)be}\mylabel{L02106h}  \normalsize

\doendnotes{C}
\bigskip
\vfill

\clearpage

\footnotesize

\lohead{\textsc{register}}

% Definiere theindex-Environment komplett neu ohne reledmac
\makeatletter
\renewenvironment{theindex}{%
  \section*{\indexname}%
  \setlength{\parindent}{0pt}%
  \setlength{\parskip}{0pt plus 0.3pt}%
  \let\item\@idxitem
}{%
  \clearpage
}
\makeatother

\IfFileExists{\jobname-pw.ind}{\input{\jobname-pw.ind}}{}

\end{document}

      