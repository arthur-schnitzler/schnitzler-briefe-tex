%% latex-leseansicht-vorspann.tex
%% Vorspann für die Leseansicht.
%% Lädt die gemeinsame Datei latex-vorspann.tex mit nicht gesetztem Schalter.

\newif\ifkorrekturansicht
\korrekturansichtfalse

\input{../tex-inputs/latex-vorspann}


         \renewcommand{\erwaehnteOrte}{Orte: Semmering, Wien}
         \renewcommand{\erwaehnteWerke}{
               \section[Peter Altenberg an Arthur Schnitzler, 1. 12. 191{[}2{]}]{ Peter Altenberg an Arthur Schnitzler, 1. 12. 191{[}2{]}}\nopagebreak\mylabel{v}\rehead{ }\begin{ledgroupsized}[t]{13cm}\normalsize\beginnumbering \toendnotes[C]{\smallbreak\pagebreak[2]} \Standort{CUL, Schnitzler, B 2.}
\physDesc{Telegramm
\newline{}Handschrift einer Schreibkraft: blauer Buntstift, lateinische Kurrent\newline{}Versand: »\noindent{}Semmering\oindex{Semmering@\textbf{Semmering}|pw}{ }\textcolor{gray}{\textbf{Nr.}} 9{ }\textcolor{gray}{\textbf{Taxw.}} 26{ }\textcolor{gray}{\textbf{(W{\dots} Ch{\dots}) aufgegeben
                                                  am}}{ }1/12 \textcolor{gray}{\textbf{191{\dots} um}} 9 \textcolor{gray}{\textbf{Uhr}} 40 \textcolor{gray}{\textbf{Mittag.}}« 
\newline{}Schnitzler: mit Bleistift datiert: »1/12 912« \newline{}Ordnung: 1) mit Bleistift von unbekannter Hand nummeriert:
                                                »11«  2) Adresszeile teilweise beschnitten}\buchAbdrucke{\weitereDrucke{1) Kurt Bergel: \emph{Arthur Schnitzlers unveröffentlichte Tragikomödie Das Wort.} In: \emph{Studies in Arthur Schnitzler. Centennial Commemorative
                                Volume}. Hg. Herbert W. Reichert und Herman Salinger. Chapel Hill: \emph{University of North Carolina Press} 1963, S. 22 (UNC Studies in the Germanic Languages and
                                Literatures, 42).} \weitereDrucke{2) Arthur Schnitzler: \emph{Das Wort. Tragikomödie in fünf Akten. Fragment}. Aus dem Nachlaß hg. und eingeleitet von Kurt Bergel. Frankfurt am Main: \emph{S. Fischer Verlag} 1966, S. 10.} }\pstart
           \noindent{}{\pb}Unter heissen tränen meinen dank
                    kann nicht schreiben es wird nicht mehr lang dauern ihr unglücklicher\pend
           \pstart \spacefill\mbox{Altenberg}\pend{}
         
         \endnumbering\mylabel{h}\end{ledgroupsized}  \newcommand{\dateiname}{L02106}\newcommand{\titel}{Peter Altenberg an Arthur Schnitzler, 1. 12. 191[2]}\newcommand{\editorInnen}{Martin Anton Müller und Gerd-Hermann Susen}%% latex-leseansicht-abspann.tex
%% Abspann für die Leseansicht.
%% Der Schalter \ifkorrekturansicht ist bereits durch den Vorspann gesetzt.

%% latex-abspann.tex
%% Gemeinsamer Abspann für Korrekturansicht und Leseansicht.
%% Setzt den Schalter \ifkorrekturansicht voraus (gesetzt in den
%% einbindenden Dateien latex-korrekturansicht-abspann.tex bzw.
%% latex-leseansicht-abspann.tex).
%% ---------------------------------------------------------------

\normalsize

% Das esempio-Environment wird nur in der Leseansicht benötigt
\ifkorrekturansicht\else
\newenvironment{esempio}[3]%
{
    \vspace{1.5ex}
    \rlap{\underline{#1}}
    \par
    \setlength{\parindent}{0cm}
    \nopagebreak
    \leftskip=#2cm
    \rightskip=#3cm
}
{
    \par
}
\fi

\doendnotes{C}
\bigskip
\vfill

\clearpage

\footnotesize

\ifkorrekturansicht
  \lohead{\textsc{register}}
\fi

% theindex-Environment neu definieren ohne reledmac
\makeatletter
\renewenvironment{theindex}{%
  \ifkorrekturansicht
    \section*{\indexname}%
  \else
    \subsubsection*{Index der erwähnten Entitäten}%
  \fi
  \setlength{\parindent}{0pt}%
  \setlength{\parskip}{0pt plus 0.3pt}%
  \let\item\@idxitem
}{%
  \ifkorrekturansicht\clearpage\fi
}
\makeatother

\IfFileExists{\jobname-pw.ind}{\input{\jobname-pw.ind}}{}

% Quellenangabe nur in der Leseansicht
\ifkorrekturansicht\else
% Fallback-Definitionen, falls die .tex-Datei \titel etc. nicht gesetzt hat
\providecommand{\titel}{}
\providecommand{\editorInnen}{}
\providecommand{\dateiname}{\jobname}

\vspace{3cm}

\vfill

\footnotesize
\textsc{Quelle}: \titel. Herausgegeben von {\editorInnen}. In: \emph{Arthur Schnitzler: Briefwechsel mit Autorinnen und Autoren}.
 Digitale Edition, https://schnitzler-briefe.acdh.oeaw.ac.at/{\dateiname}.html (Stand \today)
\fi

\end{document}


      