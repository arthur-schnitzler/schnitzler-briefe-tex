%% latex-leseansicht-vorspann.tex
%% Vorspann für die Leseansicht.
%% Lädt die gemeinsame Datei latex-vorspann.tex mit nicht gesetztem Schalter.

\newif\ifkorrekturansicht
\korrekturansichtfalse

\input{../tex-inputs/latex-vorspann}


         
         \renewcommand{\erwaehntePersonen}{Personen: Richard Beer-Hofmann, Hugo von Hofmannsthal, Olga Schnitzler, Heinrich Schnitzler, Karl Werkmann}
         \renewcommand{\erwaehnteInstitutionen}{Institutionen: Franz-Grillparzer-Preis}
         \renewcommand{\erwaehnteOrte}{Orte: Semmering, Wien}
         \renewcommand{\erwaehnteWerke}{
               \section[Arthur Schnitzler an Hugo von Hofmannsthal, 25. 1. 1908]{ Arthur Schnitzler an Hugo von Hofmannsthal, 25. 1. 1908}\nopagebreak\mylabel{v}\rehead{ }\begin{ledgroupsized}[t]{13cm}\normalsize\beginnumbering \toendnotes[C]{\smallbreak\pagebreak[2]} \Standort{FDH, Hs-30885,131.}
\physDesc{Brief, 1 Blatt, 4 Seiten
\newline{}Handschrift: blaue Tinte, deutsche Kurrent}\buchAbdrucke{\weitereDrucke{Hugo von Hofmannsthal, Arthur Schnitzler: \emph{Briefwechsel}. Hg. Therese Nickl und Heinrich Schnitzler. Frankfurt am Main: \emph{S. Fischer} 1964, S. 235–236.} }\toendnotes[C]{\smallbreak}\pstart
           \raggedleft{}{\pb}25. 1 908\pend
           \pstart{}mein lieber Hugo,\pend\pstart
           die Verhältniſſe nähern ſich ſehr allmälig dem \label{K_L01758_1v}\edtext{\textsc{soi disant}}{\lemma{\textnormal{\emph{soi disant}}}\Cendnote{\textnormal{französisch:
                  sogenannt.}}}\label{K_L01758_1h} Normalen. Die Wohnung iſt desinfizirt, Olga\pwindex{Schnitzler, Olga 17.01.1882 – 13.01.1970@\textsc{Schnitzler, Olga} (17.01.1882 – 13.01.1970), \emph{Schauspielerin, Sängerin}|pw}{ }ſchon viel außer Bett; Heini\pwindex{Schnitzler, Heinrich 09.08.1902 – 12.07.1982@\textsc{Schnitzler, Heinrich} (09.08.1902 – 12.07.1982), \emph{Regisseur, Schauspieler}|pw} noch nicht zu Haus; aber ich treffe ihn zuweilen. –\pend
           \pstart
           In etwa 10 Tagen wollen wir auf den Se{\geminationm}ering\oindex{Semmering@\textbf{Semmering}|pw} (jetzt, heißt es, iſt Influenza oben) und
               etwa 8 Tage oder länger, mit Heini\pwindex{Schnitzler, Heinrich 09.08.1902 – 12.07.1982@\textsc{Schnitzler, Heinrich} (09.08.1902 – 12.07.1982), \emph{Regisseur, Schauspieler}|pw} oben bleiben –
                  da{\geminationn} erſt oeffnen ſich wieder unſeres Hauſes
               Pforten.\pend
           \pstart
           {\pb}Vielleicht ſieht man ſich vorher ſchon in neutralem
               Gebiet –? Ich möchte gern näheres über Sie, von Ihnen wiſſen, von andern, ſelbſt we{\geminationn} die andern Richard\pwindex{Beer-Hofmann, Richard 1866-07-11 – 1945-09-26@\textsc{Beer-Hofmann, Richard} (1866-07-11 – 1945-09-26), \emph{Schriftsteller}|pw}s ſind, erfährt man doch nicht genug.\pend
           \pstart
           Mit edler Geste ſchuppſen Sie mir den Grillparzerpreis\orgindex{Franz-Grillparzer-Preis@Franz-Grillparzer-Preis|pw} wieder zurück – i{\geminationm}erhin bin
               ich froh, daſs ich ihn direct bekommen hab – es vereinfacht die Einkaſſirung. Mit
                  »\textsc{Interviewern}\pwindex{Werkmann, Karl 14.09.1878 – 24.12.1951@\textsc{Werkmann, Karl} (14.09.1878 – 24.12.1951), \emph{Journalist}|pwv}« ſoll man natürlich nie ſprechen (we{\geminationn} man ihnen
               nicht dictirt, wie es andere thun) {\pb}ja man ſoll ſie nicht
               empfangen, was aber ſchwer iſt, wenn ſie hinter einem Stubenmädl die öffnet, direct
               ins Zimmer ſtürzen, ohne Meldung abzuwarten, – oder man ſoll ſie hinauswerfen – was
               auch wieder ſchwer iſt, we{\geminationn} man nicht weiſs, wer ſie
               ſind und ſie plötzlich aus heiterm oder vielmehr bewölktem Himmel einem Glückwünſche
               zu unvermutet erſchienenen fünftauſend Kronen (nebſt Ehre, Auszeichnung u Lorbeer) zu
               Füßen legen. Übrigens werd ich Ihnen {\pb}nächſtens noch
               etwas Komiſches vom Vormittag des 15. Januar erzählen.\pend
           \pstart
           Zur Arbeit fühl ich mich ſchon ſehr bereit; an Tagen, da man innerlich u äußerlich
               allerlei ordnen konnte, un\textcolor{gray}{d} ſelbſt an Einfällen hat es \strikeout{mir} nicht gefehlt.\pend
           \pstart
           Wie gehts Ihnen Allen? Olga\pwindex{Schnitzler, Olga 17.01.1882 – 13.01.1970@\textsc{Schnitzler, Olga} (17.01.1882 – 13.01.1970), \emph{Schauspielerin, Sängerin}|pw} iſt über die
               prachtvolle Schale ſehr froh. Ich hab sie \introOben{}ihr\introOben{} erſt im
               desinfizirten Raum übergeben.\pend
           \pstart
           Wir grüßen Euch! Laßt was hören!{\\[\baselineskip]}\spacefill\mbox{Arthur}\pend
           \leftskip=0em{}
         
         \endnumbering\mylabel{h}\end{ledgroupsized}  \newcommand{\dateiname}{L01758}\newcommand{\titel}{Arthur Schnitzler an Hugo von Hofmannsthal, 25. 1. 1908}\newcommand{\editorInnen}{Martin Anton Müller und Gerd-Hermann Susen}%% latex-leseansicht-abspann.tex
%% Abspann für die Leseansicht.
%% Der Schalter \ifkorrekturansicht ist bereits durch den Vorspann gesetzt.

%% latex-abspann.tex
%% Gemeinsamer Abspann für Korrekturansicht und Leseansicht.
%% Setzt den Schalter \ifkorrekturansicht voraus (gesetzt in den
%% einbindenden Dateien latex-korrekturansicht-abspann.tex bzw.
%% latex-leseansicht-abspann.tex).
%% ---------------------------------------------------------------

\normalsize

% Das esempio-Environment wird nur in der Leseansicht benötigt
\ifkorrekturansicht\else
\newenvironment{esempio}[3]%
{
    \vspace{1.5ex}
    \rlap{\underline{#1}}
    \par
    \setlength{\parindent}{0cm}
    \nopagebreak
    \leftskip=#2cm
    \rightskip=#3cm
}
{
    \par
}
\fi

\doendnotes{C}
\bigskip
\vfill

\clearpage

\footnotesize

\ifkorrekturansicht
  \lohead{\textsc{register}}
\fi

% theindex-Environment neu definieren ohne reledmac
\makeatletter
\renewenvironment{theindex}{%
  \ifkorrekturansicht
    \section*{\indexname}%
  \else
    \subsubsection*{Index der erwähnten Entitäten}%
  \fi
  \setlength{\parindent}{0pt}%
  \setlength{\parskip}{0pt plus 0.3pt}%
  \let\item\@idxitem
}{%
  \ifkorrekturansicht\clearpage\fi
}
\makeatother

\IfFileExists{\jobname-pw.ind}{\input{\jobname-pw.ind}}{}

% Quellenangabe nur in der Leseansicht
\ifkorrekturansicht\else
% Fallback-Definitionen, falls die .tex-Datei \titel etc. nicht gesetzt hat
\providecommand{\titel}{}
\providecommand{\editorInnen}{}
\providecommand{\dateiname}{\jobname}

\vspace{3cm}

\vfill

\footnotesize
\textsc{Quelle}: \titel. Herausgegeben von {\editorInnen}. In: \emph{Arthur Schnitzler: Briefwechsel mit Autorinnen und Autoren}.
 Digitale Edition, https://schnitzler-briefe.acdh.oeaw.ac.at/{\dateiname}.html (Stand \today)
\fi

\end{document}


      