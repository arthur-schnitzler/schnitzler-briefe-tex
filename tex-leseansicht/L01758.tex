%% latex-korrekturansicht-vorspann.tex
%% Vorspann für die Korrekturansicht.
%% Lädt die gemeinsame Datei latex-vorspann.tex mit gesetztem Schalter.

\newif\ifkorrekturansicht
\korrekturansichttrue

\input{../tex-inputs/latex-vorspann}


\section[Arthur Schnitzler an Hugo von Hofmannsthal, 25. 1. 1908]{L01758 Arthur Schnitzler an Hugo von Hofmannsthal, 25. 1. 1908}
\nopagebreak\mylabel{L01758v}
\rehead{ }\normalsize\beginnumbering\briefempfaengerindex{Hofmannsthal, Hugo von@\textsc{Hofmannsthal, Hugo von}!zzzSchnitzler, Arthur@\emph{von Arthur Schnitzler}!1908-01-251@{25. 1. 1908}|(be}
\toendnotes[C]{\smallbreak\pagebreak[2]}\Standort{FDH, Hs-30885,131.}
\physDesc{Brief, 1 Blatt, 4 Seiten, 1659 Zeichen
\newline{}Handschrift: blaue Tinte, deutsche Kurrent}
\buchAbdrucke{\weitereDrucke{Hugo von Hofmannsthal, Arthur Schnitzler: \emph{Briefwechsel}. Frankfurt am Main: \emph{S. Fischer} 1964, S. 235–236.} }\toendnotes[C]{\smallbreak}
\pstart
           \raggedleft{}{\pb}25. 1 908\pend
           
\pstart{}mein lieber Hugo,\pend\vspace{0.5em}
\pstart
           die Verhältniſſe nähern ſich ſehr allmälig dem \label{K_L01758-1v}\edtext{\textsc{soi disant}}{\lemma{\textnormal{\emph{soi disant}}}\Cendnote{\textnormal{französisch: sogenannt}}}\label{K_L01758-1} Normalen.
               Die Wohnung iſt desinfizirt, Olga\pwindex{Schnitzler, Olga 17.01.1882 – 13.01.1970@\textsc{Schnitzler, Olga} (17.01.1882 – 13.01.1970), \emph{Schauspieler/Schauspielerin, Sänger/Sängerin}|pw}{ }ſchon viel außer Bett; Heini\pwindex{Schnitzler, Heinrich 09.08.1902 – 12.07.1982@\textsc{Schnitzler, Heinrich} (09.08.1902 – 12.07.1982), \emph{Regisseur/Regisseurin, Schauspieler/Schauspielerin}|pw} noch nicht zu Haus; aber ich treffe ihn zuweilen. –\pend
           
\pstart
           In etwa 10 Tagen wollen wir auf den Se{\geminationm}ering\oindex{Semmering@\textbf{Semmering}, \emph{A.ADM3}|pw} (jetzt, heißt es, iſt Influenza oben) und
               etwa 8 Tage oder länger, mit Heini\pwindex{Schnitzler, Heinrich 09.08.1902 – 12.07.1982@\textsc{Schnitzler, Heinrich} (09.08.1902 – 12.07.1982), \emph{Regisseur/Regisseurin, Schauspieler/Schauspielerin}|pw} oben
               bleiben – da{\geminationn} erſt oeffnen ſich wieder unſeres Hauſes
               Pforten.\pend
           
\pstart
           {\pb}Vielleicht ſieht man ſich vorher ſchon in neutralem
               Gebiet –? Ich möchte gern näheres über Sie, von Ihnen wiſſen, von andern, ſelbſt
                  \label{K_L01758-2v}\edtext{we{\geminationn}
               die andern Richards\pwindex{Beer-Hofmann, Richard 1866-07-11 – 1945-09-26@\textsc{Beer-Hofmann, Richard} (1866-07-11 – 1945-09-26), \emph{Schriftsteller/Schriftstellerin}|pw} ſind}{\lemma{\textnormal{\emph{wenn … ſind}}}\Cendnote{\textnormal{Beer-Hofmann\pwindex{Beer-Hofmann, Richard 1866-07-11 – 1945-09-26@\textsc{Beer-Hofmann, Richard} (1866-07-11 – 1945-09-26), \emph{Schriftsteller/Schriftstellerin}|pwk} und Schnitzler waren am 23. 1. 1908 gemeinsam
                  spazieren.}}}\label{K_L01758-2},
               erfährt man doch nicht genug.\pend
           
\pstart
           Mit edler Geste ſchuppſen Sie mir den Grillparzerpreis\orgindex{Franz-Grillparzer-Preis@Franz-Grillparzer-Preis|pw} wieder zurück – i{\geminationm}erhin bin
               ich froh, daſs ich ihn direct bekommen hab – es vereinfacht die Einkaſſirung. Mit
                  »\textsc{Interviewern}\pwindex{Werkmann, Karl 14.09.1878 – 24.12.1951@\textsc{Werkmann, Karl} (14.09.1878 – 24.12.1951), \emph{Journalist/Journalistin}|pwv}« ſoll man natürlich nie ſprechen (we{\geminationn} man ihnen
               nicht dictirt, wie es andere thun) {\pb}ja man ſoll ſie nicht
               empfangen, was aber ſchwer iſt, wenn ſie hinter einem Stubenmädl die öffnet, direct
               ins Zimmer ſtürzen, ohne Meldung abzuwarten, – oder man ſoll ſie hinauswerfen – was
               auch wieder ſchwer iſt, we{\geminationn} man nicht weiſs, wer ſie
               ſind und ſie plötzlich aus heiterm oder vielmehr bewölktem Himmel einem Glückwünſche
               zu unvermutet erſchienenen fünftauſend Kronen (nebſt Ehre, Auszeichnung u Lorbeer) zu
               Füßen legen. Übrigens werd ich Ihnen {\pb}nächſtens noch
               etwas Komiſches vom Vormittag des 15. Januar erzählen.\pend
           
\pstart
           Zur Arbeit fühl ich mich ſchon ſehr bereit; an Tagen, da man innerlich u äußerlich
               allerlei ordnen konnte, un\textcolor{gray}{d} ſelbſt an Einfällen hat es \strikeout{mir} nicht gefehlt.\pend
           
\pstart
           Wie gehts Ihnen Allen? Olga\pwindex{Schnitzler, Olga 17.01.1882 – 13.01.1970@\textsc{Schnitzler, Olga} (17.01.1882 – 13.01.1970), \emph{Schauspieler/Schauspielerin, Sänger/Sängerin}|pw} iſt über die
               prachtvolle \label{K_L01758-3v}\edtext{Schale}{\lemma{\textnormal{\emph{Schale}}}\Cendnote{\textnormal{Vgl. Hugo von Hofmannsthal an Arthur Schnitzler, 25. 12. 1907.
               }}}\label{K_L01758-3} ſehr froh. Ich hab sie \introOben{}ihr\introOben{}
               erſt im desinfizirten Raum übergeben.\pend
           
\pstart
           Wir grüßen Euch! Laßt was hören!{\\[\baselineskip]}\spacefill\mbox{Arthur}\pend
           \leftskip=0em{}\selectlanguage{ngerman}\endnumbering\briefempfaengerindex{Hofmannsthal, Hugo von@\textsc{Hofmannsthal, Hugo von}!zzzSchnitzler, Arthur@\emph{von Arthur Schnitzler}!1908-01-251@{25. 1. 1908}|)be}\mylabel{L01758h}  \normalsize

\doendnotes{C}
\bigskip
\vfill

\clearpage

\footnotesize

\lohead{\textsc{register}}

% Definiere theindex-Environment komplett neu ohne reledmac
\makeatletter
\renewenvironment{theindex}{%
  \section*{\indexname}%
  \setlength{\parindent}{0pt}%
  \setlength{\parskip}{0pt plus 0.3pt}%
  \let\item\@idxitem
}{%
  \clearpage
}
\makeatother

\IfFileExists{\jobname-pw.ind}{\input{\jobname-pw.ind}}{}

\end{document}

      