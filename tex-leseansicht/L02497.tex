%% latex-leseansicht-vorspann.tex
%% Vorspann für die Leseansicht.
%% Lädt die gemeinsame Datei latex-vorspann.tex mit nicht gesetztem Schalter.

\newif\ifkorrekturansicht
\korrekturansichtfalse

\input{../tex-inputs/latex-vorspann}


               \section[Robert Adam an Arthur Schnitzler, 30. 12. 1927]{ Robert Adam an Arthur Schnitzler, 30. 12. 1927}\nopagebreak\mylabel{v}\rehead{ }\begin{ledgroupsized}[t]{13cm}\normalsize\beginnumbering\briefempfaengerindex{Schnitzler, Arthur@\textsc{Schnitzler, Arthur}!zzzAdam, Robert@\emph{von Robert Adam}!1927-12-301@{30. 12. 1927}|(be} \toendnotes[C]{\smallbreak\pagebreak[2]} \Standort{CUL, Schnitzler, B 1.}
\physDesc{Brief, 1 Blatt, 4 Seiten
\newline{}Handschrift: schwarze Tinte, deutsche Kurrent
\newline{}Schnitzler: 1) mit Bleistift beschriftet: »\textsc{Adam}« 2) mit rotem Buntstift Vermerk: »\textsc{Aph}{[}orismen{]}« und vereinzelte Unterstreichungen\newline{}Ordnung: mit Bleistift von unbekannter Hand nummeriert:
                                                »19« }\Standort{Wien, Österreichische Nationalbibliothek, Cod.ser. 52.268, 347 verso – 349 recto.}
\physDesc{handschriftliche Abschrift, Entwurf
\newline{}Handschrift: schwarze Tinte, Gabelsberger Kurzschrift}\Standort{Wien, Österreichische Nationalbibliothek, Cod.ser. 52.268, 347 verso – 349 recto.}
\physDesc{maschinelle Abschrift, Entwurf
\newline{}Schreibmaschine}\pstart
           \raggedleft{}{\pb}Wien\oindex{Wien@\textbf{Wien}|pw}, am 30. Dezember 1927\pend
           \pstart{}Hochverehrter Herr Doktor!\pend\pstart
           Wenn ich es unternehme, Ihnen für die Überſendung Ihres Buches der Sprüche und Bedenken\pwindex{Schnitzler, Arthur 15.05.1862 – 21.10.1931@\textsc{Schnitzler, Arthur} (15.05.1862 – 21.10.1931), \emph{Schriftsteller, Mediziner}!Buch der Sprueche und Bedenken1927@\strich\emph{Buch der Sprüche und Bedenken} {[}1927{]}|pw} meinen Dank zu ſagen, ſo
                    verführt mich die alteingefleiſchte Gewohnheit meines Berufes, dem nicht leicht
                    etwas ohne Begründung entſchlüpft, dazu, meinen Dank nicht etwa bloß zu äußern,
                    sondern auch zu begründen. Will ich aber Sätze einer Begründung formen, ſo iſt
                    es mir, als müßte ich einen unüberſehbaren Tatbeſtand in wenige Worte
                    zuſammenfaſſen und leichthin erledigen. Ein Einzelbild mag man nach einmaliger
                    längerer Betrachtung {\pb}kühn
                    beurteilen; um aber zu einer Bilderſammlung, die viele Säle füllt, klare
                    Stellung zu nehmen, bedarf’s wiederholter Begehung und vergleichenden Hin- und
                    Herwandelns. Und Ihr Buch iſt eine in klarer Systematik zuſammengefaßte
                    Aneinanderreihung der weſentlichen Ergebniſſe eines langen und reichen
                    Dichterlebens, dem nichts Menſchenerhebliches fremd blieb, der Abriß Ihrer
                    Lebensphiloſophie, und zwiſchen den Abſchnitten Ihrer Aphorismen eröffnen sich
                    Ausblicke, verlockend zu verbindender Gedankenarbeit. Wenn der Aphorismus, der
                    in der Literatur das iſt, was die Bleiſtiftſkizze in der bildenden Kunſt, die
                    redlichſte Art des Schrifttums iſt, weil er entgegen allen andern Arten, vom
                    lyriſchen Gedicht bis zum philoſophiſchen Wälzer, keiner Lüge und keiner Maske
                    Raum \strikeout{läßt} gibt, {\pb}da ſich alles poſieren läßt, Gefühl
                    wie Erlebnis, Gründlichkeit wie Gewalt, nur nicht der Gedanke ſelbſt und ſeine
                    Form, und nirgends wie bei ihm jeder kleine Satz den ganzen Autor zeigt: wie
                    verehrungs- und liebenswürdig erſcheint der Autor dieser Sprüche und Aphorismen,
                    wie lebt ſeine uns aus unſerer Jugend schon vertraute Erſcheinung in jedem
                    dieſer klaren Worte! Welche Erlebtheit, welche Liebe zur Wahrheit und zur Form,
                    welche Herrſchaft des Geiſtes und über alles Geiſtige ſpricht aus jedem
                    Spruch!\pend
           \pstart
           Ich muß mich mit dieser Begründung beſcheiden und mit einer Wiederholung meines
                    Dankes, der mir Gelegenheit gibt, Ihnen, hochverehrter Herr Doktor, zur
                    Jahreswende alles Freudige zu wünſchen!\pend
           \pstart
           Mit den besten Empfehlungen und {\pb}dem
                    Ausdruck meiner tiefen Ergebenheit\pend
           \pstart
           Ihr{\\[\baselineskip]}\spacefill\mbox{D\textsuperscript{r}RAdam}\pend
           \leftskip=0em{}          \endnumbering\briefempfaengerindex{Schnitzler, Arthur@\textsc{Schnitzler, Arthur}!zzzAdam, Robert@\emph{von Robert Adam}!1927-12-301@{30. 12. 1927}|)be}\mylabel{h}\end{ledgroupsized}  \newcommand{\dateiname}{L02497}\newcommand{\titel}{Robert Adam an Arthur Schnitzler, 30. 12. 1927}\newcommand{\editorInnen}{Martin Anton Müller und Gerd-Hermann Susen}%% latex-leseansicht-abspann.tex
%% Abspann für die Leseansicht.
%% Der Schalter \ifkorrekturansicht ist bereits durch den Vorspann gesetzt.

%% latex-abspann.tex
%% Gemeinsamer Abspann für Korrekturansicht und Leseansicht.
%% Setzt den Schalter \ifkorrekturansicht voraus (gesetzt in den
%% einbindenden Dateien latex-korrekturansicht-abspann.tex bzw.
%% latex-leseansicht-abspann.tex).
%% ---------------------------------------------------------------

\normalsize

% Das esempio-Environment wird nur in der Leseansicht benötigt
\ifkorrekturansicht\else
\newenvironment{esempio}[3]%
{
    \vspace{1.5ex}
    \rlap{\underline{#1}}
    \par
    \setlength{\parindent}{0cm}
    \nopagebreak
    \leftskip=#2cm
    \rightskip=#3cm
}
{
    \par
}
\fi

\doendnotes{C}
\bigskip
\vfill

\clearpage

\footnotesize

\ifkorrekturansicht
  \lohead{\textsc{register}}
\fi

% theindex-Environment neu definieren ohne reledmac
\makeatletter
\renewenvironment{theindex}{%
  \ifkorrekturansicht
    \section*{\indexname}%
  \else
    \subsubsection*{Index der erwähnten Entitäten}%
  \fi
  \setlength{\parindent}{0pt}%
  \setlength{\parskip}{0pt plus 0.3pt}%
  \let\item\@idxitem
}{%
  \ifkorrekturansicht\clearpage\fi
}
\makeatother

\IfFileExists{\jobname-pw.ind}{\input{\jobname-pw.ind}}{}

% Quellenangabe nur in der Leseansicht
\ifkorrekturansicht\else
% Fallback-Definitionen, falls die .tex-Datei \titel etc. nicht gesetzt hat
\providecommand{\titel}{}
\providecommand{\editorInnen}{}
\providecommand{\dateiname}{\jobname}

\vspace{3cm}

\vfill

\footnotesize
\textsc{Quelle}: \titel. Herausgegeben von {\editorInnen}. In: \emph{Arthur Schnitzler: Briefwechsel mit Autorinnen und Autoren}.
 Digitale Edition, https://schnitzler-briefe.acdh.oeaw.ac.at/{\dateiname}.html (Stand \today)
\fi

\end{document}


      