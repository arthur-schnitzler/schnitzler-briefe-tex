%% latex-leseansicht-vorspann.tex
%% Vorspann für die Leseansicht.
%% Lädt die gemeinsame Datei latex-vorspann.tex mit nicht gesetztem Schalter.

\newif\ifkorrekturansicht
\korrekturansichtfalse

\input{../tex-inputs/latex-vorspann}


         
         \renewcommand{\erwaehntePersonen}{Personen: Paul Goldmann, Olga Schnitzler, Heinrich Schnitzler}
         \renewcommand{\erwaehnteOrte}{Orte: Berlin, Dessauer Straße, Wien}
         \renewcommand{\erwaehnteWerke}{Werke: Die Zeit, Die Zeit. Wiener Wochenschrift}
               \section[ Paul Goldmann an Arthur Schnitzler, 16. 9. {[}1902{]}]{ Paul Goldmann an Arthur Schnitzler, 16. 9. {[}1902{]}}\nopagebreak\mylabel{v}\rehead{ }\begin{ledgroupsized}[t]{13cm}\normalsize\beginnumbering\briefempfaengerindex{Schnitzler, Arthur@\textsc{Schnitzler, Arthur}!zzzGoldmann, Paul@\emph{von Paul Goldmann}!1902-09-161@{16. 9. {[}1902{]}}|(be} \toendnotes[C]{\smallbreak\pagebreak[2]} \Standort{DLA, A:Schnitzler, HS.NZ85.1.3172.}
\physDesc{Brief, 1 Blatt, 2 Seiten, 522 Zeichen
\newline{}Handschrift: blaue Tinte, deutsche Kurrent
\newline{}Schnitzler: mit Bleistift das Jahr »902« vermerkt }\toendnotes[C]{\smallbreak}\pstart
           \noindent{}\raggedleft{}{\pb}\textcolor{gray}{\textbf{DESSAUERSTRASSE 19}}\oindex{Dessauer Strasse@\textbf{Dessauer Straße}|pw}\pend
           \pstart
           Berlin\oindex{Berlin@\textbf{Berlin}|pw}, 16. September.\pend
           \pstart\center{}Mein lieber Freund,\pend\pstart
           Erſt heut komme ich dazu, Dir für Deine lieben Karten
               und Brief zu danken. Ich habe hier eine tolle Arbeit vorgefunden. Das \label{K_L03224-1v}\edtext{bevorſtehende Erſcheinen der »Zeit\pwindex{Zeit1902-09-27 – 1919@\emph{Die Zeit} {[}1902-09-27 – 1919{]}|pw}«}{\lemma{\textnormal{\emph{bevorſtehende … »Zeit«}}}\Cendnote{\textnormal{Zusätzlich zur Wochenzeitung\pwindex{Zeit. Wiener Wochenschrift1894 – 1904@\emph{Die Zeit. Wiener Wochenschrift} {[}1894 – 1904{]}|pwkv} erschien ab dem 27. 9. 1902
                  eine gleichnamige Tageszeitung\pwindex{Zeit1902-09-27 – 1919@\emph{Die Zeit} {[}1902-09-27 – 1919{]}|pwkv}. Siehe Paul Goldmann an Arthur Schnitzler und Olga
               Gussmann, 7. 7. [1901].}}}\label{K_L03224-1h} wird mein \textsc{Pensum}{ }\strikeout{zu} wahrſcheinlich verdoppeln.\pend
           \pstart
           Ich freue mich unendlich {\pb}zu hören, daß es Dir und
                  \textsc{Olga\pwindex{Schnitzler, Olga 17.01.1882 – 13.01.1970@\textsc{Schnitzler, Olga} (17.01.1882 – 13.01.1970), \emph{Schauspielerin, Sängerin}|pw}} ſowie Eurem Sohn\pwindex{Schnitzler, Heinrich 09.08.1902 – 12.07.1982@\textsc{Schnitzler, Heinrich} (09.08.1902 – 12.07.1982), \emph{Regisseur, Schauspieler}|pwv} gut
               geht und freue mich ganz beſonders über die Ausſicht, Dir \label{K_L03224-2v}\edtext{Anfang Oktober}{\lemma{\textnormal{\emph{Anfang Oktober}}}\Cendnote{\textnormal{Schnitzler\pwindex{Schnitzler, Arthur 15.05.1862 – 21.10.1931@\textsc{Schnitzler, Arthur} (15.05.1862 – 21.10.1931), \emph{Schriftsteller, Mediziner}|pwk} war vom 13. 10. 1902 bis zum 18. 10. 1902 in Berlin\oindex{Berlin@\textbf{Berlin}|pwk}. Die beiden trafen sich in dieser Zeit
                  täglich. }}}\label{K_L03224-2h}{ }hier\oindex{Berlin@\textbf{Berlin}|pwv} die Hand drücken zu
               können.\pend
           \pstart
           Schreib’ mir bald, – und nicht ſo kurz und ſo \strikeout{eilig}
               eilig, wie ich es thun muß.\pend
           \pstart
           Viele treue Grüße {\\[\baselineskip]}Dein {\\[\baselineskip]}\spacefill\mbox{Paul Goldmn}\pend
           \leftskip=0em{}
         
         \endnumbering\mylabel{h}\end{ledgroupsized}  \newcommand{\dateiname}{L03224}\newcommand{\titel}{Paul Goldmann an Arthur Schnitzler, 16. 9. [1902]}\newcommand{\editorInnen}{Martin Anton Müller und Laura Untner}%% latex-leseansicht-abspann.tex
%% Abspann für die Leseansicht.
%% Der Schalter \ifkorrekturansicht ist bereits durch den Vorspann gesetzt.

%% latex-abspann.tex
%% Gemeinsamer Abspann für Korrekturansicht und Leseansicht.
%% Setzt den Schalter \ifkorrekturansicht voraus (gesetzt in den
%% einbindenden Dateien latex-korrekturansicht-abspann.tex bzw.
%% latex-leseansicht-abspann.tex).
%% ---------------------------------------------------------------

\normalsize

% Das esempio-Environment wird nur in der Leseansicht benötigt
\ifkorrekturansicht\else
\newenvironment{esempio}[3]%
{
    \vspace{1.5ex}
    \rlap{\underline{#1}}
    \par
    \setlength{\parindent}{0cm}
    \nopagebreak
    \leftskip=#2cm
    \rightskip=#3cm
}
{
    \par
}
\fi

\doendnotes{C}
\bigskip
\vfill

\clearpage

\footnotesize

\ifkorrekturansicht
  \lohead{\textsc{register}}
\fi

% theindex-Environment neu definieren ohne reledmac
\makeatletter
\renewenvironment{theindex}{%
  \ifkorrekturansicht
    \section*{\indexname}%
  \else
    \subsubsection*{Index der erwähnten Entitäten}%
  \fi
  \setlength{\parindent}{0pt}%
  \setlength{\parskip}{0pt plus 0.3pt}%
  \let\item\@idxitem
}{%
  \ifkorrekturansicht\clearpage\fi
}
\makeatother

\IfFileExists{\jobname-pw.ind}{\input{\jobname-pw.ind}}{}

% Quellenangabe nur in der Leseansicht
\ifkorrekturansicht\else
% Fallback-Definitionen, falls die .tex-Datei \titel etc. nicht gesetzt hat
\providecommand{\titel}{}
\providecommand{\editorInnen}{}
\providecommand{\dateiname}{\jobname}

\vspace{3cm}

\vfill

\footnotesize
\textsc{Quelle}: \titel. Herausgegeben von {\editorInnen}. In: \emph{Arthur Schnitzler: Briefwechsel mit Autorinnen und Autoren}.
 Digitale Edition, https://schnitzler-briefe.acdh.oeaw.ac.at/{\dateiname}.html (Stand \today)
\fi

\end{document}


      