%% latex-korrekturansicht-vorspann.tex
%% Vorspann für die Korrekturansicht.
%% Lädt die gemeinsame Datei latex-vorspann.tex mit gesetztem Schalter.

\newif\ifkorrekturansicht
\korrekturansichttrue

\input{../tex-inputs/latex-vorspann}


\section[ Paul Goldmann an Arthur Schnitzler, 16. 9. {[}1902{]}]{L03224 Paul Goldmann an Arthur Schnitzler, 16. 9. {[}1902{]}}
\nopagebreak\mylabel{L03224v}
\rehead{ }\normalsize\beginnumbering\briefempfaengerindex{Schnitzler, Arthur@\textsc{Schnitzler, Arthur}!zzzGoldmann, Paul@\emph{von Paul Goldmann}!1902-09-161@{16. 9. {[}1902{]}}|(be}
\toendnotes[C]{\smallbreak\pagebreak[2]}\Standort{DLA, A:Schnitzler, HS.NZ85.1.3172.}
\physDesc{Brief, 1 Blatt, 2 Seiten, 522 Zeichen
\newline{}Handschrift: blaue Tinte, deutsche Kurrent
\newline{}Schnitzler: mit Bleistift das Jahr »902« vermerkt }\toendnotes[C]{\smallbreak}
\pstart
           \raggedleft{}{\pb}\textcolor{gray}{\textbf{DESSAUERSTRASSE 19}}\oindex{Dessauer Strasse@\textbf{Dessauer Straße}, \emph{Straße (K.STR)}|pw}\pend
           
\pstart
           Berlin\oindex{Berlin@\textbf{Berlin}, \emph{P.PPLC}|pw}, 16. September.\pend
           
\pstart\center{}Mein lieber Freund,\pend\vspace{0.5em}
\pstart
           Erſt heut komme ich dazu, Dir für Deine lieben Karten
               und Brief zu danken. Ich habe hier eine tolle Arbeit vorgefunden. Das \label{K_L03224-1v}\edtext{bevorſtehende Erſcheinen der »Zeit\pwindex{Zeit@\emph{Die Zeit}|pw}«}{\lemma{\textnormal{\emph{bevorſtehende … »Zeit«}}}\Cendnote{\textnormal{Zusätzlich zur Wochenzeitung\pwindex{Zeit. Wiener Wochenschrift@\emph{Die Zeit. Wiener Wochenschrift}|pwkv} erschien ab dem 27. 9. 1902
                  eine gleichnamige Tageszeitung\pwindex{Zeit@\emph{Die Zeit}|pwkv}. Siehe Paul Goldmann an Arthur Schnitzler und Olga
               Gussmann, 7. 7. [1901].}}}\label{K_L03224-1} wird mein \textsc{Pensum}{ }\strikeout{zu} wahrſcheinlich verdoppeln.\pend
           
\pstart
           Ich freue mich unendlich {\pb}zu hören, daß es Dir und
                  \textsc{Olga\pwindex{Schnitzler, Olga 17.01.1882 – 13.01.1970@\textsc{Schnitzler, Olga} (17.01.1882 – 13.01.1970), \emph{Schauspieler/Schauspielerin, Sänger/Sängerin}|pw}} ſowie Eurem Sohn\pwindex{Schnitzler, Heinrich 09.08.1902 – 12.07.1982@\textsc{Schnitzler, Heinrich} (09.08.1902 – 12.07.1982), \emph{Regisseur/Regisseurin, Schauspieler/Schauspielerin}|pwv} gut
               geht und freue mich ganz beſonders über die Ausſicht, Dir \label{K_L03224-2v}\edtext{Anfang Oktober}{\lemma{\textnormal{\emph{Anfang Oktober}}}\Cendnote{\textnormal{Schnitzler war vom 13. 10. 1902 bis zum 18. 10. 1902 in Berlin\oindex{Berlin@\textbf{Berlin}, \emph{P.PPLC}|pwk}. Die beiden trafen sich in dieser Zeit
                  täglich. }}}\label{K_L03224-2}{ }hier\oindex{Berlin@\textbf{Berlin}, \emph{P.PPLC}|pwv} die Hand drücken zu
               können.\pend
           
\pstart
           Schreib’ mir bald, – und nicht ſo kurz und ſo \strikeout{eilig}
               eilig, wie ich es thun muß.\pend
           
\pstart
           Viele treue Grüße {\\[\baselineskip]}Dein {\\[\baselineskip]}\spacefill\mbox{Paul Goldmn}\pend
           \leftskip=0em{}\selectlanguage{ngerman}\endnumbering\briefempfaengerindex{Schnitzler, Arthur@\textsc{Schnitzler, Arthur}!zzzGoldmann, Paul@\emph{von Paul Goldmann}!1902-09-161@{16. 9. {[}1902{]}}|)be}\mylabel{L03224h}  \normalsize

\doendnotes{C}
\bigskip
\vfill

\clearpage

\footnotesize

\lohead{\textsc{register}}

% Definiere theindex-Environment komplett neu ohne reledmac
\makeatletter
\renewenvironment{theindex}{%
  \section*{\indexname}%
  \setlength{\parindent}{0pt}%
  \setlength{\parskip}{0pt plus 0.3pt}%
  \let\item\@idxitem
}{%
  \clearpage
}
\makeatother

\IfFileExists{\jobname-pw.ind}{\input{\jobname-pw.ind}}{}

\end{document}

      