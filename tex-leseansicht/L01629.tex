%% latex-korrekturansicht-vorspann.tex
%% Vorspann für die Korrekturansicht.
%% Lädt die gemeinsame Datei latex-vorspann.tex mit gesetztem Schalter.

\newif\ifkorrekturansicht
\korrekturansichttrue

\input{../tex-inputs/latex-vorspann}


\section[Arthur Schnitzler an Hermann Bahr, 17. 9. 1906]{L01629 Arthur Schnitzler an Hermann Bahr, 17. 9. 1906}
\nopagebreak\mylabel{L01629v}
\rehead{ }\normalsize\beginnumbering\briefempfaengerindex{Bahr, Hermann@\textsc{Bahr, Hermann}!zzzSchnitzler, Arthur@\emph{von Arthur Schnitzler}!1906-09-171@{17. 9. 1906}|(be}
\toendnotes[C]{\smallbreak\pagebreak[2]}\Standort{TMW, HS AM 60119 Ba.}
\physDesc{Bildpostkarte, 298 Zeichen
\newline{}Handschrift: Bleistift, deutsche Kurrent
\newline{}Versand: Stempel: »\nobreak{}\oindex{Semmering@\textbf{Semmering}, \emph{A.ADM3}|pwk}Semmering, 17. IX{[}. 06{]}\nobreak{}«.  
\newline{}Ordnung: Lochung }
\buchAbdrucke{\weitereDrucke{Hermann Bahr, Arthur Schnitzler: \emph{Briefwechsel, Aufzeichnungen, Dokumente (1891–1931)}. Göttingen: \emph{Wallstein} 2018, S. 381.} }\toendnotes[C]{\smallbreak}\pstart{}{\pb}Hr \textsc{Hermann Bahr}\pend{}\pstart{}Wien XIII\oindex{XIII., Hietzing@\textbf{XIII., Hietzing}, \emph{A.ADM3}|pw}\pend{}\pstart{}\textsc{Ober St Veit}\oindex{Ober Sankt Veit@\textbf{Ober Sankt Veit}, \emph{P.PPLX}|pw}\pend{}\pstart{}\textsc{Veitlissengasse\oindex{Veitlissengasse@\textbf{Veitlissengasse}, \emph{Straße (K.STR)}|pw}}\pend{}{\bigskip}
\pstart
           \noindent{}\centering{}{\pb}\textcolor{gray}{\textbf{Südbahnhotel Semmering\oindex{Suedbahnhotel [Semmering]@\textbf{Südbahnhotel [Semmering]}, \emph{Hotel (K.HTL)}|pw}}}\pend
           \vspace{1em}
\pstart
           \noindent{}{\pb}lieber Hermann, das \label{K_L01629-1v}\edtext{\textsc{Mscrpt\pwindex{Ringelspiel. In drei Akten@\emph{Ringelspiel. In drei Akten}|pwv}}}{\lemma{\textnormal{\emph{Mscrpt}}}\Cendnote{\textnormal{zu \emph{Das
                     Ringelspiel}\pwindex{Ringelspiel. In drei Akten@\emph{Ringelspiel. In drei Akten}|pwk}}}}\label{K_L01629-1} haſt du hoffentlich rechtzeitig erhalten. Herzlichen Dank – beſonders für die
               erſten 2 Akte. Gegen den 3. hab ich viel auf dem Herzen.\pend
           
\pstart
           Sind Ende der Woche daheim – wir ſehn uns doch beſti{\geminationm}t
               vor deiner \label{K_L01629-2v}\edtext{Abreiſe}{\lemma{\textnormal{\emph{Abreiſe}}}\Cendnote{\textnormal{Ursprünglich meinte Bahr\pwindex{Bahr, Hermann 19.07.1863 – 15.01.1934@\textsc{Bahr, Hermann} (19.07.1863 – 15.01.1934), \emph{Schriftsteller/Schriftstellerin, Kritiker/Kritikerin}|pwk} sein Engagement als Regisseur bei Max Reinhardt\pwindex{Reinhardt, Max 09.09.1873 – 30.10.1943@\textsc{Reinhardt, Max} (09.09.1873 – 30.10.1943), \emph{Theaterleiter/Theaterleiterin, Regisseur/Regisseurin, Schauspieler/Schauspielerin}|pwk} mit Anfang Oktober anzutreten, es
                  sollte aber erst einen Monat später beginnen.}}}\label{K_L01629-2}?\pend
           
\pstart
           Dein{\\[\baselineskip]}\spacefill\mbox{A. S.}\pend
           \leftskip=0em{}
\pstart
           \noindent{}\label{T_L01629-1v}\edtext{17/9 906}{\lemma{\textnormal{\emph{17/9 906}}}\Cendnote{\textnormal{seitlich am Textrand}}}\label{T_L01629-1}\pend
           \selectlanguage{ngerman}\endnumbering\briefempfaengerindex{Bahr, Hermann@\textsc{Bahr, Hermann}!zzzSchnitzler, Arthur@\emph{von Arthur Schnitzler}!1906-09-171@{17. 9. 1906}|)be}\mylabel{L01629h}  \normalsize

\doendnotes{C}
\bigskip
\vfill

\clearpage

\footnotesize

\lohead{\textsc{register}}

% Definiere theindex-Environment komplett neu ohne reledmac
\makeatletter
\renewenvironment{theindex}{%
  \section*{\indexname}%
  \setlength{\parindent}{0pt}%
  \setlength{\parskip}{0pt plus 0.3pt}%
  \let\item\@idxitem
}{%
  \clearpage
}
\makeatother

\IfFileExists{\jobname-pw.ind}{\input{\jobname-pw.ind}}{}

\end{document}

      