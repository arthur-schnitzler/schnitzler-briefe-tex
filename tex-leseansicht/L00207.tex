%% latex-korrekturansicht-vorspann.tex
%% Vorspann für die Korrekturansicht.
%% Lädt die gemeinsame Datei latex-vorspann.tex mit gesetztem Schalter.

\newif\ifkorrekturansicht
\korrekturansichttrue

\input{../tex-inputs/latex-vorspann}


\section[Peter Altenberg an Arthur Schnitzler, {[}2.? 5. 1893{]}]{L00207 Peter Altenberg an Arthur Schnitzler, {[}2.? 5. 1893{]}}
\nopagebreak\mylabel{L00207v}
\rehead{ }\normalsize\beginnumbering\briefempfaengerindex{Schnitzler, Arthur@\textsc{Schnitzler, Arthur}!zzzAltenberg, Peter@\emph{von Peter Altenberg}!1893-05-022@{{[}2.? 5. 1893{]}}|(be}
\toendnotes[C]{\smallbreak\pagebreak[2]}\Standort{CUL, Schnitzler, B 2.}
\physDesc{Brief, 1 Blatt, 1 Seite, 107 Zeichen
\newline{}Handschrift: schwarze Tinte, deutsche Kurrent
\newline{}Schnitzler: mit Bleistift doppelt beschriftet: am oberen Rand »\textsc{Altenberg}«, datiert: »\textsc{Mai} 93« und nummeriert:
                                    »1« sowie unter der Unterschrift: »\textsc{(Altenberg)}« 
\newline{}Ordnung: mit Bleistift von unbekannter Hand nummeriert:
                                 »1« }\toendnotes[C]{\smallbreak}
\pstart{}{\pb}Lieber \textsc{D\textsuperscript{r.}} Arthur Schnitzler.\pend\vspace{0.5em}
\pstart
           Nehmen Sie den Ausdruck meiner innigſten \label{K_L00207-1v}\edtext{Theilnahme}{\lemma{\textnormal{\emph{Theilnahme}}}\Cendnote{\textnormal{zum Tod
                  des Vaters Johann Schnitzler\pwindex{Schnitzler, Johann 10.04.1835 – 02.05.1893@\textsc{Schnitzler, Johann} (10.04.1835 – 02.05.1893), \emph{Laryngologe/Laryngologin}|pwk} am
                     2. 5. 1893}}}\label{K_L00207-1} entgegen von\pend
           \pstart \spacefill\mbox{Richard Engländer.}\pend{}\selectlanguage{ngerman}\endnumbering\briefempfaengerindex{Schnitzler, Arthur@\textsc{Schnitzler, Arthur}!zzzAltenberg, Peter@\emph{von Peter Altenberg}!1893-05-022@{{[}2.? 5. 1893{]}}|)be}\mylabel{L00207h}  \normalsize

\doendnotes{C}
\bigskip
\vfill

\clearpage

\footnotesize

\lohead{\textsc{register}}

% Definiere theindex-Environment komplett neu ohne reledmac
\makeatletter
\renewenvironment{theindex}{%
  \section*{\indexname}%
  \setlength{\parindent}{0pt}%
  \setlength{\parskip}{0pt plus 0.3pt}%
  \let\item\@idxitem
}{%
  \clearpage
}
\makeatother

\IfFileExists{\jobname-pw.ind}{\input{\jobname-pw.ind}}{}

\end{document}

      