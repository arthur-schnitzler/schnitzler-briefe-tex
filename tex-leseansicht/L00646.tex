\input{../tex-inputs/latex-pdf-vorspann}
\begin{center}
            \textcolor{red}{ENTWURF. ENTZIFFERUNG NOCH NICHT KORREKTURGELESEN}
                      \end{center}
            
               \section[Arthur Schnitzler an Hermann Bahr, 26. 2. 1897]{ Arthur Schnitzler an Hermann Bahr, 26. 2. 1897}\nopagebreak\mylabel{v}\rehead{ }\begin{ledgroupsized}[t]{13cm}\normalsize\beginnumbering\briefempfaengerindex{Bahr, Hermann@\textsc{Bahr, Hermann}!zzzSchnitzler, Arthur@\emph{von Arthur Schnitzler}!1897-02-261@{26. 2. 1897}|(be} \toendnotes[C]{\smallbreak\pagebreak[2]} \Standort{TMW, HS AM 39903 Ba.}
\physDesc{Briefkarte
\newline{}Handschrift: Bleistift, deutsche Kurrent}\buchAbdrucke{\weitereDrucke{1) \emph{24. 2. 1897.} In: Arthur Schnitzler: \emph{The Letters of Arthur Schnitzler to Hermann Bahr}. Edited, annotated, and with an introduction, by Donald G.
                        Daviau. Chapel Hill: \emph{The University of North Carolina Press} 1978, S. 60 (University of North Carolina studies in the Germanic languages
                        and literatures, 89).} \weitereDrucke{2) Hermann Bahr, Arthur Schnitzler: \emph{Briefwechsel, Aufzeichnungen, Dokumente (1891–1931)}. Hg. Kurt Ifkovits und Martin Anton Müller. Göttingen: \emph{Wallstein} 2018, S. 136.} }\pstart
           \noindent{}{\pb}Lieber Hermann, auf regulärem Weg beko{\geminationm} ich an der Kaſſe nichts
               ordentliches mehr fürs Tſchapperl\pwindex{Bahr, Hermann 19.07.1863 – 15.01.1934@\textsc{Bahr, Hermann} (19.07.1863 – 15.01.1934), \emph{Schriftsteller, Kritiker}!Tschaperl1896@\strich\emph{Das Tschaperl} {[}1896{]}|pw}. Ka{\geminationn} ich
               durch deine Protektion einen guten Sitz (am {\pb}liebſten
               Orcheſter 1. Reihe) angewieſen erhalten? Thäteſt mir einen großen Gefallen.\pend
           \pstart Herzlich dein \spacefill\mbox{ArthSchn}\pend{}\pstart
           Wien\oindex{Wien@\textbf{Wien}|pw}{ }26. 2. 97.\pend
           \endnumbering\briefempfaengerindex{Bahr, Hermann@\textsc{Bahr, Hermann}!zzzSchnitzler, Arthur@\emph{von Arthur Schnitzler}!1897-02-261@{26. 2. 1897}|)be}\mylabel{h}\end{ledgroupsized}  \newcommand{\dateiname}{L00646}\newcommand{\titel}{Arthur Schnitzler an Hermann Bahr, 26. 2. 1897}\newcommand{\editorInnen}{ Kurt Ifkovits,  Martin Anton Müller}\input{../tex-inputs/latex-pdf-abspann}
      