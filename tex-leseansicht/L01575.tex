%% latex-leseansicht-vorspann.tex
%% Vorspann für die Leseansicht.
%% Lädt die gemeinsame Datei latex-vorspann.tex mit nicht gesetztem Schalter.

\newif\ifkorrekturansicht
\korrekturansichtfalse

\input{../tex-inputs/latex-vorspann}


\section[Adolf Treibl an Arthur Schnitzler, {[}22.? 1. 1906{]}]{L01575 Adolf Treibl an Arthur Schnitzler, {[}22.? 1. 1906{]}}
\nopagebreak\mylabel{L01575v}
\rehead{ }\normalsize\beginnumbering\briefempfaengerindex{Schnitzler, Arthur@\textsc{Schnitzler, Arthur}!zzzTreibl, Adolf@\emph{von Adolf Treibl}!1906-01-221@{{[}22.? 1. 1906{]}}|(be}
\toendnotes[C]{\smallbreak\pagebreak[2]}
\correspDesc{Versand  durch Adolf Treibl am [22.? 1. 1906] in Wien
\newline{}Erhalt  durch Arthur Schnitzler im Zeitraum [22. 1. 1906
                  – 26. 1. 1906?] in Wien}\toendnotes[C]{\smallbreak}
\Standort{DLA, A:Schnitzler, HS.NZ85.1.4815,1.}
\physDesc{Brief, 2 Blätter, 5 Seiten, 1590 Zeichen
\newline{}Handschrift: schwarze Tinte, deutsche Kurrent
\newline{}Schnitzler: mit Bleistift beschriftet: »\textsc{Ehrenstein (Treibl}« }\toendnotes[C]{\smallbreak}
\pstart
           {\pb}\textsc{Euer Hochwohlgeboren}\pend
           
\pstart{}Hochverehrter Herr \textsc{Doctor}\pend\vspace{0.5em}
\pstart
           Die Woche fängt für mich gut an. Schon am \textsc{Montag}{ }morgen muß ich ein Vergehen beichten. Dieſer Brief hätte Euer
               Hochwohlgeboren{ }ſchon \textsc{Samstag} zugehen{ }ſollen. Aber{ }ſo{ }ſind wir Menſchen. Im Unglück zerknirſcht und
               demütig, wird doch {\pb}kaum daß es beſſer geht, der alte
               Schlendrian eingeſchlagen und die kleine, kleinliche Tagesarbeit erſcheint wichtiger,
               als Treue und Dankbarkeit zu bezeugen. Das iſt nur eine Selbſtanklage. Die Familie
                  Ehrenstein\pwindex{Ehrenstein, Alexander 29.\,3.\,1857 Skalice – 29.\,5.\,1925 Wien@\textsc{Ehrenstein, Alexander} (29.\,3.\,1857 Skalice – 29.\,5.\,1925 Wien), \emph{Kassier}|pw}\pwindex{Ehrenstein, Charlotte 21.\,4.\,1867 Vrádište – 2.\,2.\,1941 New York City@\textsc{Ehrenstein, Charlotte} (21.\,4.\,1867 Vrádište – 2.\,2.\,1941 New York City)|pw} trifft kein
               Verſchulden.\pend
           
\pstart
           \textsc{Albert}\pwindex{Ehrenstein, Albert 23.\,12.\,1886 Wien – 8.\,4.\,1950 New York City@\textsc{Ehrenstein, Albert} (23.\,12.\,1886 Wien – 8.\,4.\,1950 New York City), \emph{Schriftsteller}|pw} befindet{ }ſich am Wege der Beſſerung und iſt mit Zuſtimmung des \textsc{Prima{\pb}rius D\textsuperscript{r}}{ }\textsc{Kornfeld}\pwindex{Kornfeld, Sigmund 21.\,4.\,1859 Golčův Jeníkov – 15.\,4.\,1927 Wien@\textsc{Kornfeld, Sigmund} (21.\,4.\,1859 Golčův Jeníkov – 15.\,4.\,1927 Wien), \emph{Psychiater}|pw}, der vorgeſtern dort war und heute wieder kommt in häuslicher Pflege belaſſen
               worden. Der krankhafte Erregungszuſtand iſt im Abflauen. Seine Handlungsweiſe vom
                  \label{K_L01575-1v}\edtext{vorigen Sonntag}{\lemma{\textnormal{\emph{vorigen Sonntag}}}\Cendnote{\textnormal{Vgl. A. S.: \emph{Tagebuch}, 14. 1. 1906.
               }}}\label{K_L01575-1} erkennt \textsc{Albert}\pwindex{Ehrenstein, Albert 23.\,12.\,1886 Wien – 8.\,4.\,1950 New York City@\textsc{Ehrenstein, Albert} (23.\,12.\,1886 Wien – 8.\,4.\,1950 New York City), \emph{Schriftsteller}|pw}{ }ſchon als abnormal. Sein Gang iſt{ }ſchon
               natürlicher, drückt bei weitem nicht mehr die gehobene Stimmung eines Siegers aus.
               Unnützes {\pb}Lachen kommt nicht vor, doch hat er noch
               namentlich abends Angſtgefühle und findet auch noch – wenn auch{ }ſeltener –
               Beziehungen litterariſcher Größen zu{ }ſich und seinem Verhalten.\pend
           
\pstart
           \textsc{D\textsuperscript{r}}{ }\textsc{Kornfeld}\pwindex{Kornfeld, Sigmund 21.\,4.\,1859 Golčův Jeníkov – 15.\,4.\,1927 Wien@\textsc{Kornfeld, Sigmund} (21.\,4.\,1859 Golčův Jeníkov – 15.\,4.\,1927 Wien), \emph{Psychiater}|pw} ordnete unter anderem auch gelinde geiſtige Beſchäftigung an und \textsc{Albert}\pwindex{Ehrenstein, Albert 23.\,12.\,1886 Wien – 8.\,4.\,1950 New York City@\textsc{Ehrenstein, Albert} (23.\,12.\,1886 Wien – 8.\,4.\,1950 New York City), \emph{Schriftsteller}|pw} hat geſtern im \textsc{Herder}\pwindex{Herder, Johann Gottfried von 25.\,8.\,1744 Morąg – 18.\,12.\,1803 Weimar@\textsc{Herder, Johann Gottfried von} (25.\,8.\,1744 Morąg – 18.\,12.\,1803 Weimar), \emph{Schriftsteller, Philosoph, Theologe}|pw} geleſen u darüber eine \textsc{Kritik} zu liefern gehabt. Daß
               Gott erbarme wie Herder\pwindex{Herder, Johann Gottfried von 25.\,8.\,1744 Morąg – 18.\,12.\,1803 Weimar@\textsc{Herder, Johann Gottfried von} (25.\,8.\,1744 Morąg – 18.\,12.\,1803 Weimar), \emph{Schriftsteller, Philosoph, Theologe}|pw} wegkam. Er{ }ſelbſt be{\pb}zeichnete die Arbeit ironiſierend als »Schularbeit«
               und klaſſifizierte{ }ſie mit »nicht genügend«.\pend
           
\pstart
           Mit vielem und herzlichen Dank für Ihre Teilnahme an das Geſchick des Kranken\pwindex{Ehrenstein, Albert 23.\,12.\,1886 Wien – 8.\,4.\,1950 New York City@\textsc{Ehrenstein, Albert} (23.\,12.\,1886 Wien – 8.\,4.\,1950 New York City), \emph{Schriftsteller}|pwv} bitte ich um
               Entſchuldigung, wenn ich{ }ſo frei{ }ſein werde dieſer Tage weiter zu berichten\pend
           
\pstart
           In vollkommener Hochachtung{\\[\baselineskip]}ergebſt{\\[\baselineskip]}\spacefill\mbox{Ad. Treibl}\pend
           \leftskip=0em{}\selectlanguage{ngerman}\endnumbering\briefempfaengerindex{Schnitzler, Arthur@\textsc{Schnitzler, Arthur}!zzzTreibl, Adolf@\emph{von Adolf Treibl}!1906-01-221@{{[}22.? 1. 1906{]}}|)be}\mylabel{L01575h}  \newcommand{\dateiname}{L01575}\newcommand{\titel}{Adolf Treibl an Arthur Schnitzler, [22.? 1. 1906]}\newcommand{\editorInnen}{Martin Anton Müller und Gerd-Hermann Susen}%% latex-leseansicht-abspann.tex
%% Abspann für die Leseansicht.
%% Der Schalter \ifkorrekturansicht ist bereits durch den Vorspann gesetzt.

%% latex-abspann.tex
%% Gemeinsamer Abspann für Korrekturansicht und Leseansicht.
%% Setzt den Schalter \ifkorrekturansicht voraus (gesetzt in den
%% einbindenden Dateien latex-korrekturansicht-abspann.tex bzw.
%% latex-leseansicht-abspann.tex).
%% ---------------------------------------------------------------

\normalsize

% Das esempio-Environment wird nur in der Leseansicht benötigt
\ifkorrekturansicht\else
\newenvironment{esempio}[3]%
{
    \vspace{1.5ex}
    \rlap{\underline{#1}}
    \par
    \setlength{\parindent}{0cm}
    \nopagebreak
    \leftskip=#2cm
    \rightskip=#3cm
}
{
    \par
}
\fi

\doendnotes{C}
\bigskip
\vfill

\clearpage

\footnotesize

\ifkorrekturansicht
  \lohead{\textsc{register}}
\fi

% theindex-Environment neu definieren ohne reledmac
\makeatletter
\renewenvironment{theindex}{%
  \ifkorrekturansicht
    \section*{\indexname}%
  \else
    \subsubsection*{Index der erwähnten Entitäten}%
  \fi
  \setlength{\parindent}{0pt}%
  \setlength{\parskip}{0pt plus 0.3pt}%
  \let\item\@idxitem
}{%
  \ifkorrekturansicht\clearpage\fi
}
\makeatother

\IfFileExists{\jobname-pw.ind}{\input{\jobname-pw.ind}}{}

% Quellenangabe nur in der Leseansicht
\ifkorrekturansicht\else
% Fallback-Definitionen, falls die .tex-Datei \titel etc. nicht gesetzt hat
\providecommand{\titel}{}
\providecommand{\editorInnen}{}
\providecommand{\dateiname}{\jobname}

\vspace{3cm}

\vfill

\footnotesize
\textsc{Quelle}: \titel. Herausgegeben von {\editorInnen}. In: \emph{Arthur Schnitzler: Briefwechsel mit Autorinnen und Autoren}.
 Digitale Edition, https://schnitzler-briefe.acdh.oeaw.ac.at/{\dateiname}.html (Stand \today)
\fi

\end{document}


