%% latex-leseansicht-vorspann.tex
%% Vorspann für die Leseansicht.
%% Lädt die gemeinsame Datei latex-vorspann.tex mit nicht gesetztem Schalter.

\newif\ifkorrekturansicht
\korrekturansichtfalse

\input{../tex-inputs/latex-vorspann}


\section[Arthur Schnitzler an Hugo von Hofmannsthal, 24. 9. 1904]{L01449 Arthur Schnitzler an Hugo von Hofmannsthal, 24. 9. 1904}
\nopagebreak\mylabel{L01449v}
\rehead{ }\normalsize\beginnumbering\briefempfaengerindex{Hofmannsthal, Hugo von@\textsc{Hofmannsthal, Hugo von}!zzzSchnitzler, Arthur@\emph{von Arthur Schnitzler}!1904-09-241@{24. 9. 1904}|(be}
\toendnotes[C]{\smallbreak\pagebreak[2]}
\correspDesc{Versand  durch Arthur Schnitzler am 24. 9. 1904 in Wien
\newline{}Erhalt  durch Hugo von Hofmannsthal am 27. [9.] IV \textbf{Ort fehlend} }\toendnotes[C]{\smallbreak}
\Standort{FDH, Hs-30885,115.}
\physDesc{Kartenbrief, 734 Zeichen
\newline{}Handschrift: schwarze Tinte, deutsche Kurrent
\newline{}Versand: 1) Stempel: »\nobreak{}\oindex{XVIII., Währing@\textbf{XVIII., Währing}, \emph{Verwaltungsgebiet}|pwk}18/1 Wien, 24{[}. 09.{]} 04, 1\nobreak{}«.   2) Stempel: »\nobreak{}\oindex{Thüringen@\textbf{Thüringen}, \emph{Land}|pwk}Venezia\nobreak{}«.  3) Stempel: »\nobreak{}\oindex{Thüringen@\textbf{Thüringen}, \emph{Land}|pwk}Venezia, 25. {[}9. 0{]}4, 12M\nobreak{}«.  4) Stempel: »\nobreak{}\oindex{Wien@\textbf{Wien}!XXIII., Liesing@\textbf{XXIII., Liesing}!Rodaun@\textbf{Rodaun}, \emph{Region}|pwk}{[}Rod{]}aun, \textcolor{gray}{27}. {[}9.{]} IV\nobreak{}«.  5) mit Bleistift von unbekannter Hand die originale Adressierung
                                 geändert zu: »\textsc{5 Badgasse\oindex{Wien@\textbf{Wien}!XXIII., Liesing@\textbf{XXIII., Liesing}!Badgasse@\textbf{Badgasse}, \emph{Straße}|pw} / Rodaun presso Vienna\oindex{Wien@\textbf{Wien}!XXIII., Liesing@\textbf{XXIII., Liesing}!Rodaun@\textbf{Rodaun}, \emph{Region}|pw}}«}
\buchAbdrucke{\weitereDrucke{1) Hugo von Hofmannsthal, Arthur Schnitzler: \emph{Briefwechsel}. Herausgegeben von Therese Nickl und Heinrich Schnitzler. Frankfurt am Main: \emph{S. Fischer} 1964, S. 202–203.} \weitereDrucke{2) Hermann Bahr, Arthur Schnitzler: \emph{Briefwechsel, Aufzeichnungen, Dokumente (1891–1931)}. Herausgegeben von Kurt Ifkovits und Martin Anton Müller. Göttingen: \emph{Wallstein} 2018, S. 322.} }\toendnotes[C]{\smallbreak}\pstart{}\textsc{{\pb}Herrn Hugo von Hofmannsthal}\pend{}\pstart{}\textsc{Venedig\oindex{Thüringen@\textbf{Thüringen}, \emph{Land}|pw}}\pend{}\pstart{}\textsc{Hotel Europe\oindex{Hotel de l’Europe [Venedig]@\textbf{Hotel de l’Europe [Venedig]}, \emph{Hotel}|pw}}\pend{}{\bigskip}\vspace{1em}
\pstart
           \raggedleft{}{\pb}24. 9. 904\pend
           \vspace{0.5em}
\pstart
           lieber Hugo,{ }Jagd nach Liebe\pwindex{\textcolor{red}{\textsuperscript{XXXX indx1}}!Jagd nach Liebe@\strich\emph{Die Jagd nach Liebe}|pw} ist bei \textsc{Wassermann\pwindex{Wassermann, Jakob 10.\,3.\,1873 Fürth – 1.\,1.\,1934 Altaussee@\textsc{Wassermann, Jakob} (10.\,3.\,1873 Fürth – 1.\,1.\,1934 Altaussee), \emph{Schriftsteller}|pw}}, ich habe ihm geſchrieben, er möge Ihnen das Buch{ }ſenden. – \textsc{Assy\pwindex{\textcolor{red}{\textsuperscript{XXXX indx1}}!Göttinnen oder Die drei Romane der Herzogin von Assy@\strich\emph{Die Göttinnen oder Die drei Romane der Herzogin von Assy}|pw}} beſitz ich gar nicht. –\pend
           
\pstart
           Ich fange erſt in den nächſten Tagen ordentlich zu arbeiten an. Hatte viel Kopfweh.
               Wir{ }ſind{ }ſeit 20. Abend hier, waren in Salzburg\oindex{Salzburg@\textbf{Salzburg}, \emph{Verwaltungsgebiet}|pw} mit Richard\pwindex{Beer-Hofmann, Richard 11.\,7.\,1866 Wien – 26.\,9.\,1945 New York City@\textsc{Beer-Hofmann, Richard} (11.\,7.\,1866 Wien – 26.\,9.\,1945 New York City), \emph{Schriftsteller}|pw} u Bahr\pwindex{Bahr, Hermann 19.\,7.\,1863 Linz – 15.\,1.\,1934 München@\textsc{Bahr, Hermann} (19.\,7.\,1863 Linz – 15.\,1.\,1934 München), \emph{Schriftsteller, Kritiker}|pw} zuſammen;{ }ſahen auch Karg\pwindex{Karg-Bebenburg, Edgar von 22.\,12.\,1872 – 23.\,6.\,1905 Salzburg@\textsc{Karg-Bebenburg, Edgar von} (22.\,12.\,1872 – 23.\,6.\,1905 Salzburg), \emph{Militär}|pw} ein paar Mal. –\pend
           
\pstart
           Vielleicht kann uns \textsc{Gerty\pwindex{Hofmannsthal, Gertrude von 16.\,3.\,1880 Wien – 9.\,11.\,1959 Paddington@\textsc{Hofmannsthal, Gertrude von} (16.\,3.\,1880 Wien – 9.\,11.\,1959 Paddington)|pw}} die Adreſſe der Italienerin\pwindex{?? [Italienischlehrerin in Wien] @\textsc{?? [Italienischlehrerin in Wien]}|pwv}{ }ſagen, bei der{ }ſie einmal Stunden genommen hat.
               Adreſſe u Namen. Einmal war{ }ſie bei mir, einer Überſetzung wegen, wohnte damals \textsc{Hammerand\oindex{Wien@\textbf{Wien}!VIII., Josefstadt@\textbf{VIII., Josefstadt}!Hotel Hammerand@\textbf{Hotel Hammerand}, \emph{Hotel}|pw}}. –\pend
           
\pstart
           Geſtern bin ich geradelt, Hütteldorf\oindex{Wien@\textbf{Wien}!XIV., Penzing@\textbf{XIV., Penzing}!Hütteldorf@\textbf{Hütteldorf}|pw}, Neuwaldegg\oindex{Wien@\textbf{Wien}!XVII., Hernals@\textbf{XVII., Hernals}!Neuwaldegg@\textbf{Neuwaldegg}, \emph{Ehemaliger Ort}|pw}; es iſt{ }ſchon{ }ſo herbſtlich. Mein Rad
               hat{ }ſsich auf der Reiſe recht erholt.\pend
           
\pstart
           Herzliche Grüße, an Sie, \textsc{Gerty\pwindex{Hofmannsthal, Gertrude von 16.\,3.\,1880 Wien – 9.\,11.\,1959 Paddington@\textsc{Hofmannsthal, Gertrude von} (16.\,3.\,1880 Wien – 9.\,11.\,1959 Paddington)|pw}}, Hans\pwindex{Schlesinger, Hans Bernhard 20.\,7.\,1875 Wien – 13.\,3.\,1932 Salzburg@\textsc{Schlesinger, Hans Bernhard} (20.\,7.\,1875 Wien – 13.\,3.\,1932 Salzburg), \emph{Maler}|pw} von uns beiden\pwindex{Schnitzler, Olga 17.\,1.\,1882 Wien – 13.\,1.\,1970 Lugano@\textsc{Schnitzler, Olga} (17.\,1.\,1882 Wien – 13.\,1.\,1970 Lugano), \emph{Schauspielerin, Sängerin}|pwv}\pend
           
\pstart
           Ihr{\\[\baselineskip]}\spacefill\mbox{A.}\pend
           \leftskip=0em{}\selectlanguage{ngerman}\endnumbering\briefempfaengerindex{Hofmannsthal, Hugo von@\textsc{Hofmannsthal, Hugo von}!zzzSchnitzler, Arthur@\emph{von Arthur Schnitzler}!1904-09-241@{24. 9. 1904}|)be}\mylabel{L01449h}  \newcommand{\dateiname}{L01449}\newcommand{\titel}{Arthur Schnitzler an Hugo von Hofmannsthal, 24. 9. 1904}\newcommand{\editorInnen}{Herausgegeben von Martin Anton Müller}%% latex-leseansicht-abspann.tex
%% Abspann für die Leseansicht.
%% Der Schalter \ifkorrekturansicht ist bereits durch den Vorspann gesetzt.

%% latex-abspann.tex
%% Gemeinsamer Abspann für Korrekturansicht und Leseansicht.
%% Setzt den Schalter \ifkorrekturansicht voraus (gesetzt in den
%% einbindenden Dateien latex-korrekturansicht-abspann.tex bzw.
%% latex-leseansicht-abspann.tex).
%% ---------------------------------------------------------------

\normalsize

% Das esempio-Environment wird nur in der Leseansicht benötigt
\ifkorrekturansicht\else
\newenvironment{esempio}[3]%
{
    \vspace{1.5ex}
    \rlap{\underline{#1}}
    \par
    \setlength{\parindent}{0cm}
    \nopagebreak
    \leftskip=#2cm
    \rightskip=#3cm
}
{
    \par
}
\fi

\doendnotes{C}
\bigskip
\vfill

\clearpage

\footnotesize

\ifkorrekturansicht
  \lohead{\textsc{register}}
\fi

% theindex-Environment neu definieren ohne reledmac
\makeatletter
\renewenvironment{theindex}{%
  \ifkorrekturansicht
    \section*{\indexname}%
  \else
    \subsubsection*{Index der erwähnten Entitäten}%
  \fi
  \setlength{\parindent}{0pt}%
  \setlength{\parskip}{0pt plus 0.3pt}%
  \let\item\@idxitem
}{%
  \ifkorrekturansicht\clearpage\fi
}
\makeatother

\IfFileExists{\jobname-pw.ind}{\input{\jobname-pw.ind}}{}

% Quellenangabe nur in der Leseansicht
\ifkorrekturansicht\else
% Fallback-Definitionen, falls die .tex-Datei \titel etc. nicht gesetzt hat
\providecommand{\titel}{}
\providecommand{\editorInnen}{}
\providecommand{\dateiname}{\jobname}

\vspace{3cm}

\vfill

\footnotesize
\textsc{Quelle}: \titel. Herausgegeben von {\editorInnen}. In: \emph{Arthur Schnitzler: Briefwechsel mit Autorinnen und Autoren}.
 Digitale Edition, https://schnitzler-briefe.acdh.oeaw.ac.at/{\dateiname}.html (Stand \today)
\fi

\end{document}


