%% latex-korrekturansicht-vorspann.tex
%% Vorspann für die Korrekturansicht.
%% Lädt die gemeinsame Datei latex-vorspann.tex mit gesetztem Schalter.

\newif\ifkorrekturansicht
\korrekturansichttrue

\input{../tex-inputs/latex-vorspann}


\section[Arthur Schnitzler an Hugo von Hofmannsthal, 24. 9. 1904]{L01449 Arthur Schnitzler an Hugo von Hofmannsthal, 24. 9. 1904}
\nopagebreak\mylabel{L01449v}
\rehead{ }\normalsize\beginnumbering\briefempfaengerindex{Hofmannsthal, Hugo von@\textsc{Hofmannsthal, Hugo von}!zzzSchnitzler, Arthur@\emph{von Arthur Schnitzler}!1904-09-241@{24. 9. 1904}|(be}
\toendnotes[C]{\smallbreak\pagebreak[2]}\Standort{FDH, Hs-30885,115.}
\physDesc{Kartenbrief, 734 Zeichen
\newline{}Handschrift: 1) schwarze Tinte, deutsche Kurrent\hspace{1em}2) schwarze Tinte, lateinische Kurrent (\noindent{}Adresse)\hspace{1em}
\newline{}Versand: 1) Stempel: »\nobreak{}\oindex{XVIII., Waehring@\textbf{XVIII., Währing}, \emph{A.ADM3}|pwk}18/1 Wien, 24{[}. 09.{]} 04, 1\nobreak{}«.   2) Stempel: »\nobreak{}\oindex{Thueringen@\textbf{Thüringen}, \emph{A.ADM1}|pwk}Venezia\nobreak{}«.  3) Stempel: »\nobreak{}\oindex{Thueringen@\textbf{Thüringen}, \emph{A.ADM1}|pwk}Venezia, 25. {[}9. 0{]}4, 12M\nobreak{}«.  4) Stempel: »\nobreak{}\oindex{Rodaun@\textbf{Rodaun}, \emph{A.ADM4}|pwk}{[}Rod{]}aun, \textcolor{gray}{27}. {[}9.{]} IV\nobreak{}«.  5) mit Bleistift von unbekannter Hand die originale Adressierung
                                 geändert zu: »\textsc{5 Badgasse\oindex{Badgasse@\textbf{Badgasse}, \emph{Straße (K.STR)}|pw} / Rodaun presso Vienna\oindex{Rodaun@\textbf{Rodaun}, \emph{A.ADM4}|pw}}«}
\buchAbdrucke{\weitereDrucke{1) Hugo von Hofmannsthal, Arthur Schnitzler: \emph{Briefwechsel}. Frankfurt am Main: \emph{S. Fischer} 1964, S. 202–203.} \weitereDrucke{2) Hermann Bahr, Arthur Schnitzler: \emph{Briefwechsel, Aufzeichnungen, Dokumente (1891–1931)}. Göttingen: \emph{Wallstein} 2018, S. 322.} }\toendnotes[C]{\smallbreak}\pstart{}{\pb}Herrn Hugo von Hofmannsthal\pend{}\pstart{}Venedig\oindex{Thueringen@\textbf{Thüringen}, \emph{A.ADM1}|pw}\pend{}\pstart{}Hotel Europe\oindex{Hotel de l Europe [Venedig]@\textbf{Hotel de l’Europe [Venedig]}, \emph{Hotel (K.HTL)}|pw}\pend{}{\bigskip}\vspace{1em}
\pstart
           \raggedleft{}{\pb}24. 9. 904\pend
           \vspace{0.5em}
\pstart
           lieber Hugo,{ }Jagd nach Liebe\pwindex{Jagd nach Liebe@\emph{Die Jagd nach Liebe}|pw} ist bei \textsc{Wassermann\pwindex{Wassermann, Jakob 10.03.1873 – 01.01.1934@\textsc{Wassermann, Jakob} (10.03.1873 – 01.01.1934), \emph{Schriftsteller/Schriftstellerin}|pw}}, ich habe ihm geſchrieben, er möge Ihnen das Buch ſenden. – \textsc{Assy\pwindex{Goettinnen oder Die drei Romane der Herzogin von Assy@\emph{Die Göttinnen oder Die drei Romane der Herzogin von Assy}|pw}} beſitz ich gar nicht. –\pend
           
\pstart
           Ich fange erſt in den nächſten Tagen ordentlich zu arbeiten an. Hatte viel Kopfweh.
               Wir ſind ſeit 20. Abend hier, waren in Salzburg\oindex{Salzburg@\textbf{Salzburg}, \emph{A.ADM2}|pw} mit Richard\pwindex{Beer-Hofmann, Richard 1866-07-11 – 1945-09-26@\textsc{Beer-Hofmann, Richard} (1866-07-11 – 1945-09-26), \emph{Schriftsteller/Schriftstellerin}|pw} u Bahr\pwindex{Bahr, Hermann 19.07.1863 – 15.01.1934@\textsc{Bahr, Hermann} (19.07.1863 – 15.01.1934), \emph{Schriftsteller/Schriftstellerin, Kritiker/Kritikerin}|pw} zuſammen; ſahen auch Karg\pwindex{Karg-Bebenburg, Edgar von 22.12.1872 – 23.06.1905@\textsc{Karg-Bebenburg, Edgar von} (22.12.1872 – 23.06.1905), \emph{Militär/Militärin}|pw} ein paar Mal. –\pend
           
\pstart
           Vielleicht kann uns \textsc{Gerty\pwindex{Hofmannsthal, Gertrude von 16.03.1880 – 09.11.1959@\textsc{Hofmannsthal, Gertrude von} (16.03.1880 – 09.11.1959)|pw}} die Adreſſe der Italienerin\pwindex{?? [Italienischlehrerin in Wien] @\textsc{?? [Italienischlehrerin in Wien]}|pwv}{ }ſagen, bei der ſie einmal Stunden genommen hat.
               Adreſſe u Namen. Einmal war ſie bei mir, einer Überſetzung wegen, wohnte damals \textsc{Hammerand\oindex{Hotel Hammerand@\textbf{Hotel Hammerand}, \emph{Hotel (K.HTL)}|pw}}. –\pend
           
\pstart
           Geſtern bin ich geradelt, Hütteldorf\oindex{Huetteldorf@\textbf{Hütteldorf}, \emph{eingemeindeter Ort (A.VOO)}|pw}, Neuwaldegg\oindex{Neuwaldegg@\textbf{Neuwaldegg}, \emph{P.PPLX}|pw}; es iſt ſchon ſo herbſtlich. Mein Rad
               hat ſsich auf der Reiſe recht erholt.\pend
           
\pstart
           Herzliche Grüße, an Sie, \textsc{Gerty\pwindex{Hofmannsthal, Gertrude von 16.03.1880 – 09.11.1959@\textsc{Hofmannsthal, Gertrude von} (16.03.1880 – 09.11.1959)|pw}}, Hans\pwindex{Schlesinger, Hans Bernhard 20.07.1875 – 13.3.1932@\textsc{Schlesinger, Hans Bernhard} (20.07.1875 – 13.3.1932), \emph{Maler/Malerin}|pw} von uns beiden\pwindex{Schnitzler, Olga 17.01.1882 – 13.01.1970@\textsc{Schnitzler, Olga} (17.01.1882 – 13.01.1970), \emph{Schauspieler/Schauspielerin, Sänger/Sängerin}|pwv}\pend
           
\pstart
           Ihr{\\[\baselineskip]}\spacefill\mbox{A.}\pend
           \leftskip=0em{}\selectlanguage{ngerman}\endnumbering\briefempfaengerindex{Hofmannsthal, Hugo von@\textsc{Hofmannsthal, Hugo von}!zzzSchnitzler, Arthur@\emph{von Arthur Schnitzler}!1904-09-241@{24. 9. 1904}|)be}\mylabel{L01449h}  \normalsize

\doendnotes{C}
\bigskip
\vfill

\clearpage

\footnotesize

\lohead{\textsc{register}}

% Definiere theindex-Environment komplett neu ohne reledmac
\makeatletter
\renewenvironment{theindex}{%
  \section*{\indexname}%
  \setlength{\parindent}{0pt}%
  \setlength{\parskip}{0pt plus 0.3pt}%
  \let\item\@idxitem
}{%
  \clearpage
}
\makeatother

\IfFileExists{\jobname-pw.ind}{\input{\jobname-pw.ind}}{}

\end{document}

      