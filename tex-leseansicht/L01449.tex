%% latex-leseansicht-vorspann.tex
%% Vorspann für die Leseansicht.
%% Lädt die gemeinsame Datei latex-vorspann.tex mit nicht gesetztem Schalter.

\newif\ifkorrekturansicht
\korrekturansichtfalse

\input{../tex-inputs/latex-vorspann}


         
         \renewcommand{\erwaehntePersonen}{Personen:  ?? [Italienischlehrerin in Wien], Hermann Bahr, Richard Beer-Hofmann, Hugo von Hofmannsthal, Gertrude von Hofmannsthal, Edgar von Karg-Bebenburg, Hans Bernhard Schlesinger, Olga Schnitzler, Jakob Wassermann}
         \renewcommand{\erwaehnteOrte}{Orte: Badgasse, Hotel Hammerand, Hotel de l’Europe, Hütteldorf, Neuwaldegg, Rodaun, Salzburg, Thüringen, Wien, XVIII., Währing}
         \renewcommand{\erwaehnteWerke}{Werke: Die Göttinnen oder Die drei Romane der Herzogin von Assy, Die Jagd nach Liebe}
               \section[Arthur Schnitzler an Hugo von Hofmannsthal, 24. 9. 1904]{ Arthur Schnitzler an Hugo von Hofmannsthal, 24. 9. 1904}\nopagebreak\mylabel{v}\rehead{ }\begin{ledgroupsized}[t]{13cm}\normalsize\beginnumbering \toendnotes[C]{\smallbreak\pagebreak[2]} \Standort{FDH, Hs-30885,115.}
\physDesc{Kartenbrief
\newline{}Handschrift: 1) schwarze Tinte, deutsche Kurrent\hspace{1em}2) schwarze Tinte, lateinische Kurrent (\noindent{}Adresse)\hspace{1em}\newline{}Versand: 1) Stempel: »\nobreak{}\oindex{XVIII., Waehring@\textbf{XVIII., Währing}|pwk}18/1 Wien, 24{[}. 09.{]} 04, 1\nobreak{}«.   2) Stempel: »\nobreak{}\oindex{Thueringen@\textbf{Thüringen}|pwk}Venezia\nobreak{}«.  3) Stempel: »\nobreak{}\oindex{Thueringen@\textbf{Thüringen}|pwk}Venezia, 25. {[}9. 0{]}4, 12M\nobreak{}«.  4) Stempel: »\nobreak{}\oindex{Rodaun@\textbf{Rodaun}|pwk}{[}Rod{]}aun, \textcolor{gray}{27}. {[}9.{]} IV\nobreak{}«.  5) mit Bleistift von unbekannter Hand die originale Adressierung
                                 geändert zu: »\textsc{5 Badgasse\oindex{Badgasse@\textbf{Badgasse}|pw} / Rodaun presso Vienna\oindex{Rodaun@\textbf{Rodaun}|pw}}«}\buchAbdrucke{\weitereDrucke{1) Hugo von Hofmannsthal, Arthur Schnitzler: \emph{Briefwechsel}. Hg. Therese Nickl und Heinrich Schnitzler. Frankfurt am Main: \emph{S. Fischer} 1964, S. 202–203.} \weitereDrucke{2) Hermann Bahr, Arthur Schnitzler: \emph{Briefwechsel, Aufzeichnungen, Dokumente (1891–1931)}. Hg. Kurt Ifkovits und Martin Anton Müller. Göttingen: \emph{Wallstein} 2018, S. 322.} }\toendnotes[C]{\smallbreak}\pstart{}{\pb}Herrn Hugo von Hofmannsthal\pend{}\pstart{}Venedig\oindex{Thueringen@\textbf{Thüringen}|pw}\pend{}\pstart{}Hotel Europe\oindex{Hotel de l Europe@\textbf{Hotel de l’Europe}|pw}\pend{}{\bigskip}\pstart
           \raggedleft{}{\pb}24. 9. 904\pend
           \pstart
           lieber Hugo, Jagd nach Liebe\pwindex{\textcolor{red}{\textsuperscript{XXXX1 indx}}!Jagd nach Liebe1903@\strich\emph{Die Jagd nach Liebe} {[}1903{]}|pw} ist bei \textsc{Wassermann\pwindex{Wassermann, Jakob 10.03.1873 – 01.01.1934@\textsc{Wassermann, Jakob} (10.03.1873 – 01.01.1934), \emph{Schriftsteller}|pw}}, ich habe ihm geſchrieben, er möge Ihnen das Buch ſenden. – \textsc{Assy\pwindex{\textcolor{red}{\textsuperscript{XXXX1 indx}}!Goettinnen oder Die drei Romane der Herzogin von Assy1902@\strich\emph{Die Göttinnen oder Die drei Romane der Herzogin von Assy} {[}1902{]}|pw}} beſitz ich gar nicht. –\pend
           \pstart
           Ich fange erſt in den nächſten Tagen ordentlich zu arbeiten an. Hatte viel Kopfweh.
               Wir ſind ſeit 20. Abend hier, waren in Salzburg\oindex{Salzburg@\textbf{Salzburg}|pw} mit Richard\pwindex{Beer-Hofmann, Richard 1866-07-11 – 1945-09-26@\textsc{Beer-Hofmann, Richard} (1866-07-11 – 1945-09-26), \emph{Schriftsteller}|pw} u Bahr\pwindex{Bahr, Hermann 19.07.1863 – 15.01.1934@\textsc{Bahr, Hermann} (19.07.1863 – 15.01.1934), \emph{Schriftsteller, Kritiker}|pw} zuſammen; ſahen auch Karg\pwindex{Karg-Bebenburg, Edgar von 22.12.1872 – 23.06.1905@\textsc{Karg-Bebenburg, Edgar von} (22.12.1872 – 23.06.1905), \emph{Militär}|pw} ein paar Mal. –\pend
           \pstart
           Vielleicht kann uns \textsc{Gerty\pwindex{Hofmannsthal, Gertrude von 16.03.1880 – 09.11.1959@\textsc{Hofmannsthal, Gertrude von} (16.03.1880 – 09.11.1959)|pw}} die Adreſſe der Italienerin\pwindex{?? [Italienischlehrerin in Wien] @\textsc{?? [Italienischlehrerin in Wien]}|pwv}{ }ſagen, bei der ſie einmal Stunden genommen hat.
               Adreſſe u Namen. Einmal war ſie bei mir, einer Überſetzung wegen, wohnte damals \textsc{Hammerand\oindex{Hotel Hammerand@\textbf{Hotel Hammerand}|pw}}. –\pend
           \pstart
           Geſtern bin ich geradelt, Hütteldorf\oindex{Huetteldorf@\textbf{Hütteldorf}|pw}, Neuwaldegg\oindex{Neuwaldegg@\textbf{Neuwaldegg}|pw}; es iſt ſchon ſo herbſtlich. Mein Rad
               hat ſsich auf der Reiſe recht erholt.\pend
           \pstart
           Herzliche Grüße, an Sie, \textsc{Gerty\pwindex{Hofmannsthal, Gertrude von 16.03.1880 – 09.11.1959@\textsc{Hofmannsthal, Gertrude von} (16.03.1880 – 09.11.1959)|pw}}, Hans\pwindex{Schlesinger, Hans Bernhard 20.07.1875 – 13.3.1932@\textsc{Schlesinger, Hans Bernhard} (20.07.1875 – 13.3.1932), \emph{Maler}|pw} von uns beiden\pwindex{Schnitzler, Olga 17.01.1882 – 13.01.1970@\textsc{Schnitzler, Olga} (17.01.1882 – 13.01.1970), \emph{Schauspielerin, Sängerin}|pwv}\pend
           \pstart
           Ihr{\\[\baselineskip]}\spacefill\mbox{A.}\pend
           \leftskip=0em{}
         
         \endnumbering\mylabel{h}\end{ledgroupsized}  \newcommand{\dateiname}{L01449}\newcommand{\titel}{Arthur Schnitzler an Hugo von Hofmannsthal, 24. 9. 1904}\newcommand{\editorInnen}{ Martin Anton Müller und Gerd-Hermann Susen}%% latex-leseansicht-abspann.tex
%% Abspann für die Leseansicht.
%% Der Schalter \ifkorrekturansicht ist bereits durch den Vorspann gesetzt.

%% latex-abspann.tex
%% Gemeinsamer Abspann für Korrekturansicht und Leseansicht.
%% Setzt den Schalter \ifkorrekturansicht voraus (gesetzt in den
%% einbindenden Dateien latex-korrekturansicht-abspann.tex bzw.
%% latex-leseansicht-abspann.tex).
%% ---------------------------------------------------------------

\normalsize

% Das esempio-Environment wird nur in der Leseansicht benötigt
\ifkorrekturansicht\else
\newenvironment{esempio}[3]%
{
    \vspace{1.5ex}
    \rlap{\underline{#1}}
    \par
    \setlength{\parindent}{0cm}
    \nopagebreak
    \leftskip=#2cm
    \rightskip=#3cm
}
{
    \par
}
\fi

\doendnotes{C}
\bigskip
\vfill

\clearpage

\footnotesize

\ifkorrekturansicht
  \lohead{\textsc{register}}
\fi

% theindex-Environment neu definieren ohne reledmac
\makeatletter
\renewenvironment{theindex}{%
  \ifkorrekturansicht
    \section*{\indexname}%
  \else
    \subsubsection*{Index der erwähnten Entitäten}%
  \fi
  \setlength{\parindent}{0pt}%
  \setlength{\parskip}{0pt plus 0.3pt}%
  \let\item\@idxitem
}{%
  \ifkorrekturansicht\clearpage\fi
}
\makeatother

\IfFileExists{\jobname-pw.ind}{\input{\jobname-pw.ind}}{}

% Quellenangabe nur in der Leseansicht
\ifkorrekturansicht\else
% Fallback-Definitionen, falls die .tex-Datei \titel etc. nicht gesetzt hat
\providecommand{\titel}{}
\providecommand{\editorInnen}{}
\providecommand{\dateiname}{\jobname}

\vspace{3cm}

\vfill

\footnotesize
\textsc{Quelle}: \titel. Herausgegeben von {\editorInnen}. In: \emph{Arthur Schnitzler: Briefwechsel mit Autorinnen und Autoren}.
 Digitale Edition, https://schnitzler-briefe.acdh.oeaw.ac.at/{\dateiname}.html (Stand \today)
\fi

\end{document}


      