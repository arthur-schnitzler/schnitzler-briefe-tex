%% latex-leseansicht-vorspann.tex
%% Vorspann für die Leseansicht.
%% Lädt die gemeinsame Datei latex-vorspann.tex mit nicht gesetztem Schalter.

\newif\ifkorrekturansicht
\korrekturansichtfalse

\input{../tex-inputs/latex-vorspann}


\section[Arthur Schnitzler an Richard Beer-Hofmann, 26. 10. 1894]{L00392 Arthur Schnitzler an Richard Beer-Hofmann, 26. 10. 1894}
\nopagebreak\mylabel{L00392v}
\rehead{ }\normalsize\beginnumbering\briefempfaengerindex{Beer-Hofmann, Richard@\textsc{Beer-Hofmann, Richard}!zzzSchnitzler, Arthur@\emph{von Arthur Schnitzler}!1894-10-261@{26. 10. 1894}|(be}
\toendnotes[C]{\smallbreak\pagebreak[2]}
\correspDesc{Versand  durch Arthur Schnitzler am 26. 10. 1894 in Wien
\newline{}Erhalt  durch Richard Beer-Hofmann am 28. 10. 1894 in Venedig}\toendnotes[C]{\smallbreak}
\Standort{YCGL, MSS 31.}
\physDesc{Brief, 2 Blätter, 6 Seiten, Kuvert, 3000 Zeichen
\newline{}Handschrift: schwarze Tinte, deutsche Kurrent
\newline{}Versand: 1) Stempel: »\nobreak{}\oindex{I., Innere Stadt@\textbf{I., Innere Stadt}, \emph{Verwaltungsgebiet}|pwk}Wien 1/1, {[}26.{]} 10. 94\nobreak{}«.   2) Stempel: »\nobreak{}\oindex{Venedig@\textbf{Venedig}|pwk}Venezia, 28 10-94, 7 N\nobreak{}«. }
\buchAbdrucke{\weitereDrucke{Arthur Schnitzler, Richard Beer-Hofmann: \emph{Briefwechsel 1891–1931}. Herausgegeben von Konstanze Fliedl. Wien, Zürich: \emph{Europaverlag} 1992, S. 68–69.} }\toendnotes[C]{\smallbreak}\pstart{}{\pb}\textsc{Dr. Arthur Schnitzler}, Wien, IX. Frankgaſſe 1\oindex{Wien@\textbf{Wien}!IX., Alsergrund@\textbf{IX., Alsergrund}!Frankgasse 1@\textbf{Frankgasse 1}, \emph{Wohngebäude}|pw}.\pend{}{\bigskip}\pstart{}Herrn Dr. \textsc{Richard Beer Hofmann}\pend{}\pstart{}\textsc{Venedig\oindex{Venedig@\textbf{Venedig}|pw}}\pend{}\pstart{}\textsc{Hotel Bauer u. Grünwald}\oindex{Grand Hotel Bauer-Grünwald@\textbf{Grand Hotel Bauer-Grünwald}, \emph{Hotel}|pw}\pend{}\pstart{}\textsc{Italien\oindex{Italien@\textbf{Italien}|pw}}\pend{}{\bigskip}\vspace{1em}
\pstart
           \raggedleft{}{\pb}\uline{26. 10. 94}\pend
           \vspace{0.5em}
\pstart
           Lieber Richard, ich denke, der Brief da trifft noch vor Ihnen in Venedig\oindex{Venedig@\textbf{Venedig}|pw} ein –{ }ſo bin ich alſo aller peinvollen
               Gedanken ledig, die Sie mir für den Fall dſs \textsc{etc}
               profezeihen. – Heut hab ich Ihren Brief über Pompeji\oindex{Pompeji@\textbf{Pompeji}, \emph{Ausgrabung}|pw} bekommen. »Ueber Pompeji\oindex{Pompeji@\textbf{Pompeji}, \emph{Ausgrabung}|pw}« –
               d. h. wo Sie{ }ſagen, daſs Sie{ }ſich nach wirklichen römiſchen\oindex{Rom@\textbf{Rom}, \emph{Hauptstadt}|pwv} Bädern{ }ſehnen. –\pend
           
\pstart
           Von mir iſt nichts neues zu{ }ſagen; nicht viel. – Sie wiſſen, dſs »Sterben\pwindex{Schnitzler, Arthur 15.\,5.\,1862 Wien – 21.\,10.\,1931 ebd.@\textsc{Schnitzler, Arthur} (15.\,5.\,1862 Wien – 21.\,10.\,1931 ebd.), \emph{Schriftsteller, Mediziner}!Sterben. Novelle@\strich\emph{Sterben. Novelle}|pw}« jetzt allmälig erſcheint, wiſſen auch, dſs ich große
               Angſt vor den Correctur{\pb}bogen hatte. Ich bin aber
               angenehm enttäuſcht; es ist einiges wirklich{ }ſchön\substVorne{}\textsuperscript{s}\substDazwischen{}e\substHinten{} drin. – Geben Sie nur Acht, was die Kritik{ }ſagen wird. Ich bin feſt
               überzeugt, daſs man mich viel{ }ſchlechter, d. h. frecher behandeln wird als zu Anatols\pwindex{Schnitzler, Arthur 15.\,5.\,1862 Wien – 21.\,10.\,1931 ebd.@\textsc{Schnitzler, Arthur} (15.\,5.\,1862 Wien – 21.\,10.\,1931 ebd.), \emph{Schriftsteller, Mediziner}!Anatol@\strich\emph{Anatol}|pw} Zeiten.\pend
           
\pstart
           – Die »\textsc{Liebelei}\pwindex{Schnitzler, Arthur 15.\,5.\,1862 Wien – 21.\,10.\,1931 ebd.@\textsc{Schnitzler, Arthur} (15.\,5.\,1862 Wien – 21.\,10.\,1931 ebd.), \emph{Schriftsteller, Mediziner}!Liebelei. Schauspiel in drei Akten@\strich\emph{Liebelei. Schauspiel in drei Akten}|pw}« werd ich Anfang nächſter Woche einreichen (d. i. alſo vor
                  1. November.) –\pend
           
\pstart
           Meine Sti{\geminationm}ung iſt nicht{ }ſehr gut. Ich{ }ſpüre die Enge
               meiner Exiſtenz zuweilen{ }ſchmerzlich. Und we{\geminationn} man{ }ſich
               über die Enge{ }ſchon hinwegtäuſcht durch ehrliche Verſuche, wenigſtens mit des Geiſtes
               Flügeln (zu denen – ach{ }ſo leicht kein körperlicher u. ſ. w.) allem davon-zu{\pb}flattern; da kommt plötzlich das gewiſſe
               Damoklesgefühl über einen. Sie wiſſen: die vielen, vielen Schwerter – aber{ }ſie tödten
               nicht einmal alle gleich. –\pend
           
\pstart
           Es wird gut{ }ſein, we{\geminationn} ich möglichſt bald wieder was
               großes zu{ }ſchreiben anfange, was vielleicht weder gut noch groß{ }ſein wird, was ein
               Wortſpiel iſt oder auch kein Wortſpiel oder doch ein Wortſpiel wie \textsc{R. B.-H.}{ }ſchreiben würde, daſs A. S.{ }ſchreiben würde –\pend
           
\pstart
           Ich war bei der \label{K_L00392-1v}\edtext{\textsc{Première}}{\lemma{\textnormal{\emph{Première}}}\Cendnote{\textnormal{Die Premiere fand am 20. 10. 1894 am \emph{Deutschen Volkstheater}\orgindex{Volkstheater@Volkstheater|pwk} statt.
               }}}\label{K_L00392-1} der Comödianten\pwindex{\textcolor{red}{\textsuperscript{XXXX indx1}}!Comödianten@\strich\emph{Comödianten}|pw}. Es iſt ein{ }ſchlechtes
               Stück mit einigen gut angelegten Figuren, einer dramatiſch {\pb}vortrefflichen Scene, (– die \introOben{}ſich\introOben{} wie ein lebendiges Auge, das leuchtet, \strikeout{ausnimmt} in einer Wachspuppe ausnimmt;) mit ein paar vortrefflichen
               Wendungen – \substVorne{}\textsuperscript{aber}\substDazwischen{}ſogar\substHinten{} mit etwas Elan im Beginn; im ganzen aber doch nur{ }ſpringende Epiſoden und
               keine{ }ſchreitende Handlung. Was{ }ſich als letztere ausgibt,{ }ſtört geradezu. Es iſt der
               Holzſtab, der durch die verzuckerten Mandeln geſteckt wird – freilich fallen die
               Mandeln ohne das Holz auseinander; – aber gegeſſen werden doch nur die Mandeln – und
               das Holz – nun?? man leckt es ab, woran dieser Vergleich,{ }ſcheint mir, {\pb}ſchmählich zu Grunde geht. –\pend
           
\pstart
           Geſtern hab ich wieder einmal Kabale u Liebe\pwindex{Schiller, Friedrich von 10.\,11.\,1759 Marbach am Neckar – 9.\,5.\,1805 Weimar@\textsc{Schiller, Friedrich von} (10.\,11.\,1759 Marbach am Neckar – 9.\,5.\,1805 Weimar), \emph{Schriftsteller, Historiker, Philosoph}!Kabale und Liebe. Ein bürgerliches Trauerspiel in fünf Aufzügen@\strich\emph{Kabale und Liebe. Ein bürgerliches Trauerspiel in fünf Aufzügen}|pw}
               geſehn. Es iſt unbegreiflich, daſs man einen{ }ſo raffinirt guten und auch innerlich
               großartigen erſten und zweiten Akt – und einen{ }ſo unſäglich du{\geminationm}en fünften Akt{ }ſchreiben kann. – Und dann – die Liebe
               bei Schiller\pwindex{Schiller, Friedrich von 10.\,11.\,1759 Marbach am Neckar – 9.\,5.\,1805 Weimar@\textsc{Schiller, Friedrich von} (10.\,11.\,1759 Marbach am Neckar – 9.\,5.\,1805 Weimar), \emph{Schriftsteller, Historiker, Philosoph}|pw} geht mir auf die Nerven. Ihre
               Bemerkung über »Lebt wohl, ihr Berge« – (ſind Sie geſchmeichelt?) läßt{ }ſich auch da
               hundertmal machen. –\pend
           
\pstart
           Kennen Sie den Komödiantenroman\pwindex{Scarron, Paul 14.\,7.\,1610 Paris – 7.\,10.\,1660 ebd.@\textsc{Scarron, Paul} (14.\,7.\,1610 Paris – 7.\,10.\,1660 ebd.), \emph{Schriftsteller}!Komödianten-Roman@\strich\emph{Der Komödianten-Roman}|pw} von \textsc{Scarron}\pwindex{Scarron, Paul 14.\,7.\,1610 Paris – 7.\,10.\,1660 ebd.@\textsc{Scarron, Paul} (14.\,7.\,1610 Paris – 7.\,10.\,1660 ebd.), \emph{Schriftsteller}|pw}? Eben leſe ich ihn mit viel Vergnügen. – Ich werde zum Nachtmahl {\pb}gerufen. Leben Sie wohl, ko{\geminationm}en Sie bald zurück, und{ }ſchämen Sie{ }ſich nicht, daſs
               Sie{ }ſich sogar – nach den Wien\oindex{Wien@\textbf{Wien}, \emph{Verwaltungsgebiet}|pw}er Kaffeehausecken{ }ſehnen. –\pend
           \pstart Herzlich der Ihre \spacefill\mbox{Arthur.}\pend{}
\pstart
           \noindent{}Sie{ }ſchreiben mir natürlich auch noch eine Zeile aus Venedig\oindex{Venedig@\textbf{Venedig}|pw}? –\pend
           \selectlanguage{ngerman}\endnumbering\briefempfaengerindex{Beer-Hofmann, Richard@\textsc{Beer-Hofmann, Richard}!zzzSchnitzler, Arthur@\emph{von Arthur Schnitzler}!1894-10-261@{26. 10. 1894}|)be}\mylabel{L00392h}  \newcommand{\dateiname}{L00392}\newcommand{\titel}{Arthur Schnitzler an Richard Beer-Hofmann, 26. 10. 1894}\newcommand{\editorInnen}{Martin Anton Müller und Gerd-Hermann Susen}%% latex-leseansicht-abspann.tex
%% Abspann für die Leseansicht.
%% Der Schalter \ifkorrekturansicht ist bereits durch den Vorspann gesetzt.

%% latex-abspann.tex
%% Gemeinsamer Abspann für Korrekturansicht und Leseansicht.
%% Setzt den Schalter \ifkorrekturansicht voraus (gesetzt in den
%% einbindenden Dateien latex-korrekturansicht-abspann.tex bzw.
%% latex-leseansicht-abspann.tex).
%% ---------------------------------------------------------------

\normalsize

% Das esempio-Environment wird nur in der Leseansicht benötigt
\ifkorrekturansicht\else
\newenvironment{esempio}[3]%
{
    \vspace{1.5ex}
    \rlap{\underline{#1}}
    \par
    \setlength{\parindent}{0cm}
    \nopagebreak
    \leftskip=#2cm
    \rightskip=#3cm
}
{
    \par
}
\fi

\doendnotes{C}
\bigskip
\vfill

\clearpage

\footnotesize

\ifkorrekturansicht
  \lohead{\textsc{register}}
\fi

% theindex-Environment neu definieren ohne reledmac
\makeatletter
\renewenvironment{theindex}{%
  \ifkorrekturansicht
    \section*{\indexname}%
  \else
    \subsubsection*{Index der erwähnten Entitäten}%
  \fi
  \setlength{\parindent}{0pt}%
  \setlength{\parskip}{0pt plus 0.3pt}%
  \let\item\@idxitem
}{%
  \ifkorrekturansicht\clearpage\fi
}
\makeatother

\IfFileExists{\jobname-pw.ind}{\input{\jobname-pw.ind}}{}

% Quellenangabe nur in der Leseansicht
\ifkorrekturansicht\else
% Fallback-Definitionen, falls die .tex-Datei \titel etc. nicht gesetzt hat
\providecommand{\titel}{}
\providecommand{\editorInnen}{}
\providecommand{\dateiname}{\jobname}

\vspace{3cm}

\vfill

\footnotesize
\textsc{Quelle}: \titel. Herausgegeben von {\editorInnen}. In: \emph{Arthur Schnitzler: Briefwechsel mit Autorinnen und Autoren}.
 Digitale Edition, https://schnitzler-briefe.acdh.oeaw.ac.at/{\dateiname}.html (Stand \today)
\fi

\end{document}


