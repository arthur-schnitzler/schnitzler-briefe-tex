%% latex-korrekturansicht-vorspann.tex
%% Vorspann für die Korrekturansicht.
%% Lädt die gemeinsame Datei latex-vorspann.tex mit gesetztem Schalter.

\newif\ifkorrekturansicht
\korrekturansichttrue

\input{../tex-inputs/latex-vorspann}


\section[Arthur Schnitzler an Richard Beer-Hofmann, 26. 10. 1894]{L00392 Arthur Schnitzler an Richard Beer-Hofmann, 26. 10. 1894}
\nopagebreak\mylabel{L00392v}
\rehead{ }\normalsize\beginnumbering\briefempfaengerindex{Beer-Hofmann, Richard@\textsc{Beer-Hofmann, Richard}!zzzSchnitzler, Arthur@\emph{von Arthur Schnitzler}!1894-10-261@{26. 10. 1894}|(be}
\toendnotes[C]{\smallbreak\pagebreak[2]}\Standort{YCGL, MSS 31.}
\physDesc{Brief, 2 Blätter, 6 Seiten, Umschlag, 3000 Zeichen
\newline{}Handschrift: schwarze Tinte, deutsche Kurrent
\newline{}Versand: 1) Stempel: »\nobreak{}\oindex{I., Innere Stadt@\textbf{I., Innere Stadt}, \emph{A.ADM3}|pwk}Wien 1/1, {[}26.{]} 10. 94\nobreak{}«.   2) Stempel: »\nobreak{}\oindex{Venedig@\textbf{Venedig}, \emph{P.PPLA}|pwk}Venezia, 28 10-94, 7 N\nobreak{}«. }
\buchAbdrucke{\weitereDrucke{Arthur Schnitzler, Richard Beer-Hofmann: \emph{Briefwechsel 1891–1931}. Wien, Zürich: \emph{Europaverlag} 1992, S. 68–69.} }\toendnotes[C]{\smallbreak}\pstart{}{\pb}\textsc{Dr. Arthur Schnitzler}, Wien, IX. Frankgaſſe 1\oindex{Frankgasse 1@\textbf{Frankgasse 1}, \emph{Wohngebäude (K.WHS)}|pw}.\pend{}{\bigskip}\pstart{}Herrn Dr. \textsc{Richard Beer Hofmann}\pend{}\pstart{}\textsc{Venedig\oindex{Venedig@\textbf{Venedig}, \emph{P.PPLA}|pw}}\pend{}\pstart{}\textsc{Hotel Bauer u. Grünwald}\oindex{Grand Hotel Bauer-Gruenwald@\textbf{Grand Hotel Bauer-Grünwald}, \emph{Hotel (K.HTL)}|pw}\pend{}\pstart{}\textsc{Italien\oindex{Italien@\textbf{Italien}, \emph{A.PCLI}|pw}}\pend{}{\bigskip}\vspace{1em}
\pstart
           \raggedleft{}{\pb}\uline{26. 10. 94}\pend
           \vspace{0.5em}
\pstart
           Lieber Richard, ich denke, der Brief da trifft noch vor Ihnen in Venedig\oindex{Venedig@\textbf{Venedig}, \emph{P.PPLA}|pw} ein – ſo bin ich alſo aller peinvollen
               Gedanken ledig, die Sie mir für den Fall dſs \textsc{etc}
               profezeihen. – Heut hab ich Ihren Brief über Pompeji\oindex{Pompeji@\textbf{Pompeji}, \emph{S.ANS}|pw} bekommen. »Ueber Pompeji\oindex{Pompeji@\textbf{Pompeji}, \emph{S.ANS}|pw}« –
               d. h. wo Sie ſagen, daſs Sie ſich nach wirklichen römiſchen\oindex{Rom@\textbf{Rom}, \emph{P.PPLC}|pwv} Bädern ſehnen. –\pend
           
\pstart
           Von mir iſt nichts neues zu ſagen; nicht viel. – Sie wiſſen, dſs »Sterben\pwindex{Sterben. Novelle@\emph{Sterben. Novelle}|pw}« jetzt allmälig erſcheint, wiſſen auch, dſs ich große
               Angſt vor den Correctur{\pb}bogen hatte. Ich bin aber
               angenehm enttäuſcht; es ist einiges wirklich ſchön\substVorne{}\textsuperscript{s}\substDazwischen{}e\substHinten{} drin. – Geben Sie nur Acht, was die Kritik ſagen wird. Ich bin feſt
               überzeugt, daſs man mich viel ſchlechter, d. h. frecher behandeln wird als zu Anatols\pwindex{Anatol@\emph{Anatol}|pw} Zeiten.\pend
           
\pstart
           – Die »\textsc{Liebelei}\pwindex{Liebelei. Schauspiel in drei Akten@\emph{Liebelei. Schauspiel in drei Akten}|pw}« werd ich Anfang nächſter Woche einreichen (d. i. alſo vor
                  1. November.) –\pend
           
\pstart
           Meine Sti{\geminationm}ung iſt nicht ſehr gut. Ich ſpüre die Enge
               meiner Exiſtenz zuweilen ſchmerzlich. Und we{\geminationn} man ſich
               über die Enge ſchon hinwegtäuſcht durch ehrliche Verſuche, wenigſtens mit des Geiſtes
               Flügeln (zu denen – ach ſo leicht kein körperlicher u. ſ. w.) allem davon-zu{\pb}flattern; da kommt plötzlich das gewiſſe
               Damoklesgefühl über einen. Sie wiſſen: die vielen, vielen Schwerter – aber ſie tödten
               nicht einmal alle gleich. –\pend
           
\pstart
           Es wird gut ſein, we{\geminationn} ich möglichſt bald wieder was
               großes zu ſchreiben anfange, was vielleicht weder gut noch groß ſein wird, was ein
               Wortſpiel iſt oder auch kein Wortſpiel oder doch ein Wortſpiel wie \textsc{R. B.-H.}{ }ſchreiben würde, daſs A. S. ſchreiben würde –\pend
           
\pstart
           Ich war bei der \label{K_L00392-1v}\edtext{\textsc{Première}}{\lemma{\textnormal{\emph{Première}}}\Cendnote{\textnormal{Die Premiere fand am 20. 10. 1894 am \emph{Deutschen Volkstheater}\orgindex{Volkstheater@Volkstheater|pwk} statt.
               }}}\label{K_L00392-1} der Comödianten\pwindex{Comoedianten@\emph{Comödianten}|pw}. Es iſt ein ſchlechtes
               Stück mit einigen gut angelegten Figuren, einer dramatiſch {\pb}vortrefflichen Scene, (– die \introOben{}ſich\introOben{} wie ein lebendiges Auge, das leuchtet, \strikeout{ausnimmt} in einer Wachspuppe ausnimmt;) mit ein paar vortrefflichen
               Wendungen – \substVorne{}\textsuperscript{aber}\substDazwischen{}ſogar\substHinten{} mit etwas Elan im Beginn; im ganzen aber doch nur ſpringende Epiſoden und
               keine ſchreitende Handlung. Was ſich als letztere ausgibt, ſtört geradezu. Es iſt der
               Holzſtab, der durch die verzuckerten Mandeln geſteckt wird – freilich fallen die
               Mandeln ohne das Holz auseinander; – aber gegeſſen werden doch nur die Mandeln – und
               das Holz – nun?? man leckt es ab, woran dieser Vergleich, ſcheint mir, {\pb}ſchmählich zu Grunde geht. –\pend
           
\pstart
           Geſtern hab ich wieder einmal Kabale u Liebe\pwindex{Kabale und Liebe@\emph{Kabale und Liebe}|pw}
               geſehn. Es iſt unbegreiflich, daſs man einen ſo raffinirt guten und auch innerlich
               großartigen erſten und zweiten Akt – und einen ſo unſäglich du{\geminationm}en fünften Akt ſchreiben kann. – Und dann – die Liebe
               bei Schiller\pwindex{Schiller, Friedrich von 10.11.1759 – 09.05.1805@\textsc{Schiller, Friedrich von} (10.11.1759 – 09.05.1805), \emph{Schriftsteller/Schriftstellerin, Historiker/Historikerin, Philosoph/Philosophin}|pw} geht mir auf die Nerven. Ihre
               Bemerkung über »Lebt wohl, ihr Berge« – (ſind Sie geſchmeichelt?) läßt ſich auch da
               hundertmal machen. –\pend
           
\pstart
           Kennen Sie den Komödiantenroman\pwindex{Komoedianten-Roman@\emph{Der Komödianten-Roman}|pw} von \textsc{Scarron}\pwindex{Scarron, Paul 1610-07-14 – 1660-10-07@\textsc{Scarron, Paul} (1610-07-14 – 1660-10-07), \emph{Schriftsteller/Schriftstellerin}|pw}? Eben leſe ich ihn mit viel Vergnügen. – Ich werde zum Nachtmahl {\pb}gerufen. Leben Sie wohl, ko{\geminationm}en Sie bald zurück, und ſchämen Sie ſich nicht, daſs
               Sie ſich sogar – nach den Wien\oindex{Wien@\textbf{Wien}, \emph{A.ADM2}|pw}er Kaffeehausecken
               ſehnen. –\pend
           \pstart Herzlich der Ihre \spacefill\mbox{Arthur.}\pend{}
\pstart
           \noindent{}Sie ſchreiben mir natürlich auch noch eine Zeile aus Venedig\oindex{Venedig@\textbf{Venedig}, \emph{P.PPLA}|pw}? –\pend
           \selectlanguage{ngerman}\endnumbering\briefempfaengerindex{Beer-Hofmann, Richard@\textsc{Beer-Hofmann, Richard}!zzzSchnitzler, Arthur@\emph{von Arthur Schnitzler}!1894-10-261@{26. 10. 1894}|)be}\mylabel{L00392h}  \normalsize

\doendnotes{C}
\bigskip
\vfill

\clearpage

\footnotesize

\lohead{\textsc{register}}

% Definiere theindex-Environment komplett neu ohne reledmac
\makeatletter
\renewenvironment{theindex}{%
  \section*{\indexname}%
  \setlength{\parindent}{0pt}%
  \setlength{\parskip}{0pt plus 0.3pt}%
  \let\item\@idxitem
}{%
  \clearpage
}
\makeatother

\IfFileExists{\jobname-pw.ind}{\input{\jobname-pw.ind}}{}

\end{document}

      