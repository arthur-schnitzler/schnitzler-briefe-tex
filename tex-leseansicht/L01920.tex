%% latex-korrekturansicht-vorspann.tex
%% Vorspann für die Korrekturansicht.
%% Lädt die gemeinsame Datei latex-vorspann.tex mit gesetztem Schalter.

\newif\ifkorrekturansicht
\korrekturansichttrue

\input{../tex-inputs/latex-vorspann}


\section[Stefan Großmann an Arthur Schnitzler, 31. 3. 1910]{L01920 Stefan Großmann an Arthur Schnitzler, 31. 3. 1910}
\nopagebreak\mylabel{L01920v}
\rehead{ }\normalsize\beginnumbering\briefempfaengerindex{Schnitzler, Arthur@\textsc{Schnitzler, Arthur}!zzzGrossmann, Stefan@\emph{von Stefan Großmann}!1910-03-311@{31. 3. 1910}|(be}
\toendnotes[C]{\smallbreak\pagebreak[2]}\Standort{CUL, Schnitzler, B 34.}
\physDesc{Brief, 1 Blatt, 2 Seiten, 1064 Zeichen
\newline{}Handschrift: 1) schwarze Tinte, deutsche Kurrent\hspace{1em}2) schwarze Tinte, lateinische Kurrent (\noindent{}bis einschließlich der Aufzählung der Schauspieler)\hspace{1em}
\newline{}Schnitzler: mit rotem Buntstift eine Unterstreichung 
\newline{}Ordnung: mit Bleistift von unbekannter Hand nummeriert:
                                 »7« }\toendnotes[C]{\smallbreak}
\pstart
           {\pb}\textcolor{gray}{\textbf{STEFAN GROSSMANN}}\hfill WIEN, I.\oindex{I., Innere Stadt@\textbf{I., Innere Stadt}, \emph{A.ADM3}|pw}, GRABEN 29a\oindex{Graben@\textbf{Graben}, \emph{Straße (K.STR)}|pw}\pend
           
\pstart
           \centering{}31. III. 10\pend
           
\pstart{}Sehr verehrter Herr!\pend\vspace{0.5em}
\pstart
           Aufrichtigen Dank für Ihre gütige Erlaubnis. Der Verein\orgindex{Wiener Freie Volksbuehne@Wiener Freie Volksbühne|pwv} (der langsam in eine bürgerliche Breite kommt, es
               gehören ihm heute schon 12000 Mitglieder an) bittet Sie, zu gestatten, dass wir dem
                  Verleger\pwindex{Fischer, Samuel 24.12.1859 – 15.10.1934@\textsc{Fischer, Samuel} (24.12.1859 – 15.10.1934), \emph{Verleger/Verlegerin}|pwv} 5{\%} Tantieme zahlen. Reicher sind wir noch nicht.\pend
           
\pstart
           Ich verstehe vollkommen, dass Ihnen die Anfügung der »Frage an das Schicksal\pwindex{Frage an das Schicksal@\emph{Die Frage an das Schicksal}|pw}« nicht gefällt. Aber die Neue W\textsuperscript{r} Bühne\orgindex{Neue Wiener Buehne@Neue Wiener Bühne|pw} behauptet, für den
                  {[}»{]}Puppenspieler\pwindex{Puppenspieler. Studie in einem Aufzuge@\emph{Der Puppenspieler. Studie in einem Aufzuge}|pw}« absolut nicht die Zeit für
               nöthige Proben zu haben. So musste ich, wider besseres Wissen, {\pb}im Interesse der guten Ausarbeitung der »Literatur\pwindex{Literatur@\emph{Literatur}|pw}« und »Masken\pwindex{letzten Masken@\emph{Die letzten Masken}|pw}« einwilligen.\pend
           
\pstart
           In Literatur\pwindex{Literatur@\emph{Literatur}|pw} sind \uline{Charlé\pwindex{Charle, Gustav 28.02.1871 – 1940?@\textsc{Charlé, Gustav} (28.02.1871 – 1940?), \emph{Theaterleiter/Theaterleiterin, Schauspieler/Schauspielerin}|pw}}, Fr v. \uline{Linden}\pwindex{Linden, Constance von *~14.05.1877@\textsc{Linden, Constance von} (*~14.05.1877), \emph{Schauspieler/Schauspielerin}|pw} (die ausgezeichnet wird), Hr \uline{Ziegler}\pwindex{Ziegler, Hans 09.05.1879 – 25.12.1961@\textsc{Ziegler, Hans} (09.05.1879 – 25.12.1961), \emph{Theaterleiter/Theaterleiterin, Schauspieler/Schauspielerin, Komponist/Komponistin}|pw}, – in Masken\pwindex{letzten Masken@\emph{Die letzten Masken}|pw} Herr \uline{Charlé}\pwindex{Charle, Gustav 28.02.1871 – 1940?@\textsc{Charlé, Gustav} (28.02.1871 – 1940?), \emph{Theaterleiter/Theaterleiterin, Schauspieler/Schauspielerin}|pw}, Herr \uline{Heyse}\pwindex{Heyse, Emil 1875-05-17 – 1949-04-28@\textsc{Heyse, Emil} (1875-05-17 – 1949-04-28), \emph{Schauspieler/Schauspielerin}|pw} (Weihgast\pwindex{letzten Masken@\emph{Die letzten Masken}|pwv})
               beſchäftigt.\pend
           
\pstart
           Gern würde ich Sie einmal als Gaſt bei einer Aufführung des \uline{\textsc{Halben Held}\pwindex{halber Held. Tragoedie in fuenf Aufzuegen@\emph{Ein halber Held. Tragödie in fünf Aufzügen}|pw}} v \textsc{H Eulenberg}\pwindex{Eulenberg, Herbert 25.01.1876 – 04.09.1949@\textsc{Eulenberg, Herbert} (25.01.1876 – 04.09.1949), \emph{Schriftsteller/Schriftstellerin}|pw} begrüßen, auch deshalb, weil es eine paſſable Regiſſeurarbeit von mir iſt.
               Wollen Sie unſer Gaſt ſein?\pend
           
\pstart
           Ich habe die Hoffnung, daſs Sie mich als Regiſſeur noch einmal werden brauchen
               können. – – –\pend
           
\pstart
           Mit den beſten Gefühlen{\\[\baselineskip]}\uline{aufrichtig ergeben}: \spacefill\mbox{Stefan Großmann}\pend
           \leftskip=0em{}\selectlanguage{ngerman}\endnumbering\briefempfaengerindex{Schnitzler, Arthur@\textsc{Schnitzler, Arthur}!zzzGrossmann, Stefan@\emph{von Stefan Großmann}!1910-03-311@{31. 3. 1910}|)be}\mylabel{L01920h}  \normalsize

\doendnotes{C}
\bigskip
\vfill

\clearpage

\footnotesize

\lohead{\textsc{register}}

% Definiere theindex-Environment komplett neu ohne reledmac
\makeatletter
\renewenvironment{theindex}{%
  \section*{\indexname}%
  \setlength{\parindent}{0pt}%
  \setlength{\parskip}{0pt plus 0.3pt}%
  \let\item\@idxitem
}{%
  \clearpage
}
\makeatother

\IfFileExists{\jobname-pw.ind}{\input{\jobname-pw.ind}}{}

\end{document}

      