%% latex-leseansicht-vorspann.tex
%% Vorspann für die Leseansicht.
%% Lädt die gemeinsame Datei latex-vorspann.tex mit nicht gesetztem Schalter.

\newif\ifkorrekturansicht
\korrekturansichtfalse

\input{../tex-inputs/latex-vorspann}


\section[Stefan Großmann an Arthur Schnitzler, 31. 3. 1910]{L01920 Stefan Großmann an Arthur Schnitzler, 31. 3. 1910}
\nopagebreak\mylabel{L01920v}
\rehead{ }\normalsize\beginnumbering\briefempfaengerindex{Schnitzler, Arthur@\textsc{Schnitzler, Arthur}!zzzGroßmann, Stefan@\emph{von Stefan Großmann}!1910-03-311@{31. 3. 1910}|(be}
\toendnotes[C]{\smallbreak\pagebreak[2]}
\correspDesc{Versand  durch Stefan Großmann am 31. 3. 1910 in Wien
\newline{}Erhalt  durch Arthur Schnitzler im Zeitraum [31. 3. 1910
                  – 4. 4. 1910?] in Wien}\toendnotes[C]{\smallbreak}
\Standort{CUL, Schnitzler, B 34.}
\physDesc{Brief, 1 Blatt, 2 Seiten, 1064 Zeichen
\newline{}Handschrift: 1) schwarze Tinte, deutsche Kurrent\hspace{1em}2) schwarze Tinte, lateinische Kurrent (\noindent{}bis einschließlich der Aufzählung der Schauspieler)\hspace{1em}
\newline{}Schnitzler: mit rotem Buntstift eine Unterstreichung 
\newline{}Ordnung: mit Bleistift von unbekannter Hand nummeriert:
                                 »7« }\toendnotes[C]{\smallbreak}
\pstart
           {\pb}\textcolor{gray}{\textbf{STEFAN GROSSMANN}}\hfill WIEN, I.\oindex{I., Innere Stadt@\textbf{I., Innere Stadt}, \emph{Verwaltungsgebiet}|pw}, GRABEN 29a\oindex{Wien@\textbf{Wien}!I., Innere Stadt@\textbf{I., Innere Stadt}!Graben@\textbf{Graben}, \emph{Straße}|pw}\pend
           
\pstart
           \centering{}31. III. 10\pend
           
\pstart{}Sehr verehrter Herr!\pend\vspace{0.5em}
\pstart
           Aufrichtigen Dank für Ihre gütige Erlaubnis. Der Verein\orgindex{Wiener Freie Volksbühne@Wiener Freie Volksbühne|pwv} (der langsam in eine bürgerliche Breite kommt, es
               gehören ihm heute schon 12000 Mitglieder an) bittet Sie, zu gestatten, dass wir dem
                  Verleger\pwindex{Fischer, Samuel 24.\,12.\,1859 Liptovský Mikuláš – 15.\,10.\,1934 Berlin@\textsc{Fischer, Samuel} (24.\,12.\,1859 Liptovský Mikuláš – 15.\,10.\,1934 Berlin), \emph{Verleger}|pwv} 5{\%} Tantieme zahlen. Reicher sind wir noch nicht.\pend
           
\pstart
           Ich verstehe vollkommen, dass Ihnen die Anfügung der »Frage an das Schicksal\pwindex{Schnitzler, Arthur 15.\,5.\,1862 Wien – 21.\,10.\,1931 ebd.@\textsc{Schnitzler, Arthur} (15.\,5.\,1862 Wien – 21.\,10.\,1931 ebd.), \emph{Schriftsteller, Mediziner}!Frage an das Schicksal@\strich\emph{Die Frage an das Schicksal}|pw}« nicht gefällt. Aber die Neue W\textsuperscript{r} Bühne\orgindex{Neue Wiener Bühne@Neue Wiener Bühne|pw} behauptet, für den
                  {[}»{]}Puppenspieler\pwindex{Schnitzler, Arthur 15.\,5.\,1862 Wien – 21.\,10.\,1931 ebd.@\textsc{Schnitzler, Arthur} (15.\,5.\,1862 Wien – 21.\,10.\,1931 ebd.), \emph{Schriftsteller, Mediziner}!Puppenspieler. Studie in einem Aufzuge@\strich\emph{Der Puppenspieler. Studie in einem Aufzuge}|pw}« absolut nicht die Zeit für
               nöthige Proben zu haben. So musste ich, wider besseres Wissen, {\pb}im Interesse der guten Ausarbeitung der »Literatur\pwindex{Schnitzler, Arthur 15.\,5.\,1862 Wien – 21.\,10.\,1931 ebd.@\textsc{Schnitzler, Arthur} (15.\,5.\,1862 Wien – 21.\,10.\,1931 ebd.), \emph{Schriftsteller, Mediziner}!Literatur@\strich\emph{Literatur}|pw}« und »Masken\pwindex{Schnitzler, Arthur 15.\,5.\,1862 Wien – 21.\,10.\,1931 ebd.@\textsc{Schnitzler, Arthur} (15.\,5.\,1862 Wien – 21.\,10.\,1931 ebd.), \emph{Schriftsteller, Mediziner}!letzten Masken@\strich\emph{Die letzten Masken}|pw}« einwilligen.\pend
           
\pstart
           In Literatur\pwindex{Schnitzler, Arthur 15.\,5.\,1862 Wien – 21.\,10.\,1931 ebd.@\textsc{Schnitzler, Arthur} (15.\,5.\,1862 Wien – 21.\,10.\,1931 ebd.), \emph{Schriftsteller, Mediziner}!Literatur@\strich\emph{Literatur}|pw} sind \uline{Charlé\pwindex{Charlé, Gustav 28.\,2.\,1871 Wien – 1940?@\textsc{Charlé, Gustav} (28.\,2.\,1871 Wien – 1940?), \emph{Theaterleiter, Schauspieler}|pw}}, Fr v. \uline{Linden}\pwindex{Linden, Constance von *~14.\,5.\,1877 Bukarest@\textsc{Linden, Constance von} (*~14.\,5.\,1877 Bukarest), \emph{Schauspielerin}|pw} (die ausgezeichnet wird), Hr \uline{Ziegler}\pwindex{Ziegler, Hans 9.\,5.\,1879 Karlsruhe – 25.\,12.\,1961 Wien@\textsc{Ziegler, Hans} (9.\,5.\,1879 Karlsruhe – 25.\,12.\,1961 Wien), \emph{Theaterleiter, Schauspieler, Komponist}|pw}, – in Masken\pwindex{Schnitzler, Arthur 15.\,5.\,1862 Wien – 21.\,10.\,1931 ebd.@\textsc{Schnitzler, Arthur} (15.\,5.\,1862 Wien – 21.\,10.\,1931 ebd.), \emph{Schriftsteller, Mediziner}!letzten Masken@\strich\emph{Die letzten Masken}|pw} Herr \uline{Charlé}\pwindex{Charlé, Gustav 28.\,2.\,1871 Wien – 1940?@\textsc{Charlé, Gustav} (28.\,2.\,1871 Wien – 1940?), \emph{Theaterleiter, Schauspieler}|pw}, Herr \uline{Heyse}\pwindex{Heyse, Emil 17.\,5.\,1875 Sankt Petersburg – 28.\,4.\,1949 Gmunden@\textsc{Heyse, Emil} (17.\,5.\,1875 Sankt Petersburg – 28.\,4.\,1949 Gmunden), \emph{Schauspieler}|pw} (Weihgast\pwindex{Schnitzler, Arthur 15.\,5.\,1862 Wien – 21.\,10.\,1931 ebd.@\textsc{Schnitzler, Arthur} (15.\,5.\,1862 Wien – 21.\,10.\,1931 ebd.), \emph{Schriftsteller, Mediziner}!letzten Masken@\strich\emph{Die letzten Masken}|pwv})
               beſchäftigt.\pend
           
\pstart
           Gern würde ich Sie einmal als Gaſt bei einer Aufführung des \uline{\textsc{Halben Held}\pwindex{Eulenberg, Herbert 25.\,1.\,1876 Mülheim [Köln] – 4.\,9.\,1949 Düsseldorf@\textsc{Eulenberg, Herbert} (25.\,1.\,1876 Mülheim [Köln] – 4.\,9.\,1949 Düsseldorf), \emph{Schriftsteller}!halber Held. Tragödie in fünf Aufzügen@\strich\emph{Ein halber Held. Tragödie in fünf Aufzügen}|pw}} v \textsc{H Eulenberg}\pwindex{Eulenberg, Herbert 25.\,1.\,1876 Mülheim [Köln] – 4.\,9.\,1949 Düsseldorf@\textsc{Eulenberg, Herbert} (25.\,1.\,1876 Mülheim [Köln] – 4.\,9.\,1949 Düsseldorf), \emph{Schriftsteller}|pw} begrüßen, auch deshalb, weil es eine paſſable Regiſſeurarbeit von mir iſt.
               Wollen Sie unſer Gaſt{ }ſein?\pend
           
\pstart
           Ich habe die Hoffnung, daſs Sie mich als Regiſſeur noch einmal werden brauchen
               können. – – –\pend
           
\pstart
           Mit den beſten Gefühlen{\\[\baselineskip]}\uline{aufrichtig ergeben}: \spacefill\mbox{Stefan Großmann}\pend
           \leftskip=0em{}\selectlanguage{ngerman}\endnumbering\briefempfaengerindex{Schnitzler, Arthur@\textsc{Schnitzler, Arthur}!zzzGroßmann, Stefan@\emph{von Stefan Großmann}!1910-03-311@{31. 3. 1910}|)be}\mylabel{L01920h}  \newcommand{\dateiname}{L01920}\newcommand{\titel}{Stefan Großmann an Arthur Schnitzler, 31. 3. 1910}\newcommand{\editorInnen}{Herausgegeben von Martin Anton Müller}%% latex-leseansicht-abspann.tex
%% Abspann für die Leseansicht.
%% Der Schalter \ifkorrekturansicht ist bereits durch den Vorspann gesetzt.

%% latex-abspann.tex
%% Gemeinsamer Abspann für Korrekturansicht und Leseansicht.
%% Setzt den Schalter \ifkorrekturansicht voraus (gesetzt in den
%% einbindenden Dateien latex-korrekturansicht-abspann.tex bzw.
%% latex-leseansicht-abspann.tex).
%% ---------------------------------------------------------------

\normalsize

% Das esempio-Environment wird nur in der Leseansicht benötigt
\ifkorrekturansicht\else
\newenvironment{esempio}[3]%
{
    \vspace{1.5ex}
    \rlap{\underline{#1}}
    \par
    \setlength{\parindent}{0cm}
    \nopagebreak
    \leftskip=#2cm
    \rightskip=#3cm
}
{
    \par
}
\fi

\doendnotes{C}
\bigskip
\vfill

\clearpage

\footnotesize

\ifkorrekturansicht
  \lohead{\textsc{register}}
\fi

% theindex-Environment neu definieren ohne reledmac
\makeatletter
\renewenvironment{theindex}{%
  \ifkorrekturansicht
    \section*{\indexname}%
  \else
    \subsubsection*{Index der erwähnten Entitäten}%
  \fi
  \setlength{\parindent}{0pt}%
  \setlength{\parskip}{0pt plus 0.3pt}%
  \let\item\@idxitem
}{%
  \ifkorrekturansicht\clearpage\fi
}
\makeatother

\IfFileExists{\jobname-pw.ind}{\input{\jobname-pw.ind}}{}

% Quellenangabe nur in der Leseansicht
\ifkorrekturansicht\else
% Fallback-Definitionen, falls die .tex-Datei \titel etc. nicht gesetzt hat
\providecommand{\titel}{}
\providecommand{\editorInnen}{}
\providecommand{\dateiname}{\jobname}

\vspace{3cm}

\vfill

\footnotesize
\textsc{Quelle}: \titel. Herausgegeben von {\editorInnen}. In: \emph{Arthur Schnitzler: Briefwechsel mit Autorinnen und Autoren}.
 Digitale Edition, https://schnitzler-briefe.acdh.oeaw.ac.at/{\dateiname}.html (Stand \today)
\fi

\end{document}


