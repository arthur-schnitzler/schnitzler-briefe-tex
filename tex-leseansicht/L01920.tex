\input{../tex-inputs/latex-pdf-vorspann}
\begin{center}
            \textcolor{red}{ENTWURF. ENTZIFFERUNG NOCH NICHT KORREKTURGELESEN}
                      \end{center}
            
               \section[Stefan Großmann an Arthur Schnitzler, 31. 3. 1910]{ Stefan Großmann an Arthur Schnitzler, 31. 3. 1910}\nopagebreak\mylabel{v}\rehead{ }\begin{ledgroupsized}[t]{13cm}\normalsize\beginnumbering\briefempfaengerindex{Schnitzler, Arthur@\textsc{Schnitzler, Arthur}!zzzGrossmann, Stefan@\emph{von Stefan Großmann}!1910-03-311@{31. 3. 1910}|(be} \toendnotes[C]{\smallbreak\pagebreak[2]} \Standort{CUL, Schnitzler, B 34.}
\physDesc{Brief, 1 Blatt, 2 Seiten
\newline{}Handschrift: 1) schwarze Tinte, deutsche Kurrent\hspace{1em}2) schwarze Tinte, lateinische Kurrent (\noindent{}bis einschließlich der
                           Aufzählung der Schauspieler)\hspace{1em}
\newline{}Schnitzler: mit rotem Buntstift eine Unterstreichung \newline{}Ordnung: mit Bleistift von unbekannter Hand nummeriert: »7« }\toendnotes[C]{\smallbreak}\pstart
           \noindent{}{\pb}\textcolor{gray}{\textbf{STEFAN GROSSMANN}}\hfill WIEN, I.\oindex{I., Innere Stadt@\textbf{I., Innere Stadt}|pw}, GRABEN
                        29a\oindex{Graben@\textbf{Graben}|pw}\pend
           \pstart
           \centering{}31. III. 10\pend
           \pstart{}Sehr verehrter Herr!\pend\pstart
           Aufrichtigen Dank für Ihre gütige Erlaubnis. Der Verein\orgindex{Wiener Freie Volksbuehne@Wiener Freie Volksbühne|pwv} (der langsam in eine bürgerliche Breite kommt, es
               gehören ihm heute schon 12000 Mitglieder an) bittet Sie, zu gestatten, dass wir dem
                  Verleger\pwindex{Fischer, Samuel 24.12.1859 – 15.10.1934@\textsc{Fischer, Samuel} (24.12.1859 – 15.10.1934), \emph{Verleger}|pwv} 5{\%} Tantieme zahlen. Reicher sind wir noch nicht.\pend
           \pstart
           Ich verstehe vollkommen, dass Ihnen die Anfügung der »Frage an das Schicksal\pwindex{Schnitzler, Arthur 15.05.1862 – 21.10.1931@\textsc{Schnitzler, Arthur} (15.05.1862 – 21.10.1931), \emph{Schriftsteller, Mediziner}!Frage an das Schicksal01. 05. 1890@\strich\emph{Die Frage an das Schicksal} {[}01. 05. 1890{]}|pw}« nicht gefällt. Aber die Neue W\textsuperscript{r} Bühne\orgindex{Neue Wiener Buehne@Neue Wiener Bühne|pw} behauptet, für den
                  {[}»{]}Puppenspieler\pwindex{Schnitzler, Arthur 15.05.1862 – 21.10.1931@\textsc{Schnitzler, Arthur} (15.05.1862 – 21.10.1931), \emph{Schriftsteller, Mediziner}!Puppenspieler31. 05. 1903@\strich\emph{Der Puppenspieler} {[}31. 05. 1903{]}|pw}« absolut nicht die Zeit für nöthige
               Proben zu haben. So musste ich, wider besseres Wissen, {\pb}im Interesse der guten Ausarbeitung der »Literatur\pwindex{Schnitzler, Arthur 15.05.1862 – 21.10.1931@\textsc{Schnitzler, Arthur} (15.05.1862 – 21.10.1931), \emph{Schriftsteller, Mediziner}!Literatur1901@\strich\emph{Literatur} {[}1901{]}|pw}« und »Masken\pwindex{Schnitzler, Arthur 15.05.1862 – 21.10.1931@\textsc{Schnitzler, Arthur} (15.05.1862 – 21.10.1931), \emph{Schriftsteller, Mediziner}!letzten Masken1901@\strich\emph{Die letzten Masken} {[}1901{]}|pw}« einwilligen.\pend
           \pstart
           In Literatur\pwindex{Schnitzler, Arthur 15.05.1862 – 21.10.1931@\textsc{Schnitzler, Arthur} (15.05.1862 – 21.10.1931), \emph{Schriftsteller, Mediziner}!Literatur1901@\strich\emph{Literatur} {[}1901{]}|pw} sind \uline{Charlé\pwindex{Charle, Gustav 28.02.1871 – 1940?@\textsc{Charlé, Gustav} (28.02.1871 – 1940?), \emph{Theaterleiter, Schauspieler}|pw}}, Fr v. \uline{Linden}\pwindex{Linden, Constance von *~14.05.1877@\textsc{Linden, Constance von} (*~14.05.1877), \emph{Schauspielerin}|pw} (die ausgezeichnet wird), Hr \uline{Ziegler}\pwindex{Ziegler, Hans 09.05.1879 – 25.12.1961@\textsc{Ziegler, Hans} (09.05.1879 – 25.12.1961), \emph{Theaterleiter, Schauspieler}|pw}, – in Masken\pwindex{Schnitzler, Arthur 15.05.1862 – 21.10.1931@\textsc{Schnitzler, Arthur} (15.05.1862 – 21.10.1931), \emph{Schriftsteller, Mediziner}!letzten Masken1901@\strich\emph{Die letzten Masken} {[}1901{]}|pw} Herr \uline{Charlé}\pwindex{Charle, Gustav 28.02.1871 – 1940?@\textsc{Charlé, Gustav} (28.02.1871 – 1940?), \emph{Theaterleiter, Schauspieler}|pw}, Herr \uline{Heyse}\pwindex{Heyse, Emil 1875-05-17 – 1949-04-28@\textsc{Heyse, Emil} (1875-05-17 – 1949-04-28), \emph{Schauspieler}|pw} (Weihgast\pwindex{Schnitzler, Arthur 15.05.1862 – 21.10.1931@\textsc{Schnitzler, Arthur} (15.05.1862 – 21.10.1931), \emph{Schriftsteller, Mediziner}!letzten Masken1901@\strich\emph{Die letzten Masken} {[}1901{]}|pwv}) beſchäftigt.\pend
           \pstart
           Gern würde ich Sie einmal als Gaſt bei einer Aufführung des \uline{\textsc{Halben Held}\pwindex{Eulenberg, Herbert 25.01.1876 – 04.09.1949@\textsc{Eulenberg, Herbert} (25.01.1876 – 04.09.1949), \emph{Schriftsteller}!halber Held. Tragoedie1904@\strich\emph{Ein halber Held. Tragödie} {[}1904{]}|pw}} v \textsc{H Eulenberg}\pwindex{Eulenberg, Herbert 25.01.1876 – 04.09.1949@\textsc{Eulenberg, Herbert} (25.01.1876 – 04.09.1949), \emph{Schriftsteller}|pw} begrüßen, auch deshalb, weil es eine paſſable
               Regiſſeurarbeit von mir iſt. Wollen Sie unſer Gaſt ſein?\pend
           \pstart
           Ich habe die Hoffnung, daſs Sie mich als Regiſſeur noch einmal werden brauchen
               können. – – –\pend
           \pstart
           Mit den beſten Gefühlen{\\[\baselineskip]}\uline{aufrichtig ergeben}:
                  \spacefill\mbox{Stefan Großmann}\pend
           \leftskip=0em{}\endnumbering\briefempfaengerindex{Schnitzler, Arthur@\textsc{Schnitzler, Arthur}!zzzGrossmann, Stefan@\emph{von Stefan Großmann}!1910-03-311@{31. 3. 1910}|)be}\mylabel{h}\end{ledgroupsized}  \newcommand{\dateiname}{L01920}\newcommand{\titel}{Stefan Großmann an Arthur Schnitzler, 31. 3. 1910}\newcommand{\editorInnen}{ Martin Anton Müller und Gerd-Hermann Susen}\input{../tex-inputs/latex-pdf-abspann}
      