%% latex-leseansicht-vorspann.tex
%% Vorspann für die Leseansicht.
%% Lädt die gemeinsame Datei latex-vorspann.tex mit nicht gesetztem Schalter.

\newif\ifkorrekturansicht
\korrekturansichtfalse

\input{../tex-inputs/latex-vorspann}

\begin{center}
            \textcolor{red}{ENTWURF, NICHT FERTIG KORRIGIERT}
                      \end{center}
            
         
         \newcommand{\erwaehntePersonen}{Personen: Felix Salten}
         \newcommand{\erwaehnteInstitutionen}{}
         \newcommand{\erwaehnteOrte}{Orte: Café Pucher, Hörlgasse, London, Wien}
         \newcommand{\erwaehnteWerke}{
               \section[Arthur Schnitzler an Felix Salten, 29. 5. 1897]{ Arthur Schnitzler an Felix Salten, 29. 5. 1897}\nopagebreak\mylabel{v}\rehead{ }\begin{ledgroupsized}[t]{13cm}\normalsize\beginnumbering \toendnotes[C]{\smallbreak\pagebreak[2]} \Standort{Wienbibliothek im Rathaus, ZPH 1681, 2.1.516.}
\physDesc{
\newline{}Handschrift: , deutsche Kurrent}\pstart{}{\pb}Austria \pend{}\pstart{}Mr Felix Salten\pend{}\pstart{}Wien\oindex{Wien@\textbf{Wien}|pw}\pend{}\pstart{}IX Hoerlgasse 16\oindex{Hoerlgasse@\textbf{Hörlgasse}|pw}\pend{}{\bigskip}\pstart
           \noindent{}{\pb}Lieber Freund, Ihr lieber Brief, den ich nicht mehr ſo ausführlich
               beantworten kann, als ich ſollte u möchte, iſt mir hier nachgeſchickt wirden. Es wird
               ſich ja ſehr bald in Wien\oindex{Wien@\textbf{Wien}|pw} zu allerlei Ausſprache
               Gelegenheit \introOben{}er\introOben{}geben. Wede hoffentlich Mittwoch Abd. \textsc{resp.} Do{\geminationn}erſtag in Wien\oindex{Wien@\textbf{Wien}|pw} ſein. Finde vielleicht ein Wort von Ihnen. – Jetzt eben hab
               ich mir ein Rad beſtellt – glauben Sie mir, daſs es echt engliſch ſein wird?– Ich
               möchte Pucher\oindex{Cafe Pucher@\textbf{Café Pucher}|pw} womöglich ganz aufgeben. – Auf frohes
               Wiederſehen. Herzlichſt Ihr \pend
           \pstart \spacefill\mbox{Arthur Sch}\pend{}\pstart
           \raggedleft{}London\oindex{London@\textbf{London}|pw}{ }29. 5. 97.\pend
           
         
         \endnumbering\mylabel{h}\end{ledgroupsized}\begin{anhang}\end{anhang}\newcommand{\dateiname}{L02964}\newcommand{\titel}{Arthur Schnitzler an Felix Salten, 29. 5. 1897}\newcommand{\editorInnen}{Martin Anton Müller und Laura Untner}%% latex-leseansicht-abspann.tex
%% Abspann für die Leseansicht.
%% Der Schalter \ifkorrekturansicht ist bereits durch den Vorspann gesetzt.

%% latex-abspann.tex
%% Gemeinsamer Abspann für Korrekturansicht und Leseansicht.
%% Setzt den Schalter \ifkorrekturansicht voraus (gesetzt in den
%% einbindenden Dateien latex-korrekturansicht-abspann.tex bzw.
%% latex-leseansicht-abspann.tex).
%% ---------------------------------------------------------------

\normalsize

% Das esempio-Environment wird nur in der Leseansicht benötigt
\ifkorrekturansicht\else
\newenvironment{esempio}[3]%
{
    \vspace{1.5ex}
    \rlap{\underline{#1}}
    \par
    \setlength{\parindent}{0cm}
    \nopagebreak
    \leftskip=#2cm
    \rightskip=#3cm
}
{
    \par
}
\fi

\doendnotes{C}
\bigskip
\vfill

\clearpage

\footnotesize

\ifkorrekturansicht
  \lohead{\textsc{register}}
\fi

% theindex-Environment neu definieren ohne reledmac
\makeatletter
\renewenvironment{theindex}{%
  \ifkorrekturansicht
    \section*{\indexname}%
  \else
    \subsubsection*{Index der erwähnten Entitäten}%
  \fi
  \setlength{\parindent}{0pt}%
  \setlength{\parskip}{0pt plus 0.3pt}%
  \let\item\@idxitem
}{%
  \ifkorrekturansicht\clearpage\fi
}
\makeatother

\IfFileExists{\jobname-pw.ind}{\input{\jobname-pw.ind}}{}

% Quellenangabe nur in der Leseansicht
\ifkorrekturansicht\else
% Fallback-Definitionen, falls die .tex-Datei \titel etc. nicht gesetzt hat
\providecommand{\titel}{}
\providecommand{\editorInnen}{}
\providecommand{\dateiname}{\jobname}

\vspace{3cm}

\vfill

\footnotesize
\textsc{Quelle}: \titel. Herausgegeben von {\editorInnen}. In: \emph{Arthur Schnitzler: Briefwechsel mit Autorinnen und Autoren}.
 Digitale Edition, https://schnitzler-briefe.acdh.oeaw.ac.at/{\dateiname}.html (Stand \today)
\fi

\end{document}


      