%% latex-korrekturansicht-vorspann.tex
%% Vorspann für die Korrekturansicht.
%% Lädt die gemeinsame Datei latex-vorspann.tex mit gesetztem Schalter.

\newif\ifkorrekturansicht
\korrekturansichttrue

\input{../tex-inputs/latex-vorspann}


\section[ Arthur Schnitzler an Felix Salten, 29. 5. 1897]{L02964 Arthur Schnitzler an Felix Salten, 29. 5. 1897}
\nopagebreak\mylabel{L02964v}
\rehead{ }\normalsize\beginnumbering\briefempfaengerindex{Salten, Felix@\textsc{Salten, Felix}!zzzSchnitzler, Arthur@\emph{von Arthur Schnitzler}!1897-05-292@{29. 5. 1897}|(be}
\toendnotes[C]{\smallbreak\pagebreak[2]}\Standort{Wienbibliothek im Rathaus, ZPH 1681, 2.1.516.}
\physDesc{Postkarte, 554 Zeichen
\newline{}Handschrift: 1) schwarze Tinte, deutsche Kurrent\hspace{1em}2) schwarze Tinte, lateinische Kurrent (\noindent{}Adresse)\hspace{1em}
\newline{}Versand: 1) Stempel: »\nobreak{}\oindex{Forest Hill@\textbf{Forest Hill}, \emph{Bezirk (A.BZK)}|pwk}Forest-Hill S.E., MY 29 97\nobreak{}«.   2) Stempel: »\nobreak{}\oindex{IX., Alsergrund@\textbf{IX., Alsergrund}, \emph{A.ADM3}|pwk}Wien 9/1, 1/6. 9\textcolor{gray}{7}, 8–9½ V., Bestellt\nobreak{}«. 
\newline{}Ordnung: mit Bleistift von unbekannter Hand nummeriert: »75« }\toendnotes[C]{\smallbreak}\pstart{}{\pb}Austria\pend{}\pstart{}Mr. Felix Salten\pend{}\pstart{}Wien\oindex{Wien@\textbf{Wien}, \emph{A.ADM2}|pw}\pend{}\pstart{}IX. Hoerlgasse 16\oindex{Hoerlgasse@\textbf{Hörlgasse}, \emph{Straße (K.STR)}|pw}\pend{}{\bigskip}\vspace{1em}
\pstart
           \noindent{}{\pb}Lieber Freund, Ihr lieber \label{K_L02964-1v}\edtext{Brief, den ich nicht mehr ſo ausführlich beantworten kann,
               als ich ſollte u möchte, iſt mir hieher\oindex{London@\textbf{London}, \emph{P.PPLC}|pwv}}{\lemma{\textnormal{\emph{Brief, … hieher}}}\Cendnote{\textnormal{Schnitzler war am 24. 5. 1897 von Paris\oindex{Paris@\textbf{Paris}, \emph{P.PPLC}|pwk} weiter nach London\oindex{London@\textbf{London}, \emph{P.PPLC}|pwk} gereist. Goldmann\pwindex{Goldmann, Paul 31.01.1865 – 25.09.1935@\textsc{Goldmann, Paul} (31.01.1865 – 25.09.1935), \emph{Schriftsteller/Schriftstellerin, Journalist/Journalistin}|pwk}
                  sandte ihm am 26. 5. [1897]
                  einen Brief nach, aller Wahrscheinlichkeit nach diesen: Felix Salten an Arthur Schnitzler, 23. 5. 1897.}}}\label{K_L02964-1} nachgeſchickt worden. Es wird ſich ja ſehr bald in Wien\oindex{Wien@\textbf{Wien}, \emph{A.ADM2}|pw} zu allerlei Ausſprache Gelegenheit \introOben{}er\introOben{}geben. Werde hoffentlich \label{K_L02964-2v}\edtext{Mittwoch}{\lemma{\textnormal{\emph{Mittwoch}}}\Cendnote{\textnormal{Schnitzler kehrte am Mittwoch, dem 2. 6. 1897 nach Wien\oindex{Wien@\textbf{Wien}, \emph{A.ADM2}|pwk} zurück.}}}\label{K_L02964-2}{ }Abd{ }\textsc{resp.}{ }Do{\geminationn}erſtag in Wien\oindex{Wien@\textbf{Wien}, \emph{A.ADM2}|pw} ſein. Finde vielleicht ein Wort von Ihnen.–
               Jetzt eben hab ich mir ein Rad beſtellt – glauben Sie mir, daſs es echt engliſch\oindex{England@\textbf{England}, \emph{A.ADM1}|pwv} ſein wird? – Ich möchte
                  \label{K_L02964-3v}\edtext{Pucher\oindex{Cafe Pucher@\textbf{Café Pucher}, \emph{Kaffeehaus (K.KAF)}|pw}}{\lemma{\textnormal{\emph{Pucher}}}\Cendnote{\textnormal{Die Stelle bleibt weitgehend kryptisch.
                  Naheliegend scheint vor allem diese Auflösung: Am 21. 1. 1897 hatte das
                     Café Griensteidl\oindex{Cafe Griensteidl@\textbf{Café Griensteidl}, \emph{Kaffeehaus (K.KAF)}|pwk} geschlossen, folglich
                  musste ein neues Stammkaffeehaus gefunden werden. Eventuell war dies in
                  den ersten Tagen bis zu Schnitzlers Abreise
                  das Café Pucher\oindex{Cafe Pucher@\textbf{Café Pucher}, \emph{Kaffeehaus (K.KAF)}|pwk}, vgl. Felix Salten an Arthur Schnitzler, 1. [6.] 1897.}}}\label{K_L02964-3} womöglich ganz
               aufgeben.– Auf frohes Wiederſehen. Herzlich Ihr\pend
           \pstart \spacefill\mbox{Arthur Sch}\pend{}
\pstart
           London\oindex{London@\textbf{London}, \emph{P.PPLC}|pw}{ }29. 5. 97.\pend
           \selectlanguage{ngerman}\endnumbering\briefempfaengerindex{Salten, Felix@\textsc{Salten, Felix}!zzzSchnitzler, Arthur@\emph{von Arthur Schnitzler}!1897-05-292@{29. 5. 1897}|)be}\mylabel{L02964h}  \normalsize

\doendnotes{C}
\bigskip
\vfill

\clearpage

\footnotesize

\lohead{\textsc{register}}

% Definiere theindex-Environment komplett neu ohne reledmac
\makeatletter
\renewenvironment{theindex}{%
  \section*{\indexname}%
  \setlength{\parindent}{0pt}%
  \setlength{\parskip}{0pt plus 0.3pt}%
  \let\item\@idxitem
}{%
  \clearpage
}
\makeatother

\IfFileExists{\jobname-pw.ind}{\input{\jobname-pw.ind}}{}

\end{document}

      