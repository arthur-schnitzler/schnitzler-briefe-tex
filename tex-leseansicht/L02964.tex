%% latex-leseansicht-vorspann.tex
%% Vorspann für die Leseansicht.
%% Lädt die gemeinsame Datei latex-vorspann.tex mit nicht gesetztem Schalter.

\newif\ifkorrekturansicht
\korrekturansichtfalse

\input{../tex-inputs/latex-vorspann}


         
         \renewcommand{\erwaehntePersonen}{Personen: Paul Goldmann, Felix Salten}
         \renewcommand{\erwaehnteOrte}{Orte: Café Griensteidl, Café Pucher, England, Forest Hill, Hörlgasse, IX., Alsergrund, London, Paris, Wien}
         \renewcommand{\erwaehnteWerke}{}
               \section[ Arthur Schnitzler an Felix Salten, 29. 5. 1897]{ Arthur Schnitzler an Felix Salten, 29. 5. 1897}\nopagebreak\mylabel{v}\rehead{ }\begin{ledgroupsized}[t]{13cm}\normalsize\beginnumbering\briefempfaengerindex{Salten, Felix@\textsc{Salten, Felix}!zzzSchnitzler, Arthur@\emph{von Arthur Schnitzler}!1897-05-292@{29. 5. 1897}|(be} \toendnotes[C]{\smallbreak\pagebreak[2]} \Standort{Wienbibliothek im Rathaus, ZPH 1681, 2.1.516.}
\physDesc{Postkarte, 554 Zeichen
\newline{}Handschrift: 1) schwarze Tinte, deutsche Kurrent\hspace{1em}2) schwarze Tinte, lateinische Kurrent (\noindent{}Adresse)\hspace{1em}
\newline{}Versand: 1) Stempel: »\nobreak{}\oindex{Forest Hill@\textbf{Forest Hill}|pwk}Forest-Hill S.E., MY 29 97\nobreak{}«.   2) Stempel: »\nobreak{}\oindex{IX., Alsergrund@\textbf{IX., Alsergrund}|pwk}Wien 9/1, 1/6. 9\textcolor{gray}{7}, 8–9½ V., Bestellt\nobreak{}«. 
\newline{}Ordnung: mit Bleistift von unbekannter Hand nummeriert: »75« }\toendnotes[C]{\smallbreak}\pstart{}{\pb}Austria\pend{}\pstart{}Mr. Felix Salten\pend{}\pstart{}Wien\oindex{Wien@\textbf{Wien}|pw}\pend{}\pstart{}IX. Hoerlgasse 16\oindex{Hoerlgasse@\textbf{Hörlgasse}|pw}\pend{}{\bigskip}\pstart
           \noindent{}{\pb}Lieber Freund, Ihr lieber \label{K_L02964-1v}\edtext{Brief, den ich nicht mehr ſo ausführlich beantworten kann,
               als ich ſollte u möchte, iſt mir hieher\oindex{London@\textbf{London}|pwv}}{\lemma{\textnormal{\emph{Brief, … hieher}}}\Cendnote{\textnormal{Schnitzler\pwindex{Schnitzler, Arthur 15.05.1862 – 21.10.1931@\textsc{Schnitzler, Arthur} (15.05.1862 – 21.10.1931), \emph{Schriftsteller, Mediziner}|pwk} reiste am 24. 5. 1897 von Paris\oindex{Paris@\textbf{Paris}|pwk} weiter nach London\oindex{London@\textbf{London}|pwk}. Goldmann\pwindex{Goldmann, Paul 31.01.1865 – 25.09.1935@\textsc{Goldmann, Paul} (31.01.1865 – 25.09.1935), \emph{Schriftsteller, Journalist}|pwk}
                  sandte ihm am 26. 5. [1897]
                  einen Brief nach, aller Wahrscheinlichkeit nach diesen: Felix Salten an Arthur Schnitzler, 23. 5. 1897.}}}\label{K_L02964-1h} nachgeſchickt worden. Es wird ſich ja ſehr bald in Wien\oindex{Wien@\textbf{Wien}|pw} zu allerlei Ausſprache Gelegenheit \introOben{}er\introOben{}geben. Werde hoffentlich \label{K_L02964-2v}\edtext{Mittwoch}{\lemma{\textnormal{\emph{Mittwoch}}}\Cendnote{\textnormal{Schnitzler\pwindex{Schnitzler, Arthur 15.05.1862 – 21.10.1931@\textsc{Schnitzler, Arthur} (15.05.1862 – 21.10.1931), \emph{Schriftsteller, Mediziner}|pwk} kehrte am Mittwoch, dem 2. 6. 1897, nach Wien\oindex{Wien@\textbf{Wien}|pwk} zurück.}}}\label{K_L02964-2h}{ }Abd{ }\textsc{resp.}{ }Do{\geminationn}erſtag in Wien\oindex{Wien@\textbf{Wien}|pw} ſein. Finde vielleicht ein Wort von Ihnen.–
               Jetzt eben hab ich mir ein Rad beſtellt – glauben Sie mir, daſs es echt engliſch\oindex{England@\textbf{England}|pwv} ſein wird? – Ich möchte
                  \label{K_L02964-3v}\edtext{Pucher\oindex{Cafe Pucher@\textbf{Café Pucher}|pw}}{\lemma{\textnormal{\emph{Pucher}}}\Cendnote{\textnormal{Die Stelle bleibt weitgehend kryptisch.
                  Naheliegend scheint vor allem diese Auflösung: Am 21. 1. 1897 hatte das
                     Café Griensteidl\oindex{Cafe Griensteidl@\textbf{Café Griensteidl}|pwk} geschlossen, folglich
                  musste in Folge ein neues Stammkaffeehaus gefunden werden. Eventuell war dies in
                  den ersten Tagen bis zu Schnitzler\pwindex{Schnitzler, Arthur 15.05.1862 – 21.10.1931@\textsc{Schnitzler, Arthur} (15.05.1862 – 21.10.1931), \emph{Schriftsteller, Mediziner}|pwk}s Abreise
                  das Café Pucher\oindex{Cafe Pucher@\textbf{Café Pucher}|pwk}, vgl. Felix Salten an Arthur Schnitzler, 1. [6.] 1897.}}}\label{K_L02964-3h} womöglich ganz
               aufgeben.– Auf frohes Wiederſehen. Herzlich Ihr\pend
           \pstart \spacefill\mbox{Arthur Sch}\pend{}\pstart
           London\oindex{London@\textbf{London}|pw}{ }29. 5. 97.\pend
           
         
         \endnumbering\mylabel{h}\end{ledgroupsized}  \newcommand{\dateiname}{L02964}\newcommand{\titel}{Arthur Schnitzler an Felix Salten, 29. 5. 1897}\newcommand{\editorInnen}{Martin Anton Müller und Laura Untner}%% latex-leseansicht-abspann.tex
%% Abspann für die Leseansicht.
%% Der Schalter \ifkorrekturansicht ist bereits durch den Vorspann gesetzt.

%% latex-abspann.tex
%% Gemeinsamer Abspann für Korrekturansicht und Leseansicht.
%% Setzt den Schalter \ifkorrekturansicht voraus (gesetzt in den
%% einbindenden Dateien latex-korrekturansicht-abspann.tex bzw.
%% latex-leseansicht-abspann.tex).
%% ---------------------------------------------------------------

\normalsize

% Das esempio-Environment wird nur in der Leseansicht benötigt
\ifkorrekturansicht\else
\newenvironment{esempio}[3]%
{
    \vspace{1.5ex}
    \rlap{\underline{#1}}
    \par
    \setlength{\parindent}{0cm}
    \nopagebreak
    \leftskip=#2cm
    \rightskip=#3cm
}
{
    \par
}
\fi

\doendnotes{C}
\bigskip
\vfill

\clearpage

\footnotesize

\ifkorrekturansicht
  \lohead{\textsc{register}}
\fi

% theindex-Environment neu definieren ohne reledmac
\makeatletter
\renewenvironment{theindex}{%
  \ifkorrekturansicht
    \section*{\indexname}%
  \else
    \subsubsection*{Index der erwähnten Entitäten}%
  \fi
  \setlength{\parindent}{0pt}%
  \setlength{\parskip}{0pt plus 0.3pt}%
  \let\item\@idxitem
}{%
  \ifkorrekturansicht\clearpage\fi
}
\makeatother

\IfFileExists{\jobname-pw.ind}{\input{\jobname-pw.ind}}{}

% Quellenangabe nur in der Leseansicht
\ifkorrekturansicht\else
% Fallback-Definitionen, falls die .tex-Datei \titel etc. nicht gesetzt hat
\providecommand{\titel}{}
\providecommand{\editorInnen}{}
\providecommand{\dateiname}{\jobname}

\vspace{3cm}

\vfill

\footnotesize
\textsc{Quelle}: \titel. Herausgegeben von {\editorInnen}. In: \emph{Arthur Schnitzler: Briefwechsel mit Autorinnen und Autoren}.
 Digitale Edition, https://schnitzler-briefe.acdh.oeaw.ac.at/{\dateiname}.html (Stand \today)
\fi

\end{document}


      