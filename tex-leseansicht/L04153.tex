%% latex-leseansicht-vorspann.tex
%% Vorspann für die Leseansicht.
%% Lädt die gemeinsame Datei latex-vorspann.tex mit nicht gesetztem Schalter.

\newif\ifkorrekturansicht
\korrekturansichtfalse

\input{../tex-inputs/latex-vorspann}


\section[Arthur Schnitzler an Gustav Schwarzkopf, 3. 3. 1909]{L04153 Arthur Schnitzler an Gustav Schwarzkopf, 3. 3. 1909}
\nopagebreak\mylabel{L04153v}
\rehead{ }\normalsize\beginnumbering\briefempfaengerindex{Schwarzkopf, Gustav@\textsc{Schwarzkopf, Gustav}!zzzSchnitzler, Arthur@\emph{von Arthur Schnitzler}!1909-03-031@{3. 3. 1909}|(be}
\toendnotes[C]{\smallbreak\pagebreak[2]}
\correspDesc{Versand  durch Arthur Schnitzler am 3. 3. 1909 in Wien
\newline{}Erhalt  durch Gustav Schwarzkopf im Zeitraum [3. 3. 1909 – 4. 3. 1909] in Wien}\toendnotes[C]{\smallbreak}
\Standort{CUL, Schnitzler, B 96.}
\physDesc{Brief, 1 Blatt, 1 Seite, 244 Zeichen
\newline{}Handschrift: schwarze Tinte, deutsche Kurrent}\toendnotes[C]{\smallbreak}
\pstart
           {\pb}\textcolor{gray}{\textbf{Dr. Arthur Schnitzler}}\hfill 3/3 09\pend
           
\pstart
           \textcolor{gray}{\textbf{Wien XVIII. Spoettelgasse 7\oindex{Wien@\textbf{Wien}!XVIII., Währing@\textbf{XVIII., Währing}!Edmund-Weiß-Gasse@\textbf{Edmund-Weiß-Gasse}, \emph{Straße}|pw}.}}\pend
           
\pstart{}lieber Guſtav!\pend\vspace{0.5em}
\pstart
           hoffentlich hält Sie nichts ab, von der beifolgd \label{K_L04153-1v}\edtext{Karte\eventindex{Cabaret Fledermaus@\textbf{Cabaret Fledermaus}!Generalprobe von Le devin du village, 4.3.1909@Generalprobe von Le devin du village, 4.3.1909|pwv}}{\lemma{\textnormal{\emph{Karte}}}\Cendnote{\textnormal{Beilage nicht erhalten. Es dürfte sich
                  um eine Eintrittskarte für die Generalprobe von
                     \emph{Le devin du village}\pwindex{Rousseau, Jean-Jacques 28.\,6.\,1712 Genf – 2.\,7.\,1778 Ermenonville@\textsc{Rousseau, Jean-Jacques} (28.\,6.\,1712 Genf – 2.\,7.\,1778 Ermenonville), \emph{Philosoph}!Le devin du village [Bearbeitung Robert Gound]@\strich\emph{Le devin du village [Bearbeitung Robert Gound]}|pwk} von Jean-Jacques Rousseau\pwindex{Rousseau, Jean-Jacques 28.\,6.\,1712 Genf – 2.\,7.\,1778 Ermenonville@\textsc{Rousseau, Jean-Jacques} (28.\,6.\,1712 Genf – 2.\,7.\,1778 Ermenonville), \emph{Philosoph}|pwk}\eventindex{Cabaret Fledermaus@\textbf{Cabaret Fledermaus}!Generalprobe von Le devin du village, 4.3.1909@Generalprobe von Le devin du village, 4.3.1909|pwk} (in der Bearbeitung von Robert Gund\pwindex{Gund, Robert 18.\,11.\,1865 Neuhausen – 26.\,6.\,1927 Wien@\textsc{Gund, Robert} (18.\,11.\,1865 Neuhausen – 26.\,6.\,1927 Wien), \emph{Gesangspädagoge, Pianist}|pwk})
                   am 4. 3. 1909 im Cabaret Fledermaus\oindex{Wien@\textbf{Wien}!I., Innere Stadt@\textbf{I., Innere Stadt}!Cabaret Fledermaus@\textbf{Cabaret Fledermaus}, \emph{Kabarett}|pwk} gehandelt haben. Schwarzkopf\pwindex{Schwarzkopf, Gustav 7.\,11.\,1853 Wien – 13.\,11.\,1939 ebd.@\textsc{Schwarzkopf, Gustav} (7.\,11.\,1853 Wien – 13.\,11.\,1939 ebd.), \emph{Schriftsteller}|pwk} nahm teil.}}}\label{K_L04153-1} Gebrauch zu
               machen.\pend
           
\pstart
           Auf Wiederſehen dortſelbſt.\pend
           
\pstart
           Herzliche Grüße, auch von Olga\pwindex{Schnitzler, Olga 17.\,1.\,1882 Wien – 13.\,1.\,1970 Lugano@\textsc{Schnitzler, Olga} (17.\,1.\,1882 Wien – 13.\,1.\,1970 Lugano), \emph{Schauspielerin, Sängerin}|pw}, der es, wie
               Sie ſchon aus der frei gewordenen Karte\eventindex{Cabaret Fledermaus@\textbf{Cabaret Fledermaus}!Generalprobe von Le devin du village, 4.3.1909@Generalprobe von Le devin du village, 4.3.1909|pwv} merken, noch i{\geminationm}er ſehr bettlägerig geht\pend
           
\pstart
           Ihr{\\[\baselineskip]}\spacefill\mbox{A.}\pend
           \leftskip=0em{}\selectlanguage{ngerman}\endnumbering\briefempfaengerindex{Schwarzkopf, Gustav@\textsc{Schwarzkopf, Gustav}!zzzSchnitzler, Arthur@\emph{von Arthur Schnitzler}!1909-03-031@{3. 3. 1909}|)be}\mylabel{L04153h}
\begin{anhang}
\end{anhang}\newcommand{\dateiname}{L04153}\newcommand{\titel}{Arthur Schnitzler an Gustav Schwarzkopf, 3. 3. 1909}\newcommand{\editorInnen}{Herausgegeben von Jahnke, SelmaMüller, Martin Anton}%% latex-leseansicht-abspann.tex
%% Abspann für die Leseansicht.
%% Der Schalter \ifkorrekturansicht ist bereits durch den Vorspann gesetzt.

%% latex-abspann.tex
%% Gemeinsamer Abspann für Korrekturansicht und Leseansicht.
%% Setzt den Schalter \ifkorrekturansicht voraus (gesetzt in den
%% einbindenden Dateien latex-korrekturansicht-abspann.tex bzw.
%% latex-leseansicht-abspann.tex).
%% ---------------------------------------------------------------

\normalsize

% Das esempio-Environment wird nur in der Leseansicht benötigt
\ifkorrekturansicht\else
\newenvironment{esempio}[3]%
{
    \vspace{1.5ex}
    \rlap{\underline{#1}}
    \par
    \setlength{\parindent}{0cm}
    \nopagebreak
    \leftskip=#2cm
    \rightskip=#3cm
}
{
    \par
}
\fi

\doendnotes{C}
\bigskip
\vfill

\clearpage

\footnotesize

\ifkorrekturansicht
  \lohead{\textsc{register}}
\fi

% theindex-Environment neu definieren ohne reledmac
\makeatletter
\renewenvironment{theindex}{%
  \ifkorrekturansicht
    \section*{\indexname}%
  \else
    \subsubsection*{Index der erwähnten Entitäten}%
  \fi
  \setlength{\parindent}{0pt}%
  \setlength{\parskip}{0pt plus 0.3pt}%
  \let\item\@idxitem
}{%
  \ifkorrekturansicht\clearpage\fi
}
\makeatother

\IfFileExists{\jobname-pw.ind}{\input{\jobname-pw.ind}}{}

% Quellenangabe nur in der Leseansicht
\ifkorrekturansicht\else
% Fallback-Definitionen, falls die .tex-Datei \titel etc. nicht gesetzt hat
\providecommand{\titel}{}
\providecommand{\editorInnen}{}
\providecommand{\dateiname}{\jobname}

\vspace{3cm}

\vfill

\footnotesize
\textsc{Quelle}: \titel. Herausgegeben von {\editorInnen}. In: \emph{Arthur Schnitzler: Briefwechsel mit Autorinnen und Autoren}.
 Digitale Edition, https://schnitzler-briefe.acdh.oeaw.ac.at/{\dateiname}.html (Stand \today)
\fi

\end{document}


