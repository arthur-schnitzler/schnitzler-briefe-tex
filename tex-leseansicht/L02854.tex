%% latex-leseansicht-vorspann.tex
%% Vorspann für die Leseansicht.
%% Lädt die gemeinsame Datei latex-vorspann.tex mit nicht gesetztem Schalter.

\newif\ifkorrekturansicht
\korrekturansichtfalse

\input{../tex-inputs/latex-vorspann}


         
         \renewcommand{\erwaehntePersonen}{Personen: Hermann Bahr, Richard Beer-Hofmann, Clementine Goldmann, Hugo von Hofmannsthal, Marie Reinhard, Leo Van-Jung}
         \renewcommand{\erwaehnteInstitutionen}{Institutionen: S. Fischer Verlag}
         \renewcommand{\erwaehnteOrte}{Orte: Berlin, China, Deutsches Theater Berlin, Europa, Japan, Peking, Ringstraße, Rossertstraße, Shanghai, Wien, Yantai}
         \renewcommand{\erwaehnteWerke}{Werke: Das Vermächtnis. Schauspiel in drei Akten, Der Schleier der Beatrice. Schauspiel in fünf Akten, Der Tod des Tizian, Der junge Medardus. Dramatische Historie in einem Vorspiel und fünf Aufzügen, Renaissance. Neue Studien zur Kritik der Moderne, Über moderne englische Malerei. Rückblick auf die internationale Ausstellung Wien 1894}
               \section[ Paul Goldmann an Arthur Schnitzler, 24. 8. {[}1898{]}]{ Paul Goldmann an Arthur Schnitzler, 24. 8. {[}1898{]}}\nopagebreak\mylabel{v}\rehead{ }\begin{ledgroupsized}[t]{13cm}\normalsize\beginnumbering \toendnotes[C]{\smallbreak\pagebreak[2]} \Standort{DLA, A:Schnitzler, HS.NZ85.1.3168.}
\physDesc{Brief, 2 Blätter, 7 Seiten, 2970 Zeichen
\newline{}Handschrift: blaue Tinte, deutsche Kurrent
\newline{}Schnitzler: 1) mit Bleistift das Jahr »98« vermerkt  2) mit rotem Buntstift vier Unterstreichungen}\toendnotes[C]{\smallbreak}\pstart
           \raggedleft{}{\pb}\textsc{Tschifu\oindex{Yantai@\textbf{Yantai}|pw}}, 24. Auguſt.\pend
           \pstart\center{}Mein lieber Freund,\pend\pstart
           Hier erhielt ich Deine lieben Briefe vom 28. Juni u.
               vom 10. Juli. Ich hoffe, daß Deine \label{K_L02854-1v}\edtext{Reiſe}{\lemma{\textnormal{\emph{Reiſe}}}\Cendnote{\textnormal{siehe Paul Goldmann an Arthur Schnitzler, 16. 5. 1898}}}\label{K_L02854-1h} Dir Erfriſchung und Abziehung von Deinen trüben, Deinen ſo unnöthig trüben
               Gedanken gebracht hat. Wie gern wäre ich \strikeout{\textcolor{gray}{mt}} mitgekommen, wie alljährlich! Hoffentlich können wir \label{K_L02854-2v}\edtext{nächſtes Jahr}{\lemma{\textnormal{\emph{nächſtes Jahr}}}\Cendnote{\textnormal{Sie sahen sich bereits Anfang des
                  nächsten Jahres wieder: Goldmann\pwindex{Goldmann, Paul 31.01.1865 – 25.09.1935@\textsc{Goldmann, Paul} (31.01.1865 – 25.09.1935), \emph{Schriftsteller, Journalist}|pwk} überraschte Schnitzler\pwindex{Schnitzler, Arthur 15.05.1862 – 21.10.1931@\textsc{Schnitzler, Arthur} (15.05.1862 – 21.10.1931), \emph{Schriftsteller, Mediziner}|pwk} am
                     14. 1. 1899 mit
                  einem Besuch in Wien\oindex{Wien@\textbf{Wien}|pwk}.}}}\label{K_L02854-2h} wieder zuſammen
               ſein.\pend
           \pstart
           Mit wahrer Freude habe ich aus Deinen lieben Briefen geſehen, wie reich das
               literariſche Erträgniß dieſes Jahres für Dich ſein wird.
               Wenn Dich Deine Hypochondrie {\pb}ſo arbeitſam macht, ſo
               will ich mich recht gern mit ihr \strikeout{abf\textcolor{gray}{i}nden} abfinden. Dieſer Brief erreicht Dich
               wahrſcheinlich ſchon nach der \label{K_L02854-3v}\edtext{\begin{otherlanguage}{french}\textsc{Première\pwindex{Schnitzler, Arthur 15.05.1862 – 21.10.1931@\textsc{Schnitzler, Arthur} (15.05.1862 – 21.10.1931), \emph{Schriftsteller, Mediziner}!Vermaechtnis. Schauspiel in drei Akten1898@\strich\emph{Das Vermächtnis. Schauspiel in drei Akten} {[}1898{]}|pwv}}\end{otherlanguage}{ }in Berlin\oindex{Berlin@\textbf{Berlin}|pw}}{\lemma{\textnormal{\emph{Première in Berlin}}}\Cendnote{\textnormal{Die Uraufführung von \emph{Das Vermächtnis}\pwindex{Schnitzler, Arthur 15.05.1862 – 21.10.1931@\textsc{Schnitzler, Arthur} (15.05.1862 – 21.10.1931), \emph{Schriftsteller, Mediziner}!Vermaechtnis. Schauspiel in drei Akten1898@\strich\emph{Das Vermächtnis. Schauspiel in drei Akten} {[}1898{]}|pwk} fand am 8. 10. 1898 am Deutschen Theater\oindex{Deutsches Theater Berlin@\textbf{Deutsches Theater Berlin}|pwk} in Berlin\oindex{Berlin@\textbf{Berlin}|pwk} statt und
                  war ein Erfolg.}}}\label{K_L02854-3h}, und ich bin überzeugt, daß Du \strikeout{\textcolor{gray}{×}\-\textcolor{gray}{×}\-\textcolor{gray}{×}\-\textcolor{gray}{×}{ }\textcolor{gray}{×}\-\textcolor{gray}{×}\-\textcolor{gray}{×}\-\textcolor{gray}{×}\-\textcolor{gray}{×}\-\textcolor{gray}{×}\-\textcolor{gray}{×}\-\textcolor{gray}{×}} einen neuen ſchönen Erfolg erringen wirſt, zu dem ich Dich im Voraus von
               ganzem Herzen beglückwünſche. Der Titel des Stück\pwindex{Schnitzler, Arthur 15.05.1862 – 21.10.1931@\textsc{Schnitzler, Arthur} (15.05.1862 – 21.10.1931), \emph{Schriftsteller, Mediziner}!Vermaechtnis. Schauspiel in drei Akten1898@\strich\emph{Das Vermächtnis. Schauspiel in drei Akten} {[}1898{]}|pwv}es iſt vielverſprechend. Aber was ſteht darin? \strikeout{Sob} Sobald Du nur irgend kannſt, ſendeſt Du mir ein Exemplar\pwindex{Schnitzler, Arthur 15.05.1862 – 21.10.1931@\textsc{Schnitzler, Arthur} (15.05.1862 – 21.10.1931), \emph{Schriftsteller, Mediziner}!Vermaechtnis. Schauspiel in drei Akten1898@\strich\emph{Das Vermächtnis. Schauspiel in drei Akten} {[}1898{]}|pwv}, nicht wahr? Deine
               Idee, ein \label{K_L02854-12v}\edtext{Renaiſſance-Stück\pwindex{Schnitzler, Arthur 15.05.1862 – 21.10.1931@\textsc{Schnitzler, Arthur} (15.05.1862 – 21.10.1931), \emph{Schriftsteller, Mediziner}!Schleier der Beatrice. Schauspiel in fuenf Akten1900-12-01@\strich\emph{Der Schleier der Beatrice. Schauspiel in fünf Akten} {[}1900-12-01{]}|pwv}}{\lemma{\textnormal{\emph{Renaiſſance-Stück}}}\Cendnote{\textnormal{siehe A. S.: \emph{Tagebuch}, 5. 7. 1898}}}\label{K_L02854-12h} zu {\pb}ſchreiben, gefällt mir weniger. Mir
               kommt \substVorne{}\textsuperscript{vo\textcolor{gray}{r},}\substDazwischen{}vor,\substHinten{} als würde Dir das nicht liegen, und ſeit die \textsc{Renaissance} von den \textsc{Bahr\pwindex{Bahr, Hermann 19.07.1863 – 15.01.1934@\textsc{Bahr, Hermann} (19.07.1863 – 15.01.1934), \emph{Schriftsteller, Kritiker}|pw}} und \textsc{Hoffmannsthal\pwindex{Hofmannsthal, Hugo von 1874-02-01 – 1929-07-15@\textsc{Hofmannsthal, Hugo von} (1874-02-01 – 1929-07-15), \emph{Schriftsteller}|pw}} zum \label{K_L02854-4v}\edtext{Dogma}{\lemma{\textnormal{\emph{Dogma}}}\Cendnote{\textnormal{Hermann Bahr\pwindex{Bahr, Hermann 19.07.1863 – 15.01.1934@\textsc{Bahr, Hermann} (19.07.1863 – 15.01.1934), \emph{Schriftsteller, Kritiker}|pwk} hatte seine jüngste Sammlung
                  von Kritiken \emph{Renaissance. Neue Studien zur Kritik
                     der Moderne}\pwindex{Bahr, Hermann 19.07.1863 – 15.01.1934@\textsc{Bahr, Hermann} (19.07.1863 – 15.01.1934), \emph{Schriftsteller, Kritiker}!Renaissance. Neue Studien zur Kritik der Moderne1897@\strich\emph{Renaissance. Neue Studien zur Kritik der Moderne} {[}1897{]}|pwk} (Berlin\oindex{Berlin@\textbf{Berlin}|pwk}: \emph{S.
                        Fischer}\orgindex{S. Fischer Verlag@S. Fischer Verlag|pwk}{ }1897) betitelt. Hofmannsthal\pwindex{Hofmannsthal, Hugo von 1874-02-01 – 1929-07-15@\textsc{Hofmannsthal, Hugo von} (1874-02-01 – 1929-07-15), \emph{Schriftsteller}|pwk} hatte in
                  seinem Essay \emph{Über moderne englische Malerei.
                     Rückblick auf die internationale Ausstellung Wien 1894}\pwindex{Ueber moderne englische Malerei. Rueckblick auf die internationale Ausstellung
                  Wien 18941894-06-13@\emph{Über moderne englische Malerei. Rückblick auf die internationale Ausstellung Wien 1894} {[}1894-06-13{]}|pwk} und in seinem
                  Dramenfragment \emph{Der Tod des Tizian}\pwindex{Hofmannsthal, Hugo von 1874-02-01 – 1929-07-15@\textsc{Hofmannsthal, Hugo von} (1874-02-01 – 1929-07-15), \emph{Schriftsteller}!Tod des TizianOktober 1892@\strich\emph{Der Tod des Tizian} {[}Oktober 1892{]}|pwk} Interesse
                  an der Renaissance kundgetan.}}}\label{K_L02854-4h} erhoben worden iſt, iſt ſie mir verleidet.
               Wenn Dich die \strikeout{alt\textcolor{gray}{e}}{ }\strikeout{alt\textcolor{gray}{en}} alten Zeiten locken, was ich begreife, ſo ſchreibe Du ein \label{K_L02854-112v}\edtext{Alt-Wien\oindex{Wien@\textbf{Wien}|pw}er-Stück}{\lemma{\textnormal{\emph{Alt-Wiener-Stück}}}\Cendnote{\textnormal{Gemeint war das
                     Wien\oindex{Wien@\textbf{Wien}|pwk} vor der Stadterneuerung durch die Ringstraße\oindex{Ringstrasse@\textbf{Ringstraße}|pwk}nbauten. Am ehesten kann \emph{Der junge Medardus}\pwindex{Schnitzler, Arthur 15.05.1862 – 21.10.1931@\textsc{Schnitzler, Arthur} (15.05.1862 – 21.10.1931), \emph{Schriftsteller, Mediziner}!junge Medardus. Dramatische Historie in einem Vorspiel und fuenf
                  Aufzuegen1910-10-26@\strich\emph{Der junge Medardus. Dramatische Historie in einem Vorspiel und fünf Aufzügen} {[}1910-10-26{]}|pwk} (1910) als
                     Alt-Wien\oindex{Wien@\textbf{Wien}|pwk}er Stück gelten.}}}\label{K_L02854-112h}. Ich meine,
               Du könnteſt da etwas Entzückendes machen. Folge mir und laſſe Dich von den Zünftlern
               nicht aus Deinem Leben und Deiner Wärme ins »Literariſche« hineinlocken!\pend
           \pstart
           {\pb}Wann ich zurück komme? Ich habe keine Ahnung. Wenn
               ich im ſelben Tempo fortarbeite, kann der nächſte Sommer herankommen. Denn ich
               arbeite qualvoll ſchwer, da ich es ſo gern vermeiden möchte, Banalitäten zu ſagen,
               und ſitze über einem Feuilleton manchmal 14 Tage. Freilich beginne ich die Geſchichte
               ſatt zu bekommen, – die ewige Feuilleton-Schmiererei ebenſo wie den Miſthaufen China\oindex{China@\textbf{China}|pw}; und da \strikeout{i\textcolor{gray}{c}h} auch meine Familie auf Abkürzung meiner Reiſe {\pb}dringt, ſo könnte es geſchehen, daß ich nach \textsc{Peking\oindex{Peking@\textbf{Peking}|pw}} einfach kurz abbreche und heimkehre, ohne Japan\oindex{Japan@\textbf{Japan}|pw} geſehen zu haben. Das wäre ein ſchweres Opfer, aber es iſt nicht
               unmöglich, daß ich es bringen muß. In dieſem Falle wäre ich etwa im Februar wieder in Europa\oindex{Europa@\textbf{Europa}|pw}. Jedenfalls bitte ich Dich, mir nur noch bis \uline{Ende Oktober} nach \textsc{Shanghai\oindex{Shanghai@\textbf{Shanghai}|pw}} zu ſchreiben. Was \uline{bis zum 20. Oktober} von \textsc{Wien\oindex{Wien@\textbf{Wien}|pw}} abgeht, erreicht mich ſicher noch in China\oindex{China@\textbf{China}|pw}. {\pb}\substVorne{}\textsuperscript{v}\substDazwischen{}V\substHinten{}on da ab bitte ich Dich, alle Deine \strikeout{lieben}
               lieben Briefe meiner Mutter\pwindex{Goldmann, Clementine 1842-05-15 – 1924-02-24@\textsc{Goldmann, Clementine} (1842-05-15 – 1924-02-24)|pwv}
               zu ſenden (\textsc{Frankfurt am Main, \strikeout{Rossert} Rossertstraſse 15\oindex{Rossertstrasse@\textbf{Rossertstraße}|pw}}), welche \strikeout{all\textcolor{gray}{e}s} immer meine
               Adreſſe kennen und mir Alles nachſenden wird.\pend
           \pstart
           Willſt Du glauben, daß \textsc{Richard\pwindex{Beer-Hofmann, Richard 1866-07-11 – 1945-09-26@\textsc{Beer-Hofmann, Richard} (1866-07-11 – 1945-09-26), \emph{Schriftsteller}|pw}} mir mit keiner Sylbe ſeine \label{K_L02854-7v}\edtext{Verheirathung}{\lemma{\textnormal{\emph{Verheirathung}}}\Cendnote{\textnormal{siehe Paul Goldmann an Arthur Schnitzler, 26. 6. [1898]}}}\label{K_L02854-7h} angezeigt hat? Es gibt Fälle, wo man ſchreiben muß, ſelbſt wenn man niemals
               ſchreibt. Und mich kränkt {\pb}beſonders der Gedanke,
               daß er weder Dich noch den jungen Herrn \textsc{von Hoffmannsthal\pwindex{Hofmannsthal, Hugo von 1874-02-01 – 1929-07-15@\textsc{Hofmannsthal, Hugo von} (1874-02-01 – 1929-07-15), \emph{Schriftsteller}|pw}} in dieſer Weiſe vernachläſſigt haben würde. \label{K_L02854-11v}\edtext{\begin{otherlanguage}{french}\textsc{Avec moi, on en prend à son aise!}\end{otherlanguage}}{\lemma{\textnormal{\emph{Avec … aise!}}}\Cendnote{\textnormal{französisch: Mit mir muss man es nicht
                  so genau nehmen!}}}\label{K_L02854-11h}\pend
           \pstart
           Das iſt aber nur zwiſchen Dir und mir geſagt, und Du ſollſt ihm, wie \textsc{Leo\pwindex{Van-Jung, Leo 15.10.1866 – 02.07.1939@\textsc{Van-Jung, Leo} (15.10.1866 – 02.07.1939), \emph{Gesangspädagoge, Mathematiker}|pw}} die herzlichſten Grüße von mir übermitteln.\pend
           \pstart
           Auch Dir, mein lieber Freund, herzlichſte und treueſte Grüße!\pend
           \pstart
           Dein{\\[\baselineskip]}\spacefill\mbox{\strikeout{Pa\textcolor{gray}{u}l} Paul Goldmann}\pend
           \leftskip=0em{}\pstart
           \noindent{}Viele Grüße an Deine Freundin\pwindex{Reinhard, Marie 1871-03-13 – 1899-03-18@\textsc{Reinhard, Marie} (1871-03-13 – 1899-03-18), \emph{Gesangspädagogin}|pwv}!\pend
           
         
         \endnumbering\mylabel{h}\end{ledgroupsized}  \newcommand{\dateiname}{L02854}\newcommand{\titel}{Paul Goldmann an Arthur Schnitzler, 24. 8. [1898]}\newcommand{\editorInnen}{Martin Anton Müller und Laura Untner}%% latex-leseansicht-abspann.tex
%% Abspann für die Leseansicht.
%% Der Schalter \ifkorrekturansicht ist bereits durch den Vorspann gesetzt.

%% latex-abspann.tex
%% Gemeinsamer Abspann für Korrekturansicht und Leseansicht.
%% Setzt den Schalter \ifkorrekturansicht voraus (gesetzt in den
%% einbindenden Dateien latex-korrekturansicht-abspann.tex bzw.
%% latex-leseansicht-abspann.tex).
%% ---------------------------------------------------------------

\normalsize

% Das esempio-Environment wird nur in der Leseansicht benötigt
\ifkorrekturansicht\else
\newenvironment{esempio}[3]%
{
    \vspace{1.5ex}
    \rlap{\underline{#1}}
    \par
    \setlength{\parindent}{0cm}
    \nopagebreak
    \leftskip=#2cm
    \rightskip=#3cm
}
{
    \par
}
\fi

\doendnotes{C}
\bigskip
\vfill

\clearpage

\footnotesize

\ifkorrekturansicht
  \lohead{\textsc{register}}
\fi

% theindex-Environment neu definieren ohne reledmac
\makeatletter
\renewenvironment{theindex}{%
  \ifkorrekturansicht
    \section*{\indexname}%
  \else
    \subsubsection*{Index der erwähnten Entitäten}%
  \fi
  \setlength{\parindent}{0pt}%
  \setlength{\parskip}{0pt plus 0.3pt}%
  \let\item\@idxitem
}{%
  \ifkorrekturansicht\clearpage\fi
}
\makeatother

\IfFileExists{\jobname-pw.ind}{\input{\jobname-pw.ind}}{}

% Quellenangabe nur in der Leseansicht
\ifkorrekturansicht\else
% Fallback-Definitionen, falls die .tex-Datei \titel etc. nicht gesetzt hat
\providecommand{\titel}{}
\providecommand{\editorInnen}{}
\providecommand{\dateiname}{\jobname}

\vspace{3cm}

\vfill

\footnotesize
\textsc{Quelle}: \titel. Herausgegeben von {\editorInnen}. In: \emph{Arthur Schnitzler: Briefwechsel mit Autorinnen und Autoren}.
 Digitale Edition, https://schnitzler-briefe.acdh.oeaw.ac.at/{\dateiname}.html (Stand \today)
\fi

\end{document}


      