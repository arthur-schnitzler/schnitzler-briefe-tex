%% latex-leseansicht-vorspann.tex
%% Vorspann für die Leseansicht.
%% Lädt die gemeinsame Datei latex-vorspann.tex mit nicht gesetztem Schalter.

\newif\ifkorrekturansicht
\korrekturansichtfalse

\input{../tex-inputs/latex-vorspann}


\section[ Paul Goldmann an Arthur Schnitzler, 24. 8. {[}1898{]}]{L02854 Paul Goldmann an Arthur Schnitzler,  24. 8. [1898]}
\nopagebreak\mylabel{L02854v}
\rehead{ }\normalsize\beginnumbering\briefempfaengerindex{Schnitzler, Arthur@\textsc{Schnitzler, Arthur}!zzzGoldmann, Paul@\emph{von Paul Goldmann}!1898-08-243@{24. 8. [1898]}|(be}
\toendnotes[C]{\smallbreak\pagebreak[2]}
\correspDesc{Versand  durch Paul Goldmann am 24. 8. [1898] in Yantai
\newline{}Erhalt  durch Arthur Schnitzler im Zeitraum [15. 9. 1898
                  – 30. 10. 1898?] in Wien}\toendnotes[C]{\smallbreak}
\Standort{DLA, A:Schnitzler, HS.NZ85.1.3168.}
\physDesc{Brief, 2 Blätter, 7 Seiten, 2970 Zeichen
\newline{}Handschrift: blaue Tinte, deutsche Kurrent
\newline{}Schnitzler: 1) mit Bleistift das Jahr »98« vermerkt  2) mit rotem Buntstift vier Unterstreichungen}\toendnotes[C]{\smallbreak}
\pstart
           \raggedleft{}{\pb}\textsc{Tschifu\oindex{Yantai@\textbf{Yantai}|pw}}, 24. Auguſt.\pend
           
\pstart\center{}Mein lieber Freund,\pend\vspace{0.5em}
\pstart
           Hier erhielt ich Deine lieben Briefe vom 28. Juni u.
               vom 10. Juli. Ich hoffe, daß Deine \label{K_L02854-1v}\edtext{Reiſe}{\lemma{\textnormal{\emph{Reise}}}\Cendnote{\textnormal{Siehe XXXX Auszeichnungsfehler: Dokument L02845 nicht gefunden.
               }}}\label{K_L02854-1} Dir Erfriſchung und Abziehung von Deinen trüben, Deinen{ }ſo unnöthig trüben
               Gedanken gebracht hat. Wie gern wäre ich \strikeout{\textcolor{gray}{mt}} mitgekommen, wie alljährlich! Hoffentlich können wir \label{K_L02854-2v}\edtext{nächſtes Jahr}{\lemma{\textnormal{\emph{nächstes Jahr}}}\Cendnote{\textnormal{Sie sahen sich bereits Anfang des
                  nächsten Jahres wieder: Goldmann\pwindex{Goldmann, Paul 31.\,1.\,1865 Breslau – 25.\,9.\,1935 Wien@\textsc{Goldmann, Paul} (31.\,1.\,1865 Breslau – 25.\,9.\,1935 Wien), \emph{Schriftsteller, Journalist}|pwk} überraschte Schnitzler am
                     14. 1. 1899 mit
                  einem Besuch in Wien\oindex{Wien@\textbf{Wien}, \emph{Verwaltungsgebiet}|pwk}.}}}\label{K_L02854-2} wieder zuſammen{ }ſein.\pend
           
\pstart
           Mit wahrer Freude habe ich aus Deinen lieben Briefen geſehen, wie reich das
               literariſche Erträgniß dieſes Jahres für Dich{ }ſein wird.
               Wenn Dich Deine Hypochondrie {\pb}ſo arbeitſam macht,{ }ſo
               will ich mich recht gern mit ihr \strikeout{abf\textcolor{gray}{i}nden} abfinden. Dieſer Brief erreicht Dich
               wahrſcheinlich{ }ſchon nach der \label{K_L02854-3v}\edtext{\begin{otherlanguage}{french}\textsc{Première\pwindex{Schnitzler, Arthur 15.\,5.\,1862 Wien – 21.\,10.\,1931 ebd.@\textsc{Schnitzler, Arthur} (15.\,5.\,1862 Wien – 21.\,10.\,1931 ebd.), \emph{Schriftsteller, Mediziner}!Vermächtnis. Schauspiel in drei Akten@\strich\emph{Das Vermächtnis. Schauspiel in drei Akten}|pwv}\eventindex{Deutsches Theater Berlin@\textbf{Deutsches Theater Berlin}!Uraufführung von Das Vermächtnis, 8.10.1898@Uraufführung von Das Vermächtnis, 8.10.1898|pwv}}\end{otherlanguage}{ }in Berlin\oindex{Berlin@\textbf{Berlin}, \emph{Hauptstadt}|pw}}{\lemma{\textnormal{\emph{Première in Berlin}}}\Cendnote{\textnormal{Die Uraufführung von \emph{Das Vermächtnis}\pwindex{Schnitzler, Arthur 15.\,5.\,1862 Wien – 21.\,10.\,1931 ebd.@\textsc{Schnitzler, Arthur} (15.\,5.\,1862 Wien – 21.\,10.\,1931 ebd.), \emph{Schriftsteller, Mediziner}!Vermächtnis. Schauspiel in drei Akten@\strich\emph{Das Vermächtnis. Schauspiel in drei Akten}|pwk}\eventindex{Deutsches Theater Berlin@\textbf{Deutsches Theater Berlin}!Uraufführung von Das Vermächtnis, 8.10.1898@Uraufführung von Das Vermächtnis, 8.10.1898|pwk} fand am 8. 10. 1898 am \emph{Deutschen Theater}\orgindex{Deutsches Theater Berlin@Deutsches Theater Berlin|pwk} in Berlin\oindex{Berlin@\textbf{Berlin}, \emph{Hauptstadt}|pwk} statt und
                  war ein Erfolg.}}}\label{K_L02854-3}, und ich bin überzeugt, daß Du \strikeout{\textcolor{gray}{×}\-\textcolor{gray}{×}\-\textcolor{gray}{×}\-\textcolor{gray}{×}{ }\textcolor{gray}{×}\-\textcolor{gray}{×}\-\textcolor{gray}{×}\-\textcolor{gray}{×}\-\textcolor{gray}{×}\-\textcolor{gray}{×}\-\textcolor{gray}{×}\-\textcolor{gray}{×}} einen neuen{ }ſchönen Erfolg erringen wirſt, zu dem ich Dich im Voraus von
               ganzem Herzen beglückwünſche. Der Titel des Stück\pwindex{Schnitzler, Arthur 15.\,5.\,1862 Wien – 21.\,10.\,1931 ebd.@\textsc{Schnitzler, Arthur} (15.\,5.\,1862 Wien – 21.\,10.\,1931 ebd.), \emph{Schriftsteller, Mediziner}!Vermächtnis. Schauspiel in drei Akten@\strich\emph{Das Vermächtnis. Schauspiel in drei Akten}|pwv}es iſt vielverſprechend. Aber was{ }ſteht darin? \strikeout{Sob} Sobald Du nur irgend kannſt,{ }ſendeſt Du mir ein Exemplar\pwindex{Schnitzler, Arthur 15.\,5.\,1862 Wien – 21.\,10.\,1931 ebd.@\textsc{Schnitzler, Arthur} (15.\,5.\,1862 Wien – 21.\,10.\,1931 ebd.), \emph{Schriftsteller, Mediziner}!Vermächtnis. Schauspiel in drei Akten@\strich\emph{Das Vermächtnis. Schauspiel in drei Akten}|pwv}, nicht wahr? Deine
               Idee, ein \label{K_L02854-4v}\edtext{Renaiſſance-Stück\pwindex{Schnitzler, Arthur 15.\,5.\,1862 Wien – 21.\,10.\,1931 ebd.@\textsc{Schnitzler, Arthur} (15.\,5.\,1862 Wien – 21.\,10.\,1931 ebd.), \emph{Schriftsteller, Mediziner}!Schleier der Beatrice. Schauspiel in fünf Akten@\strich\emph{Der Schleier der Beatrice. Schauspiel in fünf Akten}|pwv}}{\lemma{\textnormal{\emph{Renaissance-Stück}}}\Cendnote{\textnormal{Siehe A. S.: \emph{Tagebuch}, 5. 7. 1898.
               }}}\label{K_L02854-4} zu {\pb}ſchreiben, gefällt mir weniger. Mir
               kommt \substVorne{}\textsuperscript{vo\textcolor{gray}{r},}\substDazwischen{}vor,\substHinten{} als würde Dir das nicht liegen, und{ }ſeit die \textsc{Renaissance} von den \textsc{Bahr\pwindex{Bahr, Hermann 19.\,7.\,1863 Linz – 15.\,1.\,1934 München@\textsc{Bahr, Hermann} (19.\,7.\,1863 Linz – 15.\,1.\,1934 München), \emph{Schriftsteller, Kritiker}|pw}} und \textsc{Hoffmannsthal\pwindex{Hofmannsthal, Hugo von 1.\,2.\,1874 Wien – 15.\,7.\,1929 Rodaun@\textsc{Hofmannsthal, Hugo von} (1.\,2.\,1874 Wien – 15.\,7.\,1929 Rodaun), \emph{Schriftsteller}|pw}} zum \label{K_L02854-5v}\edtext{Dogma}{\lemma{\textnormal{\emph{Dogma}}}\Cendnote{\textnormal{Hermann Bahr\pwindex{Bahr, Hermann 19.\,7.\,1863 Linz – 15.\,1.\,1934 München@\textsc{Bahr, Hermann} (19.\,7.\,1863 Linz – 15.\,1.\,1934 München), \emph{Schriftsteller, Kritiker}|pwk} hatte seine jüngste Sammlung
                  von Kritiken \emph{Renaissance. Neue Studien zur Kritik
                     der Moderne}\pwindex{Bahr, Hermann 19.\,7.\,1863 Linz – 15.\,1.\,1934 München@\textsc{Bahr, Hermann} (19.\,7.\,1863 Linz – 15.\,1.\,1934 München), \emph{Schriftsteller, Kritiker}!Renaissance. Neue Studien zur Kritik der Moderne@\strich\emph{Renaissance. Neue Studien zur Kritik der Moderne}|pwk} (Berlin\oindex{Berlin@\textbf{Berlin}, \emph{Hauptstadt}|pwk}: \emph{S.                         Fischer}\orgindex{S. Fischer Verlag@S. Fischer Verlag|pwk}{ }1897) betitelt. Hofmannsthal\pwindex{Hofmannsthal, Hugo von 1.\,2.\,1874 Wien – 15.\,7.\,1929 Rodaun@\textsc{Hofmannsthal, Hugo von} (1.\,2.\,1874 Wien – 15.\,7.\,1929 Rodaun), \emph{Schriftsteller}|pwk} hatte in
                  seinem Essay \emph{Über moderne englische Malerei.
                     Rückblick auf die internationale Ausstellung Wien 1894}\pwindex{Hofmannsthal, Hugo von 1.\,2.\,1874 Wien – 15.\,7.\,1929 Rodaun@\textsc{Hofmannsthal, Hugo von} (1.\,2.\,1874 Wien – 15.\,7.\,1929 Rodaun), \emph{Schriftsteller}!Über moderne englische Malerei. Rückblick auf die internationale Ausstellung Wien 1894@\strich\emph{Über moderne englische Malerei. Rückblick auf die internationale Ausstellung Wien 1894}|pwk} und in seinem
                  Dramenfragment \emph{Der Tod des Tizian}\pwindex{Hofmannsthal, Hugo von 1.\,2.\,1874 Wien – 15.\,7.\,1929 Rodaun@\textsc{Hofmannsthal, Hugo von} (1.\,2.\,1874 Wien – 15.\,7.\,1929 Rodaun), \emph{Schriftsteller}!Tod des Tizian. Ein Bruchstück@\strich\emph{Der Tod des Tizian. Ein Bruchstück}|pwk} Interesse
                  an der Renaissance kundgetan.}}}\label{K_L02854-5} erhoben worden iſt, iſt{ }ſie mir verleidet.
               Wenn Dich die \strikeout{alt\textcolor{gray}{e}}{ }\strikeout{alt\textcolor{gray}{en}} alten Zeiten locken, was ich begreife,{ }ſo{ }ſchreibe Du ein \label{K_L02854-6v}\edtext{Alt-Wien\oindex{Wien@\textbf{Wien}, \emph{Verwaltungsgebiet}|pw}er-Stück}{\lemma{\textnormal{\emph{Alt-Wiener-Stück}}}\Cendnote{\textnormal{Gemeint war das
                     Wien\oindex{Wien@\textbf{Wien}, \emph{Verwaltungsgebiet}|pwk} vor der Stadterneuerung durch die Ringstraße\oindex{Wien@\textbf{Wien}!I., Innere Stadt@\textbf{I., Innere Stadt}!Ringstraße@\textbf{Ringstraße}, \emph{Straße}|pwk}nbauten. Am ehesten kann \emph{Der junge Medardus}\pwindex{Schnitzler, Arthur 15.\,5.\,1862 Wien – 21.\,10.\,1931 ebd.@\textsc{Schnitzler, Arthur} (15.\,5.\,1862 Wien – 21.\,10.\,1931 ebd.), \emph{Schriftsteller, Mediziner}!junge Medardus. Dramatische Historie in einem Vorspiel und fünf Aufzügen@\strich\emph{Der junge Medardus. Dramatische Historie in einem Vorspiel und fünf Aufzügen}|pwk} (1910) als
                     Alt-Wien\oindex{Wien@\textbf{Wien}, \emph{Verwaltungsgebiet}|pwk}er Stück gelten.}}}\label{K_L02854-6}. Ich meine,
               Du könnteſt da etwas Entzückendes machen. Folge mir und laſſe Dich von den Zünftlern
               nicht aus Deinem Leben und Deiner Wärme ins »Literariſche« hineinlocken!\pend
           
\pstart
           {\pb}Wann ich zurück komme? Ich habe keine Ahnung. Wenn
               ich im{ }ſelben Tempo fortarbeite, kann der nächſte Sommer herankommen. Denn ich
               arbeite qualvoll{ }ſchwer, da ich es{ }ſo gern vermeiden möchte, Banalitäten zu{ }ſagen,
               und{ }ſitze über einem Feuilleton manchmal 14 Tage. Freilich beginne ich die Geſchichte{ }ſatt zu bekommen, – die ewige Feuilleton-Schmiererei ebenſo wie den Miſthaufen China\oindex{China@\textbf{China}|pw}; und da \strikeout{i\textcolor{gray}{c}h} auch meine Familie auf Abkürzung meiner Reiſe {\pb}dringt,{ }ſo könnte es geſchehen, daß ich nach \textsc{Peking\oindex{Peking@\textbf{Peking}, \emph{Hauptstadt}|pw}} einfach kurz abbreche und heimkehre, ohne Japan\oindex{Japan@\textbf{Japan}|pw} geſehen zu haben. Das wäre ein{ }ſchweres Opfer, aber es iſt nicht
               unmöglich, daß ich es bringen muß. In dieſem Falle wäre ich etwa im Februar wieder in Europa\oindex{Europa@\textbf{Europa}|pw}. Jedenfalls bitte ich Dich, mir nur noch bis \uline{Ende Oktober} nach \textsc{Shanghai\oindex{Shanghai@\textbf{Shanghai}|pw}} zu{ }ſchreiben. Was \uline{bis zum 20. Oktober} von \textsc{Wien\oindex{Wien@\textbf{Wien}, \emph{Verwaltungsgebiet}|pw}} abgeht, erreicht mich{ }ſicher noch in China\oindex{China@\textbf{China}|pw}. {\pb}\substVorne{}\textsuperscript{v}\substDazwischen{}V\substHinten{}on da ab bitte ich Dich, alle Deine \strikeout{lieben}
               lieben Briefe meiner Mutter\pwindex{Goldmann, Clementine 15.\,5.\,1842 Breslau – 24.\,2.\,1924 Frankfurt am Main@\textsc{Goldmann, Clementine} (15.\,5.\,1842 Breslau – 24.\,2.\,1924 Frankfurt am Main)|pwv}
               zu{ }ſenden (\textsc{Frankfurt am Main, \strikeout{Rossert} Rossertstraſse 15\oindex{Rossertstraße@\textbf{Rossertstraße}, \emph{Straße}|pw}}), welche \strikeout{all\textcolor{gray}{e}s} immer meine
               Adreſſe kennen und mir Alles nachſenden wird.\pend
           
\pstart
           Willſt Du glauben, daß \textsc{Richard\pwindex{Beer-Hofmann, Richard 11.\,7.\,1866 Wien – 26.\,9.\,1945 New York City@\textsc{Beer-Hofmann, Richard} (11.\,7.\,1866 Wien – 26.\,9.\,1945 New York City), \emph{Schriftsteller}|pw}} mir mit keiner Sylbe{ }ſeine \label{K_L02854-7v}\edtext{Verheirathung}{\lemma{\textnormal{\emph{Verheirathung}}}\Cendnote{\textnormal{Siehe XXXX Auszeichnungsfehler: Dokument L02848 nicht gefunden.
               }}}\label{K_L02854-7} angezeigt hat? Es gibt Fälle, wo man{ }ſchreiben muß,{ }ſelbſt wenn man niemals{ }ſchreibt. Und mich kränkt {\pb}beſonders der Gedanke,
               daß er weder Dich noch den jungen Herrn \textsc{von Hoffmannsthal\pwindex{Hofmannsthal, Hugo von 1.\,2.\,1874 Wien – 15.\,7.\,1929 Rodaun@\textsc{Hofmannsthal, Hugo von} (1.\,2.\,1874 Wien – 15.\,7.\,1929 Rodaun), \emph{Schriftsteller}|pw}} in dieſer Weiſe vernachläſſigt haben würde. \label{K_L02854-8v}\edtext{\begin{otherlanguage}{french}\textsc{Avec moi, on en prend à son aise!}\end{otherlanguage}}{\lemma{\textnormal{\emph{Avec … aise!}}}\Cendnote{\textnormal{französisch: Mit mir muss man es nicht
                  so genau nehmen!}}}\label{K_L02854-8}\pend
           
\pstart
           Das iſt aber nur zwiſchen Dir und mir geſagt, und Du{ }ſollſt ihm, wie \textsc{Leo\pwindex{Van-Jung, Leo 15.\,10.\,1866 Odessa – 2.\,7.\,1939 Riga@\textsc{Van-Jung, Leo} (15.\,10.\,1866 Odessa – 2.\,7.\,1939 Riga), \emph{Gesangspädagoge, Mathematiker}|pw}} die herzlichſten Grüße von mir übermitteln.\pend
           
\pstart
           Auch Dir, mein lieber Freund, herzlichſte und treueſte Grüße!\pend
           
\pstart
           Dein{\\[\baselineskip]}\spacefill\mbox{\strikeout{Pa\textcolor{gray}{u}l} Paul Goldmann}\pend
           \leftskip=0em{}
\pstart
           \noindent{}Viele Grüße an Deine Freundin\pwindex{Reinhard, Marie 13.\,3.\,1871 Wien – 18.\,3.\,1899 ebd.@\textsc{Reinhard, Marie} (13.\,3.\,1871 Wien – 18.\,3.\,1899 ebd.), \emph{Gesangspädagogin}|pwv}!\pend
           \selectlanguage{ngerman}\endnumbering\briefempfaengerindex{Schnitzler, Arthur@\textsc{Schnitzler, Arthur}!zzzGoldmann, Paul@\emph{von Paul Goldmann}!1898-08-243@{24. 8. [1898]}|)be}\mylabel{L02854h}  \newcommand{\dateiname}{L02854}\newcommand{\titel}{Paul Goldmann an Arthur Schnitzler, 24. 8. [1898]}\newcommand{\editorInnen}{Martin Anton Müller und Laura Untner}%% latex-leseansicht-abspann.tex
%% Abspann für die Leseansicht.
%% Der Schalter \ifkorrekturansicht ist bereits durch den Vorspann gesetzt.

%% latex-abspann.tex
%% Gemeinsamer Abspann für Korrekturansicht und Leseansicht.
%% Setzt den Schalter \ifkorrekturansicht voraus (gesetzt in den
%% einbindenden Dateien latex-korrekturansicht-abspann.tex bzw.
%% latex-leseansicht-abspann.tex).
%% ---------------------------------------------------------------

\normalsize

% Das esempio-Environment wird nur in der Leseansicht benötigt
\ifkorrekturansicht\else
\newenvironment{esempio}[3]%
{
    \vspace{1.5ex}
    \rlap{\underline{#1}}
    \par
    \setlength{\parindent}{0cm}
    \nopagebreak
    \leftskip=#2cm
    \rightskip=#3cm
}
{
    \par
}
\fi

\doendnotes{C}
\bigskip
\vfill

\clearpage

\footnotesize

\ifkorrekturansicht
  \lohead{\textsc{register}}
\fi

% theindex-Environment neu definieren ohne reledmac
\makeatletter
\renewenvironment{theindex}{%
  \ifkorrekturansicht
    \section*{\indexname}%
  \else
    \subsubsection*{Index der erwähnten Entitäten}%
  \fi
  \setlength{\parindent}{0pt}%
  \setlength{\parskip}{0pt plus 0.3pt}%
  \let\item\@idxitem
}{%
  \ifkorrekturansicht\clearpage\fi
}
\makeatother

\IfFileExists{\jobname-pw.ind}{\input{\jobname-pw.ind}}{}

% Quellenangabe nur in der Leseansicht
\ifkorrekturansicht\else
% Fallback-Definitionen, falls die .tex-Datei \titel etc. nicht gesetzt hat
\providecommand{\titel}{}
\providecommand{\editorInnen}{}
\providecommand{\dateiname}{\jobname}

\vspace{3cm}

\vfill

\footnotesize
\textsc{Quelle}: \titel. Herausgegeben von {\editorInnen}. In: \emph{Arthur Schnitzler: Briefwechsel mit Autorinnen und Autoren}.
 Digitale Edition, https://schnitzler-briefe.acdh.oeaw.ac.at/{\dateiname}.html (Stand \today)
\fi

\end{document}


