%% latex-korrekturansicht-vorspann.tex
%% Vorspann für die Korrekturansicht.
%% Lädt die gemeinsame Datei latex-vorspann.tex mit gesetztem Schalter.

\newif\ifkorrekturansicht
\korrekturansichttrue

\input{../tex-inputs/latex-vorspann}


\section[Arthur Schnitzler an Robert Adam, 13. {[}7. 1915?{]}]{L02214 Arthur Schnitzler an Robert Adam, 13. {[}7. 1915?{]}}
\nopagebreak\mylabel{L02214v}
\rehead{ }\normalsize\beginnumbering\briefempfaengerindex{Adam, Robert@\textsc{Adam, Robert}!zzzSchnitzler, Arthur@\emph{von Arthur Schnitzler}!1915-07-131@{13. {[}7. 1915?{]}}|(be}
\toendnotes[C]{\smallbreak\pagebreak[2]}\Standort{DLA, 96.34.1–2.}
\physDesc{Sonderfall, Umschlag, 130 Zeichen
\newline{}Handschrift: 1) schwarze Tinte, deutsche Kurrent\hspace{1em}2) Bleistift, deutsche Kurrent (\noindent{}»Druckſache«)\hspace{1em}
\newline{}Versand: 1) Stempel: »\nobreak{}\oindex{XVIII., Waehring@\textbf{XVIII., Währing}, \emph{A.ADM3}|pwk}18/\textsubscript{1} Wien
                                       111, 13 \textcolor{gray}{VII.}{[}1915{]}\nobreak{}«.   2) das erhöhte Porto und der Kleber »R« (für »Rekommandiert«,
                                 eingeschrieben) deuten darauf hin, dass damit das Manuskript retour
                                 gesandt wurde}\pstart{}{\pb}Abſender\pend{}\pstart{}\textsc{Dr Arthur Schnitzler.}{ }Wien XVIII\oindex{XVIII., Waehring@\textbf{XVIII., Währing}, \emph{A.ADM3}|pw}\pend{}\pstart{}\textsc{Sternwartestr 71.\oindex{Sternwartestrasse 71@\textbf{Sternwartestraße 71}, \emph{Wohngebäude (K.WHS)}|pw}}\pend{}{\bigskip}\pstart{}Herrn Dr. \textsc{Robert Adam Pollak}\pend{}\pstart{}Bezirksrichter in\pend{}\pstart{}\textsc{Zistersdorf\oindex{Zistersdorf@\textbf{Zistersdorf}, \emph{A.ADM3}|pw}}\pend{}\pstart{}\textsc{N. Oe.}\oindex{Niederoesterreich@\textbf{Niederösterreich}, \emph{A.ADM1}|pw} – \pend{}\pstart{}Druckſache\pend{}{\bigskip}\endnumbering\briefempfaengerindex{Adam, Robert@\textsc{Adam, Robert}!zzzSchnitzler, Arthur@\emph{von Arthur Schnitzler}!1915-07-131@{13. {[}7. 1915?{]}}|)be}\mylabel{L02214h}  \normalsize

\doendnotes{C}
\bigskip
\vfill

\clearpage

\footnotesize

\lohead{\textsc{register}}

% Definiere theindex-Environment komplett neu ohne reledmac
\makeatletter
\renewenvironment{theindex}{%
  \section*{\indexname}%
  \setlength{\parindent}{0pt}%
  \setlength{\parskip}{0pt plus 0.3pt}%
  \let\item\@idxitem
}{%
  \clearpage
}
\makeatother

\IfFileExists{\jobname-pw.ind}{\input{\jobname-pw.ind}}{}

\end{document}

      