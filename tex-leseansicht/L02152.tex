%% latex-leseansicht-vorspann.tex
%% Vorspann für die Leseansicht.
%% Lädt die gemeinsame Datei latex-vorspann.tex mit nicht gesetztem Schalter.

\newif\ifkorrekturansicht
\korrekturansichtfalse

\input{../tex-inputs/latex-vorspann}


\section[Arthur Schnitzler an Hermann Bahr, 12. 10. 1913]{L02152 Arthur Schnitzler an Hermann Bahr, 12. 10. 1913}
\nopagebreak\mylabel{L02152v}
\rehead{ }\normalsize\beginnumbering\briefempfaengerindex{Bahr, Hermann@\textsc{Bahr, Hermann}!zzzSchnitzler, Arthur@\emph{von Arthur Schnitzler}!1913-10-121@{12. 10. 1913}|(be}
\toendnotes[C]{\smallbreak\pagebreak[2]}
\correspDesc{Versand  durch Arthur Schnitzler am 12. 10. 1913 in Wien
\newline{}Erhalt  durch Hermann Bahr im Zeitraum [13. 10. 1913 – 17. 10. 1913?] in Salzburg}\toendnotes[C]{\smallbreak}
\Standort{TMW, HS AM 23394 Ba.}
\physDesc{Kartenbrief, 632 Zeichen
\newline{}Handschrift: schwarze Tinte, deutsche Kurrent
\newline{}Versand: Briefmarke nicht gestempelt 
\newline{}Ordnung: Lochung }
\buchAbdrucke{\weitereDrucke{1) \emph{12. 10. 1913.} In: Arthur Schnitzler: \emph{The Letters of Arthur Schnitzler to Hermann Bahr}. Edited, annotated, and with an introduction, by Donald G. Daviau. Chapel Hill: \emph{The University of North Carolina Press} 1978, S. 112 (University of North Carolina studies in the Germanic languages
                        and literatures, 89).} \weitereDrucke{2) Hermann Bahr, Arthur Schnitzler: \emph{Briefwechsel, Aufzeichnungen, Dokumente (1891–1931)}. Herausgegeben von Kurt Ifkovits und Martin Anton Müller. Göttingen: \emph{Wallstein} 2018, S. 491.} }\toendnotes[C]{\smallbreak}\pstart{}{\pb}Herrn Hermann Bahr, \pend{}\pstart{}Salzburg\oindex{Salzburg@\textbf{Salzburg}, \emph{Verwaltungsgebiet}|pw}\pend{}\pstart{}\textsc{Schloss Arenberg\oindex{Schloss Arenberg@\textbf{Schloss Arenberg}, \emph{Schloss}|pw}}\pend{}{\bigskip}\vspace{1em}
\pstart
           \raggedleft{}{\pb}Wien\oindex{Wien@\textbf{Wien}, \emph{Verwaltungsgebiet}|pw}, 12. X. 913\pend
           
\pstart{}Mein lieber Hermann,\pend\vspace{0.5em}
\pstart
           dein{ }ſchönes Burkhardbuch\pwindex{Bahr, Hermann 19.\,7.\,1863 Linz – 15.\,1.\,1934 München@\textsc{Bahr, Hermann} (19.\,7.\,1863 Linz – 15.\,1.\,1934 München), \emph{Schriftsteller, Kritiker}!Erinnerung an Burckhard@\strich\emph{Erinnerung an Burckhard}|pwv}, von
               dem mir die \label{K_L02152-1v}\edtext{meiſten Kapitel{ }ſchon
                  bekannt}{\lemma{\textnormal{\emph{meisten … bekannt}}}\Cendnote{\textnormal{Vorabdrucke aus \emph{Erinnerung an Burckhard}\pwindex{Bahr, Hermann 19.\,7.\,1863 Linz – 15.\,1.\,1934 München@\textsc{Bahr, Hermann} (19.\,7.\,1863 Linz – 15.\,1.\,1934 München), \emph{Schriftsteller, Kritiker}!Erinnerung an Burckhard@\strich\emph{Erinnerung an Burckhard}|pwk} waren in \emph{Der Merker}\orgindex{Merker@Der Merker|pwk}, \emph{Neue
                     Freie Presse}\orgindex{Neue Freie Presse@Neue Freie Presse|pwk} und \emph{Die neue Rundschau}\pwindex{neue Rundschau@\emph{Die neue Rundschau}|pwk}
                  erschienen.}}}\label{K_L02152-1} waren hab ich nun als ganzes, mit neuer Ergriffenheit geleſen,
               und danke dir von Herzen. Wenn es überhaupt möglich iſt \introOben{}einen\introOben{} Menſchen Leuten, die \substVorne{}\textsuperscript{\textcolor{gray}{Burckhar\pwindex{Burckhard, Max Eugen 14.\,7.\,1854 Korneuburg – 16.\,3.\,1912 Wien@\textsc{Burckhard, Max Eugen} (14.\,7.\,1854 Korneuburg – 16.\,3.\,1912 Wien), \emph{Schriftsteller, Rechtswissenschaftler, Theaterleiter}|pwu}}}\substDazwischen{}ihn\substHinten{} nicht gekannt haben, näher zu bringen – ich glaube, mit deiner Geſtaltung
                  Burckhards\pwindex{Burckhard, Max Eugen 14.\,7.\,1854 Korneuburg – 16.\,3.\,1912 Wien@\textsc{Burckhard, Max Eugen} (14.\,7.\,1854 Korneuburg – 16.\,3.\,1912 Wien), \emph{Schriftsteller, Rechtswissenschaftler, Theaterleiter}|pw} m\substVorne{}\textsuperscript{uſs}\substDazwischen{}üßte\substHinten{} es gelungen sein. Dir und einigen wenigen andern bleibt ja in jedem Fall das
               Glück ihn gekannt und erkannt zu haben. Wie{ }ſehr{ }ſind die zu bedauern, die das eine
               verſäumt, das andre nicht vermocht haben! –\pend
           
\pstart
           Viele Grüße von uns\pwindex{Schnitzler, Olga 17.\,1.\,1882 Wien – 13.\,1.\,1970 Lugano@\textsc{Schnitzler, Olga} (17.\,1.\,1882 Wien – 13.\,1.\,1970 Lugano), \emph{Schauspielerin, Sängerin}|pwv} zu
                  Euch\pwindex{Bahr-Mildenburg, Anna 29.\,11.\,1872 Wien – 27.\,1.\,1947 ebd.@\textsc{Bahr-Mildenburg, Anna} (29.\,11.\,1872 Wien – 27.\,1.\,1947 ebd.), \emph{Sängerin}|pwv}!{\\[\baselineskip]}Dein
                  \spacefill\mbox{Arthur}\pend
           \leftskip=0em{}\selectlanguage{ngerman}\endnumbering\briefempfaengerindex{Bahr, Hermann@\textsc{Bahr, Hermann}!zzzSchnitzler, Arthur@\emph{von Arthur Schnitzler}!1913-10-121@{12. 10. 1913}|)be}\mylabel{L02152h}  \newcommand{\dateiname}{L02152}\newcommand{\titel}{Arthur Schnitzler an Hermann Bahr, 12. 10. 1913}\newcommand{\editorInnen}{Herausgegeben von Martin Anton Müller}%% latex-leseansicht-abspann.tex
%% Abspann für die Leseansicht.
%% Der Schalter \ifkorrekturansicht ist bereits durch den Vorspann gesetzt.

%% latex-abspann.tex
%% Gemeinsamer Abspann für Korrekturansicht und Leseansicht.
%% Setzt den Schalter \ifkorrekturansicht voraus (gesetzt in den
%% einbindenden Dateien latex-korrekturansicht-abspann.tex bzw.
%% latex-leseansicht-abspann.tex).
%% ---------------------------------------------------------------

\normalsize

% Das esempio-Environment wird nur in der Leseansicht benötigt
\ifkorrekturansicht\else
\newenvironment{esempio}[3]%
{
    \vspace{1.5ex}
    \rlap{\underline{#1}}
    \par
    \setlength{\parindent}{0cm}
    \nopagebreak
    \leftskip=#2cm
    \rightskip=#3cm
}
{
    \par
}
\fi

\doendnotes{C}
\bigskip
\vfill

\clearpage

\footnotesize

\ifkorrekturansicht
  \lohead{\textsc{register}}
\fi

% theindex-Environment neu definieren ohne reledmac
\makeatletter
\renewenvironment{theindex}{%
  \ifkorrekturansicht
    \section*{\indexname}%
  \else
    \subsubsection*{Index der erwähnten Entitäten}%
  \fi
  \setlength{\parindent}{0pt}%
  \setlength{\parskip}{0pt plus 0.3pt}%
  \let\item\@idxitem
}{%
  \ifkorrekturansicht\clearpage\fi
}
\makeatother

\IfFileExists{\jobname-pw.ind}{\input{\jobname-pw.ind}}{}

% Quellenangabe nur in der Leseansicht
\ifkorrekturansicht\else
% Fallback-Definitionen, falls die .tex-Datei \titel etc. nicht gesetzt hat
\providecommand{\titel}{}
\providecommand{\editorInnen}{}
\providecommand{\dateiname}{\jobname}

\vspace{3cm}

\vfill

\footnotesize
\textsc{Quelle}: \titel. Herausgegeben von {\editorInnen}. In: \emph{Arthur Schnitzler: Briefwechsel mit Autorinnen und Autoren}.
 Digitale Edition, https://schnitzler-briefe.acdh.oeaw.ac.at/{\dateiname}.html (Stand \today)
\fi

\end{document}


