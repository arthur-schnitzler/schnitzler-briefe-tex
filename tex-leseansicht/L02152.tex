\input{../tex-inputs/latex-pdf-vorspann}
\begin{center}
            \textcolor{red}{ENTWURF. ENTZIFFERUNG NOCH NICHT KORREKTURGELESEN}
                      \end{center}
            
               \section[Arthur Schnitzler an Hermann Bahr, 12. 10. 1913]{ Arthur Schnitzler an Hermann Bahr, 12. 10. 1913}\nopagebreak\mylabel{v}\rehead{ }\begin{ledgroupsized}[t]{13cm}\normalsize\beginnumbering\briefempfaengerindex{Bahr, Hermann@\textsc{Bahr, Hermann}!zzzSchnitzler, Arthur@\emph{von Arthur Schnitzler}!1913-10-121@{12. 10. 1913}|(be} \toendnotes[C]{\smallbreak\pagebreak[2]} \Standort{TMW, HS AM 23394 Ba.}
\physDesc{Kartenbrief
\newline{}Handschrift: schwarze Tinte, deutsche Kurrent\newline{}Versand: Briefmarke nicht gestempelt \newline{}Ordnung: Lochung }\buchAbdrucke{\weitereDrucke{1) \emph{12. 10. 1913.} In: Arthur Schnitzler: \emph{The Letters of Arthur Schnitzler to Hermann Bahr}. Edited, annotated, and with an introduction, by Donald G.
                        Daviau. Chapel Hill: \emph{The University of North Carolina Press} 1978, S. 112 (University of North Carolina studies in the Germanic languages
                        and literatures, 89).} \weitereDrucke{2) Hermann Bahr, Arthur Schnitzler: \emph{Briefwechsel, Aufzeichnungen, Dokumente (1891–1931)}. Hg. Kurt Ifkovits und Martin Anton Müller. Göttingen: \emph{Wallstein} 2018, S. 491.} }\toendnotes[C]{\smallbreak}\pstart{}{\pb}Herrn Hermann Bahr, \pend{}\pstart{}Salzburg\oindex{Salzburg@\textbf{Salzburg}|pw}\pend{}\pstart{}\textsc{Schloss Arenberg\oindex{Schloss Arenberg@\textbf{Schloss Arenberg}|pw}}\pend{}{\bigskip}\pstart
           \raggedleft{}{\pb}Wien\oindex{Wien@\textbf{Wien}|pw}, 12. X. 913\pend
           \pstart{}Mein lieber Hermann,\pend\pstart
           dein ſchönes Burkhardbuch\pwindex{Bahr, Hermann 19.07.1863 – 15.01.1934@\textsc{Bahr, Hermann} (19.07.1863 – 15.01.1934), \emph{Schriftsteller, Kritiker}!Erinnerung an Burckhard1913@\strich\emph{Erinnerung an Burckhard} {[}1913{]}|pwv}, von
               dem mir die \label{K_L02152_1v}\edtext{meiſten Kapitel ſchon
                  bekannt}{\lemma{\textnormal{\emph{meiſten … bekannt}}}\Cendnote{\textnormal{Vorabdrucke aus \emph{Erinnerung an Burckhard}\pwindex{Bahr, Hermann 19.07.1863 – 15.01.1934@\textsc{Bahr, Hermann} (19.07.1863 – 15.01.1934), \emph{Schriftsteller, Kritiker}!Erinnerung an Burckhard1913@\strich\emph{Erinnerung an Burckhard} {[}1913{]}|pwk} waren in \emph{Der Merker}\orgindex{Merker@Der Merker|pwk}, \emph{Neue Freie Presse}\orgindex{Neue Freie Presse@Neue Freie Presse|pwk} und \emph{Die neue Rundschau}\pwindex{neue Rundschau1904@\emph{Die neue Rundschau}|pwk} erschienen.}}}\label{K_L02152_1h} waren hab
               ich nun als ganzes, mit neuer Ergriffenheit geleſen, und danke dir von Herzen. Wenn
               es überhaupt möglich iſt \introOben{}einen\introOben{} Menſchen Leuten, die \substVorne{}\textsuperscript{\textcolor{gray}{Burckhar\pwindex{Burckhard, Max Eugen 14.07.1854 – 16.03.1912@\textsc{Burckhard, Max Eugen} (14.07.1854 – 16.03.1912), \emph{Schriftsteller, Rechtswissenschaftler, Theaterleiter}|pwu}}}{\allowbreak}\substDazwischen{}ihn\substHinten{} nicht gekannt haben, näher zu bringen – ich glaube, mit deiner Geſtaltung
                  Burckhards\pwindex{Burckhard, Max Eugen 14.07.1854 – 16.03.1912@\textsc{Burckhard, Max Eugen} (14.07.1854 – 16.03.1912), \emph{Schriftsteller, Rechtswissenschaftler, Theaterleiter}|pw} m\substVorne{}\textsuperscript{uſs}\substDazwischen{}üßte\substHinten{} es gelungen sein. Dir und einigen wenigen andern bleibt ja in jedem Fall das
               Glück ihn gekannt und erkannt zu haben. Wie ſehr ſind die zu bedauern, die das eine
               verſäumt, das andre nicht vermocht haben! –\pend
           \pstart
           Viele Grüße von uns\pwindex{Schnitzler, Olga 17.01.1882 – 13.01.1970@\textsc{Schnitzler, Olga} (17.01.1882 – 13.01.1970), \emph{Schauspielerin, Sängerin}|pwv} zu Euch\pwindex{Bahr-Mildenburg, Anna 29.11.1872 – 27.01.1947@\textsc{Bahr-Mildenburg, Anna} (29.11.1872 – 27.01.1947), \emph{Sängerin}|pwv}!{\\[\baselineskip]}Dein
                  \spacefill\mbox{Arthur}\pend
           \leftskip=0em{}\endnumbering\briefempfaengerindex{Bahr, Hermann@\textsc{Bahr, Hermann}!zzzSchnitzler, Arthur@\emph{von Arthur Schnitzler}!1913-10-121@{12. 10. 1913}|)be}\mylabel{h}\end{ledgroupsized}  \newcommand{\dateiname}{L02152}\newcommand{\titel}{Arthur Schnitzler an Hermann Bahr, 12. 10. 1913}\newcommand{\editorInnen}{ Kurt Ifkovits,  Martin Anton Müller}\input{../tex-inputs/latex-pdf-abspann}
      