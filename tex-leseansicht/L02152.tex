%% latex-korrekturansicht-vorspann.tex
%% Vorspann für die Korrekturansicht.
%% Lädt die gemeinsame Datei latex-vorspann.tex mit gesetztem Schalter.

\newif\ifkorrekturansicht
\korrekturansichttrue

\input{../tex-inputs/latex-vorspann}


\section[Arthur Schnitzler an Hermann Bahr, 12. 10. 1913]{L02152 Arthur Schnitzler an Hermann Bahr, 12. 10. 1913}
\nopagebreak\mylabel{L02152v}
\rehead{ }\normalsize\beginnumbering\briefempfaengerindex{Bahr, Hermann@\textsc{Bahr, Hermann}!zzzSchnitzler, Arthur@\emph{von Arthur Schnitzler}!1913-10-121@{12. 10. 1913}|(be}
\toendnotes[C]{\smallbreak\pagebreak[2]}\Standort{TMW, HS AM 23394 Ba.}
\physDesc{Kartenbrief, 632 Zeichen
\newline{}Handschrift: schwarze Tinte, deutsche Kurrent
\newline{}Versand: Briefmarke nicht gestempelt 
\newline{}Ordnung: Lochung }
\buchAbdrucke{\weitereDrucke{1) Arthur Schnitzler: \emph{The Letters of Arthur Schnitzler to Hermann Bahr}. Chapel Hill: \emph{The University of North Carolina Press} 1978, S. 112.} \weitereDrucke{2) Hermann Bahr, Arthur Schnitzler: \emph{Briefwechsel, Aufzeichnungen, Dokumente (1891–1931)}. Göttingen: \emph{Wallstein} 2018, S. 491.} }\toendnotes[C]{\smallbreak}\pstart{}{\pb}Herrn Hermann Bahr, \pend{}\pstart{}Salzburg\oindex{Salzburg@\textbf{Salzburg}, \emph{A.ADM2}|pw}\pend{}\pstart{}\textsc{Schloss Arenberg\oindex{Schloss Arenberg@\textbf{Schloss Arenberg}, \emph{Schloss (K.SLS)}|pw}}\pend{}{\bigskip}\vspace{1em}
\pstart
           \raggedleft{}{\pb}Wien\oindex{Wien@\textbf{Wien}, \emph{A.ADM2}|pw}, 12. X. 913\pend
           
\pstart{}Mein lieber Hermann,\pend\vspace{0.5em}
\pstart
           dein ſchönes Burkhardbuch\pwindex{Erinnerung an Burckhard@\emph{Erinnerung an Burckhard}|pwv}, von
               dem mir die \label{K_L02152-1v}\edtext{meiſten Kapitel ſchon
                  bekannt}{\lemma{\textnormal{\emph{meiſten … bekannt}}}\Cendnote{\textnormal{Vorabdrucke aus \emph{Erinnerung an Burckhard}\pwindex{Erinnerung an Burckhard@\emph{Erinnerung an Burckhard}|pwk} waren in \emph{Der Merker}\orgindex{Merker@Der Merker|pwk}, \emph{Neue
                     Freie Presse}\orgindex{Neue Freie Presse@Neue Freie Presse|pwk} und \emph{Die neue Rundschau}\pwindex{neue Rundschau@\emph{Die neue Rundschau}|pwk}
                  erschienen.}}}\label{K_L02152-1} waren hab ich nun als ganzes, mit neuer Ergriffenheit geleſen,
               und danke dir von Herzen. Wenn es überhaupt möglich iſt \introOben{}einen\introOben{} Menſchen Leuten, die \substVorne{}\textsuperscript{\textcolor{gray}{Burckhar\pwindex{Burckhard, Max Eugen 14.07.1854 – 16.03.1912@\textsc{Burckhard, Max Eugen} (14.07.1854 – 16.03.1912), \emph{Schriftsteller/Schriftstellerin, Rechtswissenschaftler/Rechtswissenschaftlerin, Theaterleiter/Theaterleiterin}|pwu}}}\substDazwischen{}ihn\substHinten{} nicht gekannt haben, näher zu bringen – ich glaube, mit deiner Geſtaltung
                  Burckhards\pwindex{Burckhard, Max Eugen 14.07.1854 – 16.03.1912@\textsc{Burckhard, Max Eugen} (14.07.1854 – 16.03.1912), \emph{Schriftsteller/Schriftstellerin, Rechtswissenschaftler/Rechtswissenschaftlerin, Theaterleiter/Theaterleiterin}|pw} m\substVorne{}\textsuperscript{uſs}\substDazwischen{}üßte\substHinten{} es gelungen sein. Dir und einigen wenigen andern bleibt ja in jedem Fall das
               Glück ihn gekannt und erkannt zu haben. Wie ſehr ſind die zu bedauern, die das eine
               verſäumt, das andre nicht vermocht haben! –\pend
           
\pstart
           Viele Grüße von uns\pwindex{Schnitzler, Olga 17.01.1882 – 13.01.1970@\textsc{Schnitzler, Olga} (17.01.1882 – 13.01.1970), \emph{Schauspieler/Schauspielerin, Sänger/Sängerin}|pwv} zu
                  Euch\pwindex{Bahr-Mildenburg, Anna 29.11.1872 – 27.01.1947@\textsc{Bahr-Mildenburg, Anna} (29.11.1872 – 27.01.1947), \emph{Sänger/Sängerin}|pwv}!{\\[\baselineskip]}Dein
                  \spacefill\mbox{Arthur}\pend
           \leftskip=0em{}\selectlanguage{ngerman}\endnumbering\briefempfaengerindex{Bahr, Hermann@\textsc{Bahr, Hermann}!zzzSchnitzler, Arthur@\emph{von Arthur Schnitzler}!1913-10-121@{12. 10. 1913}|)be}\mylabel{L02152h}  \normalsize

\doendnotes{C}
\bigskip
\vfill

\clearpage

\footnotesize

\lohead{\textsc{register}}

% Definiere theindex-Environment komplett neu ohne reledmac
\makeatletter
\renewenvironment{theindex}{%
  \section*{\indexname}%
  \setlength{\parindent}{0pt}%
  \setlength{\parskip}{0pt plus 0.3pt}%
  \let\item\@idxitem
}{%
  \clearpage
}
\makeatother

\IfFileExists{\jobname-pw.ind}{\input{\jobname-pw.ind}}{}

\end{document}

      