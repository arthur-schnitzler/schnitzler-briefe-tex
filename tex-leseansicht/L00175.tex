%% latex-leseansicht-vorspann.tex
%% Vorspann für die Leseansicht.
%% Lädt die gemeinsame Datei latex-vorspann.tex mit nicht gesetztem Schalter.

\newif\ifkorrekturansicht
\korrekturansichtfalse

\input{../tex-inputs/latex-vorspann}


\section[Eduard Michael Kafka an Arthur Schnitzler, 11. 2. 1893]{L00175 Eduard Michael Kafka an Arthur Schnitzler, 11. 2. 1893}
\nopagebreak\mylabel{L00175v}
\rehead{ }\normalsize\beginnumbering\briefempfaengerindex{Schnitzler, Arthur@\textsc{Schnitzler, Arthur}!zzzKafka, Eduard Michael@\emph{von Eduard Michael Kafka}!1893-02-121@{12. 2. 1893}|(be}
\toendnotes[C]{\smallbreak\pagebreak[2]}
\correspDesc{Versand  durch Eduard Michael Kafka am 12. 2. 1893 in Leipzig
\newline{}Erhalt  durch Arthur Schnitzler im Zeitraum [13. 2. 1893
                  – 17. 2. 1893?] in Wien}\toendnotes[C]{\smallbreak}
\Standort{DLA, A:Schnitzler, HS.NZ85.1.3604.}
\physDesc{Bildpostkarte, 574 Zeichen
\newline{}Handschrift: schwarze Tinte, deutsche Kurrent
\newline{}Versand: 1) Stempel: »\nobreak{}\oindex{Leipzig@\textbf{Leipzig}, \emph{Hauptstadt}|pwk}Leipzig, 12. 2. 93, 5–6V\nobreak{}«.   2) Stempel: »\nobreak{}\oindex{I., Innere Stadt@\textbf{I., Innere Stadt}, \emph{Verwaltungsgebiet}|pwk}Wien 1/1 1, 13 2 93, 10–11½V.\nobreak{}«. }\toendnotes[C]{\smallbreak}\pstart{}{\pb}Herrn\pend{}\pstart{}\textsc{D\textsuperscript{r.} Arthur Schnitzler}\pend{}\pstart{}\textsc{Wien}\oindex{Wien@\textbf{Wien}, \emph{Verwaltungsgebiet}|pw}\pend{}\pstart{}I. Grillparzerſtraße 7\oindex{Wien@\textbf{Wien}!I., Innere Stadt@\textbf{I., Innere Stadt}!Grillparzerstraße@\textbf{Grillparzerstraße}, \emph{Straße}|pw}. \pend{}{\bigskip}
\pstart
           \noindent{}\centering{}{\pb}\textcolor{gray}{\textbf{Gruss aus Auerbach’s Keller,
                     Leipzig\oindex{Auerbachs Keller@\textbf{Auerbachs Keller}, \emph{Lokal}|pw}.}}\pend
           \vspace{1em}
\pstart
           \raggedleft{}{\pb}11/II 93.{\\}Ständige Adreſse: \introOben{}bis gegen Ende des
                     Monats\introOben{}{ }Berlin, Wienerhof\oindex{Wienerhof@\textbf{Wienerhof}, \emph{Gebäude}|pw}{\\}Marienstraße 20\oindex{Marienstraße@\textbf{Marienstraße}, \emph{Straße}|pw}.\pend
           
\pstart{}Lieber Schnitzler,\pend\vspace{0.5em}
\pstart
           Senden Sie, bitte unverzüglich 1 Ex. des »\textsc{Anatol}\pwindex{Schnitzler, Arthur 15.\,5.\,1862 Wien – 21.\,10.\,1931 ebd.@\textsc{Schnitzler, Arthur} (15.\,5.\,1862 Wien – 21.\,10.\,1931 ebd.), \emph{Schriftsteller, Mediziner}!Anatol@\strich\emph{Anatol}|pw}« an \textsc{\uline{J. Simon}}\pwindex{Simon, Josef 22.\,2.\,1854 Hořice – 29.\,12.\,1926 Wien@\textsc{Simon, Josef} (22.\,2.\,1854 Hořice – 29.\,12.\,1926 Wien), \emph{Kunstsammler, Bankier}|pw} (\textsc{Prag}\oindex{Prag@\textbf{Prag}, \emph{Land}|pw}) \textsc{\strikeout{Raffa\oindex{Raffaelova@\textbf{Raffaelova}, \emph{Straße}|pw}}}{ }\textsc{Park}ſtraße 9\oindex{Wilsonova@\textbf{Wilsonova}, \emph{Straße}|pw} er will \uline{Neumann}\pwindex{Neumann, Angelo 18.\,8.\,1838 Stupava – 20.\,12.\,1910 Prag@\textsc{Neumann, Angelo} (18.\,8.\,1838 Stupava – 20.\,12.\,1910 Prag), \emph{Theaterleiter, Sänger}|pw} dafür intereſſiren. Herr \textsc{\uline{Simon}}\pwindex{Simon, Josef 22.\,2.\,1854 Hořice – 29.\,12.\,1926 Wien@\textsc{Simon, Josef} (22.\,2.\,1854 Hořice – 29.\,12.\,1926 Wien), \emph{Kunstsammler, Bankier}|pw} iſt der Schwager von Joh. \textsc{Strauß}\pwindex{Strauss, Johann 25.\,10.\,1825 Wien – 3.\,6.\,1899 ebd.@\textsc{Strauss, Johann} (25.\,10.\,1825 Wien – 3.\,6.\,1899 ebd.), \emph{Komponist, Dirigent}|pw}. – Herr \textsc{\uline{Jarno}}\pwindex{Jarno, Josef 24.\,8.\,1865 Budapest – 11.\,1.\,1932 Wien@\textsc{Jarno, Josef} (24.\,8.\,1865 Budapest – 11.\,1.\,1932 Wien), \emph{Theaterleiter, Schauspieler}|pw} vom \textsc{Residenz}theater\oindex{Residenztheater Berlin@\textbf{Residenztheater Berlin}, \emph{Theater}|pw} in \textsc{Berlin}\oindex{Berlin@\textbf{Berlin}, \emph{Hauptstadt}|pw} läßt Ihnen{ }ſagen, er werde Ihre »Frage an das
                  Schickſal\pwindex{Schnitzler, Arthur 15.\,5.\,1862 Wien – 21.\,10.\,1931 ebd.@\textsc{Schnitzler, Arthur} (15.\,5.\,1862 Wien – 21.\,10.\,1931 ebd.), \emph{Schriftsteller, Mediziner}!Frage an das Schicksal@\strich\emph{Die Frage an das Schicksal}|pw}« u. »Abſchieds\textsc{souper}\pwindex{Schnitzler, Arthur 15.\,5.\,1862 Wien – 21.\,10.\,1931 ebd.@\textsc{Schnitzler, Arthur} (15.\,5.\,1862 Wien – 21.\,10.\,1931 ebd.), \emph{Schriftsteller, Mediziner}!Abschiedssouper@\strich\emph{Abschiedssouper}|pw}« heuer im \textsc{So{\geminationm}er}\orgindex{Saisontheater Ischl@Saisontheater Ischl|pwv} in \substVorne{}\textsuperscript{\textsc{Ishl}\oindex{Bad Ischl@\textbf{Bad Ischl}|pw}}\substDazwischen{}\textsc{Ischl}\oindex{Bad Ischl@\textbf{Bad Ischl}|pw}\substHinten{}{ }ſpielen. Warum{ }ſenden Sie Nichts an das »\textsc{Magazin}\orgindex{Magazin für die Literatur des Auslandes@Magazin für die Literatur des Auslandes|pw}« in Berlin\oindex{Berlin@\textbf{Berlin}, \emph{Hauptstadt}|pw}? \textsc{Lehmann}\orgindex{F. und P. Lehmann@F. und P. Lehmann|pw} u. Neumann-Hofer\pwindex{Neumann-Hofer, Gilbert Otto 4.\,2.\,1857 Bol’shiye Berezhki – 14.\,4.\,1941 Detmold@\textsc{Neumann-Hofer, Gilbert Otto} (4.\,2.\,1857 Bol’shiye Berezhki – 14.\,4.\,1941 Detmold), \emph{Kritiker, Theaterleiter}|pw} intereſſiren{ }ſich{ }ſehr
               für Sie.\pend
           \pstart Gruß \spacefill\mbox{Kafka}\pend{}\selectlanguage{ngerman}\endnumbering\briefempfaengerindex{Schnitzler, Arthur@\textsc{Schnitzler, Arthur}!zzzKafka, Eduard Michael@\emph{von Eduard Michael Kafka}!1893-02-121@{12. 2. 1893}|)be}\mylabel{L00175h}  \newcommand{\dateiname}{L00175}\newcommand{\titel}{Eduard Michael Kafka an Arthur Schnitzler, 11. 2. 1893}\newcommand{\editorInnen}{Martin Anton Müller und Gerd-Hermann Susen}%% latex-leseansicht-abspann.tex
%% Abspann für die Leseansicht.
%% Der Schalter \ifkorrekturansicht ist bereits durch den Vorspann gesetzt.

%% latex-abspann.tex
%% Gemeinsamer Abspann für Korrekturansicht und Leseansicht.
%% Setzt den Schalter \ifkorrekturansicht voraus (gesetzt in den
%% einbindenden Dateien latex-korrekturansicht-abspann.tex bzw.
%% latex-leseansicht-abspann.tex).
%% ---------------------------------------------------------------

\normalsize

% Das esempio-Environment wird nur in der Leseansicht benötigt
\ifkorrekturansicht\else
\newenvironment{esempio}[3]%
{
    \vspace{1.5ex}
    \rlap{\underline{#1}}
    \par
    \setlength{\parindent}{0cm}
    \nopagebreak
    \leftskip=#2cm
    \rightskip=#3cm
}
{
    \par
}
\fi

\doendnotes{C}
\bigskip
\vfill

\clearpage

\footnotesize

\ifkorrekturansicht
  \lohead{\textsc{register}}
\fi

% theindex-Environment neu definieren ohne reledmac
\makeatletter
\renewenvironment{theindex}{%
  \ifkorrekturansicht
    \section*{\indexname}%
  \else
    \subsubsection*{Index der erwähnten Entitäten}%
  \fi
  \setlength{\parindent}{0pt}%
  \setlength{\parskip}{0pt plus 0.3pt}%
  \let\item\@idxitem
}{%
  \ifkorrekturansicht\clearpage\fi
}
\makeatother

\IfFileExists{\jobname-pw.ind}{\input{\jobname-pw.ind}}{}

% Quellenangabe nur in der Leseansicht
\ifkorrekturansicht\else
% Fallback-Definitionen, falls die .tex-Datei \titel etc. nicht gesetzt hat
\providecommand{\titel}{}
\providecommand{\editorInnen}{}
\providecommand{\dateiname}{\jobname}

\vspace{3cm}

\vfill

\footnotesize
\textsc{Quelle}: \titel. Herausgegeben von {\editorInnen}. In: \emph{Arthur Schnitzler: Briefwechsel mit Autorinnen und Autoren}.
 Digitale Edition, https://schnitzler-briefe.acdh.oeaw.ac.at/{\dateiname}.html (Stand \today)
\fi

\end{document}


