%% latex-leseansicht-vorspann.tex
%% Vorspann für die Leseansicht.
%% Lädt die gemeinsame Datei latex-vorspann.tex mit nicht gesetztem Schalter.

\newif\ifkorrekturansicht
\korrekturansichtfalse

\input{../tex-inputs/latex-vorspann}


         
         \newcommand{\erwaehntePersonen}{Personen: Josef Jarno, Angelo Neumann, Gilbert Otto Neumann-Hofer, Josef Simon, Johann Strauss}
         \newcommand{\erwaehnteInstitutionen}{Institutionen: F. und P. Lehmann, Magazin für die Literatur des Auslandes, Saisontheater Ischl}
         \newcommand{\erwaehnteOrte}{Orte: Auerbachs Keller, Bad Ischl, Berlin, Grillparzerstraße, I., Innere Stadt, Leipzig, Marienstraße, Prag, Raffaelova, Residenztheater Berlin, Wien, Wienerhof, Wilsonova}
         \newcommand{\erwaehnteWerke}{Werke: Abschiedssouper, Anatol, Die Frage an das Schicksal}
               \section[Eduard Michael Kafka an Arthur Schnitzler, 11. 2. 1893]{ Eduard Michael Kafka an Arthur Schnitzler, 11. 2. 1893}\nopagebreak\mylabel{v}\rehead{ }\begin{ledgroupsized}[t]{13cm}\normalsize\beginnumbering \toendnotes[C]{\smallbreak\pagebreak[2]} \Standort{DLA, A:Schnitzler, HS.NZ85.1.3604.}
\physDesc{Bildpostkarte
\newline{}Handschrift: schwarze Tinte, deutsche Kurrent\newline{}Versand: 1) Stempel: »\nobreak{}\oindex{Leipzig@\textbf{Leipzig}|pwk}Leipzig, 12. 2. 93, 5–6V\nobreak{}«.   2) Stempel: »\nobreak{}\oindex{I., Innere Stadt@\textbf{I., Innere Stadt}|pwk}Wien 1/1 1, 13 2 93, 10–11½V.\nobreak{}«. }\toendnotes[C]{\smallbreak}\pstart{}{\pb}Herrn\pend{}\pstart{}\textsc{D\textsuperscript{r.} Arthur Schnitzler}\pend{}\pstart{}\textsc{Wien}\oindex{Wien@\textbf{Wien}|pw}\pend{}\pstart{}I. Grillparzerſtraße 7\oindex{Grillparzerstrasse@\textbf{Grillparzerstraße}|pw}. \pend{}{\bigskip}\pstart
           \noindent{}\centering{}{\pb}\textcolor{gray}{\textbf{Gruss aus Auerbach’s
                                Keller, Leipzig\oindex{Auerbachs Keller@\textbf{Auerbachs Keller}|pw}.}}\pend
           \pstart
           \raggedleft{}11/II 93.{\\}Ständige Adreſse: \introOben{}bis gegen
                            Ende des Monats\introOben{}{ }Berlin, Wienerhof\oindex{Wienerhof@\textbf{Wienerhof}|pw}{\\}Marienstraße 20\oindex{Marienstrasse@\textbf{Marienstraße}|pw}.\pend
           \pstart{}Lieber Schnitzler,\pend\pstart
           Senden Sie, bitte unverzüglich 1 Ex. des »\textsc{Anatol}\pwindex{Schnitzler, Arthur 15.05.1862 – 21.10.1931@\textsc{Schnitzler, Arthur} (15.05.1862 – 21.10.1931), \emph{Schriftsteller, Mediziner}!Anatol1892-10-29@\strich\emph{Anatol} {[}1892-10-29{]}|pw}« an \textsc{\uline{J. Simon}}\pwindex{Simon, Josef 22.02.1854 – 29.12.1926@\textsc{Simon, Josef} (22.02.1854 – 29.12.1926), \emph{Kunstsammler, Bankier}|pw} (\textsc{Prag}\oindex{Prag@\textbf{Prag}|pw}) \textsc{\strikeout{Raffa\oindex{Raffaelova@\textbf{Raffaelova}|pw}}}{ }\textsc{Park}ſtraße 9\oindex{Wilsonova@\textbf{Wilsonova}|pw} er will \uline{Neumann}\pwindex{Neumann, Angelo 18.08.1838 – 20.12.1910@\textsc{Neumann, Angelo} (18.08.1838 – 20.12.1910), \emph{Theaterleiter, Sänger}|pw} dafür intereſſiren. Herr \textsc{\uline{Simon}}\pwindex{Simon, Josef 22.02.1854 – 29.12.1926@\textsc{Simon, Josef} (22.02.1854 – 29.12.1926), \emph{Kunstsammler, Bankier}|pw} iſt der Schwager von Joh. \textsc{Strauß}\pwindex{Strauss, Johann 25.10.1825 – 03.06.1899@\textsc{Strauss, Johann} (25.10.1825 – 03.06.1899), \emph{Komponist, Dirigent}|pw}. – Herr \textsc{\uline{Jarno}}\pwindex{Jarno, Josef 24.08.1865 – 11.01.1932@\textsc{Jarno, Josef} (24.08.1865 – 11.01.1932), \emph{Theaterleiter, Schauspieler}|pw} vom \textsc{Residenz}theater\oindex{Residenztheater Berlin@\textbf{Residenztheater Berlin}|pw} in \textsc{Berlin}\oindex{Berlin@\textbf{Berlin}|pw} läßt Ihnen ſagen, er werde Ihre »Frage an das
                        Schickſal\pwindex{Schnitzler, Arthur 15.05.1862 – 21.10.1931@\textsc{Schnitzler, Arthur} (15.05.1862 – 21.10.1931), \emph{Schriftsteller, Mediziner}!Frage an das Schicksal01. 05. 1890@\strich\emph{Die Frage an das Schicksal} {[}01. 05. 1890{]}|pw}« u. »Abſchieds\textsc{souper}\pwindex{Schnitzler, Arthur 15.05.1862 – 21.10.1931@\textsc{Schnitzler, Arthur} (15.05.1862 – 21.10.1931), \emph{Schriftsteller, Mediziner}!Abschiedssouper1892@\strich\emph{Abschiedssouper} {[}1892{]}|pw}« heuer im \textsc{So{\geminationm}er}\orgindex{Saisontheater Ischl@Saisontheater Ischl|pwv} in \substVorne{}\textsuperscript{\textsc{Ishl}\oindex{Bad Ischl@\textbf{Bad Ischl}|pw}}\substDazwischen{}\textsc{Ishl}\oindex{Bad Ischl@\textbf{Bad Ischl}|pw}\substHinten{}{ }ſpielen. Warum ſenden Sie Nichts an das »\textsc{Magazin}\orgindex{Magazin fuer die Literatur des Auslandes@Magazin für die Literatur des Auslandes|pw}« in Berlin\oindex{Berlin@\textbf{Berlin}|pw}? \textsc{Lehmann}\orgindex{F. und P. Lehmann@F. und P. Lehmann|pw} u. Neumann-Hofer\pwindex{Neumann-Hofer, Gilbert Otto 04.02.1857 – 14.04.1941@\textsc{Neumann-Hofer, Gilbert Otto} (04.02.1857 – 14.04.1941), \emph{Kritiker, Theaterleiter}|pw} intereſſiren ſich
                    ſehr für Sie.\pend
           \pstart Gruß \spacefill\mbox{Kafka}\pend{}
         
         \endnumbering\mylabel{h}\end{ledgroupsized}  \newcommand{\dateiname}{L00175}\newcommand{\titel}{Eduard Michael Kafka an Arthur Schnitzler, 11. 2. 1893}\newcommand{\editorInnen}{Martin Anton Müller und Gerd-Hermann Susen}%% latex-leseansicht-abspann.tex
%% Abspann für die Leseansicht.
%% Der Schalter \ifkorrekturansicht ist bereits durch den Vorspann gesetzt.

%% latex-abspann.tex
%% Gemeinsamer Abspann für Korrekturansicht und Leseansicht.
%% Setzt den Schalter \ifkorrekturansicht voraus (gesetzt in den
%% einbindenden Dateien latex-korrekturansicht-abspann.tex bzw.
%% latex-leseansicht-abspann.tex).
%% ---------------------------------------------------------------

\normalsize

% Das esempio-Environment wird nur in der Leseansicht benötigt
\ifkorrekturansicht\else
\newenvironment{esempio}[3]%
{
    \vspace{1.5ex}
    \rlap{\underline{#1}}
    \par
    \setlength{\parindent}{0cm}
    \nopagebreak
    \leftskip=#2cm
    \rightskip=#3cm
}
{
    \par
}
\fi

\doendnotes{C}
\bigskip
\vfill

\clearpage

\footnotesize

\ifkorrekturansicht
  \lohead{\textsc{register}}
\fi

% theindex-Environment neu definieren ohne reledmac
\makeatletter
\renewenvironment{theindex}{%
  \ifkorrekturansicht
    \section*{\indexname}%
  \else
    \subsubsection*{Index der erwähnten Entitäten}%
  \fi
  \setlength{\parindent}{0pt}%
  \setlength{\parskip}{0pt plus 0.3pt}%
  \let\item\@idxitem
}{%
  \ifkorrekturansicht\clearpage\fi
}
\makeatother

\IfFileExists{\jobname-pw.ind}{\input{\jobname-pw.ind}}{}

% Quellenangabe nur in der Leseansicht
\ifkorrekturansicht\else
% Fallback-Definitionen, falls die .tex-Datei \titel etc. nicht gesetzt hat
\providecommand{\titel}{}
\providecommand{\editorInnen}{}
\providecommand{\dateiname}{\jobname}

\vspace{3cm}

\vfill

\footnotesize
\textsc{Quelle}: \titel. Herausgegeben von {\editorInnen}. In: \emph{Arthur Schnitzler: Briefwechsel mit Autorinnen und Autoren}.
 Digitale Edition, https://schnitzler-briefe.acdh.oeaw.ac.at/{\dateiname}.html (Stand \today)
\fi

\end{document}


      