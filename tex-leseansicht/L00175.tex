%% latex-korrekturansicht-vorspann.tex
%% Vorspann für die Korrekturansicht.
%% Lädt die gemeinsame Datei latex-vorspann.tex mit gesetztem Schalter.

\newif\ifkorrekturansicht
\korrekturansichttrue

\input{../tex-inputs/latex-vorspann}


\section[Eduard Michael Kafka an Arthur Schnitzler, 11. 2. 1893]{L00175 Eduard Michael Kafka an Arthur Schnitzler, 11. 2. 1893}
\nopagebreak\mylabel{L00175v}
\rehead{ }\normalsize\beginnumbering\briefempfaengerindex{Schnitzler, Arthur@\textsc{Schnitzler, Arthur}!zzzKafka, Eduard Michael@\emph{von Eduard Michael Kafka}!1893-02-121@{12. 2. 1893}|(be}
\toendnotes[C]{\smallbreak\pagebreak[2]}\Standort{DLA, A:Schnitzler, HS.NZ85.1.3604.}
\physDesc{Bildpostkarte, 574 Zeichen
\newline{}Handschrift: schwarze Tinte, deutsche Kurrent
\newline{}Versand: 1) Stempel: »\nobreak{}\oindex{Leipzig@\textbf{Leipzig}, \emph{P.PPLA3}|pwk}Leipzig, 12. 2. 93, 5–6V\nobreak{}«.   2) Stempel: »\nobreak{}\oindex{I., Innere Stadt@\textbf{I., Innere Stadt}, \emph{A.ADM3}|pwk}Wien 1/1 1, 13 2 93, 10–11½V.\nobreak{}«. }\toendnotes[C]{\smallbreak}\pstart{}{\pb}Herrn\pend{}\pstart{}\textsc{D\textsuperscript{r.} Arthur Schnitzler}\pend{}\pstart{}\textsc{Wien}\oindex{Wien@\textbf{Wien}, \emph{A.ADM2}|pw}\pend{}\pstart{}I. Grillparzerſtraße 7\oindex{Grillparzerstrasse@\textbf{Grillparzerstraße}, \emph{R.ST}|pw}. \pend{}{\bigskip}
\pstart
           \noindent{}\centering{}{\pb}\textcolor{gray}{\textbf{Gruss aus Auerbach’s Keller,
                     Leipzig\oindex{Auerbachs Keller@\textbf{Auerbachs Keller}, \emph{Lokal (K.LKL)}|pw}.}}\pend
           \vspace{1em}
\pstart
           \raggedleft{}{\pb}11/II 93.{\\}Ständige Adreſse: \introOben{}bis gegen Ende des
                     Monats\introOben{}{ }Berlin, Wienerhof\oindex{Wienerhof@\textbf{Wienerhof}, \emph{Gebäude (K.GBD)}|pw}{\\}Marienstraße 20\oindex{Marienstrasse@\textbf{Marienstraße}, \emph{Straße (K.STR)}|pw}.\pend
           
\pstart{}Lieber Schnitzler,\pend\vspace{0.5em}
\pstart
           Senden Sie, bitte unverzüglich 1 Ex. des »\textsc{Anatol}\pwindex{Anatol@\emph{Anatol}|pw}« an \textsc{\uline{J. Simon}}\pwindex{Simon, Josef 22.02.1854 – 29.12.1926@\textsc{Simon, Josef} (22.02.1854 – 29.12.1926), \emph{Kunstsammler/Kunstsammlerin, Bankier/Bankierin}|pw} (\textsc{Prag}\oindex{Prag@\textbf{Prag}, \emph{A.ADM1}|pw}) \textsc{\strikeout{Raffa\oindex{Raffaelova@\textbf{Raffaelova}, \emph{Straße (K.STR)}|pw}}}{ }\textsc{Park}ſtraße 9\oindex{Wilsonova@\textbf{Wilsonova}, \emph{Straße (K.STR)}|pw} er will \uline{Neumann}\pwindex{Neumann, Angelo 18.08.1838 – 20.12.1910@\textsc{Neumann, Angelo} (18.08.1838 – 20.12.1910), \emph{Theaterleiter/Theaterleiterin, Sänger/Sängerin}|pw} dafür intereſſiren. Herr \textsc{\uline{Simon}}\pwindex{Simon, Josef 22.02.1854 – 29.12.1926@\textsc{Simon, Josef} (22.02.1854 – 29.12.1926), \emph{Kunstsammler/Kunstsammlerin, Bankier/Bankierin}|pw} iſt der Schwager von Joh. \textsc{Strauß}\pwindex{Strauss, Johann 25.10.1825 – 03.06.1899@\textsc{Strauss, Johann} (25.10.1825 – 03.06.1899), \emph{Komponist/Komponistin, Dirigent/Dirigentin}|pw}. – Herr \textsc{\uline{Jarno}}\pwindex{Jarno, Josef 24.08.1865 – 11.01.1932@\textsc{Jarno, Josef} (24.08.1865 – 11.01.1932), \emph{Theaterleiter/Theaterleiterin, Schauspieler/Schauspielerin}|pw} vom \textsc{Residenz}theater\oindex{Residenztheater Berlin@\textbf{Residenztheater Berlin}, \emph{Theater (K.THE)}|pw} in \textsc{Berlin}\oindex{Berlin@\textbf{Berlin}, \emph{P.PPLC}|pw} läßt Ihnen ſagen, er werde Ihre »Frage an das
                  Schickſal\pwindex{Frage an das Schicksal@\emph{Die Frage an das Schicksal}|pw}« u. »Abſchieds\textsc{souper}\pwindex{Abschiedssouper@\emph{Abschiedssouper}|pw}« heuer im \textsc{So{\geminationm}er}\orgindex{Saisontheater Ischl@Saisontheater Ischl|pwv} in \substVorne{}\textsuperscript{\textsc{Ishl}\oindex{Bad Ischl@\textbf{Bad Ischl}, \emph{P.PPL}|pw}}\substDazwischen{}\textsc{Ischl}\oindex{Bad Ischl@\textbf{Bad Ischl}, \emph{P.PPL}|pw}\substHinten{}{ }ſpielen. Warum ſenden Sie Nichts an das »\textsc{Magazin}\orgindex{Magazin fuer die Literatur des Auslandes@Magazin für die Literatur des Auslandes|pw}« in Berlin\oindex{Berlin@\textbf{Berlin}, \emph{P.PPLC}|pw}? \textsc{Lehmann}\orgindex{F. und P. Lehmann@F. und P. Lehmann|pw} u. Neumann-Hofer\pwindex{Neumann-Hofer, Gilbert Otto 04.02.1857 – 14.04.1941@\textsc{Neumann-Hofer, Gilbert Otto} (04.02.1857 – 14.04.1941), \emph{Kritiker/Kritikerin, Theaterleiter/Theaterleiterin}|pw} intereſſiren ſich ſehr
               für Sie.\pend
           \pstart Gruß \spacefill\mbox{Kafka}\pend{}\selectlanguage{ngerman}\endnumbering\briefempfaengerindex{Schnitzler, Arthur@\textsc{Schnitzler, Arthur}!zzzKafka, Eduard Michael@\emph{von Eduard Michael Kafka}!1893-02-121@{12. 2. 1893}|)be}\mylabel{L00175h}  \normalsize

\doendnotes{C}
\bigskip
\vfill

\clearpage

\footnotesize

\lohead{\textsc{register}}

% Definiere theindex-Environment komplett neu ohne reledmac
\makeatletter
\renewenvironment{theindex}{%
  \section*{\indexname}%
  \setlength{\parindent}{0pt}%
  \setlength{\parskip}{0pt plus 0.3pt}%
  \let\item\@idxitem
}{%
  \clearpage
}
\makeatother

\IfFileExists{\jobname-pw.ind}{\input{\jobname-pw.ind}}{}

\end{document}

      