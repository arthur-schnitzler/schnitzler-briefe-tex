%% latex-korrekturansicht-vorspann.tex
%% Vorspann für die Korrekturansicht.
%% Lädt die gemeinsame Datei latex-vorspann.tex mit gesetztem Schalter.

\newif\ifkorrekturansicht
\korrekturansichttrue

\input{../tex-inputs/latex-vorspann}


\section[Arthur Schnitzler an Stefan Großmann, 2. 12. 1913]{L02159 Arthur Schnitzler an Stefan Großmann, 2. 12. 1913}
\nopagebreak\mylabel{L02159v}
\rehead{ }\normalsize\beginnumbering\briefempfaengerindex{Grossmann, Stefan@\textsc{Großmann, Stefan}!zzzSchnitzler, Arthur@\emph{von Arthur Schnitzler}!1913-12-021@{2. 12. 1913}|(be}
\toendnotes[C]{\smallbreak\pagebreak[2]}\Standort{DLA, A:Schnitzler, HS.NZ85.1.896.}
\physDesc{Brief, Durchschlag1 Blatt, 1 Seite, 475 Zeichen
\newline{}Schreibmaschine
\newline{}Handschrift: roter Buntstift, deutsche Kurrent (\noindent{}drei Unterstreichungen)}\toendnotes[C]{\smallbreak}
\pstart
           \raggedleft{}{\pb}2. 12. 1913.\pend
           
\pstart\center{}Verehrter Herr Grossmann.\pend\vspace{0.5em}
\pstart
           Besten Dank für die freundliche Einladung des Verlags Ullstein\orgindex{Ullstein Verlag@Ullstein Verlag|pw}. Ich bin jetzt so sehr mit einer grösseren Arbeit\pwindex{Komoedie der Worte. Drei Einakter@\emph{Komödie der Worte. Drei Einakter}|pwuv} beschäftigt, dass ich in
               der nächsten Zeit kaum dazu kommen werde, etwas für die Jubiläumsnummer der Morgenpost\orgindex{Berliner Morgenpost@Berliner Morgenpost|pw} zu verfassen. Uebrigens merke ich eben,
               dass Sie das Datum des Jubiläums nicht angegeben haben. Vielleicht sagen Sie mir
               darüber noch ein Wort.\pend
           
\pstart
           Mit vorzüglicher Hochachtung{\\[\baselineskip]}Ihr sehr ergebener\pend
           \leftskip=0em{}{\vspace{1\baselineskip}}
\pstart
           Herrn Stefan Grossmann, Wien\oindex{Wien@\textbf{Wien}, \emph{A.ADM2}|pw}.\pend
           \selectlanguage{ngerman}\endnumbering\briefempfaengerindex{Grossmann, Stefan@\textsc{Großmann, Stefan}!zzzSchnitzler, Arthur@\emph{von Arthur Schnitzler}!1913-12-021@{2. 12. 1913}|)be}\mylabel{L02159h}  \normalsize

\doendnotes{C}
\bigskip
\vfill

\clearpage

\footnotesize

\lohead{\textsc{register}}

% Definiere theindex-Environment komplett neu ohne reledmac
\makeatletter
\renewenvironment{theindex}{%
  \section*{\indexname}%
  \setlength{\parindent}{0pt}%
  \setlength{\parskip}{0pt plus 0.3pt}%
  \let\item\@idxitem
}{%
  \clearpage
}
\makeatother

\IfFileExists{\jobname-pw.ind}{\input{\jobname-pw.ind}}{}

\end{document}

      