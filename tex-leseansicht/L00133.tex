%% latex-leseansicht-vorspann.tex
%% Vorspann für die Leseansicht.
%% Lädt die gemeinsame Datei latex-vorspann.tex mit nicht gesetztem Schalter.

\newif\ifkorrekturansicht
\korrekturansichtfalse

\input{../tex-inputs/latex-vorspann}


         
         \newcommand{\erwaehntePersonen}{Personen: Hermann Bahr, Wilhelm Bölsche, Robert Ehrhart-Ehrhartstein, Hugo von Hofmannsthal}
         \newcommand{\erwaehnteInstitutionen}{}
         \newcommand{\erwaehnteOrte}{Orte: Café Pfob, Café Union, Grillparzerstraße, Wien}
         \newcommand{\erwaehnteWerke}{Werke: Musotte}
               \section[Arthur Schnitzler an Hugo von Hofmannsthal, 9. 11. 1892]{ Arthur Schnitzler an Hugo von Hofmannsthal, 9. 11. 1892}\nopagebreak\mylabel{v}\rehead{ }\begin{ledgroupsized}[t]{13cm}\normalsize\beginnumbering \toendnotes[C]{\smallbreak\pagebreak[2]} \Standort{FDH, Hs-30885,26.}
\physDesc{Brief, 1 Blatt, 2 Seiten
\newline{}Handschrift: schwarze Tinte, deutsche Kurrent\newline{}Ordnung: auf der ersten Seite von
                                  Schnitzler mutmaßlich bei der Durchsicht der Korrespondenz 1929 mit Bleistift datiert: »9/11 92« }\buchAbdrucke{\weitereDrucke{1) Hugo von Hofmannsthal, Arthur Schnitzler: \emph{Briefwechsel}. Hg. Therese Nickl und Heinrich Schnitzler. Frankfurt am Main: \emph{S. Fischer} 1964, S. 30–31.} \weitereDrucke{2) Hermann Bahr, Arthur Schnitzler: \emph{Briefwechsel, Aufzeichnungen, Dokumente
                                (1891–1931)}. Hg. Kurt Ifkovits und Martin Anton Müller. Göttingen: \emph{Wallstein} 2018.} }\pstart{}{\pb}Liebſter Hugo,\pend\pstart
           zu \textsc{Musotte}\pwindex{\textcolor{red}{\textsuperscript{XXXX1 indx}}!Musotte1891@\strich\emph{Musotte} {[}1891{]}|pw}\pwindex{\textcolor{red}{\textsuperscript{XXXX1 indx}}!Musotte1891@\strich\emph{Musotte} {[}1891{]}|pw} geh ich beinahe ſicher. –\pend
           \pstart
           Wir ſoupiren alſo miteinander. –\pend
           \pstart
           Rendezvous einfach im Parterre Foyer. –\pend
           \pstart
           Herrn von \textsc{Ehrhardt}\pwindex{Ehrhart-Ehrhartstein, Robert 12.09.1870 – 11.11.1956@\textsc{Ehrhart-Ehrhartstein, Robert} (12.09.1870 – 11.11.1956), \emph{Schriftsteller, Ministerialbeamter}|pw} hab ich alles ausgerichtet. – Wiſſen Sie ſchon? Dienſtag \textcolor{gray}{{\kaufmannsund}}{ }Samſtag{ }\textsc{Cafe Pfob}\oindex{Cafe Pfob@\textbf{Café Pfob}|pw}. – Die andern Abende \textsc{Café Union}\oindex{Cafe Union@\textbf{Café Union}|pw} – \introOben{}lies \textsc{\uline{Union}}\introOben{} (\textsc{Grillparzerstraße}\oindex{Grillparzerstrasse@\textbf{Grillparzerstraße}|pw}.) –\pend
           \pstart
           {\pb}Hat Ihnen Bölſche\pwindex{Boelsche, Wilhelm 02.01.1861 – 31.08.1939@\textsc{Bölsche, Wilhelm} (02.01.1861 – 31.08.1939), \emph{Schriftsteller, Publizist}|pw} geantwortet? –\pend
           \pstart
           Was treiben Sie überhaupt? –\pend
           \pstart
           Eigentlich habe ich gehofft, Sie heuer öfters zu ſehen. Ich arbeite; bin aber
                    leider ſehr talentlos.\pend
           \pstart
           Herzlichſt der Ihre{\\[\baselineskip]}\spacefill\mbox{Arthur}\pend
           \leftskip=0em{}\pstart
           9/XI. 92\pend
           \pstart
           Grüßen Sie Bahr\pwindex{Bahr, Hermann 19.07.1863 – 15.01.1934@\textsc{Bahr, Hermann} (19.07.1863 – 15.01.1934), \emph{Schriftsteller, Kritiker}|pw}!\pend
           
         
         \endnumbering\mylabel{h}\end{ledgroupsized}  \newcommand{\dateiname}{L00133}\newcommand{\titel}{Arthur Schnitzler an Hugo von Hofmannsthal, 9. 11. 1892}\newcommand{\editorInnen}{ Martin Anton Müller und Gerd-Hermann Susen}%% latex-leseansicht-abspann.tex
%% Abspann für die Leseansicht.
%% Der Schalter \ifkorrekturansicht ist bereits durch den Vorspann gesetzt.

%% latex-abspann.tex
%% Gemeinsamer Abspann für Korrekturansicht und Leseansicht.
%% Setzt den Schalter \ifkorrekturansicht voraus (gesetzt in den
%% einbindenden Dateien latex-korrekturansicht-abspann.tex bzw.
%% latex-leseansicht-abspann.tex).
%% ---------------------------------------------------------------

\normalsize

% Das esempio-Environment wird nur in der Leseansicht benötigt
\ifkorrekturansicht\else
\newenvironment{esempio}[3]%
{
    \vspace{1.5ex}
    \rlap{\underline{#1}}
    \par
    \setlength{\parindent}{0cm}
    \nopagebreak
    \leftskip=#2cm
    \rightskip=#3cm
}
{
    \par
}
\fi

\doendnotes{C}
\bigskip
\vfill

\clearpage

\footnotesize

\ifkorrekturansicht
  \lohead{\textsc{register}}
\fi

% theindex-Environment neu definieren ohne reledmac
\makeatletter
\renewenvironment{theindex}{%
  \ifkorrekturansicht
    \section*{\indexname}%
  \else
    \subsubsection*{Index der erwähnten Entitäten}%
  \fi
  \setlength{\parindent}{0pt}%
  \setlength{\parskip}{0pt plus 0.3pt}%
  \let\item\@idxitem
}{%
  \ifkorrekturansicht\clearpage\fi
}
\makeatother

\IfFileExists{\jobname-pw.ind}{\input{\jobname-pw.ind}}{}

% Quellenangabe nur in der Leseansicht
\ifkorrekturansicht\else
% Fallback-Definitionen, falls die .tex-Datei \titel etc. nicht gesetzt hat
\providecommand{\titel}{}
\providecommand{\editorInnen}{}
\providecommand{\dateiname}{\jobname}

\vspace{3cm}

\vfill

\footnotesize
\textsc{Quelle}: \titel. Herausgegeben von {\editorInnen}. In: \emph{Arthur Schnitzler: Briefwechsel mit Autorinnen und Autoren}.
 Digitale Edition, https://schnitzler-briefe.acdh.oeaw.ac.at/{\dateiname}.html (Stand \today)
\fi

\end{document}


      