%% latex-korrekturansicht-vorspann.tex
%% Vorspann für die Korrekturansicht.
%% Lädt die gemeinsame Datei latex-vorspann.tex mit gesetztem Schalter.

\newif\ifkorrekturansicht
\korrekturansichttrue

\input{../tex-inputs/latex-vorspann}


\section[Arthur Schnitzler an Hugo von Hofmannsthal, 9. 11. 1892]{L00133 Arthur Schnitzler an Hugo von Hofmannsthal, 9. 11. 1892}
\nopagebreak\mylabel{L00133v}
\rehead{ }\normalsize\beginnumbering\briefempfaengerindex{Hofmannsthal, Hugo von@\textsc{Hofmannsthal, Hugo von}!zzzSchnitzler, Arthur@\emph{von Arthur Schnitzler}!1892-11-091@{9. 11. 1892}|(be}
\toendnotes[C]{\smallbreak\pagebreak[2]}\Standort{FDH, Hs-30885,26.}
\physDesc{Brief, 1 Blatt, 2 Seiten, 480 Zeichen
\newline{}Handschrift: schwarze Tinte, deutsche Kurrent
\newline{}Ordnung: mit Bleistift auf der ersten Seite von Schnitzler mutmaßlich bei der
                                 Durchsicht der Korrespondenz 1929
                                 datiert: »9/11 92« }
\buchAbdrucke{\weitereDrucke{1) Hugo von Hofmannsthal, Arthur Schnitzler: \emph{Briefwechsel}. Frankfurt am Main: \emph{S. Fischer} 1964, S. 30–31.} \weitereDrucke{2) Hermann Bahr, Arthur Schnitzler: \emph{Briefwechsel, Aufzeichnungen, Dokumente (1891–1931)}. Göttingen: \emph{Wallstein} 2018.} }
\pstart{}{\pb}Liebſter Hugo,\pend\vspace{0.5em}
\pstart
           zu \textsc{Musotte}\pwindex{Musotte@\emph{Musotte}|pw} geh ich beinahe ſicher. –\pend
           
\pstart
           Wir ſoupiren alſo miteinander. –\pend
           
\pstart
           Rendezvous einfach im Parterre Foyer. –\pend
           
\pstart
           Herrn von \textsc{Ehrhardt}\pwindex{Ehrhart-Ehrhartstein, Robert 12.09.1870 – 11.11.1956@\textsc{Ehrhart-Ehrhartstein, Robert} (12.09.1870 – 11.11.1956), \emph{Schriftsteller/Schriftstellerin, Ministerialbeamter/Ministerialbeamte}|pw} hab ich alles ausgerichtet. – Wiſſen Sie ſchon? Dienſtag \textcolor{gray}{{\kaufmannsund}}{ }Samſtag{ }\textsc{Cafe Pfob}\oindex{Cafe Pfob@\textbf{Café Pfob}, \emph{Kaffeehaus (K.KAF)}|pw}. – Die andern Abende \textsc{Café Union}\oindex{Cafe Union@\textbf{Café Union}, \emph{Kaffeehaus (K.KAF)}|pw} – \introOben{}lies \textsc{\uline{Union}}\introOben{} (\textsc{Grillparzerstraße}\oindex{Grillparzerstrasse@\textbf{Grillparzerstraße}, \emph{R.ST}|pw}.) –\pend
           
\pstart
           {\pb}Hat Ihnen Bölſche\pwindex{Boelsche, Wilhelm 02.01.1861 – 31.08.1939@\textsc{Bölsche, Wilhelm} (02.01.1861 – 31.08.1939), \emph{Schriftsteller/Schriftstellerin, Publizist/Publizistin}|pw}
               geantwortet? –\pend
           
\pstart
           Was treiben Sie überhaupt? –\pend
           
\pstart
           Eigentlich habe ich gehofft, Sie heuer öfters zu ſehen. Ich arbeite; bin aber leider
               ſehr talentlos.\pend
           
\pstart
           Herzlichſt der Ihre{\\[\baselineskip]}\spacefill\mbox{Arthur}\pend
           \leftskip=0em{}
\pstart
           9/XI. 92\pend
           
\pstart
           Grüßen Sie Bahr\pwindex{Bahr, Hermann 19.07.1863 – 15.01.1934@\textsc{Bahr, Hermann} (19.07.1863 – 15.01.1934), \emph{Schriftsteller/Schriftstellerin, Kritiker/Kritikerin}|pw}!\pend
           \selectlanguage{ngerman}\endnumbering\briefempfaengerindex{Hofmannsthal, Hugo von@\textsc{Hofmannsthal, Hugo von}!zzzSchnitzler, Arthur@\emph{von Arthur Schnitzler}!1892-11-091@{9. 11. 1892}|)be}\mylabel{L00133h}  \normalsize

\doendnotes{C}
\bigskip
\vfill

\clearpage

\footnotesize

\lohead{\textsc{register}}

% Definiere theindex-Environment komplett neu ohne reledmac
\makeatletter
\renewenvironment{theindex}{%
  \section*{\indexname}%
  \setlength{\parindent}{0pt}%
  \setlength{\parskip}{0pt plus 0.3pt}%
  \let\item\@idxitem
}{%
  \clearpage
}
\makeatother

\IfFileExists{\jobname-pw.ind}{\input{\jobname-pw.ind}}{}

\end{document}

      