%% latex-leseansicht-vorspann.tex
%% Vorspann für die Leseansicht.
%% Lädt die gemeinsame Datei latex-vorspann.tex mit nicht gesetztem Schalter.

\newif\ifkorrekturansicht
\korrekturansichtfalse

\input{../tex-inputs/latex-vorspann}


\section[Hugo von Hofmannsthal an Arthur Schnitzler, 6. 8. {[}1898{]}]{L00831 Hugo von Hofmannsthal an Arthur Schnitzler, 6. 8. [1898]}
\nopagebreak\mylabel{L00831v}
\rehead{ }\normalsize\beginnumbering\briefempfaengerindex{Schnitzler, Arthur@\textsc{Schnitzler, Arthur}!zzzHofmannsthal, Hugo von@\emph{von Hugo von Hofmannsthal}!1898-08-061@{6. 8. [1898]}|(be}
\toendnotes[C]{\smallbreak\pagebreak[2]}
\correspDesc{Versand  durch Hugo von Hofmannsthal am 6. 8. [1898] in Hinterbrühl
\newline{}Erhalt  durch Arthur Schnitzler im Zeitraum [7. 8. 1898
                  – 11. 8. 1898?] in Tegernsee}\toendnotes[C]{\smallbreak}
\Standort{CUL, Schnitzler, B 43.}
\physDesc{Brief, 1 Blatt, 4 Seiten, 944 Zeichen
\newline{}Handschrift: schwarze Tinte, deutsche Kurrent
\newline{}Schnitzler: mit Bleistift die Jahreszahl ergänzt: »98« 
\newline{}Ordnung: 1) mit Bleistift von unbekannter Hand nummeriert: »\strikeout{133}«  2) mit Bleistift von unbekannter Hand nummeriert:
                                    »119a«}
\buchAbdrucke{\weitereDrucke{Hugo von Hofmannsthal, Arthur Schnitzler: \emph{Briefwechsel}. Herausgegeben von Therese Nickl und Heinrich Schnitzler. Frankfurt am Main: \emph{S. Fischer} 1964, S. 109.} }\toendnotes[C]{\smallbreak}
\pstart
           \raggedleft{}{\pb}Brühl\oindex{Hinterbrühl@\textbf{Hinterbrühl}, \emph{Hauptstadt}|pw}{ }6\textsc{\textsuperscript{ten}} VIII.\pend
           
\pstart{}mein lieber Arthur\pend\vspace{0.5em}
\pstart
           auf meinen letzten Brief \introOben{}nach Tegernſee\oindex{Tegernsee@\textbf{Tegernsee}|pw}\introOben{} bin ich noch ohne Antwort, aber gar nicht beunruhigend, da ja Ihr letzter die
               Verſicherung enthielt, daſs Ihnen unſer Rendezvous 10–15
               recht iſt. Nun fange ich an mich{ }ſchon{ }ſehr nach dem Arbeiten zu{ }ſehnen und mit den
               Tagen geizig zu{ }ſein.\pend
           
\pstart
           {\pb}Ich möchte daher{ }ſchon
                  Mittwoch d. 10\textsc{\textsuperscript{ten}} vormittag (circa 10\textsc{\textsuperscript{h}} glaub ich) von Zell am See\oindex{Zell am See@\textbf{Zell am See}, \emph{Hauptstadt}|pw} her in Innsbruck\oindex{Innsbruck@\textbf{Innsbruck}, \emph{Verwaltungsgebiet}|pw} anko{\geminationm}en.
               Werden Sie da{ }ſchon dort{ }ſein? und am Bahnhof\oindex{Innsbruck Hauptbahnhof@\textbf{Innsbruck Hauptbahnhof}, \emph{Bahnhofsgebäude}|pwv} oder wo treffen wir uns? Ich nehme an daſs wir am{ }ſelben Tag weiterfahren gegen Bregenz\oindex{Bregenz@\textbf{Bregenz}|pw}. Sollte
               es practiſch{ }ſein mit demſelben {\pb}Zug weiterzufahren, in dem ich anko{\geminationm}e,{ }ſo müſsten Sie
               mich natürlich auch das wiſſen laſſen. Ich reiſe Montag 8\textsc{\textsuperscript{ten}} von Wien\oindex{Wien@\textbf{Wien}, \emph{Verwaltungsgebiet}|pw} abends ab, bin 9\textsc{\textsuperscript{ten}}{ }früh bis 9\textsc{\textsuperscript{ten}}{ }abends{ }Bad Fuſch\oindex{Bad Fusch@\textbf{Bad Fusch}|pw}. Entweder{ }ſchreiben Sie alſo umgehend
               in die Fuſch\oindex{Bad Fusch@\textbf{Bad Fusch}|pw} oder was mir noch lieber wäre {\pb}telegrafieren in die Saleſianergaſſe\oindex{Wien@\textbf{Wien}!III., Landstraße@\textbf{III., Landstraße}!Salesianergasse 12@\textbf{Salesianergasse 12}, \emph{Wohngebäude}|pw} (am Montag) das
               Dringendſte, ob Sie Mittwoch{ }Innsbruck\oindex{Innsbruck@\textbf{Innsbruck}, \emph{Verwaltungsgebiet}|pw} und wo.\pend
           
\pstart
           Von Herzen Ihr{\\[\baselineskip]}\spacefill\mbox{Hugo.}\pend
           \leftskip=0em{}\selectlanguage{ngerman}\endnumbering\briefempfaengerindex{Schnitzler, Arthur@\textsc{Schnitzler, Arthur}!zzzHofmannsthal, Hugo von@\emph{von Hugo von Hofmannsthal}!1898-08-061@{6. 8. [1898]}|)be}\mylabel{L00831h}  \newcommand{\dateiname}{L00831}\newcommand{\titel}{Hugo von Hofmannsthal an Arthur Schnitzler, 6. 8. [1898]}\newcommand{\editorInnen}{Martin Anton Müller und Gerd-Hermann Susen}%% latex-leseansicht-abspann.tex
%% Abspann für die Leseansicht.
%% Der Schalter \ifkorrekturansicht ist bereits durch den Vorspann gesetzt.

%% latex-abspann.tex
%% Gemeinsamer Abspann für Korrekturansicht und Leseansicht.
%% Setzt den Schalter \ifkorrekturansicht voraus (gesetzt in den
%% einbindenden Dateien latex-korrekturansicht-abspann.tex bzw.
%% latex-leseansicht-abspann.tex).
%% ---------------------------------------------------------------

\normalsize

% Das esempio-Environment wird nur in der Leseansicht benötigt
\ifkorrekturansicht\else
\newenvironment{esempio}[3]%
{
    \vspace{1.5ex}
    \rlap{\underline{#1}}
    \par
    \setlength{\parindent}{0cm}
    \nopagebreak
    \leftskip=#2cm
    \rightskip=#3cm
}
{
    \par
}
\fi

\doendnotes{C}
\bigskip
\vfill

\clearpage

\footnotesize

\ifkorrekturansicht
  \lohead{\textsc{register}}
\fi

% theindex-Environment neu definieren ohne reledmac
\makeatletter
\renewenvironment{theindex}{%
  \ifkorrekturansicht
    \section*{\indexname}%
  \else
    \subsubsection*{Index der erwähnten Entitäten}%
  \fi
  \setlength{\parindent}{0pt}%
  \setlength{\parskip}{0pt plus 0.3pt}%
  \let\item\@idxitem
}{%
  \ifkorrekturansicht\clearpage\fi
}
\makeatother

\IfFileExists{\jobname-pw.ind}{\input{\jobname-pw.ind}}{}

% Quellenangabe nur in der Leseansicht
\ifkorrekturansicht\else
% Fallback-Definitionen, falls die .tex-Datei \titel etc. nicht gesetzt hat
\providecommand{\titel}{}
\providecommand{\editorInnen}{}
\providecommand{\dateiname}{\jobname}

\vspace{3cm}

\vfill

\footnotesize
\textsc{Quelle}: \titel. Herausgegeben von {\editorInnen}. In: \emph{Arthur Schnitzler: Briefwechsel mit Autorinnen und Autoren}.
 Digitale Edition, https://schnitzler-briefe.acdh.oeaw.ac.at/{\dateiname}.html (Stand \today)
\fi

\end{document}


