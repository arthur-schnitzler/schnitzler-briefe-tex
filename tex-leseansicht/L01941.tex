%% latex-korrekturansicht-vorspann.tex
%% Vorspann für die Korrekturansicht.
%% Lädt die gemeinsame Datei latex-vorspann.tex mit gesetztem Schalter.

\newif\ifkorrekturansicht
\korrekturansichttrue

\input{../tex-inputs/latex-vorspann}


\section[Arthur Schnitzler an Richard Beer-Hofmann, 29. 6. 1910]{L01941 Arthur Schnitzler an Richard Beer-Hofmann, 29. 6. 1910}
\nopagebreak\mylabel{L01941v}
\rehead{ }\normalsize\beginnumbering\briefempfaengerindex{Beer-Hofmann, Richard@\textsc{Beer-Hofmann, Richard}!zzzSchnitzler, Arthur@\emph{von Arthur Schnitzler}!1910-06-291@{29. 6. 1910}|(be}
\toendnotes[C]{\smallbreak\pagebreak[2]}\Standort{YCGL, MSS 31.}
\physDesc{Kartenbrief, 315 Zeichen
\newline{}Handschrift: Bleistift, deutsche Kurrent
\newline{}Versand: 1) Stempel: »\nobreak{}29. VI. 10, 6\nobreak{}«.   2) Stempel: »\nobreak{}\oindex{Bad Ischl@\textbf{Bad Ischl}, \emph{P.PPL}|pwk}Bad Ischl 1, \textcolor{gray}{3}0. \textcolor{gray}{VI.} 10\nobreak{}«.  3) Weil dem Postbediensteten in Ischl\oindex{Bad Ischl@\textbf{Bad Ischl}, \emph{P.PPL}|pw} die Adresse nicht geläufig war, strich dieser mit
                                 Bleistift diese Ortsangabe durch und vermerkte: »\textsc{retour}« und »\textsc{wenden}« (zweiteres bezieht
                                 sich auf die auf der Rückseite angebrachte Absenderangabe) und das
                                 Korrespondenzstück ging wieder nach Wien\oindex{Wien@\textbf{Wien}, \emph{A.ADM2}|pw}, von wo es neuerlich gesandt wurde und am
                                    6. 7. 1910 den Empfänger erreichte.}
\buchAbdrucke{\weitereDrucke{Arthur Schnitzler, Richard Beer-Hofmann: \emph{Briefwechsel 1891–1931}. Wien, Zürich: \emph{Europaverlag} 1992, S. 208.} }\pstart{}{\pb}Abſ.:\pend{}\pstart{}\textsc{Schnitzler}Wien\oindex{Wien@\textbf{Wien}, \emph{A.ADM2}|pw}\pend{}\pstart{}\textsc{XVIII Spoettelg. 7}\oindex{Edmund-Weiss-Gasse 7@\textbf{Edmund-Weiß-Gasse 7}, \emph{Wohngebäude (K.WHS)}|pw}\pend{}{\bigskip}\pstart{}{\pb}\textsc{Herrn Dr. Richard Beer-Hofma{\geminationn}}\pend{}\pstart{}\textsc{\strikeout{Wien\oindex{Wien@\textbf{Wien}, \emph{A.ADM2}|pw}}}\pend{}\pstart{}\textsc{Ischl\oindex{Bad Ischl@\textbf{Bad Ischl}, \emph{P.PPL}|pw}}\pend{}\pstart{}\textsc{Steinfeld Nr 6\oindex{Steinfeld@\textbf{Steinfeld}, \emph{P.PPL}|pw}}.\pend{}{\bigskip}\vspace{1em}
\pstart
           \raggedleft{}{\pb}29. 6. 1910\pend
           
\pstart{}lieber Richard,\pend\vspace{0.5em}
\pstart
           würd es Ihnen Mühe machen, mir geſchwind eine Abſchrift von »\textsc{Mirjams Wiegenlied}\pwindex{Schlaflied fuer Mirjam@\emph{Schlaflied für Mirjam}|pw}« zu ſenden, um das ich von Paul Marx\pwindex{Marx, Paul 21.07.1879 – 1956-10-30@\textsc{Marx, Paul} (21.07.1879 – 1956-10-30), \emph{Regisseur/Regisseurin, Schauspieler/Schauspielerin}|pw}
               dringend gebeten wurde u das ich nicht beſitze?\pend
           
\pstart
           Hoffe Sie wohl am Ort!{\\}Herzlichſt\pend
           \pstart Ihr \spacefill\mbox{A.}\pend{}\selectlanguage{ngerman}\endnumbering\briefempfaengerindex{Beer-Hofmann, Richard@\textsc{Beer-Hofmann, Richard}!zzzSchnitzler, Arthur@\emph{von Arthur Schnitzler}!1910-06-291@{29. 6. 1910}|)be}\mylabel{L01941h}  \normalsize

\doendnotes{C}
\bigskip
\vfill

\clearpage

\footnotesize

\lohead{\textsc{register}}

% Definiere theindex-Environment komplett neu ohne reledmac
\makeatletter
\renewenvironment{theindex}{%
  \section*{\indexname}%
  \setlength{\parindent}{0pt}%
  \setlength{\parskip}{0pt plus 0.3pt}%
  \let\item\@idxitem
}{%
  \clearpage
}
\makeatother

\IfFileExists{\jobname-pw.ind}{\input{\jobname-pw.ind}}{}

\end{document}

      