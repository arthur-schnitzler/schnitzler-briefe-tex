%% latex-leseansicht-vorspann.tex
%% Vorspann für die Leseansicht.
%% Lädt die gemeinsame Datei latex-vorspann.tex mit nicht gesetztem Schalter.

\newif\ifkorrekturansicht
\korrekturansichtfalse

\input{../tex-inputs/latex-vorspann}


               \section[Arthur Schnitzler an Richard Beer-Hofmann, 29. 6. 1910]{ Arthur Schnitzler an Richard Beer-Hofmann, 29. 6. 1910}\nopagebreak\mylabel{v}\rehead{ }\begin{ledgroupsized}[t]{13cm}\normalsize\beginnumbering\briefempfaengerindex{Beer-Hofmann, Richard@\textsc{Beer-Hofmann, Richard}!zzzSchnitzler, Arthur@\emph{von Arthur Schnitzler}!1910-06-291@{29. 6. 1910}|(be} \toendnotes[C]{\smallbreak\pagebreak[2]} \Standort{YCGL, MSS 31.}
\physDesc{Kartenbrief
\newline{}Handschrift: Bleistift, deutsche Kurrent\newline{}Versand: 1) Stempel: »\nobreak{}29. VI. 10, 6\nobreak{}«.  2) Stempel: »\nobreak{}\oindex{Bad Ischl@\textbf{Bad Ischl}|pwk}Bad Ischl 1, \textcolor{gray}{3}0. \textcolor{gray}{VI.} 10\nobreak{}«. 3) Weil dem Postbediensteten in Ischl
                              \oindex{Bad Ischl@\textbf{Bad Ischl}|pw}die Adresse nicht geläufig war, strich dieser mit Bleistift diese Ortsangabe
                                 durch und vermerkte: »\textsc{retour}« und »\textsc{wenden}« (zweiteres
                                 bezieht sich auf die auf der Rückseite angebrachte Absenderangabe)
                                 und das Korrespondenzstück ging wieder nach Wien\oindex{Wien@\textbf{Wien}|pw}, von wo es neuerlich gesandt wurde und am
                                    6. 7. 1910 den Empfänger erreichte.}\buchAbdrucke{\weitereDrucke{Arthur Schnitzler, Richard Beer-Hofmann: \emph{Briefwechsel 1891–1931}. Hg. Konstanze Fliedl. Wien, Zürich: \emph{Europaverlag} 1992, S. 208.} }\pstart{}{\pb}Abſ.:\pend{}\pstart{}\textsc{Schnitzler}Wien\oindex{Wien@\textbf{Wien}|pw}\pend{}\pstart{}\textsc{XVIII Spoettelg. 7}\oindex{Edmund-Weiss-Gasse@\textbf{Edmund-Weiß-Gasse}|pw}\pend{}{\bigskip}\pstart{}{\pb}\textsc{Herrn Dr. Richard Beer-Hofma{\geminationn}}\pend{}\pstart{}\textsc{\strikeout{Wien\oindex{Wien@\textbf{Wien}|pw}}}\pend{}\pstart{}\textsc{Ischl\oindex{Bad Ischl@\textbf{Bad Ischl}|pw}}\pend{}\pstart{}\textsc{Steinfeld Nr 6\oindex{Steinfeld@\textbf{Steinfeld}|pw}}.\pend{}{\bigskip}\pstart
           \raggedleft{}{\pb}29. 6. 1910\pend
           \pstart{}lieber Richard,\pend\pstart
           würd es Ihnen Mühe machen, mir geſchwind eine Abſchrift von »\textsc{Mirjams Wiegenlied}\pwindex{Beer-Hofmann, Richard 11.07.1866 – 26.09.1945@\textsc{Beer-Hofmann, Richard} (11.07.1866 – 26.09.1945), \emph{Schriftsteller}!Schlaflied fuer Mirjam15.11.1898 – 15.11.1898@\strich\emph{Schlaflied für Mirjam} {[}15.11.1898 – 15.11.1898{]}|pw}« zu ſenden, um das ich von Paul Marx\pwindex{Marx, Paul 04.06.1861 – 27.11.1919@\textsc{Marx, Paul} (04.06.1861 – 27.11.1919), \emph{Journalist, Kritiker}|pw}
               dringend gebeten wurde u das ich nicht beſitze?\pend
           \pstart
           Hoffe Sie wohl am Ort!{\\}Herzlichſt\pend
           \pstart Ihr \spacefill\mbox{A.}\pend{}          \endnumbering\briefempfaengerindex{Beer-Hofmann, Richard@\textsc{Beer-Hofmann, Richard}!zzzSchnitzler, Arthur@\emph{von Arthur Schnitzler}!1910-06-291@{29. 6. 1910}|)be}\mylabel{h}\end{ledgroupsized}  \newcommand{\dateiname}{L01941}\newcommand{\titel}{Arthur Schnitzler an Richard Beer-Hofmann, 29. 6. 1910}\newcommand{\editorInnen}{Martin Anton Müller und Gerd-Hermann Susen}%% latex-leseansicht-abspann.tex
%% Abspann für die Leseansicht.
%% Der Schalter \ifkorrekturansicht ist bereits durch den Vorspann gesetzt.

%% latex-abspann.tex
%% Gemeinsamer Abspann für Korrekturansicht und Leseansicht.
%% Setzt den Schalter \ifkorrekturansicht voraus (gesetzt in den
%% einbindenden Dateien latex-korrekturansicht-abspann.tex bzw.
%% latex-leseansicht-abspann.tex).
%% ---------------------------------------------------------------

\normalsize

% Das esempio-Environment wird nur in der Leseansicht benötigt
\ifkorrekturansicht\else
\newenvironment{esempio}[3]%
{
    \vspace{1.5ex}
    \rlap{\underline{#1}}
    \par
    \setlength{\parindent}{0cm}
    \nopagebreak
    \leftskip=#2cm
    \rightskip=#3cm
}
{
    \par
}
\fi

\doendnotes{C}
\bigskip
\vfill

\clearpage

\footnotesize

\ifkorrekturansicht
  \lohead{\textsc{register}}
\fi

% theindex-Environment neu definieren ohne reledmac
\makeatletter
\renewenvironment{theindex}{%
  \ifkorrekturansicht
    \section*{\indexname}%
  \else
    \subsubsection*{Index der erwähnten Entitäten}%
  \fi
  \setlength{\parindent}{0pt}%
  \setlength{\parskip}{0pt plus 0.3pt}%
  \let\item\@idxitem
}{%
  \ifkorrekturansicht\clearpage\fi
}
\makeatother

\IfFileExists{\jobname-pw.ind}{\input{\jobname-pw.ind}}{}

% Quellenangabe nur in der Leseansicht
\ifkorrekturansicht\else
% Fallback-Definitionen, falls die .tex-Datei \titel etc. nicht gesetzt hat
\providecommand{\titel}{}
\providecommand{\editorInnen}{}
\providecommand{\dateiname}{\jobname}

\vspace{3cm}

\vfill

\footnotesize
\textsc{Quelle}: \titel. Herausgegeben von {\editorInnen}. In: \emph{Arthur Schnitzler: Briefwechsel mit Autorinnen und Autoren}.
 Digitale Edition, https://schnitzler-briefe.acdh.oeaw.ac.at/{\dateiname}.html (Stand \today)
\fi

\end{document}


      