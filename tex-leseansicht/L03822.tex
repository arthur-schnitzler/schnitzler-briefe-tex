%% latex-leseansicht-vorspann.tex
%% Vorspann für die Leseansicht.
%% Lädt die gemeinsame Datei latex-vorspann.tex mit nicht gesetztem Schalter.

\newif\ifkorrekturansicht
\korrekturansichtfalse

\input{../tex-inputs/latex-vorspann}


\section[Theodor Herzl an Arthur Schnitzler, 2. 10. 1886]{L03822 Theodor Herzl an Arthur Schnitzler, 2. 10. 1886}
\nopagebreak\mylabel{L03822v}
\rehead{ }\normalsize\beginnumbering\briefempfaengerindex{Schnitzler, Arthur@\textsc{Schnitzler, Arthur}!zzzHerzl, Theodor@\emph{von Theodor Herzl}!1886-10-021@{2. 10. 1886}|(be}
\toendnotes[C]{\smallbreak\pagebreak[2]}
\correspDesc{Versand  durch Theodor Herzl am 2. 10. 1886 in Wien
\newline{}Erhalt  durch Arthur Schnitzler im Zeitraum [3. 10. 1886
                  – 7. 10. 1886?] in Wien}\toendnotes[C]{\smallbreak}
\Standort{CUL, Schnitzler, B 39.}
\physDesc{Brief, 1 Blatt, 1 Seite, 443 Zeichen
\newline{}Handschrift: schwarze Tinte, lateinische Kurrent
\newline{}Ordnung: mit Bleistift von unbekannter Hand nummeriert: »3« }
\buchAbdrucke{\weitereDrucke{Theodor Herzl: \emph{Briefe und
                        autobiographische Notizen 1866–1895}. Bearbeitet von Johannes Wachten in Zusammenarbeit mit Chaya Harel, Daisy Tycho und Manfred Winkler. Berlin, Frankfurt am Main, Wien: \emph{Propyläen} 1983, S. 235 (Briefe und Tagebücher. Herausgegeben von Alex Bein, Hermann Greive, Moshe Schaerf, Julius H. Schoeps und Johannes Wachten, 1).} }\toendnotes[C]{\smallbreak}
\pstart{}{\pb}Lieber Selbereiner!\pend\vspace{0.5em}
\pstart
           Leider hatte ich die Karte schon zurückgegeben. Zugleich – an Sie denkend – hatte ich
               jedoch gefragt, wie es mit dem Premièreschein stehe. Sie präsentiren ihn ganz einfach
               am \uline{Vortage} der Première – so diesmal wie künftig –
               bis inclus. 12 Uhr Mittags an der Tageskasse\oindex{Wien@\textbf{Wien}!I., Innere Stadt@\textbf{I., Innere Stadt}!Burgtheater [Altes Burgtheater]@\textbf{Burgtheater [Altes Burgtheater]}, \emph{Theater}|pwv}, und erhalten das Billet zu \label{K_L03822-1v}\edtext{Maria und so weiter\pwindex{Lindau, Paul 3.\,6.\,1839 Magdeburg – 31.\,1.\,1919 Berlin@\textsc{Lindau, Paul} (3.\,6.\,1839 Magdeburg – 31.\,1.\,1919 Berlin), \emph{Schriftsteller, Kritiker, Theaterleiter}!Maria und Magdalena@\strich\emph{Maria und Magdalena}|pwv}}{\lemma{\textnormal{\emph{Maria und so weiter}}}\Cendnote{\textnormal{Schnitzler besuchte die Theaterpremiere von \emph{Maria und
                        Magdalena}\pwindex{Lindau, Paul 3.\,6.\,1839 Magdeburg – 31.\,1.\,1919 Berlin@\textsc{Lindau, Paul} (3.\,6.\,1839 Magdeburg – 31.\,1.\,1919 Berlin), \emph{Schriftsteller, Kritiker, Theaterleiter}!Maria und Magdalena@\strich\emph{Maria und Magdalena}|pwk} von Paul Lindau\pwindex{Lindau, Paul 3.\,6.\,1839 Magdeburg – 31.\,1.\,1919 Berlin@\textsc{Lindau, Paul} (3.\,6.\,1839 Magdeburg – 31.\,1.\,1919 Berlin), \emph{Schriftsteller, Kritiker, Theaterleiter}|pwk}\eventindex{Burgtheater [Altes Burgtheater]@\textbf{Burgtheater [Altes Burgtheater]}!Premiere von Maria und Magdalena, 5.10.1886@Premiere von Maria und Magdalena, 5.10.1886|pwk}, die am 5. 10. 1886 am \emph{Burgtheater}\orgindex{Burgtheater@Burgtheater|pwk} stattfand, A. S.: \emph{Kulturveranstaltungen}, 5. 10. 1886. Der Theaterzettel vom 2. 10. 1886 gibt
                  Auskunft, dass sie ursprünglich für diesen Tag angesetzt war, aber aufgrund der
                  Unpässlichkeit der Darstellerin Josephine
                     Wessely\pwindex{Wessely, Josefine 8.\,3.\,1860 Wien – 12.\,8.\,1887@\textsc{Wessely, Josefine} (8.\,3.\,1860 Wien – 12.\,8.\,1887), \emph{Schauspielerin, Sängerin}|pwk} auf den 5. 10. 1886 verschoben wurde. Das könnte der
                  Grund dafür sein, dass Herzl\pwindex{Herzl, Theodor 2.\,5.\,1860 Budapest – 3.\,7.\,1904 Edlach@\textsc{Herzl, Theodor} (2.\,5.\,1860 Budapest – 3.\,7.\,1904 Edlach), \emph{Schriftsteller, Journalist}|pwk}, der vom 3. oder 4.  bis zum 21. 10. 1886 nach Berlin\oindex{Berlin@\textbf{Berlin}, \emph{Hauptstadt}|pwk} reiste, den
                  Premierenschein zur Verfügung stellte.}}}\label{K_L03822-1}.\pend
           
\pstart
           Mögen sie Ihnen leicht werden!\pend
           
\pstart
           Ich bleibe mit cordialem Gruss{\\[\baselineskip]} Ihr aufrichtig ergebener{\\[\baselineskip]}\spacefill\mbox{Herzl}\pend
           \leftskip=0em{}
\pstart
           Zelinkag. 11\oindex{Wien@\textbf{Wien}!I., Innere Stadt@\textbf{I., Innere Stadt}!Zelinkagasse 11@\textbf{Zelinkagasse 11}, \emph{Wohngebäude}|pw}, 2 Octob 86\pend
           \selectlanguage{ngerman}\endnumbering\briefempfaengerindex{Schnitzler, Arthur@\textsc{Schnitzler, Arthur}!zzzHerzl, Theodor@\emph{von Theodor Herzl}!1886-10-021@{2. 10. 1886}|)be}\mylabel{L03822h}
\begin{anhang}
\end{anhang}\newcommand{\dateiname}{L03822}\newcommand{\titel}{Theodor Herzl an Arthur Schnitzler, 2. 10. 1886}\newcommand{\editorInnen}{Selma Jahnke und Martin Anton Müller}%% latex-leseansicht-abspann.tex
%% Abspann für die Leseansicht.
%% Der Schalter \ifkorrekturansicht ist bereits durch den Vorspann gesetzt.

%% latex-abspann.tex
%% Gemeinsamer Abspann für Korrekturansicht und Leseansicht.
%% Setzt den Schalter \ifkorrekturansicht voraus (gesetzt in den
%% einbindenden Dateien latex-korrekturansicht-abspann.tex bzw.
%% latex-leseansicht-abspann.tex).
%% ---------------------------------------------------------------

\normalsize

% Das esempio-Environment wird nur in der Leseansicht benötigt
\ifkorrekturansicht\else
\newenvironment{esempio}[3]%
{
    \vspace{1.5ex}
    \rlap{\underline{#1}}
    \par
    \setlength{\parindent}{0cm}
    \nopagebreak
    \leftskip=#2cm
    \rightskip=#3cm
}
{
    \par
}
\fi

\doendnotes{C}
\bigskip
\vfill

\clearpage

\footnotesize

\ifkorrekturansicht
  \lohead{\textsc{register}}
\fi

% theindex-Environment neu definieren ohne reledmac
\makeatletter
\renewenvironment{theindex}{%
  \ifkorrekturansicht
    \section*{\indexname}%
  \else
    \subsubsection*{Index der erwähnten Entitäten}%
  \fi
  \setlength{\parindent}{0pt}%
  \setlength{\parskip}{0pt plus 0.3pt}%
  \let\item\@idxitem
}{%
  \ifkorrekturansicht\clearpage\fi
}
\makeatother

\IfFileExists{\jobname-pw.ind}{\input{\jobname-pw.ind}}{}

% Quellenangabe nur in der Leseansicht
\ifkorrekturansicht\else
% Fallback-Definitionen, falls die .tex-Datei \titel etc. nicht gesetzt hat
\providecommand{\titel}{}
\providecommand{\editorInnen}{}
\providecommand{\dateiname}{\jobname}

\vspace{3cm}

\vfill

\footnotesize
\textsc{Quelle}: \titel. Herausgegeben von {\editorInnen}. In: \emph{Arthur Schnitzler: Briefwechsel mit Autorinnen und Autoren}.
 Digitale Edition, https://schnitzler-briefe.acdh.oeaw.ac.at/{\dateiname}.html (Stand \today)
\fi

\end{document}


