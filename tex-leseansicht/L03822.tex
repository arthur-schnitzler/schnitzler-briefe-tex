%% latex-korrekturansicht-vorspann.tex
%% Vorspann für die Korrekturansicht.
%% Lädt die gemeinsame Datei latex-vorspann.tex mit gesetztem Schalter.

\newif\ifkorrekturansicht
\korrekturansichttrue

\input{../tex-inputs/latex-vorspann}


\section[Theodor Herzl an Arthur Schnitzler, 2. 10. 1886]{L03822 Theodor Herzl an Arthur Schnitzler, 2. 10. 1886}
\nopagebreak\mylabel{L03822v}
\rehead{ }\normalsize\beginnumbering\briefempfaengerindex{Schnitzler, Arthur@\textsc{Schnitzler, Arthur}!zzzHerzl, Theodor@\emph{von Theodor Herzl}!1886-10-021@{2. 10. 1886}|(be}
\toendnotes[C]{\smallbreak\pagebreak[2]}\Standort{CUL, Schnitzler, B 39.}
\physDesc{Brief, 1 Blatt, 1 Seite, 443 Zeichen
\newline{}Handschrift: schwarze Tinte, lateinische Kurrent
\newline{}Ordnung: mit Bleistift von unbekannter Hand nummeriert: »3« }\toendnotes[C]{\smallbreak}
\pstart{}{\pb}Lieber Selbereiner!\pend\vspace{0.5em}
\pstart
           Leider hatte ich die Karte schon zurückgegeben. Zugleich – an Sie denkend – hatte ich
               jedoch gefragt, wie es mit dem Premièreschein stehe. Sie präsentiren ihn ganz einfach
               am \uline{Vortage} der Première – so diesmal wie künftig –
               bis inclus. 12 Uhr Mittags an der Tageskasse\oindex{Burgtheater [Altes Burgtheater]@\textbf{Burgtheater [Altes Burgtheater]}, \emph{Theater (K.THE)}|pwv}, und erhalten das Billet zu \label{K_L03822-1v}\edtext{Maria und so weiter\pwindex{Maria und Magdalena@\emph{Maria und Magdalena}|pwv}}{\lemma{\textnormal{\emph{Maria und so weiter}}}\Cendnote{\textnormal{Schnitzler besuchte die \emph{Theaterpremiere von \emph{Maria und
                        Magdalena}\pwindex{Maria und Magdalena@\emph{Maria und Magdalena}|pwk} von Paul Lindau\pwindex{Lindau, Paul 03.06.1839 – 31.01.1919@\textsc{Lindau, Paul} (03.06.1839 – 31.01.1919), \emph{Schriftsteller/Schriftstellerin, Kritiker/Kritikerin, Theaterleiter/Theaterleiterin}|pwk}}\eventindex{Burgtheater [Altes Burgtheater]@\textbf{Burgtheater [Altes Burgtheater]}!Premiere von Maria und Magdalena, 5.10.1886@Premiere von Maria und Magdalena, 5.10.1886|pwk}, die am 5. 10. 1886 am \emph{Burgtheater}\orgindex{Burgtheater@Burgtheater|pwk} stattfand, A. S.: \emph{Kulturveranstaltungen}, 5. 10. 1886. Der Theaterzettel vom 2. 10. 1886 gibt
                  Auskunft, dass sie ursprünglich für diesen Tag angesetzt war, aber aufgrund der
                  Unpässlichkeit der Darstellerin Josephine
                     Wessely\pwindex{Wessely, Josefine 1860-03-08 – 1887-08-12@\textsc{Wessely, Josefine} (1860-03-08 – 1887-08-12), \emph{Schauspieler/Schauspielerin, Sänger/Sängerin}|pwk} auf den 5. 10. 1886 verschoben wurde. Das könnte der
                  Grund dafür sein, dass Herzl\pwindex{Herzl, Theodor 1860-05-02 – 1904-07-03@\textsc{Herzl, Theodor} (1860-05-02 – 1904-07-03), \emph{Schriftsteller/Schriftstellerin, Journalist/Journalistin}|pwk}, der vom 3. oder 4.  bis zum 21. 10. 1886 nach Berlin\oindex{Berlin@\textbf{Berlin}, \emph{P.PPLC}|pwk} reiste, den
                  Premierenschein zur Verfügung stellte.}}}\label{K_L03822-1}. \pend
           
\pstart
           Mögen sie Ihnen leicht werden! \pend
           
\pstart
           Ich bleibe mit cordialem Gruss{\\[\baselineskip]} Ihr aufrichtig ergebener{\\[\baselineskip]}\spacefill\mbox{Herzl}\pend
           \leftskip=0em{}
\pstart
           Zelinkag. 11\oindex{Zelinkagasse 11@\textbf{Zelinkagasse 11}, \emph{Wohngebäude (K.WHS)}|pw}, 2 Octob 86\pend
           \selectlanguage{ngerman}\endnumbering\briefempfaengerindex{Schnitzler, Arthur@\textsc{Schnitzler, Arthur}!zzzHerzl, Theodor@\emph{von Theodor Herzl}!1886-10-021@{2. 10. 1886}|)be}\mylabel{L03822h}
\begin{anhang}
\end{anhang}\normalsize

\doendnotes{C}
\bigskip
\vfill

\clearpage

\footnotesize

\lohead{\textsc{register}}

% Definiere theindex-Environment komplett neu ohne reledmac
\makeatletter
\renewenvironment{theindex}{%
  \section*{\indexname}%
  \setlength{\parindent}{0pt}%
  \setlength{\parskip}{0pt plus 0.3pt}%
  \let\item\@idxitem
}{%
  \clearpage
}
\makeatother

\IfFileExists{\jobname-pw.ind}{\input{\jobname-pw.ind}}{}

\end{document}

      