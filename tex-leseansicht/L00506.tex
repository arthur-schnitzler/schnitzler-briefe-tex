%% latex-leseansicht-vorspann.tex
%% Vorspann für die Leseansicht.
%% Lädt die gemeinsame Datei latex-vorspann.tex mit nicht gesetztem Schalter.

\newif\ifkorrekturansicht
\korrekturansichtfalse

\input{../tex-inputs/latex-vorspann}


         
         \newcommand{\erwaehntePersonen}{Personen: Gésa D’Ouvenou, Karl Gutzkow, Ludwig Speidel}
         \newcommand{\erwaehnteOrte}{Orte: Café Metropol, Frankgasse, IX., Alsergrund, Irrgarten (Irrgänge), Wien, Zürich, Österreich}
         \newcommand{\erwaehnteWerke}{Werke: Burgtheater. (»Liebelei«, Schauspiel in drei Aufzügen von Arthur Schnitzler. – »Rechte der Seele«, Schauspiel in einem Act von Giuseppe Giacosa, deutsch von Otto Eisenschitz.), Neue Freie Presse, Theater- und Kunstnachrichten. [Burgtheater] [Liebelei, Rechte der Seele], Uriel Acosta. Trauerspiel in fünf Aufzügen}
               \section[Friedrich M. Fels und Jenny Nordegg an Arthur Schnitzler, 15. 10. 1895]{ Friedrich M. Fels und Jenny Nordegg an Arthur Schnitzler,
               15. 10. 1895}\nopagebreak\mylabel{v}\rehead{ }\begin{ledgroupsized}[t]{13cm}\normalsize\beginnumbering \toendnotes[C]{\smallbreak\pagebreak[2]} \Standort{DLA, A:Schnitzler, HS.NZ85.1.2956.}
\physDesc{Postkarte
\newline{}Handschrift Friedrich Michael Fels: schwarze Tinte, lateinische Kurrent\newline{}Handschrift Jenny Nordegg: schwarze Tinte\newline{}Versand: 1) Stempel: »\nobreak{}\oindex{Zuerich@\textbf{Zürich}|pwk}Zürich Bhf. Exp., 15. X. 95, 11\nobreak{}«.   2) Stempel: »\nobreak{}\oindex{IX., Alsergrund@\textbf{IX., Alsergrund}|pwk}Wien 9/3, 17 10. 95, 9.V, Bestellt\nobreak{}«. 
\newline{}Schnitzler: mit Bleistift nummeriert: »27« }\toendnotes[C]{\smallbreak}\pstart{}{\pb}Herrn Dr. med. Arthur Schnitzler\pend{}\pstart{}Schriftsteller\pend{}\pstart{}Wien\oindex{Wien@\textbf{Wien}|pw}\pend{}\pstart{}IX, Frankgaſse 1\oindex{Frankgasse@\textbf{Frankgasse}|pw}\pend{}\pstart{}Österreich\oindex{Oesterreich@\textbf{Österreich}|pw}\pend{}{\bigskip}\pstart
           \noindent{}\centering{}{\pb}\textcolor{gray}{\textbf{Grand Restaurant et Café Metropol Zurich
                        Auböck {\kaufmannsund} Ziegler Pr.\oindex{Cafe Metropol@\textbf{Café Metropol}|pw}}}\pend
           \pstart
           \noindent{}\centering{}\textcolor{gray}{\textbf{Irrgarten (Labyrinth)\oindex{Irrgarten (Irrgaenge)@\textbf{Irrgarten (Irrgänge)}|pw} D\textsuperscript{ir}{ }G. D.’Ouvenou\pwindex{DOuvenou, Gesa @\textsc{D’Ouvenou, Gésa}, \emph{Schausteller}|pw}.}}\pend
           \pstart{}Lieber Dr. Schnitzler!\pend\pstart
           Soeben lesen wir Speidel\pwindex{Speidel, Ludwig 1830-04-11 – 1906-02-03@\textsc{Speidel, Ludwig} (1830-04-11 – 1906-02-03), \emph{Journalist, Kritiker}|pw}s \label{K_L00506-8v}\edtext{Kritik\pwindex{Burgtheater. (»Liebelei«, Schauspiel in drei Aufzuegen von Arthur Schnitzler. – »Rechte der Seele«, Schauspiel in einem Act von Giuseppe Giacosa, deutsch von Otto Eisenschitz.)1895-10-13@\emph{Burgtheater. (»Liebelei«, Schauspiel in drei Aufzügen von Arthur Schnitzler. – »Rechte der Seele«, Schauspiel in einem Act von Giuseppe Giacosa, deutsch von Otto Eisenschitz.)} {[}1895-10-13{]}|pwv}}{\lemma{\textnormal{\emph{Kritik}}}\Cendnote{\textnormal{L. Sp.\pwindex{Speidel, Ludwig 1830-04-11 – 1906-02-03@\textsc{Speidel, Ludwig} (1830-04-11 – 1906-02-03), \emph{Journalist, Kritiker}|pwkv} [=Ludwig Speidel\pwindex{Speidel, Ludwig 1830-04-11 – 1906-02-03@\textsc{Speidel, Ludwig} (1830-04-11 – 1906-02-03), \emph{Journalist, Kritiker}|pwk}]: \emph{Burgtheater. (»Liebelei«, Schauspiel in drei Aufzügen von
                              Arthur Schnitzler. – »Rechte der Seele«, Schauspiel in einem Act von
                              Giuseppe Giacosa, deutsch von Otto Eisenschitz.)}\pwindex{Burgtheater. (»Liebelei«, Schauspiel in drei Aufzuegen von Arthur Schnitzler. – »Rechte der Seele«, Schauspiel in einem Act von Giuseppe Giacosa, deutsch von Otto Eisenschitz.)1895-10-13@\emph{Burgtheater. (»Liebelei«, Schauspiel in drei Aufzügen von Arthur Schnitzler. – »Rechte der Seele«, Schauspiel in einem Act von Giuseppe Giacosa, deutsch von Otto Eisenschitz.)} {[}1895-10-13{]}|pwk}. In: \emph{Neue Freie Presse}\pwindex{Neue Freie Presse1864 – 1939@\emph{Neue Freie Presse} {[}1864 – 1939{]}|pwk}, Nr. 11.184, 13. 10. 1895, Morgenblatt, S. 1–3. Eher unwahrscheinlich
                     ist, dass sie sich auf die erste Reaktion Speidel\pwindex{Speidel, Ludwig 1830-04-11 – 1906-02-03@\textsc{Speidel, Ludwig} (1830-04-11 – 1906-02-03), \emph{Journalist, Kritiker}|pwk}s,
                     dessen Nachtkritik beziehen: [Ludwig Speidel\pwindex{Speidel, Ludwig 1830-04-11 – 1906-02-03@\textsc{Speidel, Ludwig} (1830-04-11 – 1906-02-03), \emph{Journalist, Kritiker}|pwk}]: \emph{Theater- und Kunstnachrichten.
                        [Burgtheater]}\pwindex{Theater- und Kunstnachrichten. [Burgtheater] [Liebelei, Rechte der Seele]1895-10-10@\emph{Theater- und Kunstnachrichten. [Burgtheater] [Liebelei, Rechte der Seele]} {[}1895-10-10{]}|pwk}. In: \emph{Neue Freie
                        Presse}\pwindex{Neue Freie Presse1864 – 1939@\emph{Neue Freie Presse} {[}1864 – 1939{]}|pwk}, Nr. 11.181, 10. 10. 1895,
                     S. 7.}}}\label{K_L00506-8h} und freuen uns riesig über Ihren Erfolg. Fahren Sie so
               weiter, junger Ma{\geminationn}, und vergeſsen Sie im Glücke nicht
                  »\label{K_L00506-22v}\edtext{derer, die am Wege
                     sterben\pwindex{Gutzkow, Karl 17.03.1811 – 16.12.1878@\textsc{Gutzkow, Karl} (17.03.1811 – 16.12.1878), \emph{Schriftsteller}!Uriel Acosta. Trauerspiel in fuenf Aufzuegen1846@\strich\emph{Uriel Acosta. Trauerspiel in fünf Aufzügen} {[}1846{]}|pwv}}{\lemma{\textnormal{\emph{derer, … sterben}}}\Cendnote{\textnormal{Zitat aus \emph{Uriel Acosta}\pwindex{Gutzkow, Karl 17.03.1811 – 16.12.1878@\textsc{Gutzkow, Karl} (17.03.1811 – 16.12.1878), \emph{Schriftsteller}!Uriel Acosta. Trauerspiel in fuenf Aufzuegen1846@\strich\emph{Uriel Acosta. Trauerspiel in fünf Aufzügen} {[}1846{]}|pwk}
                        von Karl Gutzkow\pwindex{Gutzkow, Karl 17.03.1811 – 16.12.1878@\textsc{Gutzkow, Karl} (17.03.1811 – 16.12.1878), \emph{Schriftsteller}|pwk} (1846)}}}\label{K_L00506-22h}«.\pend
           \pstart
           Herzlichst{\\[\baselineskip]}{\\[\baselineskip]}\spacefill\mbox{{[}hs. Nordegg:{]} Jenny Nordegg}{\\[\baselineskip]}{[}hs. Fels:{]} und \spacefill\mbox{Friedr. M. Fels}\pend
           \leftskip=0em{}
         
         \endnumbering\mylabel{h}\end{ledgroupsized}  \newcommand{\dateiname}{L00506}\newcommand{\titel}{Friedrich M. Fels und Jenny Nordegg an Arthur Schnitzler, 15. 10. 1895}\newcommand{\editorInnen}{Martin Anton Müller und Gerd-Hermann Susen}%% latex-leseansicht-abspann.tex
%% Abspann für die Leseansicht.
%% Der Schalter \ifkorrekturansicht ist bereits durch den Vorspann gesetzt.

%% latex-abspann.tex
%% Gemeinsamer Abspann für Korrekturansicht und Leseansicht.
%% Setzt den Schalter \ifkorrekturansicht voraus (gesetzt in den
%% einbindenden Dateien latex-korrekturansicht-abspann.tex bzw.
%% latex-leseansicht-abspann.tex).
%% ---------------------------------------------------------------

\normalsize

% Das esempio-Environment wird nur in der Leseansicht benötigt
\ifkorrekturansicht\else
\newenvironment{esempio}[3]%
{
    \vspace{1.5ex}
    \rlap{\underline{#1}}
    \par
    \setlength{\parindent}{0cm}
    \nopagebreak
    \leftskip=#2cm
    \rightskip=#3cm
}
{
    \par
}
\fi

\doendnotes{C}
\bigskip
\vfill

\clearpage

\footnotesize

\ifkorrekturansicht
  \lohead{\textsc{register}}
\fi

% theindex-Environment neu definieren ohne reledmac
\makeatletter
\renewenvironment{theindex}{%
  \ifkorrekturansicht
    \section*{\indexname}%
  \else
    \subsubsection*{Index der erwähnten Entitäten}%
  \fi
  \setlength{\parindent}{0pt}%
  \setlength{\parskip}{0pt plus 0.3pt}%
  \let\item\@idxitem
}{%
  \ifkorrekturansicht\clearpage\fi
}
\makeatother

\IfFileExists{\jobname-pw.ind}{\input{\jobname-pw.ind}}{}

% Quellenangabe nur in der Leseansicht
\ifkorrekturansicht\else
% Fallback-Definitionen, falls die .tex-Datei \titel etc. nicht gesetzt hat
\providecommand{\titel}{}
\providecommand{\editorInnen}{}
\providecommand{\dateiname}{\jobname}

\vspace{3cm}

\vfill

\footnotesize
\textsc{Quelle}: \titel. Herausgegeben von {\editorInnen}. In: \emph{Arthur Schnitzler: Briefwechsel mit Autorinnen und Autoren}.
 Digitale Edition, https://schnitzler-briefe.acdh.oeaw.ac.at/{\dateiname}.html (Stand \today)
\fi

\end{document}


      