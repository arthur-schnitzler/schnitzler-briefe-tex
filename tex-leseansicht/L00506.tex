%% latex-korrekturansicht-vorspann.tex
%% Vorspann für die Korrekturansicht.
%% Lädt die gemeinsame Datei latex-vorspann.tex mit gesetztem Schalter.

\newif\ifkorrekturansicht
\korrekturansichttrue

\input{../tex-inputs/latex-vorspann}


\section[Friedrich M. Fels und Jenny Nordegg an Arthur Schnitzler, 15. 10. 1895]{L00506 Friedrich M. Fels und Jenny Nordegg an Arthur Schnitzler,
               15. 10. 1895}
\nopagebreak\mylabel{L00506v}
\rehead{ }\normalsize\beginnumbering\briefempfaengerindex{Schnitzler, Arthur@\textsc{Schnitzler, Arthur}!zzzNordegg, Jenny@\emph{von Jenny Nordegg}!1895-10-152@{15. 10. 1895}|(be}\briefempfaengerindex{Schnitzler, Arthur@\textsc{Schnitzler, Arthur}!zzzFels, Friedrich Michael@\emph{von Friedrich Michael Fels}!1895-10-152@{15. 10. 1895}|(be}
\toendnotes[C]{\smallbreak\pagebreak[2]}\Standort{DLA, A:Schnitzler, HS.NZ85.1.2956.}
\physDesc{Postkarte, 307 Zeichen
\newline{}Handschrift Friedrich Michael Fels: schwarze Tinte, lateinische Kurrent
\newline{}Handschrift Jenny Nordegg: schwarze Tinte
\newline{}Versand: 1) Stempel: »\nobreak{}\oindex{Zuerich@\textbf{Zürich}, \emph{P.PPLA}|pwk}Zürich Bhf. Exp., 15. X. 95, 11\nobreak{}«.   2) Stempel: »\nobreak{}\oindex{IX., Alsergrund@\textbf{IX., Alsergrund}, \emph{A.ADM3}|pwk}Wien 9/3, 17 10. 95, 9.V, Bestellt\nobreak{}«. 
\newline{}Schnitzler: mit Bleistift nummeriert: »27« }\toendnotes[C]{\smallbreak}\pstart{}{\pb}Herrn Dr. med. Arthur Schnitzler\pend{}\pstart{}Schriftsteller\pend{}\pstart{}Wien\oindex{Wien@\textbf{Wien}, \emph{A.ADM2}|pw}\pend{}\pstart{}IX, Frankgaſse 1\oindex{Frankgasse 1@\textbf{Frankgasse 1}, \emph{Wohngebäude (K.WHS)}|pw}\pend{}\pstart{}Österreich\oindex{Oesterreich@\textbf{Österreich}, \emph{A.PCLI}|pw}\pend{}{\bigskip}\vspace{1em}
\pstart
           \centering{}{\pb}\textcolor{gray}{\textbf{Grand Restaurant et Café Metropol Zurich
                        Auböck {\kaufmannsund} Ziegler Pr.\oindex{Cafe Metropol@\textbf{Café Metropol}, \emph{Kaffeehaus (K.KAF)}|pw}}}\pend
           
\pstart
           \centering{}\textcolor{gray}{\textbf{Irrgarten (Labyrinth)\oindex{Irrgarten (Irrgaenge)@\textbf{Irrgarten (Irrgänge)}, \emph{Vergnügungspark (K.VGN)}|pw} D\textsuperscript{ir}{ }G. D.’Ouvenou\pwindex{DOuvenou, Gesa @\textsc{D’Ouvenou, Gésa}, \emph{Schausteller/Schaustellerin}|pw}.}}\pend
           
\pstart{}Lieber Dr. Schnitzler!\pend\vspace{0.5em}
\pstart
           Soeben lesen wir Speidels\pwindex{Speidel, Ludwig 1830-04-11 – 1906-02-03@\textsc{Speidel, Ludwig} (1830-04-11 – 1906-02-03), \emph{Journalist/Journalistin, Kritiker/Kritikerin}|pw}{ }\label{K_L00506-1v}\edtext{Kritik\pwindex{Burgtheater. (»Liebelei«, Schauspiel in drei Aufzuegen von Arthur Schnitzler. – »Rechte der Seele«, Schauspiel in einem Act von Giuseppe Giacosa, deutsch von Otto Eisenschitz.)@\emph{Burgtheater. (»Liebelei«, Schauspiel in drei Aufzügen von Arthur Schnitzler. – »Rechte der Seele«, Schauspiel in einem Act von Giuseppe Giacosa, deutsch von Otto Eisenschitz.)}|pwv}}{\lemma{\textnormal{\emph{Kritik}}}\Cendnote{\textnormal{L. Sp.\pwindex{Speidel, Ludwig 1830-04-11 – 1906-02-03@\textsc{Speidel, Ludwig} (1830-04-11 – 1906-02-03), \emph{Journalist/Journalistin, Kritiker/Kritikerin}|pwkv} [ = Ludwig Speidel\pwindex{Speidel, Ludwig 1830-04-11 – 1906-02-03@\textsc{Speidel, Ludwig} (1830-04-11 – 1906-02-03), \emph{Journalist/Journalistin, Kritiker/Kritikerin}|pwk}]: \emph{Burgtheater. (»Liebelei«, Schauspiel in drei Aufzügen von
                        Arthur Schnitzler. – »Rechte der Seele«, Schauspiel in einem Act von
                        Giuseppe Giacosa, deutsch von Otto Eisenschitz)}\pwindex{Burgtheater. (»Liebelei«, Schauspiel in drei Aufzuegen von Arthur Schnitzler. – »Rechte der Seele«, Schauspiel in einem Act von Giuseppe Giacosa, deutsch von Otto Eisenschitz.)@\emph{Burgtheater. (»Liebelei«, Schauspiel in drei Aufzügen von Arthur Schnitzler. – »Rechte der Seele«, Schauspiel in einem Act von Giuseppe Giacosa, deutsch von Otto Eisenschitz.)}|pwk}. In: \emph{Neue Freie Presse}\pwindex{Neue Freie Presse@\emph{Neue Freie Presse}|pwk}, Nr. 11.184, 13. 10. 1895, Morgenblatt, S. 1–3. Eher unwahrscheinlich ist,
                  dass sich Nordegg\pwindex{Nordegg, Jenny @\textsc{Nordegg, Jenny}, \emph{Schauspieler/Schauspielerin}|pwk} und Fels\pwindex{Fels, Friedrich Michael *~1864@\textsc{Fels, Friedrich Michael} (*~1864), \emph{Journalist/Journalistin}|pwk} auf die erste Reaktion Speidels\pwindex{Speidel, Ludwig 1830-04-11 – 1906-02-03@\textsc{Speidel, Ludwig} (1830-04-11 – 1906-02-03), \emph{Journalist/Journalistin, Kritiker/Kritikerin}|pwk}, dessen Nachtkritik, beziehen: [Ludwig Speidel\pwindex{Speidel, Ludwig 1830-04-11 – 1906-02-03@\textsc{Speidel, Ludwig} (1830-04-11 – 1906-02-03), \emph{Journalist/Journalistin, Kritiker/Kritikerin}|pwk}]: \emph{Theater- und Kunstnachrichten. [Burgtheater]}\pwindex{Theater- und Kunstnachrichten. [Burgtheater] [Liebelei, Rechte der Seele]@\emph{Theater- und Kunstnachrichten. [Burgtheater] [Liebelei, Rechte der Seele]}|pwk}. In: \emph{Neue Freie Presse}\pwindex{Neue Freie Presse@\emph{Neue Freie Presse}|pwk}, Nr. 11.181, 10. 10. 1895, S. 7.}}}\label{K_L00506-1} und freuen uns
               riesig über Ihren Erfolg. Fahren Sie so weiter, junger Ma{\geminationn}, und vergeſsen Sie im Glücke nicht »\label{K_L00506-2v}\edtext{derer, die am Wege sterben\pwindex{Uriel Acosta. Trauerspiel in fuenf Aufzuegen@\emph{Uriel Acosta. Trauerspiel in fünf Aufzügen}|pwv}}{\lemma{\textnormal{\emph{derer, … sterben}}}\Cendnote{\textnormal{Zitat aus \emph{Uriel Acosta}\pwindex{Uriel Acosta. Trauerspiel in fuenf Aufzuegen@\emph{Uriel Acosta. Trauerspiel in fünf Aufzügen}|pwk} von Karl
                     Gutzkow\pwindex{Gutzkow, Karl 17.03.1811 – 16.12.1878@\textsc{Gutzkow, Karl} (17.03.1811 – 16.12.1878), \emph{Schriftsteller/Schriftstellerin}|pwk} (1846)}}}\label{K_L00506-2}«.\pend
           
\pstart
           Herzlichst{\\[\baselineskip]}{\\[\baselineskip]}\spacefill\mbox{{[}hs. :{]} Jenny Nordegg}{\\[\baselineskip]}{[}hs. :{]} und \spacefill\mbox{Friedr. M. Fels}\pend
           \leftskip=0em{}\selectlanguage{ngerman}\endnumbering\briefempfaengerindex{Schnitzler, Arthur@\textsc{Schnitzler, Arthur}!zzzNordegg, Jenny@\emph{von Jenny Nordegg}!1895-10-152@{15. 10. 1895}|)be}\briefempfaengerindex{Schnitzler, Arthur@\textsc{Schnitzler, Arthur}!zzzFels, Friedrich Michael@\emph{von Friedrich Michael Fels}!1895-10-152@{15. 10. 1895}|)be}\mylabel{L00506h}  \normalsize

\doendnotes{C}
\bigskip
\vfill

\clearpage

\footnotesize

\lohead{\textsc{register}}

% Definiere theindex-Environment komplett neu ohne reledmac
\makeatletter
\renewenvironment{theindex}{%
  \section*{\indexname}%
  \setlength{\parindent}{0pt}%
  \setlength{\parskip}{0pt plus 0.3pt}%
  \let\item\@idxitem
}{%
  \clearpage
}
\makeatother

\IfFileExists{\jobname-pw.ind}{\input{\jobname-pw.ind}}{}

\end{document}

      