%% latex-leseansicht-vorspann.tex
%% Vorspann für die Leseansicht.
%% Lädt die gemeinsame Datei latex-vorspann.tex mit nicht gesetztem Schalter.

\newif\ifkorrekturansicht
\korrekturansichtfalse

\input{../tex-inputs/latex-vorspann}


\section[Friedrich M. Fels und Jenny Nordegg an Arthur Schnitzler, 15. 10. 1895]{L00506 Friedrich M. Fels und Jenny Nordegg an Arthur Schnitzler, 15. 10. 1895}
\nopagebreak\mylabel{L00506v}
\rehead{ }\normalsize\beginnumbering\briefempfaengerindex{Schnitzler, Arthur@\textsc{Schnitzler, Arthur}!zzzNordegg, Jenny@\emph{von Jenny Nordegg}!1895-10-152@{15. 10. 1895}|(be}\briefempfaengerindex{Schnitzler, Arthur@\textsc{Schnitzler, Arthur}!zzzFels, Friedrich Michael@\emph{von Friedrich Michael Fels}!1895-10-152@{15. 10. 1895}|(be}
\toendnotes[C]{\smallbreak\pagebreak[2]}
\correspDesc{Versand  durch Friedrich M. Fels, Jenny Nordegg am 15. 10. 1895 in Zürich
\newline{}Erhalt  durch Arthur Schnitzler am 17. 10. 1895 in Wien}\toendnotes[C]{\smallbreak}
\Standort{DLA, A:Schnitzler, HS.NZ85.1.2956.}
\physDesc{Postkarte, 307 Zeichen
\newline{}Handschrift Friedrich Michael Fels: schwarze Tinte, lateinische Kurrent
\newline{}Handschrift Jenny Nordegg: schwarze Tinte
\newline{}Versand: 1) Stempel: »\nobreak{}\oindex{Zürich@\textbf{Zürich}|pwk}Zürich Bhf. Exp., 15. X. 95, 11\nobreak{}«.   2) Stempel: »\nobreak{}\oindex{IX., Alsergrund@\textbf{IX., Alsergrund}, \emph{Verwaltungsgebiet}|pwk}Wien 9/3, 17 10. 95, 9.V, Bestellt\nobreak{}«. 
\newline{}Schnitzler: mit Bleistift nummeriert: »27« }\toendnotes[C]{\smallbreak}\pstart{}{\pb}Herrn Dr. med. Arthur Schnitzler\pend{}\pstart{}Schriftsteller\pend{}\pstart{}Wien\oindex{Wien@\textbf{Wien}, \emph{Verwaltungsgebiet}|pw}\pend{}\pstart{}IX, Frankgaſse 1\oindex{Wien@\textbf{Wien}!IX., Alsergrund@\textbf{IX., Alsergrund}!Frankgasse 1@\textbf{Frankgasse 1}, \emph{Wohngebäude}|pw}\pend{}\pstart{}Österreich\oindex{Österreich@\textbf{Österreich}|pw}\pend{}{\bigskip}\vspace{1em}
\pstart
           \centering{}{\pb}\textcolor{gray}{\textbf{Grand Restaurant et Café Metropol Zurich
                        Auböck {\kaufmannsund} Ziegler Pr.\oindex{Café Metropol@\textbf{Café Metropol}, \emph{Kaffeehaus}|pw}}}\pend
           
\pstart
           \centering{}\textcolor{gray}{\textbf{Irrgarten (Labyrinth)\oindex{Irrgarten (Irrgänge)@\textbf{Irrgarten (Irrgänge)}, \emph{Vergnügungspark}|pw} D\textsuperscript{ir}{ }G. D.’Ouvenou\pwindex{D’Ouvenou, Gésa @\textsc{D’Ouvenou, Gésa}, \emph{Schausteller}|pw}.}}\pend
           
\pstart{}Lieber Dr. Schnitzler!\pend\vspace{0.5em}
\pstart
           Soeben lesen wir Speidels\pwindex{Speidel, Ludwig 11.\,4.\,1830 Ulm – 3.\,2.\,1906 Wien@\textsc{Speidel, Ludwig} (11.\,4.\,1830 Ulm – 3.\,2.\,1906 Wien), \emph{Journalist, Kritiker}|pw}{ }\label{K_L00506-1v}\edtext{Kritik\pwindex{Speidel, Ludwig 11.\,4.\,1830 Ulm – 3.\,2.\,1906 Wien@\textsc{Speidel, Ludwig} (11.\,4.\,1830 Ulm – 3.\,2.\,1906 Wien), \emph{Journalist, Kritiker}!Burgtheater. (»Liebelei«, Schauspiel in drei Aufzügen von Arthur Schnitzler. – »Rechte der Seele«, Schauspiel in einem Act von Giuseppe Giacosa, deutsch von Otto Eisenschitz.)@\strich\emph{Burgtheater. (»Liebelei«, Schauspiel in drei Aufzügen von Arthur Schnitzler. – »Rechte der Seele«, Schauspiel in einem Act von Giuseppe Giacosa, deutsch von Otto Eisenschitz.)}|pwv}}{\lemma{\textnormal{\emph{Kritik}}}\Cendnote{\textnormal{L. Sp.\pwindex{Speidel, Ludwig 11.\,4.\,1830 Ulm – 3.\,2.\,1906 Wien@\textsc{Speidel, Ludwig} (11.\,4.\,1830 Ulm – 3.\,2.\,1906 Wien), \emph{Journalist, Kritiker}|pwkv} [ = Ludwig Speidel\pwindex{Speidel, Ludwig 11.\,4.\,1830 Ulm – 3.\,2.\,1906 Wien@\textsc{Speidel, Ludwig} (11.\,4.\,1830 Ulm – 3.\,2.\,1906 Wien), \emph{Journalist, Kritiker}|pwk}]: \emph{Burgtheater. (»Liebelei«, Schauspiel in drei Aufzügen von
                        Arthur Schnitzler. – »Rechte der Seele«, Schauspiel in einem Act von
                        Giuseppe Giacosa, deutsch von Otto Eisenschitz)}\pwindex{Speidel, Ludwig 11.\,4.\,1830 Ulm – 3.\,2.\,1906 Wien@\textsc{Speidel, Ludwig} (11.\,4.\,1830 Ulm – 3.\,2.\,1906 Wien), \emph{Journalist, Kritiker}!Burgtheater. (»Liebelei«, Schauspiel in drei Aufzügen von Arthur Schnitzler. – »Rechte der Seele«, Schauspiel in einem Act von Giuseppe Giacosa, deutsch von Otto Eisenschitz.)@\strich\emph{Burgtheater. (»Liebelei«, Schauspiel in drei Aufzügen von Arthur Schnitzler. – »Rechte der Seele«, Schauspiel in einem Act von Giuseppe Giacosa, deutsch von Otto Eisenschitz.)}|pwk}. In: \emph{Neue Freie Presse}\pwindex{Neue Freie Presse@\emph{Neue Freie Presse}|pwk}, Nr. 11.184, 13. 10. 1895, Morgenblatt, S. 1–3. Eher unwahrscheinlich ist,
                  dass sich Nordegg\pwindex{Nordegg, Jenny @\textsc{Nordegg, Jenny}, \emph{Schauspielerin}|pwk} und Fels\pwindex{Fels, Friedrich Michael *~1864 Bad Dürkheim@\textsc{Fels, Friedrich Michael} (*~1864 Bad Dürkheim), \emph{Journalist}|pwk} auf die erste Reaktion Speidels\pwindex{Speidel, Ludwig 11.\,4.\,1830 Ulm – 3.\,2.\,1906 Wien@\textsc{Speidel, Ludwig} (11.\,4.\,1830 Ulm – 3.\,2.\,1906 Wien), \emph{Journalist, Kritiker}|pwk}, dessen Nachtkritik, beziehen: [Ludwig Speidel\pwindex{Speidel, Ludwig 11.\,4.\,1830 Ulm – 3.\,2.\,1906 Wien@\textsc{Speidel, Ludwig} (11.\,4.\,1830 Ulm – 3.\,2.\,1906 Wien), \emph{Journalist, Kritiker}|pwk}]: \emph{Theater- und Kunstnachrichten. [Burgtheater]}\pwindex{Theater- und Kunstnachrichten. [Burgtheater] [Liebelei, Rechte der Seele]@\emph{Theater- und Kunstnachrichten. [Burgtheater] [Liebelei, Rechte der Seele]}|pwk}. In: \emph{Neue Freie Presse}\pwindex{Neue Freie Presse@\emph{Neue Freie Presse}|pwk}, Nr. 11.181, 10. 10. 1895, S. 7.}}}\label{K_L00506-1} und freuen uns
               riesig über Ihren Erfolg. Fahren Sie so weiter, junger Ma{\geminationn}, und vergeſsen Sie im Glücke nicht »\label{K_L00506-2v}\edtext{derer, die am Wege sterben\pwindex{Gutzkow, Karl 17.\,3.\,1811 Berlin – 16.\,12.\,1878 Frankfurt am Main@\textsc{Gutzkow, Karl} (17.\,3.\,1811 Berlin – 16.\,12.\,1878 Frankfurt am Main), \emph{Schriftsteller}!Uriel Acosta. Trauerspiel in fünf Aufzügen@\strich\emph{Uriel Acosta. Trauerspiel in fünf Aufzügen}|pwv}\pwindex{Gutzkow, Karl 17.\,3.\,1811 Berlin – 16.\,12.\,1878 Frankfurt am Main@\textsc{Gutzkow, Karl} (17.\,3.\,1811 Berlin – 16.\,12.\,1878 Frankfurt am Main), \emph{Schriftsteller}!Uriel Acosta. Trauerspiel in fünf Aufzügen@\strich\emph{Uriel Acosta. Trauerspiel in fünf Aufzügen}|pwv}}{\lemma{\textnormal{\emph{derer, … sterben}}}\Cendnote{\textnormal{Zitat aus \emph{Uriel Acosta}\pwindex{Gutzkow, Karl 17.\,3.\,1811 Berlin – 16.\,12.\,1878 Frankfurt am Main@\textsc{Gutzkow, Karl} (17.\,3.\,1811 Berlin – 16.\,12.\,1878 Frankfurt am Main), \emph{Schriftsteller}!Uriel Acosta. Trauerspiel in fünf Aufzügen@\strich\emph{Uriel Acosta. Trauerspiel in fünf Aufzügen}|pwk}\pwindex{Gutzkow, Karl 17.\,3.\,1811 Berlin – 16.\,12.\,1878 Frankfurt am Main@\textsc{Gutzkow, Karl} (17.\,3.\,1811 Berlin – 16.\,12.\,1878 Frankfurt am Main), \emph{Schriftsteller}!Uriel Acosta. Trauerspiel in fünf Aufzügen@\strich\emph{Uriel Acosta. Trauerspiel in fünf Aufzügen}|pwk} von Karl
                     Gutzkow\pwindex{Gutzkow, Karl 17.\,3.\,1811 Berlin – 16.\,12.\,1878 Frankfurt am Main@\textsc{Gutzkow, Karl} (17.\,3.\,1811 Berlin – 16.\,12.\,1878 Frankfurt am Main), \emph{Schriftsteller}|pwk} (1846)}}}\label{K_L00506-2}«.\pend
           
\pstart
           Herzlichst{\\[\baselineskip]}{\\[\baselineskip]}\spacefill\mbox{{[}hs. Nordegg:{]} Jenny Nordegg}{\\[\baselineskip]}{[}hs. Fels:{]} und \spacefill\mbox{Friedr. M. Fels}\pend
           \leftskip=0em{}\selectlanguage{ngerman}\endnumbering\briefempfaengerindex{Schnitzler, Arthur@\textsc{Schnitzler, Arthur}!zzzNordegg, Jenny@\emph{von Jenny Nordegg}!1895-10-152@{15. 10. 1895}|)be}\briefempfaengerindex{Schnitzler, Arthur@\textsc{Schnitzler, Arthur}!zzzFels, Friedrich Michael@\emph{von Friedrich Michael Fels}!1895-10-152@{15. 10. 1895}|)be}\mylabel{L00506h}  \newcommand{\dateiname}{L00506}\newcommand{\titel}{Friedrich M. Fels und Jenny Nordegg an Arthur Schnitzler, 15. 10. 1895}\newcommand{\editorInnen}{Martin Anton Müller und Gerd-Hermann Susen}%% latex-leseansicht-abspann.tex
%% Abspann für die Leseansicht.
%% Der Schalter \ifkorrekturansicht ist bereits durch den Vorspann gesetzt.

%% latex-abspann.tex
%% Gemeinsamer Abspann für Korrekturansicht und Leseansicht.
%% Setzt den Schalter \ifkorrekturansicht voraus (gesetzt in den
%% einbindenden Dateien latex-korrekturansicht-abspann.tex bzw.
%% latex-leseansicht-abspann.tex).
%% ---------------------------------------------------------------

\normalsize

% Das esempio-Environment wird nur in der Leseansicht benötigt
\ifkorrekturansicht\else
\newenvironment{esempio}[3]%
{
    \vspace{1.5ex}
    \rlap{\underline{#1}}
    \par
    \setlength{\parindent}{0cm}
    \nopagebreak
    \leftskip=#2cm
    \rightskip=#3cm
}
{
    \par
}
\fi

\doendnotes{C}
\bigskip
\vfill

\clearpage

\footnotesize

\ifkorrekturansicht
  \lohead{\textsc{register}}
\fi

% theindex-Environment neu definieren ohne reledmac
\makeatletter
\renewenvironment{theindex}{%
  \ifkorrekturansicht
    \section*{\indexname}%
  \else
    \subsubsection*{Index der erwähnten Entitäten}%
  \fi
  \setlength{\parindent}{0pt}%
  \setlength{\parskip}{0pt plus 0.3pt}%
  \let\item\@idxitem
}{%
  \ifkorrekturansicht\clearpage\fi
}
\makeatother

\IfFileExists{\jobname-pw.ind}{\input{\jobname-pw.ind}}{}

% Quellenangabe nur in der Leseansicht
\ifkorrekturansicht\else
% Fallback-Definitionen, falls die .tex-Datei \titel etc. nicht gesetzt hat
\providecommand{\titel}{}
\providecommand{\editorInnen}{}
\providecommand{\dateiname}{\jobname}

\vspace{3cm}

\vfill

\footnotesize
\textsc{Quelle}: \titel. Herausgegeben von {\editorInnen}. In: \emph{Arthur Schnitzler: Briefwechsel mit Autorinnen und Autoren}.
 Digitale Edition, https://schnitzler-briefe.acdh.oeaw.ac.at/{\dateiname}.html (Stand \today)
\fi

\end{document}


