%% latex-leseansicht-vorspann.tex
%% Vorspann für die Leseansicht.
%% Lädt die gemeinsame Datei latex-vorspann.tex mit nicht gesetztem Schalter.

\newif\ifkorrekturansicht
\korrekturansichtfalse

\input{../tex-inputs/latex-vorspann}


\section[Arthur Schnitzler an Gustav Schwarzkopf, 15. 7. 1898]{L04125 Arthur Schnitzler an Gustav Schwarzkopf, 15. 7. 1898}
\nopagebreak\mylabel{L04125v}
\rehead{ }\normalsize\beginnumbering\briefempfaengerindex{Schwarzkopf, Gustav@\textsc{Schwarzkopf, Gustav}!zzzSchnitzler, Arthur@\emph{von Arthur Schnitzler}!1898-07-153@{15. 7. 1898}|(be}
\toendnotes[C]{\smallbreak\pagebreak[2]}
\correspDesc{Versand  durch Arthur Schnitzler am 15. 7. 1898 in Graz
\newline{}Erhalt  durch Gustav Schwarzkopf am 16. 7. 1898 in Wien}\toendnotes[C]{\smallbreak}
\Standort{CUL, Schnitzler, B 96.}
\physDesc{Postkarte, 467 Zeichen
\newline{}Handschrift: Bleistift, deutsche Kurrent
\newline{}Versand: 1) Stempel: »\nobreak{}\oindex{Graz@\textbf{Graz}, \emph{Verwaltungsgebiet}|pwk}Graz Murvorstadt, 15/7 98, 11V\nobreak{}«.   2) Stempel: »\nobreak{}\oindex{I., Innere Stadt@\textbf{I., Innere Stadt}, \emph{Verwaltungsgebiet}|pwk}Wien 1, 16. 7. 98, 6–9½V, Bestellt\nobreak{}«. }\pstart{}{\pb}Herrn \textsc{Gustav Schwarzkopf}\pend{}\pstart{}Wien I\oindex{I., Innere Stadt@\textbf{I., Innere Stadt}, \emph{Verwaltungsgebiet}|pw}\pend{}\pstart{}\textsc{Tiefer
                        Graben 23\oindex{Wien@\textbf{Wien}!I., Innere Stadt@\textbf{I., Innere Stadt}!Tiefer Graben 23@\textbf{Tiefer Graben 23}, \emph{Wohngebäude}|pw}}.\pend{}{\bigskip}\vspace{1em}
\pstart
           \noindent{}{\pb}Lieber Guſtav,das ist mein vorläufiges Programm\pend
           
\pstart
           \uline{So{\geminationn}tag}
                (17) Graz\oindex{Graz@\textbf{Graz}, \emph{Verwaltungsgebiet}|pw} – Steinach\oindex{Steinach@\textbf{Steinach}|pw} (Bahn) – Schladming\oindex{Schladming@\textbf{Schladming}, \emph{Verwaltungsgebiet}|pw} (Rad)\pend
           
\pstart
           Montag 18.{ }Schladming\oindex{Schladming@\textbf{Schladming}, \emph{Verwaltungsgebiet}|pw} – Biſchofshofen\oindex{Bischofshofen@\textbf{Bischofshofen}, \emph{Hauptstadt}|pw} (Rad–) – Zell a S.\oindex{Zell am See@\textbf{Zell am See}, \emph{Hauptstadt}|pw}
                  (Bahn{[}){]}\pend
           
\pstart
           \uline{Dinſtg} 19.{ }Zell am See\oindex{Zell am See@\textbf{Zell am See}, \emph{Hauptstadt}|pw}, Bruck\oindex{Bruck an der Großglocknerstraße@\textbf{Bruck an der Großglocknerstraße}, \emph{Hauptstadt}|pw} – Fuſch\oindex{Bad Fusch@\textbf{Bad Fusch}|pw}.\pend
           
\pstart
           \uline{Mittwoch} 20 (Ferleiten\oindex{Ferleiten@\textbf{Ferleiten}|pw}), Bruck\oindex{Bruck an der Großglocknerstraße@\textbf{Bruck an der Großglocknerstraße}, \emph{Hauptstadt}|pw} – Bad-Gaſtein\oindex{Bad Gastein@\textbf{Bad Gastein}, \emph{Hauptstadt}|pw}\pend
           
\pstart
           Bis So{\geminationn}tag etwa Gaſtein\oindex{Bad Gastein@\textbf{Bad Gastein}, \emph{Hauptstadt}|pw}; da{\geminationn} per Rad Salzburg\oindex{Salzburg@\textbf{Salzburg}, \emph{Verwaltungsgebiet}|pw}; wo ich jedenfalls von Montag (25) – Do{\geminationn}erſtg 28. bin. –
               Eventuelle \uline{telegr.} Adreſſe Steinach\oindex{Steinach@\textbf{Steinach}|pw}; Briefe Wildbad
                  Gaſtein, Villa Waſſing\oindex{Villa Dr. Wassing@\textbf{Villa Dr. Wassing}, \emph{Sanatorium}|pw}.\pend
           
\pstart
           Auf Wiederſehn{\\[\baselineskip]} Herzlichſt Ihr \spacefill\mbox{A.}\pend
           \leftskip=0em{}\selectlanguage{ngerman}\endnumbering\briefempfaengerindex{Schwarzkopf, Gustav@\textsc{Schwarzkopf, Gustav}!zzzSchnitzler, Arthur@\emph{von Arthur Schnitzler}!1898-07-153@{15. 7. 1898}|)be}\mylabel{L04125h}
\begin{anhang}
\end{anhang}\newcommand{\dateiname}{L04125}\newcommand{\titel}{Arthur Schnitzler an Gustav Schwarzkopf, 15. 7. 1898}\newcommand{\editorInnen}{Herausgegeben von Jahnke, SelmaMüller, Martin Anton}%% latex-leseansicht-abspann.tex
%% Abspann für die Leseansicht.
%% Der Schalter \ifkorrekturansicht ist bereits durch den Vorspann gesetzt.

%% latex-abspann.tex
%% Gemeinsamer Abspann für Korrekturansicht und Leseansicht.
%% Setzt den Schalter \ifkorrekturansicht voraus (gesetzt in den
%% einbindenden Dateien latex-korrekturansicht-abspann.tex bzw.
%% latex-leseansicht-abspann.tex).
%% ---------------------------------------------------------------

\normalsize

% Das esempio-Environment wird nur in der Leseansicht benötigt
\ifkorrekturansicht\else
\newenvironment{esempio}[3]%
{
    \vspace{1.5ex}
    \rlap{\underline{#1}}
    \par
    \setlength{\parindent}{0cm}
    \nopagebreak
    \leftskip=#2cm
    \rightskip=#3cm
}
{
    \par
}
\fi

\doendnotes{C}
\bigskip
\vfill

\clearpage

\footnotesize

\ifkorrekturansicht
  \lohead{\textsc{register}}
\fi

% theindex-Environment neu definieren ohne reledmac
\makeatletter
\renewenvironment{theindex}{%
  \ifkorrekturansicht
    \section*{\indexname}%
  \else
    \subsubsection*{Index der erwähnten Entitäten}%
  \fi
  \setlength{\parindent}{0pt}%
  \setlength{\parskip}{0pt plus 0.3pt}%
  \let\item\@idxitem
}{%
  \ifkorrekturansicht\clearpage\fi
}
\makeatother

\IfFileExists{\jobname-pw.ind}{\input{\jobname-pw.ind}}{}

% Quellenangabe nur in der Leseansicht
\ifkorrekturansicht\else
% Fallback-Definitionen, falls die .tex-Datei \titel etc. nicht gesetzt hat
\providecommand{\titel}{}
\providecommand{\editorInnen}{}
\providecommand{\dateiname}{\jobname}

\vspace{3cm}

\vfill

\footnotesize
\textsc{Quelle}: \titel. Herausgegeben von {\editorInnen}. In: \emph{Arthur Schnitzler: Briefwechsel mit Autorinnen und Autoren}.
 Digitale Edition, https://schnitzler-briefe.acdh.oeaw.ac.at/{\dateiname}.html (Stand \today)
\fi

\end{document}


