%% latex-leseansicht-vorspann.tex
%% Vorspann für die Leseansicht.
%% Lädt die gemeinsame Datei latex-vorspann.tex mit nicht gesetztem Schalter.

\newif\ifkorrekturansicht
\korrekturansichtfalse

\input{../tex-inputs/latex-vorspann}


         
         \renewcommand{\erwaehntePersonen}{Personen: Albert Steinrück}
         \renewcommand{\erwaehnteOrte}{Orte: Deutschland, Hotel Vier Jahreszeiten, München, Wien}
         \renewcommand{\erwaehnteWerke}{Werke: Arbeiter-Zeitung, Das weite Land. Tragikomödie in fünf Akten, Schnitzlers »Weites Land«. Erste Aufführung im Burgtheater}
               \section[Stefan Großmann an Arthur Schnitzler, 5. 2. 1912]{ Stefan Großmann an Arthur Schnitzler, 5. 2. 1912}\nopagebreak\mylabel{v}\rehead{ }\begin{ledgroupsized}[t]{13cm}\normalsize\beginnumbering \toendnotes[C]{\smallbreak\pagebreak[2]} \Standort{CUL, Schnitzler, B 34.}
\physDesc{Brief, 1 Blatt, 4 Seiten, 1143 Zeichen
\newline{}Schreibmaschine
\newline{}Handschrift: schwarze Tinte, deutsche Kurrent
\newline{}Schnitzler: 1) mit Bleistift beschriftet: »\textsc{Großma{\geminationn}}«  2) mit rotem Buntstift eine Unterstreichung
\newline{}Ordnung: mit Bleistift von unbekannter Hand nummeriert:
                                    »11« }\toendnotes[C]{\smallbreak}\pstart
           \noindent{}\centering{}{\pb}\textcolor{gray}{\textbf{\textsc{Hotel}}}\pend
           \pstart
           \noindent{}\centering{}\textcolor{gray}{\textbf{\textsc{Vier Jahreszeiten}\oindex{Hotel Vier Jahreszeiten@\textbf{Hotel Vier Jahreszeiten}|pw}}}\pend
           \pstart
           \noindent{}\centering{}\textcolor{gray}{\textbf{TELEGRAMM-ADRESSE: JAHRESZEITENTYP, MÜNCHEN\oindex{Muenchen@\textbf{München}|pw}.}}\pend
           \pstart
           \noindent{}\centering{}\textcolor{gray}{\textbf{Lieber’s Code – International Hôtel-Code.}}\pend
           \pstart
           \noindent{}\centering{}\textcolor{gray}{\textbf{TELEFON 23073–23076}}\pend
           \pstart
           \raggedleft{}\textcolor{gray}{\textbf{MÜNCHEN\oindex{Muenchen@\textbf{München}|pw},}}{ }5 Februar 1912\pend
           \pstart
           \uline{Auf der Durchreiſe.}\pend
           \pstart
           Nachdem ich nun in München\oindex{Muenchen@\textbf{München}|pw}{ }\strikeout{\textcolor{gray}{ſa}} »Das weite Land\pwindex{Schnitzler, Arthur 15.05.1862 – 21.10.1931@\textsc{Schnitzler, Arthur} (15.05.1862 – 21.10.1931), \emph{Schriftsteller, Mediziner}!weite Land. Tragikomoedie in fuenf Akten1910-10-20@\strich\emph{Das weite Land. Tragikomödie in fünf Akten} {[}1910-10-20{]}|pw}« mit Hrn \textsc{Steinrück}\pwindex{Steinrueck, Albert 20.05.1872 – 11.02.1929@\textsc{Steinrück, Albert} (20.05.1872 – 11.02.1929), \emph{Schauspieler}|pw} ſah, möchte ich Ihnen, verehrter Herr Schnitzler, – wiewohl Sie gewiſs auf
               dieſe Correctur gar kein Gewicht legen – ſagen, daſs ich nun erſt das Werk wirklich
               gefühlt habe und das verfluchte Zeitungshandwerk anklage, welches Einen zwingt, im
               Handumdrehen {\pb}ein paar \label{K_L02052_1v}\edtext{leicht-fertige Dinge\pwindex{Schnitzlers »Weites Land«. Erste Auffuehrung im Burgtheater1911-10-15@\emph{Schnitzlers »Weites Land«. Erste Aufführung im Burgtheater} {[}1911-10-15{]}|pwv}}{\lemma{\textnormal{\emph{leicht-fertige Dinge}}}\Cendnote{\textnormal{Seine Kritik fasste er am Ende der
                  Rezension der Uraufführung (st. gr.\pwindex{Grossmann, Stefan 19.05.1875 – 03.01.1935@\textsc{Großmann, Stefan} (19.05.1875 – 03.01.1935), \emph{Schriftsteller, Journalist}|pwk}: \emph{Schnitzlers »Weites Land«. Erste Aufführung im Burgtheater}\pwindex{Schnitzlers »Weites Land«. Erste Auffuehrung im Burgtheater1911-10-15@\emph{Schnitzlers »Weites Land«. Erste Aufführung im Burgtheater} {[}1911-10-15{]}|pwk}. In: \emph{Arbeiter-Zeitung}\pwindex{Arbeiter-Zeitung12.7.1881 – 31.10.1991@\emph{Arbeiter-Zeitung} {[}12.7.1881 – 31.10.1991{]}|pwk}, Jg. 23, Nr. 284,
                        15. 11. 1910, S. 3–4.) zusammen: »Das Publikum
                     nahm das übergrübelte Schauspiel mit großem Interesse auf und gab sich auch den
                     zarten, eigentlich novellistischen Reizen der Dichtung mit außerordentlicher
                     Bereitwilligkeit hin. Nach jedem Akt wurde Schnitzler\pwindex{Schnitzler, Arthur 15.05.1862 – 21.10.1931@\textsc{Schnitzler, Arthur} (15.05.1862 – 21.10.1931), \emph{Schriftsteller, Mediziner}|pw} hervorgerufen und dankte in etwas müder
                  Haltung.«}}}\label{K_L02052_1h} innerhalb einiger Stunden über eine Dichtung zu
               ſagen.\pend
           \pstart
           Durch Hrn \uline{Steinrück}\pwindex{Steinrueck, Albert 20.05.1872 – 11.02.1929@\textsc{Steinrück, Albert} (20.05.1872 – 11.02.1929), \emph{Schauspieler}|pw} ſah ich erſt, wie viel menſchliche Stärke im Hofreiter\pwindex{Schnitzler, Arthur 15.05.1862 – 21.10.1931@\textsc{Schnitzler, Arthur} (15.05.1862 – 21.10.1931), \emph{Schriftsteller, Mediziner}!weite Land. Tragikomoedie in fuenf Akten1910-10-20@\strich\emph{Das weite Land. Tragikomödie in fünf Akten} {[}1910-10-20{]}|pw} ſteckt, wie viel ſittliche Energie bei aller Freiheit, wie viel
               Willens-training bei aller Ungebundenheit.\pend
           \pstart
           Das verdammte Geſetz der Nähe verwirrt Einen oft, ich ſah nur {\pb}das Äußerliche, die Wien\oindex{Wien@\textbf{Wien}|pw}er Nichtsthuer-atmosphäre, das war oberflächlich und
               anmaßend.\pend
           \pstart
           Es liegt mir daran, Ihnen zu ſagen, daſs ich das Werk geſtern mit einer Art Bangen
               mitgefühlt habe und einen tiefen, nicht ſchnell zu verwiſchenden Eindruck nach Hauſe
               trage.\pend
           \pstart
           Ich ſchreibe Ihnen dies mitten auf einer Forſchungsreiſe nach Talenten durch ganz Deutſchland\oindex{Deutschland@\textbf{Deutschland}|pw} und nur deshalb, {\pb}weil ich \uline{mir}
               durch dieſes Geſtändnis eine erleichterte Viertelſtunde machen will.\pend
           \pstart
           Sehr ergeben:{\\[\baselineskip]}\spacefill\mbox{Stefan Großmann}\pend
           \leftskip=0em{}
         
         \endnumbering\mylabel{h}\end{ledgroupsized}  \newcommand{\dateiname}{L02052}\newcommand{\titel}{Stefan Großmann an Arthur Schnitzler, 5. 2. 1912}\newcommand{\editorInnen}{ Martin Anton Müller und Gerd-Hermann Susen}%% latex-leseansicht-abspann.tex
%% Abspann für die Leseansicht.
%% Der Schalter \ifkorrekturansicht ist bereits durch den Vorspann gesetzt.

%% latex-abspann.tex
%% Gemeinsamer Abspann für Korrekturansicht und Leseansicht.
%% Setzt den Schalter \ifkorrekturansicht voraus (gesetzt in den
%% einbindenden Dateien latex-korrekturansicht-abspann.tex bzw.
%% latex-leseansicht-abspann.tex).
%% ---------------------------------------------------------------

\normalsize

% Das esempio-Environment wird nur in der Leseansicht benötigt
\ifkorrekturansicht\else
\newenvironment{esempio}[3]%
{
    \vspace{1.5ex}
    \rlap{\underline{#1}}
    \par
    \setlength{\parindent}{0cm}
    \nopagebreak
    \leftskip=#2cm
    \rightskip=#3cm
}
{
    \par
}
\fi

\doendnotes{C}
\bigskip
\vfill

\clearpage

\footnotesize

\ifkorrekturansicht
  \lohead{\textsc{register}}
\fi

% theindex-Environment neu definieren ohne reledmac
\makeatletter
\renewenvironment{theindex}{%
  \ifkorrekturansicht
    \section*{\indexname}%
  \else
    \subsubsection*{Index der erwähnten Entitäten}%
  \fi
  \setlength{\parindent}{0pt}%
  \setlength{\parskip}{0pt plus 0.3pt}%
  \let\item\@idxitem
}{%
  \ifkorrekturansicht\clearpage\fi
}
\makeatother

\IfFileExists{\jobname-pw.ind}{\input{\jobname-pw.ind}}{}

% Quellenangabe nur in der Leseansicht
\ifkorrekturansicht\else
% Fallback-Definitionen, falls die .tex-Datei \titel etc. nicht gesetzt hat
\providecommand{\titel}{}
\providecommand{\editorInnen}{}
\providecommand{\dateiname}{\jobname}

\vspace{3cm}

\vfill

\footnotesize
\textsc{Quelle}: \titel. Herausgegeben von {\editorInnen}. In: \emph{Arthur Schnitzler: Briefwechsel mit Autorinnen und Autoren}.
 Digitale Edition, https://schnitzler-briefe.acdh.oeaw.ac.at/{\dateiname}.html (Stand \today)
\fi

\end{document}


      