%% latex-leseansicht-vorspann.tex
%% Vorspann für die Leseansicht.
%% Lädt die gemeinsame Datei latex-vorspann.tex mit nicht gesetztem Schalter.

\newif\ifkorrekturansicht
\korrekturansichtfalse

\input{../tex-inputs/latex-vorspann}


\section[Stefan Großmann an Arthur Schnitzler, 5. 2. 1912]{L02052 Stefan Großmann an Arthur Schnitzler, 5. 2. 1912}
\nopagebreak\mylabel{L02052v}
\rehead{ }\normalsize\beginnumbering\briefempfaengerindex{Schnitzler, Arthur@\textsc{Schnitzler, Arthur}!zzzGroßmann, Stefan@\emph{von Stefan Großmann}!1912-02-051@{5. 2. 1912}|(be}
\toendnotes[C]{\smallbreak\pagebreak[2]}
\correspDesc{Versand  durch Stefan Großmann am 5. 2. 1912 in München
\newline{}Erhalt  durch Arthur Schnitzler im Zeitraum [6. 2. 1912
                  – 10. 2. 1912?] in Wien}\toendnotes[C]{\smallbreak}
\Standort{CUL, Schnitzler, B 34.}
\physDesc{Brief, 1 Blatt, 4 Seiten, 1143 Zeichen
\newline{}Schreibmaschine
\newline{}Handschrift: schwarze Tinte, deutsche Kurrent
\newline{}Schnitzler: 1) mit Bleistift beschriftet: »\textsc{Großma{\geminationn}}«  2) mit rotem Buntstift eine Unterstreichung
\newline{}Ordnung: mit Bleistift von unbekannter Hand nummeriert:
                                    »11« }\toendnotes[C]{\smallbreak}
\pstart
           \centering{}{\pb}\textcolor{gray}{\textbf{\textsc{Hotel}}}\pend
           
\pstart
           \centering{}\textcolor{gray}{\textbf{\textsc{Vier Jahreszeiten}\oindex{Hotel Vier Jahreszeiten@\textbf{Hotel Vier Jahreszeiten}, \emph{Hotel}|pw}}}\pend
           
\pstart
           \centering{}\textcolor{gray}{\textbf{TELEGRAMM-ADRESSE: JAHRESZEITENTYP, MÜNCHEN\oindex{München@\textbf{München}|pw}.}}\pend
           
\pstart
           \centering{}\textcolor{gray}{\textbf{Lieber’s Code – International Hôtel-Code.}}\pend
           
\pstart
           \centering{}\textcolor{gray}{\textbf{TELEFON 23073–23076}}\pend
           
\pstart
           \raggedleft{}\textcolor{gray}{\textbf{MÜNCHEN\oindex{München@\textbf{München}|pw},}}{ }5 Februar 1912\pend
           
\pstart
           \uline{Auf der Durchreiſe.}\pend
           \vspace{0.5em}
\pstart
           Nachdem ich nun in München\oindex{München@\textbf{München}|pw}{ }\strikeout{\textcolor{gray}{ſa}} »Das weite Land\pwindex{Schnitzler, Arthur 15.\,5.\,1862 Wien – 21.\,10.\,1931 ebd.@\textsc{Schnitzler, Arthur} (15.\,5.\,1862 Wien – 21.\,10.\,1931 ebd.), \emph{Schriftsteller, Mediziner}!weite Land. Tragikomödie in fünf Akten@\strich\emph{Das weite Land. Tragikomödie in fünf Akten}|pw}« mit Hrn \textsc{Steinrück}\pwindex{Steinrück, Albert 20.\,5.\,1872 Wetterburg – 11.\,2.\,1929 Berlin@\textsc{Steinrück, Albert} (20.\,5.\,1872 Wetterburg – 11.\,2.\,1929 Berlin), \emph{Schauspieler}|pw}{ }ſah, möchte ich Ihnen, verehrter Herr Schnitzler, – wiewohl Sie gewiſs auf
               dieſe Correctur gar kein Gewicht legen –{ }ſagen, daſs ich nun erſt das Werk wirklich
               gefühlt habe und das verfluchte Zeitungshandwerk anklage, welches Einen zwingt, im
               Handumdrehen {\pb}ein paar \label{K_L02052-1v}\edtext{leicht-fertige Dinge\pwindex{Großmann, Stefan 19.\,5.\,1875 Wien – 3.\,1.\,1935 ebd.@\textsc{Großmann, Stefan} (19.\,5.\,1875 Wien – 3.\,1.\,1935 ebd.), \emph{Schriftsteller, Journalist}!Schnitzlers »Weites Land«. Erste Aufführung im Burgtheater@\strich\emph{Schnitzlers »Weites Land«. Erste Aufführung im Burgtheater}|pwv}}{\lemma{\textnormal{\emph{leicht-fertige Dinge}}}\Cendnote{\textnormal{Seine Kritik fasste er am Ende der
                   Rezension der Uraufführung\eventindex{Burgtheater@\textbf{Burgtheater}!Premiere von Das weite Land, 14.10.1911 [I.]@Premiere von Das weite Land, 14.10.1911 [I.]|pwkv} zusammen: »Das Publikum
                     nahm das übergrübelte Schauspiel mit großem Interesse auf und gab sich auch den
                     zarten, eigentlich novellistischen Reizen der Dichtung mit außerordentlicher
                     Bereitwilligkeit hin. Nach jedem Akt wurde Schnitzler hervorgerufen und dankte in etwas müder
                     Haltung.« st. gr.\pwindex{Großmann, Stefan 19.\,5.\,1875 Wien – 3.\,1.\,1935 ebd.@\textsc{Großmann, Stefan} (19.\,5.\,1875 Wien – 3.\,1.\,1935 ebd.), \emph{Schriftsteller, Journalist}|pwk}: \emph{Schnitzlers »Weites Land«. Erste Aufführung im Burgtheater}\pwindex{Großmann, Stefan 19.\,5.\,1875 Wien – 3.\,1.\,1935 ebd.@\textsc{Großmann, Stefan} (19.\,5.\,1875 Wien – 3.\,1.\,1935 ebd.), \emph{Schriftsteller, Journalist}!Schnitzlers »Weites Land«. Erste Aufführung im Burgtheater@\strich\emph{Schnitzlers »Weites Land«. Erste Aufführung im Burgtheater}|pwk}. In: \emph{Arbeiter-Zeitung}\pwindex{Arbeiter-Zeitung@\emph{Arbeiter-Zeitung}|pwk}, Jg. 23, Nr. 284,
                        15. 11. 1910, S. 3–4.}}}\label{K_L02052-1} innerhalb einiger Stunden über eine Dichtung zu{ }ſagen.\pend
           
\pstart
           Durch Hrn \uline{Steinrück}\pwindex{Steinrück, Albert 20.\,5.\,1872 Wetterburg – 11.\,2.\,1929 Berlin@\textsc{Steinrück, Albert} (20.\,5.\,1872 Wetterburg – 11.\,2.\,1929 Berlin), \emph{Schauspieler}|pw}{ }ſah ich erſt, wie viel menſchliche Stärke im Hofreiter\pwindex{Schnitzler, Arthur 15.\,5.\,1862 Wien – 21.\,10.\,1931 ebd.@\textsc{Schnitzler, Arthur} (15.\,5.\,1862 Wien – 21.\,10.\,1931 ebd.), \emph{Schriftsteller, Mediziner}!weite Land. Tragikomödie in fünf Akten@\strich\emph{Das weite Land. Tragikomödie in fünf Akten}|pw}{ }ſteckt, wie viel{ }ſittliche Energie bei aller Freiheit, wie viel
               Willens-training bei aller Ungebundenheit.\pend
           
\pstart
           Das verdammte Geſetz der Nähe verwirrt Einen oft, ich{ }ſah nur {\pb}das Äußerliche, die Wien\oindex{Wien@\textbf{Wien}, \emph{Verwaltungsgebiet}|pw}er Nichtsthuer-atmosphäre, das war oberflächlich und
               anmaßend.\pend
           
\pstart
           Es liegt mir daran, Ihnen zu{ }ſagen, daſs ich das Werk geſtern mit einer Art Bangen
               mitgefühlt habe und einen tiefen, nicht{ }ſchnell zu verwiſchenden Eindruck nach Hauſe
               trage.\pend
           
\pstart
           Ich{ }ſchreibe Ihnen dies mitten auf einer Forſchungsreiſe nach Talenten durch ganz Deutſchland\oindex{Deutschland@\textbf{Deutschland}|pw} und nur deshalb, {\pb}weil ich \uline{mir}
               durch dieſes Geſtändnis eine erleichterte Viertelſtunde machen will.\pend
           
\pstart
           Sehr ergeben:{\\[\baselineskip]}\spacefill\mbox{Stefan Großmann}\pend
           \leftskip=0em{}\selectlanguage{ngerman}\endnumbering\briefempfaengerindex{Schnitzler, Arthur@\textsc{Schnitzler, Arthur}!zzzGroßmann, Stefan@\emph{von Stefan Großmann}!1912-02-051@{5. 2. 1912}|)be}\mylabel{L02052h}  \newcommand{\dateiname}{L02052}\newcommand{\titel}{Stefan Großmann an Arthur Schnitzler, 5. 2. 1912}\newcommand{\editorInnen}{Herausgegeben von Martin Anton Müller}%% latex-leseansicht-abspann.tex
%% Abspann für die Leseansicht.
%% Der Schalter \ifkorrekturansicht ist bereits durch den Vorspann gesetzt.

%% latex-abspann.tex
%% Gemeinsamer Abspann für Korrekturansicht und Leseansicht.
%% Setzt den Schalter \ifkorrekturansicht voraus (gesetzt in den
%% einbindenden Dateien latex-korrekturansicht-abspann.tex bzw.
%% latex-leseansicht-abspann.tex).
%% ---------------------------------------------------------------

\normalsize

% Das esempio-Environment wird nur in der Leseansicht benötigt
\ifkorrekturansicht\else
\newenvironment{esempio}[3]%
{
    \vspace{1.5ex}
    \rlap{\underline{#1}}
    \par
    \setlength{\parindent}{0cm}
    \nopagebreak
    \leftskip=#2cm
    \rightskip=#3cm
}
{
    \par
}
\fi

\doendnotes{C}
\bigskip
\vfill

\clearpage

\footnotesize

\ifkorrekturansicht
  \lohead{\textsc{register}}
\fi

% theindex-Environment neu definieren ohne reledmac
\makeatletter
\renewenvironment{theindex}{%
  \ifkorrekturansicht
    \section*{\indexname}%
  \else
    \subsubsection*{Index der erwähnten Entitäten}%
  \fi
  \setlength{\parindent}{0pt}%
  \setlength{\parskip}{0pt plus 0.3pt}%
  \let\item\@idxitem
}{%
  \ifkorrekturansicht\clearpage\fi
}
\makeatother

\IfFileExists{\jobname-pw.ind}{\input{\jobname-pw.ind}}{}

% Quellenangabe nur in der Leseansicht
\ifkorrekturansicht\else
% Fallback-Definitionen, falls die .tex-Datei \titel etc. nicht gesetzt hat
\providecommand{\titel}{}
\providecommand{\editorInnen}{}
\providecommand{\dateiname}{\jobname}

\vspace{3cm}

\vfill

\footnotesize
\textsc{Quelle}: \titel. Herausgegeben von {\editorInnen}. In: \emph{Arthur Schnitzler: Briefwechsel mit Autorinnen und Autoren}.
 Digitale Edition, https://schnitzler-briefe.acdh.oeaw.ac.at/{\dateiname}.html (Stand \today)
\fi

\end{document}


