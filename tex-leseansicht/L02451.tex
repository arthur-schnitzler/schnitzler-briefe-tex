\input{../tex-inputs/latex-pdf-vorspann}
\begin{center}
            \textcolor{red}{ENTWURF. ENTZIFFERUNG NOCH NICHT KORREKTURGELESEN}
                      \end{center}
            
               \section[Stefan Großmann an Arthur Schnitzler, {[}nach dem 25. 9. 1925{]}]{ Stefan Großmann an Arthur Schnitzler, {[}nach dem 25. 9. 1925{]}}\nopagebreak\mylabel{v}\rehead{ }\begin{ledgroupsized}[t]{13cm}\normalsize\beginnumbering\briefempfaengerindex{Schnitzler, Arthur@\textsc{Schnitzler, Arthur}!zzzGrossmann, Stefan@\emph{von Stefan Großmann}!1925-09-251@{{[}nach dem
                  25. 9. 1925{]}}|(be} \toendnotes[C]{\smallbreak\pagebreak[2]} \Standort{DLA, A:Schnitzler, HS.NZ85.1.3232.}
\physDesc{Brief, 1 Blatt, 1 Seite
\newline{}Schreibmaschine
\newline{}Handschrift Arthur Schnitzler: roter Buntstift, deutsche Kurrent (\noindent{}Nummerierung:
                                    »25«; eine Unterstreichung)}\toendnotes[C]{\smallbreak}\pstart
           \noindent{}\centering{}{\pb}\textcolor{gray}{\textbf{Das Tage-Buch\orgindex{Tage-Buch@Das Tage-Buch|pw}}}\pend
           \pstart
           \noindent{}\centering{}\textcolor{gray}{\textbf{\emph{Herausgeber: Stefan Großmann und Leopold Schwarzschild\pwindex{Schwarzschild, Leopold 1891-12-08 – 1950-10-02@\textsc{Schwarzschild, Leopold} (1891-12-08 – 1950-10-02), \emph{Publizist}|pw}}}}\pend
           \pstart
           \noindent{}\centering{}\textcolor{gray}{\textbf{Tagebuchverlag m. b. H., Berlin
                        SW 19\oindex{Berlin@\textbf{Berlin}|pw}}}\pend
           \pstart
           \noindent{}\centering{}\textcolor{gray}{\textbf{BEUTHSTRASSE 19\oindex{Beuthstrasse@\textbf{Beuthstrasse}|pw}}}\pend
           \pstart
           \noindent{}\centering{}\textcolor{gray}{\textbf{\emph{Telegramm-Adresse: Tagebuch Berlin\oindex{Berlin@\textbf{Berlin}|pw} ⋅ Fernsprecher: Merkur 8790–8792}}}\pend
           \pstart
           \noindent{}\centering{}\textcolor{gray}{\textbf{\emph{\so{Sprechstunde der Redaktion: 12–1 Uhr}}}}\pend
           \pstart
           \noindent{}\centering{}\textcolor{gray}{\textbf{*}}\pend
           \pstart
           \noindent{}\raggedleft{}Herrn\pend
           \pstart
           \noindent{}\raggedleft{}Dr. Arnold \so{Schnitzler}\pend
           \pstart
           \noindent{}\raggedleft{}\so{Wien } XVIII\oindex{XVIII., Waehring@\textbf{XVIII., Währing}|pw}\pend
           \pstart
           \noindent{}\raggedleft{}Sternwartestr. 71\oindex{Sternwartestrasse@\textbf{Sternwartestraße}|pw}. \pend
           \pstart\center{}Sehr verehrter Herr Doktor!\pend\pstart
           Herzlichen Dank für Ihre prinzipielle Zusage. Mein Dank wäre noch grösser, wenn Sie
               sich entschliessen würden, recht bald die nun versprochenen \label{K_L02451_1v}\edtext{Beiträge}{\lemma{\textnormal{\emph{Beiträge}}}\Cendnote{\textnormal{Schnitzler\pwindex{Schnitzler, Arthur 15.05.1862 – 21.10.1931@\textsc{Schnitzler, Arthur} (15.05.1862 – 21.10.1931), \emph{Schriftsteller, Mediziner}|pwk} hielt seine Zusage nicht. Ein halbes
                  Jahr später erschien ein Nachdruck
                     von Aphorismen\pwindex{?? Werk@Nicht ermittelte Verfasserinnen und Verfasser!Bemerkungen1926-05-29 – 1926-05-29@\emph{Bemerkungen} {[}1926-05-29 – 1926-05-29{]}|pwkv} im \emph{Tage-Buch}\pwindex{Tage-Buch1920-01-01 – 1933-01-01@\emph{Das Tage-Buch}|pwk}.}}}\label{K_L02451_1h} zu
               senden. Ich wäre Ihnen für die Uebersendung von Beiträgen gerade jetzt, im Herbst,
               ganz besonders dankbar.\pend
           \pstart
           Mit ergebensten Grüssen{\\[\baselineskip]}Ihr{\\[\baselineskip]}\spacefill\mbox{{[}hs.:{]} Stefan Großmann}\pend
           \leftskip=0em{}\endnumbering\briefempfaengerindex{Schnitzler, Arthur@\textsc{Schnitzler, Arthur}!zzzGrossmann, Stefan@\emph{von Stefan Großmann}!1925-09-251@{{[}nach dem
                  25. 9. 1925{]}}|)be}\mylabel{h}\end{ledgroupsized}  \newcommand{\dateiname}{L02451}\newcommand{\titel}{Stefan Großmann an Arthur Schnitzler, [nach dem 25. 9. 1925]}\newcommand{\editorInnen}{ Martin Anton Müller und Gerd-Hermann Susen}\input{../tex-inputs/latex-pdf-abspann}
      