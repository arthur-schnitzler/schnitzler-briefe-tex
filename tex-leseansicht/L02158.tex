%% latex-leseansicht-vorspann.tex
%% Vorspann für die Leseansicht.
%% Lädt die gemeinsame Datei latex-vorspann.tex mit nicht gesetztem Schalter.

\newif\ifkorrekturansicht
\korrekturansichtfalse

\input{../tex-inputs/latex-vorspann}


\section[Stefan Großmann an Arthur Schnitzler, 30. 11. 1913]{L02158 Stefan Großmann an Arthur Schnitzler, 30. 11. 1913}
\nopagebreak\mylabel{L02158v}
\rehead{ }\normalsize\beginnumbering\briefempfaengerindex{Schnitzler, Arthur@\textsc{Schnitzler, Arthur}!zzzGroßmann, Stefan@\emph{von Stefan Großmann}!1913-11-301@{30. 11. 1913}|(be}
\toendnotes[C]{\smallbreak\pagebreak[2]}
\correspDesc{Versand  durch Stefan Großmann am 30. 11. 1913 in Wien
\newline{}Erhalt  durch Arthur Schnitzler im Zeitraum [30. 11. 1913 – 4. 12. 1913?] in Wien}\toendnotes[C]{\smallbreak}
\Standort{CUL, Schnitzler, B 34.}
\physDesc{Brief, 1 Blatt, 1 Seite, 753 Zeichen
\newline{}Handschrift: schwarze Tinte, deutsche Kurrent
\newline{}Ordnung: mit Bleistift von unbekannter Hand nummeriert:
                                    »13« }
\pstart
           {\pb}\textcolor{gray}{\textbf{STEFAN GROSSMANN}}\hfill \textcolor{gray}{\textbf{WIEN,\oindex{Wien@\textbf{Wien}, \emph{Verwaltungsgebiet}|pw}}}{ }30. \strikeout{Dece} Nov. 1913\pend
           
\pstart
           \raggedleft{}I. \textsc{Dominikanerbastei}
                     5\oindex{Dominikanerbastei@\textbf{Dominikanerbastei}, \emph{Straße}|pw}\pend
           
\pstart
           Herrn D\textsuperscript{r} Arthur Schnitzler\pend
           
\pstart
           \raggedleft{}\uline{Wien}\oindex{Wien@\textbf{Wien}, \emph{Verwaltungsgebiet}|pw}\pend
           
\pstart{}Sehr verehrter Herr.\pend\vspace{0.5em}
\pstart
           Im Aufträge des Verlegers \uline{Ullſtein\orgindex{Ullstein Verlag@Ullstein Verlag|pw}} in Berlin\oindex{Berlin@\textbf{Berlin}, \emph{Hauptstadt}|pw}, der die »\textsc{Vossische Zeitung\orgindex{Vossische Zeitung@Vossische Zeitung|pw}}«, die »\textsc{B. Z. am Mittag\orgindex{B.Z. am Mittag@B.Z. am Mittag|pw}}« u die »\textsc{Morgenpost}\orgindex{Berliner Morgenpost@Berliner Morgenpost|pw}« herausgibt und in deſſen Dienſte ich getreten bin, habe ich an Sie die
               höflichſte und dringende Bitte zu richten, ob Sie, verehrter Herr Doktor, dem Verlage
               aus Anlaſs einer Jubiläumsnummer der \textsc{Morgenpost}\orgindex{Berliner Morgenpost@Berliner Morgenpost|pw}, die jetzt einen Abonnentenſtand von 400.000 Abonennten erreicht, einen kleinen
               Beitrag überlaſſen wollten.\pend
           
\pstart
           Der Verlag Ullſtein\orgindex{Ullstein Verlag@Ullstein Verlag|pw} bittet Sie darum inſtändig.
               Die beſten deutſchen Autoren{ }ſind in dieser N\textsuperscript{o} vertreten.
               Wollen Sie{ }ſelbſt, wenn Sie Ihren Beitrag ankündigen, das Honorar beſtimmen.\pend
           
\pstart
           Mit außerordentlicher Hochschätzung{\\[\baselineskip]}sehr ergeben: \spacefill\mbox{Stefan
                  Großmann}\pend
           \leftskip=0em{}\selectlanguage{ngerman}\endnumbering\briefempfaengerindex{Schnitzler, Arthur@\textsc{Schnitzler, Arthur}!zzzGroßmann, Stefan@\emph{von Stefan Großmann}!1913-11-301@{30. 11. 1913}|)be}\mylabel{L02158h}  \newcommand{\dateiname}{L02158}\newcommand{\titel}{Stefan Großmann an Arthur Schnitzler, 30. 11. 1913}\newcommand{\editorInnen}{Martin Anton Müller und Gerd-Hermann Susen}%% latex-leseansicht-abspann.tex
%% Abspann für die Leseansicht.
%% Der Schalter \ifkorrekturansicht ist bereits durch den Vorspann gesetzt.

%% latex-abspann.tex
%% Gemeinsamer Abspann für Korrekturansicht und Leseansicht.
%% Setzt den Schalter \ifkorrekturansicht voraus (gesetzt in den
%% einbindenden Dateien latex-korrekturansicht-abspann.tex bzw.
%% latex-leseansicht-abspann.tex).
%% ---------------------------------------------------------------

\normalsize

% Das esempio-Environment wird nur in der Leseansicht benötigt
\ifkorrekturansicht\else
\newenvironment{esempio}[3]%
{
    \vspace{1.5ex}
    \rlap{\underline{#1}}
    \par
    \setlength{\parindent}{0cm}
    \nopagebreak
    \leftskip=#2cm
    \rightskip=#3cm
}
{
    \par
}
\fi

\doendnotes{C}
\bigskip
\vfill

\clearpage

\footnotesize

\ifkorrekturansicht
  \lohead{\textsc{register}}
\fi

% theindex-Environment neu definieren ohne reledmac
\makeatletter
\renewenvironment{theindex}{%
  \ifkorrekturansicht
    \section*{\indexname}%
  \else
    \subsubsection*{Index der erwähnten Entitäten}%
  \fi
  \setlength{\parindent}{0pt}%
  \setlength{\parskip}{0pt plus 0.3pt}%
  \let\item\@idxitem
}{%
  \ifkorrekturansicht\clearpage\fi
}
\makeatother

\IfFileExists{\jobname-pw.ind}{\input{\jobname-pw.ind}}{}

% Quellenangabe nur in der Leseansicht
\ifkorrekturansicht\else
% Fallback-Definitionen, falls die .tex-Datei \titel etc. nicht gesetzt hat
\providecommand{\titel}{}
\providecommand{\editorInnen}{}
\providecommand{\dateiname}{\jobname}

\vspace{3cm}

\vfill

\footnotesize
\textsc{Quelle}: \titel. Herausgegeben von {\editorInnen}. In: \emph{Arthur Schnitzler: Briefwechsel mit Autorinnen und Autoren}.
 Digitale Edition, https://schnitzler-briefe.acdh.oeaw.ac.at/{\dateiname}.html (Stand \today)
\fi

\end{document}


