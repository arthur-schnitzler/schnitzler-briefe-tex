%% latex-leseansicht-vorspann.tex
%% Vorspann für die Leseansicht.
%% Lädt die gemeinsame Datei latex-vorspann.tex mit nicht gesetztem Schalter.

\newif\ifkorrekturansicht
\korrekturansichtfalse

\input{../tex-inputs/latex-vorspann}


\section[Arthur Schnitzler an Gustav Schwarzkopf, 15. 6. 1905]{L04070 Arthur Schnitzler an Gustav Schwarzkopf, 15. 6. 1905}
\nopagebreak\mylabel{L04070v}
\rehead{ }\normalsize\beginnumbering\briefempfaengerindex{Schwarzkopf, Gustav@\textsc{Schwarzkopf, Gustav}!zzzSchnitzler, Arthur@\emph{von Arthur Schnitzler}!1905-06-151@{15. 6. 1905}|(be}
\toendnotes[C]{\smallbreak\pagebreak[2]}
\correspDesc{Versand  durch Arthur Schnitzler am 15. 6. 1905 in Wien
\newline{}Erhalt  durch Gustav Schwarzkopf im Zeitraum [15. 6. 1905
                  – 18. 6. 1905?] in Wien}\toendnotes[C]{\smallbreak}
\Standort{CUL, Schnitzler, B 96.}
\physDesc{Brief, 1 Blatt, 2 Seiten, 398 Zeichen
\newline{}Handschrift: Bleistift, deutsche Kurrent}\toendnotes[C]{\smallbreak}
\pstart
           \raggedleft{}{\pb}Wien\oindex{Wien@\textbf{Wien}, \emph{Verwaltungsgebiet}|pw}{ }15. 6. 905.\pend
           \vspace{0.5em}
\pstart
           lieber Guſtav, der Dramaturg und \textsc{Secretair}{ }\label{K_L04070-1v}\edtext{Kahane\pwindex{Kahane, Arthur 2.\,5.\,1872 Iași – 7.\,10.\,1932 Berlin@\textsc{Kahane, Arthur} (2.\,5.\,1872 Iași – 7.\,10.\,1932 Berlin), \emph{Schriftsteller, Dramaturg}|pw} iſt geſtern erſt{ }ſpät ins Theater\oindex{Wien@\textbf{Wien}!VI., Mariahilf@\textbf{VI., Mariahilf}!Theater an der Wien@\textbf{Theater an der Wien}, \emph{Theater}|pw}}{\lemma{\textnormal{\emph{Kahane … Theater}}}\Cendnote{\textnormal{Max Reinhardt\pwindex{Reinhardt, Max 9.\,9.\,1873 Baden bei Wien – 30.\,10.\,1943 New York City@\textsc{Reinhardt, Max} (9.\,9.\,1873 Baden bei Wien – 30.\,10.\,1943 New York City), \emph{Theaterleiter, Regisseur, Schauspieler}|pwk} gastierte mit dem \emph{Kleinen Theater}\orgindex{Kleines Theater@Kleines Theater|pwk} und dem \emph{Neuen Theater}\orgindex{Neues Theater@Neues Theater|pwk} im Theater an
                     der Wien\oindex{Wien@\textbf{Wien}!VI., Mariahilf@\textbf{VI., Mariahilf}!Theater an der Wien@\textbf{Theater an der Wien}, \emph{Theater}|pwk}. Schnitzler dürfte aber am
                  Vortag selbst nicht im Theater gewesen sein, auch wenn das hier implizit zum
                  Ausdruck zu kommen scheint.}}}\label{K_L04070-1} gekommen und hat Ihnen die \label{K_L04070-2v}\edtext{Sitze}{\lemma{\textnormal{\emph{Sitze}}}\Cendnote{\textnormal{Am 15. 6. 1905 wurde als letzter Auftritt des
                  Gastspiels \emph{Ein Sommernachtstraum}\pwindex{Shakespeare, William 23.\,4.\,1564? Stratford-upon-Avon – 3.\,5.\,1616 ebd.@\textsc{Shakespeare, William} (23.\,4.\,1564? Stratford-upon-Avon – 3.\,5.\,1616 ebd.), \emph{Schauspieler, Dramatiker}!Midsommer nights dreame@\strich\emph{A Midsommer nights dreame}|pwk}\eventindex{Theater an der Wien@\textbf{Theater an der Wien}!Aufführung von Ein Sommernachtstraum, 15.6.1905@Aufführung von Ein Sommernachtstraum, 15.6.1905|pwkv} gegeben.}}}\label{K_L04070-2} dann gleich hinunterlegen laſſen – Sie waren ſchon dageweſen
               u wieder fortgegangen – er bittet mich Ihnen das zu ſagen und ihn {\pb}in dieſem Sinn \introOben{}bei
                  Ihnen\introOben{} zu entſchuldigen. Was hiemit geſchieht. Mir hat es ſehr Leid
               gethan.\pend
           
\pstart
           Auf Wiedersehen also am{\\[\baselineskip]}\label{K_L04070-3v}\edtext{So{\geminationn}tag}{\lemma{\textnormal{\emph{Sonntag}}}\Cendnote{\textnormal{Vgl. A. S.: \emph{Kulturveranstaltungen}, 18. 6. 1905. }}}\label{K_L04070-3} und
               herzlichen{\\[\baselineskip]} Gruſs.{\\[\baselineskip]} Ihr{\\[\baselineskip]}\spacefill\mbox{A.}\pend
           \leftskip=0em{}\selectlanguage{ngerman}\endnumbering\briefempfaengerindex{Schwarzkopf, Gustav@\textsc{Schwarzkopf, Gustav}!zzzSchnitzler, Arthur@\emph{von Arthur Schnitzler}!1905-06-151@{15. 6. 1905}|)be}\mylabel{L04070h}
\begin{anhang}
\end{anhang}\newcommand{\dateiname}{L04070}\newcommand{\titel}{Arthur Schnitzler an Gustav Schwarzkopf, 15. 6. 1905}\newcommand{\editorInnen}{Herausgegeben von Jahnke, SelmaMüller, Martin Anton}%% latex-leseansicht-abspann.tex
%% Abspann für die Leseansicht.
%% Der Schalter \ifkorrekturansicht ist bereits durch den Vorspann gesetzt.

%% latex-abspann.tex
%% Gemeinsamer Abspann für Korrekturansicht und Leseansicht.
%% Setzt den Schalter \ifkorrekturansicht voraus (gesetzt in den
%% einbindenden Dateien latex-korrekturansicht-abspann.tex bzw.
%% latex-leseansicht-abspann.tex).
%% ---------------------------------------------------------------

\normalsize

% Das esempio-Environment wird nur in der Leseansicht benötigt
\ifkorrekturansicht\else
\newenvironment{esempio}[3]%
{
    \vspace{1.5ex}
    \rlap{\underline{#1}}
    \par
    \setlength{\parindent}{0cm}
    \nopagebreak
    \leftskip=#2cm
    \rightskip=#3cm
}
{
    \par
}
\fi

\doendnotes{C}
\bigskip
\vfill

\clearpage

\footnotesize

\ifkorrekturansicht
  \lohead{\textsc{register}}
\fi

% theindex-Environment neu definieren ohne reledmac
\makeatletter
\renewenvironment{theindex}{%
  \ifkorrekturansicht
    \section*{\indexname}%
  \else
    \subsubsection*{Index der erwähnten Entitäten}%
  \fi
  \setlength{\parindent}{0pt}%
  \setlength{\parskip}{0pt plus 0.3pt}%
  \let\item\@idxitem
}{%
  \ifkorrekturansicht\clearpage\fi
}
\makeatother

\IfFileExists{\jobname-pw.ind}{\input{\jobname-pw.ind}}{}

% Quellenangabe nur in der Leseansicht
\ifkorrekturansicht\else
% Fallback-Definitionen, falls die .tex-Datei \titel etc. nicht gesetzt hat
\providecommand{\titel}{}
\providecommand{\editorInnen}{}
\providecommand{\dateiname}{\jobname}

\vspace{3cm}

\vfill

\footnotesize
\textsc{Quelle}: \titel. Herausgegeben von {\editorInnen}. In: \emph{Arthur Schnitzler: Briefwechsel mit Autorinnen und Autoren}.
 Digitale Edition, https://schnitzler-briefe.acdh.oeaw.ac.at/{\dateiname}.html (Stand \today)
\fi

\end{document}


