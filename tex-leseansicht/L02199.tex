%% latex-leseansicht-vorspann.tex
%% Vorspann für die Leseansicht.
%% Lädt die gemeinsame Datei latex-vorspann.tex mit nicht gesetztem Schalter.

\newif\ifkorrekturansicht
\korrekturansichtfalse

\input{../tex-inputs/latex-vorspann}


               \section[Arthur Schnitzler an Georg Brandes, 20. 10. 1914]{ Arthur Schnitzler an Georg Brandes, 20. 10. 1914}\nopagebreak\mylabel{v}\rehead{ }\begin{ledgroupsized}[t]{13cm}\normalsize\beginnumbering\briefempfaengerindex{Brandes, Georg@\textsc{Brandes, Georg}!zzzSchnitzler, Arthur@\emph{von Arthur Schnitzler}!1914-10-201@{20. 10. 1914}|(be} \toendnotes[C]{\smallbreak\pagebreak[2]} \Standort{Kopenhagen, Det Kongelige Bibliotek, Georg Brandes Arkiv, box 125.}
\physDesc{Brief, 3 Blätter (Blatt 2 und 3 mit Schreibmaschine paginiert), 5 Seiten
\newline{}Schreibmaschine
\newline{}Handschrift: schwarze Tinte, lateinische Kurrent (\noindent{}fünf Sofortkorrekturen, Überarbeitung, Unterschrift)\newline{}Ordnung: mit Bleistift am
                                    ersten Blatt nummeriert: »36.«. Das
                                    zweite und dritte Blatt datiert mit »20/10 14« \newline{}Editorischer Hinweis: Die Sofortkorrekturen sind nicht ausgewiesen. }\buchAbdrucke{\weitereDrucke{1) Georg Brandes, Arthur Schnitzler: \emph{Ein Briefwechsel}. Hg. Kurt Bergel. Bern: \emph{Francke} 1956, S. 111–113.} \weitereDrucke{2) Arthur Schnitzler: \emph{Briefe 1913–1931}. Hg. Peter Michael Braunwarth, Richard Miklin, Susanne Pertlik und Heinrich Schnitzler. Frankfurt am Main: \emph{S. Fischer} 1984, S. 49–51.} }\toendnotes[C]{\smallbreak}\pstart
           \noindent{}{\pb}\textcolor{gray}{\textbf{Dr. Arthur Schnitzler}}{\\}\textcolor{gray}{\textbf{Wien XVIII. Sternwartestrasse 71\oindex{Sternwartestrasse@\textbf{Sternwartestraße}|pw}}}\pend
           \pstart
           \raggedleft{}20. 10. 1914. \pend
           \pstart\center{}Lieber und verehrter Freund.\pend\pstart
           Erst heute danke ich Ihnen für Ihren Brief vom 23. August d.J., der
                    immerhin schon am 10. September bei mir eingetroffen ist. Wie
                    lächerlich muss Ihnen mein Schreiben aus Celerina\oindex{Celerina@\textbf{Celerina}|pw} erschienen sein, das schon mitten in den Stürmen des
                    Weltkrieges bei Ihnen eintraf; kam es doch mir selbst schon am Tage, da ich es
                    absandte, recht unzeitgemäss vor. Aber die Gründe, die mich bewogen es doch
                    nicht zurückzuhalten habe ich Ihnen ja \strikeout{doch}
               schon damals dargelegt\introOben{},\introOben{} und ich kann heute wirklich nur um Entschuldigung
                    bitten, dass ich Sie in solcher Zeit überhaupt mit einem Privatwunsch bemüht
                    oder wenigstens gelangweilt habe. Wenn aber auch Privatwünsche jetzt notwendig
                    und gerne zum Schweigen verurteilt sind und jedes Privatinteresse irgendwie und
                    irgendwo mit dem Allgemeininteresse verbunden scheint; es führt {\pb}doch jeder, ob er nun will oder nicht, auch
                    seine Privatexistenz weiter und man gibt am Ende der Zeit auch etwas, indem man
                    sich selber zu bewahren sucht, so weit es ohne Schaden für die Allgemeinheit
                    möglich ist. Allzu viele sieht man heute, die in einem ins Leere gewandten
                    Betätigungstrieb sich nutzlos verschwenden und Neigung zeigen sich einem äussern
                    oder innern Beruf zu entfremden, innerhalb dessen, \substVorne{}\textsuperscript{beim}\substDazwischen{}durch den\substHinten{} Versuch ruhiger Weiterarbeit, sie die Sache ihres Vaterlandes, wenn
                    auch nur mittelbar, aufs Beste fördern könnten. Freilich ist \introOben{}es eine sehr begreifliche\introOben{}\strikeout{die}{ }Sehnsucht
                    von Vielen, die nicht gerade in der Front stehen, oder sonstwie zu
                    Kriegsdienstleistungen herangezogen sind, am allgemeinen Schicksal in
                    deutlicherer Weise teilzunehmen, als es sich durch Fortführung ihrer
                    Friedensarbeiten offenbaren würde, und man kann sagen, dass auch auf diese Art
                    heute viel Gutes, besonders auf dem Gebiete der Wohltätigkeit,geleistet
                    wird.\pend
           \pstart
           Es ist sehr wahrscheinlich, dass {\pb}Dänemark\oindex{Daenemark@\textbf{Dänemark}|pw}, sowie die andern neutralen Staaten,
                    mehr von dem für Deutschland\oindex{Deutschland@\textbf{Deutschland}|pw} und Oesterreich\oindex{Oesterreich@\textbf{Österreich}|pw} Ungünstigen als von dem Günstigen
                    zu lesen bekommt. Wir sind hier jedenfalls immer wieder von Neuem starr über die
                    ungeheuerlichen Lügen, die in der ausländischen Presse nicht nur von den
                    Ereignissen im Feld, sondern auch von den inneren Zuständen unseres Landes
                    verbreitet werden. Obwohl ich annehmen kann, dass Sie im Ganzen \label{LL710-1v}\label{LL710-1h}leidlich orientiert sein werden, so möchte
                    ich Ihnen doch jedesfalls mitteilen, dass in Wien\oindex{Wien@\textbf{Wien}|pw}, im Gegensatz zu den Gerüchten von Teuerung und dergleichen,
                    völlig geordnete Zustände herrschen, dass beinahe nirgends nennenswerte
                    Preissteigerungen erfolgt sind, dass die ökonomischen \substVorne{}\textsuperscript{Zustände}{\allowbreak}\substDazwischen{}Verhältnisse\substHinten{} hier nicht nur nicht schlechter, sondern besser sind als im vorigen
                    Jahr, dass für die Arbeitslosen, deren Zahl, wie mir selbst ganz merkwürdig
                    erscheint, in diesen Jahr nicht grösser sein soll als im vorigen, von
                    öffentlicher und privater Seite ausreichend {\pb}gesorgt wird, dass alle Theater spielen, die meisten bei sehr gutem Besuch,
                    und dass einem das ganze Elend des Kriegs eigentlich nur dort vor Augen tritt,
                    wo die einzigen, bisher unbezweifelbaren Resultate desselben zu sehen sind,
                    nämlich in den Spitälern, wo die Verwundeten liegen{\dotstwo}
                    Aber auch von dort bringt man keineswegs ausschliesslich trübe Eindrücke nach
                    Hause. Denn beinahe alle Soldaten und Offiziere, die vom Kriegsschauplatz nach
                    Hause kommen, auch wenn sie sehr Grauenhaftes zu erzählen wissen, sind von
                    grosser Zuversicht erfüllt, ja, sie waren es auch schon zu einer Zeit, wo die
                    Stimmung der Bevölkerung in manchen Kreisen zu wünschen übrig liess. Aber auch
                    das ist in den letzten Wochen, in denen durchaus gute, glaubwürdig gute
                    Nachrichten bei uns eintreffen, anders geworden und diese Hoffnungsfreudigkeit
                    ist im äussern und innern Leben unserer Stadt nicht zu verkennen.\pend
           \pstart
           Ich wünschte sehr zu erfahren, ob {\pb}Sie von
                    Ihrem im Feld stehenden Schwiegersohn\pwindex{Philipp, Reinhold 15.08.1883 – 1968@\textsc{Philipp, Reinhold} (15.08.1883 – 1968), \emph{Fabrikant}|pwv} Gutes hören. Was uns bisher die Feldpost
                    bisher gebracht hat, so weit es sich auf persönliche Bekannte und
                    Freunde bezieht, ist von beruhigender Art gewesen.\pend
           \pstart
           Vor Voraussagen wollen wir uns hüten; unsere Wünsche sind zu selbstverständlich,
                    als dass wir sie erst aussprechen müssten. Und doch, wieviel Unheil, nicht nur
                    für Schuldige, sondern auch für Unschuldige flehen wir, nicht einmal ganz
                    gedankenlos, durch unsere Wünsche herab. Ja, nach den Einrichtungen dieser Welt
                    ist sogar zu befürchten, dass mancher von den Allerschuldigsten ganz ohne Strafe
                    ausgehen wird. Aber ziemt es sich \introOben{}denn\introOben{} in dieser
                    überwältigend grauenhaften Epoche derartige Worte \introOben{}wie\introOben{}{ }Schuld, Strafe, Verantwortung,
                    zu gebrauchen\substVorne{}\textsuperscript{, a}\substDazwischen{}? A\substHinten{}lles Philosophische und Ethische verlischt im Sturmhauch der
                    Geschichte.\pend
           \pstart
           Bitte schreiben Sie mir bald wie es Ihnen geht. Meine Frau\pwindex{Schnitzler, Olga 17.01.1882 – 13.01.1970@\textsc{Schnitzler, Olga} (17.01.1882 – 13.01.1970), \emph{Schauspielerin, Sängerin}|pwv} und ich grüssen Sie herzlichst{\\[\baselineskip]}{[}hs.:{]} Ihr{\\[\baselineskip]}\spacefill\mbox{Arthur Schnitzler}\pend
           \leftskip=0em{}\endnumbering\briefempfaengerindex{Brandes, Georg@\textsc{Brandes, Georg}!zzzSchnitzler, Arthur@\emph{von Arthur Schnitzler}!1914-10-201@{20. 10. 1914}|)be}\mylabel{h}\end{ledgroupsized}  \newcommand{\dateiname}{L02199}\newcommand{\titel}{Arthur Schnitzler an Georg Brandes, 20. 10. 1914}\newcommand{\editorInnen}{Martin Anton Müller und Gerd-Hermann Susen}%% latex-leseansicht-abspann.tex
%% Abspann für die Leseansicht.
%% Der Schalter \ifkorrekturansicht ist bereits durch den Vorspann gesetzt.

%% latex-abspann.tex
%% Gemeinsamer Abspann für Korrekturansicht und Leseansicht.
%% Setzt den Schalter \ifkorrekturansicht voraus (gesetzt in den
%% einbindenden Dateien latex-korrekturansicht-abspann.tex bzw.
%% latex-leseansicht-abspann.tex).
%% ---------------------------------------------------------------

\normalsize

% Das esempio-Environment wird nur in der Leseansicht benötigt
\ifkorrekturansicht\else
\newenvironment{esempio}[3]%
{
    \vspace{1.5ex}
    \rlap{\underline{#1}}
    \par
    \setlength{\parindent}{0cm}
    \nopagebreak
    \leftskip=#2cm
    \rightskip=#3cm
}
{
    \par
}
\fi

\doendnotes{C}
\bigskip
\vfill

\clearpage

\footnotesize

\ifkorrekturansicht
  \lohead{\textsc{register}}
\fi

% theindex-Environment neu definieren ohne reledmac
\makeatletter
\renewenvironment{theindex}{%
  \ifkorrekturansicht
    \section*{\indexname}%
  \else
    \subsubsection*{Index der erwähnten Entitäten}%
  \fi
  \setlength{\parindent}{0pt}%
  \setlength{\parskip}{0pt plus 0.3pt}%
  \let\item\@idxitem
}{%
  \ifkorrekturansicht\clearpage\fi
}
\makeatother

\IfFileExists{\jobname-pw.ind}{\input{\jobname-pw.ind}}{}

% Quellenangabe nur in der Leseansicht
\ifkorrekturansicht\else
% Fallback-Definitionen, falls die .tex-Datei \titel etc. nicht gesetzt hat
\providecommand{\titel}{}
\providecommand{\editorInnen}{}
\providecommand{\dateiname}{\jobname}

\vspace{3cm}

\vfill

\footnotesize
\textsc{Quelle}: \titel. Herausgegeben von {\editorInnen}. In: \emph{Arthur Schnitzler: Briefwechsel mit Autorinnen und Autoren}.
 Digitale Edition, https://schnitzler-briefe.acdh.oeaw.ac.at/{\dateiname}.html (Stand \today)
\fi

\end{document}


      