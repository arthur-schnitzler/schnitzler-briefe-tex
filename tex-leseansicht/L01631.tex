%% latex-leseansicht-vorspann.tex
%% Vorspann für die Leseansicht.
%% Lädt die gemeinsame Datei latex-vorspann.tex mit nicht gesetztem Schalter.

\newif\ifkorrekturansicht
\korrekturansichtfalse

\input{../tex-inputs/latex-vorspann}


         
         \renewcommand{\erwaehntePersonen}{Personen: Josef Kainz, Max Mell, Maria Mell, Olga Schnitzler}
         \renewcommand{\erwaehnteInstitutionen}{Institutionen: Frankfurter Zeitung}
         \renewcommand{\erwaehnteOrte}{Orte: Berlin, II., Leopoldstadt, Wien, Wittelsbachstraße}
         \renewcommand{\erwaehnteWerke}{Werke: Die Komödianten, Über die Briefe Beethovens}
               \section[Max Mell an Arthur Schnitzler, 11. 10. 1906]{ Max Mell an Arthur Schnitzler, 11. 10. 1906}\nopagebreak\mylabel{v}\rehead{ }\begin{ledgroupsized}[t]{13cm}\normalsize\beginnumbering \toendnotes[C]{\smallbreak\pagebreak[2]} \Standort{CUL, Schnitzler, B 70.}
\physDesc{Brief, 1 Blatt, 1 Seite, 856 Zeichen
\newline{}Handschrift: schwarze Tinte, deutsche Kurrent
\newline{}Schnitzler: 1) mit rotem Buntstift beschriftet: »\textsc{Mell}«  2) mit rotem Buntstift eine Unterstreichung}\toendnotes[C]{\smallbreak}\pstart
           \noindent{}\raggedleft{}{\pb}Wien II.\oindex{II., Leopoldstadt@\textbf{II., Leopoldstadt}|pw}{ }Wittelsbacherſtr\oindex{Wittelsbachstrasse@\textbf{Wittelsbachstraße}|pw}. 5.\pend
           \pstart
           \raggedleft{}11. Oktober 1906.\pend
           \pstart{}Sehr verehrter Herr Doktor,\pend\pstart
           ich nehme mir die Freiheit, Ihnen mein Stück\pwindex{Mell, Max 10.11.1882 – 13.12.1971@\textsc{Mell, Max} (10.11.1882 – 13.12.1971), \emph{Schriftsteller}!Komoedianten@\strich\emph{Die Komödianten}|pwv} zu überreichen, ermutigt durch Sie ſelbſt und in
               Ungeduld, denen auch als Dramatiker bekannt zu werden, die ſich meiner Novellen
               erinnern. Mein Ziel ist die Komödie; und hoffentlich werden Sie mir die Fähigkeit, es
               zu erreichen, zuſprechen.\pend
           \pstart
           Darf ich auch einen kleinen Aufſatz\pwindex{Mell, Max 10.11.1882 – 13.12.1971@\textsc{Mell, Max} (10.11.1882 – 13.12.1971), \emph{Schriftsteller}!Ueber die Briefe Beethovens1906-10-04@\strich\emph{Über die Briefe Beethovens} {[}1906-10-04{]}|pwv} aus der Frankfurter Zeitung\orgindex{Frankfurter Zeitung@Frankfurter Zeitung|pw}
               beilegen?\pend
           \pstart
           Vielleicht geben Sie das Manuſkript gelegentlich meiner Schweſter\pwindex{Mell, Maria 12.07.1885 – 29.10.1954@\textsc{Mell, Maria} (12.07.1885 – 29.10.1954), \emph{Schauspielerin}|pwv} zurück, wenn sie Ihre Frau Gemahlin\pwindex{Schnitzler, Olga 17.01.1882 – 13.01.1970@\textsc{Schnitzler, Olga} (17.01.1882 – 13.01.1970), \emph{Schauspielerin, Sängerin}|pwv} beſucht, auch werde
               ich mir erlauben, Ihnen meine Berlin\oindex{Berlin@\textbf{Berlin}|pw}er Adreſſe
               mitzuteilen. Ich hab das Stück\pwindex{Mell, Max 10.11.1882 – 13.12.1971@\textsc{Mell, Max} (10.11.1882 – 13.12.1971), \emph{Schriftsteller}!Komoedianten@\strich\emph{Die Komödianten}|pwv}
               in Berlin\oindex{Berlin@\textbf{Berlin}|pw} noch nirgends eingereicht, aber es an
                  Kainz\pwindex{Kainz, Josef 02.01.1858 – 20.09.1910@\textsc{Kainz, Josef} (02.01.1858 – 20.09.1910), \emph{Schauspieler}|pw} geſchickt.\pend
           \pstart
           Es wäre mir ſehr erfreulich, wenn auch Ihre Frau Gemahlin\pwindex{Schnitzler, Olga 17.01.1882 – 13.01.1970@\textsc{Schnitzler, Olga} (17.01.1882 – 13.01.1970), \emph{Schauspielerin, Sängerin}|pwv}, der ich mich beſtens zu empfehlen bitte, es leſen
               wollte.\pend
           \pstart
           Ich bin, in aufrichtiger Hochachtung{\\[\baselineskip]}Ihr ſehr ergebener{\\[\baselineskip]}\spacefill\mbox{Max Mell.}\pend
           \leftskip=0em{}
         
         \endnumbering\mylabel{h}\end{ledgroupsized}  \newcommand{\dateiname}{L01631}\newcommand{\titel}{Max Mell an Arthur Schnitzler, 11. 10. 1906}\newcommand{\editorInnen}{Martin Anton Müller und Gerd-Hermann Susen}%% latex-leseansicht-abspann.tex
%% Abspann für die Leseansicht.
%% Der Schalter \ifkorrekturansicht ist bereits durch den Vorspann gesetzt.

%% latex-abspann.tex
%% Gemeinsamer Abspann für Korrekturansicht und Leseansicht.
%% Setzt den Schalter \ifkorrekturansicht voraus (gesetzt in den
%% einbindenden Dateien latex-korrekturansicht-abspann.tex bzw.
%% latex-leseansicht-abspann.tex).
%% ---------------------------------------------------------------

\normalsize

% Das esempio-Environment wird nur in der Leseansicht benötigt
\ifkorrekturansicht\else
\newenvironment{esempio}[3]%
{
    \vspace{1.5ex}
    \rlap{\underline{#1}}
    \par
    \setlength{\parindent}{0cm}
    \nopagebreak
    \leftskip=#2cm
    \rightskip=#3cm
}
{
    \par
}
\fi

\doendnotes{C}
\bigskip
\vfill

\clearpage

\footnotesize

\ifkorrekturansicht
  \lohead{\textsc{register}}
\fi

% theindex-Environment neu definieren ohne reledmac
\makeatletter
\renewenvironment{theindex}{%
  \ifkorrekturansicht
    \section*{\indexname}%
  \else
    \subsubsection*{Index der erwähnten Entitäten}%
  \fi
  \setlength{\parindent}{0pt}%
  \setlength{\parskip}{0pt plus 0.3pt}%
  \let\item\@idxitem
}{%
  \ifkorrekturansicht\clearpage\fi
}
\makeatother

\IfFileExists{\jobname-pw.ind}{\input{\jobname-pw.ind}}{}

% Quellenangabe nur in der Leseansicht
\ifkorrekturansicht\else
% Fallback-Definitionen, falls die .tex-Datei \titel etc. nicht gesetzt hat
\providecommand{\titel}{}
\providecommand{\editorInnen}{}
\providecommand{\dateiname}{\jobname}

\vspace{3cm}

\vfill

\footnotesize
\textsc{Quelle}: \titel. Herausgegeben von {\editorInnen}. In: \emph{Arthur Schnitzler: Briefwechsel mit Autorinnen und Autoren}.
 Digitale Edition, https://schnitzler-briefe.acdh.oeaw.ac.at/{\dateiname}.html (Stand \today)
\fi

\end{document}


      