%% latex-korrekturansicht-vorspann.tex
%% Vorspann für die Korrekturansicht.
%% Lädt die gemeinsame Datei latex-vorspann.tex mit gesetztem Schalter.

\newif\ifkorrekturansicht
\korrekturansichttrue

\input{../tex-inputs/latex-vorspann}


\section[Max Mell an Arthur Schnitzler, 11. 10. 1906]{L01631 Max Mell an Arthur Schnitzler, 11. 10. 1906}
\nopagebreak\mylabel{L01631v}
\rehead{ }\normalsize\beginnumbering\briefempfaengerindex{Schnitzler, Arthur@\textsc{Schnitzler, Arthur}!zzzMell, Max@\emph{von Max Mell}!1906-10-111@{11. 10. 1906}|(be}
\toendnotes[C]{\smallbreak\pagebreak[2]}\Standort{CUL, Schnitzler, B 70.}
\physDesc{Brief, 1 Blatt, 1 Seite, 856 Zeichen
\newline{}Handschrift: schwarze Tinte, deutsche Kurrent
\newline{}Schnitzler: 1) mit rotem Buntstift beschriftet: »\textsc{Mell}«  2) mit rotem Buntstift eine Unterstreichung}\toendnotes[C]{\smallbreak}
\pstart
           \raggedleft{}{\pb}Wien II.\oindex{II., Leopoldstadt@\textbf{II., Leopoldstadt}, \emph{A.ADM3}|pw}{ }Wittelsbacherſtr\oindex{Wittelsbachstrasse@\textbf{Wittelsbachstraße}, \emph{Straße (K.STR)}|pw}. 5.\pend
           
\pstart
           \raggedleft{}11. Oktober 1906.\pend
           
\pstart{}Sehr verehrter Herr Doktor,\pend\vspace{0.5em}
\pstart
           ich nehme mir die Freiheit, Ihnen mein Stück\pwindex{Komoedianten@\emph{Die Komödianten}|pwv} zu überreichen, ermutigt durch Sie ſelbſt und in
               Ungeduld, denen auch als Dramatiker bekannt zu werden, die ſich meiner Novellen
               erinnern. Mein Ziel ist die Komödie; und hoffentlich werden Sie mir die Fähigkeit, es
               zu erreichen, zuſprechen.\pend
           
\pstart
           Darf ich auch einen kleinen Aufſatz\pwindex{Ueber die Briefe Beethovens@\emph{Über die Briefe Beethovens}|pwv} aus der Frankfurter Zeitung\orgindex{Frankfurter Zeitung@Frankfurter Zeitung|pw}
               beilegen?\pend
           
\pstart
           Vielleicht geben Sie das Manuſkript gelegentlich meiner Schweſter\pwindex{Mell, Maria 12.07.1885 – 29.10.1954@\textsc{Mell, Maria} (12.07.1885 – 29.10.1954), \emph{Schauspieler/Schauspielerin}|pwv} zurück, wenn sie Ihre Frau Gemahlin\pwindex{Schnitzler, Olga 17.01.1882 – 13.01.1970@\textsc{Schnitzler, Olga} (17.01.1882 – 13.01.1970), \emph{Schauspieler/Schauspielerin, Sänger/Sängerin}|pwv} beſucht, auch werde
               ich mir erlauben, Ihnen meine Berlin\oindex{Berlin@\textbf{Berlin}, \emph{P.PPLC}|pw}er Adreſſe
               mitzuteilen. Ich hab das Stück\pwindex{Komoedianten@\emph{Die Komödianten}|pwv}
               in Berlin\oindex{Berlin@\textbf{Berlin}, \emph{P.PPLC}|pw} noch nirgends eingereicht, aber es an
                  Kainz\pwindex{Kainz, Josef 02.01.1858 – 20.09.1910@\textsc{Kainz, Josef} (02.01.1858 – 20.09.1910), \emph{Schauspieler/Schauspielerin}|pw} geſchickt.\pend
           
\pstart
           Es wäre mir ſehr erfreulich, wenn auch Ihre Frau Gemahlin\pwindex{Schnitzler, Olga 17.01.1882 – 13.01.1970@\textsc{Schnitzler, Olga} (17.01.1882 – 13.01.1970), \emph{Schauspieler/Schauspielerin, Sänger/Sängerin}|pwv}, der ich mich beſtens zu empfehlen bitte, es leſen
               wollte.\pend
           
\pstart
           Ich bin, in aufrichtiger Hochachtung{\\[\baselineskip]}Ihr ſehr ergebener{\\[\baselineskip]}\spacefill\mbox{Max Mell.}\pend
           \leftskip=0em{}\selectlanguage{ngerman}\endnumbering\briefempfaengerindex{Schnitzler, Arthur@\textsc{Schnitzler, Arthur}!zzzMell, Max@\emph{von Max Mell}!1906-10-111@{11. 10. 1906}|)be}\mylabel{L01631h}  \normalsize

\doendnotes{C}
\bigskip
\vfill

\clearpage

\footnotesize

\lohead{\textsc{register}}

% Definiere theindex-Environment komplett neu ohne reledmac
\makeatletter
\renewenvironment{theindex}{%
  \section*{\indexname}%
  \setlength{\parindent}{0pt}%
  \setlength{\parskip}{0pt plus 0.3pt}%
  \let\item\@idxitem
}{%
  \clearpage
}
\makeatother

\IfFileExists{\jobname-pw.ind}{\input{\jobname-pw.ind}}{}

\end{document}

      