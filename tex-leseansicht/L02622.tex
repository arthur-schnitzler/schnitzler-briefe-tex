%% latex-leseansicht-vorspann.tex
%% Vorspann für die Leseansicht.
%% Lädt die gemeinsame Datei latex-vorspann.tex mit nicht gesetztem Schalter.

\newif\ifkorrekturansicht
\korrekturansichtfalse

\input{../tex-inputs/latex-vorspann}


\section[Paul Goldmann an Arthur Schnitzler, 28. 11. 1894]{L02622 Paul Goldmann an Arthur Schnitzler, 28. 11. 1894}
\nopagebreak\mylabel{L02622v}
\rehead{ }\normalsize\beginnumbering\briefempfaengerindex{Schnitzler, Arthur@\textsc{Schnitzler, Arthur}!zzzGoldmann, Paul@\emph{von Paul Goldmann}!1894-11-282@{28. 11. 1894}|(be}
\toendnotes[C]{\smallbreak\pagebreak[2]}
\correspDesc{Versand  durch Paul Goldmann am 28. 11. 1894 in Paris
\newline{}Erhalt  durch Arthur Schnitzler im Zeitraum [29. 11. 1894 – 3. 12. 1894?] in Wien}\toendnotes[C]{\smallbreak}
\Standort{DLA, A:Schnitzler, HS.NZ85.1.3164.}
\physDesc{Brief, 3 Blätter, 12 Seiten, 5565 Zeichen
\newline{}Handschrift: schwarze Tinte, deutsche Kurrent
\newline{}Schnitzler: 1) mit Bleistift auf dem ersten Blatt die Jahreszahl »94« vermerkt  2) mit rotem Buntstift acht Unterstreichungen}\toendnotes[C]{\smallbreak}
\pstart
           {\pb}\textcolor{gray}{\textbf{\textsc{Frankfurter Zeitung\orgindex{Frankfurter Zeitung@Frankfurter Zeitung|pw}}}}\hfill \textsc{Paris\oindex{Paris@\textbf{Paris}, \emph{Hauptstadt}|pw}}, 28. November.\pend
           
\pstart
           \textcolor{gray}{\textbf{(Gazette de
                     Francfort\orgindex{Frankfurter Zeitung@Frankfurter Zeitung|pw}).}}\pend
           
\pstart
           \textcolor{gray}{\textbf{\begin{otherlanguage}{french}Fondateur\end{otherlanguage}{ }\textbf{M. L. Sonnemann\pwindex{Sonnemann, Leopold 29.\,10.\,1831 Höchberg – 30.\,10.\,1909 Frankfurt am Main@\textsc{Sonnemann, Leopold} (29.\,10.\,1831 Höchberg – 30.\,10.\,1909 Frankfurt am Main), \emph{Journalist, Herausgeber}|pw}}.}}\pend
           
\pstart
           \textcolor{gray}{\textbf{\begin{otherlanguage}{french}Journal politique, financier,\end{otherlanguage}}}\pend
           
\pstart
           \textcolor{gray}{\textbf{\begin{otherlanguage}{french}commercial et littéraire.\end{otherlanguage}}}\pend
           
\pstart
           \textcolor{gray}{\textbf{\begin{otherlanguage}{french}\textbf{Paraissant trois fois par jour}\end{otherlanguage}}}.\pend
           
\pstart
           \textcolor{gray}{\textbf{\begin{otherlanguage}{french}\textbf{Bureaux à Paris\oindex{Paris@\textbf{Paris}, \emph{Hauptstadt}|pw}:}\end{otherlanguage}}}\pend
           
\pstart
           \textcolor{gray}{\textbf{\begin{otherlanguage}{french}\textbf{24. Rue Feydeau}\oindex{rue Feydeau@\textbf{rue Feydeau}, \emph{Straße}|pw}.\end{otherlanguage}}}\pend
           
\pstart\center{}Mein lieber Freund,\pend\vspace{0.5em}
\pstart
           Ich danke Dir von Herzen für die Überſendung von »Sterben\pwindex{Schnitzler, Arthur 15.\,5.\,1862 Wien – 21.\,10.\,1931 ebd.@\textsc{Schnitzler, Arthur} (15.\,5.\,1862 Wien – 21.\,10.\,1931 ebd.), \emph{Schriftsteller, Mediziner}!Sterben. Novelle@\strich\emph{Sterben. Novelle}|pw}«. Als ich den Schluß\pwindex{Schnitzler, Arthur 15.\,5.\,1862 Wien – 21.\,10.\,1931 ebd.@\textsc{Schnitzler, Arthur} (15.\,5.\,1862 Wien – 21.\,10.\,1931 ebd.), \emph{Schriftsteller, Mediziner}!Sterben. Novelle@\strich\emph{Sterben. Novelle}|pwv} las, hatte ich das Gefühl, daß{ }ſich der durch die verfluchten
               Fortſetzungen unterbrochene Strom wieder herſtellte. Der große Schauer kam –
               Ergriffenheit und Entzücken. Das Sterben iſt meiſterhaft geſchildert. Mich{ }ſtört nur
               das Erwürgen\substVorne{}\textsuperscript{.}\substDazwischen{},\substHinten{} – dieſes plötzliche Verfallen in die kriminaliſtiſche Brutalität, nachdem
                  \strikeout{es} vorher \strikeout{Alles}
               Alles eitel Freiheit, Seele, Stimmung geweſen. Ich glaube, das {\pb}hätte zweiſelhaft bleiben müſſen. Vielleicht{ }ſtellte{ }ſich das die überhitzte Phantaſie des Mädchens \introOben{}nur\introOben{}{ }ſo vor?
               Vielleicht wollte er{ }ſie umarmen? Wie{ }ſtört das noch rückwärts etwas das Bild des
               Unglücklichen. Er{ }ſoll Einer{ }ſein, der \uline{leidet}, bis
               zum Schluß. Das Handeln iſt{ }ſo unheimlich,{ }ſo gegen{ }ſeine Natur. Der erwürgt nicht,
               glaub’ mirs. Er weint nur, weil{ }ſie nicht mit ihm{ }ſterben will, das Sterben{ }ſelbſt
               wird ihm dadurch zur noch größeren Qual, er wird noch mehr \uline{leidend} zum Schluß. So denke ichs mir. Und {\pb}das Alles könnte erreicht werden, wenn nur ein einziger kleiner Satz am Schluſſe
               geſtrichen würde, wo das Mädel es klar{ }ſagt: »\label{K_L02622-1v}\edtext{Er hatte{ }ſie erwürgen wollen.}{\lemma{\textnormal{\emph{Er … wollen.}}}\Cendnote{\textnormal{Schnitzler änderte den Satz in späteren Auflagen\pwindex{Schnitzler, Arthur 15.\,5.\,1862 Wien – 21.\,10.\,1931 ebd.@\textsc{Schnitzler, Arthur} (15.\,5.\,1862 Wien – 21.\,10.\,1931 ebd.), \emph{Schriftsteller, Mediziner}!Sterben. Novelle@\strich\emph{Sterben. Novelle}|pwkv} nicht.}}}\label{K_L02622-1}«\pend
           
\pstart
           Vielleicht habe ich übrigens Unrecht. Denn ich habe das Buch\pwindex{Schnitzler, Arthur 15.\,5.\,1862 Wien – 21.\,10.\,1931 ebd.@\textsc{Schnitzler, Arthur} (15.\,5.\,1862 Wien – 21.\,10.\,1931 ebd.), \emph{Schriftsteller, Mediziner}!Sterben. Novelle@\strich\emph{Sterben. Novelle}|pwv} mit überſcharfer Kritik geleſen, weil ich
                  \strikeout{\textcolor{gray}{mir}} Dir{ }ſelbſt gegenüber ein unparteiiſches zu fällen mich verpflichtet fühlte und{ }ſtets auf der Lauer war, um nicht von meiner Freundſchaft überrumpelt zu werden.
               Sonſt iſt es wohl gelungen, das Buch\pwindex{Schnitzler, Arthur 15.\,5.\,1862 Wien – 21.\,10.\,1931 ebd.@\textsc{Schnitzler, Arthur} (15.\,5.\,1862 Wien – 21.\,10.\,1931 ebd.), \emph{Schriftsteller, Mediziner}!Sterben. Novelle@\strich\emph{Sterben. Novelle}|pwv} –{ }ſchön und reich. In der Literatur {\pb}weiſt es Dir, meiner Anſicht nach, einen Platz neben \textsc{d’Annunzio\pwindex{D’Annunzio, Gabriele 12.\,3.\,1863 Pescara – 1.\,3.\,1938 Cargnacco@\textsc{D’Annunzio, Gabriele} (12.\,3.\,1863 Pescara – 1.\,3.\,1938 Cargnacco), \emph{Schriftsteller}|pw}} an\substVorne{}\textsuperscript{.}\substDazwischen{};\substHinten{} nur iſt Deine Art etwas blaſſer, weniger raffinirt,{ }ſanfter, als die{ }ſeine.
               Laß’ Dich von Herzen beglückwünſchen.\pend
           
\pstart
           Ich habe{ }ſofort Schritte gethan, um Dir eine Beſprechung in der Pariſ\oindex{Paris@\textbf{Paris}, \emph{Hauptstadt}|pw}er Preſſe, und zwar in der großen, zu verſchaffen. Ich bin
               zum »\textsc{Journal des Débats\orgindex{Journal des débats@Journal des débats|pw}}« gegangen und habe Sturm geläutet über die Wien\oindex{Wien@\textbf{Wien}, \emph{Verwaltungsgebiet}|pw}er Literatur. \textsc{Pierre Lalo\pwindex{Lalo, Pierre 6.\,9.\,1866 Puteaux – 9.\,6.\,1943 Paris@\textsc{Lalo, Pierre} (6.\,9.\,1866 Puteaux – 9.\,6.\,1943 Paris), \emph{Kritiker}|pw}}, ein charmanter und feinſinniger College, hat mir \label{K_L02622-2v}\edtext{Beſprechungen}{\lemma{\textnormal{\emph{Besprechungen}}}\Cendnote{\textnormal{Pierre Lalo\pwindex{Lalo, Pierre 6.\,9.\,1866 Puteaux – 9.\,6.\,1943 Paris@\textsc{Lalo, Pierre} (6.\,9.\,1866 Puteaux – 9.\,6.\,1943 Paris), \emph{Kritiker}|pwk} schrieb selbst: P. L.: \emph{Au jour le jour. M. Arthur Schnitzler}\pwindex{Lalo, Pierre 6.\,9.\,1866 Puteaux – 9.\,6.\,1943 Paris@\textsc{Lalo, Pierre} (6.\,9.\,1866 Puteaux – 9.\,6.\,1943 Paris), \emph{Kritiker}!Au jour le jour. M. Arthur Schnitzler@\strich\emph{Au jour le jour. M. Arthur Schnitzler}|pwk}. In: \emph{Journal des débats}\pwindex{Journal des débats. Politiques et littéraires@\emph{Journal des débats. Politiques et littéraires}|pwk}, Jg. 107,
                        21. 3. 1895, S. 1.}}}\label{K_L02622-2} verſprochen.
               Ob ers halten {\pb}wird, weiß ich nicht. Jedenfalls{ }ſchicke ihm ein Buch\pwindex{Schnitzler, Arthur 15.\,5.\,1862 Wien – 21.\,10.\,1931 ebd.@\textsc{Schnitzler, Arthur} (15.\,5.\,1862 Wien – 21.\,10.\,1931 ebd.), \emph{Schriftsteller, Mediziner}!Sterben. Novelle@\strich\emph{Sterben. Novelle}|pwv} und{ }ſchreibe hinein: \textsc{À Monsieur Pierre Lalo\pwindex{Lalo, Pierre 6.\,9.\,1866 Puteaux – 9.\,6.\,1943 Paris@\textsc{Lalo, Pierre} (6.\,9.\,1866 Puteaux – 9.\,6.\,1943 Paris), \emph{Kritiker}|pw}}, \textsc{\label{K_L02622-3v}\edtext{hommage de l’auteur}{\lemma{\textnormal{\emph{hommage de l’auteur}}}\Cendnote{\textnormal{französisch: Widmung des
                     Verfassers}}}\label{K_L02622-3}}, mit Deiner Unterſchrift. Ebenſo{ }ſoll \textsc{Richard\pwindex{Beer-Hofmann, Richard 11.\,7.\,1866 Wien – 26.\,9.\,1945 New York City@\textsc{Beer-Hofmann, Richard} (11.\,7.\,1866 Wien – 26.\,9.\,1945 New York City), \emph{Schriftsteller}|pw}} ihm{ }ſein Buch\pwindex{Beer-Hofmann, Richard 11.\,7.\,1866 Wien – 26.\,9.\,1945 New York City@\textsc{Beer-Hofmann, Richard} (11.\,7.\,1866 Wien – 26.\,9.\,1945 New York City), \emph{Schriftsteller}!Novellen@\strich\emph{Novellen}|pwv}{ }ſchicken.
               Er wohnt \textsc{19. Boulevard de Courcelles, Paris\oindex{Boulevard de Courcelles@\textbf{Boulevard de Courcelles}, \emph{Straße}|pw}}. Unter keinen Umſtänden aber bitte ich \textsc{Bahr\pwindex{Bahr, Hermann 19.\,7.\,1863 Linz – 15.\,1.\,1934 München@\textsc{Bahr, Hermann} (19.\,7.\,1863 Linz – 15.\,1.\,1934 München), \emph{Schriftsteller, Kritiker}|pw}} die Adreſſe\oindex{Boulevard de Courcelles@\textbf{Boulevard de Courcelles}, \emph{Straße}|pwv} zu geben.
               Ich will nicht, daß er{ }ſich durch meine Vermittelung in der Pariſ\oindex{Paris@\textbf{Paris}, \emph{Hauptstadt}|pw}er Preſſe lancirt. Sei mir nicht böſe: »\label{K_L02622-4v}\edtext{Ich weiß es wohl, es iſt ein
                  Vorurtheil\pwindex{\textcolor{red}{\textsuperscript{XXXX indx1}}!Urfaust@\strich\emph{Urfaust}|pwv}}{\lemma{\textnormal{\emph{Ich … Vorurtheil}}}\Cendnote{\textnormal{Mephistopheles im \emph{Urfaust}\pwindex{\textcolor{red}{\textsuperscript{XXXX indx1}}!Urfaust@\strich\emph{Urfaust}|pwk}: »Ich weis es wohl, es iſt ein Vorurtheil /
                  Allein genung mir iſts einmal zuwieder«. }}}\label{K_L02622-4}{ }\textsc{etc.}«.\pend
           
\pstart
           {\pb}Bei der »Frankfurter
                  Zeitung\orgindex{Frankfurter Zeitung@Frankfurter Zeitung|pw}« habe ich geſtern Schritte gethan. Ich
               hoffe, diesmal wird Alles glatt gehen. Haſt Du die liebenswürdige \label{K_L02622-5v}\edtext{Erwähnung Deines Namens durch \textsc{Uhl\pwindex{Uhl, Friedrich 14.\,5.\,1825 Cieszyn – 20.\,1.\,1906 Mondsee@\textsc{Uhl, Friedrich} (14.\,5.\,1825 Cieszyn – 20.\,1.\,1906 Mondsee), \emph{Journalist}|pw}} in{ }ſeinem Briefe\pwindex{Wiener Brief@\emph{Wiener Brief}|pwv} über
               das Stück\pwindex{Lubliner, Hugo 22.\,4.\,1846 Breslau – 19.\,12.\,1911 Berlin@\textsc{Lubliner, Hugo} (22.\,4.\,1846 Breslau – 19.\,12.\,1911 Berlin), \emph{Schriftsteller}!neue Stück. Lustspiel in 4 Acten@\strich\emph{Das neue Stück. Lustspiel in 4 Acten}|pwv} von \textsc{Lubliner\pwindex{Lubliner, Hugo 22.\,4.\,1846 Breslau – 19.\,12.\,1911 Berlin@\textsc{Lubliner, Hugo} (22.\,4.\,1846 Breslau – 19.\,12.\,1911 Berlin), \emph{Schriftsteller}|pw}}}{\lemma{\textnormal{\emph{Erwähnung … Lubliner}}}\Cendnote{\textnormal{Am 17. 11. 1894 hatte die Uraufführung\eventindex{Volkstheater@\textbf{Volkstheater}!Uraufführung von Das neue Stück, 17.11.1894@Uraufführung von Das neue Stück, 17.11.1894|pwkv} von \emph{Das neue Stück}\pwindex{Lubliner, Hugo 22.\,4.\,1846 Breslau – 19.\,12.\,1911 Berlin@\textsc{Lubliner, Hugo} (22.\,4.\,1846 Breslau – 19.\,12.\,1911 Berlin), \emph{Schriftsteller}!neue Stück. Lustspiel in 4 Acten@\strich\emph{Das neue Stück. Lustspiel in 4 Acten}|pwk} von Hugo Lubliner\pwindex{Lubliner, Hugo 22.\,4.\,1846 Breslau – 19.\,12.\,1911 Berlin@\textsc{Lubliner, Hugo} (22.\,4.\,1846 Breslau – 19.\,12.\,1911 Berlin), \emph{Schriftsteller}|pwk} am \emph{Deutschen
                     Volkstheater}\orgindex{Volkstheater@Volkstheater|pwk} stattgefunden. Uhl\pwindex{Uhl, Friedrich 14.\,5.\,1825 Cieszyn – 20.\,1.\,1906 Mondsee@\textsc{Uhl, Friedrich} (14.\,5.\,1825 Cieszyn – 20.\,1.\,1906 Mondsee), \emph{Journalist}|pwk}
                  schrieb: »Am lautesten lachten die dienstfreien Schauspieler des Volkstheaters\orgindex{Volkstheater@Volkstheater|pw} im Zuschauerraum, besonders
                     die stets Aufschauen erregende Schwärmerin Frl. \so{Sandrock}\pwindex{Sandrock, Adele 19.\,8.\,1863 Rotterdam – 30.\,8.\,1937 Berlin@\textsc{Sandrock, Adele} (19.\,8.\,1863 Rotterdam – 30.\,8.\,1937 Berlin), \emph{Schauspielerin}|pw} in einer Loge des ersten Ranges und der geistreiche Lustspieldichter Dr.
                        \so{Schnitzler}, der über das Kapitel ›Dichter im Wiener
                        Volkstheater\orgindex{Volkstheater@Volkstheater|pw}‹ eine Leidensgeschichte erzählen könnte. Aber Frl. Sandrock\pwindex{Sandrock, Adele 19.\,8.\,1863 Rotterdam – 30.\,8.\,1937 Berlin@\textsc{Sandrock, Adele} (19.\,8.\,1863 Rotterdam – 30.\,8.\,1937 Berlin), \emph{Schauspielerin}|pw} konnte auch dieses Stück\pwindex{Lubliner, Hugo 22.\,4.\,1846 Breslau – 19.\,12.\,1911 Berlin@\textsc{Lubliner, Hugo} (22.\,4.\,1846 Breslau – 19.\,12.\,1911 Berlin), \emph{Schriftsteller}!neue Stück. Lustspiel in 4 Acten@\strich\emph{Das neue Stück. Lustspiel in 4 Acten}|pwv} nicht
                     retten.« [Friedrich Uhl\pwindex{Uhl, Friedrich 14.\,5.\,1825 Cieszyn – 20.\,1.\,1906 Mondsee@\textsc{Uhl, Friedrich} (14.\,5.\,1825 Cieszyn – 20.\,1.\,1906 Mondsee), \emph{Journalist}|pwk}]:
                        \emph{Wiener Brief}\pwindex{Wiener Brief@\emph{Wiener Brief}|pwk}. In: \emph{Frankfurter Zeitung}\pwindex{Frankfurter Zeitung@\emph{Frankfurter Zeitung}|pwk}, Jg. 39, Nr. 322,
                        20. 11. 1894, Abendblatt, S. 1. }}}\label{K_L02622-5} geleſen?\pend
           
\pstart
           Ich wünſchte nur, daß ich Dir auch in den Schritten für Dein Stück\pwindex{Schnitzler, Arthur 15.\,5.\,1862 Wien – 21.\,10.\,1931 ebd.@\textsc{Schnitzler, Arthur} (15.\,5.\,1862 Wien – 21.\,10.\,1931 ebd.), \emph{Schriftsteller, Mediziner}!Liebelei. Schauspiel in drei Akten@\strich\emph{Liebelei. Schauspiel in drei Akten}|pwv} behilflich{ }ſein könnte\substVorne{}\textsuperscript{.}\substDazwischen{},\substHinten{} um Dir ein wenig von dem Paſſionswege zu erſparen. Ich habe mir den Kopf
               zerbrochen, wie ich eingreifen könnte, finde aber nichts. Oder glaubſt Du vielleicht,
               daß {\pb}\textsc{Uhl\pwindex{Uhl, Friedrich 14.\,5.\,1825 Cieszyn – 20.\,1.\,1906 Mondsee@\textsc{Uhl, Friedrich} (14.\,5.\,1825 Cieszyn – 20.\,1.\,1906 Mondsee), \emph{Journalist}|pw}} etwas in der Sache thun könnte? Dann{ }ſchreib’ mir darüber, und ich wills
               unternehmen. Jedenfalls wiederhole ich Dir von Neuem: laß’ Dich nicht niederdrücken
               und entmuthigen. Die Schwierigkeiten waren vorauszuſehen. Wenn man ein Stück nur zu{ }ſchreiben und einzureichen brauchte, um es aufgeführt zu{ }ſehen,{ }ſo wäre es ein
               Vergnügen, Theaterdichter zu{ }ſein. Außerdem bringſt Du Neues, das heißt etwas
               Anti-Dummes, folglich haſt Du die Dummheit gegen Dich. Das iſt doch ganz natürlich.
               Aber man findet{ }ſchon Mittel, {\pb}um mit der Dummheit
               fertig zu werden. Nur Zeit, Geduld und Geſchick gehört dazu. Mit dieſen drei
               Kampfmitteln \strikeout{\textcolor{gray}{we}} mußt Du Dich unter allen Umſtänden ausrüſten. Ich bin \uline{überzeugt}, Du wirſt am Ende durchdringen, und zwar gerade beim Burgtheater\orgindex{Burgtheater@Burgtheater|pw}. Laß’ Dich alſo nicht verſtimmen.
               Denk’ auch an den{ }ſchönen Haß und Hohn, den dieſe Erfahrungen in Dir aufhäufen und
               der befruchtend wirken wird für \strikeout{ſch}{ }ſpätere Werke.
               Und, bitte, mach’ mir nach wie vor von jedem weiteren Vorkomniß Mittheilung. \label{K_L02622-6v}\edtext{\textsc{Speidel\pwindex{Speidel, Ludwig 11.\,4.\,1830 Ulm – 3.\,2.\,1906 Wien@\textsc{Speidel, Ludwig} (11.\,4.\,1830 Ulm – 3.\,2.\,1906 Wien), \emph{Journalist, Kritiker}|pw}}}{\lemma{\textnormal{\emph{Speidel}}}\Cendnote{\textnormal{Speidel\pwindex{Speidel, Ludwig 11.\,4.\,1830 Ulm – 3.\,2.\,1906 Wien@\textsc{Speidel, Ludwig} (11.\,4.\,1830 Ulm – 3.\,2.\,1906 Wien), \emph{Journalist, Kritiker}|pwk} war ein enger Berater des \emph{Burgtheater}\orgindex{Burgtheater@Burgtheater|pwk}-Direktors Burckhard\pwindex{Burckhard, Max Eugen 14.\,7.\,1854 Korneuburg – 16.\,3.\,1912 Wien@\textsc{Burckhard, Max Eugen} (14.\,7.\,1854 Korneuburg – 16.\,3.\,1912 Wien), \emph{Schriftsteller, Rechtswissenschaftler, Theaterleiter}|pwk}. Vgl. A. S.: \emph{Tagebuch}, 20. 11. 1894, 14. 12. 1894. }}}\label{K_L02622-6}? {\pb}Vielleicht. \label{K_L02622-7v}\edtext{Wenn Gott will,{ }ſchießt ein Beſen}{\lemma{\textnormal{\emph{Wenn … Besen}}}\Cendnote{\textnormal{jüdisches Sprichwort}}}\label{K_L02622-7}. Und die Erfahrung lehrt, daß
               hier und da ein Beſen{ }ſchon geſchoſſen hat. Man \strikeout{ve}
               verleumdet den lieben Gott, wenn man{ }ſo ganz{ }ſeine Exiſtenz leugnet. Ein wenig
               exiſtirt er doch, auch für junge Poeten.\pend
           
\pstart
           Dringend bitte ich Dich, mich bei Frl. \textsc{Sandrock\pwindex{Sandrock, Adele 19.\,8.\,1863 Rotterdam – 30.\,8.\,1937 Berlin@\textsc{Sandrock, Adele} (19.\,8.\,1863 Rotterdam – 30.\,8.\,1937 Berlin), \emph{Schauspielerin}|pw}} zu entſchuldigen. Ich{ }ſchreibe ihr,{ }ſobald ich einen{ }ſreien Augenblick
               habe.\pend
           
\pstart
           Herr \textsc{Sokal\pwindex{Sokal, Clemens *~21.\,1.\,1867 Lviv@\textsc{Sokal, Clemens} (*~21.\,1.\,1867 Lviv), \emph{Journalist, Rechtsanwalt}|pw}}{ }ſoll \label{K_L02622-8v}\edtext{gut aufgenommen}{\lemma{\textnormal{\emph{gut aufgenommen}}}\Cendnote{\textnormal{Bezug unklar}}}\label{K_L02622-8} werden, {\pb}um deſſentwillen, von dem er kommt, und, wenn er
               will, auch{ }ſeinetwegen.\pend
           
\pstart
           Wie geht die »Zeit\orgindex{Zeit. Wiener Wochenschrift@Die Zeit. Wiener Wochenschrift|pw}«? Und was{ }ſagſt Du dazu?\pend
           
\pstart
           Unter Discretion: Ich höre, daß \textsc{Benedict\pwindex{Benedikt, Moriz 27.\,5.\,1849 Kvačice – 18.\,3.\,1920 Wien@\textsc{Benedikt, Moriz} (27.\,5.\,1849 Kvačice – 18.\,3.\,1920 Wien), \emph{Journalist, Herausgeber}|pw}} Erkundigungen über mich einzieht. Natürlich werde ich nie an \label{K_L02622-9v}\edtext{\textsc{Herzls\pwindex{Herzl, Theodor 2.\,5.\,1860 Budapest – 3.\,7.\,1904 Edlach@\textsc{Herzl, Theodor} (2.\,5.\,1860 Budapest – 3.\,7.\,1904 Edlach), \emph{Schriftsteller, Journalist}|pw}} Stelle}{\lemma{\textnormal{\emph{Herzls Stelle}}}\Cendnote{\textnormal{Siehe XXXX Auszeichnungsfehler: Dokument L02619 nicht gefunden.
               }}}\label{K_L02622-9} kommen,{ }ſchon weil \textsc{Herzl\pwindex{Herzl, Theodor 2.\,5.\,1860 Budapest – 3.\,7.\,1904 Edlach@\textsc{Herzl, Theodor} (2.\,5.\,1860 Budapest – 3.\,7.\,1904 Edlach), \emph{Schriftsteller, Journalist}|pw}} dagegen iſt, und aus andern Gründen. Aber kennſt Du zufällig Jemanden, der dem
               hochmögenden Herrn\pwindex{Benedikt, Moriz 27.\,5.\,1849 Kvačice – 18.\,3.\,1920 Wien@\textsc{Benedikt, Moriz} (27.\,5.\,1849 Kvačice – 18.\,3.\,1920 Wien), \emph{Journalist, Herausgeber}|pw}, natürlich mit unendlicher
               Vorſicht, in einem Geſpräche gelegentlich mittheilen könnte, {\pb}daß ich ein großer Mann bin? Um nicht Alles
               unverſucht zu laſſen!\pend
           
\pstart
           Die gütigen Worte, die Du über mich{ }ſchreibſt, haben mich tief bewegt. Was ich an \uline{Dir} habe, weiß ich längſt; aber es thut wohl, es
               wieder einmal zu fühlen. Wie{ }ſich mein Bild bei Andern malt,{ }ſehe ich täglich und{ }ſtündlich, und dieſe Erfahrungen{ }ſprechen{ }ſchreienden, brüllenden Hohn zu Deinen
               lieben Zeilen. Wenn ich \strikeout{dann} Dein Buch\pwindex{Schnitzler, Arthur 15.\,5.\,1862 Wien – 21.\,10.\,1931 ebd.@\textsc{Schnitzler, Arthur} (15.\,5.\,1862 Wien – 21.\,10.\,1931 ebd.), \emph{Schriftsteller, Mediziner}!Sterben. Novelle@\strich\emph{Sterben. Novelle}|pwv} leſe und dann an meine Thätigkeit denke –
                  {\pb}es iſt beinahe komiſch. Nein, ehrlich geſagt,
               das iſt es nicht: es iſt traurig{\dotsfour}\pend
           
\pstart
           Du erhälſt anbei ein paar \label{K_L02622-10v}\edtext{kurioſe
                  Artikel}{\lemma{\textnormal{\emph{kuriose
                  Artikel}}}\Cendnote{\textnormal{Beilage nicht erhalten}}}\label{K_L02622-10}
               aller Art.\pend
           
\pstart
           Was{ }ſoll ich mit den \textsc{30 Francs 30 ct.} machen, die ich Dir{ }ſchulde? Du{ }ſetzeſt mich einer{ }ſtarken Verſuchung aus. Ein Anderer hätte{ }ſie längſt
               unterſchlagen. Ich{ }ſehe mit Befriedigung, wie \strikeout{\textcolor{gray}{e}hrlich} ehrlich ich bin.\pend
           
\pstart
           Grüße, bitte, Mutter\pwindex{Schnitzler, Louise 8.\,7.\,1840 Kőszeg – 9.\,9.\,1911 Wien@\textsc{Schnitzler, Louise} (8.\,7.\,1840 Kőszeg – 9.\,9.\,1911 Wien)|pwv}, Bruder\pwindex{Schnitzler, Julius 13.\,7.\,1865 Wien – 29.\,6.\,1939 ebd.@\textsc{Schnitzler, Julius} (13.\,7.\,1865 Wien – 29.\,6.\,1939 ebd.), \emph{Chirurg}|pwv} und Schwägerin\pwindex{Schnitzler, Helene 16.\,7.\,1871 Budapest – September 1941 Atlantischer Ozean@\textsc{Schnitzler, Helene} (16.\,7.\,1871 Budapest – September 1941 Atlantischer Ozean)|pwv}.\pend
           
\pstart
           In alter Treue{\\[\baselineskip]}Dein{\\[\baselineskip]}\spacefill\mbox{Paul Goldmann.}\pend
           \leftskip=0em{}\selectlanguage{ngerman}\endnumbering\briefempfaengerindex{Schnitzler, Arthur@\textsc{Schnitzler, Arthur}!zzzGoldmann, Paul@\emph{von Paul Goldmann}!1894-11-282@{28. 11. 1894}|)be}\mylabel{L02622h}  \newcommand{\dateiname}{L02622}\newcommand{\titel}{Paul Goldmann an Arthur Schnitzler, 28. 11. 1894}\newcommand{\editorInnen}{Martin Anton Müller und Laura Untner}%% latex-leseansicht-abspann.tex
%% Abspann für die Leseansicht.
%% Der Schalter \ifkorrekturansicht ist bereits durch den Vorspann gesetzt.

%% latex-abspann.tex
%% Gemeinsamer Abspann für Korrekturansicht und Leseansicht.
%% Setzt den Schalter \ifkorrekturansicht voraus (gesetzt in den
%% einbindenden Dateien latex-korrekturansicht-abspann.tex bzw.
%% latex-leseansicht-abspann.tex).
%% ---------------------------------------------------------------

\normalsize

% Das esempio-Environment wird nur in der Leseansicht benötigt
\ifkorrekturansicht\else
\newenvironment{esempio}[3]%
{
    \vspace{1.5ex}
    \rlap{\underline{#1}}
    \par
    \setlength{\parindent}{0cm}
    \nopagebreak
    \leftskip=#2cm
    \rightskip=#3cm
}
{
    \par
}
\fi

\doendnotes{C}
\bigskip
\vfill

\clearpage

\footnotesize

\ifkorrekturansicht
  \lohead{\textsc{register}}
\fi

% theindex-Environment neu definieren ohne reledmac
\makeatletter
\renewenvironment{theindex}{%
  \ifkorrekturansicht
    \section*{\indexname}%
  \else
    \subsubsection*{Index der erwähnten Entitäten}%
  \fi
  \setlength{\parindent}{0pt}%
  \setlength{\parskip}{0pt plus 0.3pt}%
  \let\item\@idxitem
}{%
  \ifkorrekturansicht\clearpage\fi
}
\makeatother

\IfFileExists{\jobname-pw.ind}{\input{\jobname-pw.ind}}{}

% Quellenangabe nur in der Leseansicht
\ifkorrekturansicht\else
% Fallback-Definitionen, falls die .tex-Datei \titel etc. nicht gesetzt hat
\providecommand{\titel}{}
\providecommand{\editorInnen}{}
\providecommand{\dateiname}{\jobname}

\vspace{3cm}

\vfill

\footnotesize
\textsc{Quelle}: \titel. Herausgegeben von {\editorInnen}. In: \emph{Arthur Schnitzler: Briefwechsel mit Autorinnen und Autoren}.
 Digitale Edition, https://schnitzler-briefe.acdh.oeaw.ac.at/{\dateiname}.html (Stand \today)
\fi

\end{document}


