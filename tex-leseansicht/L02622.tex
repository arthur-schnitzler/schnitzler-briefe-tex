%% latex-leseansicht-vorspann.tex
%% Vorspann für die Leseansicht.
%% Lädt die gemeinsame Datei latex-vorspann.tex mit nicht gesetztem Schalter.

\newif\ifkorrekturansicht
\korrekturansichtfalse

\input{../tex-inputs/latex-vorspann}

\begin{center}
            \textcolor{red}{ENTWURF, NICHT FERTIG KORRIGIERT}
                      \end{center}
            
               \section[Paul Goldmann an Arthur Schnitzler, 28. 11. 1894]{ Paul Goldmann an Arthur Schnitzler, 28. 11. 1894}\nopagebreak\mylabel{v}\rehead{ }\begin{ledgroupsized}[t]{13cm}\normalsize\beginnumbering\briefempfaengerindex{Schnitzler, Arthur@\textsc{Schnitzler, Arthur}!zzzGoldmann, Paul@\emph{von Paul Goldmann}!1894-11-281@{28. 11. 1894}|(be} \toendnotes[C]{\smallbreak\pagebreak[2]} \Standort{DLA, A:Schnitzler, HS.NZ85.1.3164.}
\physDesc{Brief, 3 Blätter, 12 Seiten
\newline{}Handschrift: schwarze Tinte, deutsche Kurrent
\newline{}Schnitzler: 1) mit Bleistift auf dem ersten Blatt die Jahreszahl
                                       »94« vermerkt 2) mit rotem Buntstift acht Unterstreichungen}\toendnotes[C]{\smallbreak}\pstart
           \noindent{}{\pb}\textcolor{gray}{\textbf{Frankfurter Zeitung\orgindex{Frankfurter Zeitung@Frankfurter Zeitung|pw}.}}\hfill \textsc{Paris\oindex{Paris@\textbf{Paris}|pw}}, 28. November.\pend
           \pstart
           \textcolor{gray}{\textbf{(Gazette de
                  Francfort\orgindex{Frankfurter Zeitung@Frankfurter Zeitung|pw}.)}}\pend
           \pstart
           \textcolor{gray}{\textbf{\begin{otherlanguage}{french}Fondateur\end{otherlanguage}{ }\textbf{M. L.
                  Sonnemann\pwindex{Sonnemann, Leopold 1831-10-29 – 1909-10-30@\textsc{Sonnemann, Leopold} (1831-10-29 – 1909-10-30), \emph{Journalist, Herausgeber}|pw}}.}}\pend
           \pstart
           \textcolor{gray}{\textbf{\begin{otherlanguage}{french}Journal politique,
                        financier,\end{otherlanguage}}}\pend
           \pstart
           \textcolor{gray}{\textbf{\begin{otherlanguage}{french}commercial et
                     littéraire.\end{otherlanguage}}}\pend
           \pstart
           \textcolor{gray}{\textbf{\begin{otherlanguage}{french}\textbf{Paraissant trois fois
                           par jour}\end{otherlanguage}}}.\pend
           \pstart
           \textcolor{gray}{\textbf{–}}\pend
           \pstart
           \textcolor{gray}{\textbf{\begin{otherlanguage}{french}\textbf{Bureaux à Paris\oindex{Paris@\textbf{Paris}|pw}:}\end{otherlanguage}}}\pend
           \pstart
           \textcolor{gray}{\textbf{\begin{otherlanguage}{french}\textbf{24. Rue Feydeau}\oindex{rue Feydeau@\textbf{rue Feydeau}|pw}.\end{otherlanguage}}}\pend
           \pstart\center{}Mein lieber Freund,\pend\pstart
           Ich danke Dir von Herzen für die Überſendung von »Sterben\pwindex{Schnitzler, Arthur 15.05.1862 – 21.10.1931@\textsc{Schnitzler, Arthur} (15.05.1862 – 21.10.1931), \emph{Schriftsteller, Mediziner}!Sterben. Novelle1.10.1894 – 1.12.1894@\strich\emph{Sterben. Novelle} {[}1.10.1894 – 1.12.1894{]}|pw}«. Als ich den Schluß las, hatte ich das Gefühl, daß ſich der durch
               die verfluchten Fortſetzungen unterbrochene Strom wieder herſtellte. Der große
               Schauer kam – Ergriffenheit und Entzücken. Das Sterben iſt meiſterhaft geſchildert.
               Mich ſtört nur das Erwürgen, – dieſes plötzliche Verfallen in die kriminaliſtiſche
               Brutalität, nachdem \strikeout{es} vorher \strikeout{Alles} Alles eitel Freiheit, Seele, Stimmung geweſen.
               Ich glaube, das {\pb}hätte zweiſelhaft bleiben müſſen.
               Vielleicht ſtellte ſich das die überhitzte Phantaſie des Mädchens \introOben{}nur\introOben{} ſo vor? Vielleicht wollte er ſie umarmen? Mir ſtört das noch rückwärts
               etwas das Bild des Unglücklichen. Er ſoll Einer ſein, der \uline{leidet}, bis zum Schluß. Das Handeln iſt ſo unheimlich, ſo gegen ſeine Natur.
               Der erwürgt nicht, glaub’ mirs. Er weint nur, weil ſie nicht mit ihm ſterben will,
               das Sterben ſelbſt wird ihm dadurch zur noch größeren Qual, er wird noch mehr \uline{leidend} zum Schluß. So denke ichs mir. Und {\pb}das Alles könnte erreicht werden, wenn nur ein
               einziger kleiner Satz am Schluſſe geſtrichen würde, wo das Mädel es klar ſagt: »Er
               hatte ſie erwürgen wollen.«\pend
           \pstart
           Vielleicht habe ich übrigens Unrecht. Denn ich habe das Buch mit überſcharfer Kritik
               geleſen, weil ich \strikeout{\textcolor{gray}{mir}} Dir ſelbſt
               gegenüber ein unparteiiſches zu fällen mich verpflichtet fühlte und ſtets auf der
               Lauer war, um nicht von meiner Freundſchaft überrumpelt zu werden. Sonſt iſt es wohl
               gelungen, das Buch – ſchön und reich. In der Literatur {\pb}weiſt es Dir, meiner Anſicht nach, einen Platz neben
                  \textsc{d’Annunzio\pwindex{DAnnunzio, Gabriele 12.03.1863 – 01.03.1938@\textsc{D’Annunzio, Gabriele} (12.03.1863 – 01.03.1938), \emph{Schriftsteller}|pw}} an; nur
               iſt Deine Art etwas blaſſer, weniger raffinirt, ſanſter, als die ſeine. Laß’ Dich von
               Herzen beglückwünſichen.\pend
           \pstart
           Ich habe ſofort Schritte gethan, um Dir eine Beſprechung in der Pariſ\oindex{Paris@\textbf{Paris}|pw}er Preſſe, und zwar in der großen, zu verſchaffen. Ich bin
               zum »\textsc{Journal des Débats\orgindex{Journal des debats@Journal des débats|pw}}«
               gegangen und habe Sturm geläutet über die Wien\oindex{Wien@\textbf{Wien}|pw}er
               Literatur. \textsc{Pierre Lalo\pwindex{Lalo, Pierre 06.09.1866 – 09.06.1943@\textsc{Lalo, Pierre} (06.09.1866 – 09.06.1943), \emph{Kritiker}|pw}},
               ein charmanter und feinſinnger College, hat mir \label{K_L02622-1v}\edtext{Beſprechungen}{\lemma{\textnormal{\emph{Beſprechungen}}}\Cendnote{\textnormal{XXXX}}}\label{K_L02622-1h} verſprochen. Ob ers halten {\pb}wird,
               weiß ich nicht. Jedenfalls ſchicke ihm ein Buch\textcolor{red}{\textsuperscript{\textbf{KEY}}} und
               ſchreibe hinein: \textsc{À Monsieur Pierre
                     Lalo\pwindex{Lalo, Pierre 06.09.1866 – 09.06.1943@\textsc{Lalo, Pierre} (06.09.1866 – 09.06.1943), \emph{Kritiker}|pw}}, \textsc{\label{K_mets_Goldmann_94-partII-676v}\edtext{hommage de l’auteur}{\lemma{\textnormal{\emph{hommage de l’auteur}}}\Cendnote{\textnormal{französisch: Widmung
                     des Verfassers}}}\label{K_mets_Goldmann_94-partII-676h}}, mit Deiner Unterſchrift. Ebenſo ſoll \textsc{Richard\pwindex{Beer-Hofmann, Richard 11.07.1866 – 26.09.1945@\textsc{Beer-Hofmann, Richard} (11.07.1866 – 26.09.1945), \emph{Schriftsteller}|pw}} ihm ſein Buch\pwindex{Beer-Hofmann, Richard 11.07.1866 – 26.09.1945@\textsc{Beer-Hofmann, Richard} (11.07.1866 – 26.09.1945), \emph{Schriftsteller}!Novellen1. 12. 1893@\strich\emph{Novellen} {[}1. 12. 1893{]}|pwv} ſchicken. Er wohnt \textsc{19. Boulevard de Courcelles,
                     Paris\oindex{Boulevard de Courcelles@\textbf{Boulevard de Courcelles}|pw}}. Unter keinen Umſtänden aber bitte ich \textsc{Bahr\pwindex{Bahr, Hermann 19.07.1863 – 15.01.1934@\textsc{Bahr, Hermann} (19.07.1863 – 15.01.1934), \emph{Schriftsteller, Kritiker}|pw}} die Adreſſe\oindex{Boulevard de Courcelles@\textbf{Boulevard de Courcelles}|pwv} zu geben. Ich will nicht, daß er ſich durch meine
               Vermittelung in der Pariſ\oindex{Paris@\textbf{Paris}|pw}er Preſſe lancirt. Sei mir
               nicht böſe: »Ich weiß es wohl, es iſt ein Vorurtheil \textsc{etc.}«.\pend
           \pstart
           {\pb}Bei der »Frankfurter
                  Zeitung\orgindex{Frankfurter Zeitung@Frankfurter Zeitung|pw}« habe ich geſtern Schritte gethan. Ich hoffe, diesmal wird Alles
               glatt gehen. Haſt Du die liebenswürdige Erwähnung Deines Namens durch \textsc{Uhl\pwindex{Uhl, Friedrich 14.05.1825 – 20.01.1906@\textsc{Uhl, Friedrich} (14.05.1825 – 20.01.1906), \emph{Journalist}|pw}} in ſeinem Briefe\pwindex{Uhl, Friedrich 14.05.1825 – 20.01.1906@\textsc{Uhl, Friedrich} (14.05.1825 – 20.01.1906), \emph{Journalist}!?? [Wiener Brief]vor 1894-11-28@\strich\emph{?? [Wiener Brief]} {[}vor 1894-11-28{]}|pwv} über das \label{K_L02622-55v}\edtext{Stück\pwindex{Lubliner, Hugo 22.04.1846 – 19.12.1911@\textsc{Lubliner, Hugo} (22.04.1846 – 19.12.1911), \emph{Schriftsteller}!neue Stueck. Lustspiel in 4 Acten1894-11-17@\strich\emph{Das neue Stück. Lustspiel in 4 Acten} {[}1894-11-17{]}|pwv}}{\lemma{\textnormal{\emph{Stück}}}\Cendnote{\textnormal{Am 17. 11. 1894 fand die
                  Uraufführung von \emph{Das neue Stück}\pwindex{Lubliner, Hugo 22.04.1846 – 19.12.1911@\textsc{Lubliner, Hugo} (22.04.1846 – 19.12.1911), \emph{Schriftsteller}!neue Stueck. Lustspiel in 4 Acten1894-11-17@\strich\emph{Das neue Stück. Lustspiel in 4 Acten} {[}1894-11-17{]}|pwk} von Hugo Lubliner\pwindex{Lubliner, Hugo 22.04.1846 – 19.12.1911@\textsc{Lubliner, Hugo} (22.04.1846 – 19.12.1911), \emph{Schriftsteller}|pwk} am \emph{Deutschen Volkstheater}\orgindex{Volkstheater@Volkstheater|pwk} statt. Schnitzler\pwindex{Schnitzler, Arthur 15.05.1862 – 21.10.1931@\textsc{Schnitzler, Arthur} (15.05.1862 – 21.10.1931), \emph{Schriftsteller, Mediziner}|pwk} nahm teil.}}}\label{K_L02622-55h} von \textsc{Lubliner\pwindex{Lubliner, Hugo 22.04.1846 – 19.12.1911@\textsc{Lubliner, Hugo} (22.04.1846 – 19.12.1911), \emph{Schriftsteller}|pw}} geleſen?\pend
           \pstart
           Ich wünſchte nur, daß ich Dir auch in den Schritten für Dein Stück\pwindex{Schnitzler, Arthur 15.05.1862 – 21.10.1931@\textsc{Schnitzler, Arthur} (15.05.1862 – 21.10.1931), \emph{Schriftsteller, Mediziner}!Liebelei. Schauspiel in drei Akten9. 10. 1895@\strich\emph{Liebelei. Schauspiel in drei Akten} {[}9. 10. 1895{]}|pwv} behilflich ſein könnte, um Dir ein wenig
               von dem Paſſionswege zu erſparen. Ich habe mir den Kopf zerbrochen, wie ich
               eingreifen könnte, finde aber nichts. Aber glaubſt Du vielleicht, daß {\pb}\textsc{Uhl\pwindex{Uhl, Friedrich 14.05.1825 – 20.01.1906@\textsc{Uhl, Friedrich} (14.05.1825 – 20.01.1906), \emph{Journalist}|pw}} etwas in der Sache thun könnte? Dann ſchreib’ mir darüber und
               ich wills unternehmen. Jedenfalls, wiederhole ich Dir von Neuem: laß’ Dich nicht
               niederdrücken und entmuthigen. Die Schwierigkeiten waren vorauszuſehen. Wenn man ein
               Stück nur zu ſchreiben und einzureichen brauchte, um es aufgeführt zu ſehen, ſo wäre
               es ein Vergnügen, Theaterdichter zu ſein. Außerdem bringſt Du Neues, das heißt, etwas
               Anti-Dummes, folglich haſt Du die Dummheit gegen Dich. Das iſt doch ganz natürlich.
               Aber man findet ſchon Mittel, {\pb}um mit der Dummheit
               fertig zu werden. Nur Zeit, Geduld und Geſchick gehört dazu. Mit dieſen drei
               Kampfmitteln \strikeout{\textcolor{gray}{we}} mußt Du Dich unter
               allen Umſtänden ausrüſten. Ich bin \uline{überzeugt}, Du
               wirſt am Ende durchdringen, und zwar gerade bein Burgtheater\orgindex{Burgtheater@Burgtheater|pw}. Laß’ Dich alſo nicht verſtimmen. Denk’ auch an den ſchönen Haß
               und Hohn, den dieſe Erfahrungen in Dir aufhäufen und der befruchtend wirken wird für
                  \strikeout{ſch} ſpätere Werke. Und, bitte, mach’ mir nach wie
               vor von jedem weiteren Vorkomniß Mittheilung. \label{K_L02622-2v}\edtext{\textsc{Speidel\pwindex{Speidel, Ludwig 11.04.1830 – 03.02.1906@\textsc{Speidel, Ludwig} (11.04.1830 – 03.02.1906), \emph{Journalist, Kritiker}|pw}}}{\lemma{\textnormal{\emph{Speidel}}}\Cendnote{\textnormal{XXXX}}}\label{K_L02622-2h}? {\pb}Vielleicht. Wenn Gott will, ſchießt ein Beſen. Und
               die Erfahrung lehrt, daß hier und da ein Beſen ſchon geſchoſſen hat. Man \strikeout{ve} verleumdet den lieben Gott, wenn man ſo ganz ſeine
               Exiſtenz leugnet. Ein wenig exiſtirt er doch, auch für junge Poeten.\pend
           \pstart
           Dringend bitte ich dich, mich bei Frl. \textsc{Sandrock\pwindex{Sandrock, Adele 19.08.1863 – 30.08.1937@\textsc{Sandrock, Adele} (19.08.1863 – 30.08.1937), \emph{Schauspielerin}|pw}} zu entſchuldigen. Ich ſchreibe ihr, ſobald ich
               einen ſreien Augenblick habe.\pend
           \pstart
           Herr \textsc{Sokal\pwindex{Sokal, Clemens *~21.01.1867@\textsc{Sokal, Clemens} (*~21.01.1867), \emph{Journalist, Rechtsanwalt}|pw}} ſoll \label{K_L02622-3v}\edtext{gut aufgenommen}{\lemma{\textnormal{\emph{gut aufgenommen}}}\Cendnote{\textnormal{XXXX}}}\label{K_L02622-3h} werden, {\pb}um
               deſſentwillen, von dem er kommt, und, wenn er will, auch ſeinetwegen.\pend
           \pstart
           Wie geht die »Zeit\orgindex{Zeit. Wiener Wochenschrift@Die Zeit. Wiener Wochenschrift|pw}«? Und was ſagſt Du dazu?\pend
           \pstart
           Unter Discretion: Ich höre, daß \textsc{Benedict\pwindex{Benedikt, Moriz 27.05.1849 – 18.03.1920@\textsc{Benedikt, Moriz} (27.05.1849 – 18.03.1920), \emph{Journalist}|pw}} Erkundigungen über mich einzieht. Natürlich werde ich nie
               an \label{K_L02622-4v}\edtext{\textsc{Herzl\pwindex{Herzl, Theodor 02.05.1860 – 03.07.1904@\textsc{Herzl, Theodor} (02.05.1860 – 03.07.1904), \emph{Schriftsteller, Journalist}|pw}s} Stelle}{\lemma{\textnormal{\emph{Herzls Stelle}}}\Cendnote{\textnormal{siehe Paul Goldmann an Arthur Schnitzler, 1. 5. [1894]}}}\label{K_L02622-4h}
               kommen, ſchon weil \textsc{Herzl\pwindex{Herzl, Theodor 02.05.1860 – 03.07.1904@\textsc{Herzl, Theodor} (02.05.1860 – 03.07.1904), \emph{Schriftsteller, Journalist}|pw}} dagegen iſt, und aus andern Gründen. Aber kennſt Du zufällig
               Jemanden, der dem hochmögenden Herrn\pwindex{Benedikt, Moriz 27.05.1849 – 18.03.1920@\textsc{Benedikt, Moriz} (27.05.1849 – 18.03.1920), \emph{Journalist}|pw}, natürlich
               mit unendlicher Vorſicht, in einem Geſpräche gelegentlich mittheilen könnte, {\pb}daß ich ein großer Mann bin? Um nicht Alles
               unverſucht zu laſſen!\pend
           \pstart
           Die gütigen Worte, die Du über mich ſchreibſt, haben mich tief bewegt. Was ich an \uline{Dir} habe, weiß ich längſt; aber es thut wohl, es
               wieder einmal zu fühlen. Wie ſich mein Bild bei Andern malt, ſehe ich täglich und
               ſtündlich, und dieſe Erfahrungen ſprechen ſchreienden, brüllenden Hohn zu Deinen
               lieben Zeilen. Wenn ich \strikeout{dann} Dein Buch\pwindex{Schnitzler, Arthur 15.05.1862 – 21.10.1931@\textsc{Schnitzler, Arthur} (15.05.1862 – 21.10.1931), \emph{Schriftsteller, Mediziner}!Sterben. Novelle1.10.1894 – 1.12.1894@\strich\emph{Sterben. Novelle} {[}1.10.1894 – 1.12.1894{]}|pwv} leſe und dann an meine Thätigkeit denke –
                  {\pb}es iſt beinahe komiſch. Nein, ehrlich geſagt,
               das iſt es nicht: es iſt traurig{\dotsfour}\pend
           \pstart
           Du erhälſt anbei ein Paar \label{K_L02622-11v}\edtext{kurioſe
                  Artikel}{\lemma{\textnormal{\emph{kurioſe
                  Artikel}}}\Cendnote{\textnormal{Beilage nicht erhalten}}}\label{K_L02622-11h}
               aller Art.\pend
           \pstart
           Was ſoll ich mit den \textsc{30 Francs 30 ct.} machen, die
               ich Dir ſchulde? Du ſetzeſt mich einer ſtarken Verſuchung aus. Ein Anderer hätte ſie
               längſt unterſchlagen. Ich ſehe mit Befriedigung, wie \strikeout{\textcolor{gray}{e}hrlich} ehrlich ich bin.\pend
           \pstart
           Grüße, bitte, Mutter\pwindex{Schnitzler, Louise 08.07.1840 – 09.09.1911@\textsc{Schnitzler, Louise} (08.07.1840 – 09.09.1911)|pwv}, Bruder\pwindex{Schnitzler, Julius 13.07.1865 – 29.06.1939@\textsc{Schnitzler, Julius} (13.07.1865 – 29.06.1939), \emph{Chirurg}|pwv} und Schwägerin\pwindex{Schnitzler, Helene 16.07.1871 – September 1941@\textsc{Schnitzler, Helene} (16.07.1871 – September 1941)|pwv}.\pend
           \pstart
           In alter Treue{\\[\baselineskip]}Dein{\\[\baselineskip]}\spacefill\mbox{Paul Goldmann.}\pend
           \leftskip=0em{}\endnumbering\briefempfaengerindex{Schnitzler, Arthur@\textsc{Schnitzler, Arthur}!zzzGoldmann, Paul@\emph{von Paul Goldmann}!1894-11-281@{28. 11. 1894}|)be}\mylabel{h}\end{ledgroupsized}\begin{anhang}\end{anhang}\newcommand{\dateiname}{L02622}\newcommand{\titel}{Paul Goldmann an Arthur Schnitzler, 28. 11. 1894}\newcommand{\editorInnen}{Martin Anton Müller und Laura Untner}
            \footnotesize
\begin{ledgroupsized}[t]{11.5cm}
\doendnotes{C}
\end{ledgroupsized}
         %% latex-leseansicht-abspann.tex
%% Abspann für die Leseansicht.
%% Der Schalter \ifkorrekturansicht ist bereits durch den Vorspann gesetzt.

%% latex-abspann.tex
%% Gemeinsamer Abspann für Korrekturansicht und Leseansicht.
%% Setzt den Schalter \ifkorrekturansicht voraus (gesetzt in den
%% einbindenden Dateien latex-korrekturansicht-abspann.tex bzw.
%% latex-leseansicht-abspann.tex).
%% ---------------------------------------------------------------

\normalsize

% Das esempio-Environment wird nur in der Leseansicht benötigt
\ifkorrekturansicht\else
\newenvironment{esempio}[3]%
{
    \vspace{1.5ex}
    \rlap{\underline{#1}}
    \par
    \setlength{\parindent}{0cm}
    \nopagebreak
    \leftskip=#2cm
    \rightskip=#3cm
}
{
    \par
}
\fi

\doendnotes{C}
\bigskip
\vfill

\clearpage

\footnotesize

\ifkorrekturansicht
  \lohead{\textsc{register}}
\fi

% theindex-Environment neu definieren ohne reledmac
\makeatletter
\renewenvironment{theindex}{%
  \ifkorrekturansicht
    \section*{\indexname}%
  \else
    \subsubsection*{Index der erwähnten Entitäten}%
  \fi
  \setlength{\parindent}{0pt}%
  \setlength{\parskip}{0pt plus 0.3pt}%
  \let\item\@idxitem
}{%
  \ifkorrekturansicht\clearpage\fi
}
\makeatother

\IfFileExists{\jobname-pw.ind}{\input{\jobname-pw.ind}}{}

% Quellenangabe nur in der Leseansicht
\ifkorrekturansicht\else
% Fallback-Definitionen, falls die .tex-Datei \titel etc. nicht gesetzt hat
\providecommand{\titel}{}
\providecommand{\editorInnen}{}
\providecommand{\dateiname}{\jobname}

\vspace{3cm}

\vfill

\footnotesize
\textsc{Quelle}: \titel. Herausgegeben von {\editorInnen}. In: \emph{Arthur Schnitzler: Briefwechsel mit Autorinnen und Autoren}.
 Digitale Edition, https://schnitzler-briefe.acdh.oeaw.ac.at/{\dateiname}.html (Stand \today)
\fi

\end{document}


      