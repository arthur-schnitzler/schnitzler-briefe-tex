%% latex-korrekturansicht-vorspann.tex
%% Vorspann für die Korrekturansicht.
%% Lädt die gemeinsame Datei latex-vorspann.tex mit gesetztem Schalter.

\newif\ifkorrekturansicht
\korrekturansichttrue

\input{../tex-inputs/latex-vorspann}


\section[Max Burckhard an Arthur Schnitzler, {[}1. 12. 1900?{]}]{L01083 Max Burckhard an Arthur Schnitzler, {[}1. 12. 1900?{]}}
\nopagebreak\mylabel{L01083v}
\rehead{ }\normalsize\beginnumbering\briefempfaengerindex{Schnitzler, Arthur@\textsc{Schnitzler, Arthur}!zzzBurckhard, Max Eugen@\emph{von Max Eugen Burckhard}!1900-12-011@{{[}1. 12. 1900?{]}}|(be}
\toendnotes[C]{\smallbreak\pagebreak[2]}\Standort{CUL, Schnitzler, B 20.}
\physDesc{Telegramm, 133 Zeichen
\newline{}maschinell
\newline{}Schnitzler: mit Bleistift datiert: »99? 900?
                                 902?« 
\newline{}Ordnung: beschnitten }\toendnotes[C]{\smallbreak}
\pstart
           \noindent{}{\pb}\label{T_L01083-1v}\edtext{hiesige}{\lemma{\textnormal{\emph{hiesige}}}\Cendnote{\textnormal{korrigiert aus: »hisige«}}}\label{T_L01083-1}
               theaterpflichten hielten mich leider fest. meine aufrichtigsten und besten wuensche
               fuer \label{K_L01083-1v}\edtext{heute abend}{\lemma{\textnormal{\emph{heute abend}}}\Cendnote{\textnormal{Das Telegramm könnte am
                     29. 4. 1899, 1. 12. 1900 oder 4. 1. 1902
                  verfasst worden sein – alles Tage, an denen Schnitzler zu Uraufführungen\eventindex{Deutsches Theater Berlin@\textbf{Deutsches Theater Berlin}!Premiere von Der gruene Kakadu – Paracelsus – Die Gefaehrtin. Drei Einakter, 29.4.1899@Premiere von Der grüne Kakadu – Paracelsus – Die Gefährtin. Drei Einakter, 29.4.1899|pwkv}\eventindex{Lobe-Theater@\textbf{Lobe-Theater}!Urauffuehrung von Der Schleier der Beatrice, 1.12.1900@Uraufführung von Der Schleier der Beatrice, 1.12.1900|pwkv}\eventindex{Deutsches Theater Berlin@\textbf{Deutsches Theater Berlin}!Urauffuehrung von Lebendige Stunden, 4.1.1902@Uraufführung von Lebendige Stunden, 4.1.1902|pwkv} im
                  Ausland weilte. Am wahrscheinlichsten ist die Uraufführung\eventindex{Lobe-Theater@\textbf{Lobe-Theater}!Urauffuehrung von Der Schleier der Beatrice, 1.12.1900@Uraufführung von Der Schleier der Beatrice, 1.12.1900|pwkv} von \emph{Der Schleier der
                     Beatrice}\pwindex{Schleier der Beatrice. Schauspiel in fuenf Akten@\emph{Der Schleier der Beatrice. Schauspiel in fünf Akten}|pwk} am 1. 12. 1900 in Breslau\oindex{Breslau@\textbf{Breslau}, \emph{P.PPLA}|pwk}, da viele Wien\oindex{Wien@\textbf{Wien}, \emph{A.ADM2}|pwk}er speziell
                  dafür anreisten. Bei den Theaterpflichten Burckhards\pwindex{Burckhard, Max Eugen 14.07.1854 – 16.03.1912@\textsc{Burckhard, Max Eugen} (14.07.1854 – 16.03.1912), \emph{Schriftsteller/Schriftstellerin, Rechtswissenschaftler/Rechtswissenschaftlerin, Theaterleiter/Theaterleiterin}|pwk} dürfte es sich um seine Tätigkeit als Theaterkritiker handeln.
                  Die Abschrift des Briefwechsels datiert ausschließlich auf
                  »1902?«.}}}\label{K_L01083-1}. gruesse \spacefill\mbox{doctor burckhard +}\pend
           \selectlanguage{ngerman}\endnumbering\briefempfaengerindex{Schnitzler, Arthur@\textsc{Schnitzler, Arthur}!zzzBurckhard, Max Eugen@\emph{von Max Eugen Burckhard}!1900-12-011@{{[}1. 12. 1900?{]}}|)be}\mylabel{L01083h}  \normalsize

\doendnotes{C}
\bigskip
\vfill

\clearpage

\footnotesize

\lohead{\textsc{register}}

% Definiere theindex-Environment komplett neu ohne reledmac
\makeatletter
\renewenvironment{theindex}{%
  \section*{\indexname}%
  \setlength{\parindent}{0pt}%
  \setlength{\parskip}{0pt plus 0.3pt}%
  \let\item\@idxitem
}{%
  \clearpage
}
\makeatother

\IfFileExists{\jobname-pw.ind}{\input{\jobname-pw.ind}}{}

\end{document}

      