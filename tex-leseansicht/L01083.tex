%% latex-leseansicht-vorspann.tex
%% Vorspann für die Leseansicht.
%% Lädt die gemeinsame Datei latex-vorspann.tex mit nicht gesetztem Schalter.

\newif\ifkorrekturansicht
\korrekturansichtfalse

\input{../tex-inputs/latex-vorspann}


         \newcommand{\erwaehnteOrte}{Orte: Breslau, Wien}
         \newcommand{\erwaehnteWerke}{Werke: Der Schleier der Beatrice. Schauspiel in fünf Akten}
               \section[Max Burckhard an Arthur Schnitzler, {[}1. 12. 1900?{]}]{ Max Burckhard an Arthur Schnitzler, {[}1. 12. 1900?{]}}\nopagebreak\mylabel{v}\rehead{ }\begin{ledgroupsized}[t]{13cm}\normalsize\beginnumbering \toendnotes[C]{\smallbreak\pagebreak[2]} \Standort{CUL, Schnitzler, B 20.}
\physDesc{Telegramm
\newline{}maschinell
\newline{}Schnitzler: mit Bleistift datiert: »99? 900?
                                 902?« \newline{}Ordnung: beschnitten }\toendnotes[C]{\smallbreak}\pstart
           \noindent{}{\pb}\label{T_L01083_1v}\edtext{hiesige}{\lemma{\textnormal{\emph{hiesige}}}\Cendnote{\textnormal{korrigiert aus: »hisige«}}}\label{T_L01083_1h} theaterpflichten
               hielten mich leider fest. meine aufrichtigsten und besten wuensche fuer \label{K_L01083_1v}\edtext{heute abend}{\lemma{\textnormal{\emph{heute abend}}}\Cendnote{\textnormal{Das Telegramm könnte am 29. 4. 1899,
                     1. 12. 1900 oder 4. 1. 1902 verfasst sein – alles
                  Tage, an denen Schnitzler zu Uraufführungen im Ausland weilte. Am
                  wahrscheinlichsten ist die Uraufführung von \emph{Der
                     Schleier der Beatrice}\pwindex{Schnitzler, Arthur 15.05.1862 – 21.10.1931@\textsc{Schnitzler, Arthur} (15.05.1862 – 21.10.1931), \emph{Schriftsteller, Mediziner}!Schleier der Beatrice. Schauspiel in fuenf Akten1900-12-01@\strich\emph{Der Schleier der Beatrice. Schauspiel in fünf Akten} {[}1900-12-01{]}|pwk} am 1. 12. 1900 in Breslau\oindex{Breslau@\textbf{Breslau}|pwk}, da viele Wien\oindex{Wien@\textbf{Wien}|pwk}er
                  speziell dafür anreisten. Bei den Theaterpflichten Burckhard\pwindex{Burckhard, Max Eugen 14.07.1854 – 16.03.1912@\textsc{Burckhard, Max Eugen} (14.07.1854 – 16.03.1912), \emph{Schriftsteller, Rechtswissenschaftler, Theaterleiter}|pwk}s dürfte es sich um seine Tätigkeit als Theaterkritiker handeln.
                  Die Abschrift des Briefwechsels datiert ausschließlich auf
                  »1902?«.}}}\label{K_L01083_1h}. gruesse \spacefill\mbox{doctor burckhard +}\pend
           
         
         \endnumbering\mylabel{h}\end{ledgroupsized}  \newcommand{\dateiname}{L01083}\newcommand{\titel}{Max Burckhard an Arthur Schnitzler, [1. 12. 1900?]}\newcommand{\editorInnen}{Martin Anton Müller und Gerd-Hermann Susen}%% latex-leseansicht-abspann.tex
%% Abspann für die Leseansicht.
%% Der Schalter \ifkorrekturansicht ist bereits durch den Vorspann gesetzt.

%% latex-abspann.tex
%% Gemeinsamer Abspann für Korrekturansicht und Leseansicht.
%% Setzt den Schalter \ifkorrekturansicht voraus (gesetzt in den
%% einbindenden Dateien latex-korrekturansicht-abspann.tex bzw.
%% latex-leseansicht-abspann.tex).
%% ---------------------------------------------------------------

\normalsize

% Das esempio-Environment wird nur in der Leseansicht benötigt
\ifkorrekturansicht\else
\newenvironment{esempio}[3]%
{
    \vspace{1.5ex}
    \rlap{\underline{#1}}
    \par
    \setlength{\parindent}{0cm}
    \nopagebreak
    \leftskip=#2cm
    \rightskip=#3cm
}
{
    \par
}
\fi

\doendnotes{C}
\bigskip
\vfill

\clearpage

\footnotesize

\ifkorrekturansicht
  \lohead{\textsc{register}}
\fi

% theindex-Environment neu definieren ohne reledmac
\makeatletter
\renewenvironment{theindex}{%
  \ifkorrekturansicht
    \section*{\indexname}%
  \else
    \subsubsection*{Index der erwähnten Entitäten}%
  \fi
  \setlength{\parindent}{0pt}%
  \setlength{\parskip}{0pt plus 0.3pt}%
  \let\item\@idxitem
}{%
  \ifkorrekturansicht\clearpage\fi
}
\makeatother

\IfFileExists{\jobname-pw.ind}{\input{\jobname-pw.ind}}{}

% Quellenangabe nur in der Leseansicht
\ifkorrekturansicht\else
% Fallback-Definitionen, falls die .tex-Datei \titel etc. nicht gesetzt hat
\providecommand{\titel}{}
\providecommand{\editorInnen}{}
\providecommand{\dateiname}{\jobname}

\vspace{3cm}

\vfill

\footnotesize
\textsc{Quelle}: \titel. Herausgegeben von {\editorInnen}. In: \emph{Arthur Schnitzler: Briefwechsel mit Autorinnen und Autoren}.
 Digitale Edition, https://schnitzler-briefe.acdh.oeaw.ac.at/{\dateiname}.html (Stand \today)
\fi

\end{document}


      