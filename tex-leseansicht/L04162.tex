%% latex-leseansicht-vorspann.tex
%% Vorspann für die Leseansicht.
%% Lädt die gemeinsame Datei latex-vorspann.tex mit nicht gesetztem Schalter.

\newif\ifkorrekturansicht
\korrekturansichtfalse

\input{../tex-inputs/latex-vorspann}


\section[Arthur Schnitzler an Gustav Schwarzkopf, 22. 7. 1910]{L04162 Arthur Schnitzler an Gustav Schwarzkopf, 22. 7. 1910}
\nopagebreak\mylabel{L04162v}
\rehead{ }\normalsize\beginnumbering\briefempfaengerindex{Schwarzkopf, Gustav@\textsc{Schwarzkopf, Gustav}!zzzSchnitzler, Arthur@\emph{von Arthur Schnitzler}!1910-07-221@{22. 7. 1910}|(be}
\toendnotes[C]{\smallbreak\pagebreak[2]}
\correspDesc{Versand  durch Arthur Schnitzler am 22. 7. 1910 in Wien
\newline{}Erhalt  durch Gustav Schwarzkopf im Zeitraum [23. 7. 1910 – 27. 7. 1910?] \textbf{Ort fehlend} }\toendnotes[C]{\smallbreak}
\Standort{CUL, Schnitzler, B 96.}
\physDesc{Brief, 1 Blatt, 4 Seiten, 981 Zeichen
\newline{}Handschrift: Bleistift, deutsche Kurrent}\toendnotes[C]{\smallbreak}
\pstart
           {\pb}\textcolor{gray}{\textbf{Dr. Arthur Schnitzler}}\hfill 22. 7. 910\pend
           
\pstart
           \textcolor{gray}{\textbf{Wien XVIII.
                        Spoettelgasse 7\oindex{Wien@\textbf{Wien}!XVIII., Währing@\textbf{XVIII., Währing}!Edmund-Weiß-Gasse@\textbf{Edmund-Weiß-Gasse}, \emph{Straße}|pw}.}}\pend
           
\pstart{}lieber Guſtav,\pend\vspace{0.5em}
\pstart
           ich habe Sie \label{K_L04162-1v}\edtext{geſtern aufgeſucht}{\lemma{\textnormal{\emph{gestern aufgesucht}}}\Cendnote{\textnormal{Das ist nur implizit im \emph{Tagebuch}\pwindex{Schnitzler, Arthur 15. 5. 1862 Wien – 21. 10. 1931 ebd.@\textsc{Schnitzler, Arthur} (15. 5. 1862 Wien – 21. 10. 1931 ebd.), \emph{Schriftsteller, Mediziner}!Tagebuch@\strich\emph{Tagebuch}|pwk}-Eintrag zum 21. 7. 1910 erwähnt: »Besorgungen in der
                        Stadt\oindex{I., Innere Stadt@\textbf{I., Innere Stadt}, \emph{Verwaltungsgebiet}|pwv}.«}}}\label{K_L04162-1}
               und bei dieſer Gelegenheit erfahren, daß Sie ſchon fort{ }ſind; ich dachte Sie wollten
               erſt Ende dieſer Woche abreiſen. Meinen Bruder\pwindex{Schnitzler, Julius 13.\,7.\,1865 Wien – 29.\,6.\,1939 ebd.@\textsc{Schnitzler, Julius} (13.\,7.\,1865 Wien – 29.\,6.\,1939 ebd.), \emph{Chirurg}|pwv} hatt’ ich brieflich interpellirt, aber keine Antwort
               erhalten, bis er {\pb}\label{K_L04162-2v}\edtext{vorgeſtern Abend perſönlich}{\lemma{\textnormal{\emph{vorgestern Abend persönlich}}}\Cendnote{\textnormal{Vgl. A. S.: \emph{Tagebuch}, 20. 7. 1910.}}}\label{K_L04162-2} erſchien – und ſich in der
               erwarteter Weiſe äußerte: daſs er Krankheitsfälle nicht aus der Ferne beurtheilen
               könne. Soweit ſich theoretiſch-mediziniſches daran knüpfte, entſprach es ungefähr den
               Anſichten, die ich Ihnen gegenüber ausge{\pb}ſprochen. Den \label{K_L04162-3v}\edtext{Chirurgen \textsc{Wiesinger\pwindex{Wiesinger @\textsc{Wiesinger}, \emph{Chirurg}|pw}}}{\lemma{\textnormal{\emph{Chirurgen Wiesinger}}}\Cendnote{\textnormal{In Wien\oindex{Wien@\textbf{Wien}, \emph{Verwaltungsgebiet}|pwk} ist zu dieser Zeit kein Arzt dieses Namens nachgewiesen.}}}\label{K_L04162-3} ke{\geminationn}t mein Bruder\pwindex{Schnitzler, Julius 13.\,7.\,1865 Wien – 29.\,6.\,1939 ebd.@\textsc{Schnitzler, Julius} (13.\,7.\,1865 Wien – 29.\,6.\,1939 ebd.), \emph{Chirurg}|pwv} dem Namen nach als ſehr tüchtig. –\pend
           
\pstart
           – \label{K_L04162-4v}\edtext{Montag oder
                  Dinſtag}{\lemma{\textnormal{\emph{Montag oder
                  Dinstag}}}\Cendnote{\textnormal{Es wurde
                  Dienstag, der 26. 7. 1910.}}}\label{K_L04162-4} will ich, \introOben{}(\introOben{}wahrſcheinlich\introOben{})\introOben{}{ }\strikeout{O}
               allein, da Olga\pwindex{Schnitzler, Olga 17.\,1.\,1882 Wien – 13.\,1.\,1970 Lugano@\textsc{Schnitzler, Olga} (17.\,1.\,1882 Wien – 13.\,1.\,1970 Lugano), \emph{Schauspielerin, Sängerin}|pw} zu viel im Haus\oindex{Wien@\textbf{Wien}!XVIII., Währing@\textbf{XVIII., Währing}!Sternwartestraße 71@\textbf{Sternwartestraße 71}, \emph{Wohngebäude}|pwv} zu thun hat, aus dem die Handwerker
               noch nicht verſchwunden ſind) auf ein paar {\pb}Tage auf den Se{\geminationm}ering\oindex{Semmering@\textbf{Semmering}, \emph{Verwaltungsgebiet}|pw} (vielleicht \label{K_L04162-5v}\edtext{über Mönichkirchen\oindex{Mönichkirchen [Niederösterreich]@\textbf{Mönichkirchen [Niederösterreich]}, \emph{Verwaltungsgebiet}|pw}}{\lemma{\textnormal{\emph{über Mönichkirchen}}}\Cendnote{\textnormal{Nach Mönichkirchen\oindex{Mönichkirchen [Niederösterreich]@\textbf{Mönichkirchen [Niederösterreich]}, \emph{Verwaltungsgebiet}|pwk} kam Schnitzler erst am A. S.: \emph{Wiener Schnitzler}, 28. 7. 1910.}}}\label{K_L04162-5}.) Ihr Bruder\pwindex{Schwarzkopf, Max 12.\,6.\,1857 Wien – 14.\,4.\,1928 ebd.@\textsc{Schwarzkopf, Max} (12.\,6.\,1857 Wien – 14.\,4.\,1928 ebd.), \emph{Rechtsanwalt}|pwv} gab
               mir die Hoffnung, daſs Sie \label{K_L04162-6v}\edtext{auch hinauf}{\lemma{\textnormal{\emph{auch hinauf}}}\Cendnote{\textnormal{Daraus dürfte nichts geworden sein.}}}\label{K_L04162-6} kommen wollen, we{\geminationn} ein Zimmer frei wird. Jedenfalls{ }ſehen wir uns. Meine Adreſſe: Südbahnhotel\oindex{Südbahnhotel [Semmering]@\textbf{Südbahnhotel [Semmering]}, \emph{Hotel}|pw}.\pend
           
\pstart
           Herzlichſt mit Grüßen{\\[\baselineskip]} von uns Allen Ihr \spacefill\mbox{A.}\pend
           \leftskip=0em{}\selectlanguage{ngerman}\endnumbering\briefempfaengerindex{Schwarzkopf, Gustav@\textsc{Schwarzkopf, Gustav}!zzzSchnitzler, Arthur@\emph{von Arthur Schnitzler}!1910-07-221@{22. 7. 1910}|)be}\mylabel{L04162h}
\begin{anhang}
\end{anhang}\newcommand{\dateiname}{L04162}\newcommand{\titel}{Arthur Schnitzler an Gustav Schwarzkopf, 22. 7. 1910}\newcommand{\editorInnen}{Herausgegeben von Jahnke, SelmaMüller, Martin Anton}%% latex-leseansicht-abspann.tex
%% Abspann für die Leseansicht.
%% Der Schalter \ifkorrekturansicht ist bereits durch den Vorspann gesetzt.

%% latex-abspann.tex
%% Gemeinsamer Abspann für Korrekturansicht und Leseansicht.
%% Setzt den Schalter \ifkorrekturansicht voraus (gesetzt in den
%% einbindenden Dateien latex-korrekturansicht-abspann.tex bzw.
%% latex-leseansicht-abspann.tex).
%% ---------------------------------------------------------------

\normalsize

% Das esempio-Environment wird nur in der Leseansicht benötigt
\ifkorrekturansicht\else
\newenvironment{esempio}[3]%
{
    \vspace{1.5ex}
    \rlap{\underline{#1}}
    \par
    \setlength{\parindent}{0cm}
    \nopagebreak
    \leftskip=#2cm
    \rightskip=#3cm
}
{
    \par
}
\fi

\doendnotes{C}
\bigskip
\vfill

\clearpage

\footnotesize

\ifkorrekturansicht
  \lohead{\textsc{register}}
\fi

% theindex-Environment neu definieren ohne reledmac
\makeatletter
\renewenvironment{theindex}{%
  \ifkorrekturansicht
    \section*{\indexname}%
  \else
    \subsubsection*{Index der erwähnten Entitäten}%
  \fi
  \setlength{\parindent}{0pt}%
  \setlength{\parskip}{0pt plus 0.3pt}%
  \let\item\@idxitem
}{%
  \ifkorrekturansicht\clearpage\fi
}
\makeatother

\IfFileExists{\jobname-pw.ind}{\input{\jobname-pw.ind}}{}

% Quellenangabe nur in der Leseansicht
\ifkorrekturansicht\else
% Fallback-Definitionen, falls die .tex-Datei \titel etc. nicht gesetzt hat
\providecommand{\titel}{}
\providecommand{\editorInnen}{}
\providecommand{\dateiname}{\jobname}

\vspace{3cm}

\vfill

\footnotesize
\textsc{Quelle}: \titel. Herausgegeben von {\editorInnen}. In: \emph{Arthur Schnitzler: Briefwechsel mit Autorinnen und Autoren}.
 Digitale Edition, https://schnitzler-briefe.acdh.oeaw.ac.at/{\dateiname}.html (Stand \today)
\fi

\end{document}


