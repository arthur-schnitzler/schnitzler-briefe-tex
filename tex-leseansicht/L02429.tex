%% latex-leseansicht-vorspann.tex
%% Vorspann für die Leseansicht.
%% Lädt die gemeinsame Datei latex-vorspann.tex mit nicht gesetztem Schalter.

\newif\ifkorrekturansicht
\korrekturansichtfalse

\input{../tex-inputs/latex-vorspann}


\section[Arthur Schnitzler an Thomas Mann, 5. 1. 1925]{L02429 Arthur Schnitzler an Thomas Mann, 5. 1. 1925}
\nopagebreak\mylabel{L02429v}
\rehead{ }\normalsize\beginnumbering\briefempfaengerindex{Mann, Thomas@\textsc{Mann, Thomas}!zzzSchnitzler, Arthur@\emph{von Arthur Schnitzler}!1925-01-051@{5. 1. 1925}|(be}
\toendnotes[C]{\smallbreak\pagebreak[2]}
\correspDesc{Versand  durch Arthur Schnitzler am 5. 1. 1925 in Wien
\newline{}Erhalt  durch Thomas Mann im Zeitraum [5. 1. 1925
                  – 9. 1. 1925?] \textbf{Ort fehlend} }\toendnotes[C]{\smallbreak}
\Standort{Zürich, Thomas-Mann-Archiv, B-II-SCHNM-3.}
\physDesc{Brief, 1 Blatt, 2 Seiten, 1382 Zeichen
\newline{}Handschrift: schwarze Tinte, lateinische Kurrent}
\buchAbdrucke{\weitereDrucke{1) Hertha Krotkoff: \emph{Arthur Schnitzler – Thomas Mann: Briefe.} In: \emph{Modern Austrian Literature}, Jg. 7 (1974) Nr. 1/2, S. 23–24.} \weitereDrucke{2) Hans-Ulrich Lindken: \emph{Arthur Schnitzler. Aspekte und Akzente. Materialien zu Leben
                        und Werk}. Frankfurt am Main, Bern, Göttingen: \emph{Peter Lang} 1984, S. 198 (Europäische Hochschulschriften, Reihe 1, Deutsche Sprache und
                        Literatur, 754).} }\toendnotes[C]{\smallbreak}
\pstart
           \raggedleft{}{\pb}Wien\oindex{Wien@\textbf{Wien}, \emph{Verwaltungsgebiet}|pw}, 5. 1. 925\pend
           
\pstart{}lieber und verehrter Herr Thomas Mann,\pend\vspace{0.5em}
\pstart
           daſs man auf das Geschenk eines solchen Buches, wie Sie es nun der Welt gegeben, auch
               nur mit einem Dankbrief erwidern ka{\geminationn}, ist einigermaßen
               lächerlich. Es ist ein wunderbares und ein wundersames Werk, Ihr Zauberberg\pwindex{Mann, Thomas 6.\,6.\,1875 Lübeck – 12.\,8.\,1955 Zürich@\textsc{Mann, Thomas} (6.\,6.\,1875 Lübeck – 12.\,8.\,1955 Zürich), \emph{Schriftsteller}!Zauberberg. Roman@\strich\emph{Der Zauberberg. Roman}|pw} – von Kapitel zu Kapitel war ich tiefer und
               beglückender gefangen und umfangen, und mir war schwer ums Herz, als ich Ihren Castorp\pwindex{Mann, Thomas 6.\,6.\,1875 Lübeck – 12.\,8.\,1955 Zürich@\textsc{Mann, Thomas} (6.\,6.\,1875 Lübeck – 12.\,8.\,1955 Zürich), \emph{Schriftsteller}!Zauberberg. Roman@\strich\emph{Der Zauberberg. Roman}|pwv} in sein
               blutig-unabänderliches Schicksal entlassen mußte. Aber auch wenn er nur in ein
               heitreres und hoffnungsvolleres entrückt worden wäre; – ich hätte um ihn geklagt, da
               Sie doch in keinem Fall weiter von ihm erzählen wollten. Innerhalb {\pb}eines solchen unendlich reichen Complexes ein Element gesondert hervorheben wollen
               ist in jedem Fall ein zu kühnes, u außerdem überflüssiges Unterfangen: und doch
               drängt es mich zu sagen, daſs in der Darstellung von Joachims\pwindex{Mann, Thomas 6.\,6.\,1875 Lübeck – 12.\,8.\,1955 Zürich@\textsc{Mann, Thomas} (6.\,6.\,1875 Lübeck – 12.\,8.\,1955 Zürich), \emph{Schriftsteller}!Zauberberg. Roman@\strich\emph{Der Zauberberg. Roman}|pwv} Hinscheiden und Gestorbensein mir etwas einziges,
               unvergeßliches erreicht scheint. Sie haben den Humor des Sterbens und des Todes
               erfaßt und festgehalten – ich weiß nichts ähnliches in der deutschen Romanliteratur –
               auch in keiner anderen.\pend
           
\pstart
           Manche Fragen erheben sich in der Lecture, aesthetischer, und politischer, und
               religiöser Natur; – ich wünschte sehr über manches einmal mit Ihnen reden zu dürfen.
               Hoffentlich ists mir einmal vergönnt. Zum Ende nur nochmals – Dank, Bewunderung und
               tiefste Sympathie!\pend
           
\pstart
           Ihr{\\[\baselineskip]}\spacefill\mbox{Arthur Schnitzler}\pend
           \leftskip=0em{}\selectlanguage{ngerman}\endnumbering\briefempfaengerindex{Mann, Thomas@\textsc{Mann, Thomas}!zzzSchnitzler, Arthur@\emph{von Arthur Schnitzler}!1925-01-051@{5. 1. 1925}|)be}\mylabel{L02429h}  \newcommand{\dateiname}{L02429}\newcommand{\titel}{Arthur Schnitzler an Thomas Mann, 5. 1. 1925}\newcommand{\editorInnen}{Martin Anton Müller und Gerd-Hermann Susen}%% latex-leseansicht-abspann.tex
%% Abspann für die Leseansicht.
%% Der Schalter \ifkorrekturansicht ist bereits durch den Vorspann gesetzt.

%% latex-abspann.tex
%% Gemeinsamer Abspann für Korrekturansicht und Leseansicht.
%% Setzt den Schalter \ifkorrekturansicht voraus (gesetzt in den
%% einbindenden Dateien latex-korrekturansicht-abspann.tex bzw.
%% latex-leseansicht-abspann.tex).
%% ---------------------------------------------------------------

\normalsize

% Das esempio-Environment wird nur in der Leseansicht benötigt
\ifkorrekturansicht\else
\newenvironment{esempio}[3]%
{
    \vspace{1.5ex}
    \rlap{\underline{#1}}
    \par
    \setlength{\parindent}{0cm}
    \nopagebreak
    \leftskip=#2cm
    \rightskip=#3cm
}
{
    \par
}
\fi

\doendnotes{C}
\bigskip
\vfill

\clearpage

\footnotesize

\ifkorrekturansicht
  \lohead{\textsc{register}}
\fi

% theindex-Environment neu definieren ohne reledmac
\makeatletter
\renewenvironment{theindex}{%
  \ifkorrekturansicht
    \section*{\indexname}%
  \else
    \subsubsection*{Index der erwähnten Entitäten}%
  \fi
  \setlength{\parindent}{0pt}%
  \setlength{\parskip}{0pt plus 0.3pt}%
  \let\item\@idxitem
}{%
  \ifkorrekturansicht\clearpage\fi
}
\makeatother

\IfFileExists{\jobname-pw.ind}{\input{\jobname-pw.ind}}{}

% Quellenangabe nur in der Leseansicht
\ifkorrekturansicht\else
% Fallback-Definitionen, falls die .tex-Datei \titel etc. nicht gesetzt hat
\providecommand{\titel}{}
\providecommand{\editorInnen}{}
\providecommand{\dateiname}{\jobname}

\vspace{3cm}

\vfill

\footnotesize
\textsc{Quelle}: \titel. Herausgegeben von {\editorInnen}. In: \emph{Arthur Schnitzler: Briefwechsel mit Autorinnen und Autoren}.
 Digitale Edition, https://schnitzler-briefe.acdh.oeaw.ac.at/{\dateiname}.html (Stand \today)
\fi

\end{document}


