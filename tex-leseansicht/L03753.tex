%% latex-korrekturansicht-vorspann.tex
%% Vorspann für die Korrekturansicht.
%% Lädt die gemeinsame Datei latex-vorspann.tex mit gesetztem Schalter.

\newif\ifkorrekturansicht
\korrekturansichttrue

\input{../tex-inputs/latex-vorspann}


\section[Arthur Schnitzler an Stefan Zweig, 22. 1. 1923]{L03753 Arthur Schnitzler an Stefan Zweig, 22. 1. 1923}
\nopagebreak\mylabel{L03753v}
\rehead{ }\normalsize\beginnumbering\briefempfaengerindex{Zweig, Stefan@\textsc{Zweig, Stefan}!zzzSchnitzler, Arthur@\emph{von Arthur Schnitzler}!1923-01-221@{22. 1. 1923}|(be}
\toendnotes[C]{\smallbreak\pagebreak[2]}\Standort{Jerusalem, National Library of Israel, ARC. Ms. Var. 305 1 58 Stefan Zweig Collection.}
\physDesc{Briefkarte, 1 Blatt, 1 Seite, 842 Zeichen
\newline{}Schreibmaschine
\newline{}Handschrift: Bleistift, lateinische Kurrent (\noindent{}minimale Korrekturen, Unterschrift)}
\pstart
           {\pb}\textcolor{gray}{\textbf{D\textsuperscript{R} ARTHUR SCHNITZLER}}\hfill {\pb}22. 1. 1923.\pend
           
\pstart
           \textcolor{gray}{\textbf{WIEN, XVIII.
                           STERNWARTESTRASSE 71\oindex{Sternwartestrasse 71@\textbf{Sternwartestraße 71}, \emph{Wohngebäude (K.WHS)}|pw}.}}\pend
           
\pstart{}Lieber und verehrter Herr Doktor.\pend\vspace{0.5em}
\pstart
           Herr Alzir Hella\pwindex{Hella, Alzir 1881-12-30 – 1953-07-14@\textsc{Hella, Alzir} (1881-12-30 – 1953-07-14), \emph{Übersetzer/Übersetzerin}|pw} hatte sich schon an Fischer\pwindex{Fischer, Samuel 24.12.1859 – 15.10.1934@\textsc{Fischer, Samuel} (24.12.1859 – 15.10.1934), \emph{Verleger/Verlegerin}|pw} gewandt, aber es ist mir im Grunde lieber mit ihm persönlich
               zu verhandeln. »Casanovas Heimfahrt\pwindex{Casanovas Heimfahrt@\emph{Casanovas Heimfahrt}|pw}« ist schon halb und
               halb vergeben, »Frau Beate\pwindex{Frau Beate und ihr Sohn. Novelle@\emph{Frau Beate und ihr Sohn. Novelle}|pw}« ist noch frei und ich wäre
               gern geneigt sie zur Uebersetzung ins Französische dem von Ihnen empfohlenen Herrn Hella\pwindex{Hella, Alzir 1881-12-30 – 1953-07-14@\textsc{Hella, Alzir} (1881-12-30 – 1953-07-14), \emph{Übersetzer/Übersetzerin}|pw} zu überlassen, wenn der Verleger sich zu einer
               Garantie und für einen bestimmten Termin verpflichtet\introOben{}e\introOben{}. Sonst sind alle diese Sachen
               gar zu unsicher. Vielleicht ist es das Richtigste, wenn Sie, lieber Herr Doktor, der
               ja mit Hella\pwindex{Hella, Alzir 1881-12-30 – 1953-07-14@\textsc{Hella, Alzir} (1881-12-30 – 1953-07-14), \emph{Übersetzer/Übersetzerin}|pw} in Verbindung zu stehen scheint, ihm das
               gelegentlich mitteilt\substVorne{}\textsuperscript{.}\substDazwischen{}?\substHinten{} Oder halten sie es für richtiger, dass ich ihm direkt
               schreibe?\pend
           
\pstart
           Seien Sie vielmals gegrüsst, auf baldiges Wiedersehen!{\\[\baselineskip]}Ihr herzlich ergebener{\\[\baselineskip]}\spacefill\mbox{{[}hs.:{]} Arthur Schnitzler}\pend
           \leftskip=0em{}
\pstart
           \noindent{}{[}ms.:{]} Herrn Dr. Stefan Zweig,{\\}Salzburg,
                     Kapuzinerberg 5\oindex{Paschinger Schloessl@\textbf{Paschinger Schlössl}, \emph{Wohngebäude (K.WHS)}|pw}.\pend
           \selectlanguage{ngerman}\endnumbering\briefempfaengerindex{Zweig, Stefan@\textsc{Zweig, Stefan}!zzzSchnitzler, Arthur@\emph{von Arthur Schnitzler}!1923-01-221@{22. 1. 1923}|)be}\mylabel{L03753h}
\begin{anhang}
\end{anhang}\normalsize

\doendnotes{C}
\bigskip
\vfill

\clearpage

\footnotesize

\lohead{\textsc{register}}

% Definiere theindex-Environment komplett neu ohne reledmac
\makeatletter
\renewenvironment{theindex}{%
  \section*{\indexname}%
  \setlength{\parindent}{0pt}%
  \setlength{\parskip}{0pt plus 0.3pt}%
  \let\item\@idxitem
}{%
  \clearpage
}
\makeatother

\IfFileExists{\jobname-pw.ind}{\input{\jobname-pw.ind}}{}

\end{document}

      