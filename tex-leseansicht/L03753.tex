%% latex-leseansicht-vorspann.tex
%% Vorspann für die Leseansicht.
%% Lädt die gemeinsame Datei latex-vorspann.tex mit nicht gesetztem Schalter.

\newif\ifkorrekturansicht
\korrekturansichtfalse

\input{../tex-inputs/latex-vorspann}


\section[Arthur Schnitzler an Stefan Zweig, 22. 1. 1923]{L03753 Arthur Schnitzler an Stefan Zweig, 22. 1. 1923}
\nopagebreak\mylabel{L03753v}
\rehead{ }\normalsize\beginnumbering\briefempfaengerindex{Zweig, Stefan@\textsc{Zweig, Stefan}!zzzSchnitzler, Arthur@\emph{von Arthur Schnitzler}!1923-01-221@{22. 1. 1923}|(be}
\toendnotes[C]{\smallbreak\pagebreak[2]}
\correspDesc{Versand  durch Arthur Schnitzler am 22. 1. 1923 in Wien
\newline{}Erhalt  durch Stefan Zweig im Zeitraum [23. 1. 1923 –
            27. 1. 1923?] in Salzburg}\toendnotes[C]{\smallbreak}
\Standort{Jerusalem, National Library of Israel, ARC. Ms. Var. 305 1 58 Stefan Zweig Collection.}
\physDesc{Briefkarte, 840 Zeichen
\newline{}Schreibmaschine
\newline{}Handschrift: Bleistift, lateinische Kurrent (\noindent{}minimale Korrekturen, Unterschrift)}\toendnotes[C]{\smallbreak}
\pstart
           {\pb}\textcolor{gray}{\textbf{D\textsuperscript{R} ARTHUR SCHNITZLER}}\hfill {\pb}22. 1. 1923.\pend
           
\pstart
           \textcolor{gray}{\textbf{WIEN, XVIII. STERNWARTESTRASSE 71\oindex{Wien@\textbf{Wien}!XVIII., Währing@\textbf{XVIII., Währing}!Sternwartestraße 71@\textbf{Sternwartestraße 71}, \emph{Wohngebäude}|pw}.}}\pend
           
\pstart{}Lieber und verehrter Herr Doktor.\pend\vspace{0.5em}
\pstart
           Herr Alzir Hella\pwindex{Hella, Alzir 30.\,12.\,1881 Vieux Condé – 14.\,7.\,1953 Paris@\textsc{Hella, Alzir} (30.\,12.\,1881 Vieux Condé – 14.\,7.\,1953 Paris), \emph{Übersetzer}|pw} hatte sich schon \label{K_L03753-1v}\edtext{an Fischer\pwindex{Fischer, Samuel 24.\,12.\,1859 Liptovský Mikuláš – 15.\,10.\,1934 Berlin@\textsc{Fischer, Samuel} (24.\,12.\,1859 Liptovský Mikuláš – 15.\,10.\,1934 Berlin), \emph{Verleger}|pw}
            gewandt}{\lemma{\textnormal{\emph{an Fischer
            gewandt}}}\Cendnote{\textnormal{Am 3. 1. 1923 schrieb
              Fischer\pwindex{Fischer, Samuel 24.\,12.\,1859 Liptovský Mikuláš – 15.\,10.\,1934 Berlin@\textsc{Fischer, Samuel} (24.\,12.\,1859 Liptovský Mikuláš – 15.\,10.\,1934 Berlin), \emph{Verleger}|pwk} an Schnitzler: »Was die Erzählungen für Frankreich\oindex{Frankreich@\textbf{Frankreich}|pw} anbetrifft, so handelt es sich um die Anfrage eines Herrn Alzir Hella\pwindex{Hella, Alzir 30.\,12.\,1881 Vieux Condé – 14.\,7.\,1953 Paris@\textsc{Hella, Alzir} (30.\,12.\,1881 Vieux Condé – 14.\,7.\,1953 Paris), \emph{Übersetzer}|pw} in Paris\oindex{Paris@\textbf{Paris}, \emph{Hauptstadt}|pw}, die Sache taugt wohl nicht allzuviel. Ich werde mit dem Herrn
                korrespondieren und Ihnen dann eventuell Weiteres mitteilen«, siehe Arthur Schnitzler: \emph{Mikrofilme}, \url{https://schnitzler\_mikrofilme.acdh.oeaw.ac.at/1416743\_0576}. Schnitzler schrieb am 22. 1. 1923 an
              Leo Greiner\pwindex{Greiner, Leo 1.\,4.\,1876 Brünn – 22.\,8.\,1928 Berlin@\textsc{Greiner, Leo} (1.\,4.\,1876 Brünn – 22.\,8.\,1928 Berlin), \emph{Schriftsteller, Verlagslektor}|pwk} vom \emph{Fischer-Verlag}\orgindex{S. Fischer Verlag@S. Fischer Verlag|pwk}: »Herr Alzir Hella\pwindex{Hella, Alzir 30.\,12.\,1881 Vieux Condé – 14.\,7.\,1953 Paris@\textsc{Hella, Alzir} (30.\,12.\,1881 Vieux Condé – 14.\,7.\,1953 Paris), \emph{Übersetzer}|pw}
              hat sich nun, von Stefan Zweig\pwindex{Zweig, Stefan 28.\,11.\,1881 Wien – 23.\,2.\,1942 Petrópolis@\textsc{Zweig, Stefan} (28.\,11.\,1881 Wien – 23.\,2.\,1942 Petrópolis), \emph{Schriftsteller}|pw} empfohlen,
              direkt an mich gewandt und ich werde ihm auch direkt schreiben. Um Missverständnissen
              ein für alle Mal vorzubeugen, möchte ich heute nur prinzipiell feststellen: der
              Umstand, dass sich Leute mit Anfragen über meine Werke öfters an den Verlag Fischer\orgindex{S. Fischer Verlag@S. Fischer Verlag|pw} wenden, in Unkenntnis, dass ihm für Abschlüsse
              mit dem Ausland nur im Falle ausdrücklicher Genehmigung meinerseits, also von Fall zu
              Fall ein Recht zusteht, verleiht dem Verlag
                Fischer\orgindex{S. Fischer Verlag@S. Fischer Verlag|pw} natürlich nicht, sozusagen automatisch sich in einem solchen Fall als
                meinen Vertreter zu betrachten«, siehe Arthur Schnitzler: \emph{Mikrofilme}, \url{https://schnitzler\_mikrofilme.acdh.oeaw.ac.at/1416739\_0263}. }}}\label{K_L03753-1}, aber es ist mir im Grunde lieber mit
          ihm persönlich zu verhandeln. »Casanovas Heimfahrt\pwindex{Schnitzler, Arthur 15.\,5.\,1862 Wien – 21.\,10.\,1931 ebd.@\textsc{Schnitzler, Arthur} (15.\,5.\,1862 Wien – 21.\,10.\,1931 ebd.), \emph{Schriftsteller, Mediziner}!Casanovas Heimfahrt@\strich\emph{Casanovas Heimfahrt}|pw}«
          ist schon halb und halb vergeben, »Frau Beate\pwindex{Schnitzler, Arthur 15.\,5.\,1862 Wien – 21.\,10.\,1931 ebd.@\textsc{Schnitzler, Arthur} (15.\,5.\,1862 Wien – 21.\,10.\,1931 ebd.), \emph{Schriftsteller, Mediziner}!Frau Beate und ihr Sohn. Novelle@\strich\emph{Frau Beate und ihr Sohn. Novelle}|pw}« ist
          noch frei und ich wäre gern geneigt sie zur Uebersetzung ins Französische dem von Ihnen
          empfohlenen Herrn Hella\pwindex{Hella, Alzir 30.\,12.\,1881 Vieux Condé – 14.\,7.\,1953 Paris@\textsc{Hella, Alzir} (30.\,12.\,1881 Vieux Condé – 14.\,7.\,1953 Paris), \emph{Übersetzer}|pw} zu überlassen, wenn der
          Verleger sich zu einer Garantie und für einen bestimmten Termin verpflichtet\introOben{}e\introOben{}. Sonst sind alle diese Sachen gar zu unsicher. Vielleicht ist es
          das Richtigste, wenn Sie, lieber Herr Doktor, der ja mit Hella\pwindex{Hella, Alzir 30.\,12.\,1881 Vieux Condé – 14.\,7.\,1953 Paris@\textsc{Hella, Alzir} (30.\,12.\,1881 Vieux Condé – 14.\,7.\,1953 Paris), \emph{Übersetzer}|pw} in Verbindung zu stehen scheint, ihm das gelegentlich mitteilt\substVorne{}\textsuperscript{.}\substDazwischen{}?\substHinten{} Oder halten sie es für richtiger, dass \label{K_L03753-2v}\edtext{ich ihm direkt schreibe}{\lemma{\textnormal{\emph{ich ihm direkt schreibe}}}\Cendnote{\textnormal{Im Nachlass Schnitzlers befindet sich der
            Durchschlag eines Briefes an Hella\pwindex{Hella, Alzir 30.\,12.\,1881 Vieux Condé – 14.\,7.\,1953 Paris@\textsc{Hella, Alzir} (30.\,12.\,1881 Vieux Condé – 14.\,7.\,1953 Paris), \emph{Übersetzer}|pwk}, das mit dem
            Vortag datiert ist. Die Formulierung im vorliegenden Brief lässt es aber als unklar
            erscheinen, ob das Schreiben an Hella\pwindex{Hella, Alzir 30.\,12.\,1881 Vieux Condé – 14.\,7.\,1953 Paris@\textsc{Hella, Alzir} (30.\,12.\,1881 Vieux Condé – 14.\,7.\,1953 Paris), \emph{Übersetzer}|pwk} überhaupt
            abgeschickt wurde. »19. 2. 1923{ / }Sehr geehrter Herr Hella\pwindex{Hella, Alzir 30.\,12.\,1881 Vieux Condé – 14.\,7.\,1953 Paris@\textsc{Hella, Alzir} (30.\,12.\,1881 Vieux Condé – 14.\,7.\,1953 Paris), \emph{Übersetzer}|pw}.{ / }In den nächsten Tagen kommt Frau Hofrätin Bertha
                  Zuckerkandl\pwindex{Zuckerkandl, Berta 13.\,4.\,1864 Wien – 16.\,10.\,1945 Paris@\textsc{Zuckerkandl, Berta} (13.\,4.\,1864 Wien – 16.\,10.\,1945 Paris), \emph{Schriftstellerin, Journalistin, Übersetzerin}|pw} nach Paris\oindex{Paris@\textbf{Paris}, \emph{Hauptstadt}|pw} und wird dort bei
                ihrer Schwester, MMe. Paul Clemenceau\pwindex{Clemenceau, Sophie 25.\,5.\,1862 – 24.\,9.\,1937@\textsc{Clemenceau, Sophie} (25.\,5.\,1862 – 24.\,9.\,1937)|pw}, 12, Avenue d’Eylau\oindex{12, Avenue d’Eylau@\textbf{12, Avenue d’Eylau}, \emph{Wohngebäude}|pw} wohnen. Darf ich Sie bitten
                sich mit ihr in Verbindung zu setzen{[},{]} ich habe ihr von Ihrem
                freundlichen Antrag Mitteilung gemacht und sie ermächtigt mit Ihnen weiter darüber
                zu unterhandeln. Es wäre mir natürlich sehr willkommen, wenn eine meiner Novellen in
                  ›Monde Nouveau\pwindex{Monde nouveau@\emph{Monde nouveau}|pw}‹ zum Abdruck käme. ›Casanovas Heimfahrt\pwindex{Schnitzler, Arthur 15.\,5.\,1862 Wien – 21.\,10.\,1931 ebd.@\textsc{Schnitzler, Arthur} (15.\,5.\,1862 Wien – 21.\,10.\,1931 ebd.), \emph{Schriftsteller, Mediziner}!Casanovas Heimfahrt@\strich\emph{Casanovas Heimfahrt}|pw}‹ ist nicht frei, aber
                vielleicht erlange ich mein Rechte auch auf diese Novelle wieder zurück, da der Bewerber\pwindex{?? [Französischer Übersetzer, der Casanovas Heimfahrt übersetzen will, 1293] @\textsc{?? [Französischer Übersetzer, der Casanovas Heimfahrt übersetzen will, 1293]}|pwv} bisher meines
                Wissens die Uebersetzung nicht in Angriff genommen hat. Ueber die Honorarbedingungen
                wird Frau Hofrätin Zuckerkandl\pwindex{Zuckerkandl, Berta 13.\,4.\,1864 Wien – 16.\,10.\,1945 Paris@\textsc{Zuckerkandl, Berta} (13.\,4.\,1864 Wien – 16.\,10.\,1945 Paris), \emph{Schriftstellerin, Journalistin, Übersetzerin}|pw} mit Ihnen
                reden.{ / }Mit verbindlichem Dank für Ihr freundliches Interesse und Ihre liebenswürdigen
                Worte{ / }Ihr sehr ergebener{ / }{[}Raum für die Unterschrift{]}{ / }Herrn Alzir Hella\pwindex{Hella, Alzir 30.\,12.\,1881 Vieux Condé – 14.\,7.\,1953 Paris@\textsc{Hella, Alzir} (30.\,12.\,1881 Vieux Condé – 14.\,7.\,1953 Paris), \emph{Übersetzer}|pw}, Paris, 18, rue de l’Odéon\oindex{18, rue de l’Odéon@\textbf{18, rue de l’Odéon}, \emph{Wohngebäude}|pw}.« (Brief von Schnitzler an Alzir Hella\pwindex{Hella, Alzir 30.\,12.\,1881 Vieux Condé – 14.\,7.\,1953 Paris@\textsc{Hella, Alzir} (30.\,12.\,1881 Vieux Condé – 14.\,7.\,1953 Paris), \emph{Übersetzer}|pwk},
              19. 2. 1923, \emph{DLA},
              HS.1985.1.969). Hella\pwindex{Hella, Alzir 30.\,12.\,1881 Vieux Condé – 14.\,7.\,1953 Paris@\textsc{Hella, Alzir} (30.\,12.\,1881 Vieux Condé – 14.\,7.\,1953 Paris), \emph{Übersetzer}|pwk} übersetzte gemeinsam mit Olivier Bournac\pwindex{Bournac, Olivier 13.\,8.\,1885 Saint-Amans-du-Pech – Anfang Januar 1931 Toulon@\textsc{Bournac, Olivier} (13.\,8.\,1885 Saint-Amans-du-Pech – Anfang Januar 1931 Toulon), \emph{Schriftsteller, Übersetzer}|pwk} drei Texte von Schnitzler. Als erstes erschien 1925 mit \emph{Mourir}\pwindex{Schnitzler, Arthur 15.\,5.\,1862 Wien – 21.\,10.\,1931 ebd.@\textsc{Schnitzler, Arthur} (15.\,5.\,1862 Wien – 21.\,10.\,1931 ebd.), \emph{Schriftsteller, Mediziner}!Mourir. Roman [1925]@\strich\emph{Mourir. Roman [1925]}|pwk} eine Neuübersetzung von \emph{Sterben}\pwindex{Schnitzler, Arthur 15.\,5.\,1862 Wien – 21.\,10.\,1931 ebd.@\textsc{Schnitzler, Arthur} (15.\,5.\,1862 Wien – 21.\,10.\,1931 ebd.), \emph{Schriftsteller, Mediziner}!Sterben. Novelle@\strich\emph{Sterben. Novelle}|pwk}, danach kamen noch \emph{Madame Beate et son fils}\pwindex{Schnitzler, Arthur 15.\,5.\,1862 Wien – 21.\,10.\,1931 ebd.@\textsc{Schnitzler, Arthur} (15.\,5.\,1862 Wien – 21.\,10.\,1931 ebd.), \emph{Schriftsteller, Mediziner}!Madame Beate et son fils@\strich\emph{Madame Beate et son fils}|pwk} (Oktober–November 1928) und \emph{Le Célibataire}\pwindex{Schnitzler, Arthur 15.\,5.\,1862 Wien – 21.\,10.\,1931 ebd.@\textsc{Schnitzler, Arthur} (15.\,5.\,1862 Wien – 21.\,10.\,1931 ebd.), \emph{Schriftsteller, Mediziner}!Le Célibataire@\strich\emph{Le Célibataire}|pwk} (\emph{Der
              Tod des Junggesellen}\pwindex{Schnitzler, Arthur 15.\,5.\,1862 Wien – 21.\,10.\,1931 ebd.@\textsc{Schnitzler, Arthur} (15.\,5.\,1862 Wien – 21.\,10.\,1931 ebd.), \emph{Schriftsteller, Mediziner}!Tod des Junggesellen. Novelle@\strich\emph{Der Tod des Junggesellen. Novelle}|pwk}, März 1929). }}}\label{K_L03753-2}?\pend
           
\pstart
           Seien Sie vielmals gegrüsst, auf baldiges Wiedersehen!{\\[\baselineskip]}Ihr herzlich ergebener{\\[\baselineskip]}\spacefill\mbox{{[}hs.:{]} Arthur Schnitzler}\pend
           \leftskip=0em{}
\pstart
           \noindent{}{[}ms.:{]} Herrn Dr. Stefan Zweig,{\\}Salzburg, Kapuzinerberg 5\oindex{Paschinger Schlössl@\textbf{Paschinger Schlössl}, \emph{Wohngebäude}|pw}.\pend
           \selectlanguage{ngerman}\endnumbering\briefempfaengerindex{Zweig, Stefan@\textsc{Zweig, Stefan}!zzzSchnitzler, Arthur@\emph{von Arthur Schnitzler}!1923-01-221@{22. 1. 1923}|)be}\mylabel{L03753h}  \newcommand{\dateiname}{L03753}\newcommand{\titel}{Arthur Schnitzler an Stefan Zweig, 22. 1. 1923}\newcommand{\editorInnen}{Selma Jahnke und Martin Anton Müller}%% latex-leseansicht-abspann.tex
%% Abspann für die Leseansicht.
%% Der Schalter \ifkorrekturansicht ist bereits durch den Vorspann gesetzt.

%% latex-abspann.tex
%% Gemeinsamer Abspann für Korrekturansicht und Leseansicht.
%% Setzt den Schalter \ifkorrekturansicht voraus (gesetzt in den
%% einbindenden Dateien latex-korrekturansicht-abspann.tex bzw.
%% latex-leseansicht-abspann.tex).
%% ---------------------------------------------------------------

\normalsize

% Das esempio-Environment wird nur in der Leseansicht benötigt
\ifkorrekturansicht\else
\newenvironment{esempio}[3]%
{
    \vspace{1.5ex}
    \rlap{\underline{#1}}
    \par
    \setlength{\parindent}{0cm}
    \nopagebreak
    \leftskip=#2cm
    \rightskip=#3cm
}
{
    \par
}
\fi

\doendnotes{C}
\bigskip
\vfill

\clearpage

\footnotesize

\ifkorrekturansicht
  \lohead{\textsc{register}}
\fi

% theindex-Environment neu definieren ohne reledmac
\makeatletter
\renewenvironment{theindex}{%
  \ifkorrekturansicht
    \section*{\indexname}%
  \else
    \subsubsection*{Index der erwähnten Entitäten}%
  \fi
  \setlength{\parindent}{0pt}%
  \setlength{\parskip}{0pt plus 0.3pt}%
  \let\item\@idxitem
}{%
  \ifkorrekturansicht\clearpage\fi
}
\makeatother

\IfFileExists{\jobname-pw.ind}{\input{\jobname-pw.ind}}{}

% Quellenangabe nur in der Leseansicht
\ifkorrekturansicht\else
% Fallback-Definitionen, falls die .tex-Datei \titel etc. nicht gesetzt hat
\providecommand{\titel}{}
\providecommand{\editorInnen}{}
\providecommand{\dateiname}{\jobname}

\vspace{3cm}

\vfill

\footnotesize
\textsc{Quelle}: \titel. Herausgegeben von {\editorInnen}. In: \emph{Arthur Schnitzler: Briefwechsel mit Autorinnen und Autoren}.
 Digitale Edition, https://schnitzler-briefe.acdh.oeaw.ac.at/{\dateiname}.html (Stand \today)
\fi

\end{document}


