%% latex-korrekturansicht-vorspann.tex
%% Vorspann für die Korrekturansicht.
%% Lädt die gemeinsame Datei latex-vorspann.tex mit gesetztem Schalter.

\newif\ifkorrekturansicht
\korrekturansichttrue

\input{../tex-inputs/latex-vorspann}


\section[Paul Goldmann an Arthur Schnitzler, 26. 4. 1902]{L02634 Paul Goldmann an Arthur Schnitzler, 26. 4. 1902}
\nopagebreak\mylabel{L02634v}
\rehead{ }\normalsize\beginnumbering\briefempfaengerindex{Schnitzler, Arthur@\textsc{Schnitzler, Arthur}!zzzGoldmann, Paul@\emph{von Paul Goldmann}!1902-04-261@{26. 4. 1902}|(be}
\toendnotes[C]{\smallbreak\pagebreak[2]}\Standort{DLA, A:Schnitzler, HS.NZ85.1.3172.}
\physDesc{Telegramm, 1465 Zeichen
\newline{}maschinell
\newline{}Versand: 1) Stempel: »\nobreak{}26. April 1902, Kvasnicka\pwindex{Kvasnicka, Josef @\textsc{Kvasnicka, Josef}, \emph{Telegrafenbeamter/Telegrafenbeamtin}|pw}\nobreak{}«.   2) Stempel: »\nobreak{}12 40\nobreak{}«.  3) mit Bleistift zweites Blatt beschriftet mit: »II. Blatt
                                    ad N\textsuperscript{o} 99946 De Berlin\oindex{Berlin@\textbf{Berlin}, \emph{P.PPLC}|pw}« 4) mit Bleistift von unbekannter Hand Vermerk des Postrayons:
                                    »71«}\toendnotes[C]{\smallbreak}\pstart{}{\pb}arthur schnitzler wien\oindex{Wien@\textbf{Wien}, \emph{A.ADM2}|pw}\pend{}\pstart{}frankgasze 1\oindex{Frankgasse 1@\textbf{Frankgasse 1}, \emph{Wohngebäude (K.WHS)}|pw}=\pend{}{\bigskip}\vspace{1em}
\pstart
           \noindent{}\centering{}{\pb}de berlin\oindex{Berlin@\textbf{Berlin}, \emph{P.PPLC}|pw} 99946
               196 26/4{ }10 20 m =\pend
           
\pstart
           in ›taeglichen rundschau\orgindex{Taegliche Rundschau@Tägliche Rundschau|pw}‹ veroeffentlicht
               kritiker karl strecker\pwindex{Strecker, Karl 1862-04-08 – 1933-02-19@\textsc{Strecker, Karl} (1862-04-08 – 1933-02-19), \emph{Theaterkritiker/Theaterkritikerin}|pw} folgenden \label{K_L02634-1v}\edtext{artikel\pwindex{litterarisch-dramatisches Hochstapler-Stuecklein@\emph{Ein litterarisch-dramatisches Hochstapler-Stücklein}|pwv}}{\lemma{\textnormal{\emph{artikel}}}\Cendnote{\textnormal{Karl Strecker\pwindex{Strecker, Karl 1862-04-08 – 1933-02-19@\textsc{Strecker, Karl} (1862-04-08 – 1933-02-19), \emph{Theaterkritiker/Theaterkritikerin}|pwk}: \emph{Ein litterarisch-dramatisches Hochstapler-Stücklein}\pwindex{litterarisch-dramatisches Hochstapler-Stuecklein@\emph{Ein litterarisch-dramatisches Hochstapler-Stücklein}|pwk}.
                     In: \emph{Tägliche Rundschau}\pwindex{Taegliche Rundschau@\emph{Tägliche Rundschau}|pwk}, Jg. 22, Nr. 193,
                        26. 4. 1902, Morgen-Blatt, Erste Beilage,
                     S. 3. Siehe auch A. S.: \emph{Tagebuch}, 26. 4. 1902.}}}\label{K_L02634-1} mit fragenden ueberschrift »ein literarisch dramatisches
               hochstaplerstuecklein«? am donnerstag{ }mittag erhielt ich aus wien\oindex{Wien@\textbf{Wien}, \emph{A.ADM2}|pw} ein an
               meine persoenliche adresze gerichtetes telegramm, das also lautete: »frejtag{ }karl wejsz-theater\orgindex{Rose-Theater@Rose-Theater|pw}
                  urpremi{[}ere{]} von ›\label{K_L02634-2v}\edtext{kinder der armen\pwindex{Kinder der Armen@\emph{Die Kinder der Armen}|pw}}{\lemma{\textnormal{\emph{kinder der armen}}}\Cendnote{\textnormal{der Empfänger duplizierte bei der
                  Transkription: »kinder des kinder der armen«}}}\label{K_L02634-2}{[}‹{]} empfiehlt genejgter aufmerksamkejt ergebenst arthur
                  schnitzler.{[}«{]} von diesem telegramm wuerde ich
               selbstverstaendlich niemals oeffentlich notiz genommen haben, wenn ich
                  annehm{[}en{]} koennte, dasz es wirklich von schnitzler aus {\pb}litterarischem interesze abgesandt worden
                  sej{[}n{]} haette. lejder liegt aber fuer mich nach betrachtung
               dieses ›volksstueckes\pwindex{Kinder der Armen@\emph{Die Kinder der Armen}|pwv}‹ der
               handgrejfliche verdacht nahe, dasz hier ein arger miszbrauch mit dem namen eines
               feinfuehligen poeten getrieben worden ist. (ein kollege vom »berliner tageblatt\orgindex{Berliner Tageblatt@Berliner Tageblatt|pw}« hat uebrigens genau daszelbe telegramm
               zur selbigen stunde erhalten). unter diesen umstaenden sehe ich mich
               genoetigt, die offene frage an schnitzler zu richten, ob er diese seltsame
               aufmunterung wirklich abgefaszt hat? wenn nicht (und das nehme ich an), so liegt es
               ebenso in seinem interesze wie in dem der ehre unserer deutschen
                  dramatisch{[}e{]}n litteratur, dasz dieser herr verfaszer\pwindex{Greger-Jurco, Ernest von *~11.08.1860@\textsc{Gréger-Jurco, Ernest von} (*~11.08.1860), \emph{Schriftsteller/Schriftstellerin}|pwv}, ernest von jurco\pwindex{Greger-Jurco, Ernest von *~11.08.1860@\textsc{Gréger-Jurco, Ernest von} (*~11.08.1860), \emph{Schriftsteller/Schriftstellerin}|pw} nennt sich die kapazitaet\pwindex{Greger-Jurco, Ernest von *~11.08.1860@\textsc{Gréger-Jurco, Ernest von} (*~11.08.1860), \emph{Schriftsteller/Schriftstellerin}|pwv}, entlarvt
                  wird{[}.{]} sowejt artikel. telegraphire dementi an strecker\pwindex{Strecker, Karl 1862-04-08 – 1933-02-19@\textsc{Strecker, Karl} (1862-04-08 – 1933-02-19), \emph{Theaterkritiker/Theaterkritikerin}|pw} redaktion taeglichen rundschau\orgindex{Taegliche Rundschau@Tägliche Rundschau|pw}{ }berlin zimmerstrasze 7 und 8\oindex{Zimmerstrasse@\textbf{Zimmerstraße}, \emph{Straße (K.STR)}|pw}. grusz \spacefill\mbox{=
                  goldmann. +}\pend
           \selectlanguage{ngerman}\endnumbering\briefempfaengerindex{Schnitzler, Arthur@\textsc{Schnitzler, Arthur}!zzzGoldmann, Paul@\emph{von Paul Goldmann}!1902-04-261@{26. 4. 1902}|)be}\mylabel{L02634h}  \normalsize

\doendnotes{C}
\bigskip
\vfill

\clearpage

\footnotesize

\lohead{\textsc{register}}

% Definiere theindex-Environment komplett neu ohne reledmac
\makeatletter
\renewenvironment{theindex}{%
  \section*{\indexname}%
  \setlength{\parindent}{0pt}%
  \setlength{\parskip}{0pt plus 0.3pt}%
  \let\item\@idxitem
}{%
  \clearpage
}
\makeatother

\IfFileExists{\jobname-pw.ind}{\input{\jobname-pw.ind}}{}

\end{document}

      