%% latex-leseansicht-vorspann.tex
%% Vorspann für die Leseansicht.
%% Lädt die gemeinsame Datei latex-vorspann.tex mit nicht gesetztem Schalter.

\newif\ifkorrekturansicht
\korrekturansichtfalse

\input{../tex-inputs/latex-vorspann}


         
         \renewcommand{\erwaehntePersonen}{Personen: Ernest von Gréger-Jurco, Josef Kvasnicka, Karl Strecker}
         \renewcommand{\erwaehnteInstitutionen}{Institutionen: Berliner Tageblatt, Rose-Theater, Tägliche Rundschau}
         \renewcommand{\erwaehnteOrte}{Orte: Berlin, Frankgasse, Wien, Zimmerstraße}
         \renewcommand{\erwaehnteWerke}{Werke: Die Kinder der Armen, Ein litterarisch-dramatisches Hochstapler-Stücklein}
               \section[Paul Goldmann an Arthur Schnitzler, 26. 4. 1902]{ Paul Goldmann an Arthur Schnitzler, 26. 4. 1902}\nopagebreak\mylabel{v}\rehead{ }\begin{ledgroupsized}[t]{13cm}\normalsize\beginnumbering \toendnotes[C]{\smallbreak\pagebreak[2]} \Standort{DLA, A:Schnitzler, HS.NZ85.1.3172.}
\physDesc{Telegramm, 2 Blätter, 2 Seiten
\newline{}maschinell\newline{}Versand: 1) Stempel: »\nobreak{}26. April 1902, Kvasnicka\pwindex{Kvasnicka, Josef @\textsc{Kvasnicka, Josef}, \emph{Telegrafenbeamter}|pw}\nobreak{}«.   2) Stempel: »\nobreak{}12 40\nobreak{}«.  3) mit Bleistift zweites Blatt beschriftet mit: »II. Blatt ad N\textsuperscript{o} 99946 De Berlin\oindex{Berlin@\textbf{Berlin}|pw}« 4) mit Bleistift von unbekannter Hand Vermerk des Postrayons:
                                 »71«}\toendnotes[C]{\smallbreak}\pstart{}{\pb}arthur schnitzler wien\oindex{Wien@\textbf{Wien}|pw}\pend{}\pstart{}frankgasze 1\oindex{Frankgasse@\textbf{Frankgasse}|pw}=\pend{}{\bigskip}\pstart
           \noindent{}\centering{}{\pb}de berlin\oindex{Berlin@\textbf{Berlin}|pw} 99946 196 26/4{ }10 20 m =\pend
           \pstart
           \noindent{}in ›taeglichen rundschau\orgindex{Taegliche Rundschau@Tägliche Rundschau|pw}‹ veroeffentlicht kritiker
                  karl strecker\pwindex{Strecker, Karl 1862-04-08 – 1933-02-19@\textsc{Strecker, Karl} (1862-04-08 – 1933-02-19), \emph{Theaterkritiker}|pw} folgenden artikel\pwindex{Strecker, Karl 1862-04-08 – 1933-02-19@\textsc{Strecker, Karl} (1862-04-08 – 1933-02-19), \emph{Theaterkritiker}!litterarisch-dramatisches Hochstapler-Stuecklein1902-04-26@\strich\emph{Ein litterarisch-dramatisches Hochstapler-Stücklein} {[}1902-04-26{]}|pwv} mit fragenden ueberschrift »ein
               literarisch dramatisches hochstaplerstuecklein«? am donnerstag{ }mittag erhielt
               ich aus wien\oindex{Wien@\textbf{Wien}|pw} ein an meine persoenliche adresze
               gerichtetes telegramm, das also lautete: »frejtag{ }karl
                  wejsz-theater\orgindex{Rose-Theater@Rose-Theater|pw} urpremi{[}ere{]} von ›\label{K_L02634_1v}\edtext{kinder der armen\pwindex{Greger-Jurco, Ernest von *~11.08.1860@\textsc{Gréger-Jurco, Ernest von} (*~11.08.1860), \emph{Schriftsteller}!Kinder der Armen1902-04-25@\strich\emph{Die Kinder der Armen} {[}1902-04-25{]}|pw}}{\lemma{\textnormal{\emph{kinder der armen}}}\Cendnote{\textnormal{der Empfänger duplizierte bei der
                  Transkription: »kinder des kinder der
                  armen«}}}\label{K_L02634_1h}{[}‹{]} empfiehlt genejgter aufmerksamkejt
               ergebenst arthur schnitzler.{[}«{]} von diesem telegramm wuerde ich
               selbstverstaendlich niemals oeffentlich notiz genommen haben, wenn ich
                  annehm{[}en{]} koennte, dasz es wirklich von schnitzler aus {\pb}litterarischem interesze abgesandt worden sej{[}n{]} haette.
               lejder liegt aber fuer mich nach betrachtung dieses ›volksstueckes\pwindex{Greger-Jurco, Ernest von *~11.08.1860@\textsc{Gréger-Jurco, Ernest von} (*~11.08.1860), \emph{Schriftsteller}!Kinder der Armen1902-04-25@\strich\emph{Die Kinder der Armen} {[}1902-04-25{]}|pwv}‹ der handgrejfliche verdacht
               nahe, dasz hier ein arger miszbrauch mit dem namen eines feinfuehligen poeten
               getrieben worden ist. (ein kollege vom »berliner
                  tageblatt\orgindex{Berliner Tageblatt@Berliner Tageblatt|pw}« hat uebrigens genau daszelbe telegramm zur selbigen
                  stunde erhalten). unter diesen umstaenden sehe ich mich genoetigt,
               die offene frage an schnitzler zu richten, ob er diese seltsame aufmunterung wirklich
               abgefaszt hat? wenn nicht (und das nehme ich an), so liegt es ebenso in seinem
               interesze wie in dem der ehre unserer deutschen dramatisch{[}e{]}n
               litteratur, dasz dieser herr verfaszer\pwindex{Greger-Jurco, Ernest von *~11.08.1860@\textsc{Gréger-Jurco, Ernest von} (*~11.08.1860), \emph{Schriftsteller}|pwv}, ernest von jurco\pwindex{Greger-Jurco, Ernest von *~11.08.1860@\textsc{Gréger-Jurco, Ernest von} (*~11.08.1860), \emph{Schriftsteller}|pw} nennt sich
               die kapazitaet\pwindex{Greger-Jurco, Ernest von *~11.08.1860@\textsc{Gréger-Jurco, Ernest von} (*~11.08.1860), \emph{Schriftsteller}|pwv}, entlarvt
                  wird{[}.{]} sowejt artikel. telegraphire dementi an strecker\pwindex{Strecker, Karl 1862-04-08 – 1933-02-19@\textsc{Strecker, Karl} (1862-04-08 – 1933-02-19), \emph{Theaterkritiker}|pw} redaktion taeglichen rundschau\orgindex{Taegliche Rundschau@Tägliche Rundschau|pw}{ }berlin zimmerstrasze 7 und 8\oindex{Zimmerstrasse@\textbf{Zimmerstraße}|pw}. grusz \spacefill\mbox{=
                  goldmann. +}\pend
           
         
         \endnumbering\mylabel{h}\end{ledgroupsized}  \newcommand{\dateiname}{L02634}\newcommand{\titel}{Paul Goldmann an Arthur Schnitzler, 26. 4. 1902}\newcommand{\editorInnen}{Martin Anton Müller und Laura Untner}%% latex-leseansicht-abspann.tex
%% Abspann für die Leseansicht.
%% Der Schalter \ifkorrekturansicht ist bereits durch den Vorspann gesetzt.

%% latex-abspann.tex
%% Gemeinsamer Abspann für Korrekturansicht und Leseansicht.
%% Setzt den Schalter \ifkorrekturansicht voraus (gesetzt in den
%% einbindenden Dateien latex-korrekturansicht-abspann.tex bzw.
%% latex-leseansicht-abspann.tex).
%% ---------------------------------------------------------------

\normalsize

% Das esempio-Environment wird nur in der Leseansicht benötigt
\ifkorrekturansicht\else
\newenvironment{esempio}[3]%
{
    \vspace{1.5ex}
    \rlap{\underline{#1}}
    \par
    \setlength{\parindent}{0cm}
    \nopagebreak
    \leftskip=#2cm
    \rightskip=#3cm
}
{
    \par
}
\fi

\doendnotes{C}
\bigskip
\vfill

\clearpage

\footnotesize

\ifkorrekturansicht
  \lohead{\textsc{register}}
\fi

% theindex-Environment neu definieren ohne reledmac
\makeatletter
\renewenvironment{theindex}{%
  \ifkorrekturansicht
    \section*{\indexname}%
  \else
    \subsubsection*{Index der erwähnten Entitäten}%
  \fi
  \setlength{\parindent}{0pt}%
  \setlength{\parskip}{0pt plus 0.3pt}%
  \let\item\@idxitem
}{%
  \ifkorrekturansicht\clearpage\fi
}
\makeatother

\IfFileExists{\jobname-pw.ind}{\input{\jobname-pw.ind}}{}

% Quellenangabe nur in der Leseansicht
\ifkorrekturansicht\else
% Fallback-Definitionen, falls die .tex-Datei \titel etc. nicht gesetzt hat
\providecommand{\titel}{}
\providecommand{\editorInnen}{}
\providecommand{\dateiname}{\jobname}

\vspace{3cm}

\vfill

\footnotesize
\textsc{Quelle}: \titel. Herausgegeben von {\editorInnen}. In: \emph{Arthur Schnitzler: Briefwechsel mit Autorinnen und Autoren}.
 Digitale Edition, https://schnitzler-briefe.acdh.oeaw.ac.at/{\dateiname}.html (Stand \today)
\fi

\end{document}


      