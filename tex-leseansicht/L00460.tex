%% latex-korrekturansicht-vorspann.tex
%% Vorspann für die Korrekturansicht.
%% Lädt die gemeinsame Datei latex-vorspann.tex mit gesetztem Schalter.

\newif\ifkorrekturansicht
\korrekturansichttrue

\input{../tex-inputs/latex-vorspann}


\section[Richard Beer-Hofmann an Arthur Schnitzler, 9. 7. 1895]{L00460 Richard Beer-Hofmann an Arthur Schnitzler, 9. 7. 1895}
\nopagebreak\mylabel{L00460v}
\rehead{ }\normalsize\beginnumbering\briefempfaengerindex{Schnitzler, Arthur@\textsc{Schnitzler, Arthur}!zzzBeer-Hofmann, Richard@\emph{von Richard Beer-Hofmann}!1895-07-091@{9. 7. 1895}|(be}
\toendnotes[C]{\smallbreak\pagebreak[2]}\Standort{CUL, Schnitzler, B 8.}
\physDesc{Briefkarte, 628 Zeichen
\newline{}Handschrift: Bleistift, lateinische Kurrent
\newline{}Schnitzler: mit Bleistift nummeriert: »63« }
\buchAbdrucke{\weitereDrucke{Arthur Schnitzler, Richard Beer-Hofmann: \emph{Briefwechsel 1891–1931}. Wien, Zürich: \emph{Europaverlag} 1992, S. 78.} }\toendnotes[C]{\smallbreak}
\pstart
           \raggedleft{}{\pb}Ischl\oindex{Bad Ischl@\textbf{Bad Ischl}, \emph{P.PPL}|pw}{ }9/VII 95\pend
           \vspace{0.5em}
\pstart
           Lieber Arthur! Natürlich hab ich Ihnen nicht geschrieben, und ebenso
               natürlich hab ich Gewissensbisse. Blumenthal\pwindex{Blumenthal, Oskar 13.03.1852 – 24.04.1917@\textsc{Blumenthal, Oskar} (13.03.1852 – 24.04.1917), \emph{Schriftsteller/Schriftstellerin, Journalist/Journalistin, Theaterleiter/Theaterleiterin}|pw}
               ist hier – in eigener Villa\oindex{Villa Blumenthal@\textbf{Villa Blumenthal}, \emph{Gebäude (K.GBD)}|pwv}–.
                  Jarno\pwindex{Jarno, Josef 24.08.1865 – 11.01.1932@\textsc{Jarno, Josef} (24.08.1865 – 11.01.1932), \emph{Theaterleiter/Theaterleiterin, Schauspieler/Schauspielerin}|pw} hat heute die Première \uline{seines} Stückes »der
                  Rabenvater\pwindex{Rabenvater. Schwank in drei Akten@\emph{Der Rabenvater. Schwank in drei Akten}|pw}« (noch irgend ein Compagnon\pwindex{Fischer, Hanns Friedrich 26.07.1865 – 21.08.1952@\textsc{Fischer, Hanns Friedrich} (26.07.1865 – 21.08.1952), \emph{Schriftsteller/Schriftstellerin, Theaterleiter/Theaterleiterin, Regisseur/Regisseurin}|pwv} ist dabei). Es lebe der neue Kadelburg\pwindex{Kadelburg, Gustav 26.07.1851 – 11.09.1925@\textsc{Kadelburg, Gustav} (26.07.1851 – 11.09.1925), \emph{Schriftsteller/Schriftstellerin, Schauspieler/Schauspielerin}|pw}!\pend
           
\pstart
           {\pb}Er hatte die ungeheuerliche Idee
                  »Liebelei\pwindex{Liebelei. Schauspiel in drei Akten@\emph{Liebelei. Schauspiel in drei Akten}|pw}« hier aufführen zu wollen. In Berlin\oindex{Berlin@\textbf{Berlin}, \emph{P.PPLC}|pw} soll er darin mitspielen. Nhil\pwindex{Nhil, Robert 18.07.1858 – 31.10.1938@\textsc{Nhil, Robert} (18.07.1858 – 31.10.1938), \emph{Schauspieler/Schauspielerin}|pw} war, – ist möglicherweise noch hier. Der
               kleine Kraus\pwindex{Kraus, Karl 28.04.1874 – 12.06.1936@\textsc{Kraus, Karl} (28.04.1874 – 12.06.1936), \emph{Schriftsteller/Schriftstellerin, Publizist/Publizistin, Schriftsteller/Schriftstellerin}|pw} hat bereits 3 mal mit tiefer
               Herzlichkeit mir die Hand geschüttelt. Es waren i{\geminationm}er
               andere dabei. Er ist köstlich verlegen, nur ich schweige was ihn sehr beunruhigt. Sie
                  ko{\geminationm}en bald?\pend
           \pstart Herzlichst Ihr \spacefill\mbox{R.}\pend{}\selectlanguage{ngerman}\endnumbering\briefempfaengerindex{Schnitzler, Arthur@\textsc{Schnitzler, Arthur}!zzzBeer-Hofmann, Richard@\emph{von Richard Beer-Hofmann}!1895-07-091@{9. 7. 1895}|)be}\mylabel{L00460h}  \normalsize

\doendnotes{C}
\bigskip
\vfill

\clearpage

\footnotesize

\lohead{\textsc{register}}

% Definiere theindex-Environment komplett neu ohne reledmac
\makeatletter
\renewenvironment{theindex}{%
  \section*{\indexname}%
  \setlength{\parindent}{0pt}%
  \setlength{\parskip}{0pt plus 0.3pt}%
  \let\item\@idxitem
}{%
  \clearpage
}
\makeatother

\IfFileExists{\jobname-pw.ind}{\input{\jobname-pw.ind}}{}

\end{document}

      