%% latex-leseansicht-vorspann.tex
%% Vorspann für die Leseansicht.
%% Lädt die gemeinsame Datei latex-vorspann.tex mit nicht gesetztem Schalter.

\newif\ifkorrekturansicht
\korrekturansichtfalse

\input{../tex-inputs/latex-vorspann}


               \section[Richard Beer-Hofmann an Arthur Schnitzler, 9. 7. 1895]{ Richard Beer-Hofmann an Arthur Schnitzler, 9. 7. 1895}\nopagebreak\mylabel{v}\rehead{ }\begin{ledgroupsized}[t]{13cm}\normalsize\beginnumbering\briefempfaengerindex{Schnitzler, Arthur@\textsc{Schnitzler, Arthur}!zzzBeer-Hofmann, Richard@\emph{von Richard Beer-Hofmann}!1895-07-091@{9. 7. 1895}|(be} \toendnotes[C]{\smallbreak\pagebreak[2]} \Standort{CUL, Schnitzler, B 8.}
\physDesc{Briefkarte
\newline{}Handschrift: Bleistift, lateinische Kurrent
\newline{}Schnitzler: mit Bleistift nummeriert: »63« }\buchAbdrucke{\weitereDrucke{Arthur Schnitzler, Richard Beer-Hofmann: \emph{Briefwechsel 1891–1931}. Hg. Konstanze Fliedl. Wien, Zürich: \emph{Europaverlag} 1992, S. 78.} }\toendnotes[C]{\smallbreak}\pstart
           \raggedleft{}{\pb}Ischl\oindex{Bad Ischl@\textbf{Bad Ischl}|pw}{ }9/VII 95\pend
           \pstart
           Lieber Arthur! Natürlich hab ich Ihnen nicht geschrieben, und ebenso
               natürlich hab ich Gewissensbisse. Blumenthal\pwindex{Blumenthal, Oskar 13.03.1852 – 24.04.1917@\textsc{Blumenthal, Oskar} (13.03.1852 – 24.04.1917), \emph{Schriftsteller, Journalist, Theaterleiter}|pw} ist
               hier – in eigener Villa\oindex{Villa Blumenthal@\textbf{Villa Blumenthal}|pwv}–. Jarno\pwindex{Jarno, Josef 24.08.1865 – 11.01.1932@\textsc{Jarno, Josef} (24.08.1865 – 11.01.1932), \emph{Theaterleiter, Schauspieler}|pw} hat heute die
               Première \uline{seines}
                stückes »der Rabenvater\pwindex{Jarno, Josef 24.08.1865 – 11.01.1932@\textsc{Jarno, Josef} (24.08.1865 – 11.01.1932), \emph{Theaterleiter, Schauspieler}!Rabenvater. Schwank in drei Akten9.7.1895 – 9.7.1895@\strich\emph{Der Rabenvater. Schwank in drei Akten} {[}9.7.1895 – 9.7.1895{]}|pw}\pwindex{Fischer, Hanns Friedrich 26.07.1865 – 21.08.1952@\textsc{Fischer, Hanns Friedrich} (26.07.1865 – 21.08.1952), \emph{Schriftsteller, Theaterleiter, Regisseur}!Rabenvater. Schwank in drei Akten9.7.1895 – 9.7.1895@\strich\emph{Der Rabenvater. Schwank in drei Akten} {[}9.7.1895 – 9.7.1895{]}|pw}« (noch irgend ein Compagnon\pwindex{Fischer, Hanns Friedrich 26.07.1865 – 21.08.1952@\textsc{Fischer, Hanns Friedrich} (26.07.1865 – 21.08.1952), \emph{Schriftsteller, Theaterleiter, Regisseur}|pwv} ist dabei). Es lebe der neue Kadelburg\pwindex{Kadelburg, Gustav 26.07.1851 – 11.09.1925@\textsc{Kadelburg, Gustav} (26.07.1851 – 11.09.1925), \emph{Schriftsteller, Schauspieler}|pw}!\pend
           \pstart
           {\pb}Er hatte die ungeheuerliche Idee »Liebelei\pwindex{Schnitzler, Arthur 15.05.1862 – 21.10.1931@\textsc{Schnitzler, Arthur} (15.05.1862 – 21.10.1931), \emph{Schriftsteller, Mediziner}!Liebelei. Schauspiel in drei Akten9. 10. 1895@\strich\emph{Liebelei. Schauspiel in drei Akten} {[}9. 10. 1895{]}|pw}« hier
               aufführen zu wollen. In Berlin\oindex{Berlin@\textbf{Berlin}|pw}
                soll er darin
               mitspielen. Nhil\pwindex{Nhil, Robert 18.07.1858 – 31.10.1938@\textsc{Nhil, Robert} (18.07.1858 – 31.10.1938), \emph{Schauspieler}|pw} war, – ist möglicherweise
               noch hier. Der kleine Kraus\pwindex{Kraus, Karl 28.04.1874 – 12.06.1936@\textsc{Kraus, Karl} (28.04.1874 – 12.06.1936), \emph{Schriftsteller, Publizist}|pw} hat bereits
               3 mal mit tiefer Herzlichkeit mir die Hand geschüttelt. Es waren i{\geminationm}er andere
               dabei. Er ist köstlich verlegen, nur ich schweige was ihn sehr beunruhigt. Sie ko{\geminationm}en bald?\pend
           \pstart Herzlichst Ihr \spacefill\mbox{R.}\pend{}\endnumbering\briefempfaengerindex{Schnitzler, Arthur@\textsc{Schnitzler, Arthur}!zzzBeer-Hofmann, Richard@\emph{von Richard Beer-Hofmann}!1895-07-091@{9. 7. 1895}|)be}\mylabel{h}\end{ledgroupsized}  \newcommand{\dateiname}{L00460}\newcommand{\titel}{Richard Beer-Hofmann an Arthur Schnitzler, 9. 7. 1895}\newcommand{\editorInnen}{Martin Anton Müller und Gerd-Hermann Susen}%% latex-leseansicht-abspann.tex
%% Abspann für die Leseansicht.
%% Der Schalter \ifkorrekturansicht ist bereits durch den Vorspann gesetzt.

%% latex-abspann.tex
%% Gemeinsamer Abspann für Korrekturansicht und Leseansicht.
%% Setzt den Schalter \ifkorrekturansicht voraus (gesetzt in den
%% einbindenden Dateien latex-korrekturansicht-abspann.tex bzw.
%% latex-leseansicht-abspann.tex).
%% ---------------------------------------------------------------

\normalsize

% Das esempio-Environment wird nur in der Leseansicht benötigt
\ifkorrekturansicht\else
\newenvironment{esempio}[3]%
{
    \vspace{1.5ex}
    \rlap{\underline{#1}}
    \par
    \setlength{\parindent}{0cm}
    \nopagebreak
    \leftskip=#2cm
    \rightskip=#3cm
}
{
    \par
}
\fi

\doendnotes{C}
\bigskip
\vfill

\clearpage

\footnotesize

\ifkorrekturansicht
  \lohead{\textsc{register}}
\fi

% theindex-Environment neu definieren ohne reledmac
\makeatletter
\renewenvironment{theindex}{%
  \ifkorrekturansicht
    \section*{\indexname}%
  \else
    \subsubsection*{Index der erwähnten Entitäten}%
  \fi
  \setlength{\parindent}{0pt}%
  \setlength{\parskip}{0pt plus 0.3pt}%
  \let\item\@idxitem
}{%
  \ifkorrekturansicht\clearpage\fi
}
\makeatother

\IfFileExists{\jobname-pw.ind}{\input{\jobname-pw.ind}}{}

% Quellenangabe nur in der Leseansicht
\ifkorrekturansicht\else
% Fallback-Definitionen, falls die .tex-Datei \titel etc. nicht gesetzt hat
\providecommand{\titel}{}
\providecommand{\editorInnen}{}
\providecommand{\dateiname}{\jobname}

\vspace{3cm}

\vfill

\footnotesize
\textsc{Quelle}: \titel. Herausgegeben von {\editorInnen}. In: \emph{Arthur Schnitzler: Briefwechsel mit Autorinnen und Autoren}.
 Digitale Edition, https://schnitzler-briefe.acdh.oeaw.ac.at/{\dateiname}.html (Stand \today)
\fi

\end{document}


      