%% latex-korrekturansicht-vorspann.tex
%% Vorspann für die Korrekturansicht.
%% Lädt die gemeinsame Datei latex-vorspann.tex mit gesetztem Schalter.

\newif\ifkorrekturansicht
\korrekturansichttrue

\input{../tex-inputs/latex-vorspann}


\section[Arthur Schnitzler an Max Burckhard, 14. 1. 1894]{L00290 Arthur Schnitzler an Max Burckhard, 14. 1. 1894}
\nopagebreak\mylabel{L00290v}
\rehead{ }\normalsize\beginnumbering\briefempfaengerindex{Burckhard, Max Eugen@\textsc{Burckhard, Max Eugen}!zzzSchnitzler, Arthur@\emph{von Arthur Schnitzler}!1894-01-141@{14. 1. 1894}|(be}
\toendnotes[C]{\smallbreak\pagebreak[2]}\buchAlsQuelle{\pwindex{Schnitzlers Einzug ins Burgtheater@\emph{Schnitzlers Einzug ins Burgtheater}|pwk}\pwindex{Neue Freie Presse@\emph{Neue Freie Presse}|pwk}\emph{Neue Freie Presse}, Nr. 24162, 19. 12. 1931, S. 14.}
\buchAbdrucke{\weitereDrucke{1) \pwindex{Schnitzlers Einzug ins Burgtheater@\emph{Schnitzlers Einzug ins Burgtheater}|pwk}\emph{Wiener Studien und Dokumente}. Wien: \emph{Steyrermühl} 1933, S. 166–168.} \weitereDrucke{2) Hans-Ulrich Lindken: \emph{Arthur Schnitzler. Aspekte und Akzente. Materialien zu Leben
                        und Werk}. Frankfurt am Main, Bern, Göttingen: \emph{Peter Lang} 1984, S. 243–246.} }\toendnotes[C]{\smallbreak}
\pstart
           \noindent{}{\pb}\so{Schnitzler an Burckhard}, 14. Januar 1894:
               »Sehr verehrter Herr Direktor! Vor etwa drei Vierteljahren habe ich Ihnen durch den
               Verlag \label{K_L00290-1v}\edtext{Entſch\orgindex{A. Entsch@A. Entsch|pw}}{\lemma{\textnormal{\emph{Entſch}}}\Cendnote{\textnormal{Der Verlag \emph{A. Entsch}\orgindex{A. Entsch@A. Entsch|pwk} dürfte den Bühnenvertrieb von \emph{Anatol}\pwindex{Anatol@\emph{Anatol}|pwk} verwaltet haben. Der Bühnendruck erschien bereits Ende
                     1892, vordatiert auf 1893, im \emph{Bibliographischen Bureau}\orgindex{Bibliographisches Bureau@Bibliographisches Bureau|pwk}.}}}\label{K_L00290-1} in Berlin\oindex{Berlin@\textbf{Berlin}, \emph{P.PPLC}|pw} ein Buch\pwindex{Anatol@\emph{Anatol}|pwv} einſenden laſſen, welches unter anderm drei Luſtspiele enthält, die
               ſich vielleicht zur Aufführung eignen. Erlauben Sie mir, ſehr geehrter Herr Direktor,
               Sie jetzt auf dieſelben aufmerkſam zu machen, zu einer Zeit, wo ſowohl die Stimmung
               des Publikums als auch die Geſtaltung des Repertoires Einaktern günſtiger geworden
               ſcheint. Die drei ſehr kurzen Stücke ſind: ›Frage an
                  das Schickſal\pwindex{Frage an das Schicksal@\emph{Die Frage an das Schicksal}|pw}‹, ›Epiſode\pwindex{Episode@\emph{Episode}|pw}‹ und ›Abſchiedsſouper\pwindex{Abschiedssouper@\emph{Abschiedssouper}|pw}‹, von welchen vielleicht das
               dritte in Anbetracht des etwas frivolen Tones auf der Hofbühne\oindex{Burgtheater@\textbf{Burgtheater}, \emph{S.THTR}|pwv} nicht möglich erſcheinen ſollte, ſo dürften ſich
               die zwei erſten um ſo eher für eine ſolche eignen. Ich will über die kleinen
               Stückchen weiter nichts ſagen, möchte Sie, verehrter Herr Direktor, nur bitten, ſie
               gütigſt einmal Ihrer Aufmerkſamkeit zu würdigen. Ich bin mit vorzüglicher Hochachtung
               Ihr ſehr ergebener Dr. Arthur Schnitzler.«\pend
           \selectlanguage{ngerman}\endnumbering\briefempfaengerindex{Burckhard, Max Eugen@\textsc{Burckhard, Max Eugen}!zzzSchnitzler, Arthur@\emph{von Arthur Schnitzler}!1894-01-141@{14. 1. 1894}|)be}\mylabel{L00290h}  \normalsize

\doendnotes{C}
\bigskip
\vfill

\clearpage

\footnotesize

\lohead{\textsc{register}}

% Definiere theindex-Environment komplett neu ohne reledmac
\makeatletter
\renewenvironment{theindex}{%
  \section*{\indexname}%
  \setlength{\parindent}{0pt}%
  \setlength{\parskip}{0pt plus 0.3pt}%
  \let\item\@idxitem
}{%
  \clearpage
}
\makeatother

\IfFileExists{\jobname-pw.ind}{\input{\jobname-pw.ind}}{}

\end{document}

      