%% latex-leseansicht-vorspann.tex
%% Vorspann für die Leseansicht.
%% Lädt die gemeinsame Datei latex-vorspann.tex mit nicht gesetztem Schalter.

\newif\ifkorrekturansicht
\korrekturansichtfalse

\input{../tex-inputs/latex-vorspann}


               \section[Arthur Schnitzler an Richard Beer-Hofmann, 17. 2. 1900]{ Arthur Schnitzler an Richard Beer-Hofmann, 17. 2. 1900}\nopagebreak\mylabel{v}\rehead{ }\begin{ledgroupsized}[t]{13cm}\normalsize\beginnumbering\briefempfaengerindex{Beer-Hofmann, Richard@\textsc{Beer-Hofmann, Richard}!zzzSchnitzler, Arthur@\emph{von Arthur Schnitzler}!1900-02-171@{17. 2. 1900}|(be} \toendnotes[C]{\smallbreak\pagebreak[2]} \Standort{YCGL, MSS 31.}
\physDesc{Brief, 2 Blätter, 5 Seiten, Umschlag
\newline{}Handschrift: 1) schwarze Tinte, deutsche Kurrent (\noindent{}Umschlag)\hspace{1em}2) Bleistift, deutsche Kurrent\hspace{1em}\newline{}Versand: 1) nachgesandt nach »\textsc{poste restante Sanremo\oindex{Sanremo@\textbf{Sanremo}|pw}}«  2) Stempel: »\nobreak{}\oindex{I., Innere Stadt@\textbf{I., Innere Stadt}|pwk}Wien 1, 17. 2. 00, 11–12N\nobreak{}«. 3) Stempel: »\nobreak{}\oindex{Pegli@\textbf{Pegli}|pwk}{\pb}Pegli
                                 (G\textcolor{gray}{eno}va), \textcolor{gray}{19}{[} 2. 1900{]}\nobreak{}«. 4) Stempel: »\nobreak{}\oindex{Sanremo@\textbf{Sanremo}|pwk}\textcolor{gray}{Sanremo} (Porto
                              Maurizio), 20 2 {[}0{]}0, 7M\nobreak{}«. }\buchAbdrucke{\weitereDrucke{Arthur Schnitzler, Richard Beer-Hofmann: \emph{Briefwechsel 1891–1931}. Hg. Konstanze Fliedl. Wien, Zürich: \emph{Europaverlag} 1992, S. 141–142.} }\toendnotes[C]{\smallbreak}\pstart{}{\pb}\textsc{Italia}\oindex{Italien@\textbf{Italien}|pw}\pend{}\pstart{}Herrn \textsc{Dr. Richard Beer-Hofmann}\pend{}\pstart{}\textsc{Pegli} bei \textsc{Genua}\oindex{Pegli@\textbf{Pegli}|pw}\pend{}\pstart{}\textsc{Grand Hotel Mediterranée}\oindex{Grand Hotel Mediterranee@\textbf{Grand Hotel Mediterranée}|pw}\pend{}{\bigskip}\pstart
           \raggedleft{}{\pb}17. 2. 1900.\pend
           \pstart
           Mein lieber Richard, Paul\pwindex{Goldmann, Paul 31.01.1865 – 25.09.1935@\textsc{Goldmann, Paul} (31.01.1865 – 25.09.1935), \emph{Schriftsteller, Journalist}|pw} wohnt Berlin\oindex{Berlin@\textbf{Berlin}|pw}, Hotel Saxonia\oindex{Hotel Saxonia@\textbf{Hotel Saxonia}|pw}, in der Königgrätzer Straße\oindex{Stresemannstrasse@\textbf{Stresemannstraße}|pw}; ſein Onkel heißt Fedor\pwindex{Mamroth, Fedor 21.02.1851 – 25.06.1907@\textsc{Mamroth, Fedor} (21.02.1851 – 25.06.1907), \emph{Journalist, Kritiker}|pw}, und ich komme nicht nach Italien\oindex{Italien@\textbf{Italien}|pw}. Was ich mache? – eine Novelle\pwindex{Schnitzler, Arthur 15.05.1862 – 21.10.1931@\textsc{Schnitzler, Arthur} (15.05.1862 – 21.10.1931), \emph{Schriftsteller, Mediziner}!Frau Bertha Garlan. Roman15.1.1901 – 15.3.1901@\strich\emph{Frau Bertha Garlan. Roman} {[}15.1.1901 – 15.3.1901{]}|pwv} ſchreiben, an der ich zeitweilig Freude habe,
               meinem Ohrenſauſen zuhören und dem was es bedeutet, – mich meiſtens einſam, oder
               beſſer vereinſamt, oder noch beſſer – {\pb}vereinſamend
               fühlen – Ihnen heut eine \textsc{Beatrice}\pwindex{Schnitzler, Arthur 15.05.1862 – 21.10.1931@\textsc{Schnitzler, Arthur} (15.05.1862 – 21.10.1931), \emph{Schriftsteller, Mediziner}!Schleier der Beatrice. Schauspiel in fuenf Akten1900-12-01 – 1900-12-01@\strich\emph{Der Schleier der Beatrice. Schauspiel in fünf Akten} {[}1900-12-01 – 1900-12-01{]}|pw} geſchickt haben – und Sie – ohne Neid – beneiden. –\pend
           \pstart
           Ich möchte aber auch wiſſen, was Sie machen, ob Sie ſich wohl fühlen, ob ſich Ihre
                  Frau\pwindex{Beer-Hofmann, Paula 25.02.1879 – 30.10.1939@\textsc{Beer-Hofmann, Paula} (25.02.1879 – 30.10.1939)|pwv} erholt hat, ob Sie was
               arbeiten, ob Sie Menſchen kennen gelernt haben, ob Sie ſchon eine Nachricht von Hugo\pwindex{Hofmannsthal, Hugo von 01.02.1874 – 15.07.1929@\textsc{Hofmannsthal, Hugo von} (01.02.1874 – 15.07.1929), \emph{Schriftsteller}|pw} haben. –\pend
           \pstart
           Seit Sie und Hugo\pwindex{Hofmannsthal, Hugo von 01.02.1874 – 15.07.1929@\textsc{Hofmannsthal, Hugo von} (01.02.1874 – 15.07.1929), \emph{Schriftsteller}|pw} weg ſind, bin {\pb}ich faſt nie im Club\orgindex{Wiener Schachclub@Wiener Schachclub|pwv}. \textsc{Wasserma{\geminationn}\pwindex{Wassermann, Jakob 10.03.1873 – 01.01.1934@\textsc{Wassermann, Jakob} (10.03.1873 – 01.01.1934), \emph{Schriftsteller}|pw}}, auch \textsc{Leo}\pwindex{Van-Jung, Leo 15.10.1866 – 02.07.1939@\textsc{Van-Jung, Leo} (15.10.1866 – 02.07.1939), \emph{Gesangspädagoge, Mathematiker}|pw} ſind
               beinah allabendlich bei dem aſthmatiſchen Naſchauer\pwindex{Naschauer, Paul 06.09.1866 – 20.05.1900@\textsc{Naschauer, Paul} (06.09.1866 – 20.05.1900)|pw}; ich war \label{K_L01014_1v}\edtext{2mal dort}{\lemma{\textnormal{\emph{2mal dort}}}\Cendnote{\textnormal{siehe A. S.: \emph{Tagebuch}, 4. 2. 1900 und A. S.: \emph{Tagebuch}, 12. 2. 1900}}}\label{K_L01014_1h} und habe bei dieſer Gelegenheit einmal 21,
               einmal Poker mit \textsc{Herzl}\pwindex{Herzl, Theodor 02.05.1860 – 03.07.1904@\textsc{Herzl, Theodor} (02.05.1860 – 03.07.1904), \emph{Schriftsteller, Journalist}|pw}
               und den \textsc{Naſchaueri{\geminationn}en}\pwindex{Herzl, Julie 01.02.1868 – 10.11.1907@\textsc{Herzl, Julie} (01.02.1868 – 10.11.1907)|pw}\pwindex{Czopp, Therese 13.10.1863 – 03.02.1938@\textsc{Czopp, Therese} (13.10.1863 – 03.02.1938)|pw}\pwindex{Naschauer, Ella 06.11.1875 – 17.12.1939@\textsc{Naschauer, Ella} (06.11.1875 – 17.12.1939)|pw}\pwindex{Eisner, Helene 03.07.1865 – 11.03.1937@\textsc{Eisner, Helene} (03.07.1865 – 11.03.1937)|pw} geſpielt. –\pend
           \pstart
           Ein neues Buch\pwindex{Messer, Max 05.07.1875 – 25.12.1930@\textsc{Messer, Max} (05.07.1875 – 25.12.1930), \emph{Schriftsteller, Journalist, Rechtsanwalt}!Wiener Bummelgeschichten1900 – 1900@\strich\emph{Wiener Bummelgeschichten} {[}1900 – 1900{]}|pwv}, von dem
               dampfenden Jüngling \textsc{Messer}\pwindex{Messer, Max 05.07.1875 – 25.12.1930@\textsc{Messer, Max} (05.07.1875 – 25.12.1930), \emph{Schriftsteller, Journalist, Rechtsanwalt}|pw} verfaſſt, werd ich Ihnen ſchicken, damit Ihnen auch in \textsc{Pegli}\oindex{Pegli@\textbf{Pegli}|pw} ein{\pb}mal übel wird. – Der
                  Roman\pwindex{Wolff, Ludwig 07.03.1876 – nach 1958@\textsc{Wolff, Ludwig} (07.03.1876 – nach 1958), \emph{Schriftsteller, Dramaturg}!Im toten Wasser. Ein Wiener Roman1898 – 1898@\strich\emph{Im toten Wasser. Ein Wiener Roman} {[}1898 – 1898{]}|pwv} von Wolff\pwindex{Wolff, Ludwig 07.03.1876 – nach 1958@\textsc{Wolff, Ludwig} (07.03.1876 – nach 1958), \emph{Schriftsteller, Dramaturg}|pw} iſt ſehr anſtändig intentionirt und ohne
               Geſchmackloſigkeiten\pend
           \pstart
           Mit Vergnügen les’ ich die \textsc{Kuh}\pwindex{Kuh, Emil 13.12.1828 – 30.12.1876@\textsc{Kuh, Emil} (13.12.1828 – 30.12.1876)|pw}{ }\textsc{Hebb}\pwindex{Hebbel, Friedrich 18.03.1813 – 13.12.1863@\textsc{Hebbel, Friedrich} (18.03.1813 – 13.12.1863), \emph{Schriftsteller}|pw}{[}el{]} Biographie\pwindex{Kuh, Emil 13.12.1828 – 30.12.1876@\textsc{Kuh, Emil} (13.12.1828 – 30.12.1876)!Biographie Friedrich Hebbels1877 – 1877@\strich\emph{Biographie Friedrich Hebbels} {[}1877 – 1877{]}|pwv}. Den Götterliebling\pwindex{Beer-Hofmann, Richard 11.07.1866 – 26.09.1945@\textsc{Beer-Hofmann, Richard} (11.07.1866 – 26.09.1945), \emph{Schriftsteller}!Tod Georgs1900@\strich\emph{Der Tod Georgs} {[}1900{]}|pw} heb
               ich mir auf einen Frühlingstag auf dem Land auf. Denken Sie, dſs Ihr Buch\pwindex{Beer-Hofmann, Richard 11.07.1866 – 26.09.1945@\textsc{Beer-Hofmann, Richard} (11.07.1866 – 26.09.1945), \emph{Schriftsteller}!Tod Georgs1900@\strich\emph{Der Tod Georgs} {[}1900{]}|pwv} erſt vor 2 Tagen hier in den Buchhdlg
                  angeko{\geminationm}en iſt. Frau Elly Hirſchfeld\pwindex{Petersen, Elly 26.02.1874 – 29.12.1965@\textsc{Petersen, Elly} (26.02.1874 – 29.12.1965), \emph{Schriftstellerin}|pw} – um Ihnen nichts zu verſchweigen – iſt ſchon ganz, beinah
               ganz geſund, und Georg H.\pwindex{Hirschfeld, Georg 11.02.1873 – 17.01.1942@\textsc{Hirschfeld, Georg} (11.02.1873 – 17.01.1942), \emph{Schriftsteller}|pw} iſt mir wieder viel {\pb}ſympathiſcher geworden. Frau Fulda\pwindex{DAlbert, Ida 05.12.1869 – 1926-10-06@\textsc{d’Albert, Ida} (05.12.1869 – 1926-10-06)|pw} iſt ſeit ein paar Tagen in Wien\oindex{Wien@\textbf{Wien}|pw}, \textsc{resp}. Hietzing\oindex{XIII., Hietzing@\textbf{XIII., Hietzing}|pw}. – \textsc{Schlenther}\pwindex{Schlenther, Paul 20.08.1854 – 30.04.1916@\textsc{Schlenther, Paul} (20.08.1854 – 30.04.1916), \emph{Schriftsteller, Kritiker, Theaterleiter}|pw} hat die \textsc{Bea}.\pwindex{Schnitzler, Arthur 15.05.1862 – 21.10.1931@\textsc{Schnitzler, Arthur} (15.05.1862 – 21.10.1931), \emph{Schriftsteller, Mediziner}!Schleier der Beatrice. Schauspiel in fuenf Akten1900-12-01 – 1900-12-01@\strich\emph{Der Schleier der Beatrice. Schauspiel in fünf Akten} {[}1900-12-01 – 1900-12-01{]}|pw} in im ganzen recht vernünftiger Weiſe zuſa{\geminationm}engeſtrichen u. iſt jetzt auch für Kainz\pwindex{Kainz, Josef 02.01.1858 – 20.09.1910@\textsc{Kainz, Josef} (02.01.1858 – 20.09.1910), \emph{Schauspieler}|pw}{ }Dichter\pwindex{Schnitzler, Arthur 15.05.1862 – 21.10.1931@\textsc{Schnitzler, Arthur} (15.05.1862 – 21.10.1931), \emph{Schriftsteller, Mediziner}!Schleier der Beatrice. Schauspiel in fuenf Akten1900-12-01 – 1900-12-01@\strich\emph{Der Schleier der Beatrice. Schauspiel in fünf Akten} {[}1900-12-01 – 1900-12-01{]}|pwv}, Reimers\pwindex{Reimers, Georg 04.04.1860 – 15.04.1936@\textsc{Reimers, Georg} (04.04.1860 – 15.04.1936), \emph{Schauspieler}|pw}{ }Herzog\pwindex{Schnitzler, Arthur 15.05.1862 – 21.10.1931@\textsc{Schnitzler, Arthur} (15.05.1862 – 21.10.1931), \emph{Schriftsteller, Mediziner}!Schleier der Beatrice. Schauspiel in fuenf Akten1900-12-01 – 1900-12-01@\strich\emph{Der Schleier der Beatrice. Schauspiel in fünf Akten} {[}1900-12-01 – 1900-12-01{]}|pwv}. Aber ich bin wieder
               ſchwankend geworden. – Über die \textsc{Beatrice}\pwindex{Schnitzler, Arthur 15.05.1862 – 21.10.1931@\textsc{Schnitzler, Arthur} (15.05.1862 – 21.10.1931), \emph{Schriftsteller, Mediziner}!Schleier der Beatrice. Schauspiel in fuenf Akten1900-12-01 – 1900-12-01@\strich\emph{Der Schleier der Beatrice. Schauspiel in fünf Akten} {[}1900-12-01 – 1900-12-01{]}|pw}{ }ſchreiben Sie mir nichts; vielleicht ſagen Sie mir
               noch einiges, we{\geminationn} Sie wieder zurück ſind. –\pend
           \pstart
           Leben Sie wohl. Von Herzen{\\[\baselineskip]}Ihr \spacefill\mbox{Arthur}\pend
           \leftskip=0em{}          \endnumbering\briefempfaengerindex{Beer-Hofmann, Richard@\textsc{Beer-Hofmann, Richard}!zzzSchnitzler, Arthur@\emph{von Arthur Schnitzler}!1900-02-171@{17. 2. 1900}|)be}\mylabel{h}\end{ledgroupsized}  \newcommand{\dateiname}{L01014}\newcommand{\titel}{Arthur Schnitzler an Richard Beer-Hofmann, 17. 2. 1900}\newcommand{\editorInnen}{Martin Anton Müller und Gerd-Hermann Susen}
            \footnotesize
\begin{ledgroupsized}[t]{11.5cm}
\doendnotes{C}
\end{ledgroupsized}
         %% latex-leseansicht-abspann.tex
%% Abspann für die Leseansicht.
%% Der Schalter \ifkorrekturansicht ist bereits durch den Vorspann gesetzt.

%% latex-abspann.tex
%% Gemeinsamer Abspann für Korrekturansicht und Leseansicht.
%% Setzt den Schalter \ifkorrekturansicht voraus (gesetzt in den
%% einbindenden Dateien latex-korrekturansicht-abspann.tex bzw.
%% latex-leseansicht-abspann.tex).
%% ---------------------------------------------------------------

\normalsize

% Das esempio-Environment wird nur in der Leseansicht benötigt
\ifkorrekturansicht\else
\newenvironment{esempio}[3]%
{
    \vspace{1.5ex}
    \rlap{\underline{#1}}
    \par
    \setlength{\parindent}{0cm}
    \nopagebreak
    \leftskip=#2cm
    \rightskip=#3cm
}
{
    \par
}
\fi

\doendnotes{C}
\bigskip
\vfill

\clearpage

\footnotesize

\ifkorrekturansicht
  \lohead{\textsc{register}}
\fi

% theindex-Environment neu definieren ohne reledmac
\makeatletter
\renewenvironment{theindex}{%
  \ifkorrekturansicht
    \section*{\indexname}%
  \else
    \subsubsection*{Index der erwähnten Entitäten}%
  \fi
  \setlength{\parindent}{0pt}%
  \setlength{\parskip}{0pt plus 0.3pt}%
  \let\item\@idxitem
}{%
  \ifkorrekturansicht\clearpage\fi
}
\makeatother

\IfFileExists{\jobname-pw.ind}{\input{\jobname-pw.ind}}{}

% Quellenangabe nur in der Leseansicht
\ifkorrekturansicht\else
% Fallback-Definitionen, falls die .tex-Datei \titel etc. nicht gesetzt hat
\providecommand{\titel}{}
\providecommand{\editorInnen}{}
\providecommand{\dateiname}{\jobname}

\vspace{3cm}

\vfill

\footnotesize
\textsc{Quelle}: \titel. Herausgegeben von {\editorInnen}. In: \emph{Arthur Schnitzler: Briefwechsel mit Autorinnen und Autoren}.
 Digitale Edition, https://schnitzler-briefe.acdh.oeaw.ac.at/{\dateiname}.html (Stand \today)
\fi

\end{document}


      