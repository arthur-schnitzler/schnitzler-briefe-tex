%% latex-leseansicht-vorspann.tex
%% Vorspann für die Leseansicht.
%% Lädt die gemeinsame Datei latex-vorspann.tex mit nicht gesetztem Schalter.

\newif\ifkorrekturansicht
\korrekturansichtfalse

\input{../tex-inputs/latex-vorspann}


\section[Arthur Schnitzler an Richard Beer-Hofmann, 17. 2. 1900]{L01014 Arthur Schnitzler an Richard Beer-Hofmann, 17. 2. 1900}
\nopagebreak\mylabel{L01014v}
\rehead{ }\normalsize\beginnumbering\briefempfaengerindex{Beer-Hofmann, Richard@\textsc{Beer-Hofmann, Richard}!zzzSchnitzler, Arthur@\emph{von Arthur Schnitzler}!1900-02-171@{17. 2. 1900}|(be}
\toendnotes[C]{\smallbreak\pagebreak[2]}
\correspDesc{Versand  durch Arthur Schnitzler am 17. 2. 1900 in Wien
\newline{}Weiterleitung  am 19. 2. 1900 in Pegli
\newline{}Erhalt  durch Richard Beer-Hofmann am 20. 2. 1900 in Sanremo}\toendnotes[C]{\smallbreak}
\Standort{YCGL, MSS 31.}
\physDesc{Brief, 2 Blätter, 5 Seiten, Kuvert, 1857 Zeichen
\newline{}Handschrift: 1) schwarze Tinte, deutsche Kurrent (\noindent{}Umschlag)\hspace{1em}2) Bleistift, deutsche Kurrent\hspace{1em}
\newline{}Versand: 1) nachgesandt nach »\textsc{poste restante Sanremo\oindex{Sanremo@\textbf{Sanremo}, \emph{Hauptstadt}|pw}}«   2) Stempel: »\nobreak{}\oindex{I., Innere Stadt@\textbf{I., Innere Stadt}, \emph{Verwaltungsgebiet}|pwk}Wien 1, 17. 2. 00, 11–12N\nobreak{}«.  3) Stempel: »\nobreak{}\oindex{Pegli@\textbf{Pegli}, \emph{Ehemaliger Ort}|pwk}{\pb}Pegli
                                          (G\textcolor{gray}{eno}va), \textcolor{gray}{19}{[} 2. 1900{]}\nobreak{}«.  4) Stempel: »\nobreak{}\oindex{Sanremo@\textbf{Sanremo}, \emph{Hauptstadt}|pwk}\textcolor{gray}{Sanremo} (Porto Maurizio), 20 2 {[}0{]}0, 7M\nobreak{}«. }
\buchAbdrucke{\weitereDrucke{Arthur Schnitzler, Richard Beer-Hofmann: \emph{Briefwechsel 1891–1931}. Herausgegeben von Konstanze Fliedl. Wien, Zürich: \emph{Europaverlag} 1992, S. 141–142.} }\toendnotes[C]{\smallbreak}\pstart{}{\pb}\textsc{Italia}\oindex{Italien@\textbf{Italien}|pw}\pend{}\pstart{}Herrn \textsc{Dr. Richard Beer-Hofmann}\pend{}\pstart{}\textsc{Pegli} bei \textsc{Genua}\oindex{Pegli@\textbf{Pegli}, \emph{Ehemaliger Ort}|pw}\pend{}\pstart{}\textsc{Grand Hotel Mediterranée}\oindex{Grand Hotel Mediterranée@\textbf{Grand Hotel Mediterranée}, \emph{Hotel}|pw}\pend{}{\bigskip}\vspace{1em}
\pstart
           \raggedleft{}{\pb}17. 2. 1900.\pend
           \vspace{0.5em}
\pstart
           Mein lieber Richard,{ }Paul\pwindex{Goldmann, Paul 31.\,1.\,1865 Breslau – 25.\,9.\,1935 Wien@\textsc{Goldmann, Paul} (31.\,1.\,1865 Breslau – 25.\,9.\,1935 Wien), \emph{Schriftsteller, Journalist}|pw} wohnt Berlin\oindex{Berlin@\textbf{Berlin}, \emph{Hauptstadt}|pw}, Hotel Saxonia\oindex{Hotel Saxonia@\textbf{Hotel Saxonia}, \emph{Hotel}|pw}, in der Königgrätzer Straße\oindex{Stresemannstraße@\textbf{Stresemannstraße}, \emph{Straße}|pw};{ }ſein Onkel heißt Fedor\pwindex{Mamroth, Fedor 21.\,2.\,1851 Breslau – 25.\,6.\,1907 Frankfurt am Main@\textsc{Mamroth, Fedor} (21.\,2.\,1851 Breslau – 25.\,6.\,1907 Frankfurt am Main), \emph{Journalist, Kritiker}|pw}, und ich komme nicht nach Italien\oindex{Italien@\textbf{Italien}|pw}. Was ich mache? – eine Novelle\pwindex{Schnitzler, Arthur 15.\,5.\,1862 Wien – 21.\,10.\,1931 ebd.@\textsc{Schnitzler, Arthur} (15.\,5.\,1862 Wien – 21.\,10.\,1931 ebd.), \emph{Schriftsteller, Mediziner}!Frau Bertha Garlan. Roman@\strich\emph{Frau Bertha Garlan. Roman}|pwv}{ }ſchreiben, an der ich zeitweilig
               Freude habe, meinem Ohrenſauſen zuhören und dem was es bedeutet, – mich meiſtens
               einſam, oder beſſer vereinſamt, oder noch beſſer – {\pb}vereinſamend fühlen – Ihnen heut eine \textsc{Beatrice}\pwindex{Schnitzler, Arthur 15.\,5.\,1862 Wien – 21.\,10.\,1931 ebd.@\textsc{Schnitzler, Arthur} (15.\,5.\,1862 Wien – 21.\,10.\,1931 ebd.), \emph{Schriftsteller, Mediziner}!Schleier der Beatrice. Schauspiel in fünf Akten@\strich\emph{Der Schleier der Beatrice. Schauspiel in fünf Akten}|pw} geſchickt haben – und Sie – ohne Neid – beneiden. –\pend
           
\pstart
           Ich möchte aber auch wiſſen, was Sie machen, ob Sie{ }ſich wohl fühlen, ob{ }ſich Ihre
                  Frau\pwindex{Beer-Hofmann, Paula 25.\,2.\,1879 Wien – 30.\,10.\,1939 Zürich@\textsc{Beer-Hofmann, Paula} (25.\,2.\,1879 Wien – 30.\,10.\,1939 Zürich)|pwv} erholt hat, ob Sie
               was arbeiten, ob Sie Menſchen kennen gelernt haben, ob Sie{ }ſchon eine Nachricht von
                  Hugo\pwindex{Hofmannsthal, Hugo von 1.\,2.\,1874 Wien – 15.\,7.\,1929 Rodaun@\textsc{Hofmannsthal, Hugo von} (1.\,2.\,1874 Wien – 15.\,7.\,1929 Rodaun), \emph{Schriftsteller}|pw} haben. –\pend
           
\pstart
           Seit Sie und Hugo\pwindex{Hofmannsthal, Hugo von 1.\,2.\,1874 Wien – 15.\,7.\,1929 Rodaun@\textsc{Hofmannsthal, Hugo von} (1.\,2.\,1874 Wien – 15.\,7.\,1929 Rodaun), \emph{Schriftsteller}|pw} weg{ }ſind, bin {\pb}ich faſt nie im Club\orgindex{Wiener Schachclub@Wiener Schachclub|pwv}. \textsc{Wasserma{\geminationn}\pwindex{Wassermann, Jakob 10.\,3.\,1873 Fürth – 1.\,1.\,1934 Altaussee@\textsc{Wassermann, Jakob} (10.\,3.\,1873 Fürth – 1.\,1.\,1934 Altaussee), \emph{Schriftsteller}|pw}}, auch \textsc{Leo}\pwindex{Van-Jung, Leo 15.\,10.\,1866 Odessa – 2.\,7.\,1939 Riga@\textsc{Van-Jung, Leo} (15.\,10.\,1866 Odessa – 2.\,7.\,1939 Riga), \emph{Gesangspädagoge, Mathematiker}|pw}{ }ſind beinah allabendlich bei dem aſthmatiſchen Naſchauer\pwindex{Naschauer, Paul 6.\,9.\,1866 Baden bei Wien – 20.\,5.\,1900 Wien@\textsc{Naschauer, Paul} (6.\,9.\,1866 Baden bei Wien – 20.\,5.\,1900 Wien)|pw}; ich war \label{K_L01014-1v}\edtext{2mal dort}{\lemma{\textnormal{\emph{2mal dort}}}\Cendnote{\textnormal{Siehe A. S.: \emph{Tagebuch}, 4. 2. 1900 und 12. 2. 1900.
               }}}\label{K_L01014-1} und habe bei dieſer Gelegenheit einmal 21, einmal Poker mit \textsc{Herzl}\pwindex{Herzl, Theodor 2.\,5.\,1860 Budapest – 3.\,7.\,1904 Edlach@\textsc{Herzl, Theodor} (2.\,5.\,1860 Budapest – 3.\,7.\,1904 Edlach), \emph{Schriftsteller, Journalist}|pw} und den \textsc{Naſchaueri{\geminationn}en}\pwindex{Herzl, Julie 1.\,2.\,1868 Budapest – 10.\,11.\,1907 Wien@\textsc{Herzl, Julie} (1.\,2.\,1868 Budapest – 10.\,11.\,1907 Wien)|pw}\pwindex{Czopp, Therese 13.\,10.\,1863 Wien – 3.\,2.\,1938 ebd.@\textsc{Czopp, Therese} (13.\,10.\,1863 Wien – 3.\,2.\,1938 ebd.)|pw}\pwindex{Naschauer, Ella 6.\,11.\,1875 Wien – 17.\,12.\,1939 ebd.@\textsc{Naschauer, Ella} (6.\,11.\,1875 Wien – 17.\,12.\,1939 ebd.)|pw}\pwindex{Eisner, Helene 3.\,7.\,1865 Wien – 11.\,3.\,1937 ebd.@\textsc{Eisner, Helene} (3.\,7.\,1865 Wien – 11.\,3.\,1937 ebd.)|pw} geſpielt. –\pend
           
\pstart
           Ein neues Buch\pwindex{Messer, Max 5.\,7.\,1875 Wien – 25.\,12.\,1930 ebd.@\textsc{Messer, Max} (5.\,7.\,1875 Wien – 25.\,12.\,1930 ebd.), \emph{Schriftsteller, Journalist, Rechtsanwalt}!Wiener Bummelgeschichten@\strich\emph{Wiener Bummelgeschichten}|pwv}, von dem
               dampfenden Jüngling \textsc{Messer}\pwindex{Messer, Max 5.\,7.\,1875 Wien – 25.\,12.\,1930 ebd.@\textsc{Messer, Max} (5.\,7.\,1875 Wien – 25.\,12.\,1930 ebd.), \emph{Schriftsteller, Journalist, Rechtsanwalt}|pw} verfaſſt, werd ich Ihnen{ }ſchicken, damit Ihnen auch in \textsc{Pegli}\oindex{Pegli@\textbf{Pegli}, \emph{Ehemaliger Ort}|pw} ein{\pb}mal übel wird. – Der Roman\pwindex{Wolff, Ludwig 7.\,3.\,1876 Bielsko-Biała – nach 1958 Vereinigte Staaten von Amerika [USA]@\textsc{Wolff, Ludwig} (7.\,3.\,1876 Bielsko-Biała – nach 1958 Vereinigte Staaten von Amerika [USA]), \emph{Schriftsteller, Dramaturg}!Im toten Wasser. Ein Wiener Roman@\strich\emph{Im toten Wasser. Ein Wiener Roman}|pwv} von Wolff\pwindex{Wolff, Ludwig 7.\,3.\,1876 Bielsko-Biała – nach 1958 Vereinigte Staaten von Amerika [USA]@\textsc{Wolff, Ludwig} (7.\,3.\,1876 Bielsko-Biała – nach 1958 Vereinigte Staaten von Amerika [USA]), \emph{Schriftsteller, Dramaturg}|pw} iſt{ }ſehr anſtändig intentionirt und ohne Geſchmackloſigkeiten\pend
           
\pstart
           Mit Vergnügen les’ ich die \textsc{Kuh}\pwindex{Kuh, Emil 13.\,12.\,1828 Wien – 30.\,12.\,1876 Meran@\textsc{Kuh, Emil} (13.\,12.\,1828 Wien – 30.\,12.\,1876 Meran)|pw}{ }\textsc{Hebb}\pwindex{Hebbel, Friedrich 18.\,3.\,1813 Wesselburen – 13.\,12.\,1863 Wien@\textsc{Hebbel, Friedrich} (18.\,3.\,1813 Wesselburen – 13.\,12.\,1863 Wien), \emph{Schriftsteller}|pw}{[}el{]} Biographie\pwindex{Kuh, Emil 13.\,12.\,1828 Wien – 30.\,12.\,1876 Meran@\textsc{Kuh, Emil} (13.\,12.\,1828 Wien – 30.\,12.\,1876 Meran)!Biographie Friedrich Hebbels@\strich\emph{Biographie Friedrich Hebbels}|pwv}. Den Götterliebling\pwindex{Beer-Hofmann, Richard 11.\,7.\,1866 Wien – 26.\,9.\,1945 New York City@\textsc{Beer-Hofmann, Richard} (11.\,7.\,1866 Wien – 26.\,9.\,1945 New York City), \emph{Schriftsteller}!Tod Georgs@\strich\emph{Der Tod Georgs}|pw} heb ich mir auf einen Frühlingstag auf dem Land auf. Denken
               Sie, dſs Ihr Buch\pwindex{Beer-Hofmann, Richard 11.\,7.\,1866 Wien – 26.\,9.\,1945 New York City@\textsc{Beer-Hofmann, Richard} (11.\,7.\,1866 Wien – 26.\,9.\,1945 New York City), \emph{Schriftsteller}!Tod Georgs@\strich\emph{Der Tod Georgs}|pwv} erſt vor 2
               Tagen hier in den Buchhdlg angeko{\geminationm}en iſt. Frau Elly Hirſchfeld\pwindex{Petersen, Elly 26.\,2.\,1874 Berlin – 29.\,12.\,1965 München@\textsc{Petersen, Elly} (26.\,2.\,1874 Berlin – 29.\,12.\,1965 München), \emph{Schriftstellerin}|pw} – um Ihnen nichts zu
               verſchweigen – iſt{ }ſchon ganz, beinah ganz geſund, und Georg H.\pwindex{Hirschfeld, Georg 11.\,2.\,1873 Berlin – 17.\,1.\,1942 München@\textsc{Hirschfeld, Georg} (11.\,2.\,1873 Berlin – 17.\,1.\,1942 München), \emph{Schriftsteller}|pw} iſt mir wieder viel {\pb}ſympathiſcher geworden. Frau Fulda\pwindex{d’Albert, Ida 5.\,12.\,1869 Wien – 6.\,10.\,1926 Berlin@\textsc{d’Albert, Ida} (5.\,12.\,1869 Wien – 6.\,10.\,1926 Berlin), \emph{Schauspielerin}|pw} iſt{ }ſeit
               ein paar Tagen in Wien\oindex{Wien@\textbf{Wien}, \emph{Verwaltungsgebiet}|pw}, \textsc{resp}. Hietzing\oindex{XIII., Hietzing@\textbf{XIII., Hietzing}, \emph{Verwaltungsgebiet}|pw}. – \textsc{Schlenther}\pwindex{Schlenther, Paul 20.\,8.\,1854 Chernyakhovsk – 30.\,4.\,1916 Berlin@\textsc{Schlenther, Paul} (20.\,8.\,1854 Chernyakhovsk – 30.\,4.\,1916 Berlin), \emph{Schriftsteller, Kritiker, Theaterleiter}|pw} hat die \textsc{Bea}.\pwindex{Schnitzler, Arthur 15.\,5.\,1862 Wien – 21.\,10.\,1931 ebd.@\textsc{Schnitzler, Arthur} (15.\,5.\,1862 Wien – 21.\,10.\,1931 ebd.), \emph{Schriftsteller, Mediziner}!Schleier der Beatrice. Schauspiel in fünf Akten@\strich\emph{Der Schleier der Beatrice. Schauspiel in fünf Akten}|pw} in im ganzen recht vernünftiger Weiſe zuſa{\geminationm}engeſtrichen u. iſt jetzt auch für Kainz\pwindex{Kainz, Josef 2.\,1.\,1858 Mosonmagyaróvár – 20.\,9.\,1910 Wien@\textsc{Kainz, Josef} (2.\,1.\,1858 Mosonmagyaróvár – 20.\,9.\,1910 Wien), \emph{Schauspieler}|pw}{ }Dichter\pwindex{Schnitzler, Arthur 15.\,5.\,1862 Wien – 21.\,10.\,1931 ebd.@\textsc{Schnitzler, Arthur} (15.\,5.\,1862 Wien – 21.\,10.\,1931 ebd.), \emph{Schriftsteller, Mediziner}!Schleier der Beatrice. Schauspiel in fünf Akten@\strich\emph{Der Schleier der Beatrice. Schauspiel in fünf Akten}|pwv}, Reimers\pwindex{Reimers, Georg 4.\,4.\,1860 Altona – 15.\,4.\,1936 Wien@\textsc{Reimers, Georg} (4.\,4.\,1860 Altona – 15.\,4.\,1936 Wien), \emph{Schauspieler}|pw}{ }Herzog\pwindex{Schnitzler, Arthur 15.\,5.\,1862 Wien – 21.\,10.\,1931 ebd.@\textsc{Schnitzler, Arthur} (15.\,5.\,1862 Wien – 21.\,10.\,1931 ebd.), \emph{Schriftsteller, Mediziner}!Schleier der Beatrice. Schauspiel in fünf Akten@\strich\emph{Der Schleier der Beatrice. Schauspiel in fünf Akten}|pwv}. Aber ich bin wieder{ }ſchwankend geworden. – Über die \textsc{Beatrice}\pwindex{Schnitzler, Arthur 15.\,5.\,1862 Wien – 21.\,10.\,1931 ebd.@\textsc{Schnitzler, Arthur} (15.\,5.\,1862 Wien – 21.\,10.\,1931 ebd.), \emph{Schriftsteller, Mediziner}!Schleier der Beatrice. Schauspiel in fünf Akten@\strich\emph{Der Schleier der Beatrice. Schauspiel in fünf Akten}|pw}{ }ſchreiben Sie mir nichts; vielleicht{ }ſagen Sie mir
               noch einiges, we{\geminationn} Sie wieder zurück{ }ſind. –\pend
           
\pstart
           Leben Sie wohl. Von Herzen{\\[\baselineskip]}Ihr \spacefill\mbox{Arthur}\pend
           \leftskip=0em{}\selectlanguage{ngerman}\endnumbering\briefempfaengerindex{Beer-Hofmann, Richard@\textsc{Beer-Hofmann, Richard}!zzzSchnitzler, Arthur@\emph{von Arthur Schnitzler}!1900-02-171@{17. 2. 1900}|)be}\mylabel{L01014h}  \newcommand{\dateiname}{L01014}\newcommand{\titel}{Arthur Schnitzler an Richard Beer-Hofmann, 17. 2. 1900}\newcommand{\editorInnen}{Martin Anton Müller und Gerd-Hermann Susen}%% latex-leseansicht-abspann.tex
%% Abspann für die Leseansicht.
%% Der Schalter \ifkorrekturansicht ist bereits durch den Vorspann gesetzt.

%% latex-abspann.tex
%% Gemeinsamer Abspann für Korrekturansicht und Leseansicht.
%% Setzt den Schalter \ifkorrekturansicht voraus (gesetzt in den
%% einbindenden Dateien latex-korrekturansicht-abspann.tex bzw.
%% latex-leseansicht-abspann.tex).
%% ---------------------------------------------------------------

\normalsize

% Das esempio-Environment wird nur in der Leseansicht benötigt
\ifkorrekturansicht\else
\newenvironment{esempio}[3]%
{
    \vspace{1.5ex}
    \rlap{\underline{#1}}
    \par
    \setlength{\parindent}{0cm}
    \nopagebreak
    \leftskip=#2cm
    \rightskip=#3cm
}
{
    \par
}
\fi

\doendnotes{C}
\bigskip
\vfill

\clearpage

\footnotesize

\ifkorrekturansicht
  \lohead{\textsc{register}}
\fi

% theindex-Environment neu definieren ohne reledmac
\makeatletter
\renewenvironment{theindex}{%
  \ifkorrekturansicht
    \section*{\indexname}%
  \else
    \subsubsection*{Index der erwähnten Entitäten}%
  \fi
  \setlength{\parindent}{0pt}%
  \setlength{\parskip}{0pt plus 0.3pt}%
  \let\item\@idxitem
}{%
  \ifkorrekturansicht\clearpage\fi
}
\makeatother

\IfFileExists{\jobname-pw.ind}{\input{\jobname-pw.ind}}{}

% Quellenangabe nur in der Leseansicht
\ifkorrekturansicht\else
% Fallback-Definitionen, falls die .tex-Datei \titel etc. nicht gesetzt hat
\providecommand{\titel}{}
\providecommand{\editorInnen}{}
\providecommand{\dateiname}{\jobname}

\vspace{3cm}

\vfill

\footnotesize
\textsc{Quelle}: \titel. Herausgegeben von {\editorInnen}. In: \emph{Arthur Schnitzler: Briefwechsel mit Autorinnen und Autoren}.
 Digitale Edition, https://schnitzler-briefe.acdh.oeaw.ac.at/{\dateiname}.html (Stand \today)
\fi

\end{document}


