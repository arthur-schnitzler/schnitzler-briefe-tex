%% latex-korrekturansicht-vorspann.tex
%% Vorspann für die Korrekturansicht.
%% Lädt die gemeinsame Datei latex-vorspann.tex mit gesetztem Schalter.

\newif\ifkorrekturansicht
\korrekturansichttrue

\input{../tex-inputs/latex-vorspann}


\section[Arthur Schnitzler an Richard Beer-Hofmann, 17. 2. 1900]{L01014 Arthur Schnitzler an Richard Beer-Hofmann, 17. 2. 1900}
\nopagebreak\mylabel{L01014v}
\rehead{ }\normalsize\beginnumbering\briefempfaengerindex{Beer-Hofmann, Richard@\textsc{Beer-Hofmann, Richard}!zzzSchnitzler, Arthur@\emph{von Arthur Schnitzler}!1900-02-171@{17. 2. 1900}|(be}
\toendnotes[C]{\smallbreak\pagebreak[2]}\Standort{YCGL, MSS 31.}
\physDesc{Brief, 2 Blätter, 5 Seiten, Umschlag, 1857 Zeichen
\newline{}Handschrift: 1) schwarze Tinte, deutsche Kurrent (\noindent{}Umschlag)\hspace{1em}2) Bleistift, deutsche Kurrent\hspace{1em}
\newline{}Versand: 1) nachgesandt nach »\textsc{poste restante Sanremo\oindex{Sanremo@\textbf{Sanremo}, \emph{P.PPLA3}|pw}}«   2) Stempel: »\nobreak{}\oindex{I., Innere Stadt@\textbf{I., Innere Stadt}, \emph{A.ADM3}|pwk}Wien 1, 17. 2. 00, 11–12N\nobreak{}«.  3) Stempel: »\nobreak{}\oindex{Pegli@\textbf{Pegli}, \emph{P.PPLX}|pwk}{\pb}Pegli
                                          (G\textcolor{gray}{eno}va), \textcolor{gray}{19}{[} 2. 1900{]}\nobreak{}«.  4) Stempel: »\nobreak{}\oindex{Sanremo@\textbf{Sanremo}, \emph{P.PPLA3}|pwk}\textcolor{gray}{Sanremo} (Porto Maurizio), 20 2 {[}0{]}0, 7M\nobreak{}«. }
\buchAbdrucke{\weitereDrucke{Arthur Schnitzler, Richard Beer-Hofmann: \emph{Briefwechsel 1891–1931}. Wien, Zürich: \emph{Europaverlag} 1992, S. 141–142.} }\toendnotes[C]{\smallbreak}\pstart{}{\pb}\textsc{Italia}\oindex{Italien@\textbf{Italien}, \emph{A.PCLI}|pw}\pend{}\pstart{}Herrn \textsc{Dr. Richard Beer-Hofmann}\pend{}\pstart{}\textsc{Pegli} bei \textsc{Genua}\oindex{Pegli@\textbf{Pegli}, \emph{P.PPLX}|pw}\pend{}\pstart{}\textsc{Grand Hotel Mediterranée}\oindex{Grand Hotel Mediterranee@\textbf{Grand Hotel Mediterranée}, \emph{Hotel (K.HTL)}|pw}\pend{}{\bigskip}\vspace{1em}
\pstart
           \raggedleft{}{\pb}17. 2. 1900.\pend
           \vspace{0.5em}
\pstart
           Mein lieber Richard,{ }Paul\pwindex{Goldmann, Paul 31.01.1865 – 25.09.1935@\textsc{Goldmann, Paul} (31.01.1865 – 25.09.1935), \emph{Schriftsteller/Schriftstellerin, Journalist/Journalistin}|pw} wohnt Berlin\oindex{Berlin@\textbf{Berlin}, \emph{P.PPLC}|pw}, Hotel Saxonia\oindex{Hotel Saxonia@\textbf{Hotel Saxonia}, \emph{Hotel (K.HTL)}|pw}, in der Königgrätzer Straße\oindex{Stresemannstrasse@\textbf{Stresemannstraße}, \emph{Straße (K.STR)}|pw}; ſein Onkel heißt Fedor\pwindex{Mamroth, Fedor 21.02.1851 – 25.06.1907@\textsc{Mamroth, Fedor} (21.02.1851 – 25.06.1907), \emph{Journalist/Journalistin, Kritiker/Kritikerin}|pw}, und ich komme nicht nach Italien\oindex{Italien@\textbf{Italien}, \emph{A.PCLI}|pw}. Was ich mache? – eine Novelle\pwindex{Frau Bertha Garlan. Roman@\emph{Frau Bertha Garlan. Roman}|pwv} ſchreiben, an der ich zeitweilig
               Freude habe, meinem Ohrenſauſen zuhören und dem was es bedeutet, – mich meiſtens
               einſam, oder beſſer vereinſamt, oder noch beſſer – {\pb}vereinſamend fühlen – Ihnen heut eine \textsc{Beatrice}\pwindex{Schleier der Beatrice. Schauspiel in fuenf Akten@\emph{Der Schleier der Beatrice. Schauspiel in fünf Akten}|pw} geſchickt haben – und Sie – ohne Neid – beneiden. –\pend
           
\pstart
           Ich möchte aber auch wiſſen, was Sie machen, ob Sie ſich wohl fühlen, ob ſich Ihre
                  Frau\pwindex{Beer-Hofmann, Paula 25.02.1879 – 30.10.1939@\textsc{Beer-Hofmann, Paula} (25.02.1879 – 30.10.1939)|pwv} erholt hat, ob Sie
               was arbeiten, ob Sie Menſchen kennen gelernt haben, ob Sie ſchon eine Nachricht von
                  Hugo\pwindex{Hofmannsthal, Hugo von 1874-02-01 – 1929-07-15@\textsc{Hofmannsthal, Hugo von} (1874-02-01 – 1929-07-15), \emph{Schriftsteller/Schriftstellerin}|pw} haben. –\pend
           
\pstart
           Seit Sie und Hugo\pwindex{Hofmannsthal, Hugo von 1874-02-01 – 1929-07-15@\textsc{Hofmannsthal, Hugo von} (1874-02-01 – 1929-07-15), \emph{Schriftsteller/Schriftstellerin}|pw} weg ſind, bin {\pb}ich faſt nie im Club\orgindex{Wiener Schachclub@Wiener Schachclub|pwv}. \textsc{Wasserma{\geminationn}\pwindex{Wassermann, Jakob 10.03.1873 – 01.01.1934@\textsc{Wassermann, Jakob} (10.03.1873 – 01.01.1934), \emph{Schriftsteller/Schriftstellerin}|pw}}, auch \textsc{Leo}\pwindex{Van-Jung, Leo 15.10.1866 – 02.07.1939@\textsc{Van-Jung, Leo} (15.10.1866 – 02.07.1939), \emph{Gesangspädagoge/Gesangspädagogin, Mathematiker/Mathematikerin}|pw} ſind beinah allabendlich bei dem aſthmatiſchen Naſchauer\pwindex{Naschauer, Paul 06.09.1866 – 20.05.1900@\textsc{Naschauer, Paul} (06.09.1866 – 20.05.1900)|pw}; ich war \label{K_L01014-1v}\edtext{2mal dort}{\lemma{\textnormal{\emph{2mal dort}}}\Cendnote{\textnormal{Siehe A. S.: \emph{Tagebuch}, 4. 2. 1900 und 12. 2. 1900.
               }}}\label{K_L01014-1} und habe bei dieſer Gelegenheit einmal 21, einmal Poker mit \textsc{Herzl}\pwindex{Herzl, Theodor 1860-05-02 – 1904-07-03@\textsc{Herzl, Theodor} (1860-05-02 – 1904-07-03), \emph{Schriftsteller/Schriftstellerin, Journalist/Journalistin}|pw} und den \textsc{Naſchaueri{\geminationn}en}\pwindex{Herzl, Julie 01.02.1868 – 10.11.1907@\textsc{Herzl, Julie} (01.02.1868 – 10.11.1907)|pw}\pwindex{Czopp, Therese 13.10.1863 – 03.02.1938@\textsc{Czopp, Therese} (13.10.1863 – 03.02.1938)|pw}\pwindex{Naschauer, Ella 06.11.1875 – 17.12.1939@\textsc{Naschauer, Ella} (06.11.1875 – 17.12.1939)|pw}\pwindex{Eisner, Helene 03.07.1865 – 11.03.1937@\textsc{Eisner, Helene} (03.07.1865 – 11.03.1937)|pw} geſpielt. –\pend
           
\pstart
           Ein neues Buch\pwindex{Wiener Bummelgeschichten@\emph{Wiener Bummelgeschichten}|pwv}, von dem
               dampfenden Jüngling \textsc{Messer}\pwindex{Messer, Max 05.07.1875 – 25.12.1930@\textsc{Messer, Max} (05.07.1875 – 25.12.1930), \emph{Schriftsteller/Schriftstellerin, Journalist/Journalistin, Rechtsanwalt/Rechtsanwältin}|pw} verfaſſt, werd ich Ihnen ſchicken, damit Ihnen auch in \textsc{Pegli}\oindex{Pegli@\textbf{Pegli}, \emph{P.PPLX}|pw} ein{\pb}mal übel wird. – Der Roman\pwindex{Im toten Wasser. Ein Wiener Roman@\emph{Im toten Wasser. Ein Wiener Roman}|pwv} von Wolff\pwindex{Wolff, Ludwig 07.03.1876 – nach 1958@\textsc{Wolff, Ludwig} (07.03.1876 – nach 1958), \emph{Schriftsteller/Schriftstellerin, Dramaturg/Dramaturgin}|pw} iſt ſehr anſtändig intentionirt und ohne Geſchmackloſigkeiten\pend
           
\pstart
           Mit Vergnügen les’ ich die \textsc{Kuh}\pwindex{Kuh, Emil 13.12.1828 – 30.12.1876@\textsc{Kuh, Emil} (13.12.1828 – 30.12.1876)|pw}{ }\textsc{Hebb}\pwindex{Hebbel, Friedrich 18.03.1813 – 13.12.1863@\textsc{Hebbel, Friedrich} (18.03.1813 – 13.12.1863), \emph{Schriftsteller/Schriftstellerin}|pw}{[}el{]} Biographie\pwindex{Biographie Friedrich Hebbels@\emph{Biographie Friedrich Hebbels}|pwv}. Den Götterliebling\pwindex{Tod Georgs@\emph{Der Tod Georgs}|pw} heb ich mir auf einen Frühlingstag auf dem Land auf. Denken
               Sie, dſs Ihr Buch\pwindex{Tod Georgs@\emph{Der Tod Georgs}|pwv} erſt vor 2
               Tagen hier in den Buchhdlg angeko{\geminationm}en iſt. Frau Elly Hirſchfeld\pwindex{Petersen, Elly 26.02.1874 – 29.12.1965@\textsc{Petersen, Elly} (26.02.1874 – 29.12.1965), \emph{Schriftsteller/Schriftstellerin}|pw} – um Ihnen nichts zu
               verſchweigen – iſt ſchon ganz, beinah ganz geſund, und Georg H.\pwindex{Hirschfeld, Georg 11.02.1873 – 17.01.1942@\textsc{Hirschfeld, Georg} (11.02.1873 – 17.01.1942), \emph{Schriftsteller/Schriftstellerin}|pw} iſt mir wieder viel {\pb}ſympathiſcher geworden. Frau Fulda\pwindex{DAlbert, Ida 05.12.1869 – 1926-10-06@\textsc{d’Albert, Ida} (05.12.1869 – 1926-10-06), \emph{Schauspieler/Schauspielerin}|pw} iſt ſeit
               ein paar Tagen in Wien\oindex{Wien@\textbf{Wien}, \emph{A.ADM2}|pw}, \textsc{resp}. Hietzing\oindex{XIII., Hietzing@\textbf{XIII., Hietzing}, \emph{A.ADM3}|pw}. – \textsc{Schlenther}\pwindex{Schlenther, Paul 20.08.1854 – 30.04.1916@\textsc{Schlenther, Paul} (20.08.1854 – 30.04.1916), \emph{Schriftsteller/Schriftstellerin, Kritiker/Kritikerin, Theaterleiter/Theaterleiterin}|pw} hat die \textsc{Bea}.\pwindex{Schleier der Beatrice. Schauspiel in fuenf Akten@\emph{Der Schleier der Beatrice. Schauspiel in fünf Akten}|pw} in im ganzen recht vernünftiger Weiſe zuſa{\geminationm}engeſtrichen u. iſt jetzt auch für Kainz\pwindex{Kainz, Josef 02.01.1858 – 20.09.1910@\textsc{Kainz, Josef} (02.01.1858 – 20.09.1910), \emph{Schauspieler/Schauspielerin}|pw}{ }Dichter\pwindex{Schleier der Beatrice. Schauspiel in fuenf Akten@\emph{Der Schleier der Beatrice. Schauspiel in fünf Akten}|pwv}, Reimers\pwindex{Reimers, Georg 04.04.1860 – 15.04.1936@\textsc{Reimers, Georg} (04.04.1860 – 15.04.1936), \emph{Schauspieler/Schauspielerin}|pw}{ }Herzog\pwindex{Schleier der Beatrice. Schauspiel in fuenf Akten@\emph{Der Schleier der Beatrice. Schauspiel in fünf Akten}|pwv}. Aber ich bin wieder
               ſchwankend geworden. – Über die \textsc{Beatrice}\pwindex{Schleier der Beatrice. Schauspiel in fuenf Akten@\emph{Der Schleier der Beatrice. Schauspiel in fünf Akten}|pw}{ }ſchreiben Sie mir nichts; vielleicht ſagen Sie mir
               noch einiges, we{\geminationn} Sie wieder zurück ſind. –\pend
           
\pstart
           Leben Sie wohl. Von Herzen{\\[\baselineskip]}Ihr \spacefill\mbox{Arthur}\pend
           \leftskip=0em{}\selectlanguage{ngerman}\endnumbering\briefempfaengerindex{Beer-Hofmann, Richard@\textsc{Beer-Hofmann, Richard}!zzzSchnitzler, Arthur@\emph{von Arthur Schnitzler}!1900-02-171@{17. 2. 1900}|)be}\mylabel{L01014h}  \normalsize

\doendnotes{C}
\bigskip
\vfill

\clearpage

\footnotesize

\lohead{\textsc{register}}

% Definiere theindex-Environment komplett neu ohne reledmac
\makeatletter
\renewenvironment{theindex}{%
  \section*{\indexname}%
  \setlength{\parindent}{0pt}%
  \setlength{\parskip}{0pt plus 0.3pt}%
  \let\item\@idxitem
}{%
  \clearpage
}
\makeatother

\IfFileExists{\jobname-pw.ind}{\input{\jobname-pw.ind}}{}

\end{document}

      