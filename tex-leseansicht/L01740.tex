%% latex-leseansicht-vorspann.tex
%% Vorspann für die Leseansicht.
%% Lädt die gemeinsame Datei latex-vorspann.tex mit nicht gesetztem Schalter.

\newif\ifkorrekturansicht
\korrekturansichtfalse

\input{../tex-inputs/latex-vorspann}


         
         \renewcommand{\erwaehntePersonen}{Personen: Franz Blei}
         \renewcommand{\erwaehnteInstitutionen}{Institutionen: Hyperion}
         \renewcommand{\erwaehnteOrte}{Orte: Hubertusstraße, München, Wien}
         \renewcommand{\erwaehnteWerke}{}
               \section[Franz Blei an Arthur Schnitzler, 12. 12. 1907]{ Franz Blei an Arthur Schnitzler, 12. 12. 1907}\nopagebreak\mylabel{v}\rehead{ }\begin{ledgroupsized}[t]{13cm}\normalsize\beginnumbering\briefempfaengerindex{Schnitzler, Arthur@\textsc{Schnitzler, Arthur}!zzzBlei, Franz@\emph{von Franz Blei}!1907-12-121@{12. 12. 1907}|(be} \toendnotes[C]{\smallbreak\pagebreak[2]} \Standort{CUL, Schnitzler, B 14.}
\physDesc{Brief, 1 Blatt, 1 Seite, 428 Zeichen
\newline{}Handschrift: schwarze Tinte, lateinische Kurrent
\newline{}Schnitzler: 1) mit Bleistift beschriftet: »\textsc{Blei}«  2) mit rotem Buntstift eine Unterstreichung
\newline{}Ordnung: 1) mit Bleistift von unbekannter Hand nummeriert »\strikeout{3}«  2) mit Bleistift von unbekannter Hand nummeriert
                                 »5«}\toendnotes[C]{\smallbreak}\pstart
           \noindent{}{\pb}Hubertusstrasse{ }13\hspace*{1.5em}München\oindex{Hubertusstrasse@\textbf{Hubertusstraße}|pw}\pend
           \pstart{}Sehr geehrter Herr Doctor,\pend\pstart
           ich möchte Sie um Ihre Beiträge bitten für die Zweimonatschrift »\label{K_L01740-1v}\edtext{Das Goldene Vlies\orgindex{Hyperion@Hyperion|pw}}{\lemma{\textnormal{\emph{Das Goldene Vlies}}}\Cendnote{\textnormal{als \emph{Hyperion}\orgindex{Hyperion@Hyperion|pwk} verwirklicht}}}\label{K_L01740-1h}«, die ich 1908 herausgebe. Sie
               werden unsere guten Dichter und Zeichner darin finden. Der Verlag zahlt für die
               Druckseite 15 Kronen. Alles von Ihnen soll sehr willko{\geminationm}en sein. Die Zeitschrift wird, ich muss es hinzufügen, öffentlich erscheinen.\pend
           \pstart
           Es begrüsst Sie{\\[\baselineskip]}Ihr ergebenster{\\[\baselineskip]}\spacefill\mbox{Franz Blei}\pend
           \leftskip=0em{}\pstart
           12. 12. 19\textcolor{gray}{07}\pend
           
         
         \endnumbering\mylabel{h}\end{ledgroupsized}  \newcommand{\dateiname}{L01740}\newcommand{\titel}{Franz Blei an Arthur Schnitzler, 12. 12. 1907}\newcommand{\editorInnen}{Martin Anton Müller und Gerd-Hermann Susen}%% latex-leseansicht-abspann.tex
%% Abspann für die Leseansicht.
%% Der Schalter \ifkorrekturansicht ist bereits durch den Vorspann gesetzt.

%% latex-abspann.tex
%% Gemeinsamer Abspann für Korrekturansicht und Leseansicht.
%% Setzt den Schalter \ifkorrekturansicht voraus (gesetzt in den
%% einbindenden Dateien latex-korrekturansicht-abspann.tex bzw.
%% latex-leseansicht-abspann.tex).
%% ---------------------------------------------------------------

\normalsize

% Das esempio-Environment wird nur in der Leseansicht benötigt
\ifkorrekturansicht\else
\newenvironment{esempio}[3]%
{
    \vspace{1.5ex}
    \rlap{\underline{#1}}
    \par
    \setlength{\parindent}{0cm}
    \nopagebreak
    \leftskip=#2cm
    \rightskip=#3cm
}
{
    \par
}
\fi

\doendnotes{C}
\bigskip
\vfill

\clearpage

\footnotesize

\ifkorrekturansicht
  \lohead{\textsc{register}}
\fi

% theindex-Environment neu definieren ohne reledmac
\makeatletter
\renewenvironment{theindex}{%
  \ifkorrekturansicht
    \section*{\indexname}%
  \else
    \subsubsection*{Index der erwähnten Entitäten}%
  \fi
  \setlength{\parindent}{0pt}%
  \setlength{\parskip}{0pt plus 0.3pt}%
  \let\item\@idxitem
}{%
  \ifkorrekturansicht\clearpage\fi
}
\makeatother

\IfFileExists{\jobname-pw.ind}{\input{\jobname-pw.ind}}{}

% Quellenangabe nur in der Leseansicht
\ifkorrekturansicht\else
% Fallback-Definitionen, falls die .tex-Datei \titel etc. nicht gesetzt hat
\providecommand{\titel}{}
\providecommand{\editorInnen}{}
\providecommand{\dateiname}{\jobname}

\vspace{3cm}

\vfill

\footnotesize
\textsc{Quelle}: \titel. Herausgegeben von {\editorInnen}. In: \emph{Arthur Schnitzler: Briefwechsel mit Autorinnen und Autoren}.
 Digitale Edition, https://schnitzler-briefe.acdh.oeaw.ac.at/{\dateiname}.html (Stand \today)
\fi

\end{document}


      