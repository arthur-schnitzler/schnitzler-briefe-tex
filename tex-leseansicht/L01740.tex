%% latex-korrekturansicht-vorspann.tex
%% Vorspann für die Korrekturansicht.
%% Lädt die gemeinsame Datei latex-vorspann.tex mit gesetztem Schalter.

\newif\ifkorrekturansicht
\korrekturansichttrue

\input{../tex-inputs/latex-vorspann}


\section[Franz Blei an Arthur Schnitzler, 12. 12. 1907]{L01740 Franz Blei an Arthur Schnitzler, 12. 12. 1907}
\nopagebreak\mylabel{L01740v}
\rehead{ }\normalsize\beginnumbering\briefempfaengerindex{Schnitzler, Arthur@\textsc{Schnitzler, Arthur}!zzzBlei, Franz@\emph{von Franz Blei}!1907-12-121@{12. 12. 1907}|(be}
\toendnotes[C]{\smallbreak\pagebreak[2]}\Standort{CUL, Schnitzler, B 14.}
\physDesc{Brief, 1 Blatt, 1 Seite, 428 Zeichen
\newline{}Handschrift: schwarze Tinte, lateinische Kurrent
\newline{}Schnitzler: 1) mit Bleistift beschriftet: »\textsc{Blei}«  2) mit rotem Buntstift eine Unterstreichung
\newline{}Ordnung: 1) mit Bleistift von unbekannter Hand nummeriert »\strikeout{3}«  2) mit Bleistift von unbekannter Hand nummeriert
                                 »5«}\toendnotes[C]{\smallbreak}
\pstart
           {\pb}Hubertusstrasse{ }13\hspace*{1.5em}München\oindex{Hubertusstrasse@\textbf{Hubertusstraße}, \emph{Straße (K.STR)}|pw}\pend
           
\pstart{}Sehr geehrter Herr Doctor,\pend\vspace{0.5em}
\pstart
           ich möchte Sie um Ihre Beiträge bitten für die Zweimonatschrift »\label{K_L01740-1v}\edtext{Das Goldene Vlies\orgindex{Hyperion@Hyperion|pw}}{\lemma{\textnormal{\emph{Das Goldene Vlies}}}\Cendnote{\textnormal{als \emph{Hyperion}\orgindex{Hyperion@Hyperion|pwk} verwirklicht}}}\label{K_L01740-1}«, die ich 1908 herausgebe. Sie
               werden unsere guten Dichter und Zeichner darin finden. Der Verlag zahlt für die
               Druckseite 15 Kronen. Alles von Ihnen soll sehr willko{\geminationm}en sein. Die Zeitschrift wird, ich muss es hinzufügen, öffentlich erscheinen.\pend
           
\pstart
           Es begrüsst Sie{\\[\baselineskip]}Ihr ergebenster{\\[\baselineskip]}\spacefill\mbox{Franz Blei}\pend
           \leftskip=0em{}
\pstart
           12. 12. 19\textcolor{gray}{07}\pend
           \selectlanguage{ngerman}\endnumbering\briefempfaengerindex{Schnitzler, Arthur@\textsc{Schnitzler, Arthur}!zzzBlei, Franz@\emph{von Franz Blei}!1907-12-121@{12. 12. 1907}|)be}\mylabel{L01740h}  \normalsize

\doendnotes{C}
\bigskip
\vfill

\clearpage

\footnotesize

\lohead{\textsc{register}}

% Definiere theindex-Environment komplett neu ohne reledmac
\makeatletter
\renewenvironment{theindex}{%
  \section*{\indexname}%
  \setlength{\parindent}{0pt}%
  \setlength{\parskip}{0pt plus 0.3pt}%
  \let\item\@idxitem
}{%
  \clearpage
}
\makeatother

\IfFileExists{\jobname-pw.ind}{\input{\jobname-pw.ind}}{}

\end{document}

      