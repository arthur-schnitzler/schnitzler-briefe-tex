\input{../tex-inputs/latex-pdf-vorspann}
\begin{center}
            \textcolor{red}{ENTWURF. ENTZIFFERUNG NOCH NICHT KORREKTURGELESEN}
                      \end{center}
            
               \section[Franz Blei an Arthur Schnitzler, 12. 12. 1907]{ Franz Blei an Arthur Schnitzler, 12. 12. 1907}\nopagebreak\mylabel{v}\rehead{ }\begin{ledgroupsized}[t]{13cm}\normalsize\beginnumbering\briefempfaengerindex{Schnitzler, Arthur@\textsc{Schnitzler, Arthur}!zzzBlei, Franz@\emph{von Franz Blei}!1907-12-121@{12. 12. 1907}|(be} \toendnotes[C]{\smallbreak\pagebreak[2]} \Standort{CUL, Schnitzler, B 14.}
\physDesc{Brief, 1 Blatt, 1 Seite
\newline{}Handschrift: schwarze Tinte, lateinische Kurrent
\newline{}Schnitzler: 1) mit Bleistift beschriftet: »\textsc{Blei}« 2) mit rotem Buntstift eine Unterstreichung\newline{}Ordnung: 1) mit Bleistift von unbekannter Hand nummeriert »\strikeout{3}« 2) mit Bleistift von unbekannter Hand nummeriert »5«}\toendnotes[C]{\smallbreak}\pstart
           \noindent{}{\pb}Hubertusstrasse{ }13\hspace*{1.5em}München\oindex{Hubertusstrasse@\textbf{Hubertusstraße}|pw}\pend
           \pstart{}Sehr geehrter Herr Doctor,\pend\pstart
           ich möchte Sie um Ihre Beiträge bitten für die Zweimonatschrift »\label{K_L01740_1v}\edtext{Das Goldene Vlies\orgindex{Hyperion@Hyperion|pw}}{\lemma{\textnormal{\emph{Das Goldene Vlies}}}\Cendnote{\textnormal{als \emph{Hyperion}\orgindex{Hyperion@Hyperion|pwk} verwirklicht}}}\label{K_L01740_1h}«, die ich 1908
                    herausgebe. Sie werden unsere guten Dichter und Zeichner darin finden. Der
                    Verlag zahlt für die Druckseite 15 Kronen. Alles von Ihnen soll sehr willko{\geminationm}en sein. Die Zeitschrift wird, ich muss es
                    hinzufügen, öffentlich erscheinen.\pend
           \pstart
           Es begrüsst Sie{\\[\baselineskip]}Ihr ergebenster{\\[\baselineskip]}\spacefill\mbox{Franz Blei}\pend
           \leftskip=0em{}\pstart
           12. 12. 19\textcolor{gray}{07}\pend
           \endnumbering\briefempfaengerindex{Schnitzler, Arthur@\textsc{Schnitzler, Arthur}!zzzBlei, Franz@\emph{von Franz Blei}!1907-12-121@{12. 12. 1907}|)be}\mylabel{h}\end{ledgroupsized}  \newcommand{\dateiname}{L01740}\newcommand{\titel}{Franz Blei an Arthur Schnitzler, 12. 12. 1907}\newcommand{\editorInnen}{Martin Anton Müller und Gerd-Hermann Susen}\input{../tex-inputs/latex-pdf-abspann}
      