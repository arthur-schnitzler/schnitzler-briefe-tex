%% latex-leseansicht-vorspann.tex
%% Vorspann für die Leseansicht.
%% Lädt die gemeinsame Datei latex-vorspann.tex mit nicht gesetztem Schalter.

\newif\ifkorrekturansicht
\korrekturansichtfalse

\input{../tex-inputs/latex-vorspann}


         
         \renewcommand{\erwaehntePersonen}{Personen: Gerhart Hauptmann, Alfred Kerr}
         \renewcommand{\erwaehnteOrte}{Orte: Berlin, Frankfurt am Main, Reuterweg, Wien}
         \renewcommand{\erwaehnteWerke}{Werke: Der Biberpelz. Eine Diebskomödie, Die Weber, Die versunkene Glocke, Fuhrmann Henschel, Hanneles Himmelfahrt. Traumdichtung in zwei Teilen, Michael Kramer. Drama, Neue Freie Presse, Schluck und Jau, »Michael Kramer.«}
               \section[ Paul Goldmann an Arthur Schnitzler, 31. 12. {[}1900{]}]{ Paul Goldmann an Arthur Schnitzler, 31. 12. {[}1900{]}}\nopagebreak\mylabel{v}\rehead{ }\begin{ledgroupsized}[t]{13cm}\normalsize\beginnumbering \toendnotes[C]{\smallbreak\pagebreak[2]} \Standort{DLA, A:Schnitzler, HS.NZ85.1.3170.}
\physDesc{Brief, 1 Blatt, 4 Seiten
\newline{}Handschrift: 1) blaue Tinte, deutsche Kurrent\hspace{1em}2) schwarze Tinte, deutsche Kurrent (\noindent{}sechs Zeilen auf der ersten Seite)\hspace{1em}
\newline{}Schnitzler: 1) mit Bleistift das Jahr »{[}1{]}900« vermerkt  2) mit rotem Buntstift zwei Unterstreichungen}\toendnotes[C]{\smallbreak}\pstart
           \noindent{}{\pb}Frankfurt\oindex{Frankfurt am Main@\textbf{Frankfurt am Main}|pw}, 31. December.\hfill \textcolor{gray}{\textbf{Reuterweg 59\oindex{Reuterweg@\textbf{Reuterweg}|pw}.}}\pend
           \pstart
           \centering{}Mein lieber Freund,\pend
           \pstart
           \noindent{}Ich danke Dir für Deine eingehende Erörterung meines \label{K_L02947-1v}\edtext{Feuilleton\pwindex{Goldmann, Paul 31.01.1865 – 25.09.1935@\textsc{Goldmann, Paul} (31.01.1865 – 25.09.1935), \emph{Schriftsteller, Journalist}!Michael Kramer.«1900-12-28@\strich\emph{»Michael Kramer.«} {[}1900-12-28{]}|pwv}s}{\lemma{\textnormal{\emph{Feuilletons}}}\Cendnote{\textnormal{Paul Goldmann\pwindex{Goldmann, Paul 31.01.1865 – 25.09.1935@\textsc{Goldmann, Paul} (31.01.1865 – 25.09.1935), \emph{Schriftsteller, Journalist}|pwk}: \emph{»Michael Kramer.«}\pwindex{Goldmann, Paul 31.01.1865 – 25.09.1935@\textsc{Goldmann, Paul} (31.01.1865 – 25.09.1935), \emph{Schriftsteller, Journalist}!Michael Kramer.«1900-12-28@\strich\emph{»Michael Kramer.«} {[}1900-12-28{]}|pwk}. In: \emph{Neue Freie Presse}\pwindex{Neue Freie Presse1864 – 1939@\emph{Neue Freie Presse} {[}1864 – 1939{]}|pwk}, Nr. 13055, 28. 12. 1900, Morgenblatt, S. 1–3.}}}\label{K_L02947-1h}, finde aber, daß
               ich abſolut Recht habe und würde ſelbſt jetzt, wo ich weiß, daß Dir gewiſſe Bemerkungen\pwindex{Goldmann, Paul 31.01.1865 – 25.09.1935@\textsc{Goldmann, Paul} (31.01.1865 – 25.09.1935), \emph{Schriftsteller, Journalist}!Michael Kramer.«1900-12-28@\strich\emph{»Michael Kramer.«} {[}1900-12-28{]}|pwv} unangebracht
               erſcheinen, dieſe Bemerkungen\pwindex{Goldmann, Paul 31.01.1865 – 25.09.1935@\textsc{Goldmann, Paul} (31.01.1865 – 25.09.1935), \emph{Schriftsteller, Journalist}!Michael Kramer.«1900-12-28@\strich\emph{»Michael Kramer.«} {[}1900-12-28{]}|pwv}
               nochmals mit ruhigem Gewiſſen niederſchreiben. Ich habe die Kritik\pwindex{Goldmann, Paul 31.01.1865 – 25.09.1935@\textsc{Goldmann, Paul} (31.01.1865 – 25.09.1935), \emph{Schriftsteller, Journalist}!Michael Kramer.«1900-12-28@\strich\emph{»Michael Kramer.«} {[}1900-12-28{]}|pwv} im hellen Zorn verfaßt, im Zorn
               nicht nur gegen die Kritikloſigkeit der \textsc{Hauptmann\pwindex{Hauptmann, Gerhart 15.11.1862 – 06.06.1946@\textsc{Hauptmann, Gerhart} (15.11.1862 – 06.06.1946), \emph{Schriftsteller}|pw}}-Anhänger (unter denen ſich unſer Freund \textsc{Kerr\pwindex{Kerr, Alfred 25.12.1867 – 12.10.1948@\textsc{Kerr, Alfred} (25.12.1867 – 12.10.1948), \emph{Schriftsteller, Kritiker}|pw}} beſonders hevorgethan hat), ſondern namentlich gegen den Autor\pwindex{Hauptmann, Gerhart 15.11.1862 – 06.06.1946@\textsc{Hauptmann, Gerhart} (15.11.1862 – 06.06.1946), \emph{Schriftsteller}|pwv}, der durch ſeine theils
               urtheilsunfähige und unkünſtleriſche, theils auch verlogene Anhängerſchaft {\pb}zum größten\pwindex{Hauptmann, Gerhart 15.11.1862 – 06.06.1946@\textsc{Hauptmann, Gerhart} (15.11.1862 – 06.06.1946), \emph{Schriftsteller}|pwv} der modernen deutſchen Dichter ausgerufen worden iſt,
               der dieſe Rolle als ihm gebührend widerſpruchslos acceptirt hat und der nun Stück auf
               Stück ſchreibt, \strikeout{in de} (Verſunkene Glocke\pwindex{Hauptmann, Gerhart 15.11.1862 – 06.06.1946@\textsc{Hauptmann, Gerhart} (15.11.1862 – 06.06.1946), \emph{Schriftsteller}!versunkene Glocke1896@\strich\emph{Die versunkene Glocke} {[}1896{]}|pw}, Fuhrmann
                  Henſchel\pwindex{Hauptmann, Gerhart 15.11.1862 – 06.06.1946@\textsc{Hauptmann, Gerhart} (15.11.1862 – 06.06.1946), \emph{Schriftsteller}!Fuhrmann Henschel1898@\strich\emph{Fuhrmann Henschel} {[}1898{]}|pw}, Schluck und Jau\pwindex{Hauptmann, Gerhart 15.11.1862 – 06.06.1946@\textsc{Hauptmann, Gerhart} (15.11.1862 – 06.06.1946), \emph{Schriftsteller}!Schluck und Jau1900-02-03@\strich\emph{Schluck und Jau} {[}1900-02-03{]}|pw}, Michael Kramer\pwindex{Hauptmann, Gerhart 15.11.1862 – 06.06.1946@\textsc{Hauptmann, Gerhart} (15.11.1862 – 06.06.1946), \emph{Schriftsteller}!Michael Kramer. Drama1900-12-21@\strich\emph{Michael Kramer. Drama} {[}1900-12-21{]}|pw}), in dem er ſeine Mittelmäßigkeit,
               ſeine Flachheit immer deutlicher enthüllt. Der Mangel an innerem Werth iſt nirgends
               noch ſo klar hevorgetreten, als im »Michael
                  Kramer\pwindex{Hauptmann, Gerhart 15.11.1862 – 06.06.1946@\textsc{Hauptmann, Gerhart} (15.11.1862 – 06.06.1946), \emph{Schriftsteller}!Michael Kramer. Drama1900-12-21@\strich\emph{Michael Kramer. Drama} {[}1900-12-21{]}|pw}«. Ein Dichter darf ein Werk verfehlen, wenn er es als Dichter
               verfehlt. Es kann auch im verunglückten Werk \strikeout{et} etwas
               von Perſönlichkeit ſtecken, das zum Reſpekt zwingt. {\pb}Wenn aber ein Werk deutlich zeigt, daß jede Perſönlichkeit fehlt, – wenn es zeigt,
               daß keine Weltanſchauung vorhanden iſt und daß der Verſuch, eine ſolche auszudrücken,
               zu \strikeout{prä} prätentiöſem Geſchwätz führt, – wenn Alles
               hohl, albern und unfähig iſt, dann kann der Kritiker ſeine Ausdrücke nicht
               erbarmungslos genug \strikeout{feh} wählen. Das iſt nicht ein
               Irren eines Dichters, dem Großes gelungen iſt, das iſt das Zutagetreten einer
               Mediokrität, der Zeitſtimmung und allerlei andere Chancen die Möglichkeit gegeben
               haben, hier und da etwas Hübſches zu ſchreiben und ſich daraufhin als Dichter
               aufzuſpielen. Die »Weber\pwindex{Hauptmann, Gerhart 15.11.1862 – 06.06.1946@\textsc{Hauptmann, Gerhart} (15.11.1862 – 06.06.1946), \emph{Schriftsteller}!Weber1892@\strich\emph{Die Weber} {[}1892{]}|pw}« {\pb}ſind ein ſchönes Stück\pwindex{Hauptmann, Gerhart 15.11.1862 – 06.06.1946@\textsc{Hauptmann, Gerhart} (15.11.1862 – 06.06.1946), \emph{Schriftsteller}!Weber1892@\strich\emph{Die Weber} {[}1892{]}|pwv} (oder vielmehr \strikeout{wä} waren
               es ſeinerzeit; \strikeout{ob ſ} ob ſie es heut noch ſind, müßte
               man erſt \strikeout{\textcolor{gray}{noc}h} ſehen); »Hannele\pwindex{Hauptmann, Gerhart 15.11.1862 – 06.06.1946@\textsc{Hauptmann, Gerhart} (15.11.1862 – 06.06.1946), \emph{Schriftsteller}!Hanneles Himmelfahrt. Traumdichtung in zwei Teilen1893-11-14@\strich\emph{Hanneles Himmelfahrt. Traumdichtung in zwei Teilen} {[}1893-11-14{]}|pw}« \strikeout{\textcolor{gray}{i}ſ\textcolor{gray}{t}} kenne ich nicht auf der Bühne; der \label{K_L02947-4v}\edtext{»Bibelpelz\pwindex{Hauptmann, Gerhart 15.11.1862 – 06.06.1946@\textsc{Hauptmann, Gerhart} (15.11.1862 – 06.06.1946), \emph{Schriftsteller}!Biberpelz. Eine Diebskomoedie1893@\strich\emph{Der Biberpelz. Eine Diebskomödie} {[}1893{]}|pwv}«}{\lemma{\textnormal{\emph{»Bibelpelz«}}}\Cendnote{\textnormal{eigentlich \emph{Biberpelz}\pwindex{Hauptmann, Gerhart 15.11.1862 – 06.06.1946@\textsc{Hauptmann, Gerhart} (15.11.1862 – 06.06.1946), \emph{Schriftsteller}!Biberpelz. Eine Diebskomoedie1893@\strich\emph{Der Biberpelz. Eine Diebskomödie} {[}1893{]}|pwk}}}}\label{K_L02947-4h} iſt ein hübſcher Entwurf zu einem Luſtſpiel, den auszuführen die Kunft
               gemangelt hat. \textsc{Hauptmann\pwindex{Hauptmann, Gerhart 15.11.1862 – 06.06.1946@\textsc{Hauptmann, Gerhart} (15.11.1862 – 06.06.1946), \emph{Schriftsteller}|pw}s} Stern iſt im Sinken. Ich
               freue mich deſſen, weil dadurch eine der literariſchen Lügen unſerer Zeit zu Grunde
               geht, und werde es bei nächſter Gelegenheit wiederſchreiben.\pend
           \pstart
           Viele treue Grüße und nochmals von Herzen alles Glück zum neuen Jahr! Dein
                  {\\[\baselineskip]}\spacefill\mbox{Paul Goldmann}\pend
           \leftskip=0em{}\pstart
           \noindent{}\strikeout{\textcolor{gray}{Von} übe}{ }Übermorgen fahre ich wieder nach Berlin\oindex{Berlin@\textbf{Berlin}|pw}.\pend
           
         
         \endnumbering\mylabel{h}\end{ledgroupsized}  \newcommand{\dateiname}{L02947}\newcommand{\titel}{Paul Goldmann an Arthur Schnitzler, 31. 12. [1900]}\newcommand{\editorInnen}{Martin Anton Müller und Laura Untner}%% latex-leseansicht-abspann.tex
%% Abspann für die Leseansicht.
%% Der Schalter \ifkorrekturansicht ist bereits durch den Vorspann gesetzt.

%% latex-abspann.tex
%% Gemeinsamer Abspann für Korrekturansicht und Leseansicht.
%% Setzt den Schalter \ifkorrekturansicht voraus (gesetzt in den
%% einbindenden Dateien latex-korrekturansicht-abspann.tex bzw.
%% latex-leseansicht-abspann.tex).
%% ---------------------------------------------------------------

\normalsize

% Das esempio-Environment wird nur in der Leseansicht benötigt
\ifkorrekturansicht\else
\newenvironment{esempio}[3]%
{
    \vspace{1.5ex}
    \rlap{\underline{#1}}
    \par
    \setlength{\parindent}{0cm}
    \nopagebreak
    \leftskip=#2cm
    \rightskip=#3cm
}
{
    \par
}
\fi

\doendnotes{C}
\bigskip
\vfill

\clearpage

\footnotesize

\ifkorrekturansicht
  \lohead{\textsc{register}}
\fi

% theindex-Environment neu definieren ohne reledmac
\makeatletter
\renewenvironment{theindex}{%
  \ifkorrekturansicht
    \section*{\indexname}%
  \else
    \subsubsection*{Index der erwähnten Entitäten}%
  \fi
  \setlength{\parindent}{0pt}%
  \setlength{\parskip}{0pt plus 0.3pt}%
  \let\item\@idxitem
}{%
  \ifkorrekturansicht\clearpage\fi
}
\makeatother

\IfFileExists{\jobname-pw.ind}{\input{\jobname-pw.ind}}{}

% Quellenangabe nur in der Leseansicht
\ifkorrekturansicht\else
% Fallback-Definitionen, falls die .tex-Datei \titel etc. nicht gesetzt hat
\providecommand{\titel}{}
\providecommand{\editorInnen}{}
\providecommand{\dateiname}{\jobname}

\vspace{3cm}

\vfill

\footnotesize
\textsc{Quelle}: \titel. Herausgegeben von {\editorInnen}. In: \emph{Arthur Schnitzler: Briefwechsel mit Autorinnen und Autoren}.
 Digitale Edition, https://schnitzler-briefe.acdh.oeaw.ac.at/{\dateiname}.html (Stand \today)
\fi

\end{document}


      