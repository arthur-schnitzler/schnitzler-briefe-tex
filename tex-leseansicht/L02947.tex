%% latex-leseansicht-vorspann.tex
%% Vorspann für die Leseansicht.
%% Lädt die gemeinsame Datei latex-vorspann.tex mit nicht gesetztem Schalter.

\newif\ifkorrekturansicht
\korrekturansichtfalse

\input{../tex-inputs/latex-vorspann}


\section[ Paul Goldmann an Arthur Schnitzler, 31. 12. {[}1900{]}]{L02947 Paul Goldmann an Arthur Schnitzler,  31. 12. [1900]}
\nopagebreak\mylabel{L02947v}
\rehead{ }\normalsize\beginnumbering\briefempfaengerindex{Schnitzler, Arthur@\textsc{Schnitzler, Arthur}!zzzGoldmann, Paul@\emph{von Paul Goldmann}!1900-12-313@{31. 12. [1900]}|(be}
\toendnotes[C]{\smallbreak\pagebreak[2]}
\correspDesc{Versand  durch Paul Goldmann am 31. 12. [1900] in Frankfurt am Main
\newline{}Erhalt  durch Arthur Schnitzler im Zeitraum [1. 1. 1901
                  – 5. 1. 1901?] in Wien}\toendnotes[C]{\smallbreak}
\Standort{DLA, A:Schnitzler, HS.NZ85.1.3170.}
\physDesc{Brief, 1 Blatt, 4 Seiten, 2341 Zeichen
\newline{}Handschrift: 1) blaue Tinte, deutsche Kurrent\hspace{1em}2) schwarze Tinte, deutsche Kurrent (\noindent{}sechs Zeilen auf der ersten Seite)\hspace{1em}
\newline{}Schnitzler: 1) mit Bleistift das Jahr »900« vermerkt  2) mit rotem Buntstift zwei Unterstreichungen}\toendnotes[C]{\smallbreak}
\pstart
           \noindent{}
\pstart
           {\pb}Frankfurt\oindex{Frankfurt am Main@\textbf{Frankfurt am Main}, \emph{Hauptstadt}|pw}, 31. December.\pend
           
\pstart
           \raggedleft{}\textcolor{gray}{\textbf{Reuterweg 59\oindex{Reuterweg@\textbf{Reuterweg}, \emph{Straße}|pw}.}}\pend
           \pend
           
\pstart
           \centering{}Mein lieber Freund,\pend
           
\pstart
           Ich danke Dir für Deine eingehende Erörterung meines \label{K_L02947-1v}\edtext{Feuilletons\pwindex{Goldmann, Paul 31.\,1.\,1865 Breslau – 25.\,9.\,1935 Wien@\textsc{Goldmann, Paul} (31.\,1.\,1865 Breslau – 25.\,9.\,1935 Wien), \emph{Schriftsteller, Journalist}!Michael Kramer.«@\strich\emph{»Michael Kramer.«}|pwv}}{\lemma{\textnormal{\emph{Feuilletons}}}\Cendnote{\textnormal{Paul Goldmann\pwindex{Goldmann, Paul 31.\,1.\,1865 Breslau – 25.\,9.\,1935 Wien@\textsc{Goldmann, Paul} (31.\,1.\,1865 Breslau – 25.\,9.\,1935 Wien), \emph{Schriftsteller, Journalist}|pwk}: \emph{»Michael Kramer«}\pwindex{Goldmann, Paul 31.\,1.\,1865 Breslau – 25.\,9.\,1935 Wien@\textsc{Goldmann, Paul} (31.\,1.\,1865 Breslau – 25.\,9.\,1935 Wien), \emph{Schriftsteller, Journalist}!Michael Kramer.«@\strich\emph{»Michael Kramer.«}|pwk}. In: \emph{Neue Freie Presse}\pwindex{Neue Freie Presse@\emph{Neue Freie Presse}|pwk}, Nr. 13.055, 28. 12. 1900, Morgenblatt, S. 1–3.}}}\label{K_L02947-1}, finde aber, daß
               ich abſolut Recht habe und würde{ }ſelbſt jetzt, wo ich weiß, daß Dir gewiſſe Bemerkungen\pwindex{Goldmann, Paul 31.\,1.\,1865 Breslau – 25.\,9.\,1935 Wien@\textsc{Goldmann, Paul} (31.\,1.\,1865 Breslau – 25.\,9.\,1935 Wien), \emph{Schriftsteller, Journalist}!Michael Kramer.«@\strich\emph{»Michael Kramer.«}|pwv} unangebracht
               erſcheinen, dieſe Bemerkungen\pwindex{Goldmann, Paul 31.\,1.\,1865 Breslau – 25.\,9.\,1935 Wien@\textsc{Goldmann, Paul} (31.\,1.\,1865 Breslau – 25.\,9.\,1935 Wien), \emph{Schriftsteller, Journalist}!Michael Kramer.«@\strich\emph{»Michael Kramer.«}|pwv}
               nochmals mit ruhigem Gewiſſen niederſchreiben. Ich habe die Kritik\pwindex{Goldmann, Paul 31.\,1.\,1865 Breslau – 25.\,9.\,1935 Wien@\textsc{Goldmann, Paul} (31.\,1.\,1865 Breslau – 25.\,9.\,1935 Wien), \emph{Schriftsteller, Journalist}!Michael Kramer.«@\strich\emph{»Michael Kramer.«}|pwv} im hellen Zorn verfaßt, im Zorn
               nicht nur gegen die Kritikloſigkeit der \textsc{Hauptmann\pwindex{Hauptmann, Gerhart 15.\,11.\,1862 Szczawno-Zdrój – 6.\,6.\,1946 Jagniątków@\textsc{Hauptmann, Gerhart} (15.\,11.\,1862 Szczawno-Zdrój – 6.\,6.\,1946 Jagniątków), \emph{Schriftsteller}|pw}}-Anhänger (unter denen{ }ſich unſer Freund \textsc{Kerr\pwindex{Kerr, Alfred 25.\,12.\,1867 Breslau – 12.\,10.\,1948 Hamburg@\textsc{Kerr, Alfred} (25.\,12.\,1867 Breslau – 12.\,10.\,1948 Hamburg), \emph{Schriftsteller, Kritiker}|pw}} beſonders hevorgethan hat),{ }ſondern namentlich gegen den Autor\pwindex{Hauptmann, Gerhart 15.\,11.\,1862 Szczawno-Zdrój – 6.\,6.\,1946 Jagniątków@\textsc{Hauptmann, Gerhart} (15.\,11.\,1862 Szczawno-Zdrój – 6.\,6.\,1946 Jagniątków), \emph{Schriftsteller}|pwv}, der durch{ }ſeine theils
               urtheilsunfähige und unkünſtleriſche, theils auch verlogene Anhängerſchaft {\pb}zum größten\pwindex{Hauptmann, Gerhart 15.\,11.\,1862 Szczawno-Zdrój – 6.\,6.\,1946 Jagniątków@\textsc{Hauptmann, Gerhart} (15.\,11.\,1862 Szczawno-Zdrój – 6.\,6.\,1946 Jagniątków), \emph{Schriftsteller}|pwv} der modernen deutſchen Dichter ausgerufen worden iſt,
               der dieſe Rolle als ihm gebührend widerſpruchslos acceptirt hat und der nun Stück auf
               Stück{ }ſchreibt, \strikeout{in de} (Verſunkene Glocke\pwindex{Hauptmann, Gerhart 15.\,11.\,1862 Szczawno-Zdrój – 6.\,6.\,1946 Jagniątków@\textsc{Hauptmann, Gerhart} (15.\,11.\,1862 Szczawno-Zdrój – 6.\,6.\,1946 Jagniątków), \emph{Schriftsteller}!versunkene Glocke. Ein deutsches Märchendrama in fünf Aufzügen@\strich\emph{Die versunkene Glocke. Ein deutsches Märchendrama in fünf Aufzügen}|pw}, Fuhrmann
                  Henſchel\pwindex{Hauptmann, Gerhart 15.\,11.\,1862 Szczawno-Zdrój – 6.\,6.\,1946 Jagniątków@\textsc{Hauptmann, Gerhart} (15.\,11.\,1862 Szczawno-Zdrój – 6.\,6.\,1946 Jagniątków), \emph{Schriftsteller}!Fuhrmann Henschel@\strich\emph{Fuhrmann Henschel}|pw}, Schluck und Jau\pwindex{Hauptmann, Gerhart 15.\,11.\,1862 Szczawno-Zdrój – 6.\,6.\,1946 Jagniątków@\textsc{Hauptmann, Gerhart} (15.\,11.\,1862 Szczawno-Zdrój – 6.\,6.\,1946 Jagniątków), \emph{Schriftsteller}!Schluck und Jau@\strich\emph{Schluck und Jau}|pw}, Michael Kramer\pwindex{Hauptmann, Gerhart 15.\,11.\,1862 Szczawno-Zdrój – 6.\,6.\,1946 Jagniątków@\textsc{Hauptmann, Gerhart} (15.\,11.\,1862 Szczawno-Zdrój – 6.\,6.\,1946 Jagniątków), \emph{Schriftsteller}!Michael Kramer. Drama@\strich\emph{Michael Kramer. Drama}|pw}), in dem er{ }ſeine Mittelmäßigkeit,{ }ſeine Flachheit immer deutlicher enthüllt. Der Mangel an innerem Werth iſt nirgends
               noch{ }ſo klar hevorgetreten, als im »Michael
                  Kramer\pwindex{Hauptmann, Gerhart 15.\,11.\,1862 Szczawno-Zdrój – 6.\,6.\,1946 Jagniątków@\textsc{Hauptmann, Gerhart} (15.\,11.\,1862 Szczawno-Zdrój – 6.\,6.\,1946 Jagniątków), \emph{Schriftsteller}!Michael Kramer. Drama@\strich\emph{Michael Kramer. Drama}|pw}«. Ein Dichter darf ein Werk verfehlen, wenn er es als Dichter
               verfehlt. Es kann auch im verunglückten Werk \strikeout{et} etwas
               von Perſönlichkeit{ }ſtecken, das zum Reſpekt zwingt. {\pb}Wenn aber ein Werk deutlich zeigt, daß jede Perſönlichkeit fehlt, – wenn es zeigt,
               daß keine Weltanſchauung vorhanden iſt und daß der Verſuch, eine{ }ſolche auszudrücken,
               zu \strikeout{prä} prätentiöſem Geſchwätz führt, – wenn Alles
               hohl, albern und unfähig iſt, dann kann der Kritiker{ }ſeine Ausdrücke nicht
               erbarmungslos genug \strikeout{feh} wählen. Das iſt nicht ein
               Irren eines Dichters, dem Großes gelungen iſt, das iſt das Zutagetreten einer
               Mediokrität, der Zeitſtimmung und allerlei andere Chancen die Möglichkeit gegeben
               haben, hier und da etwas Hübſches zu{ }ſchreiben und{ }ſich daraufhin als Dichter
               aufzuſpielen. Die »Weber\pwindex{Hauptmann, Gerhart 15.\,11.\,1862 Szczawno-Zdrój – 6.\,6.\,1946 Jagniątków@\textsc{Hauptmann, Gerhart} (15.\,11.\,1862 Szczawno-Zdrój – 6.\,6.\,1946 Jagniątków), \emph{Schriftsteller}!Weber@\strich\emph{Die Weber}|pw}« {\pb}ſind ein{ }ſchönes Stück\pwindex{Hauptmann, Gerhart 15.\,11.\,1862 Szczawno-Zdrój – 6.\,6.\,1946 Jagniątków@\textsc{Hauptmann, Gerhart} (15.\,11.\,1862 Szczawno-Zdrój – 6.\,6.\,1946 Jagniątków), \emph{Schriftsteller}!Weber@\strich\emph{Die Weber}|pwv} (oder vielmehr \strikeout{wä} waren
               es{ }ſeinerzeit; \strikeout{ob{ }ſ} ob{ }ſie es heut noch{ }ſind, müßte
               man erſt \strikeout{\textcolor{gray}{noc}h}{ }ſehen); »Hannele\pwindex{Hauptmann, Gerhart 15.\,11.\,1862 Szczawno-Zdrój – 6.\,6.\,1946 Jagniątków@\textsc{Hauptmann, Gerhart} (15.\,11.\,1862 Szczawno-Zdrój – 6.\,6.\,1946 Jagniątków), \emph{Schriftsteller}!Hanneles Himmelfahrt. Traumdichtung in zwei Teilen@\strich\emph{Hanneles Himmelfahrt. Traumdichtung in zwei Teilen}|pw}« \strikeout{\textcolor{gray}{i}ſ\textcolor{gray}{t}} kenne ich nicht auf der Bühne; der \label{K_L02947-2v}\edtext{»Bibelpelz\pwindex{Hauptmann, Gerhart 15.\,11.\,1862 Szczawno-Zdrój – 6.\,6.\,1946 Jagniątków@\textsc{Hauptmann, Gerhart} (15.\,11.\,1862 Szczawno-Zdrój – 6.\,6.\,1946 Jagniątków), \emph{Schriftsteller}!Biberpelz. Eine Diebskomödie@\strich\emph{Der Biberpelz. Eine Diebskomödie}|pwv}«}{\lemma{\textnormal{\emph{»Bibelpelz«}}}\Cendnote{\textnormal{eigentlich \emph{Biberpelz}\pwindex{Hauptmann, Gerhart 15.\,11.\,1862 Szczawno-Zdrój – 6.\,6.\,1946 Jagniątków@\textsc{Hauptmann, Gerhart} (15.\,11.\,1862 Szczawno-Zdrój – 6.\,6.\,1946 Jagniątków), \emph{Schriftsteller}!Biberpelz. Eine Diebskomödie@\strich\emph{Der Biberpelz. Eine Diebskomödie}|pwk}}}}\label{K_L02947-2} iſt ein hübſcher Entwurf zu einem Luſtſpiel, den auszuführen die Kunft
               gemangelt hat. \textsc{Hauptmanns\pwindex{Hauptmann, Gerhart 15.\,11.\,1862 Szczawno-Zdrój – 6.\,6.\,1946 Jagniątków@\textsc{Hauptmann, Gerhart} (15.\,11.\,1862 Szczawno-Zdrój – 6.\,6.\,1946 Jagniątków), \emph{Schriftsteller}|pw}} Stern iſt im Sinken. Ich
               freue mich deſſen, weil dadurch eine der literariſchen Lügen unſerer Zeit zu Grunde
               geht, und werde es bei nächſter Gelegenheit wiederſchreiben.\pend
           
\pstart
           Viele treue Grüße und nochmals von Herzen alles Glück zum neuen Jahr! Dein {\\[\baselineskip]}\spacefill\mbox{Paul Goldmann}\pend
           \leftskip=0em{}
\pstart
           \noindent{}\strikeout{\textcolor{gray}{Von} übe}{ }Übermorgen fahre ich wieder nach Berlin\oindex{Berlin@\textbf{Berlin}, \emph{Hauptstadt}|pw}.\pend
           \selectlanguage{ngerman}\endnumbering\briefempfaengerindex{Schnitzler, Arthur@\textsc{Schnitzler, Arthur}!zzzGoldmann, Paul@\emph{von Paul Goldmann}!1900-12-313@{31. 12. [1900]}|)be}\mylabel{L02947h}  \newcommand{\dateiname}{L02947}\newcommand{\titel}{Paul Goldmann an Arthur Schnitzler, 31. 12. [1900]}\newcommand{\editorInnen}{Martin Anton Müller und Laura Untner}%% latex-leseansicht-abspann.tex
%% Abspann für die Leseansicht.
%% Der Schalter \ifkorrekturansicht ist bereits durch den Vorspann gesetzt.

%% latex-abspann.tex
%% Gemeinsamer Abspann für Korrekturansicht und Leseansicht.
%% Setzt den Schalter \ifkorrekturansicht voraus (gesetzt in den
%% einbindenden Dateien latex-korrekturansicht-abspann.tex bzw.
%% latex-leseansicht-abspann.tex).
%% ---------------------------------------------------------------

\normalsize

% Das esempio-Environment wird nur in der Leseansicht benötigt
\ifkorrekturansicht\else
\newenvironment{esempio}[3]%
{
    \vspace{1.5ex}
    \rlap{\underline{#1}}
    \par
    \setlength{\parindent}{0cm}
    \nopagebreak
    \leftskip=#2cm
    \rightskip=#3cm
}
{
    \par
}
\fi

\doendnotes{C}
\bigskip
\vfill

\clearpage

\footnotesize

\ifkorrekturansicht
  \lohead{\textsc{register}}
\fi

% theindex-Environment neu definieren ohne reledmac
\makeatletter
\renewenvironment{theindex}{%
  \ifkorrekturansicht
    \section*{\indexname}%
  \else
    \subsubsection*{Index der erwähnten Entitäten}%
  \fi
  \setlength{\parindent}{0pt}%
  \setlength{\parskip}{0pt plus 0.3pt}%
  \let\item\@idxitem
}{%
  \ifkorrekturansicht\clearpage\fi
}
\makeatother

\IfFileExists{\jobname-pw.ind}{\input{\jobname-pw.ind}}{}

% Quellenangabe nur in der Leseansicht
\ifkorrekturansicht\else
% Fallback-Definitionen, falls die .tex-Datei \titel etc. nicht gesetzt hat
\providecommand{\titel}{}
\providecommand{\editorInnen}{}
\providecommand{\dateiname}{\jobname}

\vspace{3cm}

\vfill

\footnotesize
\textsc{Quelle}: \titel. Herausgegeben von {\editorInnen}. In: \emph{Arthur Schnitzler: Briefwechsel mit Autorinnen und Autoren}.
 Digitale Edition, https://schnitzler-briefe.acdh.oeaw.ac.at/{\dateiname}.html (Stand \today)
\fi

\end{document}


