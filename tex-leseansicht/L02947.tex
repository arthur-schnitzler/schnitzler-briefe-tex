%% latex-korrekturansicht-vorspann.tex
%% Vorspann für die Korrekturansicht.
%% Lädt die gemeinsame Datei latex-vorspann.tex mit gesetztem Schalter.

\newif\ifkorrekturansicht
\korrekturansichttrue

\input{../tex-inputs/latex-vorspann}


\section[ Paul Goldmann an Arthur Schnitzler, 31. 12. {[}1900{]}]{L02947 Paul Goldmann an Arthur Schnitzler, 31. 12. {[}1900{]}}
\nopagebreak\mylabel{L02947v}
\rehead{ }\normalsize\beginnumbering\briefempfaengerindex{Schnitzler, Arthur@\textsc{Schnitzler, Arthur}!zzzGoldmann, Paul@\emph{von Paul Goldmann}!1900-12-311@{31. 12. {[}1900{]}}|(be}
\toendnotes[C]{\smallbreak\pagebreak[2]}\Standort{DLA, A:Schnitzler, HS.NZ85.1.3170.}
\physDesc{Brief, 1 Blatt, 4 Seiten, 2341 Zeichen
\newline{}Handschrift: 1) blaue Tinte, deutsche Kurrent\hspace{1em}2) schwarze Tinte, deutsche Kurrent (\noindent{}sechs Zeilen auf der ersten Seite)\hspace{1em}
\newline{}Schnitzler: 1) mit Bleistift das Jahr »900« vermerkt  2) mit rotem Buntstift zwei Unterstreichungen}\toendnotes[C]{\smallbreak}
\pstart
           \noindent{}
\pstart
           {\pb}Frankfurt\oindex{Frankfurt am Main@\textbf{Frankfurt am Main}, \emph{P.PPLA3}|pw}, 31. December.\pend
           
\pstart
           \raggedleft{}\textcolor{gray}{\textbf{Reuterweg 59\oindex{Reuterweg@\textbf{Reuterweg}, \emph{Straße (K.STR)}|pw}.}}\pend
           \pend
           
\pstart
           \centering{}Mein lieber Freund,\pend
           
\pstart
           Ich danke Dir für Deine eingehende Erörterung meines \label{K_L02947-1v}\edtext{Feuilletons\pwindex{Michael Kramer.«@\emph{»Michael Kramer.«}|pwv}}{\lemma{\textnormal{\emph{Feuilletons}}}\Cendnote{\textnormal{Paul Goldmann\pwindex{Goldmann, Paul 31.01.1865 – 25.09.1935@\textsc{Goldmann, Paul} (31.01.1865 – 25.09.1935), \emph{Schriftsteller/Schriftstellerin, Journalist/Journalistin}|pwk}: \emph{»Michael Kramer«}\pwindex{Michael Kramer.«@\emph{»Michael Kramer.«}|pwk}. In: \emph{Neue Freie Presse}\pwindex{Neue Freie Presse@\emph{Neue Freie Presse}|pwk}, Nr. 13.055, 28. 12. 1900, Morgenblatt, S. 1–3.}}}\label{K_L02947-1}, finde aber, daß
               ich abſolut Recht habe und würde ſelbſt jetzt, wo ich weiß, daß Dir gewiſſe Bemerkungen\pwindex{Michael Kramer.«@\emph{»Michael Kramer.«}|pwv} unangebracht
               erſcheinen, dieſe Bemerkungen\pwindex{Michael Kramer.«@\emph{»Michael Kramer.«}|pwv}
               nochmals mit ruhigem Gewiſſen niederſchreiben. Ich habe die Kritik\pwindex{Michael Kramer.«@\emph{»Michael Kramer.«}|pwv} im hellen Zorn verfaßt, im Zorn
               nicht nur gegen die Kritikloſigkeit der \textsc{Hauptmann\pwindex{Hauptmann, Gerhart 15.11.1862 – 06.06.1946@\textsc{Hauptmann, Gerhart} (15.11.1862 – 06.06.1946), \emph{Schriftsteller/Schriftstellerin}|pw}}-Anhänger (unter denen ſich unſer Freund \textsc{Kerr\pwindex{Kerr, Alfred 25.12.1867 – 12.10.1948@\textsc{Kerr, Alfred} (25.12.1867 – 12.10.1948), \emph{Schriftsteller/Schriftstellerin, Kritiker/Kritikerin}|pw}} beſonders hevorgethan hat), ſondern namentlich gegen den Autor\pwindex{Hauptmann, Gerhart 15.11.1862 – 06.06.1946@\textsc{Hauptmann, Gerhart} (15.11.1862 – 06.06.1946), \emph{Schriftsteller/Schriftstellerin}|pwv}, der durch ſeine theils
               urtheilsunfähige und unkünſtleriſche, theils auch verlogene Anhängerſchaft {\pb}zum größten\pwindex{Hauptmann, Gerhart 15.11.1862 – 06.06.1946@\textsc{Hauptmann, Gerhart} (15.11.1862 – 06.06.1946), \emph{Schriftsteller/Schriftstellerin}|pwv} der modernen deutſchen Dichter ausgerufen worden iſt,
               der dieſe Rolle als ihm gebührend widerſpruchslos acceptirt hat und der nun Stück auf
               Stück ſchreibt, \strikeout{in de} (Verſunkene Glocke\pwindex{versunkene Glocke. Ein deutsches Maerchendrama in fuenf Aufzuegen@\emph{Die versunkene Glocke. Ein deutsches Märchendrama in fünf Aufzügen}|pw}, Fuhrmann
                  Henſchel\pwindex{Fuhrmann Henschel@\emph{Fuhrmann Henschel}|pw}, Schluck und Jau\pwindex{Schluck und Jau@\emph{Schluck und Jau}|pw}, Michael Kramer\pwindex{Michael Kramer. Drama@\emph{Michael Kramer. Drama}|pw}), in dem er ſeine Mittelmäßigkeit,
               ſeine Flachheit immer deutlicher enthüllt. Der Mangel an innerem Werth iſt nirgends
               noch ſo klar hevorgetreten, als im »Michael
                  Kramer\pwindex{Michael Kramer. Drama@\emph{Michael Kramer. Drama}|pw}«. Ein Dichter darf ein Werk verfehlen, wenn er es als Dichter
               verfehlt. Es kann auch im verunglückten Werk \strikeout{et} etwas
               von Perſönlichkeit ſtecken, das zum Reſpekt zwingt. {\pb}Wenn aber ein Werk deutlich zeigt, daß jede Perſönlichkeit fehlt, – wenn es zeigt,
               daß keine Weltanſchauung vorhanden iſt und daß der Verſuch, eine ſolche auszudrücken,
               zu \strikeout{prä} prätentiöſem Geſchwätz führt, – wenn Alles
               hohl, albern und unfähig iſt, dann kann der Kritiker ſeine Ausdrücke nicht
               erbarmungslos genug \strikeout{feh} wählen. Das iſt nicht ein
               Irren eines Dichters, dem Großes gelungen iſt, das iſt das Zutagetreten einer
               Mediokrität, der Zeitſtimmung und allerlei andere Chancen die Möglichkeit gegeben
               haben, hier und da etwas Hübſches zu ſchreiben und ſich daraufhin als Dichter
               aufzuſpielen. Die »Weber\pwindex{Weber@\emph{Die Weber}|pw}« {\pb}ſind ein ſchönes Stück\pwindex{Weber@\emph{Die Weber}|pwv} (oder vielmehr \strikeout{wä} waren
               es ſeinerzeit; \strikeout{ob ſ} ob ſie es heut noch ſind, müßte
               man erſt \strikeout{\textcolor{gray}{noc}h} ſehen); »Hannele\pwindex{Hanneles Himmelfahrt. Traumdichtung in zwei Teilen@\emph{Hanneles Himmelfahrt. Traumdichtung in zwei Teilen}|pw}« \strikeout{\textcolor{gray}{i}ſ\textcolor{gray}{t}} kenne ich nicht auf der Bühne; der \label{K_L02947-2v}\edtext{»Bibelpelz\pwindex{Biberpelz. Eine Diebskomoedie@\emph{Der Biberpelz. Eine Diebskomödie}|pwv}«}{\lemma{\textnormal{\emph{»Bibelpelz«}}}\Cendnote{\textnormal{eigentlich \emph{Biberpelz}\pwindex{Biberpelz. Eine Diebskomoedie@\emph{Der Biberpelz. Eine Diebskomödie}|pwk}}}}\label{K_L02947-2} iſt ein hübſcher Entwurf zu einem Luſtſpiel, den auszuführen die Kunft
               gemangelt hat. \textsc{Hauptmanns\pwindex{Hauptmann, Gerhart 15.11.1862 – 06.06.1946@\textsc{Hauptmann, Gerhart} (15.11.1862 – 06.06.1946), \emph{Schriftsteller/Schriftstellerin}|pw}} Stern iſt im Sinken. Ich
               freue mich deſſen, weil dadurch eine der literariſchen Lügen unſerer Zeit zu Grunde
               geht, und werde es bei nächſter Gelegenheit wiederſchreiben.\pend
           
\pstart
           Viele treue Grüße und nochmals von Herzen alles Glück zum neuen Jahr! Dein {\\[\baselineskip]}\spacefill\mbox{Paul Goldmann}\pend
           \leftskip=0em{}
\pstart
           \noindent{}\strikeout{\textcolor{gray}{Von} übe}{ }Übermorgen fahre ich wieder nach Berlin\oindex{Berlin@\textbf{Berlin}, \emph{P.PPLC}|pw}.\pend
           \selectlanguage{ngerman}\endnumbering\briefempfaengerindex{Schnitzler, Arthur@\textsc{Schnitzler, Arthur}!zzzGoldmann, Paul@\emph{von Paul Goldmann}!1900-12-311@{31. 12. {[}1900{]}}|)be}\mylabel{L02947h}  \normalsize

\doendnotes{C}
\bigskip
\vfill

\clearpage

\footnotesize

\lohead{\textsc{register}}

% Definiere theindex-Environment komplett neu ohne reledmac
\makeatletter
\renewenvironment{theindex}{%
  \section*{\indexname}%
  \setlength{\parindent}{0pt}%
  \setlength{\parskip}{0pt plus 0.3pt}%
  \let\item\@idxitem
}{%
  \clearpage
}
\makeatother

\IfFileExists{\jobname-pw.ind}{\input{\jobname-pw.ind}}{}

\end{document}

      