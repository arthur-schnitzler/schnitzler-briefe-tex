%% latex-korrekturansicht-vorspann.tex
%% Vorspann für die Korrekturansicht.
%% Lädt die gemeinsame Datei latex-vorspann.tex mit gesetztem Schalter.

\newif\ifkorrekturansicht
\korrekturansichttrue

\input{../tex-inputs/latex-vorspann}


\section[Arthur Schnitzler an Hugo von Hofmannsthal, {[}24. 8. 1904{]}]{L01432 Arthur Schnitzler an Hugo von Hofmannsthal, {[}24. 8. 1904{]}}
\nopagebreak\mylabel{L01432v}
\rehead{ }\normalsize\beginnumbering\briefempfaengerindex{Hofmannsthal, Hugo von@\textsc{Hofmannsthal, Hugo von}!zzzSchnitzler, Arthur@\emph{von Arthur Schnitzler}!1904-08-241@{{[}24. 8. 1904{]}}|(be}
\toendnotes[C]{\smallbreak\pagebreak[2]}\Standort{FDH, Hs-30885,113.}
\physDesc{Brief, 2 Blätter, 6 Seiten, 1912 Zeichen
\newline{}Handschrift: Bleistift, deutsche Kurrent
\newline{}Ordnung: mit Bleistift von Schnitzler – mutmaßlich bei der
                                 Durchsicht der Briefe 1929 – beide Blätter datiert: »24/8 904« respektive »24/8 04« und das zweite Blatt auch mit »II«
                                 kenntlich gemacht }
\buchAbdrucke{\weitereDrucke{Hugo von Hofmannsthal, Arthur Schnitzler: \emph{Briefwechsel}. Frankfurt am Main: \emph{S. Fischer} 1964, S. 200.} }\toendnotes[C]{\smallbreak}
\pstart
           \noindent{}{\pb}lieber Hugo, we{\geminationn} es irgend möglich iſt,
               ſo werden wir am 3. bereit ſein – jedenfalls wird es \textsc{Gerty}\pwindex{Hofmannsthal, Gertrude von 16.03.1880 – 09.11.1959@\textsc{Hofmannsthal, Gertrude von} (16.03.1880 – 09.11.1959)|pw} 3–4 Tage früher wiſſen. Wir wollen jedenfalls einige Zeit in Iſchl\oindex{Bad Ischl@\textbf{Bad Ischl}, \emph{P.PPL}|pw} bleiben; ja unſre eigentliche Abſicht war, uns dort in
               Ruhe niederzulaſſen und von dort hie u da auszufliegen. Die Hotels an den Salzk.gut\oindex{Salzkammergut@\textbf{Salzkammergut}, \emph{L.RGN}|pw}ſeen ſind mir ſoweit ich ſie kenne,
               zuwider, und ich denke, wir werden uns ev. auf Salz{\pb}burg\oindex{Salzburg@\textbf{Salzburg}, \emph{A.ADM2}|pw} einigen? Ich denke ja, \textsc{Gerty}\pwindex{Hofmannsthal, Gertrude von 16.03.1880 – 09.11.1959@\textsc{Hofmannsthal, Gertrude von} (16.03.1880 – 09.11.1959)|pw} bleibt auch ein paar Tage bei ihrer Mama\pwindex{Schlesinger, Franziska 17.08.1851 – 11.08.1932@\textsc{Schlesinger, Franziska} (17.08.1851 – 11.08.1932)|pwv} in Iſchl\oindex{Bad Ischl@\textbf{Bad Ischl}, \emph{P.PPL}|pw}, und Sie
               holen ſie mindeſtens ab? Oder ſind in Iſchl\oindex{Bad Ischl@\textbf{Bad Ischl}, \emph{P.PPL}|pw}, wenn
               ſie ankommt? Oder kommen aus Auſſee\oindex{Bad Aussee@\textbf{Bad Aussee}, \emph{P.PPLA3}|pw} auf ein paar
               Stunden herüber, bei welcher Gelegenheit man weiteres beſprechen könnte? – Außer Iſchl\oindex{Bad Ischl@\textbf{Bad Ischl}, \emph{P.PPL}|pw} hatten wir auch \textsc{Salegg}\oindex{Burg Salegg@\textbf{Burg Salegg}, \emph{Gebäude (K.GBD)}|pw} (bei Waidbruck\oindex{Ponte Gardena@\textbf{Ponte Gardena}, \emph{A.ADM3}|pw}) in {\pb}Erwägung gezogen, wegen der, von Olga\pwindex{Schnitzler, Olga 17.01.1882 – 13.01.1970@\textsc{Schnitzler, Olga} (17.01.1882 – 13.01.1970), \emph{Schauspieler/Schauspielerin, Sänger/Sängerin}|pw} u mir ſehr erſehnten (mäßigen) Höhe und Stille. \textsc{Salegg}\oindex{Burg Salegg@\textbf{Burg Salegg}, \emph{Gebäude (K.GBD)}|pw} hätte dann auch den Vortheil, we{\geminationn} der
                  Herbſt mit Macht hereinbricht, daſs man Bozen\oindex{Bozen@\textbf{Bozen}, \emph{P.PPLA2}|pw}, Meran\oindex{Meran@\textbf{Meran}, \emph{P.PPLA3}|pw} ganz nahe
               hat. –\pend
           
\pstart
           Worauf ich einigermaßen rechne \substVorne{}\textsuperscript{ſind}\substDazwischen{}iſt\substHinten{} aber ganz beſonders irgend eine kleine Radtour, die wir, Sie und ich, machen
               könnten, ſo von 2–3 Tagen, oder 2 kleinere, {\pb}in welchem
               Betracht ich d\substVorne{}\textsuperscript{en}\substDazwischen{}ie\substHinten{}{ }\textsc{ego}- u \textsc{olga}\pwindex{Schnitzler, Olga 17.01.1882 – 13.01.1970@\textsc{Schnitzler, Olga} (17.01.1882 – 13.01.1970), \emph{Schauspieler/Schauspielerin, Sänger/Sängerin}|pw}iſtiſche Hoffnung nicht unterdrücken kann, daſs während dieſer Zeit Olga\pwindex{Schnitzler, Olga 17.01.1882 – 13.01.1970@\textsc{Schnitzler, Olga} (17.01.1882 – 13.01.1970), \emph{Schauspieler/Schauspielerin, Sänger/Sängerin}|pw} u \textsc{Gerty}\pwindex{Hofmannsthal, Gertrude von 16.03.1880 – 09.11.1959@\textsc{Hofmannsthal, Gertrude von} (16.03.1880 – 09.11.1959)|pw} zuſa{\geminationm}en ſind oder uns gar auf hohem Einſpänner
               vorausraſen?\pend
           
\pstart
           – Aber all dies eignet ſich zu mündlicher Verſtändigg; für heute möcht ich nur
               wiſſen, \uline{wann} ich Sie in Iſchl\oindex{Bad Ischl@\textbf{Bad Ischl}, \emph{P.PPL}|pw}{ }ſprechen werde, den Fall geſetzt, daſs wir am
                  3.{ }\substVorne{}\textsuperscript{M}\substDazwischen{}Na\substHinten{}chmittag dortſelbſt eintreffen\pend
           
\pstart
           {\pb}Noch eines; \textsc{Gerty}\pwindex{Hofmannsthal, Gertrude von 16.03.1880 – 09.11.1959@\textsc{Hofmannsthal, Gertrude von} (16.03.1880 – 09.11.1959)|pw} wird ja wahrſcheinlich in Wien\oindex{Wien@\textbf{Wien}, \emph{A.ADM2}|pw} zu thun
               haben; es wäre ſehr hübſch von ihr, we{\geminationn}{ }sie, wann es ihr beliebt bei uns ſpeiſen wollte;
               wir bitten um eine vorherige telegr. Verſtändigung. –\pend
           
\pstart
           Mir ginge es ganz gut, we{\geminationn} ich nicht einen etwas
               hartnäckigen Bronchialkatarrh hätte; der übrigens vielleicht noch in meinen
                  Septemberplänen eine kleine Rolle wird ſpielen müſſen. –\pend
           
\pstart
           {\pb}Und Richard\pwindex{Beer-Hofmann, Richard 1866-07-11 – 1945-09-26@\textsc{Beer-Hofmann, Richard} (1866-07-11 – 1945-09-26), \emph{Schriftsteller/Schriftstellerin}|pw}? –
               Wird er zu bewegen ſein, nach Iſchl\oindex{Bad Ischl@\textbf{Bad Ischl}, \emph{P.PPL}|pw}{ }\introOben{}oder Salzburg\oindex{Salzburg@\textbf{Salzburg}, \emph{A.ADM2}|pw}?\introOben{} zu ko{\geminationm}en? Jedenfalls möcht ich ihn ſehn – ſein Stück\pwindex{Graf von Charolais. Ein Trauerspiel@\emph{Der Graf von Charolais. Ein Trauerspiel}|pwv} hören. –\pend
           
\pstart
           Herzliche Grüße.{\\[\baselineskip]}Ihr\spacefill\mbox{A.}\pend
           \leftskip=0em{}\selectlanguage{ngerman}\endnumbering\briefempfaengerindex{Hofmannsthal, Hugo von@\textsc{Hofmannsthal, Hugo von}!zzzSchnitzler, Arthur@\emph{von Arthur Schnitzler}!1904-08-241@{{[}24. 8. 1904{]}}|)be}\mylabel{L01432h}  \normalsize

\doendnotes{C}
\bigskip
\vfill

\clearpage

\footnotesize

\lohead{\textsc{register}}

% Definiere theindex-Environment komplett neu ohne reledmac
\makeatletter
\renewenvironment{theindex}{%
  \section*{\indexname}%
  \setlength{\parindent}{0pt}%
  \setlength{\parskip}{0pt plus 0.3pt}%
  \let\item\@idxitem
}{%
  \clearpage
}
\makeatother

\IfFileExists{\jobname-pw.ind}{\input{\jobname-pw.ind}}{}

\end{document}

      