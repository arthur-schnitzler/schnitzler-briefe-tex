%% latex-leseansicht-vorspann.tex
%% Vorspann für die Leseansicht.
%% Lädt die gemeinsame Datei latex-vorspann.tex mit nicht gesetztem Schalter.

\newif\ifkorrekturansicht
\korrekturansichtfalse

\input{../tex-inputs/latex-vorspann}


\section[Arthur Schnitzler an Hugo von Hofmannsthal, {{[}}24. 8. 1904{{]}}]{L01432 Arthur Schnitzler an Hugo von Hofmannsthal, {[}24. 8. 1904{]}}
\nopagebreak\mylabel{L01432v}
\rehead{ }\normalsize\beginnumbering\briefempfaengerindex{Hofmannsthal, Hugo von@\textsc{Hofmannsthal, Hugo von}!zzzSchnitzler, Arthur@\emph{von Arthur Schnitzler}!1904-08-241@{{[}24. 8. 1904{]}}|(be}
\toendnotes[C]{\smallbreak\pagebreak[2]}
\correspDesc{Versand  durch Arthur Schnitzler am [24. 8. 1904] in Wien
\newline{}Erhalt  durch Hugo von Hofmannsthal im Zeitraum [24. 8. 1904
                  – 28. 8. 1904?] \textbf{Ort fehlend} }\toendnotes[C]{\smallbreak}
\Standort{FDH, Hs-30885,113.}
\physDesc{Brief, 2 Blätter, 6 Seiten, 1912 Zeichen
\newline{}Handschrift: Bleistift, deutsche Kurrent
\newline{}Ordnung: mit Bleistift von Schnitzler – mutmaßlich bei der
                                 Durchsicht der Briefe 1929 – beide Blätter datiert: »24/8 904« respektive »24/8 04« und das zweite Blatt auch mit »II«
                                 kenntlich gemacht }
\buchAbdrucke{\weitereDrucke{Hugo von Hofmannsthal, Arthur Schnitzler: \emph{Briefwechsel}. Herausgegeben von Therese Nickl und Heinrich Schnitzler. Frankfurt am Main: \emph{S. Fischer} 1964, S. 200.} }\toendnotes[C]{\smallbreak}
\pstart
           \noindent{}{\pb}lieber Hugo, we{\geminationn} es irgend möglich iſt,{ }ſo werden wir am 3. bereit{ }ſein – jedenfalls wird es \textsc{Gerty}\pwindex{Hofmannsthal, Gertrude von 16.\,3.\,1880 Wien – 9.\,11.\,1959 Paddington@\textsc{Hofmannsthal, Gertrude von} (16.\,3.\,1880 Wien – 9.\,11.\,1959 Paddington)|pw} 3–4 Tage früher wiſſen. Wir wollen jedenfalls einige Zeit in Iſchl\oindex{Bad Ischl@\textbf{Bad Ischl}|pw} bleiben; ja unſre eigentliche Abſicht war, uns dort in
               Ruhe niederzulaſſen und von dort hie u da auszufliegen. Die Hotels an den Salzk.gut\oindex{Salzkammergut@\textbf{Salzkammergut}, \emph{Region}|pw}ſeen{ }ſind mir{ }ſoweit ich{ }ſie kenne,
               zuwider, und ich denke, wir werden uns ev. auf Salz{\pb}burg\oindex{Salzburg@\textbf{Salzburg}, \emph{Verwaltungsgebiet}|pw} einigen? Ich denke ja, \textsc{Gerty}\pwindex{Hofmannsthal, Gertrude von 16.\,3.\,1880 Wien – 9.\,11.\,1959 Paddington@\textsc{Hofmannsthal, Gertrude von} (16.\,3.\,1880 Wien – 9.\,11.\,1959 Paddington)|pw} bleibt auch ein paar Tage bei ihrer Mama\pwindex{Schlesinger, Franziska 17.\,8.\,1851 Wien – 11.\,8.\,1932 ebd.@\textsc{Schlesinger, Franziska} (17.\,8.\,1851 Wien – 11.\,8.\,1932 ebd.)|pwv} in Iſchl\oindex{Bad Ischl@\textbf{Bad Ischl}|pw}, und Sie
               holen{ }ſie mindeſtens ab? Oder{ }ſind in Iſchl\oindex{Bad Ischl@\textbf{Bad Ischl}|pw}, wenn{ }ſie ankommt? Oder kommen aus Auſſee\oindex{Bad Aussee@\textbf{Bad Aussee}, \emph{Hauptstadt}|pw} auf ein paar
               Stunden herüber, bei welcher Gelegenheit man weiteres beſprechen könnte? – Außer Iſchl\oindex{Bad Ischl@\textbf{Bad Ischl}|pw} hatten wir auch \textsc{Salegg}\oindex{Burg Salegg@\textbf{Burg Salegg}, \emph{Gebäude}|pw} (bei Waidbruck\oindex{Ponte Gardena@\textbf{Ponte Gardena}, \emph{Verwaltungsgebiet}|pw}) in {\pb}Erwägung gezogen, wegen der, von Olga\pwindex{Schnitzler, Olga 17.\,1.\,1882 Wien – 13.\,1.\,1970 Lugano@\textsc{Schnitzler, Olga} (17.\,1.\,1882 Wien – 13.\,1.\,1970 Lugano), \emph{Schauspielerin, Sängerin}|pw} u mir{ }ſehr erſehnten (mäßigen) Höhe und Stille. \textsc{Salegg}\oindex{Burg Salegg@\textbf{Burg Salegg}, \emph{Gebäude}|pw} hätte dann auch den Vortheil, we{\geminationn} der
                  Herbſt mit Macht hereinbricht, daſs man Bozen\oindex{Bozen@\textbf{Bozen}, \emph{Hauptstadt}|pw}, Meran\oindex{Meran@\textbf{Meran}, \emph{Hauptstadt}|pw} ganz nahe
               hat. –\pend
           
\pstart
           Worauf ich einigermaßen rechne \substVorne{}\textsuperscript{ſind}\substDazwischen{}iſt\substHinten{} aber ganz beſonders irgend eine kleine Radtour, die wir, Sie und ich, machen
               könnten,{ }ſo von 2–3 Tagen, oder 2 kleinere, {\pb}in welchem
               Betracht ich d\substVorne{}\textsuperscript{en}\substDazwischen{}ie\substHinten{}{ }\textsc{ego}- u \textsc{olga}\pwindex{Schnitzler, Olga 17.\,1.\,1882 Wien – 13.\,1.\,1970 Lugano@\textsc{Schnitzler, Olga} (17.\,1.\,1882 Wien – 13.\,1.\,1970 Lugano), \emph{Schauspielerin, Sängerin}|pw}iſtiſche Hoffnung nicht unterdrücken kann, daſs während dieſer Zeit Olga\pwindex{Schnitzler, Olga 17.\,1.\,1882 Wien – 13.\,1.\,1970 Lugano@\textsc{Schnitzler, Olga} (17.\,1.\,1882 Wien – 13.\,1.\,1970 Lugano), \emph{Schauspielerin, Sängerin}|pw} u \textsc{Gerty}\pwindex{Hofmannsthal, Gertrude von 16.\,3.\,1880 Wien – 9.\,11.\,1959 Paddington@\textsc{Hofmannsthal, Gertrude von} (16.\,3.\,1880 Wien – 9.\,11.\,1959 Paddington)|pw} zuſa{\geminationm}en{ }ſind oder uns gar auf hohem Einſpänner
               vorausraſen?\pend
           
\pstart
           – Aber all dies eignet{ }ſich zu mündlicher Verſtändigg; für heute möcht ich nur
               wiſſen, \uline{wann} ich Sie in Iſchl\oindex{Bad Ischl@\textbf{Bad Ischl}|pw}{ }ſprechen werde, den Fall geſetzt, daſs wir am
                  3.{ }\substVorne{}\textsuperscript{M}\substDazwischen{}Na\substHinten{}chmittag dortſelbſt eintreffen\pend
           
\pstart
           {\pb}Noch eines; \textsc{Gerty}\pwindex{Hofmannsthal, Gertrude von 16.\,3.\,1880 Wien – 9.\,11.\,1959 Paddington@\textsc{Hofmannsthal, Gertrude von} (16.\,3.\,1880 Wien – 9.\,11.\,1959 Paddington)|pw} wird ja wahrſcheinlich in Wien\oindex{Wien@\textbf{Wien}, \emph{Verwaltungsgebiet}|pw} zu thun
               haben; es wäre{ }ſehr hübſch von ihr, we{\geminationn}{ }sie, wann es ihr beliebt bei uns{ }ſpeiſen wollte;
               wir bitten um eine vorherige telegr. Verſtändigung. –\pend
           
\pstart
           Mir ginge es ganz gut, we{\geminationn} ich nicht einen etwas
               hartnäckigen Bronchialkatarrh hätte; der übrigens vielleicht noch in meinen
                  Septemberplänen eine kleine Rolle wird{ }ſpielen müſſen. –\pend
           
\pstart
           {\pb}Und Richard\pwindex{Beer-Hofmann, Richard 11.\,7.\,1866 Wien – 26.\,9.\,1945 New York City@\textsc{Beer-Hofmann, Richard} (11.\,7.\,1866 Wien – 26.\,9.\,1945 New York City), \emph{Schriftsteller}|pw}? –
               Wird er zu bewegen{ }ſein, nach Iſchl\oindex{Bad Ischl@\textbf{Bad Ischl}|pw}{ }\introOben{}oder Salzburg\oindex{Salzburg@\textbf{Salzburg}, \emph{Verwaltungsgebiet}|pw}?\introOben{} zu ko{\geminationm}en? Jedenfalls möcht ich ihn{ }ſehn –{ }ſein Stück\pwindex{Beer-Hofmann, Richard 11.\,7.\,1866 Wien – 26.\,9.\,1945 New York City@\textsc{Beer-Hofmann, Richard} (11.\,7.\,1866 Wien – 26.\,9.\,1945 New York City), \emph{Schriftsteller}!Graf von Charolais. Ein Trauerspiel@\strich\emph{Der Graf von Charolais. Ein Trauerspiel}|pwv} hören. –\pend
           
\pstart
           Herzliche Grüße.{\\[\baselineskip]}Ihr\spacefill\mbox{A.}\pend
           \leftskip=0em{}\selectlanguage{ngerman}\endnumbering\briefempfaengerindex{Hofmannsthal, Hugo von@\textsc{Hofmannsthal, Hugo von}!zzzSchnitzler, Arthur@\emph{von Arthur Schnitzler}!1904-08-241@{{[}24. 8. 1904{]}}|)be}\mylabel{L01432h}  \newcommand{\dateiname}{L01432}\newcommand{\titel}{Arthur Schnitzler an Hugo von Hofmannsthal, [24. 8. 1904]}\newcommand{\editorInnen}{Martin Anton Müller und Gerd-Hermann Susen}%% latex-leseansicht-abspann.tex
%% Abspann für die Leseansicht.
%% Der Schalter \ifkorrekturansicht ist bereits durch den Vorspann gesetzt.

%% latex-abspann.tex
%% Gemeinsamer Abspann für Korrekturansicht und Leseansicht.
%% Setzt den Schalter \ifkorrekturansicht voraus (gesetzt in den
%% einbindenden Dateien latex-korrekturansicht-abspann.tex bzw.
%% latex-leseansicht-abspann.tex).
%% ---------------------------------------------------------------

\normalsize

% Das esempio-Environment wird nur in der Leseansicht benötigt
\ifkorrekturansicht\else
\newenvironment{esempio}[3]%
{
    \vspace{1.5ex}
    \rlap{\underline{#1}}
    \par
    \setlength{\parindent}{0cm}
    \nopagebreak
    \leftskip=#2cm
    \rightskip=#3cm
}
{
    \par
}
\fi

\doendnotes{C}
\bigskip
\vfill

\clearpage

\footnotesize

\ifkorrekturansicht
  \lohead{\textsc{register}}
\fi

% theindex-Environment neu definieren ohne reledmac
\makeatletter
\renewenvironment{theindex}{%
  \ifkorrekturansicht
    \section*{\indexname}%
  \else
    \subsubsection*{Index der erwähnten Entitäten}%
  \fi
  \setlength{\parindent}{0pt}%
  \setlength{\parskip}{0pt plus 0.3pt}%
  \let\item\@idxitem
}{%
  \ifkorrekturansicht\clearpage\fi
}
\makeatother

\IfFileExists{\jobname-pw.ind}{\input{\jobname-pw.ind}}{}

% Quellenangabe nur in der Leseansicht
\ifkorrekturansicht\else
% Fallback-Definitionen, falls die .tex-Datei \titel etc. nicht gesetzt hat
\providecommand{\titel}{}
\providecommand{\editorInnen}{}
\providecommand{\dateiname}{\jobname}

\vspace{3cm}

\vfill

\footnotesize
\textsc{Quelle}: \titel. Herausgegeben von {\editorInnen}. In: \emph{Arthur Schnitzler: Briefwechsel mit Autorinnen und Autoren}.
 Digitale Edition, https://schnitzler-briefe.acdh.oeaw.ac.at/{\dateiname}.html (Stand \today)
\fi

\end{document}


