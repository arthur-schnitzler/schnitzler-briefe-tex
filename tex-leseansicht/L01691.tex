%% latex-leseansicht-vorspann.tex
%% Vorspann für die Leseansicht.
%% Lädt die gemeinsame Datei latex-vorspann.tex mit nicht gesetztem Schalter.

\newif\ifkorrekturansicht
\korrekturansichtfalse

\input{../tex-inputs/latex-vorspann}


         
         \renewcommand{\erwaehntePersonen}{Personen: Richard Beer-Hofmann, Paula Beer-Hofmann, Naëmah Beer-Hofmann, Gabriel Beer-Hofmann, Mirjam Beer-Hofmann, Alois Hofmann, Heinrich Schnitzler, Olga Schnitzler}
         \renewcommand{\erwaehnteOrte}{Orte: Edmund-Weiß-Gasse 7, Hasenauerstraße, Meran, Welsberg-Taisten, Wien, Wildbad Waldbrunn, XVIII., Währing}
         \renewcommand{\erwaehnteWerke}{Werke: Die Zeit, [Todesanzeige von Alois Hofmann]}
               \section[Arthur Schnitzler an Richard Beer-Hofmann, 14. 7. 1907]{ Arthur Schnitzler an Richard Beer-Hofmann, 14. 7. 1907}\nopagebreak\mylabel{v}\rehead{ }\begin{ledgroupsized}[t]{13cm}\normalsize\beginnumbering\briefempfaengerindex{Beer-Hofmann, Richard@\textsc{Beer-Hofmann, Richard}!zzzSchnitzler, Arthur@\emph{von Arthur Schnitzler}!1907-07-141@{14. 7. 1907}|(be} \toendnotes[C]{\smallbreak\pagebreak[2]} \Standort{YCGL, MSS 31.}
\physDesc{Brief, 1 Blatt, 2 Seiten, Umschlag, 810 Zeichen (Briefpapier mit Trauerrand)
\newline{}Handschrift: schwarze Tinte, deutsche Kurrent
\newline{}Versand: Stempel: »\nobreak{}\oindex{Welsberg-Taisten@\textbf{Welsberg-Taisten}|pwk}{[}Wels{]}berg, \textcolor{gray}{15}. 7. 07\nobreak{}«.  
\newline{}Beer-Hofmann: mit blauem Buntstift das Datum der Beantwortung festgehalten:
                                       »B 26/VII 07« }\buchAbdrucke{\weitereDrucke{Arthur Schnitzler, Richard Beer-Hofmann: \emph{Briefwechsel 1891–1931}. Hg. Konstanze Fliedl. Wien, Zürich: \emph{Europaverlag} 1992, S. 180.} }\toendnotes[C]{\smallbreak}\pstart{}{\pb}\textcolor{gray}{\textbf{Dr. Arthur Schnitzler}}\pend{}\pstart{}\textcolor{gray}{\textbf{Wien XVIII. Spoettelgasse 7\oindex{Edmund-Weiss-Gasse 7@\textbf{Edmund-Weiß-Gasse 7}|pw}.}}\pend{}{\bigskip}\pstart{}{\pb}\textsc{Herrn Dr. Richard Beer-Hofmann}\pend{}\pstart{}Wien XVIII\oindex{XVIII., Waehring@\textbf{XVIII., Währing}|pw}\pend{}\pstart{}\textsc{Hasenauerstr.} 59\oindex{Hasenauerstrasse@\textbf{Hasenauerstraße}|pw}.\pend{}{\bigskip}\pstart
           \raggedleft{}{\pb}\textsc{Welsberg-Waldbrunn}\oindex{Wildbad Waldbrunn@\textbf{Wildbad Waldbrunn}|pw}, 14. 7. 907\pend
           \pstart{}mein lieber Richard,\pend\pstart
           eben leſe ich in der Zeit\pwindex{Zeit1902-09-27 – 1919@\emph{Die Zeit} {[}1902-09-27 – 1919{]}|pw} die \label{K_L01691-1v}\edtext{Anzeige\pwindex{?? Werk@Nicht ermittelte Verfasserinnen und Verfasser!Todesanzeige von Alois Hofmann]1907-07-13@\emph{[Todesanzeige von Alois Hofmann]} {[}1907-07-13{]}|pwv}}{\lemma{\textnormal{\emph{Anzeige}}}\Cendnote{\textnormal{Die Anzeige erschien am
                     13. 7. 1907 (Jg. 6, Nr. 1723, Morgenblatt) auf
                  S. 12.}}}\label{K_L01691-1h} vom Tod Ihres Vaters\pwindex{Hofmann, Alois 30.3.1830 – 11.7.1907@\textsc{Hofmann, Alois} (30.3.1830 – 11.7.1907), \emph{Industrieller}|pwv}. Gerade um die Stunde, da ich Ihnen dieſe Zeilen ſchreibe, wird er zu
               Grabe getragen. Im Herzen bin ich bei Ihnen und drücke Ihnen die Hand, ſo wie Sie
               wiſſen.\pend
           \pstart
           Sie haben meine Karten wohl erhalten. Hier in \textsc{Welsberg Waldbrunn}\oindex{Wildbad Waldbrunn@\textbf{Wildbad Waldbrunn}|pw} denken wir möglichſt lange zu bleiben, bis Mitte, vielleicht {\pb}Ende Auguſt. Heini\pwindex{Schnitzler, Heinrich 09.08.1902 – 12.07.1982@\textsc{Schnitzler, Heinrich} (09.08.1902 – 12.07.1982), \emph{Regisseur, Schauspieler}|pw} iſt mit uns. Später wollen wir, Olga\pwindex{Schnitzler, Olga 17.01.1882 – 13.01.1970@\textsc{Schnitzler, Olga} (17.01.1882 – 13.01.1970), \emph{Schauspielerin, Sängerin}|pw} u ich, ſüdlicher, Meran\oindex{Meran@\textbf{Meran}|pw}
               vielleicht. Ich hoffe ſehr, daſs der Sommer nicht zu Ende geht, ohne daſs wir
               einander in ſchöner Landſchaft begegnen. Laſſen Sie bald, ſehr bald von ſich hören,
               wär es auch nur ein paar Zeilen. Von Olga\pwindex{Schnitzler, Olga 17.01.1882 – 13.01.1970@\textsc{Schnitzler, Olga} (17.01.1882 – 13.01.1970), \emph{Schauspielerin, Sängerin}|pw} an
               Sie, Paula\pwindex{Beer-Hofmann, Paula 25.02.1879 – 30.10.1939@\textsc{Beer-Hofmann, Paula} (25.02.1879 – 30.10.1939)|pw}, die Kinder\pwindex{Beer-Hofmann, Naemah 20.12.1898 – 10.11.1971@\textsc{Beer-Hofmann, Naëmah} (20.12.1898 – 10.11.1971)|pwv}\pwindex{Beer-Hofmann, Gabriel 09.01.1901 – 24.03.1971@\textsc{Beer-Hofmann, Gabriel} (09.01.1901 – 24.03.1971), \emph{Schriftsteller, Filmagent}|pwv}\pwindex{Beer-Hofmann, Mirjam 04.09.1897 – 24.12.1984@\textsc{Beer-Hofmann, Mirjam} (04.09.1897 – 24.12.1984)|pwv}, eben ſo wie von
               mir, alles herzliche, theilnehmende, gute.\pend
           \pstart
           Ihr{\\[\baselineskip]}\spacefill\mbox{Arthur.}\pend
           \leftskip=0em{}
         
         \endnumbering\mylabel{h}\end{ledgroupsized}  \newcommand{\dateiname}{L01691}\newcommand{\titel}{Arthur Schnitzler an Richard Beer-Hofmann, 14. 7. 1907}\newcommand{\editorInnen}{Martin Anton Müller und Gerd-Hermann Susen}%% latex-leseansicht-abspann.tex
%% Abspann für die Leseansicht.
%% Der Schalter \ifkorrekturansicht ist bereits durch den Vorspann gesetzt.

%% latex-abspann.tex
%% Gemeinsamer Abspann für Korrekturansicht und Leseansicht.
%% Setzt den Schalter \ifkorrekturansicht voraus (gesetzt in den
%% einbindenden Dateien latex-korrekturansicht-abspann.tex bzw.
%% latex-leseansicht-abspann.tex).
%% ---------------------------------------------------------------

\normalsize

% Das esempio-Environment wird nur in der Leseansicht benötigt
\ifkorrekturansicht\else
\newenvironment{esempio}[3]%
{
    \vspace{1.5ex}
    \rlap{\underline{#1}}
    \par
    \setlength{\parindent}{0cm}
    \nopagebreak
    \leftskip=#2cm
    \rightskip=#3cm
}
{
    \par
}
\fi

\doendnotes{C}
\bigskip
\vfill

\clearpage

\footnotesize

\ifkorrekturansicht
  \lohead{\textsc{register}}
\fi

% theindex-Environment neu definieren ohne reledmac
\makeatletter
\renewenvironment{theindex}{%
  \ifkorrekturansicht
    \section*{\indexname}%
  \else
    \subsubsection*{Index der erwähnten Entitäten}%
  \fi
  \setlength{\parindent}{0pt}%
  \setlength{\parskip}{0pt plus 0.3pt}%
  \let\item\@idxitem
}{%
  \ifkorrekturansicht\clearpage\fi
}
\makeatother

\IfFileExists{\jobname-pw.ind}{\input{\jobname-pw.ind}}{}

% Quellenangabe nur in der Leseansicht
\ifkorrekturansicht\else
% Fallback-Definitionen, falls die .tex-Datei \titel etc. nicht gesetzt hat
\providecommand{\titel}{}
\providecommand{\editorInnen}{}
\providecommand{\dateiname}{\jobname}

\vspace{3cm}

\vfill

\footnotesize
\textsc{Quelle}: \titel. Herausgegeben von {\editorInnen}. In: \emph{Arthur Schnitzler: Briefwechsel mit Autorinnen und Autoren}.
 Digitale Edition, https://schnitzler-briefe.acdh.oeaw.ac.at/{\dateiname}.html (Stand \today)
\fi

\end{document}


      