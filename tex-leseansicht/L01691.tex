%% latex-korrekturansicht-vorspann.tex
%% Vorspann für die Korrekturansicht.
%% Lädt die gemeinsame Datei latex-vorspann.tex mit gesetztem Schalter.

\newif\ifkorrekturansicht
\korrekturansichttrue

\input{../tex-inputs/latex-vorspann}


\section[Arthur Schnitzler an Richard Beer-Hofmann, 14. 7. 1907]{L01691 Arthur Schnitzler an Richard Beer-Hofmann, 14. 7. 1907}
\nopagebreak\mylabel{L01691v}
\rehead{ }\normalsize\beginnumbering\briefempfaengerindex{Beer-Hofmann, Richard@\textsc{Beer-Hofmann, Richard}!zzzSchnitzler, Arthur@\emph{von Arthur Schnitzler}!1907-07-141@{14. 7. 1907}|(be}
\toendnotes[C]{\smallbreak\pagebreak[2]}\Standort{YCGL, MSS 31.}
\physDesc{Brief, 1 Blatt, 2 Seiten, Umschlag, 810 Zeichen
\newline{}Handschrift: schwarze Tinte, deutsche Kurrent
\newline{}Versand: Stempel: »\nobreak{}\oindex{Welsberg-Taisten@\textbf{Welsberg-Taisten}, \emph{A.ADM3}|pwk}{[}Wels{]}berg, \textcolor{gray}{15}. 7. 07\nobreak{}«.  
\newline{}Beer-Hofmann: mit blauem Buntstift das Datum der Beantwortung festgehalten:
                                       »B 26/VII 07« }
\buchAbdrucke{\weitereDrucke{Arthur Schnitzler, Richard Beer-Hofmann: \emph{Briefwechsel 1891–1931}. Wien, Zürich: \emph{Europaverlag} 1992, S. 180.} }\toendnotes[C]{\smallbreak}\pstart{}{\pb}\textcolor{gray}{\textbf{Dr. Arthur Schnitzler}}\pend{}\pstart{}\textcolor{gray}{\textbf{Wien XVIII. Spoettelgasse 7\oindex{Edmund-Weiss-Gasse 7@\textbf{Edmund-Weiß-Gasse 7}, \emph{Wohngebäude (K.WHS)}|pw}.}}\pend{}{\bigskip}\pstart{}{\pb}\textsc{Herrn Dr. Richard Beer-Hofmann}\pend{}\pstart{}Wien XVIII\oindex{XVIII., Waehring@\textbf{XVIII., Währing}, \emph{A.ADM3}|pw}\pend{}\pstart{}\textsc{Hasenauerstr.} 59\oindex{Hasenauerstrasse 59@\textbf{Hasenauerstraße 59}, \emph{Wohngebäude (K.WHS)}|pw}.\pend{}{\bigskip}\vspace{1em}
\pstart
           \raggedleft{}{\pb}\textsc{Welsberg-Waldbrunn}\oindex{Wildbad Waldbrunn@\textbf{Wildbad Waldbrunn}, \emph{S.SPA}|pw}, 14. 7. 907\pend
           
\pstart{}mein lieber Richard,\pend\vspace{0.5em}
\pstart
           eben leſe ich in der Zeit\pwindex{Zeit@\emph{Die Zeit}|pw} die \label{K_L01691-1v}\edtext{Anzeige\pwindex{Todesanzeige von Alois Hofmann]@\emph{[Todesanzeige von Alois Hofmann]}|pwv}}{\lemma{\textnormal{\emph{Anzeige}}}\Cendnote{\textnormal{Die Todesanzeige war am
                     13. 7. 1907 (Jg. 6, Nr. 1723, Morgenblatt) auf
                  S. 12 erschienen.}}}\label{K_L01691-1} vom Tod Ihres Vaters\pwindex{Hofmann, Alois 30.3.1830 – 11.7.1907@\textsc{Hofmann, Alois} (30.3.1830 – 11.7.1907), \emph{Industrieller/Industrielle}|pwv}. Gerade um die Stunde, da ich Ihnen dieſe Zeilen ſchreibe, wird er zu
               Grabe getragen. Im Herzen bin ich bei Ihnen und drücke Ihnen die Hand, ſo wie Sie
               wiſſen.\pend
           
\pstart
           Sie haben meine Karten wohl erhalten. Hier in \textsc{Welsberg Waldbrunn}\oindex{Wildbad Waldbrunn@\textbf{Wildbad Waldbrunn}, \emph{S.SPA}|pw} denken wir möglichſt lange zu bleiben, bis Mitte, vielleicht {\pb}Ende Auguſt. Heini\pwindex{Schnitzler, Heinrich 09.08.1902 – 12.07.1982@\textsc{Schnitzler, Heinrich} (09.08.1902 – 12.07.1982), \emph{Regisseur/Regisseurin, Schauspieler/Schauspielerin}|pw} iſt mit uns. Später wollen wir, Olga\pwindex{Schnitzler, Olga 17.01.1882 – 13.01.1970@\textsc{Schnitzler, Olga} (17.01.1882 – 13.01.1970), \emph{Schauspieler/Schauspielerin, Sänger/Sängerin}|pw} u ich, ſüdlicher, Meran\oindex{Meran@\textbf{Meran}, \emph{P.PPLA3}|pw}
               vielleicht. Ich hoffe ſehr, daſs der Sommer nicht zu Ende geht, ohne daſs wir
               einander in ſchöner Landſchaft begegnen. Laſſen Sie bald, ſehr bald von ſich hören,
               wär es auch nur ein paar Zeilen. Von Olga\pwindex{Schnitzler, Olga 17.01.1882 – 13.01.1970@\textsc{Schnitzler, Olga} (17.01.1882 – 13.01.1970), \emph{Schauspieler/Schauspielerin, Sänger/Sängerin}|pw} an
               Sie, Paula\pwindex{Beer-Hofmann, Paula 25.02.1879 – 30.10.1939@\textsc{Beer-Hofmann, Paula} (25.02.1879 – 30.10.1939)|pw}, die Kinder\pwindex{Beer-Hofmann, Naemah 20.12.1898 – 10.11.1971@\textsc{Beer-Hofmann, Naëmah} (20.12.1898 – 10.11.1971)|pwv}\pwindex{Beer-Hofmann, Gabriel 09.01.1901 – 24.03.1971@\textsc{Beer-Hofmann, Gabriel} (09.01.1901 – 24.03.1971), \emph{Schriftsteller/Schriftstellerin, Filmagent/Filmagentin}|pwv}\pwindex{Beer-Hofmann, Mirjam 04.09.1897 – 24.12.1984@\textsc{Beer-Hofmann, Mirjam} (04.09.1897 – 24.12.1984)|pwv}, eben ſo wie von
               mir, alles herzliche, theilnehmende, gute.\pend
           
\pstart
           Ihr{\\[\baselineskip]}\spacefill\mbox{Arthur.}\pend
           \leftskip=0em{}\selectlanguage{ngerman}\endnumbering\briefempfaengerindex{Beer-Hofmann, Richard@\textsc{Beer-Hofmann, Richard}!zzzSchnitzler, Arthur@\emph{von Arthur Schnitzler}!1907-07-141@{14. 7. 1907}|)be}\mylabel{L01691h}  \normalsize

\doendnotes{C}
\bigskip
\vfill

\clearpage

\footnotesize

\lohead{\textsc{register}}

% Definiere theindex-Environment komplett neu ohne reledmac
\makeatletter
\renewenvironment{theindex}{%
  \section*{\indexname}%
  \setlength{\parindent}{0pt}%
  \setlength{\parskip}{0pt plus 0.3pt}%
  \let\item\@idxitem
}{%
  \clearpage
}
\makeatother

\IfFileExists{\jobname-pw.ind}{\input{\jobname-pw.ind}}{}

\end{document}

      