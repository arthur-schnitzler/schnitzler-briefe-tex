%% latex-korrekturansicht-vorspann.tex
%% Vorspann für die Korrekturansicht.
%% Lädt die gemeinsame Datei latex-vorspann.tex mit gesetztem Schalter.

\newif\ifkorrekturansicht
\korrekturansichttrue

\input{../tex-inputs/latex-vorspann}


\section[Arthur Schnitzler an Richard Beer-Hofmann, 7. 2. 1899]{L00885 Arthur Schnitzler an Richard Beer-Hofmann, 7. 2. 1899}
\nopagebreak\mylabel{L00885v}
\rehead{ }\normalsize\beginnumbering\briefempfaengerindex{Beer-Hofmann, Richard@\textsc{Beer-Hofmann, Richard}!zzzSchnitzler, Arthur@\emph{von Arthur Schnitzler}!1899-02-071@{7. 2. 1899}|(be}
\toendnotes[C]{\smallbreak\pagebreak[2]}\Standort{YCGL, MSS 31.}
\physDesc{Briefkarte, , Umschlag, 247 Zeichen
\newline{}Handschrift: Bleistift, deutsche Kurrent
\newline{}Versand: Stempel: »\nobreak{}\oindex{I., Innere Stadt@\textbf{I., Innere Stadt}, \emph{A.ADM3}|pwk}Wien 1/1, {[}7.{]} 2. 99, 10–11 N\nobreak{}«.  }
\buchAbdrucke{\weitereDrucke{Arthur Schnitzler, Richard Beer-Hofmann: \emph{Briefwechsel 1891–1931}. Wien, Zürich: \emph{Europaverlag} 1992, S. 126–127.} }\toendnotes[C]{\smallbreak}\pstart{}{\pb}Herrn \textsc{Dr. Rich
                     Beer-Hofmann}\pend{}\pstart{}Wien\oindex{Wien@\textbf{Wien}, \emph{A.ADM2}|pw}\pend{}\pstart{}\textsc{I. Wollzeile 15\oindex{Wollzeile@\textbf{Wollzeile}, \emph{Straße (K.STR)}|pw}}.\pend{}{\bigskip}\vspace{1em}
\pstart
           \noindent{}{\pb}Lieber Richard, für \label{K_L00885-1v}\edtext{Freitag\pwindex{Unser Kaethchen. Lustspiel in 4 Acten@\emph{Unser Käthchen. Lustspiel in 4 Acten}|pwv}}{\lemma{\textnormal{\emph{Freitag}}}\Cendnote{\textnormal{Am 10. 2. 1899 wurde \emph{Unser Käthchen}\pwindex{Unser Kaethchen. Lustspiel in 4 Acten@\emph{Unser Käthchen. Lustspiel in 4 Acten}|pwk} am \emph{Deutschen
                     Volkstheater}\orgindex{Volkstheater@Volkstheater|pwk} aufgeführt.}}}\label{K_L00885-1}{ }ſind keine ordentlichen Nebeneinander-Sitze mehr zu
               haben. Sie kö{\geminationn}en alſo \label{K_L00885-2v}\edtext{nix ä hin kommen ſtuppen}{\lemma{\textnormal{\emph{nix ä hin kommen ſtuppen}}}\Cendnote{\textnormal{umgangssprachlich: nicht einfach hinkommen, um durch belästigen (›anstuppsen‹) der
                  richtigen Person das Gewünschte erhalten}}}\label{K_L00885-2}. Werden wir noch die Erfindung
               des Teleſtupp erleben?\pend
           \pstart Herzlich Ihr \spacefill\mbox{Arthur}\pend{}
\pstart
           7/2 99\pend
           \selectlanguage{ngerman}\endnumbering\briefempfaengerindex{Beer-Hofmann, Richard@\textsc{Beer-Hofmann, Richard}!zzzSchnitzler, Arthur@\emph{von Arthur Schnitzler}!1899-02-071@{7. 2. 1899}|)be}\mylabel{L00885h}  \normalsize

\doendnotes{C}
\bigskip
\vfill

\clearpage

\footnotesize

\lohead{\textsc{register}}

% Definiere theindex-Environment komplett neu ohne reledmac
\makeatletter
\renewenvironment{theindex}{%
  \section*{\indexname}%
  \setlength{\parindent}{0pt}%
  \setlength{\parskip}{0pt plus 0.3pt}%
  \let\item\@idxitem
}{%
  \clearpage
}
\makeatother

\IfFileExists{\jobname-pw.ind}{\input{\jobname-pw.ind}}{}

\end{document}

      