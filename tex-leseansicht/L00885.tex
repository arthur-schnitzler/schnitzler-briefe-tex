%% latex-leseansicht-vorspann.tex
%% Vorspann für die Leseansicht.
%% Lädt die gemeinsame Datei latex-vorspann.tex mit nicht gesetztem Schalter.

\newif\ifkorrekturansicht
\korrekturansichtfalse

\input{../tex-inputs/latex-vorspann}


               \section[Arthur Schnitzler an Richard Beer-Hofmann, 7. 2. 1899]{ Arthur Schnitzler an Richard Beer-Hofmann, 7. 2. 1899}\nopagebreak\mylabel{v}\rehead{ }\begin{ledgroupsized}[t]{13cm}\normalsize\beginnumbering\briefempfaengerindex{Beer-Hofmann, Richard@\textsc{Beer-Hofmann, Richard}!zzzSchnitzler, Arthur@\emph{von Arthur Schnitzler}!1899-02-071@{7. 2. 1899}|(be} \toendnotes[C]{\smallbreak\pagebreak[2]} \Standort{YCGL, MSS 31.}
\physDesc{Briefkarte, Umschlag
\newline{}Handschrift: Bleistift, deutsche Kurrent\newline{}Versand: Stempel: »\nobreak{}\oindex{I., Innere Stadt@\textbf{I., Innere Stadt}|pwk}Wien 1/1, {[}7.{]} 2. 99, 10–11 N\nobreak{}«.  }\buchAbdrucke{\weitereDrucke{Arthur Schnitzler, Richard Beer-Hofmann: \emph{Briefwechsel 1891–1931}. Hg. Konstanze Fliedl. Wien, Zürich: \emph{Europaverlag} 1992, S. 126–127.} }\toendnotes[C]{\smallbreak}\pstart{}{\pb}Herrn \textsc{Dr. Rich
                     Beer-Hofmann}\pend{}\pstart{}Wien\oindex{Wien@\textbf{Wien}|pw}\pend{}\pstart{}\textsc{I. Wollzeile 15\oindex{Wollzeile@\textbf{Wollzeile}|pw}}.\pend{}{\bigskip}\pstart
           \noindent{}{\pb}Lieber Richard, für \label{K_L00885_1v}\edtext{Freitag\pwindex{\textcolor{red}{\textsuperscript{XXXX1 indx}}!Unser Kaethchen. Lustspiel in 4 Acten1899-02-04@\strich\emph{Unser Käthchen. Lustspiel in 4 Acten} {[}1899-02-04{]}|pwv}}{\lemma{\textnormal{\emph{Freitag}}}\Cendnote{\textnormal{Aufführung von \emph{Unser Käthchen}\pwindex{\textcolor{red}{\textsuperscript{XXXX1 indx}}!Unser Kaethchen. Lustspiel in 4 Acten1899-02-04@\strich\emph{Unser Käthchen. Lustspiel in 4 Acten} {[}1899-02-04{]}|pwk} im Deutschen Volkstheater\oindex{Volkstheater@\textbf{Volkstheater}|pwk}.}}}\label{K_L00885_1h}{ }ſind keine ordentlichen Nebeneinander-Sitze mehr zu
               haben. Sie kö{\geminationn}en alſo \label{K_L00885_2v}\edtext{nix ä hin kommen ſtuppen}{\lemma{\textnormal{\emph{nix ä hin kommen ſtuppen}}}\Cendnote{\textnormal{ugs. für: nicht einfach kommen, um durch Anstuppsen der
                  richtigen Person das Gewünschte erhalten.}}}\label{K_L00885_2h}. Werden wir noch die Erfindung
               des Teleſtupp erleben?\pend
           \pstart Herzlich Ihr \spacefill\mbox{Arthur}\pend{}\pstart
           7/2 99\pend
           \endnumbering\briefempfaengerindex{Beer-Hofmann, Richard@\textsc{Beer-Hofmann, Richard}!zzzSchnitzler, Arthur@\emph{von Arthur Schnitzler}!1899-02-071@{7. 2. 1899}|)be}\mylabel{h}\end{ledgroupsized}  \newcommand{\dateiname}{L00885}\newcommand{\titel}{Arthur Schnitzler an Richard Beer-Hofmann, 7. 2. 1899}\newcommand{\editorInnen}{Martin Anton Müller und Gerd-Hermann Susen}
            \footnotesize
\begin{ledgroupsized}[t]{11.5cm}
\doendnotes{C}
\end{ledgroupsized}
         %% latex-leseansicht-abspann.tex
%% Abspann für die Leseansicht.
%% Der Schalter \ifkorrekturansicht ist bereits durch den Vorspann gesetzt.

%% latex-abspann.tex
%% Gemeinsamer Abspann für Korrekturansicht und Leseansicht.
%% Setzt den Schalter \ifkorrekturansicht voraus (gesetzt in den
%% einbindenden Dateien latex-korrekturansicht-abspann.tex bzw.
%% latex-leseansicht-abspann.tex).
%% ---------------------------------------------------------------

\normalsize

% Das esempio-Environment wird nur in der Leseansicht benötigt
\ifkorrekturansicht\else
\newenvironment{esempio}[3]%
{
    \vspace{1.5ex}
    \rlap{\underline{#1}}
    \par
    \setlength{\parindent}{0cm}
    \nopagebreak
    \leftskip=#2cm
    \rightskip=#3cm
}
{
    \par
}
\fi

\doendnotes{C}
\bigskip
\vfill

\clearpage

\footnotesize

\ifkorrekturansicht
  \lohead{\textsc{register}}
\fi

% theindex-Environment neu definieren ohne reledmac
\makeatletter
\renewenvironment{theindex}{%
  \ifkorrekturansicht
    \section*{\indexname}%
  \else
    \subsubsection*{Index der erwähnten Entitäten}%
  \fi
  \setlength{\parindent}{0pt}%
  \setlength{\parskip}{0pt plus 0.3pt}%
  \let\item\@idxitem
}{%
  \ifkorrekturansicht\clearpage\fi
}
\makeatother

\IfFileExists{\jobname-pw.ind}{\input{\jobname-pw.ind}}{}

% Quellenangabe nur in der Leseansicht
\ifkorrekturansicht\else
% Fallback-Definitionen, falls die .tex-Datei \titel etc. nicht gesetzt hat
\providecommand{\titel}{}
\providecommand{\editorInnen}{}
\providecommand{\dateiname}{\jobname}

\vspace{3cm}

\vfill

\footnotesize
\textsc{Quelle}: \titel. Herausgegeben von {\editorInnen}. In: \emph{Arthur Schnitzler: Briefwechsel mit Autorinnen und Autoren}.
 Digitale Edition, https://schnitzler-briefe.acdh.oeaw.ac.at/{\dateiname}.html (Stand \today)
\fi

\end{document}


      