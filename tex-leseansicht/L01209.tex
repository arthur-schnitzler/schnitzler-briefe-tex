%% latex-korrekturansicht-vorspann.tex
%% Vorspann für die Korrekturansicht.
%% Lädt die gemeinsame Datei latex-vorspann.tex mit gesetztem Schalter.

\newif\ifkorrekturansicht
\korrekturansichttrue

\input{../tex-inputs/latex-vorspann}


\section[Hugo von Hofmannsthal an Arthur Schnitzler, 26. 3. {[}1902{]}]{L01209 Hugo von Hofmannsthal an Arthur Schnitzler, 26. 3. {[}1902{]}}
\nopagebreak\mylabel{L01209v}
\rehead{ }\normalsize\beginnumbering\briefempfaengerindex{Schnitzler, Arthur@\textsc{Schnitzler, Arthur}!zzzHofmannsthal, Hugo von@\emph{von Hugo von Hofmannsthal}!1902-03-261@{26. 3. 1902}|(be}
\toendnotes[C]{\smallbreak\pagebreak[2]}\Standort{CUL, Schnitzler, B 43.}
\physDesc{Brief, 1 Blatt, 3 Seiten, 477 Zeichen
\newline{}Handschrift: schwarze Tinte, deutsche Kurrent
\newline{}Ordnung: 1) mit Bleistift von unbekannter Hand nummeriert: »\strikeout{192}«  2) mit Bleistift von unbekannter Hand nummeriert:
                                    »185«}
\buchAbdrucke{\weitereDrucke{Hugo von Hofmannsthal, Arthur Schnitzler: \emph{Briefwechsel}. Frankfurt am Main: \emph{S. Fischer} 1964, S. 153.} }
\pstart
           \raggedleft{}{\pb}26. III{ }abends.\pend
           \vspace{0.5em}
\pstart
           lieber, wollen Sie nächſten Dinstag, Mittwoch oder Donnerstag mit
               mir, der Gräfin Christiane Thun\pwindex{Thun-Hohenstein-Salm-Reifferscheidt, Christiane von 12.06.1859 – 06.08.1935@\textsc{Thun-Hohenstein-Salm-Reifferscheidt, Christiane von} (12.06.1859 – 06.08.1935), \emph{Schriftsteller/Schriftstellerin}|pw} und Kaſſner\pwindex{Kassner, Rudolf 11.09.1873 – 01.04.1959@\textsc{Kassner, Rudolf} (11.09.1873 – 01.04.1959), \emph{Schriftsteller/Schriftstellerin}|pw} (ſonſt niemand) um 1 Uhr
               frühſtücken, und zwar nicht bei mir, sondern im \textsc{Palais Thun-Salm}\oindex{Palais Thun-Salm@\textbf{Palais Thun-Salm}, \emph{Gebäude (K.GBD)}|pw}, {\pb}\textsc{Kärntnerstrasse 41}.\oindex{Kaerntner Strasse@\textbf{Kärntner Straße}, \emph{Straße (K.STR)}|pw}?\pend
           
\pstart
           Bitte wählen Sie den Tag, der Ihnen am beſten paſst (\uline{mir} wäre Mittwoch der liebſte) und ſchreiben mir \uuline{gleich} eine Zeile.\pend
           
\pstart
           Von Herzen{\\[\baselineskip]}Ihr{\\[\baselineskip]}\spacefill\mbox{Hugo}\pend
           \leftskip=0em{}
\pstart
           \noindent{}P. S. Die 50 fl. für den Hund ſchicken Sie {\pb}am beſten direct per Poſt an
                  Frau Hofräthin von Pollanetz\pwindex{Pollanetz, Malvine von 15.2.1840 – 10.7.1926@\textsc{Pollanetz, Malvine von} (15.2.1840 – 10.7.1926)|pw}, Wien I. Domgaſſe 6\oindex{Domgasse@\textbf{Domgasse}, \emph{Straße (K.STR)}|pw}.\pend
           \selectlanguage{ngerman}\endnumbering\briefempfaengerindex{Schnitzler, Arthur@\textsc{Schnitzler, Arthur}!zzzHofmannsthal, Hugo von@\emph{von Hugo von Hofmannsthal}!1902-03-261@{26. 3. 1902}|)be}\mylabel{L01209h}  \normalsize

\doendnotes{C}
\bigskip
\vfill

\clearpage

\footnotesize

\lohead{\textsc{register}}

% Definiere theindex-Environment komplett neu ohne reledmac
\makeatletter
\renewenvironment{theindex}{%
  \section*{\indexname}%
  \setlength{\parindent}{0pt}%
  \setlength{\parskip}{0pt plus 0.3pt}%
  \let\item\@idxitem
}{%
  \clearpage
}
\makeatother

\IfFileExists{\jobname-pw.ind}{\input{\jobname-pw.ind}}{}

\end{document}

      