%% latex-leseansicht-vorspann.tex
%% Vorspann für die Leseansicht.
%% Lädt die gemeinsame Datei latex-vorspann.tex mit nicht gesetztem Schalter.

\newif\ifkorrekturansicht
\korrekturansichtfalse

\input{../tex-inputs/latex-vorspann}


\section[Hugo von Hofmannsthal an Arthur Schnitzler, 26. 3. [1902]]{L01209 Hugo von Hofmannsthal an Arthur Schnitzler, 26. 3. [1902]}
\nopagebreak\mylabel{L01209v}
\rehead{ }\normalsize\beginnumbering\briefempfaengerindex{Schnitzler, Arthur@\textsc{Schnitzler, Arthur}!zzzHofmannsthal, Hugo von@\emph{von Hugo von Hofmannsthal}!1902-03-261@{26. 3. 1902}|(be}
\toendnotes[C]{\smallbreak\pagebreak[2]}
\correspDesc{Versand  durch Hugo von Hofmannsthal am 26. 3. 1902 \textbf{Ort fehlend} 
\newline{}Erhalt  durch Arthur Schnitzler im Zeitraum [26. 3. 1902
                  – 30. 3. 1902?] in Wien}\toendnotes[C]{\smallbreak}
\Standort{CUL, Schnitzler, B 43.}
\physDesc{Brief, 1 Blatt, 3 Seiten, 477 Zeichen
\newline{}Handschrift: schwarze Tinte, deutsche Kurrent
\newline{}Ordnung: 1) mit Bleistift von unbekannter Hand nummeriert: »\strikeout{192}«  2) mit Bleistift von unbekannter Hand nummeriert:
                                    »185«}
\buchAbdrucke{\weitereDrucke{Hugo von Hofmannsthal, Arthur Schnitzler: \emph{Briefwechsel}. Herausgegeben von Therese Nickl und Heinrich Schnitzler. Frankfurt am Main: \emph{S. Fischer} 1964, S. 153.} }
\pstart
           \raggedleft{}{\pb}26. III{ }abends.\pend
           \vspace{0.5em}
\pstart
           lieber, wollen Sie nächſten Dinstag, Mittwoch oder Donnerstag mit
               mir, der Gräfin Christiane Thun\pwindex{Thun-Hohenstein-Salm-Reifferscheidt, Christiane von 12.\,6.\,1859 Doksy – 6.\,8.\,1935 Prag@\textsc{Thun-Hohenstein-Salm-Reifferscheidt, Christiane von} (12.\,6.\,1859 Doksy – 6.\,8.\,1935 Prag), \emph{Schriftstellerin}|pw} und Kaſſner\pwindex{Kassner, Rudolf 11.\,9.\,1873 Velké Pavlovice – 1.\,4.\,1959 Sierre@\textsc{Kassner, Rudolf} (11.\,9.\,1873 Velké Pavlovice – 1.\,4.\,1959 Sierre), \emph{Schriftsteller}|pw} (ſonſt niemand) um 1 Uhr
               frühſtücken, und zwar nicht bei mir, sondern im \textsc{Palais Thun-Salm}\oindex{Wien@\textbf{Wien}!I., Innere Stadt@\textbf{I., Innere Stadt}!Palais Thun-Salm@\textbf{Palais Thun-Salm}, \emph{Gebäude}|pw}, {\pb}\textsc{Kärntnerstrasse 41}.\oindex{Wien@\textbf{Wien}!I., Innere Stadt@\textbf{I., Innere Stadt}!Kärntner Straße@\textbf{Kärntner Straße}, \emph{Straße}|pw}?\pend
           
\pstart
           Bitte wählen Sie den Tag, der Ihnen am beſten paſst (\uline{mir} wäre Mittwoch der liebſte) und{ }ſchreiben mir \uuline{gleich} eine Zeile.\pend
           
\pstart
           Von Herzen{\\[\baselineskip]}Ihr{\\[\baselineskip]}\spacefill\mbox{Hugo}\pend
           \leftskip=0em{}
\pstart
           \noindent{}P. S. Die 50 fl. für den Hund{ }ſchicken Sie {\pb}am beſten direct per Poſt an
                  Frau Hofräthin von Pollanetz\pwindex{Pollanetz, Malvine von 15.\,2.\,1840 Wien – 10.\,7.\,1926 Rodaun@\textsc{Pollanetz, Malvine von} (15.\,2.\,1840 Wien – 10.\,7.\,1926 Rodaun)|pw}, Wien I. Domgaſſe 6\oindex{Wien@\textbf{Wien}!I., Innere Stadt@\textbf{I., Innere Stadt}!Domgasse@\textbf{Domgasse}, \emph{Straße}|pw}.\pend
           \selectlanguage{ngerman}\endnumbering\briefempfaengerindex{Schnitzler, Arthur@\textsc{Schnitzler, Arthur}!zzzHofmannsthal, Hugo von@\emph{von Hugo von Hofmannsthal}!1902-03-261@{26. 3. 1902}|)be}\mylabel{L01209h}  \newcommand{\dateiname}{L01209}\newcommand{\titel}{Hugo von Hofmannsthal an Arthur Schnitzler, 26. 3. [1902]}\newcommand{\editorInnen}{Martin Anton Müller und Gerd-Hermann Susen}%% latex-leseansicht-abspann.tex
%% Abspann für die Leseansicht.
%% Der Schalter \ifkorrekturansicht ist bereits durch den Vorspann gesetzt.

%% latex-abspann.tex
%% Gemeinsamer Abspann für Korrekturansicht und Leseansicht.
%% Setzt den Schalter \ifkorrekturansicht voraus (gesetzt in den
%% einbindenden Dateien latex-korrekturansicht-abspann.tex bzw.
%% latex-leseansicht-abspann.tex).
%% ---------------------------------------------------------------

\normalsize

% Das esempio-Environment wird nur in der Leseansicht benötigt
\ifkorrekturansicht\else
\newenvironment{esempio}[3]%
{
    \vspace{1.5ex}
    \rlap{\underline{#1}}
    \par
    \setlength{\parindent}{0cm}
    \nopagebreak
    \leftskip=#2cm
    \rightskip=#3cm
}
{
    \par
}
\fi

\doendnotes{C}
\bigskip
\vfill

\clearpage

\footnotesize

\ifkorrekturansicht
  \lohead{\textsc{register}}
\fi

% theindex-Environment neu definieren ohne reledmac
\makeatletter
\renewenvironment{theindex}{%
  \ifkorrekturansicht
    \section*{\indexname}%
  \else
    \subsubsection*{Index der erwähnten Entitäten}%
  \fi
  \setlength{\parindent}{0pt}%
  \setlength{\parskip}{0pt plus 0.3pt}%
  \let\item\@idxitem
}{%
  \ifkorrekturansicht\clearpage\fi
}
\makeatother

\IfFileExists{\jobname-pw.ind}{\input{\jobname-pw.ind}}{}

% Quellenangabe nur in der Leseansicht
\ifkorrekturansicht\else
% Fallback-Definitionen, falls die .tex-Datei \titel etc. nicht gesetzt hat
\providecommand{\titel}{}
\providecommand{\editorInnen}{}
\providecommand{\dateiname}{\jobname}

\vspace{3cm}

\vfill

\footnotesize
\textsc{Quelle}: \titel. Herausgegeben von {\editorInnen}. In: \emph{Arthur Schnitzler: Briefwechsel mit Autorinnen und Autoren}.
 Digitale Edition, https://schnitzler-briefe.acdh.oeaw.ac.at/{\dateiname}.html (Stand \today)
\fi

\end{document}


