%% latex-leseansicht-vorspann.tex
%% Vorspann für die Leseansicht.
%% Lädt die gemeinsame Datei latex-vorspann.tex mit nicht gesetztem Schalter.

\newif\ifkorrekturansicht
\korrekturansichtfalse

\input{../tex-inputs/latex-vorspann}


\section[Arthur Schnitzler an Gustav Schwarzkopf, 15. 2. 1909]{L04143 Arthur Schnitzler an Gustav Schwarzkopf, 15. 2. 1909}
\nopagebreak\mylabel{L04143v}
\rehead{ }\normalsize\beginnumbering\briefempfaengerindex{Schwarzkopf, Gustav@\textsc{Schwarzkopf, Gustav}!zzzSchnitzler, Arthur@\emph{von Arthur Schnitzler}!1909-02-151@{15. 2. 1909}|(be}
\toendnotes[C]{\smallbreak\pagebreak[2]}
\correspDesc{Versand  durch Arthur Schnitzler am 15. 2. 1909 in Wien
\newline{}Erhalt  durch Gustav Schwarzkopf im Zeitraum [15. 2. 1909 – 18. 2. 1909?] in Wien}\toendnotes[C]{\smallbreak}
\Standort{CUL, Schnitzler, B 96.}
\physDesc{Brief, 1 Blatt, 1 Seite, 110 Zeichen
\newline{}Handschrift: Bleistift, deutsche Kurrent}\toendnotes[C]{\smallbreak}
\pstart
           {\pb}\textcolor{gray}{\textbf{Dr Arthur Schnitzler}}\hfill 15. 2. 09\pend
           
\pstart
           \textcolor{gray}{\textbf{Wien XVIII.
                        Spoettelgasse 7\oindex{Wien@\textbf{Wien}!XVIII., Währing@\textbf{XVIII., Währing}!Edmund-Weiß-Gasse@\textbf{Edmund-Weiß-Gasse}, \emph{Straße}|pw}.}}\pend
           
\pstart{}lieber Guſtav,\pend\vspace{0.5em}
\pstart
           \label{K_L04143-1v}\edtext{Olga\pwindex{Schnitzler, Olga 17.\,1.\,1882 Wien – 13.\,1.\,1970 Lugano@\textsc{Schnitzler, Olga} (17.\,1.\,1882 Wien – 13.\,1.\,1970 Lugano), \emph{Schauspielerin, Sängerin}|pw} ka{\geminationn} nicht ausgehn}{\lemma{\textnormal{\emph{Olga kann nicht ausgehn}}}\Cendnote{\textnormal{Vgl. A. S.: \emph{Tagebuch}, 14. 2. 1909.}}}\label{K_L04143-1}; hier ihr \label{K_L04143-2v}\edtext{Sitz\eventindex{Theater in der Josefstadt@\textbf{Theater in der Josefstadt}!Aufführung von Die junge Frau, 15.2.1909@Aufführung von Die junge Frau, 15.2.1909|pwv}}{\lemma{\textnormal{\emph{Sitz}}}\Cendnote{\textnormal{Für die 
                   Aufführung von \emph{Die junge Frau}\pwindex{Guinon, Albert 15.\,4.\,1861 Paris – 7.\,3.\,1923 ebd.@\textsc{Guinon, Albert} (15.\,4.\,1861 Paris – 7.\,3.\,1923 ebd.), \emph{Schriftsteller}!junge Frau@\strich\emph{Die junge Frau}|pwk} von Albert Guinon\pwindex{Guinon, Albert 15.\,4.\,1861 Paris – 7.\,3.\,1923 ebd.@\textsc{Guinon, Albert} (15.\,4.\,1861 Paris – 7.\,3.\,1923 ebd.), \emph{Schriftsteller}|pwk}\eventindex{Theater in der Josefstadt@\textbf{Theater in der Josefstadt}!Aufführung von Die junge Frau, 15.2.1909@Aufführung von Die junge Frau, 15.2.1909|pwk} im Theater in der Josefstadt\oindex{Wien@\textbf{Wien}!VIII., Josefstadt@\textbf{VIII., Josefstadt}!Theater in der Josefstadt@\textbf{Theater in der Josefstadt}, \emph{Theater}|pwk}. Schwarzkopf\pwindex{Schwarzkopf, Gustav 7.\,11.\,1853 Wien – 13.\,11.\,1939 ebd.@\textsc{Schwarzkopf, Gustav} (7.\,11.\,1853 Wien – 13.\,11.\,1939 ebd.), \emph{Schriftsteller}|pwk} nahm die Einladung an.}}}\label{K_L04143-2} auf dem ich Sie Abends vorzufinden
               hoffe\pend
           
\pstart
           Herzlich{\\[\baselineskip]} Ihr{\\[\baselineskip]}\spacefill\mbox{A.}\pend
           \leftskip=0em{}\selectlanguage{ngerman}\endnumbering\briefempfaengerindex{Schwarzkopf, Gustav@\textsc{Schwarzkopf, Gustav}!zzzSchnitzler, Arthur@\emph{von Arthur Schnitzler}!1909-02-151@{15. 2. 1909}|)be}\mylabel{L04143h}
\begin{anhang}
\end{anhang}\newcommand{\dateiname}{L04143}\newcommand{\titel}{Arthur Schnitzler an Gustav Schwarzkopf, 15. 2. 1909}\newcommand{\editorInnen}{Herausgegeben von Jahnke, SelmaMüller, Martin Anton}%% latex-leseansicht-abspann.tex
%% Abspann für die Leseansicht.
%% Der Schalter \ifkorrekturansicht ist bereits durch den Vorspann gesetzt.

%% latex-abspann.tex
%% Gemeinsamer Abspann für Korrekturansicht und Leseansicht.
%% Setzt den Schalter \ifkorrekturansicht voraus (gesetzt in den
%% einbindenden Dateien latex-korrekturansicht-abspann.tex bzw.
%% latex-leseansicht-abspann.tex).
%% ---------------------------------------------------------------

\normalsize

% Das esempio-Environment wird nur in der Leseansicht benötigt
\ifkorrekturansicht\else
\newenvironment{esempio}[3]%
{
    \vspace{1.5ex}
    \rlap{\underline{#1}}
    \par
    \setlength{\parindent}{0cm}
    \nopagebreak
    \leftskip=#2cm
    \rightskip=#3cm
}
{
    \par
}
\fi

\doendnotes{C}
\bigskip
\vfill

\clearpage

\footnotesize

\ifkorrekturansicht
  \lohead{\textsc{register}}
\fi

% theindex-Environment neu definieren ohne reledmac
\makeatletter
\renewenvironment{theindex}{%
  \ifkorrekturansicht
    \section*{\indexname}%
  \else
    \subsubsection*{Index der erwähnten Entitäten}%
  \fi
  \setlength{\parindent}{0pt}%
  \setlength{\parskip}{0pt plus 0.3pt}%
  \let\item\@idxitem
}{%
  \ifkorrekturansicht\clearpage\fi
}
\makeatother

\IfFileExists{\jobname-pw.ind}{\input{\jobname-pw.ind}}{}

% Quellenangabe nur in der Leseansicht
\ifkorrekturansicht\else
% Fallback-Definitionen, falls die .tex-Datei \titel etc. nicht gesetzt hat
\providecommand{\titel}{}
\providecommand{\editorInnen}{}
\providecommand{\dateiname}{\jobname}

\vspace{3cm}

\vfill

\footnotesize
\textsc{Quelle}: \titel. Herausgegeben von {\editorInnen}. In: \emph{Arthur Schnitzler: Briefwechsel mit Autorinnen und Autoren}.
 Digitale Edition, https://schnitzler-briefe.acdh.oeaw.ac.at/{\dateiname}.html (Stand \today)
\fi

\end{document}


