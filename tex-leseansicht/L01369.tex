%% latex-korrekturansicht-vorspann.tex
%% Vorspann für die Korrekturansicht.
%% Lädt die gemeinsame Datei latex-vorspann.tex mit gesetztem Schalter.

\newif\ifkorrekturansicht
\korrekturansichttrue

\input{../tex-inputs/latex-vorspann}


\section[Hugo von Hofmannsthal an Arthur Schnitzler, 1. 2. 1904]{L01369 Hugo von Hofmannsthal an Arthur Schnitzler, 1. 2. 1904}
\nopagebreak\mylabel{L01369v}
\rehead{ }\normalsize\beginnumbering\briefempfaengerindex{Schnitzler, Arthur@\textsc{Schnitzler, Arthur}!zzzHofmannsthal, Hugo von@\emph{von Hugo von Hofmannsthal}!1904-02-012@{1. 2. 1904}|(be}
\toendnotes[C]{\smallbreak\pagebreak[2]}\Standort{CUL, Schnitzler, B 43.}
\physDesc{Postkarte, 356 Zeichen
\newline{}Handschrift: schwarze Tinte, deutsche Kurrent
\newline{}Versand: 1) Stempel: »\nobreak{}\oindex{Rodaun@\textbf{Rodaun}, \emph{A.ADM4}|pwk}Rodaun, 1 2 {[}1904{]}, 9–12N\nobreak{}«.   2) Stempel: »\nobreak{}\oindex{XVIII., Waehring@\textbf{XVIII., Währing}, \emph{A.ADM3}|pwk}18/1 Wien 110, 2 2 04, 8.V\nobreak{}«. 
\newline{}Ordnung: 1) mit Bleistift von unbekannter Hand nummeriert:
                                    »214«  2) mit Bleistift von unbekannter Hand nummeriert:
                                    »214«}
\buchAbdrucke{\weitereDrucke{Hugo von Hofmannsthal, Arthur Schnitzler: \emph{Briefwechsel}. Frankfurt am Main: \emph{S. Fischer} 1964, S. 182.} }\toendnotes[C]{\smallbreak}\pstart{}{\pb}\textsc{Herrn D\textsuperscript{r} Arthur Schnitzler}\pend{}\pstart{}\textsc{Wien}\oindex{Wien@\textbf{Wien}, \emph{A.ADM2}|pw}\pend{}\pstart{}\textsc{\damage{\textcolor{gray}{XV}}III Spöttelgasse 7}\oindex{Edmund-Weiss-Gasse 7@\textbf{Edmund-Weiß-Gasse 7}, \emph{Wohngebäude (K.WHS)}|pw}\pend{}\pstart{}neben \textsc{Türkenschanzstrasse\oindex{Tuerkenschanzstrasse@\textbf{Türkenschanzstraße}, \emph{Straße (K.STR)}|pw}}\pend{}{\bigskip}\vspace{1em}
\pstart{}{\pb}lieber, \pend\vspace{0.5em}
\pstart
           Edgar Karg\pwindex{Karg-Bebenburg, Edgar von 22.12.1872 – 23.06.1905@\textsc{Karg-Bebenburg, Edgar von} (22.12.1872 – 23.06.1905), \emph{Militär/Militärin}|pw}, der Marineur, hat Sie ſehr gern und
               möchte Sie ſehr gern wieder ſehen und auch Ihre Frau\pwindex{Schnitzler, Olga 17.01.1882 – 13.01.1970@\textsc{Schnitzler, Olga} (17.01.1882 – 13.01.1970), \emph{Schauspieler/Schauspielerin, Sänger/Sängerin}|pwv} kennen. Da Ihr nun nie zu uns kommt und diesmal wieder
               abgeſagt habt, ſo habe ich ihn für morgen zum Nachtmahl in die Spöttelgaſſe N\textsuperscript{o} 7\oindex{Edmund-Weiss-Gasse 7@\textbf{Edmund-Weiß-Gasse 7}, \emph{Wohngebäude (K.WHS)}|pw} eingeladen.\pend
           
\pstart
           Von Herzen Ihr\pend
           \pstart \spacefill\mbox{Hugo}\pend{}
\pstart
           \noindent{}Rodaun\oindex{Rodaun@\textbf{Rodaun}, \emph{A.ADM4}|pw}{ }Montag.\pend
           \selectlanguage{ngerman}\endnumbering\briefempfaengerindex{Schnitzler, Arthur@\textsc{Schnitzler, Arthur}!zzzHofmannsthal, Hugo von@\emph{von Hugo von Hofmannsthal}!1904-02-012@{1. 2. 1904}|)be}\mylabel{L01369h}  \normalsize

\doendnotes{C}
\bigskip
\vfill

\clearpage

\footnotesize

\lohead{\textsc{register}}

% Definiere theindex-Environment komplett neu ohne reledmac
\makeatletter
\renewenvironment{theindex}{%
  \section*{\indexname}%
  \setlength{\parindent}{0pt}%
  \setlength{\parskip}{0pt plus 0.3pt}%
  \let\item\@idxitem
}{%
  \clearpage
}
\makeatother

\IfFileExists{\jobname-pw.ind}{\input{\jobname-pw.ind}}{}

\end{document}

      