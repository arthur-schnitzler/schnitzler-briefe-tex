\input{../tex-inputs/latex-pdf-vorspann}
\begin{center}
            \textcolor{red}{ENTWURF. ENTZIFFERUNG NOCH NICHT KORREKTURGELESEN}
                      \end{center}
            
               \section[Hermann Bahr: Widmungsexemplar Die Rahl für Arthur Schnitzler, {[}17.–21. 11.{]} 1908]{ Hermann Bahr: Widmungsexemplar Die Rahl für Arthur Schnitzler,
               {[}17.–21. 11.{]} 1908}\nopagebreak\mylabel{v}\rehead{ }\begin{ledgroupsized}[t]{13cm}\normalsize\beginnumbering\briefempfaengerindex{Schnitzler, Arthur@\textsc{Schnitzler, Arthur}!zzzBahr, Hermann@\emph{von Hermann Bahr}!1908-11-171@{{[}17.–21. 11.{]} 1908}|(be} \toendnotes[C]{\smallbreak\pagebreak[2]} \Standort{DLA, G:Schnitzler, Arthur (Sammlung Heinrich Schnitzler).}
\physDesc{Widmung am Vorsatzblatt
\newline{}Handschrift: schwarze Tinte, deutsche Kurrent\newline{}Ordnung: bei der Enteignung des Exemplars 1938 von
                                 unbekannter Hand mit Bleistift ergänzte Informationen: »\noindent{}\uline{Erstausgabe: 0}{ / }Handſchriftliche Widmung d. Verf.« }\buchAbdrucke{\weitereDrucke{Hermann Bahr, Arthur Schnitzler: \emph{Briefwechsel, Aufzeichnungen, Dokumente (1891–1931)}. Hg. Kurt Ifkovits und Martin Anton Müller. Göttingen: \emph{Wallstein} 2018, S. 410.} }\pstart
           \noindent{}{\pb}Meinem lieben Arthur\pend
           \pstart
           herzlichſt{\\[\baselineskip]}\spacefill\mbox{HermannB}\pend
           \leftskip=0em{}\pstart
           \noindent{}1908\pend
           {\bigskip}\pstart
           \noindent{}\centering{}{\pb}\textcolor{gray}{\textbf{\textbf{Die Rahl}\pwindex{Bahr, Hermann 19.07.1863 – 15.01.1934@\textsc{Bahr, Hermann} (19.07.1863 – 15.01.1934), \emph{Schriftsteller, Kritiker}!Rahl. Roman1908@\strich\emph{Die Rahl. Roman} {[}1908{]}|pw}}}\pend
           \pstart
           \noindent{}\centering{}\textcolor{gray}{\textbf{Roman}}\pend
           \pstart
           \noindent{}\centering{}\textcolor{gray}{\textbf{von}}\pend
           \pstart
           \noindent{}\centering{}\textcolor{gray}{\textbf{Hermann Bahr}}\pend
           {\bigskip}\pstart
           \noindent{}\centering{}\textcolor{gray}{\textbf{S. Fiſcher, Verlag\orgindex{S. Fischer Verlag@S. Fischer Verlag|pw}, Berlin\oindex{Berlin@\textbf{Berlin}|pw}}}\pend
           \pstart
           \noindent{}\centering{}\textcolor{gray}{\textbf{1909}}\pend
           \endnumbering\briefempfaengerindex{Schnitzler, Arthur@\textsc{Schnitzler, Arthur}!zzzBahr, Hermann@\emph{von Hermann Bahr}!1908-11-171@{{[}17.–21. 11.{]} 1908}|)be}\mylabel{h}\end{ledgroupsized}  \newcommand{\dateiname}{L01804}\newcommand{\titel}{Hermann Bahr: Widmungsexemplar Die Rahl für Arthur Schnitzler, [17.–21. 11.] 1908}\newcommand{\editorInnen}{ Kurt Ifkovits,  Martin Anton Müller}\input{../tex-inputs/latex-pdf-abspann}
      