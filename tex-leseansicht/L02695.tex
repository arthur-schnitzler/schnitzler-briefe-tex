%% latex-leseansicht-vorspann.tex
%% Vorspann für die Leseansicht.
%% Lädt die gemeinsame Datei latex-vorspann.tex mit nicht gesetztem Schalter.

\newif\ifkorrekturansicht
\korrekturansichtfalse

\input{../tex-inputs/latex-vorspann}

\begin{center}
            \textcolor{red}{ENTWURF. ENTZIFFERUNG NOCH NICHT KORREKTURGELESEN}
                      \end{center}
            
               \section[Paul Goldmann an Arthur Schnitzler, 3. 12. {[}1893?{]}]{ Paul Goldmann an Arthur Schnitzler, 3. 12. {[}1893?{]}}\nopagebreak\mylabel{v}\rehead{ }\begin{ledgroupsized}[t]{13cm}\normalsize\beginnumbering\briefempfaengerindex{Schnitzler, Arthur@\textsc{Schnitzler, Arthur}!zzzGoldmann, Paul@\emph{von Paul Goldmann}!1893-12-031@{3. 12. {[}1893?{]}}|(be} \toendnotes[C]{\smallbreak\pagebreak[2]} \Standort{DLA, A:Schnitzler, HS.NZ85.1.3163.}
\physDesc{Telegramm
\newline{}maschinell\newline{}Versand: von unbekannter Hand mit Bleistift Vermerk: »\textcolor{gray}{\textbf{mit}} 27 \textcolor{gray}{\textbf{Taxworten}}« \newline{}Ordnung: beschnitten }\toendnotes[C]{\smallbreak}\pstart
           \centering{}{\pb}w\oindex{Wien@\textbf{Wien}|pw}{ }paris\oindex{Paris@\textbf{Paris}|pw} 5298 4\label{K_L02695-1v}\edtext{3 12}{\lemma{\textnormal{\emph{3 12}}}\Cendnote{\textnormal{Trotz des verschobenen Leerzeichens
                     in der Vorlage findet sich hier die Datumsangabe des Telegramms.}}}\label{K_L02695-1h}{ }17\pend
           \pstart
           tausend herzliche glueckwuensche fuer zwei ersten \label{K_L02695-2v}\edtext{acte\pwindex{Schnitzler, Arthur 15.05.1862 – 21.10.1931@\textsc{Schnitzler, Arthur} (15.05.1862 – 21.10.1931), \emph{Schriftsteller, Mediziner}!Maerchen. Schauspiel in drei Aufzuegen1891 – 1891@\strich\emph{Das Märchen. Schauspiel in drei Aufzügen} {[}1891 – 1891{]}|pwv}}{\lemma{\textnormal{\emph{acte}}}\Cendnote{\textnormal{\emph{Das Märchen}\pwindex{Schnitzler, Arthur 15.05.1862 – 21.10.1931@\textsc{Schnitzler, Arthur} (15.05.1862 – 21.10.1931), \emph{Schriftsteller, Mediziner}!Maerchen. Schauspiel in drei Aufzuegen1891 – 1891@\strich\emph{Das Märchen. Schauspiel in drei Aufzügen} {[}1891 – 1891{]}|pwk} hatte am 1. 12. 1893 Uraufführung am \emph{Deutschen Volkstheater}\orgindex{Volkstheater@Volkstheater|pwk} in Wien\oindex{Wien@\textbf{Wien}|pwk}. Die Kritik an dem abfallenden dritten Akt\pwindex{Schnitzler, Arthur 15.05.1862 – 21.10.1931@\textsc{Schnitzler, Arthur} (15.05.1862 – 21.10.1931), \emph{Schriftsteller, Mediziner}!Maerchen. Schauspiel in drei Aufzuegen1891 – 1891@\strich\emph{Das Märchen. Schauspiel in drei Aufzügen} {[}1891 – 1891{]}|pwkv} notierte sich Schnitzler\pwindex{Schnitzler, Arthur 15.05.1862 – 21.10.1931@\textsc{Schnitzler, Arthur} (15.05.1862 – 21.10.1931), \emph{Schriftsteller, Mediziner}|pwk} im \emph{Tagebuch}\pwindex{Schnitzler, Arthur 15.05.1862 – 21.10.1931@\textsc{Schnitzler, Arthur} (15.05.1862 – 21.10.1931), \emph{Schriftsteller, Mediziner}!Tagebuch1981 – 2000@\strich\emph{Tagebuch} {[}1981 – 2000{]}|pwk}
                     (2. 12. 1893). Er
                  kürzte ihn für die zweite Vorstellung am selben Tag. Trotzdem wurde das Stück\pwindex{Schnitzler, Arthur 15.05.1862 – 21.10.1931@\textsc{Schnitzler, Arthur} (15.05.1862 – 21.10.1931), \emph{Schriftsteller, Mediziner}!Maerchen. Schauspiel in drei Aufzuegen1891 – 1891@\strich\emph{Das Märchen. Schauspiel in drei Aufzügen} {[}1891 – 1891{]}|pwkv} nach dieser Vorstellung abgesetzt. In Druck erschien dann im
                  folgenden Mai ein geänderter Schluss (\emph{E. Pierson}\orgindex{E. Pierson s Verlag@E. Pierson’s Verlag|pwk}{ }1894), der 1902 für die 2. Auflage neuerlich
                  abgeändert wurde (\emph{S. Fischer Verlag}\orgindex{S. Fischer Verlag@S. Fischer Verlag|pwk}). }}}\label{K_L02695-2h} lass die dummen buben schrejben wohl dem welchem zum vollendeten
                  \label{T_L02695-1v}\edtext{dramatiker}{\lemma{\textnormal{\emph{dramatiker}}}\Cendnote{\textnormal{in der Vorlage: »dramatiken«}}}\label{T_L02695-1h} nur noch
               ein \label{T_L02695-2v}\edtext{dritter}{\lemma{\textnormal{\emph{dritter}}}\Cendnote{\textnormal{in der Vorlage: »dritten«}}}\label{T_L02695-2h} act fehlt
               jetzt geht es \label{T_L02695-3v}\edtext{unaufhaltsam}{\lemma{\textnormal{\emph{unaufhaltsam}}}\Cendnote{\textnormal{in der Vorlage:
                  »unanfhaltsam«}}}\label{T_L02695-3h} hinauf bitte schicke mir alle kritiken\pend
           \pstart dank gruesze + \spacefill\mbox{goldmann}\pend{}\endnumbering\briefempfaengerindex{Schnitzler, Arthur@\textsc{Schnitzler, Arthur}!zzzGoldmann, Paul@\emph{von Paul Goldmann}!1893-12-031@{3. 12. {[}1893?{]}}|)be}\mylabel{h}\end{ledgroupsized}\begin{anhang}\end{anhang}\newcommand{\dateiname}{L02695}\newcommand{\titel}{Paul Goldmann an Arthur Schnitzler, 3. 12. [1893?]}\newcommand{\editorInnen}{Martin Anton Müller und Laura Untner}%% latex-leseansicht-abspann.tex
%% Abspann für die Leseansicht.
%% Der Schalter \ifkorrekturansicht ist bereits durch den Vorspann gesetzt.

%% latex-abspann.tex
%% Gemeinsamer Abspann für Korrekturansicht und Leseansicht.
%% Setzt den Schalter \ifkorrekturansicht voraus (gesetzt in den
%% einbindenden Dateien latex-korrekturansicht-abspann.tex bzw.
%% latex-leseansicht-abspann.tex).
%% ---------------------------------------------------------------

\normalsize

% Das esempio-Environment wird nur in der Leseansicht benötigt
\ifkorrekturansicht\else
\newenvironment{esempio}[3]%
{
    \vspace{1.5ex}
    \rlap{\underline{#1}}
    \par
    \setlength{\parindent}{0cm}
    \nopagebreak
    \leftskip=#2cm
    \rightskip=#3cm
}
{
    \par
}
\fi

\doendnotes{C}
\bigskip
\vfill

\clearpage

\footnotesize

\ifkorrekturansicht
  \lohead{\textsc{register}}
\fi

% theindex-Environment neu definieren ohne reledmac
\makeatletter
\renewenvironment{theindex}{%
  \ifkorrekturansicht
    \section*{\indexname}%
  \else
    \subsubsection*{Index der erwähnten Entitäten}%
  \fi
  \setlength{\parindent}{0pt}%
  \setlength{\parskip}{0pt plus 0.3pt}%
  \let\item\@idxitem
}{%
  \ifkorrekturansicht\clearpage\fi
}
\makeatother

\IfFileExists{\jobname-pw.ind}{\input{\jobname-pw.ind}}{}

% Quellenangabe nur in der Leseansicht
\ifkorrekturansicht\else
% Fallback-Definitionen, falls die .tex-Datei \titel etc. nicht gesetzt hat
\providecommand{\titel}{}
\providecommand{\editorInnen}{}
\providecommand{\dateiname}{\jobname}

\vspace{3cm}

\vfill

\footnotesize
\textsc{Quelle}: \titel. Herausgegeben von {\editorInnen}. In: \emph{Arthur Schnitzler: Briefwechsel mit Autorinnen und Autoren}.
 Digitale Edition, https://schnitzler-briefe.acdh.oeaw.ac.at/{\dateiname}.html (Stand \today)
\fi

\end{document}


      