%% latex-leseansicht-vorspann.tex
%% Vorspann für die Leseansicht.
%% Lädt die gemeinsame Datei latex-vorspann.tex mit nicht gesetztem Schalter.

\newif\ifkorrekturansicht
\korrekturansichtfalse

\input{../tex-inputs/latex-vorspann}


\section[Albert Ehrenstein an Arthur Schnitzler, 15. 1. 1909]{L01822 Albert Ehrenstein an Arthur Schnitzler, 15. 1. 1909}
\nopagebreak\mylabel{L01822v}
\rehead{ }\normalsize\beginnumbering\briefempfaengerindex{Schnitzler, Arthur@\textsc{Schnitzler, Arthur}!zzzEhrenstein, Albert@\emph{von Albert Ehrenstein}!1909-01-152@{15. 1. 1909}|(be}
\toendnotes[C]{\smallbreak\pagebreak[2]}
\correspDesc{Versand  durch Albert Ehrenstein am 15. 1. 1909 \textbf{Ort fehlend} 
\newline{}Erhalt  durch Arthur Schnitzler im Zeitraum [15. 1. 1909
                  – 19. 1. 1909?] in Wien}\toendnotes[C]{\smallbreak}
\Standort{CUL, Schnitzler, B 30.}
\physDesc{Brief, 1 Blatt, 2 Seiten, 1104 Zeichen
\newline{}Handschrift: schwarze Tinte, deutsche Kurrent
\newline{}Schnitzler: mit Bleistift beschriftet: »\textsc{Ehrenstein}« }
\buchAbdrucke{\weitereDrucke{Albert Ehrenstein: \emph{Briefe}. Herausgegeben von Hanni Mittelmann. München: \emph{Boer} 1989, S. 24 (Werke, 1).} }\toendnotes[C]{\smallbreak}
\pstart
           \raggedleft{}{\pb}15. Jan. 09.\pend
           
\pstart{}\textsc{Sehr geehrter Herr Doktor!}\pend\vspace{0.5em}
\pstart
           Die hiſtoriſche Novellette\pwindex{Ehrenstein, Albert 23.\,12.\,1886 Wien – 8.\,4.\,1950 New York City@\textsc{Ehrenstein, Albert} (23.\,12.\,1886 Wien – 8.\,4.\,1950 New York City), \emph{Schriftsteller}!Tod des Zehir eddin Muhammed Baber@\strich\emph{Tod des Zehir eddin Muhammed Baber}|pwuv} zu{ }ſchreiben, von der ich das letztemal Ihnen,{ }ſehr geehrter Herr
               Doktor,{ }ſprach, iſt mir vorläufig mißlungen. Die Langeweile, welche mir die
               Beſchäftigung mit ihr verurſachte, war{ }ſo enorm, daß ich mich nicht dazu haben konnte{ }ſie zu vollenden, trotzdem der bereits von heftigem Fieber gequälte Held nur noch
               binnen drei Seiten zu{ }ſterben hatte. Glücklicherweiſe \label{K_L01822-1v}\edtext{träumte mir im vorigen Monat}{\lemma{\textnormal{\emph{träumte … Monat}}}\Cendnote{\textnormal{Am 7. 12. 1908, vgl. Ehrenstein\pwindex{Ehrenstein, Albert 23.\,12.\,1886 Wien – 8.\,4.\,1950 New York City@\textsc{Ehrenstein, Albert} (23.\,12.\,1886 Wien – 8.\,4.\,1950 New York City), \emph{Schriftsteller}|pwk}: \emph{Briefe},
                  S. 24.}}}\label{K_L01822-1} ein Märchen\pwindex{Ehrenstein, Albert 23.\,12.\,1886 Wien – 8.\,4.\,1950 New York City@\textsc{Ehrenstein, Albert} (23.\,12.\,1886 Wien – 8.\,4.\,1950 New York City), \emph{Schriftsteller}!Tai-Gin@\strich\emph{Tai-Gin}|pwv}, das den Vorzug hat, für die Öſterreichiſche Rundschau\orgindex{Österreichische Rundschau@Österreichische Rundschau|pw} nicht ganz ungeeignet zu{ }ſcheinen. Wenn nun Sie,
                  {\pb}ſehr geehrter Herr Doktor, dieſes Opusculum\pwindex{Ehrenstein, Albert 23.\,12.\,1886 Wien – 8.\,4.\,1950 New York City@\textsc{Ehrenstein, Albert} (23.\,12.\,1886 Wien – 8.\,4.\,1950 New York City), \emph{Schriftsteller}!Tai-Gin@\strich\emph{Tai-Gin}|pwv} einer geneigten
               Durchſicht zu unterziehen die Güte hätten, würde mir das eine große Freude bereiten.
               Denn bei dem nicht geringen Volumen des von mir für die Diſſertation\pwindex{Ehrenstein, Albert 23.\,12.\,1886 Wien – 8.\,4.\,1950 New York City@\textsc{Ehrenstein, Albert} (23.\,12.\,1886 Wien – 8.\,4.\,1950 New York City), \emph{Schriftsteller}!Lage in Ungarn (Siebenbürgen und Serbien ausgenommen) im Jahre 1790@\strich\emph{Die Lage in Ungarn (Siebenbürgen und Serbien ausgenommen) im Jahre 1790}|pwv} zu bearbeitenden
               Aktenmaterials, würde mir eine neuerliche Hingabe an zeitraubend-wertloſe
               literariſche Experimente gegenwärtig recht schwer fallen.\pend
           
\pstart
           Mit der Bitte, die kaum leichtfertige Inanſpruchnahme Ihrer koſtbaren Zeit nicht
               allzu ungünſtig beurteilen zu wollen, verbleibe ich ergebenſt Ihr Sie verehrender\pend
           \pstart \spacefill\mbox{Albert Ehrenstein.}\pend{}\selectlanguage{ngerman}\endnumbering\briefempfaengerindex{Schnitzler, Arthur@\textsc{Schnitzler, Arthur}!zzzEhrenstein, Albert@\emph{von Albert Ehrenstein}!1909-01-152@{15. 1. 1909}|)be}\mylabel{L01822h}  \newcommand{\dateiname}{L01822}\newcommand{\titel}{Albert Ehrenstein an Arthur Schnitzler, 15. 1. 1909}\newcommand{\editorInnen}{Martin Anton Müller und Gerd-Hermann Susen}%% latex-leseansicht-abspann.tex
%% Abspann für die Leseansicht.
%% Der Schalter \ifkorrekturansicht ist bereits durch den Vorspann gesetzt.

%% latex-abspann.tex
%% Gemeinsamer Abspann für Korrekturansicht und Leseansicht.
%% Setzt den Schalter \ifkorrekturansicht voraus (gesetzt in den
%% einbindenden Dateien latex-korrekturansicht-abspann.tex bzw.
%% latex-leseansicht-abspann.tex).
%% ---------------------------------------------------------------

\normalsize

% Das esempio-Environment wird nur in der Leseansicht benötigt
\ifkorrekturansicht\else
\newenvironment{esempio}[3]%
{
    \vspace{1.5ex}
    \rlap{\underline{#1}}
    \par
    \setlength{\parindent}{0cm}
    \nopagebreak
    \leftskip=#2cm
    \rightskip=#3cm
}
{
    \par
}
\fi

\doendnotes{C}
\bigskip
\vfill

\clearpage

\footnotesize

\ifkorrekturansicht
  \lohead{\textsc{register}}
\fi

% theindex-Environment neu definieren ohne reledmac
\makeatletter
\renewenvironment{theindex}{%
  \ifkorrekturansicht
    \section*{\indexname}%
  \else
    \subsubsection*{Index der erwähnten Entitäten}%
  \fi
  \setlength{\parindent}{0pt}%
  \setlength{\parskip}{0pt plus 0.3pt}%
  \let\item\@idxitem
}{%
  \ifkorrekturansicht\clearpage\fi
}
\makeatother

\IfFileExists{\jobname-pw.ind}{\input{\jobname-pw.ind}}{}

% Quellenangabe nur in der Leseansicht
\ifkorrekturansicht\else
% Fallback-Definitionen, falls die .tex-Datei \titel etc. nicht gesetzt hat
\providecommand{\titel}{}
\providecommand{\editorInnen}{}
\providecommand{\dateiname}{\jobname}

\vspace{3cm}

\vfill

\footnotesize
\textsc{Quelle}: \titel. Herausgegeben von {\editorInnen}. In: \emph{Arthur Schnitzler: Briefwechsel mit Autorinnen und Autoren}.
 Digitale Edition, https://schnitzler-briefe.acdh.oeaw.ac.at/{\dateiname}.html (Stand \today)
\fi

\end{document}


