%% latex-leseansicht-vorspann.tex
%% Vorspann für die Leseansicht.
%% Lädt die gemeinsame Datei latex-vorspann.tex mit nicht gesetztem Schalter.

\newif\ifkorrekturansicht
\korrekturansichtfalse

\input{../tex-inputs/latex-vorspann}


\section[Hugo von Hofmannsthal an Arthur Schnitzler, 31. 3. 1902]{L01214 Hugo von Hofmannsthal an Arthur Schnitzler, 31. 3. 1902}
\nopagebreak\mylabel{L01214v}
\rehead{ }\normalsize\beginnumbering\briefempfaengerindex{Schnitzler, Arthur@\textsc{Schnitzler, Arthur}!zzzHofmannsthal, Hugo von@\emph{von Hugo von Hofmannsthal}!1902-03-311@{31. 3. 1902}|(be}
\toendnotes[C]{\smallbreak\pagebreak[2]}
\correspDesc{Versand  durch Hugo von Hofmannsthal am 31. 3. 1902 in Wien
\newline{}Erhalt  durch Arthur Schnitzler im Zeitraum [31. 3. 1902
                  – 4. 4. 1902?] in Wien}\toendnotes[C]{\smallbreak}
\Standort{CUL, Schnitzler, B 43.}
\physDesc{Brief, 1 Blatt, 2 Seiten, 426 Zeichen
\newline{}Handschrift: schwarze Tinte, deutsche Kurrent
\newline{}Ordnung: 1) mit Bleistift von unbekannter Hand nummeriert: »\strikeout{195}«  2) mit Bleistift von unbekannter Hand nummeriert:
                                    »188«}
\buchAbdrucke{\weitereDrucke{Hugo von Hofmannsthal, Arthur Schnitzler: \emph{Briefwechsel}. Herausgegeben von Therese Nickl und Heinrich Schnitzler. Frankfurt am Main: \emph{S. Fischer} 1964, S. 158.} }\toendnotes[C]{\smallbreak}
\pstart
           \raggedleft{}{\pb}Oſtern mit Schnee. 31 III. 1902.\pend
           \vspace{0.5em}
\pstart
           lieber, ich freue mich{ }ſehr, daſs es nun doch zuſammengeht. Ich
               werde Donnerstag nach 12\textsuperscript{h} zu Ihnen kommen,{ }ſo daſs wir plaudernd und langſam zuſammen hingehen
                  können.\hspace*{1.5em}Karte abzugeben iſt ganz überflüſſig, da
               es ja eine außerhalb aller Formen{ }ſtehende Zuſammenkunft {\pb}von ein paar Menſchen{ }ſein{ }ſoll,
               und eigentlich ich als der Hausherr zu betrachten bin.\pend
           
\pstart
           Gehen Sie nicht in »\label{K_L01214-1v}\edtext{\textsc{Francesca}\pwindex{\textcolor{red}{\textsuperscript{XXXX indx1}}!Francesca da Rimini@\strich\emph{Francesca da Rimini}|pw}}{\lemma{\textnormal{\emph{Francesca}}}\Cendnote{\textnormal{Am 2. 4. 1902 besuchte Schnitzler die Vorführung von \emph{Francesca da Rimini}\pwindex{\textcolor{red}{\textsuperscript{XXXX indx1}}!Francesca da Rimini@\strich\emph{Francesca da Rimini}|pwk} mit Eleonora Duse\pwindex{Duse, Eleonora 3.\,10.\,1858 Vigevano – 21.\,4.\,1924 Pittsburgh@\textsc{Duse, Eleonora} (3.\,10.\,1858 Vigevano – 21.\,4.\,1924 Pittsburgh), \emph{Schauspielerin}|pwk} im Raimund-Theater\oindex{Wien@\textbf{Wien}!VI., Mariahilf@\textbf{VI., Mariahilf}!Raimund-Theater@\textbf{Raimund-Theater}, \emph{Theater}|pwk}.}}}\label{K_L01214-1}«?\pend
           
\pstart
           Von Herzen Ihr{\\[\baselineskip]}\spacefill\mbox{Hugo.}\pend
           \leftskip=0em{}\selectlanguage{ngerman}\endnumbering\briefempfaengerindex{Schnitzler, Arthur@\textsc{Schnitzler, Arthur}!zzzHofmannsthal, Hugo von@\emph{von Hugo von Hofmannsthal}!1902-03-311@{31. 3. 1902}|)be}\mylabel{L01214h}  \newcommand{\dateiname}{L01214}\newcommand{\titel}{Hugo von Hofmannsthal an Arthur Schnitzler, 31. 3. 1902}\newcommand{\editorInnen}{Martin Anton Müller und Gerd-Hermann Susen}%% latex-leseansicht-abspann.tex
%% Abspann für die Leseansicht.
%% Der Schalter \ifkorrekturansicht ist bereits durch den Vorspann gesetzt.

%% latex-abspann.tex
%% Gemeinsamer Abspann für Korrekturansicht und Leseansicht.
%% Setzt den Schalter \ifkorrekturansicht voraus (gesetzt in den
%% einbindenden Dateien latex-korrekturansicht-abspann.tex bzw.
%% latex-leseansicht-abspann.tex).
%% ---------------------------------------------------------------

\normalsize

% Das esempio-Environment wird nur in der Leseansicht benötigt
\ifkorrekturansicht\else
\newenvironment{esempio}[3]%
{
    \vspace{1.5ex}
    \rlap{\underline{#1}}
    \par
    \setlength{\parindent}{0cm}
    \nopagebreak
    \leftskip=#2cm
    \rightskip=#3cm
}
{
    \par
}
\fi

\doendnotes{C}
\bigskip
\vfill

\clearpage

\footnotesize

\ifkorrekturansicht
  \lohead{\textsc{register}}
\fi

% theindex-Environment neu definieren ohne reledmac
\makeatletter
\renewenvironment{theindex}{%
  \ifkorrekturansicht
    \section*{\indexname}%
  \else
    \subsubsection*{Index der erwähnten Entitäten}%
  \fi
  \setlength{\parindent}{0pt}%
  \setlength{\parskip}{0pt plus 0.3pt}%
  \let\item\@idxitem
}{%
  \ifkorrekturansicht\clearpage\fi
}
\makeatother

\IfFileExists{\jobname-pw.ind}{\input{\jobname-pw.ind}}{}

% Quellenangabe nur in der Leseansicht
\ifkorrekturansicht\else
% Fallback-Definitionen, falls die .tex-Datei \titel etc. nicht gesetzt hat
\providecommand{\titel}{}
\providecommand{\editorInnen}{}
\providecommand{\dateiname}{\jobname}

\vspace{3cm}

\vfill

\footnotesize
\textsc{Quelle}: \titel. Herausgegeben von {\editorInnen}. In: \emph{Arthur Schnitzler: Briefwechsel mit Autorinnen und Autoren}.
 Digitale Edition, https://schnitzler-briefe.acdh.oeaw.ac.at/{\dateiname}.html (Stand \today)
\fi

\end{document}


