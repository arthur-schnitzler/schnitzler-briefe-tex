\input{../tex-inputs/latex-pdf-vorspann}
\begin{center}
            \textcolor{red}{ENTWURF. ENTZIFFERUNG NOCH NICHT KORREKTURGELESEN}
                      \end{center}
            
               \section[Hugo von Hofmannsthal an Arthur Schnitzler, 31. 3. 1902]{ Hugo von Hofmannsthal an Arthur Schnitzler, 31. 3. 1902}\nopagebreak\mylabel{v}\rehead{ }\begin{ledgroupsized}[t]{13cm}\normalsize\beginnumbering\briefempfaengerindex{Schnitzler, Arthur@\textsc{Schnitzler, Arthur}!zzzHofmannsthal, Hugo von@\emph{von Hugo von Hofmannsthal}!1902-03-311@{31. 3. 1902}|(be} \toendnotes[C]{\smallbreak\pagebreak[2]} \Standort{CUL, Schnitzler, B 43.}
\physDesc{Brief, 1 Blatt, 2 Seiten
\newline{}Handschrift: schwarze Tinte, deutsche Kurrent\newline{}Ordnung: 1) mit Bleistift von unbekannter Hand nummeriert: »\strikeout{195}« 2) mit Bleistift von unbekannter Hand nummeriert: »188«}\buchAbdrucke{\weitereDrucke{Hugo von Hofmannsthal, Arthur Schnitzler: \emph{Briefwechsel}. Hg. Therese Nickl und Heinrich Schnitzler. Frankfurt am Main: \emph{S. Fischer} 1964, S. 158.} }\toendnotes[C]{\smallbreak}\pstart
           \raggedleft{}{\pb}Oſtern mit Schnee. 31 III. 1902.\pend
           \pstart
           lieber, ich freue mich ſehr, daſs es nun doch zuſammengeht. Ich
               werde Donnerstag nach 12\textsuperscript{h} zu Ihnen kommen, ſo daſs wir plaudernd und langſam zuſammen hingehen
                  können.\hspace*{1.5em}Karte abzugeben iſt ganz überflüſſig, da
               es ja eine außerhalb aller Formen ſtehende Zuſammenkunft {\pb}von ein paar Menſchen ſein ſoll,
               und eigentlich ich als der Hausherr zu betrachten bin.\pend
           \pstart
           Gehen Sie nicht in »\label{K_L01214_1v}\edtext{\textsc{Francesca}\pwindex{\textcolor{red}{\textsuperscript{XXXX1 indx}}!Francesca da Rimini1901@\strich\emph{Francesca da Rimini} {[}1901{]}|pw}}{\lemma{\textnormal{\emph{Francesca}}}\Cendnote{\textnormal{Am
                     2. 4. 1902 besuchte Schnitzler\pwindex{Schnitzler, Arthur 15.05.1862 – 21.10.1931@\textsc{Schnitzler, Arthur} (15.05.1862 – 21.10.1931), \emph{Schriftsteller, Mediziner}|pwk} die Vorführung von \emph{Francesca da
                     Rimini}\pwindex{\textcolor{red}{\textsuperscript{XXXX1 indx}}!Francesca da Rimini1901@\strich\emph{Francesca da Rimini} {[}1901{]}|pwk} mit Eleonora Duse\pwindex{Duse, Eleonora 03.10.1858 – 21.04.1924@\textsc{Duse, Eleonora} (03.10.1858 – 21.04.1924), \emph{Schauspielerin}|pwk} im Raimund-Theater\oindex{Raimund-Theater@\textbf{Raimund-Theater}|pwk}.}}}\label{K_L01214_1h}«?\pend
           \pstart
           Von Herzen Ihr{\\[\baselineskip]}\spacefill\mbox{Hugo.}\pend
           \leftskip=0em{}\endnumbering\briefempfaengerindex{Schnitzler, Arthur@\textsc{Schnitzler, Arthur}!zzzHofmannsthal, Hugo von@\emph{von Hugo von Hofmannsthal}!1902-03-311@{31. 3. 1902}|)be}\mylabel{h}\end{ledgroupsized}  \newcommand{\dateiname}{L01214}\newcommand{\titel}{Hugo von Hofmannsthal an Arthur Schnitzler, 31. 3. 1902}\newcommand{\editorInnen}{Martin Anton Müller und Gerd-Hermann Susen}\input{../tex-inputs/latex-pdf-abspann}
      