%% latex-korrekturansicht-vorspann.tex
%% Vorspann für die Korrekturansicht.
%% Lädt die gemeinsame Datei latex-vorspann.tex mit gesetztem Schalter.

\newif\ifkorrekturansicht
\korrekturansichttrue

\input{../tex-inputs/latex-vorspann}


\section[Hugo von Hofmannsthal an Arthur Schnitzler, 31. 3. 1902]{L01214 Hugo von Hofmannsthal an Arthur Schnitzler, 31. 3. 1902}
\nopagebreak\mylabel{L01214v}
\rehead{ }\normalsize\beginnumbering\briefempfaengerindex{Schnitzler, Arthur@\textsc{Schnitzler, Arthur}!zzzHofmannsthal, Hugo von@\emph{von Hugo von Hofmannsthal}!1902-03-311@{31. 3. 1902}|(be}
\toendnotes[C]{\smallbreak\pagebreak[2]}\Standort{CUL, Schnitzler, B 43.}
\physDesc{Brief, 1 Blatt, 2 Seiten, 426 Zeichen
\newline{}Handschrift: schwarze Tinte, deutsche Kurrent
\newline{}Ordnung: 1) mit Bleistift von unbekannter Hand nummeriert: »\strikeout{195}«  2) mit Bleistift von unbekannter Hand nummeriert:
                                    »188«}
\buchAbdrucke{\weitereDrucke{Hugo von Hofmannsthal, Arthur Schnitzler: \emph{Briefwechsel}. Frankfurt am Main: \emph{S. Fischer} 1964, S. 158.} }\toendnotes[C]{\smallbreak}
\pstart
           \raggedleft{}{\pb}Oſtern mit Schnee. 31 III. 1902.\pend
           \vspace{0.5em}
\pstart
           lieber, ich freue mich ſehr, daſs es nun doch zuſammengeht. Ich
               werde Donnerstag nach 12\textsuperscript{h} zu Ihnen kommen, ſo daſs wir plaudernd und langſam zuſammen hingehen
                  können.\hspace*{1.5em}Karte abzugeben iſt ganz überflüſſig, da
               es ja eine außerhalb aller Formen ſtehende Zuſammenkunft {\pb}von ein paar Menſchen ſein ſoll,
               und eigentlich ich als der Hausherr zu betrachten bin.\pend
           
\pstart
           Gehen Sie nicht in »\label{K_L01214-1v}\edtext{\textsc{Francesca}\pwindex{Francesca da Rimini@\emph{Francesca da Rimini}|pw}}{\lemma{\textnormal{\emph{Francesca}}}\Cendnote{\textnormal{Am 2. 4. 1902 besuchte Schnitzler die Vorführung von \emph{Francesca da Rimini}\pwindex{Francesca da Rimini@\emph{Francesca da Rimini}|pwk} mit Eleonora Duse\pwindex{Duse, Eleonora 03.10.1858 – 21.04.1924@\textsc{Duse, Eleonora} (03.10.1858 – 21.04.1924), \emph{Schauspieler/Schauspielerin}|pwk} im Raimund-Theater\oindex{Raimund-Theater@\textbf{Raimund-Theater}, \emph{Theater (K.THE)}|pwk}.}}}\label{K_L01214-1}«?\pend
           
\pstart
           Von Herzen Ihr{\\[\baselineskip]}\spacefill\mbox{Hugo.}\pend
           \leftskip=0em{}\selectlanguage{ngerman}\endnumbering\briefempfaengerindex{Schnitzler, Arthur@\textsc{Schnitzler, Arthur}!zzzHofmannsthal, Hugo von@\emph{von Hugo von Hofmannsthal}!1902-03-311@{31. 3. 1902}|)be}\mylabel{L01214h}  \normalsize

\doendnotes{C}
\bigskip
\vfill

\clearpage

\footnotesize

\lohead{\textsc{register}}

% Definiere theindex-Environment komplett neu ohne reledmac
\makeatletter
\renewenvironment{theindex}{%
  \section*{\indexname}%
  \setlength{\parindent}{0pt}%
  \setlength{\parskip}{0pt plus 0.3pt}%
  \let\item\@idxitem
}{%
  \clearpage
}
\makeatother

\IfFileExists{\jobname-pw.ind}{\input{\jobname-pw.ind}}{}

\end{document}

      