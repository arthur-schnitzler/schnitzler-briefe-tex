%% latex-leseansicht-vorspann.tex
%% Vorspann für die Leseansicht.
%% Lädt die gemeinsame Datei latex-vorspann.tex mit nicht gesetztem Schalter.

\newif\ifkorrekturansicht
\korrekturansichtfalse

\input{../tex-inputs/latex-vorspann}


\section[ Paul Goldmann und Theodore Rottenberg an Arthur Schnitzler, 29. 8. 1903]{L03385 Paul Goldmann und Theodore Rottenberg an Arthur
               Schnitzler,  29. 8. 1903}
\nopagebreak\mylabel{L03385v}
\rehead{ }\normalsize\beginnumbering\briefempfaengerindex{Schnitzler, Arthur@\textsc{Schnitzler, Arthur}!zzzRottenberg, Theodore@\emph{von Theodore Rottenberg}!1903-08-291@{29. 8. 1903}|(be}\briefempfaengerindex{Schnitzler, Arthur@\textsc{Schnitzler, Arthur}!zzzGoldmann, Paul@\emph{von Paul Goldmann}!1903-08-291@{29. 8. 1903}|(be}
\toendnotes[C]{\smallbreak\pagebreak[2]}
\correspDesc{Versand  durch Paul Goldmann, Theodore Rottenberg am 29. 8. 1903 in Lavarone
\newline{}Erhalt  durch Arthur Schnitzler im Zeitraum [30. 8. 1903
                  – 3. 9. 1903?] in Wien}\toendnotes[C]{\smallbreak}
\Standort{DLA, A:Schnitzler, HS.NZ85.1.3173.}
\physDesc{Brief, 1 Blatt, 2 Seiten, 1137 Zeichen
\newline{}Handschrift Paul Goldmann: schwarze Tinte, deutsche Kurrent
\newline{}Handschrift Theodore Rottenberg: schwarze Tinte, deutsche Kurrent
\newline{}Schnitzler: mit Bleistift das Jahr »903« vermerkt }\toendnotes[C]{\smallbreak}
\pstart
           {\pb}\textcolor{gray}{\textbf{Grand Hôtel Lavarone\oindex{Grand Hotel Lavarone@\textbf{Grand Hotel Lavarone}, \emph{Hotel}|pw}}}\hfill \textcolor{gray}{\textbf{\begin{otherlanguage}{italian}Li\end{otherlanguage}}}{ }29. August \textcolor{gray}{\textbf{190}}3\pend
           
\pstart
           \textcolor{gray}{\textbf{Lavarone\oindex{Lavarone@\textbf{Lavarone}, \emph{Verwaltungsgebiet}|pw} (Trentino\oindex{Trentino-Alto Adige@\textbf{Trentino-Alto Adige}, \emph{Land}|pw})}}{ }{\\}\textcolor{gray}{\textbf{m. 1200}}\pend
           
\pstart
           \textcolor{gray}{\textbf{\label{K_L03385-1v}\edtext{\begin{otherlanguage}{italian}Stessa Direzione\end{otherlanguage}}{\lemma{\textnormal{\emph{Stessa Direzione}}}\Cendnote{\textnormal{italienisch: unter selber
                        Leitung}}}\label{K_L03385-1}:}}{ }{\\}\textcolor{gray}{\textbf{Palast Hôtel Lido\oindex{Palast Hotel Lido@\textbf{Palast Hotel Lido}, \emph{Hotel}|pw}{ }Riva\oindex{Riva del Garda@\textbf{Riva del Garda}, \emph{Hauptstadt}|pw} (\begin{otherlanguage}{italian}Lago di Garda\oindex{Riva del Garda@\textbf{Riva del Garda}, \emph{Hauptstadt}|pw}\end{otherlanguage})}}\pend
           
\pstart
           \label{T_L03385-1v}\edtext{\textcolor{gray}{\textbf{Telegrammi: Grandhôtel –
                        Lavarone\oindex{Grand Hotel Lavarone@\textbf{Grand Hotel Lavarone}, \emph{Hotel}|pw}.}}}{\lemma{\textnormal{\emph{Telegrammi: … Lavarone.}}}\Cendnote{\textnormal{seitlich entlang des linken
                     Blattrands}}}\label{T_L03385-1}\pend
           
\pstart\center{}Mein lieber Freund,\pend\vspace{0.5em}
\pstart
           Ich beglückwünſche Dich und Deine Frau\pwindex{Schnitzler, Olga 17.\,1.\,1882 Wien – 13.\,1.\,1970 Lugano@\textsc{Schnitzler, Olga} (17.\,1.\,1882 Wien – 13.\,1.\,1970 Lugano), \emph{Schauspielerin, Sängerin}|pwv} auf das Herzlichſte zu Eurer \label{K_L03385-2v}\edtext{Vermählung}{\lemma{\textnormal{\emph{Vermählung}}}\Cendnote{\textnormal{Arthur Schnitzler und Olga Gussmann\pwindex{Schnitzler, Olga 17.\,1.\,1882 Wien – 13.\,1.\,1970 Lugano@\textsc{Schnitzler, Olga} (17.\,1.\,1882 Wien – 13.\,1.\,1970 Lugano), \emph{Schauspielerin, Sängerin}|pwk} hatten am 26. 8. 1903 geheiratet.}}}\label{K_L03385-2}. Ich habe mich{ }ſehr über dieſe Nachricht gefreut und wünſche Euch Beiden viele glückliche Jahre.\pend
           
\pstart
           Hier in \textsc{Lavarone\oindex{Lavarone@\textbf{Lavarone}, \emph{Verwaltungsgebiet}|pw}} iſt bisher Alles gut \strikeout{gelauf} verlaufen. Ein
               herrlicher Aufenthalt. Wir haben fleißig das Land\oindex{Südtirol@\textbf{Südtirol}, \emph{Verwaltungsgebiet}|pwv} durchſtreift. Der \label{K_L03385-3v}\edtext{Frankfurt\oindex{Frankfurt am Main@\textbf{Frankfurt am Main}, \emph{Hauptstadt}|pw}er Oberſtaatsanwalt\pwindex{?? [Frankfurter Oberstaatsanwalt] @\textsc{?? [Frankfurter Oberstaatsanwalt]}|pwv}}{\lemma{\textnormal{\emph{Frankfurter Oberstaatsanwalt}}}\Cendnote{\textnormal{nicht ermittelt}}}\label{K_L03385-3} iſt ein
               freundlicher Mann. Aber{ }ſeine Frau\pwindex{?? [Frau eines Frankfurter Oberstaatsanwalts] @\textsc{?? [Frau eines Frankfurter Oberstaatsanwalts]}|pwv} ignorirt mich, offenbar aus{ }ſittlicher Entrüſtung.\pend
           
\pstart
           Dich haben wir am Tage nach Deiner \label{K_L03385-4v}\edtext{Abreiſe}{\lemma{\textnormal{\emph{Abreise}}}\Cendnote{\textnormal{Schnitzler war am 20. 8. 1903 mit Goldmann\pwindex{Goldmann, Paul 31.\,1.\,1865 Breslau – 25.\,9.\,1935 Wien@\textsc{Goldmann, Paul} (31.\,1.\,1865 Breslau – 25.\,9.\,1935 Wien), \emph{Schriftsteller, Journalist}|pwk} und Rottenberg\pwindex{Rottenberg, Theodore 7.\,9.\,1875 – 5.\,4.\,1945 Limburg an der Lahn@\textsc{Rottenberg, Theodore} (7.\,9.\,1875 – 5.\,4.\,1945 Limburg an der Lahn)|pwk} nach Lavarone\oindex{Lavarone@\textbf{Lavarone}, \emph{Verwaltungsgebiet}|pwk} gekommen. Bereits am Folgetag war
                  er wieder abgereist. Am Morgen des 22. 8. 1903 war er wieder in Wien\oindex{Wien@\textbf{Wien}, \emph{Verwaltungsgebiet}|pwk} angekommen.}}}\label{K_L03385-4}{ }ſehr vermißt. Hätteſt wirklich noch
               ein paar Tage bleiben{ }ſollen.\pend
           
\pstart
           Seit zwei Tagen iſt das Idyll geſtört. Ich habe mir bei einem Ausflug die \introOben{}rechte\introOben{} Ferſe verletzt (bin mit dem Abſatz auf einen Stein
               geſprungen und habe mir offenbar eine \label{K_L03385-5v}\edtext{Contuſion}{\lemma{\textnormal{\emph{Contusion}}}\Cendnote{\textnormal{Organ- und
                  Körperschädigung durch äußere Gewalteinwirkung ohne sichtbare
                  Hautverletzungen}}}\label{K_L03385-5} des Knochens zugetragen.) Nun kann ich nicht mehr {\pb}gehen, muß im \textsc{Hotel\oindex{Grand Hotel Lavarone@\textbf{Grand Hotel Lavarone}, \emph{Hotel}|pwv}}{ }\strikeout{ſtilſ}{ }ſtillſitzen, –{ }ſie desgleichen. Und das iſt
               recht traurig.\pend
           
\pstart
           Immerhin, wir bleiben wohl noch acht Tage, wenn auch nur{ }ſitzend. Schreibe noch
               einmal hierher. Du machſt uns eine große Freude damit.\pend
           
\pstart
           Viele herzliche Grüße an Dich und Deine Frau\pwindex{Schnitzler, Olga 17.\,1.\,1882 Wien – 13.\,1.\,1970 Lugano@\textsc{Schnitzler, Olga} (17.\,1.\,1882 Wien – 13.\,1.\,1970 Lugano), \emph{Schauspielerin, Sängerin}|pwv}! {\\[\baselineskip]}Dein {\\[\baselineskip]}\spacefill\mbox{Paul Goldmann}\pend
           \leftskip=0em{}
\pstart
           \noindent{}{[}hs. Rottenberg:{]} Ebenfalls herzliche Glückwünſche zur Vermählung {\kaufmannsund} beſte Grüße Ihnen {\kaufmannsund}
                  Ihrer Frau Gemahlin\pwindex{Schnitzler, Olga 17.\,1.\,1882 Wien – 13.\,1.\,1970 Lugano@\textsc{Schnitzler, Olga} (17.\,1.\,1882 Wien – 13.\,1.\,1970 Lugano), \emph{Schauspielerin, Sängerin}|pw}. –\pend
           \selectlanguage{ngerman}\endnumbering\briefempfaengerindex{Schnitzler, Arthur@\textsc{Schnitzler, Arthur}!zzzRottenberg, Theodore@\emph{von Theodore Rottenberg}!1903-08-291@{29. 8. 1903}|)be}\briefempfaengerindex{Schnitzler, Arthur@\textsc{Schnitzler, Arthur}!zzzGoldmann, Paul@\emph{von Paul Goldmann}!1903-08-291@{29. 8. 1903}|)be}\mylabel{L03385h}  \newcommand{\dateiname}{L03385}\newcommand{\titel}{Paul Goldmann und Theodore Rottenberg an Arthur Schnitzler, 29. 8. 1903}\newcommand{\editorInnen}{Martin Anton Müller und Laura Untner}%% latex-leseansicht-abspann.tex
%% Abspann für die Leseansicht.
%% Der Schalter \ifkorrekturansicht ist bereits durch den Vorspann gesetzt.

%% latex-abspann.tex
%% Gemeinsamer Abspann für Korrekturansicht und Leseansicht.
%% Setzt den Schalter \ifkorrekturansicht voraus (gesetzt in den
%% einbindenden Dateien latex-korrekturansicht-abspann.tex bzw.
%% latex-leseansicht-abspann.tex).
%% ---------------------------------------------------------------

\normalsize

% Das esempio-Environment wird nur in der Leseansicht benötigt
\ifkorrekturansicht\else
\newenvironment{esempio}[3]%
{
    \vspace{1.5ex}
    \rlap{\underline{#1}}
    \par
    \setlength{\parindent}{0cm}
    \nopagebreak
    \leftskip=#2cm
    \rightskip=#3cm
}
{
    \par
}
\fi

\doendnotes{C}
\bigskip
\vfill

\clearpage

\footnotesize

\ifkorrekturansicht
  \lohead{\textsc{register}}
\fi

% theindex-Environment neu definieren ohne reledmac
\makeatletter
\renewenvironment{theindex}{%
  \ifkorrekturansicht
    \section*{\indexname}%
  \else
    \subsubsection*{Index der erwähnten Entitäten}%
  \fi
  \setlength{\parindent}{0pt}%
  \setlength{\parskip}{0pt plus 0.3pt}%
  \let\item\@idxitem
}{%
  \ifkorrekturansicht\clearpage\fi
}
\makeatother

\IfFileExists{\jobname-pw.ind}{\input{\jobname-pw.ind}}{}

% Quellenangabe nur in der Leseansicht
\ifkorrekturansicht\else
% Fallback-Definitionen, falls die .tex-Datei \titel etc. nicht gesetzt hat
\providecommand{\titel}{}
\providecommand{\editorInnen}{}
\providecommand{\dateiname}{\jobname}

\vspace{3cm}

\vfill

\footnotesize
\textsc{Quelle}: \titel. Herausgegeben von {\editorInnen}. In: \emph{Arthur Schnitzler: Briefwechsel mit Autorinnen und Autoren}.
 Digitale Edition, https://schnitzler-briefe.acdh.oeaw.ac.at/{\dateiname}.html (Stand \today)
\fi

\end{document}


