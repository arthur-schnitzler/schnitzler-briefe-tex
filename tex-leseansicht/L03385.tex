%% latex-leseansicht-vorspann.tex
%% Vorspann für die Leseansicht.
%% Lädt die gemeinsame Datei latex-vorspann.tex mit nicht gesetztem Schalter.

\newif\ifkorrekturansicht
\korrekturansichtfalse

\input{../tex-inputs/latex-vorspann}


         
         \renewcommand{\erwaehntePersonen}{Personen:  ?? [Frankfurter Oberstaatsanwalt],  ?? [Frau eines Frankfurter Oberstaatsanwalts], Olga Schnitzler}
         \renewcommand{\erwaehnteOrte}{Orte: Frankfurt am Main, Grand Hotel Lavarone, Lavarone, Palast Hotel Lido, Riva del Garda, Südtirol, Trentino-Alto Adige, Wien}
         \renewcommand{\erwaehnteWerke}{}
               \section[ Paul Goldmann und Theodore Rottenberg an Arthur Schnitzler, 29. 8. 1903]{ Paul Goldmann und Theodore Rottenberg an Arthur
               Schnitzler, 29. 8. 1903}\nopagebreak\mylabel{v}\rehead{ }\begin{ledgroupsized}[t]{13cm}\normalsize\beginnumbering \toendnotes[C]{\smallbreak\pagebreak[2]} \Standort{DLA, A:Schnitzler, HS.NZ85.1.3173.}
\physDesc{Brief, 1 Blatt, 2 Seiten, 1138 Zeichen
\newline{}Handschrift Paul Goldmann: schwarze Tinte, deutsche Kurrent\newline{}Handschrift Theodore Rottenberg: schwarze Tinte, deutsche Kurrent
\newline{}Schnitzler: mit Bleistift das Jahr »{[}1{]}903« vermerkt }\toendnotes[C]{\smallbreak}\pstart
           \noindent{}{\pb}\textcolor{gray}{\textbf{Grand Hôtel Lavarone\oindex{Grand Hotel Lavarone@\textbf{Grand Hotel Lavarone}|pw}}}\hfill \textcolor{gray}{\textbf{\begin{otherlanguage}{italian}Li\end{otherlanguage}}}{ }29. August \textcolor{gray}{\textbf{190}}3\pend
           \pstart
           \textcolor{gray}{\textbf{Lavarone\oindex{Lavarone@\textbf{Lavarone}|pw} (Trentino\oindex{Trentino-Alto Adige@\textbf{Trentino-Alto Adige}|pw})}}{ }{\\}\textcolor{gray}{\textbf{m. 1200}}\pend
           \pstart
           \textcolor{gray}{\textbf{\label{K_L03385-1v}\edtext{\begin{otherlanguage}{italian}Stessa Direzione\end{otherlanguage}}{\lemma{\textnormal{\emph{Stessa Direzione}}}\Cendnote{\textnormal{italienisch: unter selber
                        Leitung}}}\label{K_L03385-1h}:}}{ }{\\}\textcolor{gray}{\textbf{Palast Hôtel Lido\oindex{Palast Hotel Lido@\textbf{Palast Hotel Lido}|pw}{ }Riva\oindex{Riva del Garda@\textbf{Riva del Garda}|pw} (\begin{otherlanguage}{italian}Lago di Garda\oindex{Riva del Garda@\textbf{Riva del Garda}|pw}\end{otherlanguage})}}\pend
           \pstart
           \label{T_L03385-1v}\edtext{\textcolor{gray}{\textbf{Telegrammi: Grandhôtel –
                        Lavarone\oindex{Grand Hotel Lavarone@\textbf{Grand Hotel Lavarone}|pw}. }}}{\lemma{\textnormal{\emph{Telegrammi: … Lavarone.}}}\Cendnote{\textnormal{seitlich entlang des linken
                     Blattrands}}}\label{T_L03385-1h}\pend
           \pstart\center{}Mein lieber Freund,\pend\pstart
           Ich beglückwünſche Dich und Deine Frau\pwindex{Schnitzler, Olga 17.01.1882 – 13.01.1970@\textsc{Schnitzler, Olga} (17.01.1882 – 13.01.1970), \emph{Schauspielerin, Sängerin}|pwv} auf das Herzlichſte zu Eurer \label{K_L03385-2v}\edtext{Vermählung}{\lemma{\textnormal{\emph{Vermählung}}}\Cendnote{\textnormal{Arthur Schnitzler\pwindex{Schnitzler, Arthur 15.05.1862 – 21.10.1931@\textsc{Schnitzler, Arthur} (15.05.1862 – 21.10.1931), \emph{Schriftsteller, Mediziner}|pwk} und Olga Gussmann\pwindex{Schnitzler, Olga 17.01.1882 – 13.01.1970@\textsc{Schnitzler, Olga} (17.01.1882 – 13.01.1970), \emph{Schauspielerin, Sängerin}|pwk} hatten am 26. 8. 1903 geheiratet.}}}\label{K_L03385-2h}. Ich habe mich
               ſehr über dieſe Nachricht gefreut und wünſche Euch Beiden viele glückliche Jahre.\pend
           \pstart
           Hier in \textsc{Lavarone\oindex{Lavarone@\textbf{Lavarone}|pw}} iſt bisher Alles gut \strikeout{gelauf} verlaufen. Ein
               herrlicher Aufenthalt. Wir haben fleißig das Land\oindex{Suedtirol@\textbf{Südtirol}|pwv} durchſtreift. Der \label{K_L03385-3v}\edtext{Frankfurt\oindex{Frankfurt am Main@\textbf{Frankfurt am Main}|pw}er Oberſtaatsanwalt\pwindex{?? [Frankfurter Oberstaatsanwalt] @\textsc{?? [Frankfurter Oberstaatsanwalt]}|pwv}}{\lemma{\textnormal{\emph{Frankfurter Oberſtaatsanwalt}}}\Cendnote{\textnormal{nicht ermittelt}}}\label{K_L03385-3h} iſt ein
               freundlicher Mann. Aber ſeine Frau\pwindex{?? [Frau eines Frankfurter Oberstaatsanwalts] @\textsc{?? [Frau eines Frankfurter Oberstaatsanwalts]}|pwv} ignorirt mich, offenbar aus ſittlicher Entrüſtung.\pend
           \pstart
           Dich haben wir am Tage nach Deiner \label{K_L03385-5v}\edtext{Abreiſe}{\lemma{\textnormal{\emph{Abreiſe}}}\Cendnote{\textnormal{Schnitzler\pwindex{Schnitzler, Arthur 15.05.1862 – 21.10.1931@\textsc{Schnitzler, Arthur} (15.05.1862 – 21.10.1931), \emph{Schriftsteller, Mediziner}|pwk} war am 20. 8. 1903 mit Goldmann\pwindex{Goldmann, Paul 31.01.1865 – 25.09.1935@\textsc{Goldmann, Paul} (31.01.1865 – 25.09.1935), \emph{Schriftsteller, Journalist}|pwk} und Rottenberg\pwindex{Rottenberg, Theodore 1875-09-07 – 1945-04-05@\textsc{Rottenberg, Theodore} (1875-09-07 – 1945-04-05)|pwk} nach Lavarone\oindex{Lavarone@\textbf{Lavarone}|pwk} gekommen. Bereits am Folgetag war
                  er wieder abgereist. Am Morgen des 22. 8. 1903 war er wieder in Wien\oindex{Wien@\textbf{Wien}|pwk} angekommen.}}}\label{K_L03385-5h} ſehr vermißt. Hätteſt wirklich noch
               ein paar Tage bleiben ſollen.\pend
           \pstart
           Seit zwei Tagen iſt das Idyll geſtört. Ich habe mir bei einem Ausflug die \introOben{}rechte\introOben{} Ferſe verletzt (bin mit dem Abſatz auf einen Stein
               geſprungen und habe mir offenbar eine \label{K_L03385-4v}\edtext{Contuſion}{\lemma{\textnormal{\emph{Contuſion}}}\Cendnote{\textnormal{Organ- und
                  Körperschädigung durch äußere Gewalteinwirkung ohne sichtbare
                  Hautverletzungen}}}\label{K_L03385-4h} des Knochens zugetragen.) Nun kann ich nicht mehr {\pb}gehen, muß im \textsc{Hotel\oindex{Grand Hotel Lavarone@\textbf{Grand Hotel Lavarone}|pwv}}{ }\strikeout{ſtilſ} ſtillſitzen, – ſie desgleichen. Und das iſt
               recht traurig.\pend
           \pstart
           Immerhin, wir bleiben wohl noch acht Tage, wenn auch nur ſitzend. Schreibe noch
               einmal hierher. Du machſt uns eine große Freude damit.\pend
           \pstart
           Viele herzliche Grüße an Dich und Deine Frau\pwindex{Schnitzler, Olga 17.01.1882 – 13.01.1970@\textsc{Schnitzler, Olga} (17.01.1882 – 13.01.1970), \emph{Schauspielerin, Sängerin}|pwv}! {\\[\baselineskip]}Dein {\\[\baselineskip]}\spacefill\mbox{Paul Goldmann}\pend
           \leftskip=0em{}\pstart
           \noindent{}{[}hs. Rottenberg:{]} Ebenfalls herzliche Glückwünſche zur Vermählung {\kaufmannsund} beſte Grüße Ihnen {\kaufmannsund}
                  Ihrer Frau Gemahlin\pwindex{Schnitzler, Olga 17.01.1882 – 13.01.1970@\textsc{Schnitzler, Olga} (17.01.1882 – 13.01.1970), \emph{Schauspielerin, Sängerin}|pw}. –\pend
           
         
         \endnumbering\mylabel{h}\end{ledgroupsized}  \newcommand{\dateiname}{L03385}\newcommand{\titel}{Paul Goldmann und Theodore Rottenberg an Arthur Schnitzler, 29. 8. 1903}\newcommand{\editorInnen}{Martin Anton Müller und Laura Untner}%% latex-leseansicht-abspann.tex
%% Abspann für die Leseansicht.
%% Der Schalter \ifkorrekturansicht ist bereits durch den Vorspann gesetzt.

%% latex-abspann.tex
%% Gemeinsamer Abspann für Korrekturansicht und Leseansicht.
%% Setzt den Schalter \ifkorrekturansicht voraus (gesetzt in den
%% einbindenden Dateien latex-korrekturansicht-abspann.tex bzw.
%% latex-leseansicht-abspann.tex).
%% ---------------------------------------------------------------

\normalsize

% Das esempio-Environment wird nur in der Leseansicht benötigt
\ifkorrekturansicht\else
\newenvironment{esempio}[3]%
{
    \vspace{1.5ex}
    \rlap{\underline{#1}}
    \par
    \setlength{\parindent}{0cm}
    \nopagebreak
    \leftskip=#2cm
    \rightskip=#3cm
}
{
    \par
}
\fi

\doendnotes{C}
\bigskip
\vfill

\clearpage

\footnotesize

\ifkorrekturansicht
  \lohead{\textsc{register}}
\fi

% theindex-Environment neu definieren ohne reledmac
\makeatletter
\renewenvironment{theindex}{%
  \ifkorrekturansicht
    \section*{\indexname}%
  \else
    \subsubsection*{Index der erwähnten Entitäten}%
  \fi
  \setlength{\parindent}{0pt}%
  \setlength{\parskip}{0pt plus 0.3pt}%
  \let\item\@idxitem
}{%
  \ifkorrekturansicht\clearpage\fi
}
\makeatother

\IfFileExists{\jobname-pw.ind}{\input{\jobname-pw.ind}}{}

% Quellenangabe nur in der Leseansicht
\ifkorrekturansicht\else
% Fallback-Definitionen, falls die .tex-Datei \titel etc. nicht gesetzt hat
\providecommand{\titel}{}
\providecommand{\editorInnen}{}
\providecommand{\dateiname}{\jobname}

\vspace{3cm}

\vfill

\footnotesize
\textsc{Quelle}: \titel. Herausgegeben von {\editorInnen}. In: \emph{Arthur Schnitzler: Briefwechsel mit Autorinnen und Autoren}.
 Digitale Edition, https://schnitzler-briefe.acdh.oeaw.ac.at/{\dateiname}.html (Stand \today)
\fi

\end{document}


      