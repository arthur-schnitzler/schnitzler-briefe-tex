%% latex-korrekturansicht-vorspann.tex
%% Vorspann für die Korrekturansicht.
%% Lädt die gemeinsame Datei latex-vorspann.tex mit gesetztem Schalter.

\newif\ifkorrekturansicht
\korrekturansichttrue

\input{../tex-inputs/latex-vorspann}


\section[Arthur Schnitzler an Stefan Zweig, 5. 4. 1930]{L03737 Arthur Schnitzler an Stefan Zweig, 5. 4. 1930}
\nopagebreak\mylabel{L03737v}
\rehead{ }\normalsize\beginnumbering\briefempfaengerindex{Zweig, Stefan@\textsc{Zweig, Stefan}!zzzSchnitzler, Arthur@\emph{von Arthur Schnitzler}!1930-04-051@{5. 4. 1930}|(be}
\toendnotes[C]{\smallbreak\pagebreak[2]}\Standort{Jerusalem, National Library of Israel, ARC. Ms. Var. 305 1 58 Stefan Zweig Collection.}
\physDesc{Postkarte, 1 Blatt, 2 Seiten, 613 Zeichen
\newline{}Handschrift: schwarze Tinte, lateinische Kurrent
\newline{}Versand: 1) Aufkleber: »Durch Eilboten. Exprès.«  2) Stempel: »\nobreak{}\oindex{XVIII., Waehring@\textbf{XVIII., Währing}, \emph{A.ADM3}|pwk}18/\textsubscript{1} Wien
                                       110, 5. IV. 30, 17\nobreak{}«.  3) Stempel: »\nobreak{}\oindex{IX., Alsergrund@\textbf{IX., Alsergrund}, \emph{A.ADM3}|pwk}9 Wien, 5. IV. 30, 17\textsuperscript{\textcolor{gray}{4}0}\nobreak{}«. 
\newline{}Ordnung: mit Bleistift datiert: »1930« }\toendnotes[C]{\smallbreak}\pstart{}{\pb}\label{T_L03737-1v}\edtext{\textcolor{gray}{\textbf{A. S.}}}{\lemma{\textnormal{\emph{A. S.}}}\Cendnote{\textnormal{ovaler Absenderkleber}}}\label{T_L03737-1}\pend{}\pstart{}\textcolor{gray}{\textbf{WIEN, XVIII.}}\oindex{XVIII., Waehring@\textbf{XVIII., Währing}, \emph{A.ADM3}|pw}\pend{}\pstart{}\textcolor{gray}{\textbf{STERNWARTESTR. 71}}\oindex{Sternwartestrasse 71@\textbf{Sternwartestraße 71}, \emph{Wohngebäude (K.WHS)}|pw}\pend{}{\bigskip}\pstart{}Hn. Dr. Stefan Zweig\pend{}\pstart{}Wien IX\oindex{IX., Alsergrund@\textbf{IX., Alsergrund}, \emph{A.ADM3}|pw}\pend{}\pstart{}Garnisongasse 10\oindex{Garnisongasse 10@\textbf{Garnisongasse 10}, \emph{Wohngebäude (K.WHS)}|pw}\pend{}{\bigskip}\vspace{1em}
\pstart
           \raggedleft{}{\pb}Wien\oindex{Wien@\textbf{Wien}, \emph{A.ADM2}|pw}, \textcolor{gray}{5. 4. 30}\pend
           \vspace{0.5em}
\pstart
           lieber Doctor Stefan Zweig, Sie sind leider noch nicht da, ich habe
               aber wieder bei Ihnen angerufen. Meine Telef Nu{\geminationm}er lautet \textsc{\uline{A 10.0.81}}, ich hoffe Sie melden Ihre Ankunft, bald nachdem Sie eingetroffenen sind, und
               ich sehe Sie sehr bald. Danke sehr für das Stück\pwindex{Lamm des Armen. Tragikomoedie in drei Akten@\emph{Das Lamm des Armen. Tragikomödie in drei Akten}|pwv}, dessen Lecture ich noch verschoben habe; die kleinen
                  Novellen\pwindex{Kleine Chronik@\emph{Kleine Chronik}|pwv} hat man mir
               natürlich schon {\pb}davongetragen – so daſs ich den Titel der
                  Geschichte von dem Flüchtling\pwindex{Episode vom Genfer See@\emph{Episode vom Genfer See}|pwv}, die mir von
               allen die besonderste und ein Meisterstück der Erzählung überhaupt erscheint, nicht
               einmal nennen kann.\pend
           
\pstart
           Sehr herzlich{\\[\baselineskip]}Ihr \spacefill\mbox{A. S.}\pend
           \leftskip=0em{}\selectlanguage{ngerman}\endnumbering\briefempfaengerindex{Zweig, Stefan@\textsc{Zweig, Stefan}!zzzSchnitzler, Arthur@\emph{von Arthur Schnitzler}!1930-04-051@{5. 4. 1930}|)be}\mylabel{L03737h}
\begin{anhang}
\end{anhang}\normalsize

\doendnotes{C}
\bigskip
\vfill

\clearpage

\footnotesize

\lohead{\textsc{register}}

% Definiere theindex-Environment komplett neu ohne reledmac
\makeatletter
\renewenvironment{theindex}{%
  \section*{\indexname}%
  \setlength{\parindent}{0pt}%
  \setlength{\parskip}{0pt plus 0.3pt}%
  \let\item\@idxitem
}{%
  \clearpage
}
\makeatother

\IfFileExists{\jobname-pw.ind}{\input{\jobname-pw.ind}}{}

\end{document}

      