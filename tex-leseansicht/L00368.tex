%% latex-korrekturansicht-vorspann.tex
%% Vorspann für die Korrekturansicht.
%% Lädt die gemeinsame Datei latex-vorspann.tex mit gesetztem Schalter.

\newif\ifkorrekturansicht
\korrekturansichttrue

\input{../tex-inputs/latex-vorspann}


\section[Arthur Schnitzler an Richard Beer-Hofmann, 9. 9. 1894]{L00368 Arthur Schnitzler an Richard Beer-Hofmann, 9. 9. 1894}
\nopagebreak\mylabel{L00368v}
\rehead{ }\normalsize\beginnumbering\briefempfaengerindex{Beer-Hofmann, Richard@\textsc{Beer-Hofmann, Richard}!zzzSchnitzler, Arthur@\emph{von Arthur Schnitzler}!1894-09-091@{9. 9. 1894}|(be}
\toendnotes[C]{\smallbreak\pagebreak[2]}\Standort{YCGL, MSS 31.}
\physDesc{Brief, 1 Blatt, 2 Seiten, Umschlag, 660 Zeichen
\newline{}Handschrift: Bleistift, deutsche Kurrent
\newline{}Versand: 1) Stempel: »\nobreak{}\oindex{IX., Alsergrund@\textbf{IX., Alsergrund}, \emph{A.ADM3}|pwk}Wien 9/3, 9. 9. 94, 3–4 N\nobreak{}«.   2) Stempel: »\nobreak{}\oindex{Bad Ischl@\textbf{Bad Ischl}, \emph{P.PPL}|pwk}Ischl, 10/9 9{[}4{]}, 7 F\nobreak{}«. }
\buchAbdrucke{\weitereDrucke{Arthur Schnitzler, Richard Beer-Hofmann: \emph{Briefwechsel 1891–1931}. Wien, Zürich: \emph{Europaverlag} 1992, S. 59.} }\toendnotes[C]{\smallbreak}\pstart{}{\pb}Herrn Dr. \textsc{Richard
                     Beer-Hofmann}\pend{}\pstart{}\textsc{Ischl\oindex{Bad Ischl@\textbf{Bad Ischl}, \emph{P.PPL}|pw}}\pend{}\pstart{}\textsc{Egelmoos 22\oindex{Eglmoosgasse@\textbf{Eglmoosgasse}, \emph{Bezirk (A.BZK)}|pw}}\pend{}{\bigskip}\vspace{1em}
\pstart{}{\pb}Lieber Richard,\pend\vspace{0.5em}
\pstart
           1) Bolgar\pwindex{Regeln des Duells@\emph{Die Regeln des Duells}|pwv}\pwindex{Bolgár, Franz von 03.01.1851 – 23.05.1923@\textsc{Bolgár, Franz von} (03.01.1851 – 23.05.1923), \emph{Politiker/Politikerin, Publizist/Publizistin}|pw} geht eben unter Kreuzband ab.\pend
           
\pstart
           2.) an P. Horn\pwindex{Horn, Paul 13.02.1867 – 18.01.1936@\textsc{Horn, Paul} (13.02.1867 – 18.01.1936), \emph{Fabrikant/Fabrikantin}|pw}{ }ſchrieb ich, weil Schenker\orgindex{Schenker und Co.@Schenker {\kaufmannsund}  Co.|pw} immer beſetzt iſt und das telefoniren mich nervös macht. Ich bat
               ihn, Ihnen direct ſofort zu antworten.\pend
           
\pstart
           3.) Bahr\pwindex{Bahr, Hermann 19.07.1863 – 15.01.1934@\textsc{Bahr, Hermann} (19.07.1863 – 15.01.1934), \emph{Schriftsteller/Schriftstellerin, Kritiker/Kritikerin}|pw} werde ich morgen ſprechen.\pend
           
\pstart
           4.) Adele S.\pwindex{Sandrock, Adele 1863-08-19 – 1937-08-30@\textsc{Sandrock, Adele} (1863-08-19 – 1937-08-30), \emph{Schauspieler/Schauspielerin}|pw} wohnt Opernring 19\oindex{Opernring@\textbf{Opernring}, \emph{Straße (K.STR)}|pw}.\pend
           
\pstart
           5.) Der \label{K_L00368-1v}\edtext{Artikel\pwindex{Maerchen@\emph{Ein Märchen}|pwv}}{\lemma{\textnormal{\emph{Artikel}}}\Cendnote{\textnormal{Laura Marholm\pwindex{Marholm, Laura 19.04.1854 – 06.10.1928@\textsc{Marholm, Laura} (19.04.1854 – 06.10.1928), \emph{Schriftsteller/Schriftstellerin}|pwk}: \emph{Ein Märchen}\pwindex{Maerchen@\emph{Ein Märchen}|pwk}. In: \emph{Die
                        Zukunft}\pwindex{Zukunft@\emph{Die Zukunft}|pwk}, Jg. 8, 25. 8. 1894, S. 368–371.}}}\label{K_L00368-1} der
                  Marholm\pwindex{Marholm, Laura 19.04.1854 – 06.10.1928@\textsc{Marholm, Laura} (19.04.1854 – 06.10.1928), \emph{Schriftsteller/Schriftstellerin}|pw} iſt ſehr ſchön, ſehr werthvoll
               beſonders. – Hieſs »Ein Märchen\pwindex{Maerchen@\emph{Ein Märchen}|pw}« und beſchäftigt
               ſich nach 1 ½ Seiten allg. Einleitung auf 2 ½ Seiten {\pb}mit mir. – (Beſtellt; Sie kriegen ihn da{\geminationn})\pend
           
\pstart
           6.) Vergeſſen Sie nicht mir den Stock, welcher in Ihrer Hand ſo elegant wird, nach
                  Wien\oindex{Wien@\textbf{Wien}, \emph{A.ADM2}|pw} zu ſchicken.\pend
           
\pstart
           7.) Glücklicher! –\pend
           
\pstart
           Herzliche Grüße Ihr{\\[\baselineskip]}\spacefill\mbox{Arthur}\pend
           \leftskip=0em{}
\pstart
           9. Sept. 94{ }Wien\oindex{Wien@\textbf{Wien}, \emph{A.ADM2}|pw}.\pend
           \selectlanguage{ngerman}\endnumbering\briefempfaengerindex{Beer-Hofmann, Richard@\textsc{Beer-Hofmann, Richard}!zzzSchnitzler, Arthur@\emph{von Arthur Schnitzler}!1894-09-091@{9. 9. 1894}|)be}\mylabel{L00368h}  \normalsize

\doendnotes{C}
\bigskip
\vfill

\clearpage

\footnotesize

\lohead{\textsc{register}}

% Definiere theindex-Environment komplett neu ohne reledmac
\makeatletter
\renewenvironment{theindex}{%
  \section*{\indexname}%
  \setlength{\parindent}{0pt}%
  \setlength{\parskip}{0pt plus 0.3pt}%
  \let\item\@idxitem
}{%
  \clearpage
}
\makeatother

\IfFileExists{\jobname-pw.ind}{\input{\jobname-pw.ind}}{}

\end{document}

      