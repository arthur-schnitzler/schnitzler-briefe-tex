%% latex-leseansicht-vorspann.tex
%% Vorspann für die Leseansicht.
%% Lädt die gemeinsame Datei latex-vorspann.tex mit nicht gesetztem Schalter.

\newif\ifkorrekturansicht
\korrekturansichtfalse

\input{../tex-inputs/latex-vorspann}


               \section[Arthur Schnitzler an Richard Beer-Hofmann, 9. 9. 1894]{ Arthur Schnitzler an Richard Beer-Hofmann, 9. 9. 1894}\nopagebreak\mylabel{v}\rehead{ }\begin{ledgroupsized}[t]{13cm}\normalsize\beginnumbering\briefempfaengerindex{Beer-Hofmann, Richard@\textsc{Beer-Hofmann, Richard}!zzzSchnitzler, Arthur@\emph{von Arthur Schnitzler}!1894-09-091@{9. 9. 1894}|(be} \toendnotes[C]{\smallbreak\pagebreak[2]} \Standort{YCGL, MSS 31.}
\physDesc{Brief, 1 Blatt, 2 Seiten, Umschlag
\newline{}Handschrift: Bleistift, deutsche Kurrent\newline{}Versand: 1) Stempel: »\nobreak{}\oindex{IX., Alsergrund@\textbf{IX., Alsergrund}|pwk}Wien 9/3, 9. 9. 94, 3–4 N\nobreak{}«.  2) Stempel: »\nobreak{}\oindex{Bad Ischl@\textbf{Bad Ischl}|pwk}Ischl, 10/9 9{[}4{]}, 7 F\nobreak{}«. }\buchAbdrucke{\weitereDrucke{Arthur Schnitzler, Richard Beer-Hofmann: \emph{Briefwechsel 1891–1931}. Hg. Konstanze Fliedl. Wien, Zürich: \emph{Europaverlag} 1992, S. 59.} }\toendnotes[C]{\smallbreak}\pstart{}{\pb}Herrn Dr. \textsc{Richard
                     Beer-Hofmann}\pend{}\pstart{}\textsc{Ischl\oindex{Bad Ischl@\textbf{Bad Ischl}|pw}}\pend{}\pstart{}\textsc{Egelmoos 22\oindex{Eglmoosgasse@\textbf{Eglmoosgasse}|pw}}\pend{}{\bigskip}\pstart{}{\pb}Lieber Richard,\pend\pstart
           1) Bolgar\pwindex{Bolgár, Franz von 03.01.1851 – 23.05.1923@\textsc{Bolgár, Franz von} (03.01.1851 – 23.05.1923), \emph{Politiker, Publizist}!Regeln des Duells1880 – 1880@\strich\emph{Die Regeln des Duells} {[}1880 – 1880{]}|pwv}\pwindex{Bolgár, Franz von 03.01.1851 – 23.05.1923@\textsc{Bolgár, Franz von} (03.01.1851 – 23.05.1923), \emph{Politiker, Publizist}|pw} geht eben unter Kreuzband ab.\pend
           \pstart
           2.) an P. Horn\pwindex{Horn, Paul 13.02.1867 – 18.01.1936@\textsc{Horn, Paul} (13.02.1867 – 18.01.1936), \emph{Fabrikant}|pw}{ }ſchrieb ich, weil Schenker\orgindex{Schenker und Co.@Schenker {\kaufmannsund}  Co.|pw} immer beſetzt iſt und das telefoniren mich nervös macht. Ich bat
               ihn, Ihnen direct ſofort zu antworten.\pend
           \pstart
           3.) Bahr\pwindex{Bahr, Hermann 19.07.1863 – 15.01.1934@\textsc{Bahr, Hermann} (19.07.1863 – 15.01.1934), \emph{Schriftsteller, Kritiker}|pw} werde ich morgen ſprechen.\pend
           \pstart
           4.) Adele S.\pwindex{Sandrock, Adele 19.08.1863 – 30.08.1937@\textsc{Sandrock, Adele} (19.08.1863 – 30.08.1937), \emph{Schauspielerin}|pw} wohnt Opernring 19\oindex{Opernring@\textbf{Opernring}|pw}.\pend
           \pstart
           5.) Der \label{K_L00368-4v}\edtext{Artikel\pwindex{Marholm, Laura 19.04.1854 – 06.10.1928@\textsc{Marholm, Laura} (19.04.1854 – 06.10.1928), \emph{Schriftstellerin}!Maerchen25.8.1894 – 25.8.1894@\strich\emph{Ein Märchen} {[}25.8.1894 – 25.8.1894{]}|pwv}}{\lemma{\textnormal{\emph{Artikel}}}\Cendnote{\textnormal{Laura Marholm\pwindex{Marholm, Laura 19.04.1854 – 06.10.1928@\textsc{Marholm, Laura} (19.04.1854 – 06.10.1928), \emph{Schriftstellerin}|pwk}: \emph{Ein Märchen}\pwindex{Marholm, Laura 19.04.1854 – 06.10.1928@\textsc{Marholm, Laura} (19.04.1854 – 06.10.1928), \emph{Schriftstellerin}!Maerchen25.8.1894 – 25.8.1894@\strich\emph{Ein Märchen} {[}25.8.1894 – 25.8.1894{]}|pwk}. In: \emph{Die Zukunft}\pwindex{Zukunft1892 – 1922@\emph{Die Zukunft}|pwk}, Jg. 8, 25. 8. 1894, S. 368–371.}}}\label{K_L00368-4h} der Marholm\pwindex{Marholm, Laura 19.04.1854 – 06.10.1928@\textsc{Marholm, Laura} (19.04.1854 – 06.10.1928), \emph{Schriftstellerin}|pw} iſt  ſehr ſchön, ſehr werthvoll
               beſonders. – Hieſs »Ein Märchen\pwindex{Marholm, Laura 19.04.1854 – 06.10.1928@\textsc{Marholm, Laura} (19.04.1854 – 06.10.1928), \emph{Schriftstellerin}!Maerchen25.8.1894 – 25.8.1894@\strich\emph{Ein Märchen} {[}25.8.1894 – 25.8.1894{]}|pw}« und beſchäftigt
               ſich nach 1 ½ Seiten allg. Einleitung auf 2 ½ Seiten {\pb}mit mir. – (Beſtellt; Sie kriegen ihn da{\geminationn})\pend
           \pstart
           6.) Vergeſſen Sie nicht mir den Stock, welcher in Ihrer Hand ſo elegant wird, nach
                  Wien\oindex{Wien@\textbf{Wien}|pw} zu ſchicken.\pend
           \pstart
           7.) Glücklicher! –\pend
           \pstart
           Herzliche Grüße Ihr{\\[\baselineskip]}\spacefill\mbox{Arthur}\pend
           \leftskip=0em{}\pstart
           9. Sept. 94{ }Wien\oindex{Wien@\textbf{Wien}|pw}.\pend
           \endnumbering\briefempfaengerindex{Beer-Hofmann, Richard@\textsc{Beer-Hofmann, Richard}!zzzSchnitzler, Arthur@\emph{von Arthur Schnitzler}!1894-09-091@{9. 9. 1894}|)be}\mylabel{h}\end{ledgroupsized}  \newcommand{\dateiname}{L00368}\newcommand{\titel}{Arthur Schnitzler an Richard Beer-Hofmann, 9. 9. 1894}\newcommand{\editorInnen}{ Martin Anton Müller und Gerd-Hermann Susen}
            \footnotesize
\begin{ledgroupsized}[t]{11.5cm}
\doendnotes{C}
\end{ledgroupsized}
         %% latex-leseansicht-abspann.tex
%% Abspann für die Leseansicht.
%% Der Schalter \ifkorrekturansicht ist bereits durch den Vorspann gesetzt.

%% latex-abspann.tex
%% Gemeinsamer Abspann für Korrekturansicht und Leseansicht.
%% Setzt den Schalter \ifkorrekturansicht voraus (gesetzt in den
%% einbindenden Dateien latex-korrekturansicht-abspann.tex bzw.
%% latex-leseansicht-abspann.tex).
%% ---------------------------------------------------------------

\normalsize

% Das esempio-Environment wird nur in der Leseansicht benötigt
\ifkorrekturansicht\else
\newenvironment{esempio}[3]%
{
    \vspace{1.5ex}
    \rlap{\underline{#1}}
    \par
    \setlength{\parindent}{0cm}
    \nopagebreak
    \leftskip=#2cm
    \rightskip=#3cm
}
{
    \par
}
\fi

\doendnotes{C}
\bigskip
\vfill

\clearpage

\footnotesize

\ifkorrekturansicht
  \lohead{\textsc{register}}
\fi

% theindex-Environment neu definieren ohne reledmac
\makeatletter
\renewenvironment{theindex}{%
  \ifkorrekturansicht
    \section*{\indexname}%
  \else
    \subsubsection*{Index der erwähnten Entitäten}%
  \fi
  \setlength{\parindent}{0pt}%
  \setlength{\parskip}{0pt plus 0.3pt}%
  \let\item\@idxitem
}{%
  \ifkorrekturansicht\clearpage\fi
}
\makeatother

\IfFileExists{\jobname-pw.ind}{\input{\jobname-pw.ind}}{}

% Quellenangabe nur in der Leseansicht
\ifkorrekturansicht\else
% Fallback-Definitionen, falls die .tex-Datei \titel etc. nicht gesetzt hat
\providecommand{\titel}{}
\providecommand{\editorInnen}{}
\providecommand{\dateiname}{\jobname}

\vspace{3cm}

\vfill

\footnotesize
\textsc{Quelle}: \titel. Herausgegeben von {\editorInnen}. In: \emph{Arthur Schnitzler: Briefwechsel mit Autorinnen und Autoren}.
 Digitale Edition, https://schnitzler-briefe.acdh.oeaw.ac.at/{\dateiname}.html (Stand \today)
\fi

\end{document}


      