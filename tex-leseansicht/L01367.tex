%% latex-leseansicht-vorspann.tex
%% Vorspann für die Leseansicht.
%% Lädt die gemeinsame Datei latex-vorspann.tex mit nicht gesetztem Schalter.

\newif\ifkorrekturansicht
\korrekturansichtfalse

\input{../tex-inputs/latex-vorspann}


\section[Hermann Bahr an Arthur Schnitzler, 29. 1. 1904]{L01367 Hermann Bahr an Arthur Schnitzler, 29. 1. 1904}
\nopagebreak\mylabel{L01367v}
\rehead{ }\normalsize\beginnumbering\briefempfaengerindex{Schnitzler, Arthur@\textsc{Schnitzler, Arthur}!zzzBahr, Hermann@\emph{von Hermann Bahr}!1904-01-291@{29. 1. 1904}|(be}
\toendnotes[C]{\smallbreak\pagebreak[2]}
\correspDesc{Versand  durch Hermann Bahr am 29. 1. 1904 in Marbach am Bodensee
\newline{}Erhalt  durch Arthur Schnitzler im Zeitraum [30. 1. 1904
                  – 3. 2. 1904?] in Wien}\toendnotes[C]{\smallbreak}
\Standort{CUL, Schnitzler, B 5b.}
\physDesc{Brief, 1 Blatt, 2 Seiten, 1363 Zeichen
\newline{}Handschrift: schwarze Tinte, deutsche Kurrent
\newline{}Ordnung: mit Bleistift von unbekannter Hand nummeriert:
                                    »108« }
\buchAbdrucke{\weitereDrucke{Hermann Bahr, Arthur Schnitzler: \emph{Briefwechsel, Aufzeichnungen, Dokumente (1891–1931)}. Herausgegeben von Kurt Ifkovits und Martin Anton Müller. Göttingen: \emph{Wallstein} 2018, S. 292–293.} }\toendnotes[C]{\smallbreak}
\pstart
           \raggedleft{}{\pb}29. 1. 04\pend
           
\pstart\center{}Lieber Arthur!\pend\vspace{0.5em}
\pstart
           Ich »ſoll« nach Ortner\pwindex{Ortner-Rodenstätt, Norbert von 10.\,8.\,1865 Linz – 1.\,3.\,1935 Salzburg@\textsc{Ortner-Rodenstätt, Norbert von} (10.\,8.\,1865 Linz – 1.\,3.\,1935 Salzburg), \emph{Mediziner, Internist, Kardiologe}|pw} zwei bis drei Monate
               hier bleiben, glaube aber nicht es{ }ſo lang auszuhalten. Es iſt hier{ }ſehr unangenehm
               und ich überlege hin und her, \damage{ob} es nicht viel geſcheiter wäre, in Taormina\oindex{Taormina@\textbf{Taormina}, \emph{Hauptstadt}|pw} oder Kairo\oindex{Kairo@\textbf{Kairo}, \emph{Hauptstadt}|pw}{ }z\damage{u}{ }ſitzen. Ich tue übrigens nichts, ohne vorher Ortner\pwindex{Ortner-Rodenstätt, Norbert von 10.\,8.\,1865 Linz – 1.\,3.\,1935 Salzburg@\textsc{Ortner-Rodenstätt, Norbert von} (10.\,8.\,1865 Linz – 1.\,3.\,1935 Salzburg), \emph{Mediziner, Internist, Kardiologe}|pw} zu{ }ſchreiben.\pend
           
\pstart
           Der »einſame Weg\pwindex{Schnitzler, Arthur 15.\,5.\,1862 Wien – 21.\,10.\,1931 ebd.@\textsc{Schnitzler, Arthur} (15.\,5.\,1862 Wien – 21.\,10.\,1931 ebd.), \emph{Schriftsteller, Mediziner}!einsame Weg. Schauspiel in fünf Akten@\strich\emph{Der einsame Weg. Schauspiel in fünf Akten}|pw}« kam geſtern an und wurde
               sogleich geleſen. Wunderbar finde ich, wie Du da von der Peripherie der Menſchheit,
               an welcher{ }ſich die meiſten Stücke{ }ſonſt herumbewegen, in die Mitte ihres geiſtigen
               Lebens kommſt, nemlich unſeres geiſtigen Lebens, der Sachen, um die wir uns heute
               allein noch kümmern können. (Wobei ich mich an einen Satz Maeterlincks\pwindex{Maeterlinck, Maurice 29.\,8.\,1862 Gent – 6.\,5.\,1949 Nizza@\textsc{Maeterlinck, Maurice} (29.\,8.\,1862 Gent – 6.\,5.\,1949 Nizza), \emph{Schriftsteller}|pw} von dem \label{K_L01367-1v}\edtext{ſtill an{ }ſeinem Tiſche{ }ſitzenden Greiſe}{\lemma{\textnormal{\emph{still … Greise}}}\Cendnote{\textnormal{In \emph{À propos
                     de Solness le Constructeur}\pwindex{Maeterlinck, Maurice 29.\,8.\,1862 Gent – 6.\,5.\,1949 Nizza@\textsc{Maeterlinck, Maurice} (29.\,8.\,1862 Gent – 6.\,5.\,1949 Nizza), \emph{Schriftsteller}!À propos de Solness le Constructeur [Le Tragique quotidien]@\strich\emph{À propos de Solness le Constructeur [Le Tragique quotidien]}|pwk} (\emph{Le Figaro}\orgindex{Le Figaro@Le Figaro|pwk}, Jg. 40, Ser. 3, Nr. 92,
                        2. 4. 1894, S. 1, späterer Titel \emph{Le Tragique quotidien}\pwindex{Maeterlinck, Maurice 29.\,8.\,1862 Gent – 6.\,5.\,1949 Nizza@\textsc{Maeterlinck, Maurice} (29.\,8.\,1862 Gent – 6.\,5.\,1949 Nizza), \emph{Schriftsteller}!À propos de Solness le Constructeur [Le Tragique quotidien]@\strich\emph{À propos de Solness le Constructeur [Le Tragique quotidien]}|pwk}) schreibt Maeterlinck\pwindex{Maeterlinck, Maurice 29.\,8.\,1862 Gent – 6.\,5.\,1949 Nizza@\textsc{Maeterlinck, Maurice} (29.\,8.\,1862 Gent – 6.\,5.\,1949 Nizza), \emph{Schriftsteller}|pwk} über das
                  »tiefere Leben« eines Alten, der in seinem Stuhl versucht, seine Umgebung zu
                  begreifen, im Vergleich beispielsweise zu einem Liebhaber, der die Geliebte
                  erwürgt.}}}\label{K_L01367-1} und an manches erinnere, was in meinem Dialog vom Tragiſchen\pwindex{Bahr, Hermann 19.\,7.\,1863 Linz – 15.\,1.\,1934 München@\textsc{Bahr, Hermann} (19.\,7.\,1863 Linz – 15.\,1.\,1934 München), \emph{Schriftsteller, Kritiker}!Dialog vom Tragischen@\strich\emph{Dialog vom Tragischen}|pw} gefordert wird). Allerdings {\pb}hat mir geſtern, beim erſten eiligen Leſen und in
               meiner jetzigen geiſtigen Trübung, im dramatiſchen Ductus etwas gefehlt, ich kann es
               nicht anders sagen, als daß mir die Bewegung des Stückes einige Male abzubrechen und{ }ſich dann auf eine mir nicht gleich verſtändliche Art wieder zu{ }ſammeln oder zu
               erſetzen{ }ſchien. Ich leſe es nun aber in ein paar Tagen wieder und mit dieſen
               Bemerkungen iſt wol überhaupt mehr mein elender Zuſtand als das Stück kritiſiert.\pend
           
\pstart
           Grüß Brahm\pwindex{Brahm, Otto 5.\,2.\,1856 Hamburg – 28.\,11.\,1912 Berlin@\textsc{Brahm, Otto} (5.\,2.\,1856 Hamburg – 28.\,11.\,1912 Berlin), \emph{Theaterleiter, Regisseur}|pw} und wen ich{ }ſonſt in Berlin\oindex{Berlin@\textbf{Berlin}, \emph{Hauptstadt}|pw} kenne, empfiel mich Deiner Frau\pwindex{Schnitzler, Olga 17.\,1.\,1882 Wien – 13.\,1.\,1970 Lugano@\textsc{Schnitzler, Olga} (17.\,1.\,1882 Wien – 13.\,1.\,1970 Lugano), \emph{Schauspielerin, Sängerin}|pwv} und{ }ſei herzlichſt
               gegrüßt von{\\[\baselineskip]}Deinem alten{\\[\baselineskip]}\spacefill\mbox{Hermann}\pend
           \leftskip=0em{}\selectlanguage{ngerman}\endnumbering\briefempfaengerindex{Schnitzler, Arthur@\textsc{Schnitzler, Arthur}!zzzBahr, Hermann@\emph{von Hermann Bahr}!1904-01-291@{29. 1. 1904}|)be}\mylabel{L01367h}  \newcommand{\dateiname}{L01367}\newcommand{\titel}{Hermann Bahr an Arthur Schnitzler, 29. 1. 1904}\newcommand{\editorInnen}{Herausgegeben von Martin Anton Müller}%% latex-leseansicht-abspann.tex
%% Abspann für die Leseansicht.
%% Der Schalter \ifkorrekturansicht ist bereits durch den Vorspann gesetzt.

%% latex-abspann.tex
%% Gemeinsamer Abspann für Korrekturansicht und Leseansicht.
%% Setzt den Schalter \ifkorrekturansicht voraus (gesetzt in den
%% einbindenden Dateien latex-korrekturansicht-abspann.tex bzw.
%% latex-leseansicht-abspann.tex).
%% ---------------------------------------------------------------

\normalsize

% Das esempio-Environment wird nur in der Leseansicht benötigt
\ifkorrekturansicht\else
\newenvironment{esempio}[3]%
{
    \vspace{1.5ex}
    \rlap{\underline{#1}}
    \par
    \setlength{\parindent}{0cm}
    \nopagebreak
    \leftskip=#2cm
    \rightskip=#3cm
}
{
    \par
}
\fi

\doendnotes{C}
\bigskip
\vfill

\clearpage

\footnotesize

\ifkorrekturansicht
  \lohead{\textsc{register}}
\fi

% theindex-Environment neu definieren ohne reledmac
\makeatletter
\renewenvironment{theindex}{%
  \ifkorrekturansicht
    \section*{\indexname}%
  \else
    \subsubsection*{Index der erwähnten Entitäten}%
  \fi
  \setlength{\parindent}{0pt}%
  \setlength{\parskip}{0pt plus 0.3pt}%
  \let\item\@idxitem
}{%
  \ifkorrekturansicht\clearpage\fi
}
\makeatother

\IfFileExists{\jobname-pw.ind}{\input{\jobname-pw.ind}}{}

% Quellenangabe nur in der Leseansicht
\ifkorrekturansicht\else
% Fallback-Definitionen, falls die .tex-Datei \titel etc. nicht gesetzt hat
\providecommand{\titel}{}
\providecommand{\editorInnen}{}
\providecommand{\dateiname}{\jobname}

\vspace{3cm}

\vfill

\footnotesize
\textsc{Quelle}: \titel. Herausgegeben von {\editorInnen}. In: \emph{Arthur Schnitzler: Briefwechsel mit Autorinnen und Autoren}.
 Digitale Edition, https://schnitzler-briefe.acdh.oeaw.ac.at/{\dateiname}.html (Stand \today)
\fi

\end{document}


