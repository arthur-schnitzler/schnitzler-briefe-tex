%% latex-korrekturansicht-vorspann.tex
%% Vorspann für die Korrekturansicht.
%% Lädt die gemeinsame Datei latex-vorspann.tex mit gesetztem Schalter.

\newif\ifkorrekturansicht
\korrekturansichttrue

\input{../tex-inputs/latex-vorspann}


\section[Hermann Bahr an Arthur Schnitzler, 29. 1. 1904]{L01367 Hermann Bahr an Arthur Schnitzler, 29. 1. 1904}
\nopagebreak\mylabel{L01367v}
\rehead{ }\normalsize\beginnumbering\briefempfaengerindex{Schnitzler, Arthur@\textsc{Schnitzler, Arthur}!zzzBahr, Hermann@\emph{von Hermann Bahr}!1904-01-291@{29. 1. 1904}|(be}
\toendnotes[C]{\smallbreak\pagebreak[2]}\Standort{CUL, Schnitzler, B 5b.}
\physDesc{Brief, 1 Blatt, 2 Seiten, 1363 Zeichen
\newline{}Handschrift: schwarze Tinte, deutsche Kurrent
\newline{}Ordnung: mit Bleistift von unbekannter Hand nummeriert:
                                    »108« }
\buchAbdrucke{\weitereDrucke{Hermann Bahr, Arthur Schnitzler: \emph{Briefwechsel, Aufzeichnungen, Dokumente (1891–1931)}. Göttingen: \emph{Wallstein} 2018, S. 292–293.} }\toendnotes[C]{\smallbreak}
\pstart
           \raggedleft{}{\pb}29. 1. 04\pend
           
\pstart\center{}Lieber Arthur!\pend\vspace{0.5em}
\pstart
           Ich »ſoll« nach Ortner\pwindex{Ortner-Rodenstaett, Norbert von 10.08.1865 – 01.03.1935@\textsc{Ortner-Rodenstätt, Norbert von} (10.08.1865 – 01.03.1935), \emph{Mediziner/Medizinerin, Internist/Internistin, Kardiologe/Kardiologin}|pw} zwei bis drei Monate
               hier bleiben, glaube aber nicht es ſo lang auszuhalten. Es iſt hier ſehr unangenehm
               und ich überlege hin und her, \damage{ob} es nicht viel geſcheiter wäre, in Taormina\oindex{Taormina@\textbf{Taormina}, \emph{P.PPLA3}|pw} oder Kairo\oindex{Kairo@\textbf{Kairo}, \emph{P.PPLC}|pw}{ }z\damage{u}{ }ſitzen. Ich tue übrigens nichts, ohne vorher Ortner\pwindex{Ortner-Rodenstaett, Norbert von 10.08.1865 – 01.03.1935@\textsc{Ortner-Rodenstätt, Norbert von} (10.08.1865 – 01.03.1935), \emph{Mediziner/Medizinerin, Internist/Internistin, Kardiologe/Kardiologin}|pw} zu ſchreiben.\pend
           
\pstart
           Der »einſame Weg\pwindex{einsame Weg. Schauspiel in fuenf Akten@\emph{Der einsame Weg. Schauspiel in fünf Akten}|pw}« kam geſtern an und wurde
               sogleich geleſen. Wunderbar finde ich, wie Du da von der Peripherie der Menſchheit,
               an welcher ſich die meiſten Stücke ſonſt herumbewegen, in die Mitte ihres geiſtigen
               Lebens kommſt, nemlich unſeres geiſtigen Lebens, der Sachen, um die wir uns heute
               allein noch kümmern können. (Wobei ich mich an einen Satz Maeterlincks\pwindex{Maeterlinck, Maurice 29.08.1862 – 06.05.1949@\textsc{Maeterlinck, Maurice} (29.08.1862 – 06.05.1949), \emph{Schriftsteller/Schriftstellerin}|pw} von dem \label{K_L01367-1v}\edtext{ſtill an ſeinem Tiſche ſitzenden Greiſe}{\lemma{\textnormal{\emph{ſtill … Greiſe}}}\Cendnote{\textnormal{In \emph{À propos
                     de Solness le Constructeur}\pwindex{À propos de Solness le Constructeur [Le Tragique quotidien]@\emph{À propos de Solness le Constructeur [Le Tragique quotidien]}|pwk} (\emph{Le Figaro}\orgindex{Le Figaro@Le Figaro|pwk}, Jg. 40, Ser. 3, Nr. 92,
                        2. 4. 1894, S. 1, späterer Titel \emph{Le Tragique quotidien}\pwindex{À propos de Solness le Constructeur [Le Tragique quotidien]@\emph{À propos de Solness le Constructeur [Le Tragique quotidien]}|pwk}) schreibt Maeterlinck\pwindex{Maeterlinck, Maurice 29.08.1862 – 06.05.1949@\textsc{Maeterlinck, Maurice} (29.08.1862 – 06.05.1949), \emph{Schriftsteller/Schriftstellerin}|pwk} über das
                  »tiefere Leben« eines Alten, der in seinem Stuhl versucht, seine Umgebung zu
                  begreifen, im Vergleich beispielsweise zu einem Liebhaber, der die Geliebte
                  erwürgt.}}}\label{K_L01367-1} und an manches erinnere, was in meinem Dialog vom Tragiſchen\pwindex{Dialog vom Tragischen@\emph{Dialog vom Tragischen}|pw} gefordert wird). Allerdings {\pb}hat mir geſtern, beim erſten eiligen Leſen und in
               meiner jetzigen geiſtigen Trübung, im dramatiſchen Ductus etwas gefehlt, ich kann es
               nicht anders sagen, als daß mir die Bewegung des Stückes einige Male abzubrechen und
               ſich dann auf eine mir nicht gleich verſtändliche Art wieder zu ſammeln oder zu
               erſetzen ſchien. Ich leſe es nun aber in ein paar Tagen wieder und mit dieſen
               Bemerkungen iſt wol überhaupt mehr mein elender Zuſtand als das Stück kritiſiert.\pend
           
\pstart
           Grüß Brahm\pwindex{Brahm, Otto 05.02.1856 – 28.11.1912@\textsc{Brahm, Otto} (05.02.1856 – 28.11.1912), \emph{Theaterleiter/Theaterleiterin, Regisseur/Regisseurin}|pw} und wen ich ſonſt in Berlin\oindex{Berlin@\textbf{Berlin}, \emph{P.PPLC}|pw} kenne, empfiel mich Deiner Frau\pwindex{Schnitzler, Olga 17.01.1882 – 13.01.1970@\textsc{Schnitzler, Olga} (17.01.1882 – 13.01.1970), \emph{Schauspieler/Schauspielerin, Sänger/Sängerin}|pwv} und ſei herzlichſt
               gegrüßt von{\\[\baselineskip]}Deinem alten{\\[\baselineskip]}\spacefill\mbox{Hermann}\pend
           \leftskip=0em{}\selectlanguage{ngerman}\endnumbering\briefempfaengerindex{Schnitzler, Arthur@\textsc{Schnitzler, Arthur}!zzzBahr, Hermann@\emph{von Hermann Bahr}!1904-01-291@{29. 1. 1904}|)be}\mylabel{L01367h}  \normalsize

\doendnotes{C}
\bigskip
\vfill

\clearpage

\footnotesize

\lohead{\textsc{register}}

% Definiere theindex-Environment komplett neu ohne reledmac
\makeatletter
\renewenvironment{theindex}{%
  \section*{\indexname}%
  \setlength{\parindent}{0pt}%
  \setlength{\parskip}{0pt plus 0.3pt}%
  \let\item\@idxitem
}{%
  \clearpage
}
\makeatother

\IfFileExists{\jobname-pw.ind}{\input{\jobname-pw.ind}}{}

\end{document}

      