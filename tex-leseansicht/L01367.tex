%% latex-leseansicht-vorspann.tex
%% Vorspann für die Leseansicht.
%% Lädt die gemeinsame Datei latex-vorspann.tex mit nicht gesetztem Schalter.

\newif\ifkorrekturansicht
\korrekturansichtfalse

\input{../tex-inputs/latex-vorspann}


         
         \renewcommand{\erwaehntePersonen}{Personen: Otto Brahm, Maurice Maeterlinck, Norbert von Ortner-Rodenstätt, Olga Schnitzler}
         \renewcommand{\erwaehnteInstitutionen}{Institutionen: Le Figaro}
         \renewcommand{\erwaehnteOrte}{Orte: Berlin, Kairo, Marbach am Bodensee, Taormina, Wien}
         \renewcommand{\erwaehnteWerke}{Werke: Der einsame Weg. Schauspiel in fünf Akten, Dialog vom Tragischen, À propos de Solness le Constructeur [Le Tragique quotidien]}
               \section[Hermann Bahr an Arthur Schnitzler, 29. 1. 1904]{ Hermann Bahr an Arthur Schnitzler, 29. 1. 1904}\nopagebreak\mylabel{v}\rehead{ }\begin{ledgroupsized}[t]{13cm}\normalsize\beginnumbering \toendnotes[C]{\smallbreak\pagebreak[2]} \Standort{CUL, Schnitzler, B 5b.}
\physDesc{Brief, 1 Blatt, 2 Seiten
\newline{}Handschrift: schwarze Tinte, deutsche Kurrent\newline{}Ordnung: mit Bleistift von unbekannter Hand nummeriert: »108« }\buchAbdrucke{\weitereDrucke{Hermann Bahr, Arthur Schnitzler: \emph{Briefwechsel, Aufzeichnungen, Dokumente (1891–1931)}. Hg. Kurt Ifkovits und Martin Anton Müller. Göttingen: \emph{Wallstein} 2018, S. 292–293.} }\toendnotes[C]{\smallbreak}\pstart
           \raggedleft{}{\pb}29. 1. 04\pend
           \pstart\center{}Lieber Arthur!\pend\pstart
           Ich »ſoll« nach Ortner\pwindex{Ortner-Rodenstaett, Norbert von 10.08.1865 – 01.03.1935@\textsc{Ortner-Rodenstätt, Norbert von} (10.08.1865 – 01.03.1935), \emph{Mediziner, Internist, Kardiologe}|pw} zwei bis drei Monate hier
               bleiben, glaube aber nicht es ſo lang auszuhalten. Es iſt hier ſehr unangenehm und
               ich überlege hin und her, \damage{ob} es nicht viel geſcheiter wäre, in Taormina\oindex{Taormina@\textbf{Taormina}|pw}
               oder Kairo\oindex{Kairo@\textbf{Kairo}|pw}{ }z\damage{u}{ }ſitzen. Ich tue übrigens nichts, ohne vorher Ortner\pwindex{Ortner-Rodenstaett, Norbert von 10.08.1865 – 01.03.1935@\textsc{Ortner-Rodenstätt, Norbert von} (10.08.1865 – 01.03.1935), \emph{Mediziner, Internist, Kardiologe}|pw} zu ſchreiben.\pend
           \pstart
           Der »einſame Weg\pwindex{Schnitzler, Arthur 15.05.1862 – 21.10.1931@\textsc{Schnitzler, Arthur} (15.05.1862 – 21.10.1931), \emph{Schriftsteller, Mediziner}!einsame Weg. Schauspiel in fuenf Akten1904@\strich\emph{Der einsame Weg. Schauspiel in fünf Akten} {[}1904{]}|pw}« kam geſtern an und wurde sogleich
               geleſen. Wunderbar finde ich, wie Du da von der Peripherie der Menſchheit, an welcher
               ſich die meiſten Stücke ſonſt herumbewegen, in die Mitte ihres geiſtigen Lebens
               kommſt, nemlich unſeres geiſtigen Lebens, der Sachen, um die wir uns heute allein
               noch kümmern können. (Wobei ich mich an einen Satz Maeterlincks\pwindex{Maeterlinck, Maurice 29.08.1862 – 06.05.1949@\textsc{Maeterlinck, Maurice} (29.08.1862 – 06.05.1949), \emph{Schriftsteller}|pw} von dem \label{K_L01367_1v}\edtext{ſtill an
               ſeinem Tiſche ſitzenden Greiſe}{\lemma{\textnormal{\emph{ſtill … Greiſe}}}\Cendnote{\textnormal{In \emph{À propos de Solness le Constructeur}\pwindex{Maeterlinck, Maurice 29.08.1862 – 06.05.1949@\textsc{Maeterlinck, Maurice} (29.08.1862 – 06.05.1949), \emph{Schriftsteller}!À propos de Solness le Constructeur [Le Tragique quotidien]02. 04. 1894@\strich\emph{À propos de Solness le Constructeur [Le Tragique quotidien]} {[}02. 04. 1894{]}|pwk} (\emph{Le Figaro}\orgindex{Le Figaro@Le Figaro|pwk}, Jg. 40, Ser. 3, Nr. 92,
                        2. 4. 1894, S. 1, späterer Titel
                        \emph{Le Tragique quotidien}\pwindex{Maeterlinck, Maurice 29.08.1862 – 06.05.1949@\textsc{Maeterlinck, Maurice} (29.08.1862 – 06.05.1949), \emph{Schriftsteller}!À propos de Solness le Constructeur [Le Tragique quotidien]02. 04. 1894@\strich\emph{À propos de Solness le Constructeur [Le Tragique quotidien]} {[}02. 04. 1894{]}|pwk}) schreibt Maeterlinck\pwindex{Maeterlinck, Maurice 29.08.1862 – 06.05.1949@\textsc{Maeterlinck, Maurice} (29.08.1862 – 06.05.1949), \emph{Schriftsteller}|pwk} über das
                  »tiefere Leben« eines Alten, der in seinem Stuhl versucht, seine Umgebung zu
                  begreifen, im Vergleich beispielsweise zu einem Liebhaber, der die Geliebte
                  erwürgt.}}}\label{K_L01367_1h} und an manches erinnere, was in meinem Dialog vom Tragiſchen\pwindex{Bahr, Hermann 19.07.1863 – 15.01.1934@\textsc{Bahr, Hermann} (19.07.1863 – 15.01.1934), \emph{Schriftsteller, Kritiker}!Dialog vom Tragischen01. 07. 1903@\strich\emph{Dialog vom Tragischen} {[}01. 07. 1903{]}|pw} gefordert wird). Allerdings {\pb}hat mir geſtern, beim erſten eiligen Leſen und in
               meiner jetzigen geiſtigen Trübung, im dramatiſchen Ductus etwas gefehlt, ich kann es
               nicht anders sagen, als daß mir die Bewegung des Stückes einige Male abzubrechen und
               ſich dann auf eine mir nicht gleich verſtändliche Art wieder zu ſammeln oder zu
               erſetzen ſchien. Ich leſe es nun aber in ein paar Tagen wieder und mit dieſen
               Bemerkungen iſt wol überhaupt mehr mein elender Zuſtand als das Stück kritiſiert.\pend
           \pstart
           Grüß Brahm\pwindex{Brahm, Otto 05.02.1856 – 28.11.1912@\textsc{Brahm, Otto} (05.02.1856 – 28.11.1912), \emph{Theaterleiter, Regisseur}|pw} und wen ich ſonſt in Berlin\oindex{Berlin@\textbf{Berlin}|pw} kenne, empfiel mich Deiner Frau\pwindex{Schnitzler, Olga 17.01.1882 – 13.01.1970@\textsc{Schnitzler, Olga} (17.01.1882 – 13.01.1970), \emph{Schauspielerin, Sängerin}|pwv} und ſei herzlichſt gegrüßt
               von{\\[\baselineskip]}Deinem alten{\\[\baselineskip]}\spacefill\mbox{Hermann}\pend
           \leftskip=0em{}
         
         \endnumbering\mylabel{h}\end{ledgroupsized}  \newcommand{\dateiname}{L01367}\newcommand{\titel}{Hermann Bahr an Arthur Schnitzler, 29. 1. 1904}\newcommand{\editorInnen}{ Kurt Ifkovits,  Martin Anton Müller}%% latex-leseansicht-abspann.tex
%% Abspann für die Leseansicht.
%% Der Schalter \ifkorrekturansicht ist bereits durch den Vorspann gesetzt.

%% latex-abspann.tex
%% Gemeinsamer Abspann für Korrekturansicht und Leseansicht.
%% Setzt den Schalter \ifkorrekturansicht voraus (gesetzt in den
%% einbindenden Dateien latex-korrekturansicht-abspann.tex bzw.
%% latex-leseansicht-abspann.tex).
%% ---------------------------------------------------------------

\normalsize

% Das esempio-Environment wird nur in der Leseansicht benötigt
\ifkorrekturansicht\else
\newenvironment{esempio}[3]%
{
    \vspace{1.5ex}
    \rlap{\underline{#1}}
    \par
    \setlength{\parindent}{0cm}
    \nopagebreak
    \leftskip=#2cm
    \rightskip=#3cm
}
{
    \par
}
\fi

\doendnotes{C}
\bigskip
\vfill

\clearpage

\footnotesize

\ifkorrekturansicht
  \lohead{\textsc{register}}
\fi

% theindex-Environment neu definieren ohne reledmac
\makeatletter
\renewenvironment{theindex}{%
  \ifkorrekturansicht
    \section*{\indexname}%
  \else
    \subsubsection*{Index der erwähnten Entitäten}%
  \fi
  \setlength{\parindent}{0pt}%
  \setlength{\parskip}{0pt plus 0.3pt}%
  \let\item\@idxitem
}{%
  \ifkorrekturansicht\clearpage\fi
}
\makeatother

\IfFileExists{\jobname-pw.ind}{\input{\jobname-pw.ind}}{}

% Quellenangabe nur in der Leseansicht
\ifkorrekturansicht\else
% Fallback-Definitionen, falls die .tex-Datei \titel etc. nicht gesetzt hat
\providecommand{\titel}{}
\providecommand{\editorInnen}{}
\providecommand{\dateiname}{\jobname}

\vspace{3cm}

\vfill

\footnotesize
\textsc{Quelle}: \titel. Herausgegeben von {\editorInnen}. In: \emph{Arthur Schnitzler: Briefwechsel mit Autorinnen und Autoren}.
 Digitale Edition, https://schnitzler-briefe.acdh.oeaw.ac.at/{\dateiname}.html (Stand \today)
\fi

\end{document}


      