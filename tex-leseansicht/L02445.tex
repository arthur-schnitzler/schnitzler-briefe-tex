%% latex-korrekturansicht-vorspann.tex
%% Vorspann für die Korrekturansicht.
%% Lädt die gemeinsame Datei latex-vorspann.tex mit gesetztem Schalter.

\newif\ifkorrekturansicht
\korrekturansichttrue

\input{../tex-inputs/latex-vorspann}


\section[Georg Brandes an Arthur Schnitzler, 12. 7. 1925]{L02445 Georg Brandes an Arthur Schnitzler, 12. 7. 1925}
\nopagebreak\mylabel{L02445v}
\rehead{ }\normalsize\beginnumbering\briefempfaengerindex{Schnitzler, Arthur@\textsc{Schnitzler, Arthur}!zzzBrandes, Georg@\emph{von Georg Brandes}!1925-07-121@{12. 7. 1925}|(be}
\toendnotes[C]{\smallbreak\pagebreak[2]}\Standort{CUL, Schnitzler, B 17.}
\physDesc{Briefkarte, 1525 Zeichen
\newline{}Handschrift: schwarze Tinte, lateinische Kurrent
\newline{}Schnitzler: mit rotem Buntstift vereinzelte Unterstreichungen 
\newline{}Ordnung: 1) mit Bleistift von unbekannter Hand nummeriert: »\strikeout{59}«  2) mit Bleistift von unbekannter Hand nummeriert:
                                    »60«}
\buchAbdrucke{\weitereDrucke{Georg Brandes, Arthur Schnitzler: \emph{Ein Briefwechsel}. Bern: \emph{Francke} 1956, S. 149–150.} }\toendnotes[C]{\smallbreak}
\pstart
           \raggedleft{}{\pb}Kopenhagen\oindex{Kopenhagen@\textbf{Kopenhagen}, \emph{P.PPLC}|pw}{ }12 Juli 25\pend
           \vspace{0.5em}
\pstart
           Freund! Haben Sie herzlichen Dank für herzliche Worte.\hspace*{2em}Unter Ihren Uebeln scheint einem 83 Greis das Ohrleiden
               das einzige ernste. Glücklicherweise ist es nicht schlimmer, als dass Sie sich
               gastfreundlich mit den Leuten unterhalten können, und theatralischen Erfolg erleben.
               Ich las kürzlich sehr genau aufs Neue \uline{Beatrice’s Schleier}\pwindex{Schleier der Beatrice. Schauspiel in fuenf Akten@\emph{Der Schleier der Beatrice. Schauspiel in fünf Akten}|pw} und fand darin \uline{Tiefen}, eine Einsicht in die
               Frauenseele, die ich nie gehabt hab und nie erwerben könnte. Bin dazu geschaffen von
               complicirteren Frauen an der Nase herumgeführt zu werden und nur die einfachen zu
               verstehen. – Sie sind und bleiben für mich der Angelpunkt Wiens\oindex{Wien@\textbf{Wien}, \emph{A.ADM2}|pw}. Da Sie mit vielen Menschen und mit dem Theater zu tun
               haben, kennen Sie nicht mein Loos, \uline{die Einsamkeit}.
               Alle fast sind gestorben, die mir nahe standen, alle {\pb}die wenigen, an die ich Vertrauen
               haben konnte. Und ich mache keine neue Bekanntschaften, habe zu viele Täuschungen
               erlitten. Unter uns – bitte, sagen Sie es Niemand – die sogenannte Menschheit ist
               eine abscheuliche Bande. Es gibt ja glücklicherweise einige Ausnahmen. – Kopenhagen\oindex{Kopenhagen@\textbf{Kopenhagen}, \emph{P.PPLC}|pw} ist im Sommer eine Wüste, aber ich
               mag nicht reisen, arbeite stetig, aber es ist »die Arbeit des schlechten Kopfes« wie
               mein alter Schuldirector\pwindex{?? [Schuldirektor von Georg Brandes] @\textsc{?? [Schuldirektor von Georg Brandes]}|pwv}
               sagte, wenn ich meine Irrthümer mit meinem Fleiss entschuldigen wollte. – \uline{Sie} haben doch wenigstens Erfolge aufzuweisen, in
               meinem Fach gibt es keine Erfolge; ich verkaufe 1500 oder 2000 Exemplare in meinem
               Patria und die Uebersetzungen bringen nichts ein.\hspace*{2em}Doch genug geheult und seien Sie innigst bedankt. Ihr\spacefill\mbox{Georg Brandes}\pend
           \selectlanguage{ngerman}\endnumbering\briefempfaengerindex{Schnitzler, Arthur@\textsc{Schnitzler, Arthur}!zzzBrandes, Georg@\emph{von Georg Brandes}!1925-07-121@{12. 7. 1925}|)be}\mylabel{L02445h}  \normalsize

\doendnotes{C}
\bigskip
\vfill

\clearpage

\footnotesize

\lohead{\textsc{register}}

% Definiere theindex-Environment komplett neu ohne reledmac
\makeatletter
\renewenvironment{theindex}{%
  \section*{\indexname}%
  \setlength{\parindent}{0pt}%
  \setlength{\parskip}{0pt plus 0.3pt}%
  \let\item\@idxitem
}{%
  \clearpage
}
\makeatother

\IfFileExists{\jobname-pw.ind}{\input{\jobname-pw.ind}}{}

\end{document}

      