%% latex-leseansicht-vorspann.tex
%% Vorspann für die Leseansicht.
%% Lädt die gemeinsame Datei latex-vorspann.tex mit nicht gesetztem Schalter.

\newif\ifkorrekturansicht
\korrekturansichtfalse

\input{../tex-inputs/latex-vorspann}


         
         \renewcommand{\erwaehntePersonen}{Personen:  ?? [Schuldirektor von Georg Brandes]}
         \renewcommand{\erwaehnteOrte}{Orte: Kopenhagen, Wien}
         \renewcommand{\erwaehnteWerke}{Werke: Der Schleier der Beatrice. Schauspiel in fünf Akten}
               \section[Georg Brandes an Arthur Schnitzler, 12. 7. 1925]{ Georg Brandes an Arthur Schnitzler, 12. 7. 1925}\nopagebreak\mylabel{v}\rehead{ }\begin{ledgroupsized}[t]{13cm}\normalsize\beginnumbering \toendnotes[C]{\smallbreak\pagebreak[2]} \Standort{CUL, Schnitzler, B 17.}
\physDesc{Briefkarte
\newline{}Handschrift: schwarze Tinte, lateinische Kurrent
\newline{}Schnitzler: mit rotem Buntstift vereinzelte Unterstreichungen \newline{}Ordnung: 1) mit Bleistift von unbekannter Hand nummeriert: »\strikeout{59}«  2) mit Bleistift von unbekannter Hand nummeriert:
                                    »60«}\buchAbdrucke{\weitereDrucke{Georg Brandes, Arthur Schnitzler: \emph{Ein Briefwechsel}. Hg. Kurt Bergel. Bern: \emph{Francke} 1956, S. 149–150.} }\toendnotes[C]{\smallbreak}\pstart
           \raggedleft{}{\pb}Kopenhagen\oindex{Kopenhagen@\textbf{Kopenhagen}|pw}{ }12 Juli 25\pend
           \pstart
           Freund! Haben Sie herzlichen Dank für herzliche Worte.\hspace*{2em}Unter Ihren Uebeln scheint einem 83 Greis das Ohrleiden
               das einzige ernste. Glücklicherweise ist es nicht schlimmer, als dass Sie sich
               gastfreundlich mit den Leuten unterhalten können, und theatralischen Erfolg erleben.
               Ich las kürzlich sehr genau aufs Neue \uline{Beatrice’s Schleier}\pwindex{Schnitzler, Arthur 15.05.1862 – 21.10.1931@\textsc{Schnitzler, Arthur} (15.05.1862 – 21.10.1931), \emph{Schriftsteller, Mediziner}!Schleier der Beatrice. Schauspiel in fuenf Akten1900-12-01@\strich\emph{Der Schleier der Beatrice. Schauspiel in fünf Akten} {[}1900-12-01{]}|pw} und fand darin \uline{Tiefen}, eine Einsicht in die
               Frauenseele, die ich nie gehabt hab und nie erwerben könnte. Bin dazu geschaffen von
               complicirteren Frauen an der Nase herumgeführt zu werden und nur die einfachen zu
               verstehen. – Sie sind und bleiben für mich der Angelpunkt Wien\oindex{Wien@\textbf{Wien}|pw}s. Da Sie mit vielen Menschen und mit dem Theater zu tun haben, kennen
               Sie nicht mein Loos, \uline{die Einsamkeit}. Alle fast sind
               gestorben, die mir nahe standen, alle {\pb}die wenigen, an die ich Vertrauen
               haben konnte. Und ich mache keine neue Bekanntschaften, habe zu viele Täuschungen
               erlitten. Unter uns – bitte, sagen Sie es Niemand – die sogenannte Menschheit ist
               eine abscheuliche Bande. Es gibt ja glücklicherweise einige Ausnahmen. – Kopenhagen\oindex{Kopenhagen@\textbf{Kopenhagen}|pw} ist im Sommer eine Wüste, aber ich mag
               nicht reisen, arbeite stetig, aber es ist »die Arbeit des schlechten Kopfes« wie mein
               alter Schuldirector\pwindex{?? [Schuldirektor von Georg Brandes] @\textsc{?? [Schuldirektor von Georg Brandes]}|pwv} sagte, wenn
               ich meine Irrthümer mit meinem Fleiss entschuldigen wollte. – \uline{Sie} haben doch wenigstens Erfolge aufzuweisen, in meinem Fach gibt es
               keine Erfolge; ich verkaufe 1500 oder 2000 Exemplare in meinem Patria und die
               Uebersetzungen bringen nichts ein.\hspace*{2em}Doch genug
               geheult und seien Sie innigst bedankt. Ihr\spacefill\mbox{Georg Brandes}\pend
           
         
         \endnumbering\mylabel{h}\end{ledgroupsized}  \newcommand{\dateiname}{L02445}\newcommand{\titel}{Georg Brandes an Arthur Schnitzler, 12. 7. 1925}\newcommand{\editorInnen}{Martin Anton Müller und Gerd-Hermann Susen}%% latex-leseansicht-abspann.tex
%% Abspann für die Leseansicht.
%% Der Schalter \ifkorrekturansicht ist bereits durch den Vorspann gesetzt.

%% latex-abspann.tex
%% Gemeinsamer Abspann für Korrekturansicht und Leseansicht.
%% Setzt den Schalter \ifkorrekturansicht voraus (gesetzt in den
%% einbindenden Dateien latex-korrekturansicht-abspann.tex bzw.
%% latex-leseansicht-abspann.tex).
%% ---------------------------------------------------------------

\normalsize

% Das esempio-Environment wird nur in der Leseansicht benötigt
\ifkorrekturansicht\else
\newenvironment{esempio}[3]%
{
    \vspace{1.5ex}
    \rlap{\underline{#1}}
    \par
    \setlength{\parindent}{0cm}
    \nopagebreak
    \leftskip=#2cm
    \rightskip=#3cm
}
{
    \par
}
\fi

\doendnotes{C}
\bigskip
\vfill

\clearpage

\footnotesize

\ifkorrekturansicht
  \lohead{\textsc{register}}
\fi

% theindex-Environment neu definieren ohne reledmac
\makeatletter
\renewenvironment{theindex}{%
  \ifkorrekturansicht
    \section*{\indexname}%
  \else
    \subsubsection*{Index der erwähnten Entitäten}%
  \fi
  \setlength{\parindent}{0pt}%
  \setlength{\parskip}{0pt plus 0.3pt}%
  \let\item\@idxitem
}{%
  \ifkorrekturansicht\clearpage\fi
}
\makeatother

\IfFileExists{\jobname-pw.ind}{\input{\jobname-pw.ind}}{}

% Quellenangabe nur in der Leseansicht
\ifkorrekturansicht\else
% Fallback-Definitionen, falls die .tex-Datei \titel etc. nicht gesetzt hat
\providecommand{\titel}{}
\providecommand{\editorInnen}{}
\providecommand{\dateiname}{\jobname}

\vspace{3cm}

\vfill

\footnotesize
\textsc{Quelle}: \titel. Herausgegeben von {\editorInnen}. In: \emph{Arthur Schnitzler: Briefwechsel mit Autorinnen und Autoren}.
 Digitale Edition, https://schnitzler-briefe.acdh.oeaw.ac.at/{\dateiname}.html (Stand \today)
\fi

\end{document}


      