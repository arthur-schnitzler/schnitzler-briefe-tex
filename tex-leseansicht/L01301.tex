%% latex-leseansicht-vorspann.tex
%% Vorspann für die Leseansicht.
%% Lädt die gemeinsame Datei latex-vorspann.tex mit nicht gesetztem Schalter.

\newif\ifkorrekturansicht
\korrekturansichtfalse

\input{../tex-inputs/latex-vorspann}


\section[Hugo von Hofmannsthal an Arthur Schnitzler, 1. 7. [1903]]{L01301 Hugo von Hofmannsthal an Arthur Schnitzler, 1. 7. [1903]}
\nopagebreak\mylabel{L01301v}
\rehead{ }\normalsize\beginnumbering\briefempfaengerindex{Schnitzler, Arthur@\textsc{Schnitzler, Arthur}!zzzHofmannsthal, Hugo von@\emph{von Hugo von Hofmannsthal}!1903-07-011@{1. 7. [1903]}|(be}
\toendnotes[C]{\smallbreak\pagebreak[2]}
\correspDesc{Versand  durch Hugo von Hofmannsthal am 1. 7. [1903] in Brenner
\newline{}Erhalt  durch Arthur Schnitzler im Zeitraum [2. 7. 1903
                  – 6. 7. 1903?] in Wien}\toendnotes[C]{\smallbreak}
\Standort{CUL, Schnitzler, B 43.}
\physDesc{Brief, 1 Blatt, 4 Seiten, 1415 Zeichen
\newline{}Handschrift: schwarze Tinte, deutsche Kurrent
\newline{}Ordnung: 1) mit Bleistift von unbekannter Hand nummeriert: »\strikeout{263}«  2) mit Bleistift von unbekannter Hand nummeriert:
                                    »262«}
\buchAbdrucke{\weitereDrucke{Hugo von Hofmannsthal, Arthur Schnitzler: \emph{Briefwechsel}. Herausgegeben von Therese Nickl und Heinrich Schnitzler. Frankfurt am Main: \emph{S. Fischer} 1964, S. 172–173.} }\toendnotes[C]{\smallbreak}
\pstart
           \raggedleft{}{\pb}1\textsuperscript{ten} July{\\}Gaſthof Poſt, am Brenner\oindex{Gasthof Post@\textbf{Gasthof Post}, \emph{Gastgewerbegebäude}|pw}.\pend
           \vspace{0.5em}
\pstart
           lieber, hier, wo wir vor einem Jahr \label{K_L01301-1v}\edtext{zuſammen geſeſſen}{\lemma{\textnormal{\emph{zusammen gesessen}}}\Cendnote{\textnormal{Vgl. A. S.: \emph{Tagebuch}, 3. 7. 1902.
               }}}\label{K_L01301-1}{ }ſind – es iſt ein Jahr faſt auf den Tag genau – finde ich Ihren lieben Brief.
               Erinnern Sie{ }ſich? es war an dem{ }ſchönen Tag, wo wir im \textsc{Stubai}thal\oindex{Stubaital@\textbf{Stubaital}, \emph{Tal}|pw} waren und ich Ihnen Complimente gemacht
               habe, wir dann in \textsc{Windischmatrei}\oindex{Matrei in Osttirol@\textbf{Matrei in Osttirol}, \emph{Hauptstadt}|pw} Forellen gegeſſen haben und die \textsc{Lisl}\pwindex{Steinrück, Elisabeth 19.\,11.\,1885 – 7.\,4.\,1920 Partenkirchen@\textsc{Steinrück, Elisabeth} (19.\,11.\,1885 – 7.\,4.\,1920 Partenkirchen)|pw} aus Berlin\oindex{Berlin@\textbf{Berlin}, \emph{Hauptstadt}|pw}{ }{\pb}geſchrieben hat, daß der Goldmann\pwindex{Goldmann, Paul 31.\,1.\,1865 Breslau – 25.\,9.\,1935 Wien@\textsc{Goldmann, Paul} (31.\,1.\,1865 Breslau – 25.\,9.\,1935 Wien), \emph{Schriftsteller, Journalist}|pw} ihr kein Geld leiht.\pend
           
\pstart
           Wir haben ein paar{ }ſehr{ }ſchöne Tage in Italien\oindex{Italien@\textbf{Italien}|pw}
               verbracht, das Ampezzo-thal\oindex{Valle d’Ampezzo@\textbf{Valle d’Ampezzo}, \emph{Tal}|pw} hinunter bis \textsc{Vicenza}\oindex{Vicenza@\textbf{Vicenza}, \emph{Hauptstadt}|pw} und durchs \textsc{Val sugana}\oindex{Val Sugana@\textbf{Val Sugana}, \emph{Tal}|pw} zurück.\hspace*{1.5em}So{ }ſchön iſt dieſes Land!\pend
           
\pstart
           Trotzdem werde ich nicht mit Ihnen um den 10\textsuperscript{ten}
                  Auguſt in dieſe Gegenden fahren. Ich werde um den 10\textsuperscript{ten} Auguſt in Weimar\oindex{Weimar@\textbf{Weimar}, \emph{Verwaltungsgebiet}|pw}{ }ſein. Die Einladung dazu geht direct von der Erbgroßherzogin\pwindex{Sachsen-Weimar, Pauline 25.\,7.\,1852 Stuttgart – 17.\,5.\,1904 Orte@\textsc{Sachsen-Weimar, Pauline} (25.\,7.\,1852 Stuttgart – 17.\,5.\,1904 Orte)|pwv} aus, indirect
                  \strikeout{zu} von Keſſler\pwindex{Kessler, Harry von 23.\,5.\,1868 Paris – 4.\,12.\,1937 Lyon@\textsc{Kessler, Harry von} (23.\,5.\,1868 Paris – 4.\,12.\,1937 Lyon), \emph{Schriftsteller, Verleger, Diplomat}|pw}, der an dieſem {\pb}kleinen Hof{ }ſeit einiger Zeit eine nicht recht definierbare Art von
               Intendantenſtellung einnimmt. Sie wollen meinem Hinkommen zu Ehren dort auf dem
               kleinen Naturtheater in Belvedere\oindex{Schloss Belvedere [Weimar]@\textbf{Schloss Belvedere [Weimar]}, \emph{Bauernhof}|pw} – auf welchem
                  Goethe\pwindex{Goethe, Johann Wolfgang von 28.\,8.\,1749 Frankfurt am Main – 22.\,3.\,1832 Weimar@\textsc{Goethe, Johann Wolfgang von} (28.\,8.\,1749 Frankfurt am Main – 22.\,3.\,1832 Weimar), \emph{Schriftsteller}|pw} den Oreſt\pwindex{Goethe, Johann Wolfgang von 28.\,8.\,1749 Frankfurt am Main – 22.\,3.\,1832 Weimar@\textsc{Goethe, Johann Wolfgang von} (28.\,8.\,1749 Frankfurt am Main – 22.\,3.\,1832 Weimar), \emph{Schriftsteller}!Iphigenie auf Tauris@\strich\emph{Iphigenie auf Tauris}|pwv}{ }ſpielte – den Tod
                  des Tizian\pwindex{Hofmannsthal, Hugo von 1.\,2.\,1874 Wien – 15.\,7.\,1929 Rodaun@\textsc{Hofmannsthal, Hugo von} (1.\,2.\,1874 Wien – 15.\,7.\,1929 Rodaun), \emph{Schriftsteller}!Tod des Tizian. Ein Bruchstück@\strich\emph{Der Tod des Tizian. Ein Bruchstück}|pw} von den hübſcheſten Hofdamen und Pagen – wirklichen Pagen –{ }ſpielen laſſen. Es macht mir natürlich Spaß, auch kenne ich Weimar\oindex{Weimar@\textbf{Weimar}, \emph{Verwaltungsgebiet}|pw} gar nicht. –\pend
           
\pstart
           Das nähere darüber und über{ }ſonſtige Pläne mündlich.\pend
           
\pstart
           Wir gehen {\pb}noch für 10–12 Tage an
               den Grundlſee\oindex{Grundlsee [Gemeinde]@\textbf{Grundlsee [Gemeinde]}|pw}.\pend
           \settowidth{\longeste}{AdressexH. H.beimm}\settowidth{\longestz}{FrauxLili Geyger}\settowidth{\longestd}{}\settowidth{\longestv}{}\settowidth{\longestf}{}\addtolength\longeste{1em}
        \addtolength\longestz{1em}
      \pstart\noindent\makebox[\the\longeste][l]{Adreſſe \textsc{H. H.} bei}\makebox[\the\longestz][l]{\textsc{\uline{Frau Lili
                              Geyger\pwindex{Schalk, Lili 4.\,11.\,1872 – 29.\,9.\,1966@\textsc{Schalk, Lili} (4.\,11.\,1872 – 29.\,9.\,1966), \emph{Schriftstellerin, Sängerin}|pw}}}}
                  \pend\pstart\noindent\makebox[\the\longeste][l]{}\makebox[\the\longestz][l]{\textsc{Grundlsee}\oindex{Grundlsee [Gemeinde]@\textbf{Grundlsee [Gemeinde]}|pw}}
                  \pend\pstart\noindent\makebox[\the\longeste][l]{}\makebox[\the\longestz][l]{\textsc{Archkogel 13}\oindex{Archkogel@\textbf{Archkogel}, \emph{Straße}|pw}}
                  \pend
\pstart
           Von Herzen{\\[\baselineskip]}\spacefill\mbox{Hugo.}\pend
           \leftskip=0em{}
\pstart
           \noindent{}Grüße für Olga\pwindex{Schnitzler, Olga 17.\,1.\,1882 Wien – 13.\,1.\,1970 Lugano@\textsc{Schnitzler, Olga} (17.\,1.\,1882 Wien – 13.\,1.\,1970 Lugano), \emph{Schauspielerin, Sängerin}|pw} und Heinrich\pwindex{Schnitzler, Heinrich 9.\,8.\,1902 Hinterbrühl – 12.\,7.\,1982 Wien@\textsc{Schnitzler, Heinrich} (9.\,8.\,1902 Hinterbrühl – 12.\,7.\,1982 Wien), \emph{Regisseur, Schauspieler}|pw} das Kind. Es war abſolut unleſerlich, welches (franzöſiſche??) Buch\pwindex{\textcolor{red}{\textsuperscript{XXXX indx1}}!drei Musketiere@\strich\emph{Die drei Musketiere}|pwv} Sie
                  auf der Reiſe{ }ſehr genoſſen haben.\pend
           \selectlanguage{ngerman}\endnumbering\briefempfaengerindex{Schnitzler, Arthur@\textsc{Schnitzler, Arthur}!zzzHofmannsthal, Hugo von@\emph{von Hugo von Hofmannsthal}!1903-07-011@{1. 7. [1903]}|)be}\mylabel{L01301h}  \newcommand{\dateiname}{L01301}\newcommand{\titel}{Hugo von Hofmannsthal an Arthur Schnitzler, 1. 7. [1903]}\newcommand{\editorInnen}{Martin Anton Müller und Gerd-Hermann Susen}%% latex-leseansicht-abspann.tex
%% Abspann für die Leseansicht.
%% Der Schalter \ifkorrekturansicht ist bereits durch den Vorspann gesetzt.

%% latex-abspann.tex
%% Gemeinsamer Abspann für Korrekturansicht und Leseansicht.
%% Setzt den Schalter \ifkorrekturansicht voraus (gesetzt in den
%% einbindenden Dateien latex-korrekturansicht-abspann.tex bzw.
%% latex-leseansicht-abspann.tex).
%% ---------------------------------------------------------------

\normalsize

% Das esempio-Environment wird nur in der Leseansicht benötigt
\ifkorrekturansicht\else
\newenvironment{esempio}[3]%
{
    \vspace{1.5ex}
    \rlap{\underline{#1}}
    \par
    \setlength{\parindent}{0cm}
    \nopagebreak
    \leftskip=#2cm
    \rightskip=#3cm
}
{
    \par
}
\fi

\doendnotes{C}
\bigskip
\vfill

\clearpage

\footnotesize

\ifkorrekturansicht
  \lohead{\textsc{register}}
\fi

% theindex-Environment neu definieren ohne reledmac
\makeatletter
\renewenvironment{theindex}{%
  \ifkorrekturansicht
    \section*{\indexname}%
  \else
    \subsubsection*{Index der erwähnten Entitäten}%
  \fi
  \setlength{\parindent}{0pt}%
  \setlength{\parskip}{0pt plus 0.3pt}%
  \let\item\@idxitem
}{%
  \ifkorrekturansicht\clearpage\fi
}
\makeatother

\IfFileExists{\jobname-pw.ind}{\input{\jobname-pw.ind}}{}

% Quellenangabe nur in der Leseansicht
\ifkorrekturansicht\else
% Fallback-Definitionen, falls die .tex-Datei \titel etc. nicht gesetzt hat
\providecommand{\titel}{}
\providecommand{\editorInnen}{}
\providecommand{\dateiname}{\jobname}

\vspace{3cm}

\vfill

\footnotesize
\textsc{Quelle}: \titel. Herausgegeben von {\editorInnen}. In: \emph{Arthur Schnitzler: Briefwechsel mit Autorinnen und Autoren}.
 Digitale Edition, https://schnitzler-briefe.acdh.oeaw.ac.at/{\dateiname}.html (Stand \today)
\fi

\end{document}


