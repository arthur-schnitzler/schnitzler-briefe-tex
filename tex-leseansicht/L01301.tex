\input{../tex-inputs/latex-pdf-vorspann}
\begin{center}
            \textcolor{red}{ENTWURF. ENTZIFFERUNG NOCH NICHT KORREKTURGELESEN}
                      \end{center}
            
               \section[Hugo von Hofmannsthal an Arthur Schnitzler, 1. 7. {[}1903{]}]{ Hugo von Hofmannsthal an Arthur Schnitzler, 1. 7. {[}1903{]}}\nopagebreak\mylabel{v}\rehead{ }\begin{ledgroupsized}[t]{13cm}\normalsize\beginnumbering\briefempfaengerindex{Schnitzler, Arthur@\textsc{Schnitzler, Arthur}!zzzHofmannsthal, Hugo von@\emph{von Hugo von Hofmannsthal}!1903-07-011@{1. 7. {[}1903{]}}|(be} \toendnotes[C]{\smallbreak\pagebreak[2]} \Standort{CUL, Schnitzler, B 43.}
\physDesc{Brief, 1 Blatt, 4 Seiten
\newline{}Handschrift: schwarze Tinte, deutsche Kurrent\newline{}Ordnung: 1) mit Bleistift von unbekannter Hand nummeriert: »\strikeout{263}« 2) mit Bleistift von unbekannter Hand nummeriert: »262«}\buchAbdrucke{\weitereDrucke{Hugo von Hofmannsthal, Arthur Schnitzler: \emph{Briefwechsel}. Hg. Therese Nickl und Heinrich Schnitzler. Frankfurt am Main: \emph{S. Fischer} 1964, S. 172–173.} }\toendnotes[C]{\smallbreak}\pstart
           \raggedleft{}{\pb}1\textsuperscript{ten} July{\\}Gaſthof Poſt, am Brenner\oindex{Gasthof Post@\textbf{Gasthof Post}|pw}.\pend
           \pstart
           lieber, hier, wo wir vor einem Jahr \label{K_L01301_1v}\edtext{zuſammen geſeſſen}{\lemma{\textnormal{\emph{zuſammen geſeſſen}}}\Cendnote{\textnormal{vgl. A. S.: \emph{Tagebuch}, 3. 7. 1902}}}\label{K_L01301_1h} ſind – es iſt
               ein Jahr faſt auf den Tag genau – finde ich Ihren lieben Brief. Erinnern Sie ſich? es
               war an dem ſchönen Tag, wo wir im \textsc{Stubai}thal\oindex{Stubaital@\textbf{Stubaital}|pw} waren und ich Ihnen Complimente gemacht
               habe, wir dann in \textsc{Windischmatrei}\oindex{Matrei in Osttirol@\textbf{Matrei in Osttirol}|pw} Forellen gegeſſen haben und die \textsc{Lisl}\pwindex{Steinrueck, Elisabeth 19.11.1885 – 07.04.1920@\textsc{Steinrück, Elisabeth} (19.11.1885 – 07.04.1920)|pw} aus Berlin\oindex{Berlin@\textbf{Berlin}|pw}{ }{\pb}geſchrieben hat, daß der Goldmann\pwindex{Goldmann, Paul 31.01.1865 – 25.09.1935@\textsc{Goldmann, Paul} (31.01.1865 – 25.09.1935), \emph{Schriftsteller, Journalist}|pw} ihr kein Geld leiht.\pend
           \pstart
           Wir haben ein paar ſehr ſchöne Tage in Italien\oindex{Italien@\textbf{Italien}|pw}
               verbracht, das Ampezzo-thal\oindex{Valle DAmpezzo@\textbf{Valle d’Ampezzo}|pw} hinunter bis \textsc{Vicenza}\oindex{Vicenza@\textbf{Vicenza}|pw} und durchs \textsc{Val sugana}\oindex{Val Sugana@\textbf{Val Sugana}|pw} zurück.\hspace*{1.5em}So ſchön iſt dieſes Land!\pend
           \pstart
           Trotzdem werde ich nicht mit Ihnen um den 10\textsuperscript{ten}
                  Auguſt in dieſe Gegenden fahren. Ich werde um den 10\textsuperscript{ten} Auguſt in Weimar\oindex{Weimar@\textbf{Weimar}|pw}{ }ſein. Die Einladung dazu geht direct von der Erbgroßherzogin\pwindex{Sachsen-Weimar, Pauline 25.07.1852 – 17.05.1904@\textsc{Sachsen-Weimar, Pauline} (25.07.1852 – 17.05.1904)|pwv} aus, indirect
                  \strikeout{zu} von Keſſler\pwindex{Kessler, Harry von 23.05.1868 – 04.12.1937@\textsc{Kessler, Harry von} (23.05.1868 – 04.12.1937), \emph{Schriftsteller, Verleger, Diplomat}|pw}, der an dieſem {\pb}kleinen Hof ſeit einiger Zeit eine nicht recht definierbare Art von
               Intendantenſtellung einnimmt. Sie wollen meinem Hinkommen zu Ehren dort auf dem
               kleinen Naturtheater in Belvedere\oindex{Belvedere@\textbf{Belvedere}|pw} – auf welchem Goethe\pwindex{Goethe, Johann Wolfgang von 28.08.1749 – 22.03.1832@\textsc{Goethe, Johann Wolfgang von} (28.08.1749 – 22.03.1832), \emph{Schriftsteller}|pw} den Oreſt\pwindex{Goethe, Johann Wolfgang von 28.08.1749 – 22.03.1832@\textsc{Goethe, Johann Wolfgang von} (28.08.1749 – 22.03.1832), \emph{Schriftsteller}!Iphigenie auf Tauris1787@\strich\emph{Iphigenie auf Tauris} {[}1787{]}|pwv}{ }ſpielte – den Tod des
                  Tizian\pwindex{Hofmannsthal, Hugo von 01.02.1874 – 15.07.1929@\textsc{Hofmannsthal, Hugo von} (01.02.1874 – 15.07.1929), \emph{Schriftsteller}!Tod des Tizian1892.10@\strich\emph{Der Tod des Tizian} {[}1892.10{]}|pw} von den hübſcheſten Hofdamen und Pagen – wirklichen Pagen – ſpielen
               laſſen. Es macht mir natürlich Spaß, auch kenne ich Weimar\oindex{Weimar@\textbf{Weimar}|pw} gar nicht. –\pend
           \pstart
           Das nähere darüber und über ſonſtige Pläne mündlich.\pend
           \pstart
           Wir gehen {\pb}noch für 10–12 Tage an
               den Grundlſee\oindex{Grundlsee@\textbf{Grundlsee}|pw}.\pend
           \settowidth{\longeste}{Adreſſe H. H. bei}\settowidth{\longestz}{Frau Lili Geyger}\settowidth{\longestd}{}\settowidth{\longestv}{}\settowidth{\longestf}{}\addtolength\longeste{1em}
        \addtolength\longestz{1em}
      \pstart\noindent\makebox[\the\longeste][l]{Adreſſe \textsc{H. H.} bei}\makebox[\the\longestz][l]{\textsc{\uline{Frau Lili Geyger\pwindex{Schalk, Lili 04.11.1872 – 29.09.1966@\textsc{Schalk, Lili} (04.11.1872 – 29.09.1966), \emph{Schriftstellerin, Sängerin}|pw}}}}
                  \pend\pstart\noindent\makebox[\the\longeste][l]{}\makebox[\the\longestz][l]{\textsc{Grundlsee}\oindex{Grundlsee@\textbf{Grundlsee}|pw}}
                  \pend\pstart\noindent\makebox[\the\longeste][l]{}\makebox[\the\longestz][l]{\textsc{Archkogel 13}\oindex{Archkogel@\textbf{Archkogel}|pw}}
                  \pend\pstart
           Von Herzen{\\[\baselineskip]}\spacefill\mbox{Hugo.}\pend
           \leftskip=0em{}\pstart
           \noindent{}Grüße für Olga\pwindex{Schnitzler, Olga 17.01.1882 – 13.01.1970@\textsc{Schnitzler, Olga} (17.01.1882 – 13.01.1970), \emph{Schauspielerin, Sängerin}|pw} und Heinrich\pwindex{Schnitzler, Heinrich 09.08.1902 – 12.07.1982@\textsc{Schnitzler, Heinrich} (09.08.1902 – 12.07.1982), \emph{Regisseur, Schauspieler}|pw} das Kind. Es war abſolut unleſerlich, welches (franzöſiſche??) Buch\pwindex{\textcolor{red}{\textsuperscript{XXXX1 indx}}!drei Musketiere1844@\strich\emph{Die drei Musketiere} {[}1844{]}|pwv} Sie auf
                  der Reiſe ſehr genoſſen haben.\pend
           \endnumbering\briefempfaengerindex{Schnitzler, Arthur@\textsc{Schnitzler, Arthur}!zzzHofmannsthal, Hugo von@\emph{von Hugo von Hofmannsthal}!1903-07-011@{1. 7. {[}1903{]}}|)be}\mylabel{h}\end{ledgroupsized}  \newcommand{\dateiname}{L01301}\newcommand{\titel}{Hugo von Hofmannsthal an Arthur Schnitzler, 1. 7. [1903]}\newcommand{\editorInnen}{Martin Anton Müller und Gerd-Hermann Susen}\input{../tex-inputs/latex-pdf-abspann}
      