%% latex-leseansicht-vorspann.tex
%% Vorspann für die Leseansicht.
%% Lädt die gemeinsame Datei latex-vorspann.tex mit nicht gesetztem Schalter.

\newif\ifkorrekturansicht
\korrekturansichtfalse

\input{../tex-inputs/latex-vorspann}


\section[Arthur Schnitzler an Hermann Bahr, 17. 3. 1930]{L02533 Arthur Schnitzler an Hermann Bahr, 17. 3. 1930}
\nopagebreak\mylabel{L02533v}
\rehead{ }\normalsize\beginnumbering\briefempfaengerindex{Bahr, Hermann@\textsc{Bahr, Hermann}!zzzSchnitzler, Arthur@\emph{von Arthur Schnitzler}!1930-03-171@{17. 3. 1930}|(be}
\toendnotes[C]{\smallbreak\pagebreak[2]}
\correspDesc{Versand  durch Arthur Schnitzler am 17. 3. 1930 in Wien
\newline{}Erhalt  durch Hermann Bahr im Zeitraum [18. 3. 1930
                  – 22. 3. 1930?] in München}\toendnotes[C]{\smallbreak}
\Standort{TMW, HS AM 23399 Ba.}
\physDesc{Brief, 1 Blatt, 2 Seiten, 1426 Zeichen
\newline{}Handschrift: schwarze Tinte, lateinische Kurrent
\newline{}Bahr: 1) mit rotem Buntstift ergänzt: »\uline{Schnitzler}«  2) mit blauem Buntstift im Text »bindet«
                                 unterstrichen}
\buchAbdrucke{\weitereDrucke{1) \emph{17. 3. 1930.} In: Arthur Schnitzler: \emph{The Letters of Arthur Schnitzler to Hermann Bahr}. Edited, annotated, and with an introduction, by Donald G. Daviau. Chapel Hill: \emph{The University of North Carolina Press} 1978, S. 117–118 (University of North Carolina studies in the Germanic languages
                        and literatures, 89).} \weitereDrucke{2) Hermann Bahr, Arthur Schnitzler: \emph{Briefwechsel, Aufzeichnungen, Dokumente (1891–1931)}. Herausgegeben von Kurt Ifkovits und Martin Anton Müller. Göttingen: \emph{Wallstein} 2018, S. 596.} }\toendnotes[C]{\smallbreak}
\pstart
           \raggedleft{}{\pb}Wien\oindex{Wien@\textbf{Wien}, \emph{Verwaltungsgebiet}|pw}, 17. 3. 1930.\pend
           \vspace{0.5em}
\pstart
           Mein lieber Hermann, dein Heimweh nach Wien\oindex{Wien@\textbf{Wien}, \emph{Verwaltungsgebiet}|pw} und das deiner verehrten Gattin\pwindex{Bahr-Mildenburg, Anna 29.\,11.\,1872 Wien – 27.\,1.\,1947 ebd.@\textsc{Bahr-Mildenburg, Anna} (29.\,11.\,1872 Wien – 27.\,1.\,1947 ebd.), \emph{Sängerin}|pwv} hat auch mir ans Herz gegriffen, und der Hofrätin\pwindex{Zuckerkandl, Berta 13.\,4.\,1864 Wien – 16.\,10.\,1945 Paris@\textsc{Zuckerkandl, Berta} (13.\,4.\,1864 Wien – 16.\,10.\,1945 Paris), \emph{Schriftstellerin, Journalistin, Übersetzerin}|pwv}, mit der ich \label{K_L02533-1v}\edtext{neulich}{\lemma{\textnormal{\emph{neulich}}}\Cendnote{\textnormal{am 28. 2. 1930}}}\label{K_L02533-1} davon sprach. Aber so wenig ich den Nobelpreis\orgindex{Bauernfeld-Preis@Bauernfeld-Preis|pw} kriegen werde, so wenig hab ich in Oesterreich\oindex{Österreich@\textbf{Österreich}|pw} zu sagen, sonst hätt ich dich längst wieder ans Burgtheater\oindex{Wien@\textbf{Wien}!I., Innere Stadt@\textbf{I., Innere Stadt}!Burgtheater@\textbf{Burgtheater}, \emph{Theater}|pw} berufen (auf die Gefahr hin, daſs du
               mich wieder nicht aufführst, auch ohne Poldi\pwindex{Andrian-Werburg, Leopold von 9.\,5.\,1875 Berlin – 19.\,11.\,1951 Fribourg@\textsc{Andrian-Werburg, Leopold von} (9.\,5.\,1875 Berlin – 19.\,11.\,1951 Fribourg), \emph{Schriftsteller, Diplomat}|pw})
               – und wie erst Frau Mildenburg\pwindex{Bahr-Mildenburg, Anna 29.\,11.\,1872 Wien – 27.\,1.\,1947 ebd.@\textsc{Bahr-Mildenburg, Anna} (29.\,11.\,1872 Wien – 27.\,1.\,1947 ebd.), \emph{Sängerin}|pw} an die Oper\oindex{Wien@\textbf{Wien}!I., Innere Stadt@\textbf{I., Innere Stadt}!Oper@\textbf{Oper}, \emph{Oper}|pw} oder wohin sie sonst möchte, – und in der
               Musik geht ja meine Objectivität noch weiter als in der Literatur. Aber je weniger
               man versteht und je mehr man liebt, um so gerechter ist man.\pend
           
\pstart
           Aber Scherz beiseite, was bindet dich eigentlich an München\oindex{München@\textbf{München}|pw}? Ich habe das Gefühl, daſs deine Leiden und – entschuldige – deine
               Hypochondrien sich hier zumindest lindern würden. Es würde viele freuen auch manche
               die nicht in allem deines Sinnes sind, Dich wieder hier zu wissen. Denn wissen wir
               überhaupt {\pb}welchen
               Sinnes wir sind. Kaum welchen Herzens. Beziehungen, auch unterbrochene, auch
               gestörte, sind das einzige reale in der seelischen Oekonomie. \label{LL141-1v}\label{LL141-1h}Wenn mir meine Vergangenheit erscheint, bist du mir immer Einer
               der nächsten, und so ka{\geminationn} es auch in der Gegenwart nicht
               anders sein.\pend
           
\pstart
           \label{K_L02533-2v}\edtext{Klingt das nicht ein bischen nach
               fünfter Akt, erste Scene?}{\lemma{\textnormal{\emph{Klingt … Scene?}}}\Cendnote{\textnormal{Vgl. Hermann Bahr, Arthur Schnitzler: \emph{Briefwechsel, Aufzeichnungen, Dokumente (1891–1931)}, Arthur Schnitzler: [Brief an Hermann Bahr], [Anfang Juli] 1923.}}}\label{K_L02533-2} Sagen wir: Vierter, \textcolor{gray}{vor}letzte. Wir wollen nicht
               sentimental \introOben{}werden.\introOben{} Ich bemerke mit angemessener Kühle:
               Hoffentlich sieht man sich einmal wieder. Es wäre schön.\pend
           
\pstart
           Von Herzen Dein{\\[\baselineskip]}\spacefill\mbox{Arthur}\pend
           \leftskip=0em{}\selectlanguage{ngerman}\endnumbering\briefempfaengerindex{Bahr, Hermann@\textsc{Bahr, Hermann}!zzzSchnitzler, Arthur@\emph{von Arthur Schnitzler}!1930-03-171@{17. 3. 1930}|)be}\mylabel{L02533h}  \newcommand{\dateiname}{L02533}\newcommand{\titel}{Arthur Schnitzler an Hermann Bahr, 17. 3. 1930}\newcommand{\editorInnen}{Herausgegeben von Martin Anton Müller}%% latex-leseansicht-abspann.tex
%% Abspann für die Leseansicht.
%% Der Schalter \ifkorrekturansicht ist bereits durch den Vorspann gesetzt.

%% latex-abspann.tex
%% Gemeinsamer Abspann für Korrekturansicht und Leseansicht.
%% Setzt den Schalter \ifkorrekturansicht voraus (gesetzt in den
%% einbindenden Dateien latex-korrekturansicht-abspann.tex bzw.
%% latex-leseansicht-abspann.tex).
%% ---------------------------------------------------------------

\normalsize

% Das esempio-Environment wird nur in der Leseansicht benötigt
\ifkorrekturansicht\else
\newenvironment{esempio}[3]%
{
    \vspace{1.5ex}
    \rlap{\underline{#1}}
    \par
    \setlength{\parindent}{0cm}
    \nopagebreak
    \leftskip=#2cm
    \rightskip=#3cm
}
{
    \par
}
\fi

\doendnotes{C}
\bigskip
\vfill

\clearpage

\footnotesize

\ifkorrekturansicht
  \lohead{\textsc{register}}
\fi

% theindex-Environment neu definieren ohne reledmac
\makeatletter
\renewenvironment{theindex}{%
  \ifkorrekturansicht
    \section*{\indexname}%
  \else
    \subsubsection*{Index der erwähnten Entitäten}%
  \fi
  \setlength{\parindent}{0pt}%
  \setlength{\parskip}{0pt plus 0.3pt}%
  \let\item\@idxitem
}{%
  \ifkorrekturansicht\clearpage\fi
}
\makeatother

\IfFileExists{\jobname-pw.ind}{\input{\jobname-pw.ind}}{}

% Quellenangabe nur in der Leseansicht
\ifkorrekturansicht\else
% Fallback-Definitionen, falls die .tex-Datei \titel etc. nicht gesetzt hat
\providecommand{\titel}{}
\providecommand{\editorInnen}{}
\providecommand{\dateiname}{\jobname}

\vspace{3cm}

\vfill

\footnotesize
\textsc{Quelle}: \titel. Herausgegeben von {\editorInnen}. In: \emph{Arthur Schnitzler: Briefwechsel mit Autorinnen und Autoren}.
 Digitale Edition, https://schnitzler-briefe.acdh.oeaw.ac.at/{\dateiname}.html (Stand \today)
\fi

\end{document}


