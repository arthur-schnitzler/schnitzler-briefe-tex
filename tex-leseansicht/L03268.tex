%% latex-leseansicht-vorspann.tex
%% Vorspann für die Leseansicht.
%% Lädt die gemeinsame Datei latex-vorspann.tex mit nicht gesetztem Schalter.

\newif\ifkorrekturansicht
\korrekturansichtfalse

\input{../tex-inputs/latex-vorspann}


\section[ Felix Salten an Arthur Schnitzler, 13. 7. 1897]{L03268 Felix Salten an Arthur Schnitzler,  13. 7. 1897}
\nopagebreak\mylabel{L03268v}
\rehead{ }\normalsize\beginnumbering\briefempfaengerindex{Schnitzler, Arthur@\textsc{Schnitzler, Arthur}!zzzSalten, Felix@\emph{von Felix Salten}!1897-07-133@{13. 7. 1897}|(be}
\toendnotes[C]{\smallbreak\pagebreak[2]}
\correspDesc{Versand  durch Felix Salten am 13. 7. 1897 in Wien
\newline{}Erhalt  durch Arthur Schnitzler im Zeitraum [14. 7. 1897
                  – 18. 7. 1897?] in Ischl}\toendnotes[C]{\smallbreak}
\Standort{CUL, Schnitzler, B 89, A 2.}
\physDesc{Karte, 218 Zeichen
\newline{}Handschrift: Bleistift, lateinische Kurrent
\newline{}Ordnung: mit Bleistift von unbekannter Hand nummeriert: »91« }\toendnotes[C]{\smallbreak}
\pstart
           \raggedleft{}{\pb}Wien\oindex{Wien@\textbf{Wien}, \emph{Verwaltungsgebiet}|pw}, 13. Juli 97\pend
           \vspace{0.5em}
\pstart
           Mir geht’s leidlich. Der Arbeit auch. Am 25. treffen
               Sie mich wahrscheinlich noch \label{K_L03268-1v}\edtext{hier\oindex{Wien@\textbf{Wien}, \emph{Verwaltungsgebiet}|pwv}}{\lemma{\textnormal{\emph{hier}}}\Cendnote{\textnormal{Schnitzler kehrte am 25. 7. 1897 nach Wien\oindex{Wien@\textbf{Wien}, \emph{Verwaltungsgebiet}|pwk} zurück. Salten\pwindex{Salten, Felix 6.\,9.\,1869 Budapest – 8.\,10.\,1945 Zürich@\textsc{Salten, Felix} (6.\,9.\,1869 Budapest – 8.\,10.\,1945 Zürich), \emph{Schriftsteller, Journalist, Chefredakteur}|pwk} traf er nachweislich am 30. 7. 1897 wieder.}}}\label{K_L03268-1}. Sollte ich nicht da
               sein, bin ich einstweilen in Pressbaum\oindex{Pressbaum@\textbf{Pressbaum}, \emph{Hauptstadt}|pw}, wo mich
               Briefe in der Hauptstraße 7\oindex{Hauptstraße@\textbf{Hauptstraße}, \emph{Straße}|pw} erreichen.\pend
           \pstart Herzlich \spacefill\mbox{Salten}\pend{}\selectlanguage{ngerman}\endnumbering\briefempfaengerindex{Schnitzler, Arthur@\textsc{Schnitzler, Arthur}!zzzSalten, Felix@\emph{von Felix Salten}!1897-07-133@{13. 7. 1897}|)be}\mylabel{L03268h}  \newcommand{\dateiname}{L03268}\newcommand{\titel}{Felix Salten an Arthur Schnitzler, 13. 7. 1897}\newcommand{\editorInnen}{Martin Anton Müller und Laura Untner}%% latex-leseansicht-abspann.tex
%% Abspann für die Leseansicht.
%% Der Schalter \ifkorrekturansicht ist bereits durch den Vorspann gesetzt.

%% latex-abspann.tex
%% Gemeinsamer Abspann für Korrekturansicht und Leseansicht.
%% Setzt den Schalter \ifkorrekturansicht voraus (gesetzt in den
%% einbindenden Dateien latex-korrekturansicht-abspann.tex bzw.
%% latex-leseansicht-abspann.tex).
%% ---------------------------------------------------------------

\normalsize

% Das esempio-Environment wird nur in der Leseansicht benötigt
\ifkorrekturansicht\else
\newenvironment{esempio}[3]%
{
    \vspace{1.5ex}
    \rlap{\underline{#1}}
    \par
    \setlength{\parindent}{0cm}
    \nopagebreak
    \leftskip=#2cm
    \rightskip=#3cm
}
{
    \par
}
\fi

\doendnotes{C}
\bigskip
\vfill

\clearpage

\footnotesize

\ifkorrekturansicht
  \lohead{\textsc{register}}
\fi

% theindex-Environment neu definieren ohne reledmac
\makeatletter
\renewenvironment{theindex}{%
  \ifkorrekturansicht
    \section*{\indexname}%
  \else
    \subsubsection*{Index der erwähnten Entitäten}%
  \fi
  \setlength{\parindent}{0pt}%
  \setlength{\parskip}{0pt plus 0.3pt}%
  \let\item\@idxitem
}{%
  \ifkorrekturansicht\clearpage\fi
}
\makeatother

\IfFileExists{\jobname-pw.ind}{\input{\jobname-pw.ind}}{}

% Quellenangabe nur in der Leseansicht
\ifkorrekturansicht\else
% Fallback-Definitionen, falls die .tex-Datei \titel etc. nicht gesetzt hat
\providecommand{\titel}{}
\providecommand{\editorInnen}{}
\providecommand{\dateiname}{\jobname}

\vspace{3cm}

\vfill

\footnotesize
\textsc{Quelle}: \titel. Herausgegeben von {\editorInnen}. In: \emph{Arthur Schnitzler: Briefwechsel mit Autorinnen und Autoren}.
 Digitale Edition, https://schnitzler-briefe.acdh.oeaw.ac.at/{\dateiname}.html (Stand \today)
\fi

\end{document}


