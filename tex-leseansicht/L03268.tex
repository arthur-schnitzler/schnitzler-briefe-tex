%% latex-korrekturansicht-vorspann.tex
%% Vorspann für die Korrekturansicht.
%% Lädt die gemeinsame Datei latex-vorspann.tex mit gesetztem Schalter.

\newif\ifkorrekturansicht
\korrekturansichttrue

\input{../tex-inputs/latex-vorspann}


\section[ Felix Salten an Arthur Schnitzler, 13. 7. 1897]{L03268 Felix Salten an Arthur Schnitzler, 13. 7. 1897}
\nopagebreak\mylabel{L03268v}
\rehead{ }\normalsize\beginnumbering\briefempfaengerindex{Schnitzler, Arthur@\textsc{Schnitzler, Arthur}!zzzSalten, Felix@\emph{von Felix Salten}!1897-07-133@{13. 7. 1897}|(be}
\toendnotes[C]{\smallbreak\pagebreak[2]}\Standort{CUL, Schnitzler, B 89, A 2.}
\physDesc{Karte, 218 Zeichen
\newline{}Handschrift: Bleistift, lateinische Kurrent
\newline{}Ordnung: mit Bleistift von unbekannter Hand nummeriert: »91« }\toendnotes[C]{\smallbreak}
\pstart
           \raggedleft{}{\pb}Wien\oindex{Wien@\textbf{Wien}, \emph{A.ADM2}|pw}, 13. Juli 97\pend
           \vspace{0.5em}
\pstart
           Mir geht’s leidlich. Der Arbeit auch. Am 25. treffen
               Sie mich wahrscheinlich noch \label{K_L03268-1v}\edtext{hier\oindex{Wien@\textbf{Wien}, \emph{A.ADM2}|pwv}}{\lemma{\textnormal{\emph{hier}}}\Cendnote{\textnormal{Schnitzler kehrte am 25. 7. 1897 nach Wien\oindex{Wien@\textbf{Wien}, \emph{A.ADM2}|pwk} zurück. Salten\pwindex{Salten, Felix 06.09.1869 – 08.10.1945@\textsc{Salten, Felix} (06.09.1869 – 08.10.1945), \emph{Schriftsteller/Schriftstellerin, Journalist/Journalistin, Chefredakteur/Chefredakteurin}|pwk} traf er nachweislich am 30. 7. 1897 wieder.}}}\label{K_L03268-1}. Sollte ich nicht da
               sein, bin ich einstweilen in Pressbaum\oindex{Pressbaum@\textbf{Pressbaum}, \emph{P.PPLA3}|pw}, wo mich
               Briefe in der Hauptstraße 7\oindex{Hauptstrasse@\textbf{Hauptstraße}, \emph{Straße (K.STR)}|pw} erreichen.\pend
           \pstart Herzlich \spacefill\mbox{Salten}\pend{}\selectlanguage{ngerman}\endnumbering\briefempfaengerindex{Schnitzler, Arthur@\textsc{Schnitzler, Arthur}!zzzSalten, Felix@\emph{von Felix Salten}!1897-07-133@{13. 7. 1897}|)be}\mylabel{L03268h}  \normalsize

\doendnotes{C}
\bigskip
\vfill

\clearpage

\footnotesize

\lohead{\textsc{register}}

% Definiere theindex-Environment komplett neu ohne reledmac
\makeatletter
\renewenvironment{theindex}{%
  \section*{\indexname}%
  \setlength{\parindent}{0pt}%
  \setlength{\parskip}{0pt plus 0.3pt}%
  \let\item\@idxitem
}{%
  \clearpage
}
\makeatother

\IfFileExists{\jobname-pw.ind}{\input{\jobname-pw.ind}}{}

\end{document}

      