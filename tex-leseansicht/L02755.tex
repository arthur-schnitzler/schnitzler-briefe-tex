%% latex-leseansicht-vorspann.tex
%% Vorspann für die Leseansicht.
%% Lädt die gemeinsame Datei latex-vorspann.tex mit nicht gesetztem Schalter.

\newif\ifkorrekturansicht
\korrekturansichtfalse

\input{../tex-inputs/latex-vorspann}


\section[Paul Goldmann an Arthur Schnitzler, 13. 11. [1895]]{L02755 Paul Goldmann an Arthur Schnitzler, 13. 11. [1895]}
\nopagebreak\mylabel{L02755v}
\rehead{ }\normalsize\beginnumbering\briefempfaengerindex{Schnitzler, Arthur@\textsc{Schnitzler, Arthur}!zzzGoldmann, Paul@\emph{von Paul Goldmann}!1895-11-131@{13. 11. [1895]}|(be}
\toendnotes[C]{\smallbreak\pagebreak[2]}
\correspDesc{Versand  durch Paul Goldmann am 13. 11. [1895] in Paris
\newline{}Erhalt  durch Arthur Schnitzler im Zeitraum [14. 11. 1895 – 18. 11. 1895?] in Wien}\toendnotes[C]{\smallbreak}
\Standort{DLA, A:Schnitzler, HS.NZ85.1.3165.}
\physDesc{Brief, 1 Blatt, 4 Seiten, 1436 Zeichen
\newline{}Handschrift: blaue Tinte, deutsche Kurrent
\newline{}Schnitzler: 1) mit Bleistift das Jahr »95« vermerkt  2) mit rotem Buntstift drei Unterstreichungen}\toendnotes[C]{\smallbreak}
\pstart
           {\pb}\textcolor{gray}{\textbf{\textbf{Frankfurter Zeitung\orgindex{Frankfurter Zeitung@Frankfurter Zeitung|pw}}}}\pend
           
\pstart
           \textcolor{gray}{\textbf{(\begin{otherlanguage}{french}Gazette de Francfort\end{otherlanguage}\orgindex{Frankfurter Zeitung@Frankfurter Zeitung|pw}).}}\pend
           
\pstart
           \textcolor{gray}{\textbf{\textbf{\begin{otherlanguage}{french}Fondateur M. L.
                              Sonnemann\pwindex{Sonnemann, Leopold 29.\,10.\,1831 Höchberg – 30.\,10.\,1909 Frankfurt am Main@\textsc{Sonnemann, Leopold} (29.\,10.\,1831 Höchberg – 30.\,10.\,1909 Frankfurt am Main), \emph{Journalist, Herausgeber}|pw}\end{otherlanguage}.}}}\pend
           
\pstart
           \begin{otherlanguage}{french}\textcolor{gray}{\textbf{Journal\pwindex{Frankfurter Zeitung@\emph{Frankfurter Zeitung}|pwv} politique,
                           financier,}}\end{otherlanguage}\hfill \textsc{Paris\oindex{Paris@\textbf{Paris}, \emph{Hauptstadt}|pw}}, 13. November.\pend
           
\pstart
           \begin{otherlanguage}{french}\textcolor{gray}{\textbf{commercial et littéraire.}}\end{otherlanguage}\pend
           
\pstart
           \begin{otherlanguage}{french}\textcolor{gray}{\textbf{\textbf{Paraissant trois fois par jour.}}}\end{otherlanguage}\pend
           
\pstart
           \begin{otherlanguage}{french}\textcolor{gray}{\textbf{\textbf{Bureau à Paris\oindex{Paris@\textbf{Paris}, \emph{Hauptstadt}|pw}:}}}\end{otherlanguage}\pend
           
\pstart
           \begin{otherlanguage}{french}\textcolor{gray}{\textbf{\textbf{24. Rue Feydeau\oindex{rue Feydeau@\textbf{rue Feydeau}, \emph{Straße}|pw}.}}}\end{otherlanguage}\pend
           
\pstart\center{}Mein lieber Freund,\pend\vspace{0.5em}
\pstart
           Die Arbeit dauert fort, und den großen Brief kann ich noch immer nicht{ }ſchreiben.
               Alſo den kleinen.\pend
           
\pstart
           1.) Die »Kleine Komödie\pwindex{Schnitzler, Arthur 15.\,5.\,1862 Wien – 21.\,10.\,1931 ebd.@\textsc{Schnitzler, Arthur} (15.\,5.\,1862 Wien – 21.\,10.\,1931 ebd.), \emph{Schriftsteller, Mediziner}!kleine Komödie@\strich\emph{Die kleine Komödie}|pw}« iſt fertig überſetzt\pwindex{Schnitzler, Arthur 15.\,5.\,1862 Wien – 21.\,10.\,1931 ebd.@\textsc{Schnitzler, Arthur} (15.\,5.\,1862 Wien – 21.\,10.\,1931 ebd.), \emph{Schriftsteller, Mediziner}!petite comédie. Mœurs viennois@\strich\emph{La petite comédie. Mœurs viennois}|pwv}, dem \textsc{\begin{otherlanguage}{french}Directeur\pwindex{Frank, Jules @\textsc{Frank, Jules}, \emph{Zeitungsredakteur}|pwv}\end{otherlanguage}} der »\textsc{Liberté\orgindex{Liberté@La Liberté|pw}}« überreicht u. von dieſem geſtern acceptirt
               worden. Sie dürfte nächſte Woche zu \label{K_L02755-1v}\edtext{erſcheinen}{\lemma{\textnormal{\emph{erscheinen}}}\Cendnote{\textnormal{Arthur Schnitzler: \emph{La Petite comédie. Mœurs viennois}\pwindex{Schnitzler, Arthur 15.\,5.\,1862 Wien – 21.\,10.\,1931 ebd.@\textsc{Schnitzler, Arthur} (15.\,5.\,1862 Wien – 21.\,10.\,1931 ebd.), \emph{Schriftsteller, Mediziner}!petite comédie. Mœurs viennois@\strich\emph{La petite comédie. Mœurs viennois}|pwk}. Übersetzt von Mme. Georges Aubry\pwindex{Aubry, [MMe. Georges] @\textsc{Aubry, [MMe. Georges]}, \emph{Übersetzerin}|pwk}. In: \emph{La Liberté}\pwindex{Liberté@\emph{La Liberté}|pwk}, Jg. 30, Nr. 11.327,
                        19. 11. 1895 bis Nr. 11.336, 28. 11. 1895 (acht
                     Teile).}}}\label{K_L02755-1} beginnen. Außer \textsc{Sudermann\pwindex{Sudermann, Hermann 30.\,9.\,1857 Macikai – 21.\,11.\,1928 Berlin@\textsc{Sudermann, Hermann} (30.\,9.\,1857 Macikai – 21.\,11.\,1928 Berlin), \emph{Schriftsteller}|pw}} biſt Du{ }ſeit Jahren der einzige deutſche Autor, von dem eine Arbeit\pwindex{Schnitzler, Arthur 15.\,5.\,1862 Wien – 21.\,10.\,1931 ebd.@\textsc{Schnitzler, Arthur} (15.\,5.\,1862 Wien – 21.\,10.\,1931 ebd.), \emph{Schriftsteller, Mediziner}!petite comédie. Mœurs viennois@\strich\emph{La petite comédie. Mœurs viennois}|pwv} im Roman-Feuilleton eines großen Pariſ\oindex{Paris@\textbf{Paris}, \emph{Hauptstadt}|pw}er Tagesblatt\pwindex{Liberté@\emph{La Liberté}|pwv}es erſcheint. Ein neuer kleiner Erfolg, zu dem ich
               Dir gratulire.\pend
           
\pstart
           {\pb}2.) Wann erſcheint die »Liebelei\pwindex{Schnitzler, Arthur 15.\,5.\,1862 Wien – 21.\,10.\,1931 ebd.@\textsc{Schnitzler, Arthur} (15.\,5.\,1862 Wien – 21.\,10.\,1931 ebd.), \emph{Schriftsteller, Mediziner}!Liebelei. Schauspiel in drei Akten@\strich\emph{Liebelei. Schauspiel in drei Akten}|pw}« als \label{K_L02755-2v}\edtext{Buch}{\lemma{\textnormal{\emph{Buch}}}\Cendnote{\textnormal{Die erste Buchausgabe\pwindex{Schnitzler, Arthur 15.\,5.\,1862 Wien – 21.\,10.\,1931 ebd.@\textsc{Schnitzler, Arthur} (15.\,5.\,1862 Wien – 21.\,10.\,1931 ebd.), \emph{Schriftsteller, Mediziner}!Liebelei. Schauspiel in drei Akten@\strich\emph{Liebelei. Schauspiel in drei Akten}|pwkv} erschien in den Folgetagen
                  nach der Berlin\oindex{Berlin@\textbf{Berlin}, \emph{Hauptstadt}|pwk}er Premiere von \emph{Liebelei}\pwindex{Schnitzler, Arthur 15.\,5.\,1862 Wien – 21.\,10.\,1931 ebd.@\textsc{Schnitzler, Arthur} (15.\,5.\,1862 Wien – 21.\,10.\,1931 ebd.), \emph{Schriftsteller, Mediziner}!Liebelei. Schauspiel in drei Akten@\strich\emph{Liebelei. Schauspiel in drei Akten}|pwk} am 4. 2. 1896 bei \emph{S.
                     Fischer}\orgindex{S. Fischer Verlag@S. Fischer Verlag|pwk}.}}}\label{K_L02755-2}? Ich erbitte mehrere Exemplare, und eines{ }ſendeſt Du wohl
               mit einer freundlichen Widmung an \textsc{Pierre Lalo\pwindex{Lalo, Pierre 6.\,9.\,1866 Puteaux – 9.\,6.\,1943 Paris@\textsc{Lalo, Pierre} (6.\,9.\,1866 Puteaux – 9.\,6.\,1943 Paris), \emph{Kritiker}|pw}}{ }\strikeout{(19. \textsc{Bvd}}{ }\strikeout{(19} (\textsc{19. Boulevard de Courcelles\oindex{Boulevard de Courcelles@\textbf{Boulevard de Courcelles}, \emph{Straße}|pw}}), der mich dieſer Tage danach fragte u. uns hoffentlich im »\textsc{Journal des Débats\pwindex{Journal des débats. Politiques et littéraires@\emph{Journal des débats. Politiques et littéraires}|pw}}« einen \label{K_L02755-3v}\edtext{Bericht}{\lemma{\textnormal{\emph{Bericht}}}\Cendnote{\textnormal{Dazu kam es nicht, aber die Buchausgabe\pwindex{Schnitzler, Arthur 15.\,5.\,1862 Wien – 21.\,10.\,1931 ebd.@\textsc{Schnitzler, Arthur} (15.\,5.\,1862 Wien – 21.\,10.\,1931 ebd.), \emph{Schriftsteller, Mediziner}!Liebelei. Schauspiel in drei Akten@\strich\emph{Liebelei. Schauspiel in drei Akten}|pwkv} wurde angezeigt: [O. V.]: \emph{Courrier des Théâtres}\pwindex{Courrier des Théâtres [Liebelei-Buchausgabe]@\emph{Courrier des Théâtres [Liebelei-Buchausgabe]}|pwk}. In: \emph{Journal des débats politiques et
                        littéraires}\pwindex{Journal des débats. Politiques et littéraires@\emph{Journal des débats. Politiques et littéraires}|pwk}, Jg. 108, Nr. 43, 13. 2. 1895, S. 3.}}}\label{K_L02755-3} darüber{ }ſchreiben wird.\pend
           
\pstart
           3.) Ich bitte Dich oder \textsc{Richard\pwindex{Beer-Hofmann, Richard 11.\,7.\,1866 Wien – 26.\,9.\,1945 New York City@\textsc{Beer-Hofmann, Richard} (11.\,7.\,1866 Wien – 26.\,9.\,1945 New York City), \emph{Schriftsteller}|pw}} um eine gute Einführung bei \textsc{Johann Strauss\pwindex{Strauss, Johann 25.\,10.\,1825 Wien – 3.\,6.\,1899 ebd.@\textsc{Strauss, Johann} (25.\,10.\,1825 Wien – 3.\,6.\,1899 ebd.), \emph{Komponist, Dirigent}|pw}}, der dieſer Tage nach \textsc{Paris\oindex{Paris@\textbf{Paris}, \emph{Hauptstadt}|pw}} kommt. Hier wird ihn natürlich \textsc{Feldmann\pwindex{Feldmann, Siegmund @\textsc{Feldmann, Siegmund}, \emph{Schriftsteller, Journalist}|pw}} in Beſchlag nehmen, und ich will {\pb}mich von
               dieſem Menſchen nicht glücklich machen laſſen. Müßt mir aber die Empfehlung bald{ }ſchicken.\pend
           
\pstart
           4.) \textsc{Hoffmannsthals\pwindex{Hofmannsthal, Hugo von 1.\,2.\,1874 Wien – 15.\,7.\,1929 Rodaun@\textsc{Hofmannsthal, Hugo von} (1.\,2.\,1874 Wien – 15.\,7.\,1929 Rodaun), \emph{Schriftsteller}|pw}}{ }\label{K_L02755-4v}\edtext{Erzählung\pwindex{Hofmannsthal, Hugo von 1.\,2.\,1874 Wien – 15.\,7.\,1929 Rodaun@\textsc{Hofmannsthal, Hugo von} (1.\,2.\,1874 Wien – 15.\,7.\,1929 Rodaun), \emph{Schriftsteller}!Märchen der 672. Nacht@\strich\emph{Das Märchen der 672. Nacht}|pwv}}{\lemma{\textnormal{\emph{Erzählung}}}\Cendnote{\textnormal{Hugo von Hofmannsthal\pwindex{Hofmannsthal, Hugo von 1.\,2.\,1874 Wien – 15.\,7.\,1929 Rodaun@\textsc{Hofmannsthal, Hugo von} (1.\,2.\,1874 Wien – 15.\,7.\,1929 Rodaun), \emph{Schriftsteller}|pwk}: \emph{Das Märchen der 672. Nacht. Geschichte des jungen
                        Kaufmannssohnes und seiner vier Diener}\pwindex{Hofmannsthal, Hugo von 1.\,2.\,1874 Wien – 15.\,7.\,1929 Rodaun@\textsc{Hofmannsthal, Hugo von} (1.\,2.\,1874 Wien – 15.\,7.\,1929 Rodaun), \emph{Schriftsteller}!Märchen der 672. Nacht@\strich\emph{Das Märchen der 672. Nacht}|pwk}. In: \emph{Die Zeit}\pwindex{Zeit. Wiener Wochenschrift@\emph{Die Zeit. Wiener Wochenschrift}|pwk}, Bd. 5, Nr. 57, 2. 11. 1895, S. 79–80; Nr. 58, 9. 11. 1895, S. 95–96; Nr. 59, 16. 11. 1895, S. 111–112.}}}\label{K_L02755-4} in der »Zeit\pwindex{Zeit. Wiener Wochenschrift@\emph{Die Zeit. Wiener Wochenschrift}|pw}« mißfällt mir{ }ſehr.\pend
           
\pstart
           5.) Wer iſt der Maler\pwindex{Fanto, Leonhard 8.\,9.\,1874 Wien – 16.\,12.\,1940 Dresden@\textsc{Fanto, Leonhard} (8.\,9.\,1874 Wien – 16.\,12.\,1940 Dresden), \emph{Maler, Bühnenbildner}|pwv}{ }\textsc{Fanto\pwindex{Fanto, Leonhard 8.\,9.\,1874 Wien – 16.\,12.\,1940 Dresden@\textsc{Fanto, Leonhard} (8.\,9.\,1874 Wien – 16.\,12.\,1940 Dresden), \emph{Maler, Bühnenbildner}|pw}}? Er iſt zu mir gekommen mit einer Empfehlung von \textsc{Bahr\pwindex{Bahr, Hermann 19.\,7.\,1863 Linz – 15.\,1.\,1934 München@\textsc{Bahr, Hermann} (19.\,7.\,1863 Linz – 15.\,1.\,1934 München), \emph{Schriftsteller, Kritiker}|pw}}, was bereits{ }ſehr gegen ihn{ }ſpricht. Auch mag ich ihn perſönlich nicht, es{ }ſteckt in ihm viel mit Wohlwollen umwickelter Neid. Kann der Burſche was?\pend
           
\pstart
           6.) Wüßte ich nur, {\pb}\strikeout{wies} wie’s Dir geht!\pend
           
\pstart
           8.) Grüß’ Dich Gott!\pend
           
\pstart
           In Treue {\\[\baselineskip]}Dein {\\[\baselineskip]}\spacefill\mbox{Paul Goldmann}\pend
           \leftskip=0em{}\selectlanguage{ngerman}\endnumbering\briefempfaengerindex{Schnitzler, Arthur@\textsc{Schnitzler, Arthur}!zzzGoldmann, Paul@\emph{von Paul Goldmann}!1895-11-131@{13. 11. [1895]}|)be}\mylabel{L02755h}  \newcommand{\dateiname}{L02755}\newcommand{\titel}{Paul Goldmann an Arthur Schnitzler, 13. 11. [1895]}\newcommand{\editorInnen}{Martin Anton Müller und Laura Untner}%% latex-leseansicht-abspann.tex
%% Abspann für die Leseansicht.
%% Der Schalter \ifkorrekturansicht ist bereits durch den Vorspann gesetzt.

%% latex-abspann.tex
%% Gemeinsamer Abspann für Korrekturansicht und Leseansicht.
%% Setzt den Schalter \ifkorrekturansicht voraus (gesetzt in den
%% einbindenden Dateien latex-korrekturansicht-abspann.tex bzw.
%% latex-leseansicht-abspann.tex).
%% ---------------------------------------------------------------

\normalsize

% Das esempio-Environment wird nur in der Leseansicht benötigt
\ifkorrekturansicht\else
\newenvironment{esempio}[3]%
{
    \vspace{1.5ex}
    \rlap{\underline{#1}}
    \par
    \setlength{\parindent}{0cm}
    \nopagebreak
    \leftskip=#2cm
    \rightskip=#3cm
}
{
    \par
}
\fi

\doendnotes{C}
\bigskip
\vfill

\clearpage

\footnotesize

\ifkorrekturansicht
  \lohead{\textsc{register}}
\fi

% theindex-Environment neu definieren ohne reledmac
\makeatletter
\renewenvironment{theindex}{%
  \ifkorrekturansicht
    \section*{\indexname}%
  \else
    \subsubsection*{Index der erwähnten Entitäten}%
  \fi
  \setlength{\parindent}{0pt}%
  \setlength{\parskip}{0pt plus 0.3pt}%
  \let\item\@idxitem
}{%
  \ifkorrekturansicht\clearpage\fi
}
\makeatother

\IfFileExists{\jobname-pw.ind}{\input{\jobname-pw.ind}}{}

% Quellenangabe nur in der Leseansicht
\ifkorrekturansicht\else
% Fallback-Definitionen, falls die .tex-Datei \titel etc. nicht gesetzt hat
\providecommand{\titel}{}
\providecommand{\editorInnen}{}
\providecommand{\dateiname}{\jobname}

\vspace{3cm}

\vfill

\footnotesize
\textsc{Quelle}: \titel. Herausgegeben von {\editorInnen}. In: \emph{Arthur Schnitzler: Briefwechsel mit Autorinnen und Autoren}.
 Digitale Edition, https://schnitzler-briefe.acdh.oeaw.ac.at/{\dateiname}.html (Stand \today)
\fi

\end{document}


