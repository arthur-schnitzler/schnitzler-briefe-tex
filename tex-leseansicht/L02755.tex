%% latex-leseansicht-vorspann.tex
%% Vorspann für die Leseansicht.
%% Lädt die gemeinsame Datei latex-vorspann.tex mit nicht gesetztem Schalter.

\newif\ifkorrekturansicht
\korrekturansichtfalse

\input{../tex-inputs/latex-vorspann}


         
         \renewcommand{\erwaehntePersonen}{Personen: [MMe. Georges] Aubry, Hermann Bahr, Richard Beer-Hofmann, Leonhard Fanto, Siegmund Feldmann, Jules Frank, Paul Goldmann, Hugo von Hofmannsthal, Pierre Lalo, Leopold Sonnemann, Johann Strauss, Hermann Sudermann}
         \renewcommand{\erwaehnteInstitutionen}{Institutionen: Frankfurter Zeitung, La Liberté, S. Fischer Verlag}
         \renewcommand{\erwaehnteOrte}{Orte: Berlin, Boulevard de Courcelles, Paris, Wien, rue Feydeau}
         \renewcommand{\erwaehnteWerke}{Werke: Courrier des Théâtres [Liebelei-Buchausgabe], Das Märchen der 672. Nacht, Die Zeit. Wiener Wochenschrift, Die kleine Komödie, Frankfurter Zeitung, Journal des débats. Politiques et littéraires, La Liberté, La petite comédie. Mœurs viennois, Liebelei. Schauspiel in drei Akten}
               \section[Paul Goldmann an Arthur Schnitzler, 13. 11. {[}1895{]}]{ Paul Goldmann an Arthur Schnitzler, 13. 11. {[}1895{]}}\nopagebreak\mylabel{v}\rehead{ }\begin{ledgroupsized}[t]{13cm}\normalsize\beginnumbering\briefempfaengerindex{Schnitzler, Arthur@\textsc{Schnitzler, Arthur}!zzzGoldmann, Paul@\emph{von Paul Goldmann}!1895-11-131@{13. 11. {[}1895{]}}|(be} \toendnotes[C]{\smallbreak\pagebreak[2]} \Standort{DLA, A:Schnitzler, HS.NZ85.1.3165.}
\physDesc{Brief, 1 Blatt, 4 Seiten, 1436 Zeichen
\newline{}Handschrift: blaue Tinte, deutsche Kurrent
\newline{}Schnitzler: 1) mit Bleistift das Jahr »95« vermerkt  2) mit rotem Buntstift drei Unterstreichungen}\toendnotes[C]{\smallbreak}\pstart
           \noindent{}{\pb}\textcolor{gray}{\textbf{\textbf{Frankfurter Zeitung\orgindex{Frankfurter Zeitung@Frankfurter Zeitung|pw}}}}\pend
           \pstart
           \textcolor{gray}{\textbf{(\begin{otherlanguage}{french}Gazette de Francfort\end{otherlanguage}\orgindex{Frankfurter Zeitung@Frankfurter Zeitung|pw}). }}\pend
           \pstart
           \textcolor{gray}{\textbf{\textbf{\begin{otherlanguage}{french}Fondateur M. L.
                              Sonnemann\pwindex{Sonnemann, Leopold 1831-10-29 – 1909-10-30@\textsc{Sonnemann, Leopold} (1831-10-29 – 1909-10-30), \emph{Journalist, Herausgeber}|pw}\end{otherlanguage}.}}}\pend
           \pstart
           \begin{otherlanguage}{french}\textcolor{gray}{\textbf{Journal\pwindex{?? Werk@Nicht ermittelte Verfasserinnen und Verfasser!Frankfurter Zeitung1856 – 1943@\emph{Frankfurter Zeitung} {[}1856 – 1943{]}|pwv} politique,
                           financier,}}\end{otherlanguage}\hfill \textsc{Paris\oindex{Paris@\textbf{Paris}|pw}}, 13. November.\pend
           \pstart
           \begin{otherlanguage}{french}\textcolor{gray}{\textbf{commercial et littéraire.}}\end{otherlanguage}\pend
           \pstart
           \begin{otherlanguage}{french}\textcolor{gray}{\textbf{\textbf{Paraissant trois fois par jour.}}}\end{otherlanguage}\pend
           \pstart
           \begin{otherlanguage}{french}\textcolor{gray}{\textbf{\textbf{Bureau à Paris\oindex{Paris@\textbf{Paris}|pw}:}}}\end{otherlanguage}\pend
           \pstart
           \begin{otherlanguage}{french}\textcolor{gray}{\textbf{\textbf{24. Rue Feydeau\oindex{rue Feydeau@\textbf{rue Feydeau}|pw}.}}}\end{otherlanguage}\pend
           \pstart\center{}Mein lieber Freund,\pend\pstart
           Die Arbeit dauert fort, und den großen Brief kann ich noch immer nicht ſchreiben.
               Alſo den kleinen.\pend
           \pstart
           1.) Die »Kleine Komödie\pwindex{Schnitzler, Arthur 15.05.1862 – 21.10.1931@\textsc{Schnitzler, Arthur} (15.05.1862 – 21.10.1931), \emph{Schriftsteller, Mediziner}!kleine Komoedie1895-08-01@\strich\emph{Die kleine Komödie} {[}1895-08-01{]}|pw}« iſt fertig überſetzt\pwindex{Schnitzler, Arthur 15.05.1862 – 21.10.1931@\textsc{Schnitzler, Arthur} (15.05.1862 – 21.10.1931), \emph{Schriftsteller, Mediziner}!petite comedie. Mœurs viennois1895-11-19 – 1895-11-28@\strich\emph{La petite comédie. Mœurs viennois} {[}1895-11-19 – 1895-11-28{]}|pwv}, dem \textsc{\begin{otherlanguage}{french}Directeur\pwindex{Frank, Jules @\textsc{Frank, Jules}, \emph{Zeitungsredakteur}|pwv}\end{otherlanguage}} der »\textsc{Liberté\orgindex{Liberte@La Liberté|pw}}« überreicht u. von dieſem geſtern acceptirt
               worden. Sie dürfte nächſte Woche zu \label{K_L02755-1v}\edtext{erſcheinen}{\lemma{\textnormal{\emph{erſcheinen}}}\Cendnote{\textnormal{Arthur Schnitzler\pwindex{Schnitzler, Arthur 15.05.1862 – 21.10.1931@\textsc{Schnitzler, Arthur} (15.05.1862 – 21.10.1931), \emph{Schriftsteller, Mediziner}|pwk}: \emph{La Petite comédie. Mœurs viennois}\pwindex{Schnitzler, Arthur 15.05.1862 – 21.10.1931@\textsc{Schnitzler, Arthur} (15.05.1862 – 21.10.1931), \emph{Schriftsteller, Mediziner}!petite comedie. Mœurs viennois1895-11-19 – 1895-11-28@\strich\emph{La petite comédie. Mœurs viennois} {[}1895-11-19 – 1895-11-28{]}|pwk}. Übersetzt von Mme. Georges Aubry\pwindex{Aubry, [MMe. Georges] @\textsc{Aubry, [MMe. Georges]}, \emph{Übersetzerin}|pwk}. In: \emph{La Liberté}\pwindex{?? Werk@Nicht ermittelte Verfasserinnen und Verfasser!Liberte1865-07-16 – 1940-06-11@\emph{La Liberté} {[}1865-07-16 – 1940-06-11{]}|pwk}, Jg. 30, Nr. 11.327,
                        19. 11. 1895 bis Nr. 11.336, 28. 11. 1895 (acht
                     Teile).}}}\label{K_L02755-1h} beginnen. Außer \textsc{Sudermann\pwindex{Sudermann, Hermann 30.09.1857 – 21.11.1928@\textsc{Sudermann, Hermann} (30.09.1857 – 21.11.1928), \emph{Schriftsteller}|pw}} biſt Du ſeit Jahren der einzige deutſche Autor, von dem eine Arbeit\pwindex{Schnitzler, Arthur 15.05.1862 – 21.10.1931@\textsc{Schnitzler, Arthur} (15.05.1862 – 21.10.1931), \emph{Schriftsteller, Mediziner}!petite comedie. Mœurs viennois1895-11-19 – 1895-11-28@\strich\emph{La petite comédie. Mœurs viennois} {[}1895-11-19 – 1895-11-28{]}|pwv} im Roman-Feuilleton eines großen Pariſ\oindex{Paris@\textbf{Paris}|pw}er Tagesblatt\pwindex{?? Werk@Nicht ermittelte Verfasserinnen und Verfasser!Liberte1865-07-16 – 1940-06-11@\emph{La Liberté} {[}1865-07-16 – 1940-06-11{]}|pwv}es erſcheint. Ein neuer kleiner Erfolg, zu dem ich
               Dir gratulire.\pend
           \pstart
           {\pb}2.) Wann erſcheint die »Liebelei\pwindex{Schnitzler, Arthur 15.05.1862 – 21.10.1931@\textsc{Schnitzler, Arthur} (15.05.1862 – 21.10.1931), \emph{Schriftsteller, Mediziner}!Liebelei. Schauspiel in drei Akten1895-10-09@\strich\emph{Liebelei. Schauspiel in drei Akten} {[}1895-10-09{]}|pw}« als \label{K_L02755-2v}\edtext{Buch}{\lemma{\textnormal{\emph{Buch}}}\Cendnote{\textnormal{Die erste Buchausgabe\pwindex{Schnitzler, Arthur 15.05.1862 – 21.10.1931@\textsc{Schnitzler, Arthur} (15.05.1862 – 21.10.1931), \emph{Schriftsteller, Mediziner}!Liebelei. Schauspiel in drei Akten1895-10-09@\strich\emph{Liebelei. Schauspiel in drei Akten} {[}1895-10-09{]}|pwkv} erschien in den Folgetagen
                  nach der Berlin\oindex{Berlin@\textbf{Berlin}|pwk}er Premiere von \emph{Liebelei}\pwindex{Schnitzler, Arthur 15.05.1862 – 21.10.1931@\textsc{Schnitzler, Arthur} (15.05.1862 – 21.10.1931), \emph{Schriftsteller, Mediziner}!Liebelei. Schauspiel in drei Akten1895-10-09@\strich\emph{Liebelei. Schauspiel in drei Akten} {[}1895-10-09{]}|pwk} am 4. 2. 1896 bei \emph{S.
                     Fischer}\orgindex{S. Fischer Verlag@S. Fischer Verlag|pwk}.}}}\label{K_L02755-2h}? Ich erbitte mehrere Exemplare, und eines ſendeſt Du wohl
               mit einer freundlichen Widmung an \textsc{Pierre Lalo\pwindex{Lalo, Pierre 1866-09-06 – 1943-06-09@\textsc{Lalo, Pierre} (1866-09-06 – 1943-06-09), \emph{Kritiker}|pw}}{ }\strikeout{(19. \textsc{Bvd}}{ }\strikeout{(19} (\textsc{19. Boulevard de Courcelles\oindex{Boulevard de Courcelles@\textbf{Boulevard de Courcelles}|pw}}), der mich dieſer Tage danach fragte u. uns hoffentlich im »\textsc{Journal des Débats\pwindex{?? Werk@Nicht ermittelte Verfasserinnen und Verfasser!Journal des debats. Politiques et litteraires1789 – 1944@\emph{Journal des débats. Politiques et littéraires} {[}1789 – 1944{]}|pw}}« einen \label{K_L02755-3v}\edtext{Bericht}{\lemma{\textnormal{\emph{Bericht}}}\Cendnote{\textnormal{Dazu kam es nicht, aber die Buchausgabe\pwindex{Schnitzler, Arthur 15.05.1862 – 21.10.1931@\textsc{Schnitzler, Arthur} (15.05.1862 – 21.10.1931), \emph{Schriftsteller, Mediziner}!Liebelei. Schauspiel in drei Akten1895-10-09@\strich\emph{Liebelei. Schauspiel in drei Akten} {[}1895-10-09{]}|pwkv} wurde angezeigt: [O. V.]: \emph{Courrier des Théâtres}\pwindex{?? Werk@Nicht ermittelte Verfasserinnen und Verfasser!Courrier des Theâtres [Liebelei-Buchausgabe]1896-02-13@\emph{Courrier des Théâtres [Liebelei-Buchausgabe]} {[}1896-02-13{]}|pwk}. In: \emph{Journal des débats politiques et
                        littéraires}\pwindex{?? Werk@Nicht ermittelte Verfasserinnen und Verfasser!Journal des debats. Politiques et litteraires1789 – 1944@\emph{Journal des débats. Politiques et littéraires} {[}1789 – 1944{]}|pwk}, Jg. 108, Nr. 43, 13. 2. 1895, S. 3.}}}\label{K_L02755-3h} darüber ſchreiben wird.\pend
           \pstart
           3.) Ich bitte Dich oder \textsc{Richard\pwindex{Beer-Hofmann, Richard 1866-07-11 – 1945-09-26@\textsc{Beer-Hofmann, Richard} (1866-07-11 – 1945-09-26), \emph{Schriftsteller}|pw}} um eine gute Einführung bei \textsc{Johann Strauss\pwindex{Strauss, Johann 25.10.1825 – 03.06.1899@\textsc{Strauss, Johann} (25.10.1825 – 03.06.1899), \emph{Komponist, Dirigent}|pw}}, der dieſer Tage nach \textsc{Paris\oindex{Paris@\textbf{Paris}|pw}} kommt. Hier wird ihn natürlich \textsc{Feldmann\pwindex{Feldmann, Siegmund @\textsc{Feldmann, Siegmund}, \emph{Schriftsteller, Journalist}|pw}} in Beſchlag nehmen, und ich will {\pb}mich von
               dieſem Menſchen nicht glücklich machen laſſen. Müßt mir aber die Empfehlung bald
               ſchicken.\pend
           \pstart
           4.) \textsc{Hoffmannsthals\pwindex{Hofmannsthal, Hugo von 1874-02-01 – 1929-07-15@\textsc{Hofmannsthal, Hugo von} (1874-02-01 – 1929-07-15), \emph{Schriftsteller}|pw}}{ }\label{K_L02755-4v}\edtext{Erzählung\pwindex{Hofmannsthal, Hugo von 1874-02-01 – 1929-07-15@\textsc{Hofmannsthal, Hugo von} (1874-02-01 – 1929-07-15), \emph{Schriftsteller}!Maerchen der 672. Nacht2.11.1895 – 16.11.1895@\strich\emph{Das Märchen der 672. Nacht} {[}2.11.1895 – 16.11.1895{]}|pwv}}{\lemma{\textnormal{\emph{Erzählung}}}\Cendnote{\textnormal{Hugo von Hofmannsthal\pwindex{Hofmannsthal, Hugo von 1874-02-01 – 1929-07-15@\textsc{Hofmannsthal, Hugo von} (1874-02-01 – 1929-07-15), \emph{Schriftsteller}|pwk}: \emph{Das Märchen der 672. Nacht. Geschichte des jungen
                        Kaufmannssohnes und seiner vier Diener}\pwindex{Hofmannsthal, Hugo von 1874-02-01 – 1929-07-15@\textsc{Hofmannsthal, Hugo von} (1874-02-01 – 1929-07-15), \emph{Schriftsteller}!Maerchen der 672. Nacht2.11.1895 – 16.11.1895@\strich\emph{Das Märchen der 672. Nacht} {[}2.11.1895 – 16.11.1895{]}|pwk}. In: \emph{Die Zeit}\pwindex{Zeit. Wiener Wochenschrift1894 – 1904@\emph{Die Zeit. Wiener Wochenschrift} {[}1894 – 1904{]}|pwk}, Bd. 5, Nr. 57, 2. 11. 1895, S. 79–80; Nr. 58, 9. 11. 1895, S. 95–96; Nr. 59, 16. 11. 1895, S. 111–112.}}}\label{K_L02755-4h} in der »Zeit\pwindex{Zeit. Wiener Wochenschrift1894 – 1904@\emph{Die Zeit. Wiener Wochenschrift} {[}1894 – 1904{]}|pw}« mißfällt mir ſehr.\pend
           \pstart
           5.) Wer iſt der Maler\pwindex{Fanto, Leonhard 08.09.1874 – 1940-12-16@\textsc{Fanto, Leonhard} (08.09.1874 – 1940-12-16), \emph{Bildender Künstler, Bildender Künstler}|pwv}{ }\textsc{Fanto\pwindex{Fanto, Leonhard 08.09.1874 – 1940-12-16@\textsc{Fanto, Leonhard} (08.09.1874 – 1940-12-16), \emph{Bildender Künstler, Bildender Künstler}|pw}}? Er iſt zu mir gekommen mit einer Empfehlung von \textsc{Bahr\pwindex{Bahr, Hermann 19.07.1863 – 15.01.1934@\textsc{Bahr, Hermann} (19.07.1863 – 15.01.1934), \emph{Schriftsteller, Kritiker}|pw}}, was bereits ſehr gegen ihn ſpricht. Auch mag ich ihn perſönlich nicht, es
               ſteckt in ihm viel mit Wohlwollen umwickelter Neid. Kann der Burſche was?\pend
           \pstart
           6.) Wüßte ich nur, {\pb}\strikeout{wies} wie’s Dir geht!\pend
           \pstart
           8.) Grüß’ Dich Gott!\pend
           \pstart
           In Treue {\\[\baselineskip]}Dein {\\[\baselineskip]}\spacefill\mbox{Paul Goldmann}\pend
           \leftskip=0em{}
         
         \endnumbering\mylabel{h}\end{ledgroupsized}  \newcommand{\dateiname}{L02755}\newcommand{\titel}{Paul Goldmann an Arthur Schnitzler, 13. 11. [1895]}\newcommand{\editorInnen}{Martin Anton Müller und Laura Untner}%% latex-leseansicht-abspann.tex
%% Abspann für die Leseansicht.
%% Der Schalter \ifkorrekturansicht ist bereits durch den Vorspann gesetzt.

%% latex-abspann.tex
%% Gemeinsamer Abspann für Korrekturansicht und Leseansicht.
%% Setzt den Schalter \ifkorrekturansicht voraus (gesetzt in den
%% einbindenden Dateien latex-korrekturansicht-abspann.tex bzw.
%% latex-leseansicht-abspann.tex).
%% ---------------------------------------------------------------

\normalsize

% Das esempio-Environment wird nur in der Leseansicht benötigt
\ifkorrekturansicht\else
\newenvironment{esempio}[3]%
{
    \vspace{1.5ex}
    \rlap{\underline{#1}}
    \par
    \setlength{\parindent}{0cm}
    \nopagebreak
    \leftskip=#2cm
    \rightskip=#3cm
}
{
    \par
}
\fi

\doendnotes{C}
\bigskip
\vfill

\clearpage

\footnotesize

\ifkorrekturansicht
  \lohead{\textsc{register}}
\fi

% theindex-Environment neu definieren ohne reledmac
\makeatletter
\renewenvironment{theindex}{%
  \ifkorrekturansicht
    \section*{\indexname}%
  \else
    \subsubsection*{Index der erwähnten Entitäten}%
  \fi
  \setlength{\parindent}{0pt}%
  \setlength{\parskip}{0pt plus 0.3pt}%
  \let\item\@idxitem
}{%
  \ifkorrekturansicht\clearpage\fi
}
\makeatother

\IfFileExists{\jobname-pw.ind}{\input{\jobname-pw.ind}}{}

% Quellenangabe nur in der Leseansicht
\ifkorrekturansicht\else
% Fallback-Definitionen, falls die .tex-Datei \titel etc. nicht gesetzt hat
\providecommand{\titel}{}
\providecommand{\editorInnen}{}
\providecommand{\dateiname}{\jobname}

\vspace{3cm}

\vfill

\footnotesize
\textsc{Quelle}: \titel. Herausgegeben von {\editorInnen}. In: \emph{Arthur Schnitzler: Briefwechsel mit Autorinnen und Autoren}.
 Digitale Edition, https://schnitzler-briefe.acdh.oeaw.ac.at/{\dateiname}.html (Stand \today)
\fi

\end{document}


      