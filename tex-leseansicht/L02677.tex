%% latex-leseansicht-vorspann.tex
%% Vorspann für die Leseansicht.
%% Lädt die gemeinsame Datei latex-vorspann.tex mit nicht gesetztem Schalter.

\newif\ifkorrekturansicht
\korrekturansichtfalse

\input{../tex-inputs/latex-vorspann}


               \section[Paul Goldmann an Arthur Schnitzler, 24. 12. {[}1891{]}]{ Paul Goldmann an Arthur Schnitzler, 24. 12. {[}1891{]}}\nopagebreak\mylabel{v}\rehead{ }\begin{ledgroupsized}[t]{13cm}\normalsize\beginnumbering\briefempfaengerindex{Schnitzler, Arthur@\textsc{Schnitzler, Arthur}!zzzGoldmann, Paul@\emph{von Paul Goldmann}!1891-12-241@{24. 12. {[}1891{]}}|(be} \toendnotes[C]{\smallbreak\pagebreak[2]} \Standort{DLA, A:Schnitzler, HS.NZ85.1.3162.}
\physDesc{Brief, 1 Blatt, 1 Seite
\newline{}Handschrift: schwarze Tinte, deutsche Kurrent
\newline{}Schnitzler: mit Bleistift das Jahr »1891« vermerkt }\toendnotes[C]{\smallbreak}\pstart
           \raggedleft{}{\pb}24. December –\pend
           \pstart
           Weihnachtsabend. \label{K_L02677-1v}\edtext{Buden}{\lemma{\textnormal{\emph{Buden}}}\Cendnote{\textnormal{Schaubuden,
                  Verkaufsstände}}}\label{K_L02677-1h} auf den \textsc{Boulevards}, und eine
               dichte Menge an ihnen vorbei auf dem \label{K_L02677-2v}\edtext{\begin{otherlanguage}{french}Trottoir\end{otherlanguage}}{\lemma{\textnormal{\emph{Trottoir}}}\Cendnote{\textnormal{österreichisch: Bürgersteig,
                  Gehsteig}}}\label{K_L02677-2h}. Brauſen, Rauſchen, Frauenduft, Lichterglanz, Paris\oindex{Paris@\textbf{Paris}|pw}. Und ich, zur Straße verurtheilt, und ſelbſt auf der
               Straße ein Fremder. Sorgenberg, gedehmüthigt, zukunftverzweifelnd, von einer Dirne
               beſchmutzt. Ein Zufall führt mich am Hauſe vorüber. Die Zeitung\pwindex{Frankfurter Zeitung1856 – 1943@\emph{Frankfurter Zeitung}|pwv}, »\label{K_L02677-3v}\edtext{Weihnachtseinkäufe\pwindex{Schnitzler, Arthur 15.05.1862 – 21.10.1931@\textsc{Schnitzler, Arthur} (15.05.1862 – 21.10.1931), \emph{Schriftsteller, Mediziner}!Weihnachts-Einkaeufe24. 12. 1891@\strich\emph{Weihnachts-Einkäufe} {[}24. 12. 1891{]}|pw}}{\lemma{\textnormal{\emph{Weihnachtseinkäufe}}}\Cendnote{\textnormal{Arthur Schnitzler: \emph{Weihnachts-Einkäufe}\pwindex{Schnitzler, Arthur 15.05.1862 – 21.10.1931@\textsc{Schnitzler, Arthur} (15.05.1862 – 21.10.1931), \emph{Schriftsteller, Mediziner}!Weihnachts-Einkaeufe24. 12. 1891@\strich\emph{Weihnachts-Einkäufe} {[}24. 12. 1891{]}|pwk}.
                     In: \emph{Frankfurter Zeitung}\pwindex{Frankfurter Zeitung1856 – 1943@\emph{Frankfurter Zeitung}|pwk}, Jg. 36, Nr. 358,
                        24. 12. 1891, S. 1–2.}}}\label{K_L02677-3h}«. Mein lieber, lieber
               Freund, wie danke ich Dir für dieſen Weihnachtsgruß, der nicht beabſichtigt war und
               doch in’s tiefſte Herz traf. Ich gehe ſchlafen, mit ein paar Thränen in den Augen.
               Was für ein großer Künſtler biſt Du, mein Sohn!\pend
           \pstart
           Gute Nacht!\pend
           \endnumbering\briefempfaengerindex{Schnitzler, Arthur@\textsc{Schnitzler, Arthur}!zzzGoldmann, Paul@\emph{von Paul Goldmann}!1891-12-241@{24. 12. {[}1891{]}}|)be}\mylabel{h}\end{ledgroupsized}  \newcommand{\dateiname}{L02677}\newcommand{\titel}{Paul Goldmann an Arthur Schnitzler, 24. 12. [1891]}\newcommand{\editorInnen}{Martin Anton Müller und Laura Untner}
            \footnotesize
\begin{ledgroupsized}[t]{11.5cm}
\doendnotes{C}
\end{ledgroupsized}
         %% latex-leseansicht-abspann.tex
%% Abspann für die Leseansicht.
%% Der Schalter \ifkorrekturansicht ist bereits durch den Vorspann gesetzt.

%% latex-abspann.tex
%% Gemeinsamer Abspann für Korrekturansicht und Leseansicht.
%% Setzt den Schalter \ifkorrekturansicht voraus (gesetzt in den
%% einbindenden Dateien latex-korrekturansicht-abspann.tex bzw.
%% latex-leseansicht-abspann.tex).
%% ---------------------------------------------------------------

\normalsize

% Das esempio-Environment wird nur in der Leseansicht benötigt
\ifkorrekturansicht\else
\newenvironment{esempio}[3]%
{
    \vspace{1.5ex}
    \rlap{\underline{#1}}
    \par
    \setlength{\parindent}{0cm}
    \nopagebreak
    \leftskip=#2cm
    \rightskip=#3cm
}
{
    \par
}
\fi

\doendnotes{C}
\bigskip
\vfill

\clearpage

\footnotesize

\ifkorrekturansicht
  \lohead{\textsc{register}}
\fi

% theindex-Environment neu definieren ohne reledmac
\makeatletter
\renewenvironment{theindex}{%
  \ifkorrekturansicht
    \section*{\indexname}%
  \else
    \subsubsection*{Index der erwähnten Entitäten}%
  \fi
  \setlength{\parindent}{0pt}%
  \setlength{\parskip}{0pt plus 0.3pt}%
  \let\item\@idxitem
}{%
  \ifkorrekturansicht\clearpage\fi
}
\makeatother

\IfFileExists{\jobname-pw.ind}{\input{\jobname-pw.ind}}{}

% Quellenangabe nur in der Leseansicht
\ifkorrekturansicht\else
% Fallback-Definitionen, falls die .tex-Datei \titel etc. nicht gesetzt hat
\providecommand{\titel}{}
\providecommand{\editorInnen}{}
\providecommand{\dateiname}{\jobname}

\vspace{3cm}

\vfill

\footnotesize
\textsc{Quelle}: \titel. Herausgegeben von {\editorInnen}. In: \emph{Arthur Schnitzler: Briefwechsel mit Autorinnen und Autoren}.
 Digitale Edition, https://schnitzler-briefe.acdh.oeaw.ac.at/{\dateiname}.html (Stand \today)
\fi

\end{document}


      