\input{../tex-inputs/latex-pdf-vorspann}
\begin{center}
            \textcolor{red}{ENTWURF. ENTZIFFERUNG NOCH NICHT KORREKTURGELESEN}
                      \end{center}
            
               \section[Therese Rie-Andro an Arthur Schnitzler, 6. 1. 1928]{ Therese Rie-Andro an Arthur Schnitzler, 6. 1. 1928}\nopagebreak\mylabel{v}\rehead{ }\begin{ledgroupsized}[t]{13cm}\normalsize\beginnumbering\briefempfaengerindex{Schnitzler, Arthur@\textsc{Schnitzler, Arthur}!zzzRie, Therese@\emph{von Therese Rie}!1928-01-061@{6. 1. 1928}|(be} \toendnotes[C]{\smallbreak\pagebreak[2]} \Standort{CUL, Schnitzler, B 658.}
\physDesc{Brief, 1 Blatt, 1 Seite
\newline{}Handschrift: blaue Tinte, lateinische Kurrent
\newline{}Schnitzler: 1) mit Bleistift beschriftet: »\textsc{Andro}« 2) mit rotem Buntstift eine Unterstreichung}\toendnotes[C]{\smallbreak}\pstart
           \raggedleft{}{\pb}Wien\oindex{Wien@\textbf{Wien}|pw}, Dreikönig
                     1928.\pend
           \pstart
           \raggedleft{}IV, Schönburgstr. 48\oindex{Schoenburgstrasse@\textbf{Schönburgstraße}|pw}.\pend
           \pstart{}Verehrter Herr Doktor,\pend\pstart
           Ich habe mich so in Ihr \label{K_L02577-1v}\edtext{Buch\pwindex{Schnitzler, Arthur 15.05.1862 – 21.10.1931@\textsc{Schnitzler, Arthur} (15.05.1862 – 21.10.1931), \emph{Schriftsteller, Mediziner}!Geist im Wort und der Geist in der Tat1927@\strich\emph{Der Geist im Wort und der Geist in der Tat} {[}1927{]}|pwv}}{\lemma{\textnormal{\emph{Buch}}}\Cendnote{\textnormal{Schnitzler\pwindex{Schnitzler, Arthur 15.05.1862 – 21.10.1931@\textsc{Schnitzler, Arthur} (15.05.1862 – 21.10.1931), \emph{Schriftsteller, Mediziner}|pwk} übersandte ihr nach dem letzten Brief \emph{Der Geist im Wort und der Geist in der Tat}\pwindex{Schnitzler, Arthur 15.05.1862 – 21.10.1931@\textsc{Schnitzler, Arthur} (15.05.1862 – 21.10.1931), \emph{Schriftsteller, Mediziner}!Geist im Wort und der Geist in der Tat1927@\strich\emph{Der Geist im Wort und der Geist in der Tat} {[}1927{]}|pwk}.}}}\label{K_L02577-1h} verlesen,
               daſs ich vergessen habe, Ihnen zu danken – und es war doch so lieb von Ihnen! So darf
               ich Ihnen heute zweimal Dank sagen: einmal für Ihre Freundlichkeit und dann dafür,
               dass Sie den Unterschied zwischen Kontinualiſchem und Aktualiſchem (in allen Formen)
               so aufgezeigt haben, wie noch niemand vorher.\pend
           \pstart
           Ihre{\\[\baselineskip]}\spacefill\mbox{Therese Rie – Andro.}\pend
           \leftskip=0em{}\endnumbering\briefempfaengerindex{Schnitzler, Arthur@\textsc{Schnitzler, Arthur}!zzzRie, Therese@\emph{von Therese Rie}!1928-01-061@{6. 1. 1928}|)be}\mylabel{h}\end{ledgroupsized}  \newcommand{\dateiname}{L02577}\newcommand{\titel}{Therese Rie-Andro an Arthur Schnitzler, 6. 1. 1928}\newcommand{\editorInnen}{Martin Anton Müller und Gerd-Hermann Susen}\input{../tex-inputs/latex-pdf-abspann}
      