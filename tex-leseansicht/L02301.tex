%% latex-leseansicht-vorspann.tex
%% Vorspann für die Leseansicht.
%% Lädt die gemeinsame Datei latex-vorspann.tex mit nicht gesetztem Schalter.

\newif\ifkorrekturansicht
\korrekturansichtfalse

\input{../tex-inputs/latex-vorspann}


\section[Arthur Schnitzler an Richard Beer-Hofmann, 26. 8. 1918]{L02301 Arthur Schnitzler an Richard Beer-Hofmann, 26. 8. 1918}
\nopagebreak\mylabel{L02301v}
\rehead{ }\normalsize\beginnumbering\briefempfaengerindex{Beer-Hofmann, Richard@\textsc{Beer-Hofmann, Richard}!zzzSchnitzler, Arthur@\emph{von Arthur Schnitzler}!1918-08-261@{26. 8. 1918}|(be}
\toendnotes[C]{\smallbreak\pagebreak[2]}
\correspDesc{Versand  durch Arthur Schnitzler am 26. 8. 1918 in Wien
\newline{}Erhalt  durch Richard Beer-Hofmann am 28. 8. 1918 in Bad Ischl}\toendnotes[C]{\smallbreak}
\Standort{YCGL, MSS 31.}
\physDesc{Bildpostkarte, 407 Zeichen
\newline{}Handschrift: Bleistift, lateinische Kurrent
\newline{}Versand: Stempel: »\nobreak{}\oindex{Wien@\textbf{Wien}, \emph{Verwaltungsgebiet}|pwk}\textcolor{gray}{Wien}, 26. VIII. 18\nobreak{}«.  
\newline{}Beer-Hofmann: mit blauem Buntstift Erhalt und Beantwortung vermerkt: »E. B.
                                       28./VIII 18« 
\newline{}Zusatz: Postkartenmotiv mit Olga\pwindex{Schnitzler, Olga 17.\,1.\,1882 Wien – 13.\,1.\,1970 Lugano@\textsc{Schnitzler, Olga} (17.\,1.\,1882 Wien – 13.\,1.\,1970 Lugano), \emph{Schauspielerin, Sängerin}|pw}
                                 und Heinrich\pwindex{Schnitzler, Heinrich 9.\,8.\,1902 Hinterbrühl – 12.\,7.\,1982 Wien@\textsc{Schnitzler, Heinrich} (9.\,8.\,1902 Hinterbrühl – 12.\,7.\,1982 Wien), \emph{Regisseur, Schauspieler}|pw} links vor dem
                                 Haus und Schnitzler und Lili\pwindex{Cappellini, Lili 13.\,9.\,1909 Wien – 26.\,7.\,1928 Venedig@\textsc{Cappellini, Lili} (13.\,9.\,1909 Wien – 26.\,7.\,1928 Venedig)|pw}
                                 auf dem Söller }
\buchAbdrucke{\weitereDrucke{Arthur Schnitzler, Richard Beer-Hofmann: \emph{Briefwechsel 1891–1931}. Herausgegeben von Konstanze Fliedl. Wien, Zürich: \emph{Europaverlag} 1992, S. 226.} }\toendnotes[C]{\smallbreak}\pstart{}{\pb}Herrn Dr. Richard Beer Hofmann\pend{}\pstart{}Bad Ischl\oindex{Bad Ischl@\textbf{Bad Ischl}|pw}\pend{}\pstart{}Grazerstr. 56\oindex{Grazer Straße [Bad Ischl]@\textbf{Grazer Straße [Bad Ischl]}, \emph{Straße}|pw}\pend{}{\bigskip}
\pstart
           \noindent{}\centering{}{\pb}\textcolor{gray}{\textbf{Wien, XVIII, Sternwartestr. 71\oindex{Wien@\textbf{Wien}!XVIII., Währing@\textbf{XVIII., Währing}!Sternwartestraße 71@\textbf{Sternwartestraße 71}, \emph{Wohngebäude}|pw}.}}\pwindex{\textcolor{red}{\textsuperscript{XXXX indx1}}!Wien, XVIII, Sternwartestr. 71.@\strich\emph{Wien, XVIII, Sternwartestr. 71.}|pw}\pend
           \vspace{1em}
\pstart
           \raggedleft{}A. S.\pend
           \vspace{0.5em}
\pstart
           {\pb}lieber Richard, aus Salzburg\oindex{Salzburg@\textbf{Salzburg}, \emph{Verwaltungsgebiet}|pw} ist
               nun doch nichts geworden; ich fahre morgen, möglichst direct München\oindex{München@\textbf{München}|pw} – Partenkirchen\oindex{Partenkirchen@\textbf{Partenkirchen}, \emph{Teil eines besiedelten Ortes}|pw}; es
               scheint meiner Schwägerin\pwindex{Steinrück, Elisabeth 19.\,11.\,1885 – 7.\,4.\,1920 Partenkirchen@\textsc{Steinrück, Elisabeth} (19.\,11.\,1885 – 7.\,4.\,1920 Partenkirchen)|pwv}
               wieder schlechter zu gehn. Bitte um ein Wort nach P.\oindex{Partenkirchen@\textbf{Partenkirchen}, \emph{Teil eines besiedelten Ortes}|pw} (Haus Tannenberg\oindex{Haus Tannenberg@\textbf{Haus Tannenberg}, \emph{Beherbergungsgebäude}|pw}.) Hat der Herzog von Leopoldskron\oindex{Salzburg-Leopoldskron@\textbf{Salzburg-Leopoldskron}, \emph{Teil eines besiedelten Ortes}|pw}\pwindex{Reinhardt, Max 9.\,9.\,1873 Baden bei Wien – 30.\,10.\,1943 New York City@\textsc{Reinhardt, Max} (9.\,9.\,1873 Baden bei Wien – 30.\,10.\,1943 New York City), \emph{Theaterleiter, Regisseur, Schauspieler}|pwv} Ihnen einen besti{\geminationm}ten \label{K_L02301-1v}\edtext{Termin\pwindex{Beer-Hofmann, Richard 11.\,7.\,1866 Wien – 26.\,9.\,1945 New York City@\textsc{Beer-Hofmann, Richard} (11.\,7.\,1866 Wien – 26.\,9.\,1945 New York City), \emph{Schriftsteller}!Jaákobs Traum. Ein Vorspiel@\strich\emph{Jaákobs Traum. Ein Vorspiel}|pwv}}{\lemma{\textnormal{\emph{Termin}}}\Cendnote{\textnormal{Die Berlin\oindex{Berlin@\textbf{Berlin}, \emph{Hauptstadt}|pwk}er Premiere verzögerte sich bis zum
                  7. 11. 1919.}}}\label{K_L02301-1} gegeben? Ihnen ev. auch etwas über den \label{K_L02301-2v}\edtext{Termin der »Schwestern\pwindex{Schnitzler, Arthur 15.\,5.\,1862 Wien – 21.\,10.\,1931 ebd.@\textsc{Schnitzler, Arthur} (15.\,5.\,1862 Wien – 21.\,10.\,1931 ebd.), \emph{Schriftsteller, Mediziner}!Schwestern oder Casanova in Spa. Lustspiel in Versen@\strich\emph{Die Schwestern oder Casanova in Spa. Lustspiel in Versen}|pw}«}{\lemma{\textnormal{\emph{Termin der »Schwestern«}}}\Cendnote{\textnormal{Trotz eines
                  Vorvertrags vom 20. 12. 1917 kam keine Inszenierung am von Max Reinhardt\pwindex{Reinhardt, Max 9.\,9.\,1873 Baden bei Wien – 30.\,10.\,1943 New York City@\textsc{Reinhardt, Max} (9.\,9.\,1873 Baden bei Wien – 30.\,10.\,1943 New York City), \emph{Theaterleiter, Regisseur, Schauspieler}|pwk} geleiteten \emph{Deutschen Theater}\orgindex{Deutsches Theater Berlin@Deutsches Theater Berlin|pwk} zustande.}}}\label{K_L02301-2} verrathen?
               Herzlichst\pend
           \pstart \spacefill\mbox{A.}\pend{}\selectlanguage{ngerman}\endnumbering\briefempfaengerindex{Beer-Hofmann, Richard@\textsc{Beer-Hofmann, Richard}!zzzSchnitzler, Arthur@\emph{von Arthur Schnitzler}!1918-08-261@{26. 8. 1918}|)be}\mylabel{L02301h}  \newcommand{\dateiname}{L02301}\newcommand{\titel}{Arthur Schnitzler an Richard Beer-Hofmann, 26. 8. 1918}\newcommand{\editorInnen}{Martin Anton Müller und Gerd-Hermann Susen}%% latex-leseansicht-abspann.tex
%% Abspann für die Leseansicht.
%% Der Schalter \ifkorrekturansicht ist bereits durch den Vorspann gesetzt.

%% latex-abspann.tex
%% Gemeinsamer Abspann für Korrekturansicht und Leseansicht.
%% Setzt den Schalter \ifkorrekturansicht voraus (gesetzt in den
%% einbindenden Dateien latex-korrekturansicht-abspann.tex bzw.
%% latex-leseansicht-abspann.tex).
%% ---------------------------------------------------------------

\normalsize

% Das esempio-Environment wird nur in der Leseansicht benötigt
\ifkorrekturansicht\else
\newenvironment{esempio}[3]%
{
    \vspace{1.5ex}
    \rlap{\underline{#1}}
    \par
    \setlength{\parindent}{0cm}
    \nopagebreak
    \leftskip=#2cm
    \rightskip=#3cm
}
{
    \par
}
\fi

\doendnotes{C}
\bigskip
\vfill

\clearpage

\footnotesize

\ifkorrekturansicht
  \lohead{\textsc{register}}
\fi

% theindex-Environment neu definieren ohne reledmac
\makeatletter
\renewenvironment{theindex}{%
  \ifkorrekturansicht
    \section*{\indexname}%
  \else
    \subsubsection*{Index der erwähnten Entitäten}%
  \fi
  \setlength{\parindent}{0pt}%
  \setlength{\parskip}{0pt plus 0.3pt}%
  \let\item\@idxitem
}{%
  \ifkorrekturansicht\clearpage\fi
}
\makeatother

\IfFileExists{\jobname-pw.ind}{\input{\jobname-pw.ind}}{}

% Quellenangabe nur in der Leseansicht
\ifkorrekturansicht\else
% Fallback-Definitionen, falls die .tex-Datei \titel etc. nicht gesetzt hat
\providecommand{\titel}{}
\providecommand{\editorInnen}{}
\providecommand{\dateiname}{\jobname}

\vspace{3cm}

\vfill

\footnotesize
\textsc{Quelle}: \titel. Herausgegeben von {\editorInnen}. In: \emph{Arthur Schnitzler: Briefwechsel mit Autorinnen und Autoren}.
 Digitale Edition, https://schnitzler-briefe.acdh.oeaw.ac.at/{\dateiname}.html (Stand \today)
\fi

\end{document}


