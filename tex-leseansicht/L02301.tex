%% latex-korrekturansicht-vorspann.tex
%% Vorspann für die Korrekturansicht.
%% Lädt die gemeinsame Datei latex-vorspann.tex mit gesetztem Schalter.

\newif\ifkorrekturansicht
\korrekturansichttrue

\input{../tex-inputs/latex-vorspann}


\section[Arthur Schnitzler an Richard Beer-Hofmann, 26. 8. 1918]{L02301 Arthur Schnitzler an Richard Beer-Hofmann, 26. 8. 1918}
\nopagebreak\mylabel{L02301v}
\rehead{ }\normalsize\beginnumbering\briefempfaengerindex{Beer-Hofmann, Richard@\textsc{Beer-Hofmann, Richard}!zzzSchnitzler, Arthur@\emph{von Arthur Schnitzler}!1918-08-261@{26. 8. 1918}|(be}
\toendnotes[C]{\smallbreak\pagebreak[2]}\Standort{YCGL, MSS 31.}
\physDesc{Bildpostkarte, 407 Zeichen
\newline{}Handschrift: Bleistift, lateinische Kurrent
\newline{}Versand: Stempel: »\nobreak{}\textcolor{gray}{Wien}, 26. VIII. 18\nobreak{}«.  
\newline{}Beer-Hofmann: mit blauem Buntstift Erhalt und Beantwortung vermerkt: »E. B.
                                       28./VIII 18« 
\newline{}Zusatz: Postkartenmotiv mit Olga\pwindex{Schnitzler, Olga 17.01.1882 – 13.01.1970@\textsc{Schnitzler, Olga} (17.01.1882 – 13.01.1970), \emph{Schauspieler/Schauspielerin, Sänger/Sängerin}|pw}
                                 und Heinrich\pwindex{Schnitzler, Heinrich 09.08.1902 – 12.07.1982@\textsc{Schnitzler, Heinrich} (09.08.1902 – 12.07.1982), \emph{Regisseur/Regisseurin, Schauspieler/Schauspielerin}|pw} links vor dem
                                 Haus und Schnitzler und Lili\pwindex{Cappellini, Lili 13.09.1909 – 26.07.1928@\textsc{Cappellini, Lili} (13.09.1909 – 26.07.1928)|pw}
                                 auf dem Söller }
\buchAbdrucke{\weitereDrucke{Arthur Schnitzler, Richard Beer-Hofmann: \emph{Briefwechsel 1891–1931}. Wien, Zürich: \emph{Europaverlag} 1992, S. 226.} }\toendnotes[C]{\smallbreak}\pstart{}{\pb}Herrn Dr. Richard Beer Hofmann\pend{}\pstart{}Bad Ischl\oindex{Bad Ischl@\textbf{Bad Ischl}, \emph{P.PPL}|pw}\pend{}\pstart{}Grazerstr. 56\oindex{Grazer Strasse [Bad Ischl]@\textbf{Grazer Straße [Bad Ischl]}, \emph{Straße (K.STR)}|pw}\pend{}{\bigskip}
\pstart
           \noindent{}\centering{}{\pb}\textcolor{gray}{\textbf{Wien, XVIII, Sternwartestr. 71\oindex{Sternwartestrasse 71@\textbf{Sternwartestraße 71}, \emph{Wohngebäude (K.WHS)}|pw}.}}\pend
           \vspace{1em}
\pstart
           \raggedleft{}A. S.\pend
           \vspace{0.5em}
\pstart
           {\pb}lieber Richard, aus Salzburg\oindex{Salzburg@\textbf{Salzburg}, \emph{A.ADM2}|pw} ist
               nun doch nichts geworden; ich fahre morgen, möglichst direct München\oindex{Muenchen@\textbf{München}, \emph{P.PPLA}|pw} – Partenkirchen\oindex{Partenkirchen@\textbf{Partenkirchen}, \emph{Teil eines besiedelten Ortes (A.BSOX)}|pw}; es
               scheint meiner Schwägerin\pwindex{Steinrueck, Elisabeth 19.11.1885 – 07.04.1920@\textsc{Steinrück, Elisabeth} (19.11.1885 – 07.04.1920)|pwv}
               wieder schlechter zu gehn. Bitte um ein Wort nach P.\oindex{Partenkirchen@\textbf{Partenkirchen}, \emph{Teil eines besiedelten Ortes (A.BSOX)}|pw} (Haus Tannenberg\oindex{Haus Tannenberg@\textbf{Haus Tannenberg}, \emph{Beherbergungsgebäude (K.BHB)}|pw}.) Hat der Herzog von Leopoldskron\oindex{Salzburg-Leopoldskron@\textbf{Salzburg-Leopoldskron}, \emph{Teil eines besiedelten Ortes (A.BSOX)}|pw}\pwindex{Reinhardt, Max 09.09.1873 – 30.10.1943@\textsc{Reinhardt, Max} (09.09.1873 – 30.10.1943), \emph{Theaterleiter/Theaterleiterin, Regisseur/Regisseurin, Schauspieler/Schauspielerin}|pwv} Ihnen einen besti{\geminationm}ten \label{K_L02301-1v}\edtext{Termin\pwindex{Jaákobs Traum. Ein Vorspiel@\emph{Jaákobs Traum. Ein Vorspiel}|pwv}}{\lemma{\textnormal{\emph{Termin}}}\Cendnote{\textnormal{Die Berlin\oindex{Berlin@\textbf{Berlin}, \emph{P.PPLC}|pwk}er Premiere verzögerte sich bis zum
                  7. 11. 1919.}}}\label{K_L02301-1} gegeben? Ihnen ev. auch etwas über den \label{K_L02301-2v}\edtext{Termin der »Schwestern\pwindex{Schwestern oder Casanova in Spa. Lustspiel in Versen@\emph{Die Schwestern oder Casanova in Spa. Lustspiel in Versen}|pw}«}{\lemma{\textnormal{\emph{Termin der »Schwestern«}}}\Cendnote{\textnormal{Trotz eines
                  Vorvertrags vom 20. 12. 1917 kam keine Inszenierung am von Max Reinhardt\pwindex{Reinhardt, Max 09.09.1873 – 30.10.1943@\textsc{Reinhardt, Max} (09.09.1873 – 30.10.1943), \emph{Theaterleiter/Theaterleiterin, Regisseur/Regisseurin, Schauspieler/Schauspielerin}|pwk} geleiteten \emph{Deutschen Theater}\orgindex{Deutsches Theater Berlin@Deutsches Theater Berlin|pwk} zustande.}}}\label{K_L02301-2} verrathen?
               Herzlichst\pend
           \pstart \spacefill\mbox{A.}\pend{}\selectlanguage{ngerman}\endnumbering\briefempfaengerindex{Beer-Hofmann, Richard@\textsc{Beer-Hofmann, Richard}!zzzSchnitzler, Arthur@\emph{von Arthur Schnitzler}!1918-08-261@{26. 8. 1918}|)be}\mylabel{L02301h}  \normalsize

\doendnotes{C}
\bigskip
\vfill

\clearpage

\footnotesize

\lohead{\textsc{register}}

% Definiere theindex-Environment komplett neu ohne reledmac
\makeatletter
\renewenvironment{theindex}{%
  \section*{\indexname}%
  \setlength{\parindent}{0pt}%
  \setlength{\parskip}{0pt plus 0.3pt}%
  \let\item\@idxitem
}{%
  \clearpage
}
\makeatother

\IfFileExists{\jobname-pw.ind}{\input{\jobname-pw.ind}}{}

\end{document}

      