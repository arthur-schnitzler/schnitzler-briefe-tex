%% latex-leseansicht-vorspann.tex
%% Vorspann für die Leseansicht.
%% Lädt die gemeinsame Datei latex-vorspann.tex mit nicht gesetztem Schalter.

\newif\ifkorrekturansicht
\korrekturansichtfalse

\input{../tex-inputs/latex-vorspann}


         
         \newcommand{\erwaehntePersonen}{Personen: Richard Beer-Hofmann, Max Reinhardt, Olga Schnitzler, Heinrich Schnitzler, Lili Schnitzler, Elisabeth Steinrück}
         \newcommand{\erwaehnteInstitutionen}{Institutionen: Deutsches Theater Berlin}
         \newcommand{\erwaehnteOrte}{Orte: Bad Ischl, Berlin, Grazer Straße, Haus Tannenberg, München, Partenkirchen, Salzburg, Salzburg-Leopoldskron, Sternwartestraße, Wien}
         \newcommand{\erwaehnteWerke}{Werke: Die Schwestern oder Casanova in Spa. Lustspiel in Versen, Jaákobs Traum. Ein Vorspiel}
               \section[Arthur Schnitzler an Richard Beer-Hofmann, 26. 8. 1918]{ Arthur Schnitzler an Richard Beer-Hofmann, 26. 8. 1918}\nopagebreak\mylabel{v}\rehead{ }\begin{ledgroupsized}[t]{13cm}\normalsize\beginnumbering \toendnotes[C]{\smallbreak\pagebreak[2]} \Standort{YCGL, MSS 31.}
\physDesc{Bildpostkarte
\newline{}Handschrift: Bleistift, lateinische Kurrent\newline{}Versand: Stempel: »\nobreak{}\textcolor{gray}{Wien}, 26. VIII. 18\nobreak{}«.  
\newline{}Beer-Hofmann: mit blauem Buntstift Erhalt und Beantwortung
                                    vermerkt: »E. B. 28./VIII 18« \newline{}Zusatz: Postkartenmotiv mit Olga\pwindex{Schnitzler, Olga 17.01.1882 – 13.01.1970@\textsc{Schnitzler, Olga} (17.01.1882 – 13.01.1970), \emph{Schauspielerin, Sängerin}|pw} und
                                    Heinrich\pwindex{Schnitzler, Heinrich 09.08.1902 – 12.07.1982@\textsc{Schnitzler, Heinrich} (09.08.1902 – 12.07.1982), \emph{Regisseur, Schauspieler}|pw} links vor dem Haus
                                 und Schnitzler und Lili\pwindex{Schnitzler, Lili 13.09.1909 – 26.07.1928@\textsc{Schnitzler, Lili} (13.09.1909 – 26.07.1928)|pw} auf dem
                                 Söller }\buchAbdrucke{\weitereDrucke{Arthur Schnitzler, Richard Beer-Hofmann: \emph{Briefwechsel 1891–1931}. Hg. Konstanze Fliedl. Wien, Zürich: \emph{Europaverlag} 1992, S. 226.} }\toendnotes[C]{\smallbreak}\pstart{}{\pb}Herrn Dr. Richard Beer Hofmann\pend{}\pstart{}Bad Ischl\oindex{Bad Ischl@\textbf{Bad Ischl}|pw}\pend{}\pstart{}Grazerstr. 56\oindex{Grazer Strasse@\textbf{Grazer Straße}|pw}\pend{}{\bigskip}\pstart
           \noindent{}\centering{}{\pb}\textcolor{gray}{\textbf{Wien, XVIII, Sternwartestr. 71\oindex{Sternwartestrasse@\textbf{Sternwartestraße}|pw}.}}\pend
           \pstart
           \noindent{}\raggedleft{}A. S.\pend
           \pstart
           {\pb}lieber Richard, aus Salzburg\oindex{Salzburg@\textbf{Salzburg}|pw} ist nun
               doch nichts geworden; ich fahre morgen, möglichst direct München\oindex{Muenchen@\textbf{München}|pw} – Partenkirchen\oindex{Partenkirchen@\textbf{Partenkirchen}|pw}; es
               scheint meiner Schwägerin\pwindex{Steinrueck, Elisabeth 19.11.1885 – 07.04.1920@\textsc{Steinrück, Elisabeth} (19.11.1885 – 07.04.1920)|pwv} wieder
               schlechter zu gehn. Bitte um ein Wort nach P.\oindex{Partenkirchen@\textbf{Partenkirchen}|pw} (Haus Tannenberg\oindex{Haus Tannenberg@\textbf{Haus Tannenberg}|pw}.) Hat der Herzog von Leopoldskron\oindex{Salzburg-Leopoldskron@\textbf{Salzburg-Leopoldskron}|pw}\pwindex{Reinhardt, Max 09.09.1873 – 30.10.1943@\textsc{Reinhardt, Max} (09.09.1873 – 30.10.1943), \emph{Theaterleiter, Regisseur, Schauspieler}|pwv} Ihnen einen besti{\geminationm}ten \label{K_L02301-v}\edtext{Termin\pwindex{Beer-Hofmann, Richard 1866-07-11 – 1945-09-26@\textsc{Beer-Hofmann, Richard} (1866-07-11 – 1945-09-26), \emph{Schriftsteller}!Jaákobs Traum. Ein Vorspiel1918-04-05@\strich\emph{Jaákobs Traum. Ein Vorspiel} {[}1918-04-05{]}|pwv}}{\lemma{\textnormal{\emph{Termin}}}\Cendnote{\textnormal{Die Berlin\oindex{Berlin@\textbf{Berlin}|pwk}er Premiere verzögerte sich bis zum
                  7. 11. 1919.}}}\label{K_L02301-h} gegeben? Ihnen ev. auch etwas über den \label{K_L02301-2v}\edtext{Termin der »Schwestern\pwindex{Schnitzler, Arthur 15.05.1862 – 21.10.1931@\textsc{Schnitzler, Arthur} (15.05.1862 – 21.10.1931), \emph{Schriftsteller, Mediziner}!Schwestern oder Casanova in Spa. Lustspiel in Versen01. 10. 1919@\strich\emph{Die Schwestern oder Casanova in Spa. Lustspiel in Versen} {[}01. 10. 1919{]}|pw}«}{\lemma{\textnormal{\emph{Termin der »Schwestern«}}}\Cendnote{\textnormal{Trotz eines
                  Vorvertrags vom 20. 12. 1917 kam keine Inszenierung am von Max Reinhardt\pwindex{Reinhardt, Max 09.09.1873 – 30.10.1943@\textsc{Reinhardt, Max} (09.09.1873 – 30.10.1943), \emph{Theaterleiter, Regisseur, Schauspieler}|pwk} geleiteten \emph{Deutschen Theater}\orgindex{Deutsches Theater Berlin@Deutsches Theater Berlin|pwk} zustande.}}}\label{K_L02301-2h} verrathen? Herzlichst\pend
           \pstart \spacefill\mbox{A.}\pend{}
         
         \endnumbering\mylabel{h}\end{ledgroupsized}  \newcommand{\dateiname}{L02301}\newcommand{\titel}{Arthur Schnitzler an Richard Beer-Hofmann, 26. 8. 1918}\newcommand{\editorInnen}{Martin Anton Müller und Gerd-Hermann Susen}%% latex-leseansicht-abspann.tex
%% Abspann für die Leseansicht.
%% Der Schalter \ifkorrekturansicht ist bereits durch den Vorspann gesetzt.

%% latex-abspann.tex
%% Gemeinsamer Abspann für Korrekturansicht und Leseansicht.
%% Setzt den Schalter \ifkorrekturansicht voraus (gesetzt in den
%% einbindenden Dateien latex-korrekturansicht-abspann.tex bzw.
%% latex-leseansicht-abspann.tex).
%% ---------------------------------------------------------------

\normalsize

% Das esempio-Environment wird nur in der Leseansicht benötigt
\ifkorrekturansicht\else
\newenvironment{esempio}[3]%
{
    \vspace{1.5ex}
    \rlap{\underline{#1}}
    \par
    \setlength{\parindent}{0cm}
    \nopagebreak
    \leftskip=#2cm
    \rightskip=#3cm
}
{
    \par
}
\fi

\doendnotes{C}
\bigskip
\vfill

\clearpage

\footnotesize

\ifkorrekturansicht
  \lohead{\textsc{register}}
\fi

% theindex-Environment neu definieren ohne reledmac
\makeatletter
\renewenvironment{theindex}{%
  \ifkorrekturansicht
    \section*{\indexname}%
  \else
    \subsubsection*{Index der erwähnten Entitäten}%
  \fi
  \setlength{\parindent}{0pt}%
  \setlength{\parskip}{0pt plus 0.3pt}%
  \let\item\@idxitem
}{%
  \ifkorrekturansicht\clearpage\fi
}
\makeatother

\IfFileExists{\jobname-pw.ind}{\input{\jobname-pw.ind}}{}

% Quellenangabe nur in der Leseansicht
\ifkorrekturansicht\else
% Fallback-Definitionen, falls die .tex-Datei \titel etc. nicht gesetzt hat
\providecommand{\titel}{}
\providecommand{\editorInnen}{}
\providecommand{\dateiname}{\jobname}

\vspace{3cm}

\vfill

\footnotesize
\textsc{Quelle}: \titel. Herausgegeben von {\editorInnen}. In: \emph{Arthur Schnitzler: Briefwechsel mit Autorinnen und Autoren}.
 Digitale Edition, https://schnitzler-briefe.acdh.oeaw.ac.at/{\dateiname}.html (Stand \today)
\fi

\end{document}


      