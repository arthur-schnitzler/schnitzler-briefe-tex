%% latex-korrekturansicht-vorspann.tex
%% Vorspann für die Korrekturansicht.
%% Lädt die gemeinsame Datei latex-vorspann.tex mit gesetztem Schalter.

\newif\ifkorrekturansicht
\korrekturansichttrue

\input{../tex-inputs/latex-vorspann}


\section[Paul Goldmann an Arthur Schnitzler, 3. 4. {[}1894{]}]{L02615 Paul Goldmann an Arthur Schnitzler, 3. 4. {[}1894{]}}
\nopagebreak\mylabel{L02615v}
\rehead{ }\normalsize\beginnumbering\briefempfaengerindex{Schnitzler, Arthur@\textsc{Schnitzler, Arthur}!zzzGoldmann, Paul@\emph{von Paul Goldmann}!1894-04-031@{3. 4. {[}1894{]}}|(be}
\toendnotes[C]{\smallbreak\pagebreak[2]}\Standort{DLA, A:Schnitzler, HS.NZ85.1.3164.}
\physDesc{Brief, 1 Blatt, 4 Seiten, 1535 Zeichen
\newline{}Handschrift: schwarze Tinte, deutsche Kurrent
\newline{}Schnitzler: 1) mit Bleistift auf dem ersten Blatt die Jahreszahl »94« vermerkt  2) mit rotem Buntstift zwei Unterstreichungen}\toendnotes[C]{\smallbreak}
\pstart
           \raggedleft{}{\pb}\textsc{Paris\oindex{Paris@\textbf{Paris}, \emph{P.PPLC}|pw}}, 3. April.\pend
           
\pstart\center{}Mein lieber Freund,\pend\vspace{0.5em}
\pstart
           Ich habe Dir für zwei liebe Briefe zu danken\substVorne{}\textsuperscript{,}\substDazwischen{},\substHinten{} und ich muß Dir immer und immer wiederholen, wie wohl mir Deine treue
               Freundſchaft thut und Deine Antheilnahme an Allem, was ich leiſte. Es gibt mir beim
               Arbeiten eine gewiſſe Anregung, wenn ich daran denke, daß ich Dein Lob verdienen muß.
               Haſt Du mein \label{K_L02615-1v}\edtext{Feuilleton\pwindex{Charles Meunier@\emph{Charles Meunier}|pwv}{ }{\pb}über den armen \textsc{Charles Meunier\pwindex{Meunier, Karl 1864 – 1894-03-22@\textsc{Meunier, Karl} (1864 – 1894-03-22), \emph{Maler/Malerin, Kupferstecher/Kupferstecherin}|pw}}}{\lemma{\textnormal{\emph{Feuilleton … Meunier}}}\Cendnote{\textnormal{Paul Goldmann\pwindex{Goldmann, Paul 31.01.1865 – 25.09.1935@\textsc{Goldmann, Paul} (31.01.1865 – 25.09.1935), \emph{Schriftsteller/Schriftstellerin, Journalist/Journalistin}|pwk}: \emph{Charles Meunier. Ein Jugendleben}\pwindex{Charles Meunier@\emph{Charles Meunier}|pwk}. In: \emph{Frankfurter Zeitung}\pwindex{Frankfurter Zeitung@\emph{Frankfurter Zeitung}|pwk}, Jg. 38, Nr. 90,
                        1. 4. 1894, Erstes Morgenblatt, S. 1–2. Das »arm«
                  bezieht sich darauf, dass der Grafiker und Maler Karl Meunier\pwindex{Meunier, Karl 1864 – 1894-03-22@\textsc{Meunier, Karl} (1864 – 1894-03-22), \emph{Maler/Malerin, Kupferstecher/Kupferstecherin}|pwk} als Nachwuchstalent im Alter von 30 Jahren gestorben war, ohne sich
                  in weiteren Kreisen einen anderen Namen gemacht zu haben, als der Sohn des
                  Bildhauers Constantin Meunier\pwindex{Meunier, Constantin 1831-04-12 – 1905-04-04@\textsc{Meunier, Constantin} (1831-04-12 – 1905-04-04), \emph{Maler/Malerin, Bildhauer/Bildhauerin}|pwk} gewesen zu
                  sein. Auf diesen Weg aus dem Schatten des Vaters dürfte Goldmann\pwindex{Goldmann, Paul 31.01.1865 – 25.09.1935@\textsc{Goldmann, Paul} (31.01.1865 – 25.09.1935), \emph{Schriftsteller/Schriftstellerin, Journalist/Journalistin}|pwk}{ }Schnitzler aufmerksam gemacht machen.}}}\label{K_L02615-1}
               geleſen? Da habe ich auch viel für Dich geſchrieben. Wenn es Dir entgangen iſt, ſo
               will ichs Dir ſchicken.\pend
           
\pstart
           Du biſt aber auch der Einzige, der Antheil an meinem Schaffen nimmt. Sonſt verhallts
               in der Wüſte. Ich ſehe immer mehr, daß nichts aus mir wird.\pend
           
\pstart
           Gern hätte ich mich mit Dir getroffen. Seit unſerm \label{K_L02615-2v}\edtext{letzten Beiſammenſein}{\lemma{\textnormal{\emph{letzten Beiſammenſein}}}\Cendnote{\textnormal{am 14. 11. 1894}}}\label{K_L02615-2} denke ich fortwährend daran und mache allerhand Pläne. {\pb}Aber es iſt ein furchtbarer Strich durch die
               Rechnung gekommen. Ich werde immer kränker. Der aufreibende Beruf vergrößert das
               Übel, das ſtetig um ſich greift. Ich fürchte, ich werde nicht mehr lange die Feder
               führen können. Jedenfalls verlangt mein Schwager\pwindex{Rosengart, Josef 1860-02-08 – 1927-08-04@\textsc{Rosengart, Josef} (1860-02-08 – 1927-08-04), \emph{Arzt/Ärztin}|pwv}, daß ich meinen Urlaub in Frankfurt\oindex{Frankfurt am Main@\textbf{Frankfurt am Main}, \emph{P.PPLA3}|pw} verbringe, damit er mich \label{K_L02615-3v}\edtext{behandeln}{\lemma{\textnormal{\emph{behandeln}}}\Cendnote{\textnormal{Josef Rosengart\pwindex{Rosengart, Josef 1860-02-08 – 1927-08-04@\textsc{Rosengart, Josef} (1860-02-08 – 1927-08-04), \emph{Arzt/Ärztin}|pwk}, der Ehemann\pwindex{Rosengart, Josef 1860-02-08 – 1927-08-04@\textsc{Rosengart, Josef} (1860-02-08 – 1927-08-04), \emph{Arzt/Ärztin}|pwkv} von Goldmanns\pwindex{Goldmann, Paul 31.01.1865 – 25.09.1935@\textsc{Goldmann, Paul} (31.01.1865 – 25.09.1935), \emph{Schriftsteller/Schriftstellerin, Journalist/Journalistin}|pwk} Schwester Vally\pwindex{Rosengart, Vally 1866-12-29 – nach 1926@\textsc{Rosengart, Vally} (1866-12-29 – nach 1926)|pwk}, war Arzt.}}}\label{K_L02615-3} könne.\pend
           
\pstart
           \textsc{Albert\pwindex{Albert, Henri 1869-11-16 – 1921-08-03@\textsc{Albert, Henri} (1869-11-16 – 1921-08-03), \emph{Journalist/Journalistin, Kritiker/Kritikerin, Übersetzer/Übersetzerin}|pw}} will natürlich keinen \label{K_L02615-4v}\edtext{Preis
                  beſtimmen}{\lemma{\textnormal{\emph{Preis
                  beſtimmen}}}\Cendnote{\textnormal{Siehe Paul Goldmann an Arthur Schnitzler, 21. 3. [1894].
               }}}\label{K_L02615-4}. Das mittlere Überſetzungs-{\pb}Honorar für ein\substVorne{}\textsuperscript{\textcolor{gray}{s}}\substDazwischen{}en\substHinten{} Deiner kleinen Dialoge\pwindex{Weihnachts-Einkaeufe@\emph{Weihnachts-Einkäufe}|pwv} wären 25 bis 30 \textsc{Francs}. Wäre
               Dir das zu viel? Schreib’ ganz offen, ich richte die Sache ſchon ein, wie es für Dich
               am Beſten iſt.\pend
           
\pstart
           \textsc{Herzl\pwindex{Herzl, Theodor 1860-05-02 – 1904-07-03@\textsc{Herzl, Theodor} (1860-05-02 – 1904-07-03), \emph{Schriftsteller/Schriftstellerin, Journalist/Journalistin}|pw}} hat ſich ſehr mit Deiner \label{K_L02615-5v}\edtext{Anerkennung}{\lemma{\textnormal{\emph{Anerkennung}}}\Cendnote{\textnormal{Nicht ermittelt.
                  Eventuell bezog sich Goldmann\pwindex{Goldmann, Paul 31.01.1865 – 25.09.1935@\textsc{Goldmann, Paul} (31.01.1865 – 25.09.1935), \emph{Schriftsteller/Schriftstellerin, Journalist/Journalistin}|pwk} auf die Rezension\pwindex{Le nouvel almanach de M. Bierbaum@\emph{Le nouvel almanach de M. Bierbaum}|pwkv} des \emph{Modernen Musen-Almanachs für das Jahr 1894}\pwindex{Moderner Musen-Almanach auf das Jahr 1894. Ein Jahrbuch deutscher Kunst@\emph{Moderner Musen-Almanach auf das Jahr 1894. Ein Jahrbuch deutscher Kunst}|pwk}
                  durch Henri Albert\pwindex{Albert, Henri 1869-11-16 – 1921-08-03@\textsc{Albert, Henri} (1869-11-16 – 1921-08-03), \emph{Journalist/Journalistin, Kritiker/Kritikerin, Übersetzer/Übersetzerin}|pwk}, doch diese war bereits 
                   im März erschienen. In der Korrespondenz zwischen
                     Schnitzler und Herzl\pwindex{Herzl, Theodor 1860-05-02 – 1904-07-03@\textsc{Herzl, Theodor} (1860-05-02 – 1904-07-03), \emph{Schriftsteller/Schriftstellerin, Journalist/Journalistin}|pwk} findet sich in dieser Zeit nichts, was näheren
                  Aufschluss gibt.}}}\label{K_L02615-5} gefreut. Ich glaube, Du wirſt nächſtens etwas wahrhaft
                  \label{K_L02615-6v}\edtext{Schönes}{\lemma{\textnormal{\emph{Schönes}}}\Cendnote{\textnormal{Eventuell meint er den Einakter \emph{Die
                     Glosse}\pwindex{Glosse. Lustspiel in einem Act@\emph{Die Glosse. Lustspiel in einem Act}|pwk}, vgl. A. S.: \emph{Tagebuch}, 31. 8. 1894.
               }}}\label{K_L02615-6} von ihm zu genießen bekommen, darf aber nicht reden.\pend
           
\pstart
           Herzlichſt und in Treue {\\[\baselineskip]}Dein {\\[\baselineskip]}\spacefill\mbox{Paul Goldmann}\pend
           \leftskip=0em{}
\pstart
           \noindent{}Was haſt Du Oſtern gemacht?\pend
           \selectlanguage{ngerman}\endnumbering\briefempfaengerindex{Schnitzler, Arthur@\textsc{Schnitzler, Arthur}!zzzGoldmann, Paul@\emph{von Paul Goldmann}!1894-04-031@{3. 4. {[}1894{]}}|)be}\mylabel{L02615h}  \normalsize

\doendnotes{C}
\bigskip
\vfill

\clearpage

\footnotesize

\lohead{\textsc{register}}

% Definiere theindex-Environment komplett neu ohne reledmac
\makeatletter
\renewenvironment{theindex}{%
  \section*{\indexname}%
  \setlength{\parindent}{0pt}%
  \setlength{\parskip}{0pt plus 0.3pt}%
  \let\item\@idxitem
}{%
  \clearpage
}
\makeatother

\IfFileExists{\jobname-pw.ind}{\input{\jobname-pw.ind}}{}

\end{document}

      