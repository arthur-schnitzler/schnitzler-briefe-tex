%% latex-leseansicht-vorspann.tex
%% Vorspann für die Leseansicht.
%% Lädt die gemeinsame Datei latex-vorspann.tex mit nicht gesetztem Schalter.

\newif\ifkorrekturansicht
\korrekturansichtfalse

\input{../tex-inputs/latex-vorspann}


\section[Paul Goldmann an Arthur Schnitzler, 3. 4. {[}1894{]}]{L02615 Paul Goldmann an Arthur Schnitzler, 3. 4. [1894]}
\nopagebreak\mylabel{L02615v}
\rehead{ }\normalsize\beginnumbering\briefempfaengerindex{Schnitzler, Arthur@\textsc{Schnitzler, Arthur}!zzzGoldmann, Paul@\emph{von Paul Goldmann}!1894-04-031@{3. 4. [1894]}|(be}
\toendnotes[C]{\smallbreak\pagebreak[2]}
\correspDesc{Versand  durch Paul Goldmann am 3. 4. [1894] in Paris
\newline{}Erhalt  durch Arthur Schnitzler im Zeitraum [4. 4. 1894
                  – 8. 4. 1894?] in Wien}\toendnotes[C]{\smallbreak}
\Standort{DLA, A:Schnitzler, HS.NZ85.1.3164.}
\physDesc{Brief, 1 Blatt, 4 Seiten, 1535 Zeichen
\newline{}Handschrift: schwarze Tinte, deutsche Kurrent
\newline{}Schnitzler: 1) mit Bleistift auf dem ersten Blatt die Jahreszahl »94« vermerkt  2) mit rotem Buntstift zwei Unterstreichungen}\toendnotes[C]{\smallbreak}
\pstart
           \raggedleft{}{\pb}\textsc{Paris\oindex{Paris@\textbf{Paris}, \emph{Hauptstadt}|pw}}, 3. April.\pend
           
\pstart\center{}Mein lieber Freund,\pend\vspace{0.5em}
\pstart
           Ich habe Dir für zwei liebe Briefe zu danken\substVorne{}\textsuperscript{,}\substDazwischen{},\substHinten{} und ich muß Dir immer und immer wiederholen, wie wohl mir Deine treue
               Freundſchaft thut und Deine Antheilnahme an Allem, was ich leiſte. Es gibt mir beim
               Arbeiten eine gewiſſe Anregung, wenn ich daran denke, daß ich Dein Lob verdienen muß.
               Haſt Du mein \label{K_L02615-1v}\edtext{Feuilleton\pwindex{Goldmann, Paul 31.\,1.\,1865 Breslau – 25.\,9.\,1935 Wien@\textsc{Goldmann, Paul} (31.\,1.\,1865 Breslau – 25.\,9.\,1935 Wien), \emph{Schriftsteller, Journalist}!Charles Meunier@\strich\emph{Charles Meunier}|pwv}{ }{\pb}über den armen \textsc{Charles Meunier\pwindex{Meunier, Karl 1864 – 22.\,3.\,1894 Leuven@\textsc{Meunier, Karl} (1864 – 22.\,3.\,1894 Leuven), \emph{Maler, Kupferstecher}|pw}}}{\lemma{\textnormal{\emph{Feuilleton … Meunier}}}\Cendnote{\textnormal{Paul Goldmann\pwindex{Goldmann, Paul 31.\,1.\,1865 Breslau – 25.\,9.\,1935 Wien@\textsc{Goldmann, Paul} (31.\,1.\,1865 Breslau – 25.\,9.\,1935 Wien), \emph{Schriftsteller, Journalist}|pwk}: \emph{Charles Meunier. Ein Jugendleben}\pwindex{Goldmann, Paul 31.\,1.\,1865 Breslau – 25.\,9.\,1935 Wien@\textsc{Goldmann, Paul} (31.\,1.\,1865 Breslau – 25.\,9.\,1935 Wien), \emph{Schriftsteller, Journalist}!Charles Meunier@\strich\emph{Charles Meunier}|pwk}. In: \emph{Frankfurter Zeitung}\pwindex{Frankfurter Zeitung@\emph{Frankfurter Zeitung}|pwk}, Jg. 38, Nr. 90,
                        1. 4. 1894, Erstes Morgenblatt, S. 1–2. Das »arm«
                  bezieht sich darauf, dass der Grafiker und Maler Karl Meunier\pwindex{Meunier, Karl 1864 – 22.\,3.\,1894 Leuven@\textsc{Meunier, Karl} (1864 – 22.\,3.\,1894 Leuven), \emph{Maler, Kupferstecher}|pwk} als Nachwuchstalent im Alter von 30 Jahren gestorben war, ohne sich
                  in weiteren Kreisen einen anderen Namen gemacht zu haben, als der Sohn des
                  Bildhauers Constantin Meunier\pwindex{Meunier, Constantin 12.\,4.\,1831 Etterbeek – 4.\,4.\,1905 Ixelles@\textsc{Meunier, Constantin} (12.\,4.\,1831 Etterbeek – 4.\,4.\,1905 Ixelles), \emph{Maler, Bildhauer}|pwk} gewesen zu
                  sein. Auf diesen Weg aus dem Schatten des Vaters dürfte Goldmann\pwindex{Goldmann, Paul 31.\,1.\,1865 Breslau – 25.\,9.\,1935 Wien@\textsc{Goldmann, Paul} (31.\,1.\,1865 Breslau – 25.\,9.\,1935 Wien), \emph{Schriftsteller, Journalist}|pwk}{ }Schnitzler aufmerksam gemacht machen.}}}\label{K_L02615-1}
               geleſen? Da habe ich auch viel für Dich geſchrieben. Wenn es Dir entgangen iſt,{ }ſo
               will ichs Dir{ }ſchicken.\pend
           
\pstart
           Du biſt aber auch der Einzige, der Antheil an meinem Schaffen nimmt. Sonſt verhallts
               in der Wüſte. Ich{ }ſehe immer mehr, daß nichts aus mir wird.\pend
           
\pstart
           Gern hätte ich mich mit Dir getroffen. Seit unſerm \label{K_L02615-2v}\edtext{letzten Beiſammenſein}{\lemma{\textnormal{\emph{letzten Beisammensein}}}\Cendnote{\textnormal{am 14. 11. 1894}}}\label{K_L02615-2} denke ich fortwährend daran und mache allerhand Pläne. {\pb}Aber es iſt ein furchtbarer Strich durch die
               Rechnung gekommen. Ich werde immer kränker. Der aufreibende Beruf vergrößert das
               Übel, das{ }ſtetig um{ }ſich greift. Ich fürchte, ich werde nicht mehr lange die Feder
               führen können. Jedenfalls verlangt mein Schwager\pwindex{Rosengart, Josef 8.\,2.\,1860 Laupheim – 4.\,8.\,1927 Frankfurt am Main@\textsc{Rosengart, Josef} (8.\,2.\,1860 Laupheim – 4.\,8.\,1927 Frankfurt am Main), \emph{Arzt}|pwv}, daß ich meinen Urlaub in Frankfurt\oindex{Frankfurt am Main@\textbf{Frankfurt am Main}, \emph{Hauptstadt}|pw} verbringe, damit er mich \label{K_L02615-3v}\edtext{behandeln}{\lemma{\textnormal{\emph{behandeln}}}\Cendnote{\textnormal{Josef Rosengart\pwindex{Rosengart, Josef 8.\,2.\,1860 Laupheim – 4.\,8.\,1927 Frankfurt am Main@\textsc{Rosengart, Josef} (8.\,2.\,1860 Laupheim – 4.\,8.\,1927 Frankfurt am Main), \emph{Arzt}|pwk}, der Ehemann\pwindex{Rosengart, Josef 8.\,2.\,1860 Laupheim – 4.\,8.\,1927 Frankfurt am Main@\textsc{Rosengart, Josef} (8.\,2.\,1860 Laupheim – 4.\,8.\,1927 Frankfurt am Main), \emph{Arzt}|pwkv} von Goldmanns\pwindex{Goldmann, Paul 31.\,1.\,1865 Breslau – 25.\,9.\,1935 Wien@\textsc{Goldmann, Paul} (31.\,1.\,1865 Breslau – 25.\,9.\,1935 Wien), \emph{Schriftsteller, Journalist}|pwk} Schwester Vally\pwindex{Rosengart, Vally 29.\,12.\,1866 Breslau – nach 1926@\textsc{Rosengart, Vally} (29.\,12.\,1866 Breslau – nach 1926)|pwk}, war Arzt.}}}\label{K_L02615-3} könne.\pend
           
\pstart
           \textsc{Albert\pwindex{Albert, Henri 16.\,11.\,1869 Niederbronn-les-Bains – 3.\,8.\,1921 Straßburg@\textsc{Albert, Henri} (16.\,11.\,1869 Niederbronn-les-Bains – 3.\,8.\,1921 Straßburg), \emph{Journalist, Kritiker, Übersetzer}|pw}} will natürlich keinen \label{K_L02615-4v}\edtext{Preis
                  beſtimmen}{\lemma{\textnormal{\emph{Preis
                  bestimmen}}}\Cendnote{\textnormal{Siehe XXXX Auszeichnungsfehler: Dokument L02613 nicht gefunden.
               }}}\label{K_L02615-4}. Das mittlere Überſetzungs-{\pb}Honorar für ein\substVorne{}\textsuperscript{\textcolor{gray}{s}}\substDazwischen{}en\substHinten{} Deiner kleinen Dialoge\pwindex{Schnitzler, Arthur 15.\,5.\,1862 Wien – 21.\,10.\,1931 ebd.@\textsc{Schnitzler, Arthur} (15.\,5.\,1862 Wien – 21.\,10.\,1931 ebd.), \emph{Schriftsteller, Mediziner}!Weihnachts-Einkäufe@\strich\emph{Weihnachts-Einkäufe}|pwv} wären 25 bis 30 \textsc{Francs}. Wäre
               Dir das zu viel? Schreib’ ganz offen, ich richte die Sache{ }ſchon ein, wie es für Dich
               am Beſten iſt.\pend
           
\pstart
           \textsc{Herzl\pwindex{Herzl, Theodor 2.\,5.\,1860 Budapest – 3.\,7.\,1904 Edlach@\textsc{Herzl, Theodor} (2.\,5.\,1860 Budapest – 3.\,7.\,1904 Edlach), \emph{Schriftsteller, Journalist}|pw}} hat{ }ſich{ }ſehr mit Deiner \label{K_L02615-5v}\edtext{Anerkennung}{\lemma{\textnormal{\emph{Anerkennung}}}\Cendnote{\textnormal{Nicht ermittelt.
                  Eventuell bezog sich Goldmann\pwindex{Goldmann, Paul 31.\,1.\,1865 Breslau – 25.\,9.\,1935 Wien@\textsc{Goldmann, Paul} (31.\,1.\,1865 Breslau – 25.\,9.\,1935 Wien), \emph{Schriftsteller, Journalist}|pwk} auf die Rezension\pwindex{Albert, Henri 16.\,11.\,1869 Niederbronn-les-Bains – 3.\,8.\,1921 Straßburg@\textsc{Albert, Henri} (16.\,11.\,1869 Niederbronn-les-Bains – 3.\,8.\,1921 Straßburg), \emph{Journalist, Kritiker, Übersetzer}!Le nouvel almanach de M. Bierbaum@\strich\emph{Le nouvel almanach de M. Bierbaum}|pwkv} des \emph{Modernen Musen-Almanachs für das Jahr 1894}\pwindex{Moderner Musen-Almanach auf das Jahr 1894. Ein Jahrbuch deutscher Kunst@\emph{Moderner Musen-Almanach auf das Jahr 1894. Ein Jahrbuch deutscher Kunst}|pwk}
                  durch Henri Albert\pwindex{Albert, Henri 16.\,11.\,1869 Niederbronn-les-Bains – 3.\,8.\,1921 Straßburg@\textsc{Albert, Henri} (16.\,11.\,1869 Niederbronn-les-Bains – 3.\,8.\,1921 Straßburg), \emph{Journalist, Kritiker, Übersetzer}|pwk}, doch diese war bereits 
                   im März erschienen. In der Korrespondenz zwischen
                     Schnitzler und Herzl\pwindex{Herzl, Theodor 2.\,5.\,1860 Budapest – 3.\,7.\,1904 Edlach@\textsc{Herzl, Theodor} (2.\,5.\,1860 Budapest – 3.\,7.\,1904 Edlach), \emph{Schriftsteller, Journalist}|pwk} findet sich in dieser Zeit nichts, was näheren
                  Aufschluss gibt.}}}\label{K_L02615-5} gefreut. Ich glaube, Du wirſt nächſtens etwas wahrhaft
                  \label{K_L02615-6v}\edtext{Schönes}{\lemma{\textnormal{\emph{Schönes}}}\Cendnote{\textnormal{Eventuell meint er den Einakter \emph{Die
                     Glosse}\pwindex{Herzl, Theodor 2.\,5.\,1860 Budapest – 3.\,7.\,1904 Edlach@\textsc{Herzl, Theodor} (2.\,5.\,1860 Budapest – 3.\,7.\,1904 Edlach), \emph{Schriftsteller, Journalist}!Glosse. Lustspiel in einem Act@\strich\emph{Die Glosse. Lustspiel in einem Act}|pwk}, vgl. A. S.: \emph{Tagebuch}, 31. 8. 1894.
               }}}\label{K_L02615-6} von ihm zu genießen bekommen, darf aber nicht reden.\pend
           
\pstart
           Herzlichſt und in Treue {\\[\baselineskip]}Dein {\\[\baselineskip]}\spacefill\mbox{Paul Goldmann}\pend
           \leftskip=0em{}
\pstart
           \noindent{}Was haſt Du Oſtern gemacht?\pend
           \selectlanguage{ngerman}\endnumbering\briefempfaengerindex{Schnitzler, Arthur@\textsc{Schnitzler, Arthur}!zzzGoldmann, Paul@\emph{von Paul Goldmann}!1894-04-031@{3. 4. [1894]}|)be}\mylabel{L02615h}  \newcommand{\dateiname}{L02615}\newcommand{\titel}{Paul Goldmann an Arthur Schnitzler, 3. 4. [1894]}\newcommand{\editorInnen}{Martin Anton Müller und Laura Untner}%% latex-leseansicht-abspann.tex
%% Abspann für die Leseansicht.
%% Der Schalter \ifkorrekturansicht ist bereits durch den Vorspann gesetzt.

%% latex-abspann.tex
%% Gemeinsamer Abspann für Korrekturansicht und Leseansicht.
%% Setzt den Schalter \ifkorrekturansicht voraus (gesetzt in den
%% einbindenden Dateien latex-korrekturansicht-abspann.tex bzw.
%% latex-leseansicht-abspann.tex).
%% ---------------------------------------------------------------

\normalsize

% Das esempio-Environment wird nur in der Leseansicht benötigt
\ifkorrekturansicht\else
\newenvironment{esempio}[3]%
{
    \vspace{1.5ex}
    \rlap{\underline{#1}}
    \par
    \setlength{\parindent}{0cm}
    \nopagebreak
    \leftskip=#2cm
    \rightskip=#3cm
}
{
    \par
}
\fi

\doendnotes{C}
\bigskip
\vfill

\clearpage

\footnotesize

\ifkorrekturansicht
  \lohead{\textsc{register}}
\fi

% theindex-Environment neu definieren ohne reledmac
\makeatletter
\renewenvironment{theindex}{%
  \ifkorrekturansicht
    \section*{\indexname}%
  \else
    \subsubsection*{Index der erwähnten Entitäten}%
  \fi
  \setlength{\parindent}{0pt}%
  \setlength{\parskip}{0pt plus 0.3pt}%
  \let\item\@idxitem
}{%
  \ifkorrekturansicht\clearpage\fi
}
\makeatother

\IfFileExists{\jobname-pw.ind}{\input{\jobname-pw.ind}}{}

% Quellenangabe nur in der Leseansicht
\ifkorrekturansicht\else
% Fallback-Definitionen, falls die .tex-Datei \titel etc. nicht gesetzt hat
\providecommand{\titel}{}
\providecommand{\editorInnen}{}
\providecommand{\dateiname}{\jobname}

\vspace{3cm}

\vfill

\footnotesize
\textsc{Quelle}: \titel. Herausgegeben von {\editorInnen}. In: \emph{Arthur Schnitzler: Briefwechsel mit Autorinnen und Autoren}.
 Digitale Edition, https://schnitzler-briefe.acdh.oeaw.ac.at/{\dateiname}.html (Stand \today)
\fi

\end{document}


