%% latex-leseansicht-vorspann.tex
%% Vorspann für die Leseansicht.
%% Lädt die gemeinsame Datei latex-vorspann.tex mit nicht gesetztem Schalter.

\newif\ifkorrekturansicht
\korrekturansichtfalse

\input{../tex-inputs/latex-vorspann}


\section[ Paul Goldmann an Arthur Schnitzler, 16. 1. [1896]]{L02764 Paul Goldmann an Arthur Schnitzler,  16. 1. [1896]}
\nopagebreak\mylabel{L02764v}
\rehead{ }\normalsize\beginnumbering\briefempfaengerindex{Schnitzler, Arthur@\textsc{Schnitzler, Arthur}!zzzGoldmann, Paul@\emph{von Paul Goldmann}!1896-01-162@{16. 1. [1896]}|(be}
\toendnotes[C]{\smallbreak\pagebreak[2]}
\correspDesc{Versand  durch Paul Goldmann am 16. 1. [1896] in Paris
\newline{}Erhalt  durch Arthur Schnitzler im Zeitraum [17. 1. 1896
                  – 21. 1. 1896?] in Wien}\toendnotes[C]{\smallbreak}
\Standort{DLA, A:Schnitzler, HS.NZ85.1.3166.}
\physDesc{Brief, 1 Blatt, 3 Seiten, 1388 Zeichen
\newline{}Handschrift: blaue Tinte, deutsche Kurrent
\newline{}Beilage: handschriftlicher Brief: 1 Blatt, 1 Seite, schwarze Tinte,
                                 lateinische Kurrent 
\newline{}Schnitzler: 1) mit Bleistift das Jahr »96« vermerkt  2) mit rotem Buntstift eine Unterstreichung}\toendnotes[C]{\smallbreak}
\pstart
           {\pb}\textcolor{gray}{\textbf{\textbf{Frankfurter Zeitung\orgindex{Frankfurter Zeitung@Frankfurter Zeitung|pw}}}}\pend
           
\pstart
           \textcolor{gray}{\textbf{(\begin{otherlanguage}{french}Gazette de Francfort\end{otherlanguage}\orgindex{Frankfurter Zeitung@Frankfurter Zeitung|pw}).}}\pend
           
\pstart
           \textcolor{gray}{\textbf{\textbf{\begin{otherlanguage}{french}Fondateur M.\end{otherlanguage}{ }L. Sonnemann\pwindex{Sonnemann, Leopold 29.\,10.\,1831 Höchberg – 30.\,10.\,1909 Frankfurt am Main@\textsc{Sonnemann, Leopold} (29.\,10.\,1831 Höchberg – 30.\,10.\,1909 Frankfurt am Main), \emph{Journalist, Herausgeber}|pw}.}}}\pend
           
\pstart
           \begin{otherlanguage}{french}\textcolor{gray}{\textbf{Journal\pwindex{Frankfurter Zeitung@\emph{Frankfurter Zeitung}|pwv} politique,
                        financier,}}\end{otherlanguage}\pend
           
\pstart
           \begin{otherlanguage}{french}\textcolor{gray}{\textbf{commercial et littéraire.}}\end{otherlanguage}\pend
           
\pstart
           \begin{otherlanguage}{french}\textcolor{gray}{\textbf{\textbf{Paraissant trois fois par jour.}}}\end{otherlanguage}\pend
           
\pstart
           \begin{otherlanguage}{french}\textcolor{gray}{\textbf{\textbf{Bureau à Paris\oindex{Paris@\textbf{Paris}, \emph{Hauptstadt}|pw}:}}}\end{otherlanguage}\hfill \textsc{Paris\oindex{Paris@\textbf{Paris}, \emph{Hauptstadt}|pw}}, 16. Januar.\pend
           
\pstart
           \begin{otherlanguage}{french}\textcolor{gray}{\textbf{\textbf{24. Rue Feydeau\oindex{rue Feydeau@\textbf{rue Feydeau}, \emph{Straße}|pw}.}}}\end{otherlanguage}\pend
           
\pstart\center{}Mein lieber Freund,\pend\vspace{0.5em}
\pstart
           Ich hatte \textsc{Thorel\pwindex{Thorel, Jean 11.\,9.\,1859 Éragny – 20.\,8.\,1916 Enghien-les-Bains@\textsc{Thorel, Jean} (11.\,9.\,1859 Éragny – 20.\,8.\,1916 Enghien-les-Bains), \emph{Übersetzer, Dramatiker}|pw}} die Frankf. Zeit.\pwindex{Frankfurter Zeitung@\emph{Frankfurter Zeitung}|pw} mit dem \label{K_L02764-1v}\edtext{Referat\pwindex{Mamroth, Fedor 21.\,2.\,1851 Breslau – 25.\,6.\,1907 Frankfurt am Main@\textsc{Mamroth, Fedor} (21.\,2.\,1851 Breslau – 25.\,6.\,1907 Frankfurt am Main), \emph{Journalist, Kritiker}!Schauspielhaus. [Premiere von Liebelei]@\strich\emph{Schauspielhaus. [Premiere von Liebelei]}|pwv}}{\lemma{\textnormal{\emph{Referat}}}\Cendnote{\textnormal{m.\pwindex{Mamroth, Fedor 21.\,2.\,1851 Breslau – 25.\,6.\,1907 Frankfurt am Main@\textsc{Mamroth, Fedor} (21.\,2.\,1851 Breslau – 25.\,6.\,1907 Frankfurt am Main), \emph{Journalist, Kritiker}|pwkv} [ = Fedor Mamroth\pwindex{Mamroth, Fedor 21.\,2.\,1851 Breslau – 25.\,6.\,1907 Frankfurt am Main@\textsc{Mamroth, Fedor} (21.\,2.\,1851 Breslau – 25.\,6.\,1907 Frankfurt am Main), \emph{Journalist, Kritiker}|pwk}]: \emph{Schauspielhaus}\pwindex{Mamroth, Fedor 21.\,2.\,1851 Breslau – 25.\,6.\,1907 Frankfurt am Main@\textsc{Mamroth, Fedor} (21.\,2.\,1851 Breslau – 25.\,6.\,1907 Frankfurt am Main), \emph{Journalist, Kritiker}!Schauspielhaus. [Premiere von Liebelei]@\strich\emph{Schauspielhaus. [Premiere von Liebelei]}|pwk}. In: \emph{Frankfurter Zeitung}\pwindex{Frankfurter Zeitung@\emph{Frankfurter Zeitung}|pwk}, Jg. 40, Nr. 12, 12. 1. 1896, Zweites Morgenblatt, S. 1.}}}\label{K_L02764-1} geſchickt, um
               ihn zur raſcheren Erledigung anzutreiben. Das hat auch gewirkt. Heut erhalte ich beifolgenden Brief. Das iſt der erſte
               kleine Erfolg Deines Stück\pwindex{Schnitzler, Arthur 15.\,5.\,1862 Wien – 21.\,10.\,1931 ebd.@\textsc{Schnitzler, Arthur} (15.\,5.\,1862 Wien – 21.\,10.\,1931 ebd.), \emph{Schriftsteller, Mediziner}!Liebelei. Schauspiel in drei Akten@\strich\emph{Liebelei. Schauspiel in drei Akten}|pwv}es
               in Frankreich\oindex{Frankreich@\textbf{Frankreich}|pw}; mögen größere nachkommen! \textsc{Carré\pwindex{Carré, Albert 22.\,6.\,1852 Straßburg – 11.\,12.\,1938 Paris@\textsc{Carré, Albert} (22.\,6.\,1852 Straßburg – 11.\,12.\,1938 Paris), \emph{Schriftsteller, Theaterleiter, Schauspieler}|pw}} und \textsc{Torel\pwindex{Thorel, Jean 11.\,9.\,1859 Éragny – 20.\,8.\,1916 Enghien-les-Bains@\textsc{Thorel, Jean} (11.\,9.\,1859 Éragny – 20.\,8.\,1916 Enghien-les-Bains), \emph{Übersetzer, Dramatiker}|pw}}{ }ſind die Directoren\pwindex{Carré, Albert 22.\,6.\,1852 Straßburg – 11.\,12.\,1938 Paris@\textsc{Carré, Albert} (22.\,6.\,1852 Straßburg – 11.\,12.\,1938 Paris), \emph{Schriftsteller, Theaterleiter, Schauspieler}|pwv}\pwindex{Thorel, Jean 11.\,9.\,1859 Éragny – 20.\,8.\,1916 Enghien-les-Bains@\textsc{Thorel, Jean} (11.\,9.\,1859 Éragny – 20.\,8.\,1916 Enghien-les-Bains), \emph{Übersetzer, Dramatiker}|pwv} des \textsc{Vaudeville\orgindex{Théâtre du Vaudeville@Théâtre du Vaudeville|pw}}. Es wäre herrlich, wenn an dieſem vornehmen Theater\orgindex{Théâtre du Vaudeville@Théâtre du Vaudeville|pwv}, wo die \textsc{Réjane\pwindex{Réjane 5.\,6.\,1856 Paris – 14.\,6.\,1920 Asnières-sur-Seine@\textsc{Réjane} (5.\,6.\,1856 Paris – 14.\,6.\,1920 Asnières-sur-Seine), \emph{Schauspielerin}|pw}} die Hauperſon iſt, etwas {\pb}zu machen wäre. Ich
               möchte gern \strikeout{über} die freien Bühnen (\textsc{Œuvre\orgindex{Théâtre de l’Œuvre@Théâtre de l’Œuvre|pw}}, \textsc{Théâtre Libre\orgindex{Théâtre Libre@Théâtre Libre|pw}}) mit ihren Miſt-Aufführungen umgehen. Jedenfalls{ }ſchließe einſtweilen \uline{keinerlei} Überſetzungs-Engagement ab. Könnte ich nicht
               ein paar Exemplare des Stück\pwindex{Schnitzler, Arthur 15.\,5.\,1862 Wien – 21.\,10.\,1931 ebd.@\textsc{Schnitzler, Arthur} (15.\,5.\,1862 Wien – 21.\,10.\,1931 ebd.), \emph{Schriftsteller, Mediziner}!Liebelei. Schauspiel in drei Akten@\strich\emph{Liebelei. Schauspiel in drei Akten}|pwv}es
               haben?\pend
           
\pstart
           Was in Frankfurt\oindex{Frankfurt am Main@\textbf{Frankfurt am Main}, \emph{Hauptstadt}|pw} vorgegangen iſt, weiß ich nicht.
               Meine Mutter\pwindex{Goldmann, Clementine 15.\,5.\,1842 Breslau – 24.\,2.\,1924 Frankfurt am Main@\textsc{Goldmann, Clementine} (15.\,5.\,1842 Breslau – 24.\,2.\,1924 Frankfurt am Main)|pwv}, die mir{ }ſonſt
               drei Mal die Woche{ }ſchreibt, um {\pb}mir mitzutheilen,
               wenn irgend Jemandem dort die Naſe weh thut, iſt mir jeden Bericht über Deine
               Anweſenheit{ }ſchuldig geblieben. Oh,{ }ſie können Einen nervös machen, die Herrſchaften
               von der Familie!\pend
           
\pstart
           Hoffentlich biſt Du geſund heimgekehrt.\pend
           
\pstart
           Grüß’ Dich Gott, mein lieber Freund!\pend
           
\pstart
           Dein treuer {\\[\baselineskip]}\spacefill\mbox{Paul Goldmnn}\pend
           \leftskip=0em{}\selectlanguage{ngerman}\vspace{1em}{\vspace{1\baselineskip}}
\pstart
           \raggedleft{}{\pb}{[}hs. Thorel:{]} 12 rue de Milan\oindex{Rue de Milan@\textbf{Rue de Milan}, \emph{Straße}|pw}\pend
           
\pstart{}\begin{otherlanguage}{french}Cher Monsieur Goldmann\end{otherlanguage}\pend\vspace{0.5em}
\pstart
           \label{K_L02764-2v}\edtext{\begin{otherlanguage}{french}Je viens – enfin – de lire »Liebelei\pwindex{Schnitzler, Arthur 15.\,5.\,1862 Wien – 21.\,10.\,1931 ebd.@\textsc{Schnitzler, Arthur} (15.\,5.\,1862 Wien – 21.\,10.\,1931 ebd.), \emph{Schriftsteller, Mediziner}!Liebelei. Schauspiel in drei Akten@\strich\emph{Liebelei. Schauspiel in drei Akten}|pw}«{[}.{]} C’est un pur bijou\strikeout{x}, d’une délicateſse, d’une fraîcheur, et d’une
                  harmonie parfaites. Il faudra absolument que nous reparlions de cela. Auſsitôt que
                  je vais avoir un instant, je vous demanderai rendez-vous.\end{otherlanguage}}{\lemma{\textnormal{\emph{Je … rendez-vous.}}}\Cendnote{\textnormal{französisch: Ich habe – endlich – die
                  Lektüre von \emph{Liebelei}\pwindex{Schnitzler, Arthur 15.\,5.\,1862 Wien – 21.\,10.\,1931 ebd.@\textsc{Schnitzler, Arthur} (15.\,5.\,1862 Wien – 21.\,10.\,1931 ebd.), \emph{Schriftsteller, Mediziner}!Liebelei. Schauspiel in drei Akten@\strich\emph{Liebelei. Schauspiel in drei Akten}|pwk} abgeschlossen. Es ist
                  ein reines Juwel, von perfekter Zartheit, Frische und Harmonie. Wir müssen
                  unbedingt einmal darüber sprechen. Sobald ich einen Moment Zeit habe, werde ich
                  Sie um ein Treffen bitten.}}}\label{K_L02764-2}\pend
           
\pstart
           \begin{otherlanguage}{french}Votre dévoué\end{otherlanguage}{\\[\baselineskip]}\spacefill\mbox{Jean Thorel\pwindex{Thorel, Jean 11.\,9.\,1859 Éragny – 20.\,8.\,1916 Enghien-les-Bains@\textsc{Thorel, Jean} (11.\,9.\,1859 Éragny – 20.\,8.\,1916 Enghien-les-Bains), \emph{Übersetzer, Dramatiker}|pw}}\pend
           \leftskip=0em{}
\pstart
           \noindent{}\label{K_L02764-3v}\edtext{J’écris dès aujourd’hui à Thorel\pwindex{Thorel, Jean 11.\,9.\,1859 Éragny – 20.\,8.\,1916 Enghien-les-Bains@\textsc{Thorel, Jean} (11.\,9.\,1859 Éragny – 20.\,8.\,1916 Enghien-les-Bains), \emph{Übersetzer, Dramatiker}|pw} et Carré\pwindex{Carré, Albert 22.\,6.\,1852 Straßburg – 11.\,12.\,1938 Paris@\textsc{Carré, Albert} (22.\,6.\,1852 Straßburg – 11.\,12.\,1938 Paris), \emph{Schriftsteller, Theaterleiter, Schauspieler}|pw}!}{\lemma{\textnormal{\emph{J’écris … Carré!}}}\Cendnote{\textnormal{französisch: Ich schreibe schon heute an Thorel\pwindex{Thorel, Jean 11.\,9.\,1859 Éragny – 20.\,8.\,1916 Enghien-les-Bains@\textsc{Thorel, Jean} (11.\,9.\,1859 Éragny – 20.\,8.\,1916 Enghien-les-Bains), \emph{Übersetzer, Dramatiker}|pwk} und Carré\pwindex{Carré, Albert 22.\,6.\,1852 Straßburg – 11.\,12.\,1938 Paris@\textsc{Carré, Albert} (22.\,6.\,1852 Straßburg – 11.\,12.\,1938 Paris), \emph{Schriftsteller, Theaterleiter, Schauspieler}|pwk}!}}}\label{K_L02764-3}\pend
           \selectlanguage{ngerman}\endnumbering\briefempfaengerindex{Schnitzler, Arthur@\textsc{Schnitzler, Arthur}!zzzGoldmann, Paul@\emph{von Paul Goldmann}!1896-01-162@{16. 1. [1896]}|)be}\mylabel{L02764h}  \newcommand{\dateiname}{L02764}\newcommand{\titel}{Paul Goldmann an Arthur Schnitzler, 16. 1. [1896]}\newcommand{\editorInnen}{Martin Anton Müller und Laura Untner}%% latex-leseansicht-abspann.tex
%% Abspann für die Leseansicht.
%% Der Schalter \ifkorrekturansicht ist bereits durch den Vorspann gesetzt.

%% latex-abspann.tex
%% Gemeinsamer Abspann für Korrekturansicht und Leseansicht.
%% Setzt den Schalter \ifkorrekturansicht voraus (gesetzt in den
%% einbindenden Dateien latex-korrekturansicht-abspann.tex bzw.
%% latex-leseansicht-abspann.tex).
%% ---------------------------------------------------------------

\normalsize

% Das esempio-Environment wird nur in der Leseansicht benötigt
\ifkorrekturansicht\else
\newenvironment{esempio}[3]%
{
    \vspace{1.5ex}
    \rlap{\underline{#1}}
    \par
    \setlength{\parindent}{0cm}
    \nopagebreak
    \leftskip=#2cm
    \rightskip=#3cm
}
{
    \par
}
\fi

\doendnotes{C}
\bigskip
\vfill

\clearpage

\footnotesize

\ifkorrekturansicht
  \lohead{\textsc{register}}
\fi

% theindex-Environment neu definieren ohne reledmac
\makeatletter
\renewenvironment{theindex}{%
  \ifkorrekturansicht
    \section*{\indexname}%
  \else
    \subsubsection*{Index der erwähnten Entitäten}%
  \fi
  \setlength{\parindent}{0pt}%
  \setlength{\parskip}{0pt plus 0.3pt}%
  \let\item\@idxitem
}{%
  \ifkorrekturansicht\clearpage\fi
}
\makeatother

\IfFileExists{\jobname-pw.ind}{\input{\jobname-pw.ind}}{}

% Quellenangabe nur in der Leseansicht
\ifkorrekturansicht\else
% Fallback-Definitionen, falls die .tex-Datei \titel etc. nicht gesetzt hat
\providecommand{\titel}{}
\providecommand{\editorInnen}{}
\providecommand{\dateiname}{\jobname}

\vspace{3cm}

\vfill

\footnotesize
\textsc{Quelle}: \titel. Herausgegeben von {\editorInnen}. In: \emph{Arthur Schnitzler: Briefwechsel mit Autorinnen und Autoren}.
 Digitale Edition, https://schnitzler-briefe.acdh.oeaw.ac.at/{\dateiname}.html (Stand \today)
\fi

\end{document}


