%% latex-leseansicht-vorspann.tex
%% Vorspann für die Leseansicht.
%% Lädt die gemeinsame Datei latex-vorspann.tex mit nicht gesetztem Schalter.

\newif\ifkorrekturansicht
\korrekturansichtfalse

\input{../tex-inputs/latex-vorspann}

\begin{center}
            \textcolor{red}{ENTWURF, NICHT FERTIG KORRIGIERT}
                      \end{center}
            
         
         \renewcommand{\erwaehntePersonen}{Personen: Albert Carré, Clementine Goldmann, Fedor Mamroth,  Réjane, Leopold Sonnemann, Jean Thorel}
         \renewcommand{\erwaehnteInstitutionen}{Institutionen: Frankfurter Zeitung, Théâtre Libre, Théâtre de l’Œuvre, Théâtre du Vaudeville}
         \renewcommand{\erwaehnteOrte}{Orte: Frankfurt am Main, Frankreich, Paris, Rue de Milan, Wien, rue Feydeau}
         \renewcommand{\erwaehnteWerke}{Werke: Frankfurter Zeitung, Liebelei. Schauspiel in drei Akten, Schauspielhaus. [Premiere von Liebelei]}
               \section[ Paul Goldmann an Arthur Schnitzler, 16. 1. {[}1896{]}]{ Paul Goldmann an Arthur Schnitzler, 16. 1. {[}1896{]}}\nopagebreak\mylabel{v}\rehead{ }\begin{ledgroupsized}[t]{13cm}\normalsize\beginnumbering \toendnotes[C]{\smallbreak\pagebreak[2]} \Standort{DLA, A:Schnitzler, HS.NZ85.1.3166.}
\physDesc{Brief, 1 Blatt, 3 Seiten
\newline{}Handschrift: blaue Tinte, deutsche Kurrent\newline{}Beilage: handschriftlicher Brief: 1 Blatt, 1 Seite, schwarze Tinte,
                                 lateinische Kurrent 
\newline{}Schnitzler: 1) mit Bleistift das Jahr »96« vermerkt  2) mit rotem Buntstift eine Unterstreichung}\toendnotes[C]{\smallbreak}\pstart
           \noindent{}{\pb}\textcolor{gray}{\textbf{\textbf{Frankfurter Zeitung\orgindex{Frankfurter Zeitung@Frankfurter Zeitung|pw}}}}\pend
           \pstart
           \textcolor{gray}{\textbf{(\begin{otherlanguage}{french}Gazette de Francfort\end{otherlanguage}\orgindex{Frankfurter Zeitung@Frankfurter Zeitung|pw}).}}\pend
           \pstart
           \textcolor{gray}{\textbf{\textbf{\begin{otherlanguage}{french}Fondateur M.\end{otherlanguage}{ }L. Sonnemann\pwindex{Sonnemann, Leopold 1831-10-29 – 1909-10-30@\textsc{Sonnemann, Leopold} (1831-10-29 – 1909-10-30), \emph{Journalist, Herausgeber}|pw}.}}}\pend
           \pstart
           \begin{otherlanguage}{french}\textcolor{gray}{\textbf{Journal\pwindex{?? Werk@Nicht ermittelte Verfasserinnen und Verfasser!Frankfurter Zeitung1856 – 1943@\emph{Frankfurter Zeitung} {[}1856 – 1943{]}|pwv} politique,
                        financier,}}\end{otherlanguage}\pend
           \pstart
           \begin{otherlanguage}{french}\textcolor{gray}{\textbf{commercial et littéraire.}}\end{otherlanguage}\pend
           \pstart
           \begin{otherlanguage}{french}\textcolor{gray}{\textbf{\textbf{Paraissant trois fois par jour.}}}\end{otherlanguage}\pend
           \pstart
           \begin{otherlanguage}{french}\textcolor{gray}{\textbf{\textbf{Bureau à Paris\oindex{Paris@\textbf{Paris}|pw}:}}}\end{otherlanguage}\hfill \textsc{Paris\oindex{Paris@\textbf{Paris}|pw}}, 16. Januar.\pend
           \pstart
           \begin{otherlanguage}{french}\textcolor{gray}{\textbf{\textbf{24. Rue Feydeau\oindex{rue Feydeau@\textbf{rue Feydeau}|pw}.}}}\end{otherlanguage}\pend
           \pstart\center{}Mein lieber Freund,\pend\pstart
           Ich hatte \textsc{Thorel\pwindex{Thorel, Jean 1859-09-11 – 1916-08-20@\textsc{Thorel, Jean} (1859-09-11 – 1916-08-20), \emph{Übersetzer, Dramatiker}|pw}} die Frankf. Zeit.\pwindex{?? Werk@Nicht ermittelte Verfasserinnen und Verfasser!Frankfurter Zeitung1856 – 1943@\emph{Frankfurter Zeitung} {[}1856 – 1943{]}|pw} mit dem \label{K_L02764-5v}\edtext{Referat\pwindex{Schauspielhaus. [Premiere von Liebelei]1896-01-12@\emph{Schauspielhaus. [Premiere von Liebelei]} {[}1896-01-12{]}|pwv}}{\lemma{\textnormal{\emph{Referat}}}\Cendnote{\textnormal{m.\pwindex{Mamroth, Fedor 21.02.1851 – 25.06.1907@\textsc{Mamroth, Fedor} (21.02.1851 – 25.06.1907), \emph{Journalist, Kritiker}|pwkv} [=Fedor Mamroth\pwindex{Mamroth, Fedor 21.02.1851 – 25.06.1907@\textsc{Mamroth, Fedor} (21.02.1851 – 25.06.1907), \emph{Journalist, Kritiker}|pwk}]: \emph{Schauspielhaus}\pwindex{Schauspielhaus. [Premiere von Liebelei]1896-01-12@\emph{Schauspielhaus. [Premiere von Liebelei]} {[}1896-01-12{]}|pwk}. In: \emph{Frankfurter
                        Zeitung}\pwindex{?? Werk@Nicht ermittelte Verfasserinnen und Verfasser!Frankfurter Zeitung1856 – 1943@\emph{Frankfurter Zeitung} {[}1856 – 1943{]}|pwk}, Jg. 40, Nr. 12, 12. 1. 1896,
                     Zweites Morgenblatt, S. 1.}}}\label{K_L02764-5h} geſchickt, um ihn zur raſcheren
               Erledigung anzutreiben. Das hat auch gewirkt. Heut
               erhalte ich beifolgenden Brief. Das iſt der erſte kleine Erfolg Deines Stück\pwindex{Schnitzler, Arthur 15.05.1862 – 21.10.1931@\textsc{Schnitzler, Arthur} (15.05.1862 – 21.10.1931), \emph{Schriftsteller, Mediziner}!Liebelei. Schauspiel in drei Akten1895-10-09@\strich\emph{Liebelei. Schauspiel in drei Akten} {[}1895-10-09{]}|pwv}es in Frankreich\oindex{Frankreich@\textbf{Frankreich}|pw}; mögen größere nachkommen! \textsc{Carré\pwindex{Carre, Albert 22.06.1852 – 11.12.1938@\textsc{Carré, Albert} (22.06.1852 – 11.12.1938), \emph{Schriftsteller, Theaterleiter, Schauspieler}|pw}} und \textsc{Torel\pwindex{Thorel, Jean 1859-09-11 – 1916-08-20@\textsc{Thorel, Jean} (1859-09-11 – 1916-08-20), \emph{Übersetzer, Dramatiker}|pw}} ſind die Directoren\pwindex{Carre, Albert 22.06.1852 – 11.12.1938@\textsc{Carré, Albert} (22.06.1852 – 11.12.1938), \emph{Schriftsteller, Theaterleiter, Schauspieler}|pwv}\pwindex{Thorel, Jean 1859-09-11 – 1916-08-20@\textsc{Thorel, Jean} (1859-09-11 – 1916-08-20), \emph{Übersetzer, Dramatiker}|pwv} des \textsc{Vaudeville\orgindex{Theâtre du Vaudeville@Théâtre du Vaudeville|pw}}. Es wäre herrlich, wenn an dieſem vornehmen Theater\orgindex{Theâtre du Vaudeville@Théâtre du Vaudeville|pwv}, wo die \textsc{Réjane\pwindex{Rejane 1856-06-05 – 1920-06-14@\textsc{Réjane} (1856-06-05 – 1920-06-14), \emph{Schauspielerin}|pw}} die Hauperſon iſt, etwas {\pb}zu machen wäre. Ich
               möchte gern \strikeout{über} die freien Bühnen (\textsc{Œuvre\orgindex{Theâtre de l Œuvre@Théâtre de l’Œuvre|pw}}, \textsc{Théâtre Libre\orgindex{Theâtre Libre@Théâtre Libre|pw}}) mit ihren Miſt-Aufführungen umgehen. Jedenfalls ſchließe einſtweilen \uline{keinerlei} Überſetzungs-Engagement ab. Könnte ich nicht
               ein paar Exemplare des Stück\pwindex{Schnitzler, Arthur 15.05.1862 – 21.10.1931@\textsc{Schnitzler, Arthur} (15.05.1862 – 21.10.1931), \emph{Schriftsteller, Mediziner}!Liebelei. Schauspiel in drei Akten1895-10-09@\strich\emph{Liebelei. Schauspiel in drei Akten} {[}1895-10-09{]}|pwv}es
               haben?\pend
           \pstart
           Was in Frankfurt\oindex{Frankfurt am Main@\textbf{Frankfurt am Main}|pw} vorgegangen iſt, weiß ich nicht.
               Meine Mutter\pwindex{Goldmann, Clementine 1842-05-15 – 1924-02-24@\textsc{Goldmann, Clementine} (1842-05-15 – 1924-02-24)|pwv}, die mir ſonſt
               drei Mal die Woche ſchreibt, um {\pb}mir mitzutheilen,
               wenn irgend Jemandem dort die Naſe weh thut, iſt mir jeden Bericht über Deine
               Anweſenheit ſchuldig geblieben. Oh, ſie können Einen nervös machen, die Herrſchaften
               von der Familie!\pend
           \pstart
           Hoffentlich biſt Du geſund heimgekehrt.\pend
           \pstart
           Grüß’ Dich Gott, mein lieber Freund!\pend
           \pstart
           Dein treuer {\\[\baselineskip]}\spacefill\mbox{Paul Goldmnn}\pend
           \leftskip=0em{}{\bigskip}\pstart
           \raggedleft{}{\pb}{[}hs. Thorel:{]} 12 rue de Milan\oindex{Rue de Milan@\textbf{Rue de Milan}|pw}\pend
           \pstart{}\begin{otherlanguage}{french}Cher Monsieur Goldmann\end{otherlanguage}\pend\pstart
           \label{K_L02764-2v}\edtext{\begin{otherlanguage}{french}Je viens – enfin – de lire »Liebelei\pwindex{Schnitzler, Arthur 15.05.1862 – 21.10.1931@\textsc{Schnitzler, Arthur} (15.05.1862 – 21.10.1931), \emph{Schriftsteller, Mediziner}!Liebelei. Schauspiel in drei Akten1895-10-09@\strich\emph{Liebelei. Schauspiel in drei Akten} {[}1895-10-09{]}|pw}«{[}.{]} C’est un pur bijou\strikeout{x}, d’une délicateſse, d’une fraîcheur, et d’une
                  harmonie parfaite. Il faudra absolument que nous reparlions de cela. Auſsitôt que
                  je vais avoir un instant, je vous demanderai rendez-vous.\end{otherlanguage}}{\lemma{\textnormal{\emph{Je … rendez-vous.}}}\Cendnote{\textnormal{französisch: Ich habe – endlich – die
                  Lektüre von »\emph{Liebelei}\pwindex{Schnitzler, Arthur 15.05.1862 – 21.10.1931@\textsc{Schnitzler, Arthur} (15.05.1862 – 21.10.1931), \emph{Schriftsteller, Mediziner}!Liebelei. Schauspiel in drei Akten1895-10-09@\strich\emph{Liebelei. Schauspiel in drei Akten} {[}1895-10-09{]}|pwk}« abgeschlossen. Es ist
                  ein reines Juwel, von zartester, frischer und perfekter Harmonie. Wir müssen
                  unbedingt einmal darüber sprechen. Sobald ich einen Moment Zeit habe, werde ich
                  Sie um einen Treffen bitten.}}}\label{K_L02764-2h}\pend
           \pstart
           \begin{otherlanguage}{french}Votre dévoué\end{otherlanguage}{\\[\baselineskip]}\spacefill\mbox{Jean Thorel\pwindex{Thorel, Jean 1859-09-11 – 1916-08-20@\textsc{Thorel, Jean} (1859-09-11 – 1916-08-20), \emph{Übersetzer, Dramatiker}|pw}}\pend
           \leftskip=0em{}\pstart
           \noindent{}\label{K_L02764-123v}\edtext{J’écris dès aujourd’hui. – Po\textcolor{gray}{×}\-\textcolor{gray}{×}{ }\textcolor{gray}{×}\-\textcolor{gray}{×}arré!}{\lemma{\textnormal{\emph{J’écris … arré!}}}\Cendnote{\textnormal{französisch: XXXX}}}\label{K_L02764-123h}\pend
           
         
         \endnumbering\mylabel{h}\end{ledgroupsized}  \newcommand{\dateiname}{L02764}\newcommand{\titel}{Paul Goldmann an Arthur Schnitzler, 16. 1. [1896]}\newcommand{\editorInnen}{Martin Anton Müller und Laura Untner}%% latex-leseansicht-abspann.tex
%% Abspann für die Leseansicht.
%% Der Schalter \ifkorrekturansicht ist bereits durch den Vorspann gesetzt.

%% latex-abspann.tex
%% Gemeinsamer Abspann für Korrekturansicht und Leseansicht.
%% Setzt den Schalter \ifkorrekturansicht voraus (gesetzt in den
%% einbindenden Dateien latex-korrekturansicht-abspann.tex bzw.
%% latex-leseansicht-abspann.tex).
%% ---------------------------------------------------------------

\normalsize

% Das esempio-Environment wird nur in der Leseansicht benötigt
\ifkorrekturansicht\else
\newenvironment{esempio}[3]%
{
    \vspace{1.5ex}
    \rlap{\underline{#1}}
    \par
    \setlength{\parindent}{0cm}
    \nopagebreak
    \leftskip=#2cm
    \rightskip=#3cm
}
{
    \par
}
\fi

\doendnotes{C}
\bigskip
\vfill

\clearpage

\footnotesize

\ifkorrekturansicht
  \lohead{\textsc{register}}
\fi

% theindex-Environment neu definieren ohne reledmac
\makeatletter
\renewenvironment{theindex}{%
  \ifkorrekturansicht
    \section*{\indexname}%
  \else
    \subsubsection*{Index der erwähnten Entitäten}%
  \fi
  \setlength{\parindent}{0pt}%
  \setlength{\parskip}{0pt plus 0.3pt}%
  \let\item\@idxitem
}{%
  \ifkorrekturansicht\clearpage\fi
}
\makeatother

\IfFileExists{\jobname-pw.ind}{\input{\jobname-pw.ind}}{}

% Quellenangabe nur in der Leseansicht
\ifkorrekturansicht\else
% Fallback-Definitionen, falls die .tex-Datei \titel etc. nicht gesetzt hat
\providecommand{\titel}{}
\providecommand{\editorInnen}{}
\providecommand{\dateiname}{\jobname}

\vspace{3cm}

\vfill

\footnotesize
\textsc{Quelle}: \titel. Herausgegeben von {\editorInnen}. In: \emph{Arthur Schnitzler: Briefwechsel mit Autorinnen und Autoren}.
 Digitale Edition, https://schnitzler-briefe.acdh.oeaw.ac.at/{\dateiname}.html (Stand \today)
\fi

\end{document}


      