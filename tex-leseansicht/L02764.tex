%% latex-korrekturansicht-vorspann.tex
%% Vorspann für die Korrekturansicht.
%% Lädt die gemeinsame Datei latex-vorspann.tex mit gesetztem Schalter.

\newif\ifkorrekturansicht
\korrekturansichttrue

\input{../tex-inputs/latex-vorspann}


\section[ Paul Goldmann an Arthur Schnitzler, 16. 1. {[}1896{]}]{L02764 Paul Goldmann an Arthur Schnitzler, 16. 1. {[}1896{]}}
\nopagebreak\mylabel{L02764v}
\rehead{ }\normalsize\beginnumbering\briefempfaengerindex{Schnitzler, Arthur@\textsc{Schnitzler, Arthur}!zzzGoldmann, Paul@\emph{von Paul Goldmann}!1896-01-162@{16. 1. {[}1896{]}}|(be}
\toendnotes[C]{\smallbreak\pagebreak[2]}\Standort{DLA, A:Schnitzler, HS.NZ85.1.3166.}
\physDesc{Brief, 1 Blatt, 3 Seiten, 1388 Zeichen
\newline{}Handschrift: blaue Tinte, deutsche Kurrent
\newline{}Beilage: handschriftlicher Brief: 1 Blatt, 1 Seite, schwarze Tinte,
                                 lateinische Kurrent 
\newline{}Schnitzler: 1) mit Bleistift das Jahr »96« vermerkt  2) mit rotem Buntstift eine Unterstreichung}\toendnotes[C]{\smallbreak}
\pstart
           {\pb}\textcolor{gray}{\textbf{\textbf{Frankfurter Zeitung\orgindex{Frankfurter Zeitung@Frankfurter Zeitung|pw}}}}\pend
           
\pstart
           \textcolor{gray}{\textbf{(\begin{otherlanguage}{french}Gazette de Francfort\end{otherlanguage}\orgindex{Frankfurter Zeitung@Frankfurter Zeitung|pw}).}}\pend
           
\pstart
           \textcolor{gray}{\textbf{\textbf{\begin{otherlanguage}{french}Fondateur M.\end{otherlanguage}{ }L. Sonnemann\pwindex{Sonnemann, Leopold 1831-10-29 – 1909-10-30@\textsc{Sonnemann, Leopold} (1831-10-29 – 1909-10-30), \emph{Journalist/Journalistin, Herausgeber/Herausgeberin}|pw}.}}}\pend
           
\pstart
           \begin{otherlanguage}{french}\textcolor{gray}{\textbf{Journal\pwindex{Frankfurter Zeitung@\emph{Frankfurter Zeitung}|pwv} politique,
                        financier,}}\end{otherlanguage}\pend
           
\pstart
           \begin{otherlanguage}{french}\textcolor{gray}{\textbf{commercial et littéraire.}}\end{otherlanguage}\pend
           
\pstart
           \begin{otherlanguage}{french}\textcolor{gray}{\textbf{\textbf{Paraissant trois fois par jour.}}}\end{otherlanguage}\pend
           
\pstart
           \begin{otherlanguage}{french}\textcolor{gray}{\textbf{\textbf{Bureau à Paris\oindex{Paris@\textbf{Paris}, \emph{P.PPLC}|pw}:}}}\end{otherlanguage}\hfill \textsc{Paris\oindex{Paris@\textbf{Paris}, \emph{P.PPLC}|pw}}, 16. Januar.\pend
           
\pstart
           \begin{otherlanguage}{french}\textcolor{gray}{\textbf{\textbf{24. Rue Feydeau\oindex{rue Feydeau@\textbf{rue Feydeau}, \emph{Straße (K.STR)}|pw}.}}}\end{otherlanguage}\pend
           
\pstart\center{}Mein lieber Freund,\pend\vspace{0.5em}
\pstart
           Ich hatte \textsc{Thorel\pwindex{Thorel, Jean 1859-09-11 – 1916-08-20@\textsc{Thorel, Jean} (1859-09-11 – 1916-08-20), \emph{Übersetzer/Übersetzerin, Dramatiker/Dramatikerin}|pw}} die Frankf. Zeit.\pwindex{Frankfurter Zeitung@\emph{Frankfurter Zeitung}|pw} mit dem \label{K_L02764-1v}\edtext{Referat\pwindex{Schauspielhaus. [Premiere von Liebelei]@\emph{Schauspielhaus. [Premiere von Liebelei]}|pwv}}{\lemma{\textnormal{\emph{Referat}}}\Cendnote{\textnormal{m.\pwindex{Mamroth, Fedor 21.02.1851 – 25.06.1907@\textsc{Mamroth, Fedor} (21.02.1851 – 25.06.1907), \emph{Journalist/Journalistin, Kritiker/Kritikerin}|pwkv} [ = Fedor Mamroth\pwindex{Mamroth, Fedor 21.02.1851 – 25.06.1907@\textsc{Mamroth, Fedor} (21.02.1851 – 25.06.1907), \emph{Journalist/Journalistin, Kritiker/Kritikerin}|pwk}]: \emph{Schauspielhaus}\pwindex{Schauspielhaus. [Premiere von Liebelei]@\emph{Schauspielhaus. [Premiere von Liebelei]}|pwk}. In: \emph{Frankfurter Zeitung}\pwindex{Frankfurter Zeitung@\emph{Frankfurter Zeitung}|pwk}, Jg. 40, Nr. 12, 12. 1. 1896, Zweites Morgenblatt, S. 1.}}}\label{K_L02764-1} geſchickt, um
               ihn zur raſcheren Erledigung anzutreiben. Das hat auch gewirkt. Heut erhalte ich beifolgenden Brief. Das iſt der erſte
               kleine Erfolg Deines Stück\pwindex{Liebelei. Schauspiel in drei Akten@\emph{Liebelei. Schauspiel in drei Akten}|pwv}es
               in Frankreich\oindex{Frankreich@\textbf{Frankreich}, \emph{A.PCLI}|pw}; mögen größere nachkommen! \textsc{Carré\pwindex{Carre, Albert 22.06.1852 – 11.12.1938@\textsc{Carré, Albert} (22.06.1852 – 11.12.1938), \emph{Schriftsteller/Schriftstellerin, Theaterleiter/Theaterleiterin, Schauspieler/Schauspielerin}|pw}} und \textsc{Torel\pwindex{Thorel, Jean 1859-09-11 – 1916-08-20@\textsc{Thorel, Jean} (1859-09-11 – 1916-08-20), \emph{Übersetzer/Übersetzerin, Dramatiker/Dramatikerin}|pw}} ſind die Directoren\pwindex{Carre, Albert 22.06.1852 – 11.12.1938@\textsc{Carré, Albert} (22.06.1852 – 11.12.1938), \emph{Schriftsteller/Schriftstellerin, Theaterleiter/Theaterleiterin, Schauspieler/Schauspielerin}|pwv}\pwindex{Thorel, Jean 1859-09-11 – 1916-08-20@\textsc{Thorel, Jean} (1859-09-11 – 1916-08-20), \emph{Übersetzer/Übersetzerin, Dramatiker/Dramatikerin}|pwv} des \textsc{Vaudeville\orgindex{Theâtre du Vaudeville@Théâtre du Vaudeville|pw}}. Es wäre herrlich, wenn an dieſem vornehmen Theater\orgindex{Theâtre du Vaudeville@Théâtre du Vaudeville|pwv}, wo die \textsc{Réjane\pwindex{Rejane 1856-06-05 – 1920-06-14@\textsc{Réjane} (1856-06-05 – 1920-06-14), \emph{Schauspieler/Schauspielerin}|pw}} die Hauperſon iſt, etwas {\pb}zu machen wäre. Ich
               möchte gern \strikeout{über} die freien Bühnen (\textsc{Œuvre\orgindex{Theâtre de l Œuvre@Théâtre de l’Œuvre|pw}}, \textsc{Théâtre Libre\orgindex{Theâtre Libre@Théâtre Libre|pw}}) mit ihren Miſt-Aufführungen umgehen. Jedenfalls ſchließe einſtweilen \uline{keinerlei} Überſetzungs-Engagement ab. Könnte ich nicht
               ein paar Exemplare des Stück\pwindex{Liebelei. Schauspiel in drei Akten@\emph{Liebelei. Schauspiel in drei Akten}|pwv}es
               haben?\pend
           
\pstart
           Was in Frankfurt\oindex{Frankfurt am Main@\textbf{Frankfurt am Main}, \emph{P.PPLA3}|pw} vorgegangen iſt, weiß ich nicht.
               Meine Mutter\pwindex{Goldmann, Clementine 1842-05-15 – 1924-02-24@\textsc{Goldmann, Clementine} (1842-05-15 – 1924-02-24)|pwv}, die mir ſonſt
               drei Mal die Woche ſchreibt, um {\pb}mir mitzutheilen,
               wenn irgend Jemandem dort die Naſe weh thut, iſt mir jeden Bericht über Deine
               Anweſenheit ſchuldig geblieben. Oh, ſie können Einen nervös machen, die Herrſchaften
               von der Familie!\pend
           
\pstart
           Hoffentlich biſt Du geſund heimgekehrt.\pend
           
\pstart
           Grüß’ Dich Gott, mein lieber Freund!\pend
           
\pstart
           Dein treuer {\\[\baselineskip]}\spacefill\mbox{Paul Goldmnn}\pend
           \leftskip=0em{}\selectlanguage{ngerman}\vspace{1em}{\vspace{1\baselineskip}}
\pstart
           \raggedleft{}{\pb}{[}hs. :{]} 12 rue de Milan\oindex{Rue de Milan@\textbf{Rue de Milan}, \emph{Straße (K.STR)}|pw}\pend
           
\pstart{}\begin{otherlanguage}{french}Cher Monsieur Goldmann\end{otherlanguage}\pend\vspace{0.5em}
\pstart
           \label{K_L02764-2v}\edtext{\begin{otherlanguage}{french}Je viens – enfin – de lire »Liebelei\pwindex{Liebelei. Schauspiel in drei Akten@\emph{Liebelei. Schauspiel in drei Akten}|pw}«{[}.{]} C’est un pur bijou\strikeout{x}, d’une délicateſse, d’une fraîcheur, et d’une
                  harmonie parfaites. Il faudra absolument que nous reparlions de cela. Auſsitôt que
                  je vais avoir un instant, je vous demanderai rendez-vous.\end{otherlanguage}}{\lemma{\textnormal{\emph{Je … rendez-vous.}}}\Cendnote{\textnormal{französisch: Ich habe – endlich – die
                  Lektüre von \emph{Liebelei}\pwindex{Liebelei. Schauspiel in drei Akten@\emph{Liebelei. Schauspiel in drei Akten}|pwk} abgeschlossen. Es ist
                  ein reines Juwel, von perfekter Zartheit, Frische und Harmonie. Wir müssen
                  unbedingt einmal darüber sprechen. Sobald ich einen Moment Zeit habe, werde ich
                  Sie um ein Treffen bitten.}}}\label{K_L02764-2}\pend
           
\pstart
           \begin{otherlanguage}{french}Votre dévoué\end{otherlanguage}{\\[\baselineskip]}\spacefill\mbox{Jean Thorel\pwindex{Thorel, Jean 1859-09-11 – 1916-08-20@\textsc{Thorel, Jean} (1859-09-11 – 1916-08-20), \emph{Übersetzer/Übersetzerin, Dramatiker/Dramatikerin}|pw}}\pend
           \leftskip=0em{}
\pstart
           \noindent{}\label{K_L02764-3v}\edtext{J’écris dès aujourd’hui à Thorel\pwindex{Thorel, Jean 1859-09-11 – 1916-08-20@\textsc{Thorel, Jean} (1859-09-11 – 1916-08-20), \emph{Übersetzer/Übersetzerin, Dramatiker/Dramatikerin}|pw} et Carré\pwindex{Carre, Albert 22.06.1852 – 11.12.1938@\textsc{Carré, Albert} (22.06.1852 – 11.12.1938), \emph{Schriftsteller/Schriftstellerin, Theaterleiter/Theaterleiterin, Schauspieler/Schauspielerin}|pw}!}{\lemma{\textnormal{\emph{J’écris … Carré!}}}\Cendnote{\textnormal{französisch: Ich schreibe schon heute an Thorel\pwindex{Thorel, Jean 1859-09-11 – 1916-08-20@\textsc{Thorel, Jean} (1859-09-11 – 1916-08-20), \emph{Übersetzer/Übersetzerin, Dramatiker/Dramatikerin}|pwk} und Carré\pwindex{Carre, Albert 22.06.1852 – 11.12.1938@\textsc{Carré, Albert} (22.06.1852 – 11.12.1938), \emph{Schriftsteller/Schriftstellerin, Theaterleiter/Theaterleiterin, Schauspieler/Schauspielerin}|pwk}!}}}\label{K_L02764-3}\pend
           \selectlanguage{ngerman}\endnumbering\briefempfaengerindex{Schnitzler, Arthur@\textsc{Schnitzler, Arthur}!zzzGoldmann, Paul@\emph{von Paul Goldmann}!1896-01-162@{16. 1. {[}1896{]}}|)be}\mylabel{L02764h}  \normalsize

\doendnotes{C}
\bigskip
\vfill

\clearpage

\footnotesize

\lohead{\textsc{register}}

% Definiere theindex-Environment komplett neu ohne reledmac
\makeatletter
\renewenvironment{theindex}{%
  \section*{\indexname}%
  \setlength{\parindent}{0pt}%
  \setlength{\parskip}{0pt plus 0.3pt}%
  \let\item\@idxitem
}{%
  \clearpage
}
\makeatother

\IfFileExists{\jobname-pw.ind}{\input{\jobname-pw.ind}}{}

\end{document}

      