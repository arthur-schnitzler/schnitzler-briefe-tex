%% latex-leseansicht-vorspann.tex
%% Vorspann für die Leseansicht.
%% Lädt die gemeinsame Datei latex-vorspann.tex mit nicht gesetztem Schalter.

\newif\ifkorrekturansicht
\korrekturansichtfalse

\input{../tex-inputs/latex-vorspann}


\section[Hugo von Hofmannsthal an Arthur Schnitzler, {[}30. 3. 1902{]}]{L01213 Hugo von Hofmannsthal an Arthur Schnitzler, {[}30. 3. 1902{]}}
\nopagebreak\mylabel{L01213v}
\rehead{ }\normalsize\beginnumbering\briefempfaengerindex{Schnitzler, Arthur@\textsc{Schnitzler, Arthur}!zzzHofmannsthal, Hugo von@\emph{von Hugo von Hofmannsthal}!1902-03-302@{{[}30. 3. 1902{]}}|(be}
\toendnotes[C]{\smallbreak\pagebreak[2]}
\correspDesc{Versand  durch Hugo von Hofmannsthal am [30. 3. 1902] in Wien
\newline{}Erhalt  durch Arthur Schnitzler im Zeitraum [30. 3. 1902
                  – 3. 4. 1902?] in Wien}\toendnotes[C]{\smallbreak}
\Standort{CUL, Schnitzler, B 43.}
\physDesc{Brief, 2 Blätter, 8 Seiten, 1822 Zeichen
\newline{}Handschrift: schwarze Tinte, deutsche Kurrent
\newline{}Schnitzler: mit Bleistift datiert: »30/3 902« 
\newline{}Ordnung: 1) mit Bleistift von unbekannter Hand nummeriert: »\strikeout{194}«  2) mit Bleistift von unbekannter Hand nummeriert:
                                    »187.1« beziehungsweise auf dem zweiten Blatt:
                                    »187.2.«}
\buchAbdrucke{\weitereDrucke{Hugo von Hofmannsthal, Arthur Schnitzler: \emph{Briefwechsel}. Herausgegeben von Therese Nickl und Heinrich Schnitzler. Frankfurt am Main: \emph{S. Fischer} 1964, S. 157.} }\toendnotes[C]{\smallbreak}
\pstart{}{\pb}mein lieber Arthur\pend\vspace{0.5em}
\pstart
           ich danke Ihnen herzlich für Ihren lieben Brief. Ich denke, Sie müſſen wiſſen daſs
               eine{ }ſolche Heftigkeit, wie die meinige, eben nur gegen einen Menſchen ausbrechen
               kann, der einem{ }ſo nahe{ }ſteht, daſs ein »pikiert-sein« gar nicht {\pb}eintreten kann,{ }ſondern eben nur
               ein plötzlicher Ausbruch von Ungeduld, wenn man merkt, daſs der andere einem etwas
               unangenehmes thut, ohne das Bewuſstſein davon.\pend
           
\pstart
           \numberlinefalse{}\centering{}–\numberlinetrue{}\pend
           
\pstart
           Das iſt alſo vollkommen erledigt und weggeblaſen. {\pb}Aber:\pend
           
\pstart
           ich habe bis jetzt weder der Gfin Thun\pwindex{Thun-Hohenstein-Salm-Reifferscheidt, Christiane von 12.\,6.\,1859 Doksy – 6.\,8.\,1935 Prag@\textsc{Thun-Hohenstein-Salm-Reifferscheidt, Christiane von} (12.\,6.\,1859 Doksy – 6.\,8.\,1935 Prag), \emph{Schriftstellerin}|pw}, noch
                  Kaſſner\pwindex{Kassner, Rudolf 11.\,9.\,1873 Velké Pavlovice – 1.\,4.\,1959 Sierre@\textsc{Kassner, Rudolf} (11.\,9.\,1873 Velké Pavlovice – 1.\,4.\,1959 Sierre), \emph{Schriftsteller}|pw} abgeſagt.\pend
           
\pstart
           Ich frage alſo nochmals an (im Telephon verſuchte ich heute, Sie waren aber nicht in
                  Wien\oindex{Wien@\textbf{Wien}, \emph{Verwaltungsgebiet}|pw}) ob es Ihnen unbequem wäre,
                  Donnerstag{ }1\textsuperscript{h} dieſes Frühſtück zu haben? Jetzt{ }ſteht die Sache aber
                  {\pb}natürlich ganz anders: ich
               erwarte mir von Ihnen ganz gleichmäßig eine bejahende oder eine verneinende Antwort.
               Sagen Sie mir ab (ohne weitere Motivierung){ }ſo weiß ich, es iſt Ihnen wirklich{ }ſchwer, einzutheilen, bin natürlich weder erſtaunt noch im \uline{geringſten bös} (jetzt iſt ja das Formale der Sache nicht mehr exiſtierend)
                  {\pb}ſagen Sie mir aber zu,{ }ſo
               bleibt es dabei, ich bin nämlich Donnerstag ohnehin in Wien\oindex{Wien@\textbf{Wien}, \emph{Verwaltungsgebiet}|pw}.\pend
           
\pstart
           Miſsverſtehen wir uns alſo jetzt gewiſs nicht, lieber Arthur.\pend
           
\pstart
           Es wäre mir eine kleine Freude, einer lieben und nicht beſonders heiteren Frau\pwindex{Thun-Hohenstein-Salm-Reifferscheidt, Christiane von 12.\,6.\,1859 Doksy – 6.\,8.\,1935 Prag@\textsc{Thun-Hohenstein-Salm-Reifferscheidt, Christiane von} (12.\,6.\,1859 Doksy – 6.\,8.\,1935 Prag), \emph{Schriftstellerin}|pwv}{ }{\pb}dieſen Wunſch zu erfüllen, \uline{aber} wenn es zuſtande käme unter dem geringſten Zwang
               Ihrerſeits, Ungeduld, kurz Selbſtüberwindung,{ }ſo wäre das eine Überlaſtung dieſer
               kleinen Veranſtaltung und da iſt \uline{viel} geſcheidter {\pb}ſie kommt gar nicht zustande.\pend
           
\pstart
           Bitte alſo \uline{telegrafieren} Sie mir ja oder nein, ohne
               Motivierung und mit völliger innerer Freiheit.\pend
           
\pstart
           Nur bitte Telegramm oder Telefon damit ich den beiden Perſonen\pwindex{Thun-Hohenstein-Salm-Reifferscheidt, Christiane von 12.\,6.\,1859 Doksy – 6.\,8.\,1935 Prag@\textsc{Thun-Hohenstein-Salm-Reifferscheidt, Christiane von} (12.\,6.\,1859 Doksy – 6.\,8.\,1935 Prag), \emph{Schriftstellerin}|pwv}\pwindex{Kassner, Rudolf 11.\,9.\,1873 Velké Pavlovice – 1.\,4.\,1959 Sierre@\textsc{Kassner, Rudolf} (11.\,9.\,1873 Velké Pavlovice – 1.\,4.\,1959 Sierre), \emph{Schriftsteller}|pwv} rechtzeitig eventuell {\pb}abſagen kann.\pend
           
\pstart
           In die Generalprobe\pwindex{\textcolor{red}{\textsuperscript{XXXX indx1}}!Über unsere Kraft. Zweiter Teil@\strich\emph{Über unsere Kraft. Zweiter Teil}|pwv}{ }Mittwoch kann ich kaum gehen, weil ich abends zur \textsc{Duse}\pwindex{Duse, Eleonora 3.\,10.\,1858 Vigevano – 21.\,4.\,1924 Pittsburgh@\textsc{Duse, Eleonora} (3.\,10.\,1858 Vigevano – 21.\,4.\,1924 Pittsburgh), \emph{Schauspielerin}|pw} gehe, und das ein biſſl viel ist.\pend
           
\pstart
           Auf bald, hoffentlich.{\\[\baselineskip]}Von Herzen Ihr{\\[\baselineskip]}\spacefill\mbox{Hugo.}\pend
           \leftskip=0em{}\selectlanguage{ngerman}\endnumbering\briefempfaengerindex{Schnitzler, Arthur@\textsc{Schnitzler, Arthur}!zzzHofmannsthal, Hugo von@\emph{von Hugo von Hofmannsthal}!1902-03-302@{{[}30. 3. 1902{]}}|)be}\mylabel{L01213h}  \newcommand{\dateiname}{L01213}\newcommand{\titel}{Hugo von Hofmannsthal an Arthur Schnitzler, [30. 3. 1902]}\newcommand{\editorInnen}{Martin Anton Müller und Gerd-Hermann Susen}%% latex-leseansicht-abspann.tex
%% Abspann für die Leseansicht.
%% Der Schalter \ifkorrekturansicht ist bereits durch den Vorspann gesetzt.

%% latex-abspann.tex
%% Gemeinsamer Abspann für Korrekturansicht und Leseansicht.
%% Setzt den Schalter \ifkorrekturansicht voraus (gesetzt in den
%% einbindenden Dateien latex-korrekturansicht-abspann.tex bzw.
%% latex-leseansicht-abspann.tex).
%% ---------------------------------------------------------------

\normalsize

% Das esempio-Environment wird nur in der Leseansicht benötigt
\ifkorrekturansicht\else
\newenvironment{esempio}[3]%
{
    \vspace{1.5ex}
    \rlap{\underline{#1}}
    \par
    \setlength{\parindent}{0cm}
    \nopagebreak
    \leftskip=#2cm
    \rightskip=#3cm
}
{
    \par
}
\fi

\doendnotes{C}
\bigskip
\vfill

\clearpage

\footnotesize

\ifkorrekturansicht
  \lohead{\textsc{register}}
\fi

% theindex-Environment neu definieren ohne reledmac
\makeatletter
\renewenvironment{theindex}{%
  \ifkorrekturansicht
    \section*{\indexname}%
  \else
    \subsubsection*{Index der erwähnten Entitäten}%
  \fi
  \setlength{\parindent}{0pt}%
  \setlength{\parskip}{0pt plus 0.3pt}%
  \let\item\@idxitem
}{%
  \ifkorrekturansicht\clearpage\fi
}
\makeatother

\IfFileExists{\jobname-pw.ind}{\input{\jobname-pw.ind}}{}

% Quellenangabe nur in der Leseansicht
\ifkorrekturansicht\else
% Fallback-Definitionen, falls die .tex-Datei \titel etc. nicht gesetzt hat
\providecommand{\titel}{}
\providecommand{\editorInnen}{}
\providecommand{\dateiname}{\jobname}

\vspace{3cm}

\vfill

\footnotesize
\textsc{Quelle}: \titel. Herausgegeben von {\editorInnen}. In: \emph{Arthur Schnitzler: Briefwechsel mit Autorinnen und Autoren}.
 Digitale Edition, https://schnitzler-briefe.acdh.oeaw.ac.at/{\dateiname}.html (Stand \today)
\fi

\end{document}


