%% latex-korrekturansicht-vorspann.tex
%% Vorspann für die Korrekturansicht.
%% Lädt die gemeinsame Datei latex-vorspann.tex mit gesetztem Schalter.

\newif\ifkorrekturansicht
\korrekturansichttrue

\input{../tex-inputs/latex-vorspann}


\section[Hugo von Hofmannsthal an Arthur Schnitzler, {[}30. 3. 1902{]}]{L01213 Hugo von Hofmannsthal an Arthur Schnitzler, {[}30. 3. 1902{]}}
\nopagebreak\mylabel{L01213v}
\rehead{ }\normalsize\beginnumbering\briefempfaengerindex{Schnitzler, Arthur@\textsc{Schnitzler, Arthur}!zzzHofmannsthal, Hugo von@\emph{von Hugo von Hofmannsthal}!1902-03-302@{{[}30. 3. 1902{]}}|(be}
\toendnotes[C]{\smallbreak\pagebreak[2]}\Standort{CUL, Schnitzler, B 43.}
\physDesc{Brief, 2 Blätter, 8 Seiten, 1822 Zeichen
\newline{}Handschrift: schwarze Tinte, deutsche Kurrent
\newline{}Schnitzler: mit Bleistift datiert: »30/3 902« 
\newline{}Ordnung: 1) mit Bleistift von unbekannter Hand nummeriert: »\strikeout{194}«  2) mit Bleistift von unbekannter Hand nummeriert:
                                    »187.1« beziehungsweise auf dem zweiten Blatt:
                                    »187.2.«}
\buchAbdrucke{\weitereDrucke{Hugo von Hofmannsthal, Arthur Schnitzler: \emph{Briefwechsel}. Frankfurt am Main: \emph{S. Fischer} 1964, S. 157.} }\toendnotes[C]{\smallbreak}
\pstart{}{\pb}mein lieber Arthur\pend\vspace{0.5em}
\pstart
           ich danke Ihnen herzlich für Ihren lieben Brief. Ich denke, Sie müſſen wiſſen daſs
               eine ſolche Heftigkeit, wie die meinige, eben nur gegen einen Menſchen ausbrechen
               kann, der einem ſo nahe ſteht, daſs ein »pikiert-sein« gar nicht {\pb}eintreten kann, ſondern eben nur
               ein plötzlicher Ausbruch von Ungeduld, wenn man merkt, daſs der andere einem etwas
               unangenehmes thut, ohne das Bewuſstſein davon.\pend
           
\pstart
           \numberlinefalse{}\centering{}–\numberlinetrue{}\pend
           
\pstart
           Das iſt alſo vollkommen erledigt und weggeblaſen. {\pb}Aber:\pend
           
\pstart
           ich habe bis jetzt weder der Gfin Thun\pwindex{Thun-Hohenstein-Salm-Reifferscheidt, Christiane von 12.06.1859 – 06.08.1935@\textsc{Thun-Hohenstein-Salm-Reifferscheidt, Christiane von} (12.06.1859 – 06.08.1935), \emph{Schriftsteller/Schriftstellerin}|pw}, noch
                  Kaſſner\pwindex{Kassner, Rudolf 11.09.1873 – 01.04.1959@\textsc{Kassner, Rudolf} (11.09.1873 – 01.04.1959), \emph{Schriftsteller/Schriftstellerin}|pw} abgeſagt.\pend
           
\pstart
           Ich frage alſo nochmals an (im Telephon verſuchte ich heute, Sie waren aber nicht in
                  Wien\oindex{Wien@\textbf{Wien}, \emph{A.ADM2}|pw}) ob es Ihnen unbequem wäre,
                  Donnerstag{ }1\textsuperscript{h} dieſes Frühſtück zu haben? Jetzt ſteht die Sache aber
                  {\pb}natürlich ganz anders: ich
               erwarte mir von Ihnen ganz gleichmäßig eine bejahende oder eine verneinende Antwort.
               Sagen Sie mir ab (ohne weitere Motivierung) ſo weiß ich, es iſt Ihnen wirklich
               ſchwer, einzutheilen, bin natürlich weder erſtaunt noch im \uline{geringſten bös} (jetzt iſt ja das Formale der Sache nicht mehr exiſtierend)
                  {\pb}ſagen Sie mir aber zu, ſo
               bleibt es dabei, ich bin nämlich Donnerstag ohnehin in Wien\oindex{Wien@\textbf{Wien}, \emph{A.ADM2}|pw}.\pend
           
\pstart
           Miſsverſtehen wir uns alſo jetzt gewiſs nicht, lieber Arthur.\pend
           
\pstart
           Es wäre mir eine kleine Freude, einer lieben und nicht beſonders heiteren Frau\pwindex{Thun-Hohenstein-Salm-Reifferscheidt, Christiane von 12.06.1859 – 06.08.1935@\textsc{Thun-Hohenstein-Salm-Reifferscheidt, Christiane von} (12.06.1859 – 06.08.1935), \emph{Schriftsteller/Schriftstellerin}|pwv}{ }{\pb}dieſen Wunſch zu erfüllen, \uline{aber} wenn es zuſtande käme unter dem geringſten Zwang
               Ihrerſeits, Ungeduld, kurz Selbſtüberwindung, ſo wäre das eine Überlaſtung dieſer
               kleinen Veranſtaltung und da iſt \uline{viel} geſcheidter {\pb}ſie kommt gar nicht zustande.\pend
           
\pstart
           Bitte alſo \uline{telegrafieren} Sie mir ja oder nein, ohne
               Motivierung und mit völliger innerer Freiheit.\pend
           
\pstart
           Nur bitte Telegramm oder Telefon damit ich den beiden Perſonen\pwindex{Thun-Hohenstein-Salm-Reifferscheidt, Christiane von 12.06.1859 – 06.08.1935@\textsc{Thun-Hohenstein-Salm-Reifferscheidt, Christiane von} (12.06.1859 – 06.08.1935), \emph{Schriftsteller/Schriftstellerin}|pwv}\pwindex{Kassner, Rudolf 11.09.1873 – 01.04.1959@\textsc{Kassner, Rudolf} (11.09.1873 – 01.04.1959), \emph{Schriftsteller/Schriftstellerin}|pwv} rechtzeitig eventuell {\pb}abſagen kann.\pend
           
\pstart
           In die Generalprobe\pwindex{Ueber unsere Kraft. Zweiter Teil@\emph{Über unsere Kraft. Zweiter Teil}|pwv}{ }Mittwoch kann ich kaum gehen, weil ich abends zur \textsc{Duse}\pwindex{Duse, Eleonora 03.10.1858 – 21.04.1924@\textsc{Duse, Eleonora} (03.10.1858 – 21.04.1924), \emph{Schauspieler/Schauspielerin}|pw} gehe, und das ein biſſl viel ist.\pend
           
\pstart
           Auf bald, hoffentlich.{\\[\baselineskip]}Von Herzen Ihr{\\[\baselineskip]}\spacefill\mbox{Hugo.}\pend
           \leftskip=0em{}\selectlanguage{ngerman}\endnumbering\briefempfaengerindex{Schnitzler, Arthur@\textsc{Schnitzler, Arthur}!zzzHofmannsthal, Hugo von@\emph{von Hugo von Hofmannsthal}!1902-03-302@{{[}30. 3. 1902{]}}|)be}\mylabel{L01213h}  \normalsize

\doendnotes{C}
\bigskip
\vfill

\clearpage

\footnotesize

\lohead{\textsc{register}}

% Definiere theindex-Environment komplett neu ohne reledmac
\makeatletter
\renewenvironment{theindex}{%
  \section*{\indexname}%
  \setlength{\parindent}{0pt}%
  \setlength{\parskip}{0pt plus 0.3pt}%
  \let\item\@idxitem
}{%
  \clearpage
}
\makeatother

\IfFileExists{\jobname-pw.ind}{\input{\jobname-pw.ind}}{}

\end{document}

      