%% latex-korrekturansicht-vorspann.tex
%% Vorspann für die Korrekturansicht.
%% Lädt die gemeinsame Datei latex-vorspann.tex mit gesetztem Schalter.

\newif\ifkorrekturansicht
\korrekturansichttrue

\input{../tex-inputs/latex-vorspann}


\section[Arthur Schnitzler an Adalbert Seligmann, 15. 10. 1905]{L01564 Arthur Schnitzler an Adalbert Seligmann, 15. 10. 1905}
\nopagebreak\mylabel{L01564v}
\rehead{ }\normalsize\beginnumbering\briefempfaengerindex{Seligmann, Adalbert Franz@\textsc{Seligmann, Adalbert Franz}!zzzSchnitzler, Arthur@\emph{von Arthur Schnitzler}!1905-10-151@{15. 10. 1905}|(be}
\toendnotes[C]{\smallbreak\pagebreak[2]}\Standort{Wienbibliothek im Rathaus, H.I.N.-95601.}
\physDesc{Briefkarte, 147 Zeichen
\newline{}Handschrift: schwarze Tinte, deutsche Kurrent}
\pstart
           {\pb}\textcolor{gray}{\textbf{Dr. Arthur Schnitzler}}\hfill 1\substVorne{}\textsuperscript{3}\substDazwischen{}5\substHinten{}. 10. 905\pend
           
\pstart
           \textcolor{gray}{\textbf{Wien XVIII. Spoettelgasse 7\oindex{Edmund-Weiss-Gasse 7@\textbf{Edmund-Weiß-Gasse 7}, \emph{Wohngebäude (K.WHS)}|pw}.}}\pend
           \vspace{0.5em}
\pstart
           verehrteſter Herr Seligmann, über Ihren Brief habe ich mich ſehr
               gefreut. Ich danke Ihnen herzlich!\pend
           
\pstart
           Ihr aufrichtg ergebner{\\[\baselineskip]}\spacefill\mbox{Arth Schnitzler}\pend
           \leftskip=0em{}\selectlanguage{ngerman}\endnumbering\briefempfaengerindex{Seligmann, Adalbert Franz@\textsc{Seligmann, Adalbert Franz}!zzzSchnitzler, Arthur@\emph{von Arthur Schnitzler}!1905-10-151@{15. 10. 1905}|)be}\mylabel{L01564h}  \normalsize

\doendnotes{C}
\bigskip
\vfill

\clearpage

\footnotesize

\lohead{\textsc{register}}

% Definiere theindex-Environment komplett neu ohne reledmac
\makeatletter
\renewenvironment{theindex}{%
  \section*{\indexname}%
  \setlength{\parindent}{0pt}%
  \setlength{\parskip}{0pt plus 0.3pt}%
  \let\item\@idxitem
}{%
  \clearpage
}
\makeatother

\IfFileExists{\jobname-pw.ind}{\input{\jobname-pw.ind}}{}

\end{document}

      