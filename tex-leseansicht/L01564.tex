\input{../tex-inputs/latex-pdf-vorspann}
\begin{center}
            \textcolor{red}{ENTWURF. ENTZIFFERUNG NOCH NICHT KORREKTURGELESEN}
                      \end{center}
            
               \section[Arthur Schnitzler an Adalbert Seligmann, 15. 10. 1905]{ Arthur Schnitzler an Adalbert Seligmann,
                    15. 10. 1905}\nopagebreak\mylabel{v}\rehead{ }\begin{ledgroupsized}[t]{13cm}\normalsize\beginnumbering\briefempfaengerindex{Seligmann, Adalbert Franz@\textsc{Seligmann, Adalbert Franz}!zzzSchnitzler, Arthur@\emph{von Arthur Schnitzler}!1905-10-151@{15. 10. 1905}|(be} \toendnotes[C]{\smallbreak\pagebreak[2]} \Standort{Wienbibliothek im Rathaus, H.I.N.-95601.}
\physDesc{Briefkarte
\newline{}Handschrift: schwarze Tinte, deutsche Kurrent}\pstart
           \noindent{}{\pb}\textcolor{gray}{\textbf{Dr. Arthur Schnitzler}}\hfill 1\substVorne{}\textsuperscript{3}\substDazwischen{}5\substHinten{}. 10. 905\pend
           \pstart
           \textcolor{gray}{\textbf{Wien XVIII.
                                Spoettelgasse 7\oindex{Edmund-Weiss-Gasse@\textbf{Edmund-Weiß-Gasse}|pw}.}}\pend
           \pstart
           verehrteſter Herr Seligmann, über Ihren Brief habe ich mich ſehr
                    gefreut. Ich danke Ihnen herzlich!\pend
           \pstart
           Ihr aufrichtg ergebner{\\[\baselineskip]}\spacefill\mbox{Arth Schnitzler}\pend
           \leftskip=0em{}\endnumbering\briefempfaengerindex{Seligmann, Adalbert Franz@\textsc{Seligmann, Adalbert Franz}!zzzSchnitzler, Arthur@\emph{von Arthur Schnitzler}!1905-10-151@{15. 10. 1905}|)be}\mylabel{h}\end{ledgroupsized}  \newcommand{\dateiname}{L01564}\newcommand{\titel}{Arthur Schnitzler an Adalbert Seligmann, 15. 10. 1905}\newcommand{\editorInnen}{Martin Anton Müller und Gerd-Hermann Susen}\input{../tex-inputs/latex-pdf-abspann}
      