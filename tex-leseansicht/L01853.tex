%% latex-leseansicht-vorspann.tex
%% Vorspann für die Leseansicht.
%% Lädt die gemeinsame Datei latex-vorspann.tex mit nicht gesetztem Schalter.

\newif\ifkorrekturansicht
\korrekturansichtfalse

\input{../tex-inputs/latex-vorspann}


         
         \renewcommand{\erwaehntePersonen}{Personen: Richard Beer-Hofmann, Paula Beer-Hofmann, Heinrich Schnitzler}
         \renewcommand{\erwaehnteOrte}{Orte: Edlach, Hotel Pichl-Auhof, Pichl am See, Schneeberg, Wien}
         \renewcommand{\erwaehnteWerke}{}
               \section[Arthur und Olga Schnitzler an Richard Beer-Hofmann, 6. 7. 1909]{ Arthur und Olga Schnitzler an Richard Beer-Hofmann, 6. 7. 1909}\nopagebreak\mylabel{v}\rehead{ }\begin{ledgroupsized}[t]{13cm}\normalsize\beginnumbering \toendnotes[C]{\smallbreak\pagebreak[2]} \Standort{YCGL, MSS 31.}
\physDesc{Bildpostkarte, 471 Zeichen
\newline{}Handschrift Arthur Schnitzler: 1) Bleistift, deutsche Kurrent\hspace{1em}2) Bleistift, lateinische Kurrent (\noindent{}Adresse)\hspace{1em}\newline{}Handschrift Olga Schnitzler: Bleistift, lateinische Kurrent
\newline{}Versand: Stempel: »\nobreak{}7 7 09, 8–12V\nobreak{}«.  
\newline{}Beer-Hofmann: mit blauem Buntstift das Datum der Beantwortung
                                    festgehalten: »B 12/VII 09« }\buchAbdrucke{\weitereDrucke{Arthur Schnitzler, Richard Beer-Hofmann: \emph{Briefwechsel 1891–1931}. Hg. Konstanze Fliedl. Wien, Zürich: \emph{Europaverlag} 1992, S. 193.} }\toendnotes[C]{\smallbreak}\pstart{}{\pb}Herrn Dr. Richard\pend{}\pstart{}Beer-Hofmann\pend{}\pstart{}Pichl Auhof{\\}am Mondsee\oindex{Hotel Pichl-Auhof@\textbf{Hotel Pichl-Auhof}|pw}\pend{}{\bigskip}\pstart
           \noindent{}\centering{}{\pb}\textcolor{gray}{\textbf{Edlach bei Reichenau in N.-Oe.\oindex{Edlach@\textbf{Edlach}|pw}, 593 m
                     Seehöhe mit Schneeberg\oindex{Schneeberg@\textbf{Schneeberg}|pw}.}}\pend
           \pstart
           {\pb}lieber Richard, ſehr ſchön hier – und wir würden uns ganz wohl
               fühlen, we{\geminationn} wir den Buben\pwindex{Schnitzler, Heinrich 09.08.1902 – 12.07.1982@\textsc{Schnitzler, Heinrich} (09.08.1902 – 12.07.1982), \emph{Regisseur, Schauspieler}|pwv}{ }ſchon heraußen hätten (den ich heute u geſtern in
                  Wien\oindex{Wien@\textbf{Wien}|pw} beſucht habe).\pend
           \pstart
           Wie behagen Sie ſich? Ko{\geminationm}en Sie doch nachher auch hier
               her, ich glaube es gefiel Ihnen Allen.\pend
           \pstart Herzlichſt Ihr \spacefill\mbox{Arthur}\pend{}\pstart
           6/7 09\pend
           \pstart
           \noindent{}{[}hs. Olga Schnitzler:{]} Ja, das wäre wunderschön! An die Paula\pwindex{Beer-Hofmann, Paula 25.02.1879 – 30.10.1939@\textsc{Beer-Hofmann, Paula} (25.02.1879 – 30.10.1939)|pw} will ich nächstens einen längern Brief schreiben.
               Hoffentlich geht es Euch allen sehr gut.\pend
           \pstart Herzliche Grüsse! \spacefill\mbox{Olga.}\pend{}
         
         \endnumbering\mylabel{h}\end{ledgroupsized}  \newcommand{\dateiname}{L01853}\newcommand{\titel}{Arthur und Olga Schnitzler an Richard Beer-Hofmann, 6. 7. 1909}\newcommand{\editorInnen}{Martin Anton Müller und Gerd-Hermann Susen}%% latex-leseansicht-abspann.tex
%% Abspann für die Leseansicht.
%% Der Schalter \ifkorrekturansicht ist bereits durch den Vorspann gesetzt.

%% latex-abspann.tex
%% Gemeinsamer Abspann für Korrekturansicht und Leseansicht.
%% Setzt den Schalter \ifkorrekturansicht voraus (gesetzt in den
%% einbindenden Dateien latex-korrekturansicht-abspann.tex bzw.
%% latex-leseansicht-abspann.tex).
%% ---------------------------------------------------------------

\normalsize

% Das esempio-Environment wird nur in der Leseansicht benötigt
\ifkorrekturansicht\else
\newenvironment{esempio}[3]%
{
    \vspace{1.5ex}
    \rlap{\underline{#1}}
    \par
    \setlength{\parindent}{0cm}
    \nopagebreak
    \leftskip=#2cm
    \rightskip=#3cm
}
{
    \par
}
\fi

\doendnotes{C}
\bigskip
\vfill

\clearpage

\footnotesize

\ifkorrekturansicht
  \lohead{\textsc{register}}
\fi

% theindex-Environment neu definieren ohne reledmac
\makeatletter
\renewenvironment{theindex}{%
  \ifkorrekturansicht
    \section*{\indexname}%
  \else
    \subsubsection*{Index der erwähnten Entitäten}%
  \fi
  \setlength{\parindent}{0pt}%
  \setlength{\parskip}{0pt plus 0.3pt}%
  \let\item\@idxitem
}{%
  \ifkorrekturansicht\clearpage\fi
}
\makeatother

\IfFileExists{\jobname-pw.ind}{\input{\jobname-pw.ind}}{}

% Quellenangabe nur in der Leseansicht
\ifkorrekturansicht\else
% Fallback-Definitionen, falls die .tex-Datei \titel etc. nicht gesetzt hat
\providecommand{\titel}{}
\providecommand{\editorInnen}{}
\providecommand{\dateiname}{\jobname}

\vspace{3cm}

\vfill

\footnotesize
\textsc{Quelle}: \titel. Herausgegeben von {\editorInnen}. In: \emph{Arthur Schnitzler: Briefwechsel mit Autorinnen und Autoren}.
 Digitale Edition, https://schnitzler-briefe.acdh.oeaw.ac.at/{\dateiname}.html (Stand \today)
\fi

\end{document}


      