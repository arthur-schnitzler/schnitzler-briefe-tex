\input{../tex-inputs/latex-pdf-vorspann}
\begin{center}
            \textcolor{red}{ENTWURF. ENTZIFFERUNG NOCH NICHT KORREKTURGELESEN}
                      \end{center}
            
               \section[Arthur und Olga Schnitzler an Richard Beer-Hofmann, 6. 7. 1909]{ Arthur und Olga Schnitzler an Richard Beer-Hofmann, 6. 7. 1909}\nopagebreak\mylabel{v}\rehead{ }\begin{ledgroupsized}[t]{13cm}\normalsize\beginnumbering\briefempfaengerindex{Beer-Hofmann, Richard@\textsc{Beer-Hofmann, Richard}!zzzSchnitzler, Olga@\emph{von Olga Schnitzler}!1909-07-061@{6. 7. 1909}|(be}\briefempfaengerindex{Beer-Hofmann, Richard@\textsc{Beer-Hofmann, Richard}!zzzSchnitzler, Arthur@\emph{von Arthur Schnitzler}!1909-07-061@{6. 7. 1909}|(be} \toendnotes[C]{\smallbreak\pagebreak[2]} \Standort{YCGL, MSS 31.}
\physDesc{Bildpostkarte
\newline{}Handschrift Arthur Schnitzler: Bleistift, deutsche Kurrent\newline{}Handschrift Olga Schnitzler: Bleistift, lateinische Kurrent\newline{}Versand: Stempel: »\nobreak{}7 7 09, 8–12V\nobreak{}«.  
\newline{}Beer-Hofmann: mit blauem Bunstift das Datum der Beantwortung
                                    festgehalten: »B 12/VII 09« }\buchAbdrucke{\weitereDrucke{Arthur Schnitzler, Richard Beer-Hofmann: \emph{Briefwechsel 1891–1931}. Hg. Konstanze Fliedl. Wien, Zürich: \emph{Europaverlag} 1992, S. 193.} }\toendnotes[C]{\smallbreak}\pstart{}{\pb}\textsc{Herrn Dr. Richard}\pend{}\pstart{}\textsc{Beer-Hofmann}\pend{}\pstart{}\textsc{Pichl Auhof{\\}am Mondsee\oindex{Hotel Pichl-Auhof@\textbf{Hotel Pichl-Auhof}|pw}}\pend{}{\bigskip}\pstart
           \noindent{}\centering{}{\pb}\textcolor{gray}{\textbf{Edlach bei Reichenau in N.-Oe.\oindex{Edlach@\textbf{Edlach}|pw}, 593 m Seehöhe
                     mit Schneeberg\oindex{Schneeberg@\textbf{Schneeberg}|pw}.}}\pend
           \pstart
           {\pb}lieber Richard, ſehr ſchön hier – und wir würden uns ganz wohl
               fühlen, we{\geminationn} wir den Buben\pwindex{Schnitzler, Heinrich 09.08.1902 – 12.07.1982@\textsc{Schnitzler, Heinrich} (09.08.1902 – 12.07.1982), \emph{Regisseur, Schauspieler}|pwv}{ }ſchon heraußen hätten (den ich heute u geſtern in
                  Wien\oindex{Wien@\textbf{Wien}|pw} beſucht habe).\pend
           \pstart
           Wie behagen Sie ſich? Ko{\geminationm}en Sie doch nachher auch hier
               her, ich glaube es gefiel Ihnen Allen.\pend
           \pstart Herzlichſt Ihr \spacefill\mbox{Arthur}\pend{}\pstart
           6/7 09\pend
           \pstart
           \noindent{}{[}hs. O. Schnitzler:{]} Ja, das wäre wunderschön! An die Paula\pwindex{Beer-Hofmann, Paula 25.02.1879 – 30.10.1939@\textsc{Beer-Hofmann, Paula} (25.02.1879 – 30.10.1939)|pw} will ich nächstens einen längern Brief schreiben.
               Hoffentlich geht es Euch allen sehr gut.\pend
           \pstart Herzliche Grüsse! \spacefill\mbox{Olga.}\pend{}\endnumbering\briefempfaengerindex{Beer-Hofmann, Richard@\textsc{Beer-Hofmann, Richard}!zzzSchnitzler, Olga@\emph{von Olga Schnitzler}!1909-07-061@{6. 7. 1909}|)be}\briefempfaengerindex{Beer-Hofmann, Richard@\textsc{Beer-Hofmann, Richard}!zzzSchnitzler, Arthur@\emph{von Arthur Schnitzler}!1909-07-061@{6. 7. 1909}|)be}\mylabel{h}\end{ledgroupsized}  \newcommand{\dateiname}{L01853}\newcommand{\titel}{Arthur und Olga Schnitzler an Richard Beer-Hofmann, 6. 7. 1909}\newcommand{\editorInnen}{Martin Anton Müller und Gerd-Hermann Susen}\input{../tex-inputs/latex-pdf-abspann}
      