%% latex-korrekturansicht-vorspann.tex
%% Vorspann für die Korrekturansicht.
%% Lädt die gemeinsame Datei latex-vorspann.tex mit gesetztem Schalter.

\newif\ifkorrekturansicht
\korrekturansichttrue

\input{../tex-inputs/latex-vorspann}


\section[Arthur und Olga Schnitzler an Richard Beer-Hofmann, 6. 7. 1909]{L01853 Arthur und Olga Schnitzler an Richard Beer-Hofmann, 6. 7. 1909}
\nopagebreak\mylabel{L01853v}
\rehead{ }\normalsize\beginnumbering\briefempfaengerindex{Beer-Hofmann, Richard@\textsc{Beer-Hofmann, Richard}!zzzSchnitzler, Olga@\emph{von Olga Schnitzler}!1909-07-061@{6. 7. 1909}|(be}\briefempfaengerindex{Beer-Hofmann, Richard@\textsc{Beer-Hofmann, Richard}!zzzSchnitzler, Arthur@\emph{von Arthur Schnitzler}!1909-07-061@{6. 7. 1909}|(be}
\toendnotes[C]{\smallbreak\pagebreak[2]}\Standort{YCGL, MSS 31.}
\physDesc{Bildpostkarte, 471 Zeichen
\newline{}Handschrift Arthur Schnitzler: 1) Bleistift, deutsche Kurrent\hspace{1em}2) Bleistift, lateinische Kurrent (\noindent{}Adresse)\hspace{1em}
\newline{}Handschrift Olga Schnitzler: Bleistift, lateinische Kurrent
\newline{}Versand: Stempel: »\nobreak{}7 7 09, 8–12V\nobreak{}«.  
\newline{}Beer-Hofmann: mit blauem Buntstift das Datum der Beantwortung
                                    festgehalten: »B 12/VII 09« }
\buchAbdrucke{\weitereDrucke{Arthur Schnitzler, Richard Beer-Hofmann: \emph{Briefwechsel 1891–1931}. Wien, Zürich: \emph{Europaverlag} 1992, S. 193.} }\toendnotes[C]{\smallbreak}\pstart{}{\pb}Herrn Dr. Richard\pend{}\pstart{}Beer-Hofmann\pend{}\pstart{}Pichl Auhof{\\}am Mondsee\oindex{Hotel Pichl-Auhof@\textbf{Hotel Pichl-Auhof}, \emph{Beherbergungsgebäude (K.BHB)}|pw}\pend{}{\bigskip}
\pstart
           \noindent{}\centering{}{\pb}\textcolor{gray}{\textbf{Edlach bei Reichenau in N.-Oe.\oindex{Edlach@\textbf{Edlach}, \emph{P.PPL}|pw}, 593 m
                  Seehöhe mit Schneeberg\oindex{Schneeberg@\textbf{Schneeberg}, \emph{Berg (N.BRG)}|pw}.}}\pend
           \vspace{1em}
\pstart
           \noindent{}{\pb}lieber Richard, ſehr ſchön hier – und wir würden uns ganz wohl
               fühlen, we{\geminationn} wir den Buben\pwindex{Schnitzler, Heinrich 09.08.1902 – 12.07.1982@\textsc{Schnitzler, Heinrich} (09.08.1902 – 12.07.1982), \emph{Regisseur/Regisseurin, Schauspieler/Schauspielerin}|pwv}{ }ſchon heraußen hätten (den ich heute u geſtern in
                  Wien\oindex{Wien@\textbf{Wien}, \emph{A.ADM2}|pw} beſucht habe).\pend
           
\pstart
           Wie behagen Sie ſich? Ko{\geminationm}en Sie doch nachher auch hier
               her, ich glaube es gefiel Ihnen Allen.\pend
           \pstart Herzlichſt Ihr \spacefill\mbox{Arthur}\pend{}
\pstart
           6/7 09\pend
           \selectlanguage{ngerman}\vspace{1em}
\pstart
           \noindent{}{[}hs. :{]} Ja, das wäre wunderschön! An die Paula\pwindex{Beer-Hofmann, Paula 25.02.1879 – 30.10.1939@\textsc{Beer-Hofmann, Paula} (25.02.1879 – 30.10.1939)|pw} will ich nächstens einen längern Brief schreiben.
               Hoffentlich geht es Euch allen sehr gut.\pend
           \pstart Herzliche Grüsse! \spacefill\mbox{Olga.}\pend{}\selectlanguage{ngerman}\endnumbering\briefempfaengerindex{Beer-Hofmann, Richard@\textsc{Beer-Hofmann, Richard}!zzzSchnitzler, Olga@\emph{von Olga Schnitzler}!1909-07-061@{6. 7. 1909}|)be}\briefempfaengerindex{Beer-Hofmann, Richard@\textsc{Beer-Hofmann, Richard}!zzzSchnitzler, Arthur@\emph{von Arthur Schnitzler}!1909-07-061@{6. 7. 1909}|)be}\mylabel{L01853h}  \normalsize

\doendnotes{C}
\bigskip
\vfill

\clearpage

\footnotesize

\lohead{\textsc{register}}

% Definiere theindex-Environment komplett neu ohne reledmac
\makeatletter
\renewenvironment{theindex}{%
  \section*{\indexname}%
  \setlength{\parindent}{0pt}%
  \setlength{\parskip}{0pt plus 0.3pt}%
  \let\item\@idxitem
}{%
  \clearpage
}
\makeatother

\IfFileExists{\jobname-pw.ind}{\input{\jobname-pw.ind}}{}

\end{document}

      