%% latex-leseansicht-vorspann.tex
%% Vorspann für die Leseansicht.
%% Lädt die gemeinsame Datei latex-vorspann.tex mit nicht gesetztem Schalter.

\newif\ifkorrekturansicht
\korrekturansichtfalse

\input{../tex-inputs/latex-vorspann}


         
         \renewcommand{\erwaehntePersonen}{Personen: Paul Goldmann}
         \renewcommand{\erwaehnteInstitutionen}{Institutionen: Deutsches Theater Berlin}
         \renewcommand{\erwaehnteOrte}{Orte: Berlin, Brühl, Carl-Theater, Dessauer Straße, Wien}
         \renewcommand{\erwaehnteWerke}{Werke: Lebendige Stunden. Vier Einakter, Tagebuch}
               \section[ Paul Goldmann an Arthur Schnitzler, 12. 5. {[}1902{]}]{ Paul Goldmann an Arthur Schnitzler, 12. 5. {[}1902{]}}\nopagebreak\mylabel{v}\rehead{ }\begin{ledgroupsized}[t]{13cm}\normalsize\beginnumbering\briefempfaengerindex{Schnitzler, Arthur@\textsc{Schnitzler, Arthur}!zzzGoldmann, Paul@\emph{von Paul Goldmann}!1902-05-121@{12. 5. {[}1902{]}}|(be} \toendnotes[C]{\smallbreak\pagebreak[2]} \Standort{DLA, A:Schnitzler, HS.NZ85.1.3172.}
\physDesc{Brief, 1 Blatt, 1 Seite, 442 Zeichen
\newline{}Handschrift: blaue Tinte, deutsche Kurrent
\newline{}Schnitzler: mit Bleistift das Jahr »1902« vermerkt }\toendnotes[C]{\smallbreak}\pstart
           \noindent{}\raggedleft{}{\pb}\textcolor{gray}{\textbf{DESSAUERSTRASSE 19}}\oindex{Dessauer Strasse@\textbf{Dessauer Straße}|pw}\pend
           \pstart
           Berlin\oindex{Berlin@\textbf{Berlin}|pw}, 12. Mai.\pend
           \pstart\center{}Mein lieber Freund,\pend\pstart
           Ich warte vergeblich auf Deine Antwort: Biſt Du \label{K_L03208-1v}\edtext{Pfingſten in Wien\oindex{Wien@\textbf{Wien}|pw}}{\lemma{\textnormal{\emph{Pfingſten in Wien}}}\Cendnote{\textnormal{Siehe Paul Goldmann an Arthur Schnitzler, 5. 5. [1902].
               }}}\label{K_L03208-1h}? Oder wohnſt Du in der Brühl\oindex{Bruehl@\textbf{Brühl}|pw}? Ich weiß
               noch nicht, ob ich fahren werde. Wenn ja, ſo dürfte ich \label{K_L03208-2v}\edtext{Samſtag{ }Abend}{\lemma{\textnormal{\emph{Samſtag Abend}}}\Cendnote{\textnormal{Das \emph{Tagebuch}\pwindex{\textcolor{red}{\textsuperscript{XXXX1 indx}}!Tagebuch1981 – 2000@\strich\emph{Tagebuch} {[}Hrsg., 1981 – 2000{]}|pwk} vermerkt Goldmanns\pwindex{Goldmann, Paul 31.01.1865 – 25.09.1935@\textsc{Goldmann, Paul} (31.01.1865 – 25.09.1935), \emph{Schriftsteller, Journalist}|pwk} Ankunft in Wien\oindex{Wien@\textbf{Wien}|pwk} erst für den Folgetag, Sonntag,
                  den 18. 5. 1902,
                  doch könnte er bereits am 17. eingetroffen sein, vgl. Paul Goldmann an Arthur Schnitzler, 16. 5. [1902].}}}\label{K_L03208-2h} in Wien\oindex{Wien@\textbf{Wien}|pw} eintreffen. Biſt Du dann in der Stadt\oindex{Wien@\textbf{Wien}|pwv}? Selbſtverſtändlich darfſt
               Du Dich in Deinen Dispoſitionen durch mich in keiner Weiſe ſtören laſſen. \strikeout{\textcolor{gray}{Ic}} Ich beglückwünſche Dich herzlichſt zu Deinem \label{K_L03208-3v}\edtext{Wie\oindex{Wien@\textbf{Wien}|pw}ner Erfolge\pwindex{Schnitzler, Arthur 15.05.1862 – 21.10.1931@\textsc{Schnitzler, Arthur} (15.05.1862 – 21.10.1931), \emph{Schriftsteller, Mediziner}!Lebendige Stunden. Vier Einakter1901-12-23@\strich\emph{Lebendige Stunden. Vier Einakter} {[}1901-12-23{]}|pwv}}{\lemma{\textnormal{\emph{Wiener Erfolge}}}\Cendnote{\textnormal{Am 6. 5. 1902 hatte die erfolgreiche Premiere des
                  Gastspiels von \emph{Lebendige Stunden}\pwindex{Schnitzler, Arthur 15.05.1862 – 21.10.1931@\textsc{Schnitzler, Arthur} (15.05.1862 – 21.10.1931), \emph{Schriftsteller, Mediziner}!Lebendige Stunden. Vier Einakter1901-12-23@\strich\emph{Lebendige Stunden. Vier Einakter} {[}1901-12-23{]}|pwk} des \emph{Deutschen Theaters Berlin}\orgindex{Deutsches Theater Berlin@Deutsches Theater Berlin|pwk} am Wien\oindex{Wien@\textbf{Wien}|pwk}er Carl-Theater\oindex{Carl-Theater@\textbf{Carl-Theater}|pwk}
                  stattgefunden. Auch die Kritiken fielen gut aus (vgl. A. S.: \emph{Tagebuch}, 7. 5. 1902).}}}\label{K_L03208-3h}. Viele treue Grüße!\pend
           \pstart
           Dein {\\[\baselineskip]}\spacefill\mbox{Paul Goldmann}\pend
           \leftskip=0em{}
         
         \endnumbering\mylabel{h}\end{ledgroupsized}  \newcommand{\dateiname}{L03208}\newcommand{\titel}{Paul Goldmann an Arthur Schnitzler, 12. 5. [1902]}\newcommand{\editorInnen}{Martin Anton Müller und Laura Untner}%% latex-leseansicht-abspann.tex
%% Abspann für die Leseansicht.
%% Der Schalter \ifkorrekturansicht ist bereits durch den Vorspann gesetzt.

%% latex-abspann.tex
%% Gemeinsamer Abspann für Korrekturansicht und Leseansicht.
%% Setzt den Schalter \ifkorrekturansicht voraus (gesetzt in den
%% einbindenden Dateien latex-korrekturansicht-abspann.tex bzw.
%% latex-leseansicht-abspann.tex).
%% ---------------------------------------------------------------

\normalsize

% Das esempio-Environment wird nur in der Leseansicht benötigt
\ifkorrekturansicht\else
\newenvironment{esempio}[3]%
{
    \vspace{1.5ex}
    \rlap{\underline{#1}}
    \par
    \setlength{\parindent}{0cm}
    \nopagebreak
    \leftskip=#2cm
    \rightskip=#3cm
}
{
    \par
}
\fi

\doendnotes{C}
\bigskip
\vfill

\clearpage

\footnotesize

\ifkorrekturansicht
  \lohead{\textsc{register}}
\fi

% theindex-Environment neu definieren ohne reledmac
\makeatletter
\renewenvironment{theindex}{%
  \ifkorrekturansicht
    \section*{\indexname}%
  \else
    \subsubsection*{Index der erwähnten Entitäten}%
  \fi
  \setlength{\parindent}{0pt}%
  \setlength{\parskip}{0pt plus 0.3pt}%
  \let\item\@idxitem
}{%
  \ifkorrekturansicht\clearpage\fi
}
\makeatother

\IfFileExists{\jobname-pw.ind}{\input{\jobname-pw.ind}}{}

% Quellenangabe nur in der Leseansicht
\ifkorrekturansicht\else
% Fallback-Definitionen, falls die .tex-Datei \titel etc. nicht gesetzt hat
\providecommand{\titel}{}
\providecommand{\editorInnen}{}
\providecommand{\dateiname}{\jobname}

\vspace{3cm}

\vfill

\footnotesize
\textsc{Quelle}: \titel. Herausgegeben von {\editorInnen}. In: \emph{Arthur Schnitzler: Briefwechsel mit Autorinnen und Autoren}.
 Digitale Edition, https://schnitzler-briefe.acdh.oeaw.ac.at/{\dateiname}.html (Stand \today)
\fi

\end{document}


      