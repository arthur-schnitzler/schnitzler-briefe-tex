%% latex-korrekturansicht-vorspann.tex
%% Vorspann für die Korrekturansicht.
%% Lädt die gemeinsame Datei latex-vorspann.tex mit gesetztem Schalter.

\newif\ifkorrekturansicht
\korrekturansichttrue

\input{../tex-inputs/latex-vorspann}


\section[ Paul Goldmann an Arthur Schnitzler, 12. 5. {[}1902{]}]{L03208 Paul Goldmann an Arthur Schnitzler, 12. 5. {[}1902{]}}
\nopagebreak\mylabel{L03208v}
\rehead{ }\normalsize\beginnumbering\briefempfaengerindex{Schnitzler, Arthur@\textsc{Schnitzler, Arthur}!zzzGoldmann, Paul@\emph{von Paul Goldmann}!1902-05-121@{12. 5. {[}1902{]}}|(be}
\toendnotes[C]{\smallbreak\pagebreak[2]}\Standort{DLA, A:Schnitzler, HS.NZ85.1.3172.}
\physDesc{Brief, 1 Blatt, 1 Seite, 442 Zeichen
\newline{}Handschrift: blaue Tinte, deutsche Kurrent
\newline{}Schnitzler: mit Bleistift das Jahr »1902« vermerkt }\toendnotes[C]{\smallbreak}
\pstart
           \raggedleft{}{\pb}\textcolor{gray}{\textbf{DESSAUERSTRASSE 19}}\oindex{Dessauer Strasse@\textbf{Dessauer Straße}, \emph{Straße (K.STR)}|pw}\pend
           
\pstart
           Berlin\oindex{Berlin@\textbf{Berlin}, \emph{P.PPLC}|pw}, 12. Mai.\pend
           
\pstart\center{}Mein lieber Freund,\pend\vspace{0.5em}
\pstart
           Ich warte vergeblich auf Deine Antwort: Biſt Du \label{K_L03208-1v}\edtext{Pfingſten in Wien\oindex{Wien@\textbf{Wien}, \emph{A.ADM2}|pw}}{\lemma{\textnormal{\emph{Pfingſten in Wien}}}\Cendnote{\textnormal{Siehe Paul Goldmann an Arthur Schnitzler, 5. 5. [1902].
               }}}\label{K_L03208-1}? Oder wohnſt Du in der Brühl\oindex{Bruehl@\textbf{Brühl}, \emph{Tal (N.TAL)}|pw}? Ich weiß
               noch nicht, ob ich fahren werde. Wenn ja, ſo dürfte ich \label{K_L03208-2v}\edtext{Samſtag{ }Abend}{\lemma{\textnormal{\emph{Samſtag Abend}}}\Cendnote{\textnormal{Das \emph{Tagebuch}\pwindex{Tagebuch@\emph{Tagebuch}|pwk} vermerkt Goldmanns\pwindex{Goldmann, Paul 31.01.1865 – 25.09.1935@\textsc{Goldmann, Paul} (31.01.1865 – 25.09.1935), \emph{Schriftsteller/Schriftstellerin, Journalist/Journalistin}|pwk} Ankunft in Wien\oindex{Wien@\textbf{Wien}, \emph{A.ADM2}|pwk} erst für den Folgetag, Sonntag,
                  den 18. 5. 1902,
                  doch könnte er bereits am 17. eingetroffen sein, vgl. Paul Goldmann an Arthur Schnitzler, 16. 5. [1902].}}}\label{K_L03208-2} in Wien\oindex{Wien@\textbf{Wien}, \emph{A.ADM2}|pw} eintreffen. Biſt Du dann in der Stadt\oindex{Wien@\textbf{Wien}, \emph{A.ADM2}|pwv}? Selbſtverſtändlich darfſt
               Du Dich in Deinen Dispoſitionen durch mich in keiner Weiſe ſtören laſſen. \strikeout{\textcolor{gray}{Ic}} Ich beglückwünſche Dich herzlichſt zu Deinem \label{K_L03208-3v}\edtext{Wie\oindex{Wien@\textbf{Wien}, \emph{A.ADM2}|pw}ner Erfolge\pwindex{Lebendige Stunden. Vier Einakter@\emph{Lebendige Stunden. Vier Einakter}|pwv}}{\lemma{\textnormal{\emph{Wiener Erfolge}}}\Cendnote{\textnormal{Am 6. 5. 1902 hatte die erfolgreiche Premiere des
                  Gastspiels von \emph{Lebendige Stunden}\pwindex{Lebendige Stunden. Vier Einakter@\emph{Lebendige Stunden. Vier Einakter}|pwk} des \emph{Deutschen Theaters Berlin}\orgindex{Deutsches Theater Berlin@Deutsches Theater Berlin|pwk} am Wien\oindex{Wien@\textbf{Wien}, \emph{A.ADM2}|pwk}er Carl-Theater\oindex{Carl-Theater@\textbf{Carl-Theater}, \emph{Theater (K.THE)}|pwk}
                  stattgefunden. Auch die Kritiken fielen gut aus (vgl. A. S.: \emph{Tagebuch}, 7. 5. 1902).}}}\label{K_L03208-3}. Viele treue Grüße!\pend
           
\pstart
           Dein {\\[\baselineskip]}\spacefill\mbox{Paul Goldmann}\pend
           \leftskip=0em{}\selectlanguage{ngerman}\endnumbering\briefempfaengerindex{Schnitzler, Arthur@\textsc{Schnitzler, Arthur}!zzzGoldmann, Paul@\emph{von Paul Goldmann}!1902-05-121@{12. 5. {[}1902{]}}|)be}\mylabel{L03208h}  \normalsize

\doendnotes{C}
\bigskip
\vfill

\clearpage

\footnotesize

\lohead{\textsc{register}}

% Definiere theindex-Environment komplett neu ohne reledmac
\makeatletter
\renewenvironment{theindex}{%
  \section*{\indexname}%
  \setlength{\parindent}{0pt}%
  \setlength{\parskip}{0pt plus 0.3pt}%
  \let\item\@idxitem
}{%
  \clearpage
}
\makeatother

\IfFileExists{\jobname-pw.ind}{\input{\jobname-pw.ind}}{}

\end{document}

      