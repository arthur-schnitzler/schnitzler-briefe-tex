%% latex-korrekturansicht-vorspann.tex
%% Vorspann für die Korrekturansicht.
%% Lädt die gemeinsame Datei latex-vorspann.tex mit gesetztem Schalter.

\newif\ifkorrekturansicht
\korrekturansichttrue

\input{../tex-inputs/latex-vorspann}


\section[Arthur Schnitzler an Richard Beer-Hofmann, 1{[}2?{]}. 7. 1901]{L01146 Arthur Schnitzler an Richard Beer-Hofmann, 1{[}2?{]}. 7. 1901}
\nopagebreak\mylabel{L01146v}
\rehead{ }\normalsize\beginnumbering\briefempfaengerindex{Beer-Hofmann, Richard@\textsc{Beer-Hofmann, Richard}!zzzSchnitzler, Arthur@\emph{von Arthur Schnitzler}!1901-07-121@{1{[}2?{]}. 7. 1901}|(be}
\toendnotes[C]{\smallbreak\pagebreak[2]}\Standort{YCGL, MSS 31.}
\physDesc{Bildpostkarte, 110 Zeichen
\newline{}Handschrift: Bleistift, deutsche Kurrent
\newline{}Versand: 1) Stempel: »\nobreak{}\oindex{Innsbruck@\textbf{Innsbruck}, \emph{A.ADM2}|pwk}I{[}nnbru{]}\textcolor{gray}{ck}, \textcolor{gray}{13}. {[}7. 1901{]}\nobreak{}«.   2) Stempel: »\nobreak{}\oindex{Poertschach am Woerthersee@\textbf{Pörtschach am Wörthersee}, \emph{P.PPL}|pwk}Pörtschach am
                                          {[}See{]}, 14/7 01\nobreak{}«. 
\newline{}Ordnung: mit Bleistift von unbekannter Hand datiert: »14. 7.« }\toendnotes[C]{\smallbreak}\pstart{}{\pb}\strikeout{N.Oe.\oindex{Niederoesterreich@\textbf{Niederösterreich}, \emph{A.ADM1}|pw}}\pend{}\pstart{}Hrn Dr. \textsc{Rich Beer-Hofmann}\pend{}\pstart{}\textsc{Pörtschach}\oindex{Poertschach am Woerthersee@\textbf{Pörtschach am Wörthersee}, \emph{P.PPL}|pw}\pend{}\pstart{}\textsc{Villa Arnstein\oindex{Villa Arnstein@\textbf{Villa Arnstein}, \emph{Wohngebäude (K.WHS)}|pw}}\pend{}{\bigskip}
\pstart
           \noindent{}{\pb}\textcolor{gray}{\textbf{INNSBRUCK\oindex{Innsbruck@\textbf{Innsbruck}, \emph{A.ADM2}|pw} VON NORD.}}\pend
           \vspace{1em}
\pstart
           \noindent{}{\pb}Herzlichen Dank für das \label{K_L01146-1v}\edtext{Telegramm}{\lemma{\textnormal{\emph{Telegramm}}}\Cendnote{\textnormal{nicht
                  überliefert}}}\label{K_L01146-1}. \label{K_L01146-2v}\edtext{Morgen}{\lemma{\textnormal{\emph{Morgen}}}\Cendnote{\textnormal{Das erlaubt die Datierung noch vor dem
                  Poststempel, der, unsicher gelesen, vom 13. 7. 1901 stammen dürfte,
                  da Schnitzler bereits an diesem Tag in Vahrn\oindex{Vahrn@\textbf{Vahrn}, \emph{P.PPLA3}|pwk} ankam.}}}\label{K_L01146-2} fahr ich hin.\pend
           \pstart Ihr \spacefill\mbox{A.}\pend{}\selectlanguage{ngerman}\endnumbering\briefempfaengerindex{Beer-Hofmann, Richard@\textsc{Beer-Hofmann, Richard}!zzzSchnitzler, Arthur@\emph{von Arthur Schnitzler}!1901-07-121@{1{[}2?{]}. 7. 1901}|)be}\mylabel{L01146h}  \normalsize

\doendnotes{C}
\bigskip
\vfill

\clearpage

\footnotesize

\lohead{\textsc{register}}

% Definiere theindex-Environment komplett neu ohne reledmac
\makeatletter
\renewenvironment{theindex}{%
  \section*{\indexname}%
  \setlength{\parindent}{0pt}%
  \setlength{\parskip}{0pt plus 0.3pt}%
  \let\item\@idxitem
}{%
  \clearpage
}
\makeatother

\IfFileExists{\jobname-pw.ind}{\input{\jobname-pw.ind}}{}

\end{document}

      