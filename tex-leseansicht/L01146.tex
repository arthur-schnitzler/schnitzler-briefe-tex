\input{../tex-inputs/latex-pdf-vorspann}
\begin{center}
            \textcolor{red}{ENTWURF. ENTZIFFERUNG NOCH NICHT KORREKTURGELESEN}
                      \end{center}
            
               \section[Arthur Schnitzler an Richard Beer-Hofmann, 1{[}2?{]}. 7. 1901]{ Arthur Schnitzler an Richard Beer-Hofmann, 1{[}2?{]}. 7. 1901}\nopagebreak\mylabel{v}\rehead{ }\begin{ledgroupsized}[t]{13cm}\normalsize\beginnumbering\briefempfaengerindex{Beer-Hofmann, Richard@\textsc{Beer-Hofmann, Richard}!zzzSchnitzler, Arthur@\emph{von Arthur Schnitzler}!1901-07-121@{1{[}2?{]}. 7. 1901}|(be} \toendnotes[C]{\smallbreak\pagebreak[2]} \Standort{YCGL, MSS 31.}
\physDesc{Bildpostkarte
\newline{}Handschrift: Bleistift, deutsche Kurrent\newline{}Versand: 1) Stempel: »\nobreak{}\oindex{Innsbruck@\textbf{Innsbruck}|pwk}I{[}nnbru{]}\textcolor{gray}{ck}, \textcolor{gray}{13}. {[}7. 1901{]}\nobreak{}«.  2) Stempel: »\nobreak{}\oindex{Poertschach@\textbf{Pörtschach}|pwk}Pörtschach am {[}See{]}, 14/7 01\nobreak{}«. \newline{}Ordnung: mit Bleistift von unbekannter Hand datiert: »14. 7.« }\toendnotes[C]{\smallbreak}\pstart{}{\pb}\strikeout{N.Oe.\oindex{Niederoesterreich@\textbf{Niederösterreich}|pw}}\pend{}\pstart{}Hrn Dr. \textsc{Rich Beer-Hofmann}\pend{}\pstart{}\textsc{Pörtschach}\oindex{Poertschach@\textbf{Pörtschach}|pw}\pend{}\pstart{}\textsc{Villa Arnstein\oindex{Villa Arnstein@\textbf{Villa Arnstein}|pw}}\pend{}{\bigskip}\pstart
           \noindent{}{\pb}\textcolor{gray}{\textbf{INNSBRUCK\oindex{Innsbruck@\textbf{Innsbruck}|pw} VON NORD.}}\pend
           \pstart
           Herzlichen Dank für das \label{K_L01146-2v}\edtext{Telegramm}{\lemma{\textnormal{\emph{Telegramm}}}\Cendnote{\textnormal{nicht überliefert}}}\label{K_L01146-2h}. \label{K_L01146-1v}\edtext{Morgen}{\lemma{\textnormal{\emph{Morgen}}}\Cendnote{\textnormal{Das erlaubt die Datierung noch vor dem Poststempel, der,
                  unsicher gelesen, vom 13. 7. 1901 stammen dürfte, da Schnitzler\pwindex{Schnitzler, Arthur 15.05.1862 – 21.10.1931@\textsc{Schnitzler, Arthur} (15.05.1862 – 21.10.1931), \emph{Schriftsteller, Mediziner}|pwk} bereits an diesem Tag in Vahrn\oindex{Vahrn@\textbf{Vahrn}|pwk} ankommt.}}}\label{K_L01146-1h} fahr ich hin.\pend
           \pstart Ihr \spacefill\mbox{A.}\pend{}\endnumbering\briefempfaengerindex{Beer-Hofmann, Richard@\textsc{Beer-Hofmann, Richard}!zzzSchnitzler, Arthur@\emph{von Arthur Schnitzler}!1901-07-121@{1{[}2?{]}. 7. 1901}|)be}\mylabel{h}\end{ledgroupsized}  \newcommand{\dateiname}{L01146}\newcommand{\titel}{Arthur Schnitzler an Richard Beer-Hofmann, 1[2?]. 7. 1901}\newcommand{\editorInnen}{Martin Anton Müller und Gerd-Hermann Susen}\input{../tex-inputs/latex-pdf-abspann}
      