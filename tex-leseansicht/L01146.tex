%% latex-leseansicht-vorspann.tex
%% Vorspann für die Leseansicht.
%% Lädt die gemeinsame Datei latex-vorspann.tex mit nicht gesetztem Schalter.

\newif\ifkorrekturansicht
\korrekturansichtfalse

\input{../tex-inputs/latex-vorspann}


\section[Arthur Schnitzler an Richard Beer-Hofmann, 1{[}2?{]}. 7. 1901]{L01146 Arthur Schnitzler an Richard Beer-Hofmann, 1[2?]. 7. 1901}
\nopagebreak\mylabel{L01146v}
\rehead{ }\normalsize\beginnumbering\briefempfaengerindex{Beer-Hofmann, Richard@\textsc{Beer-Hofmann, Richard}!zzzSchnitzler, Arthur@\emph{von Arthur Schnitzler}!1901-07-121@{1[2?]. 7. 1901}|(be}
\toendnotes[C]{\smallbreak\pagebreak[2]}
\correspDesc{Versand  durch Arthur Schnitzler am 1[2?]. 7. 1901 in Innsbruck
\newline{}Übermittlung  am 1[3?]. 7. 1901 in Innsbruck
\newline{}Erhalt  durch Richard Beer-Hofmann am 14. 7. 1901 in Pörtschach}\toendnotes[C]{\smallbreak}
\Standort{YCGL, MSS 31.}
\physDesc{Bildpostkarte, 110 Zeichen
\newline{}Handschrift: Bleistift, deutsche Kurrent
\newline{}Versand: 1) Stempel: »\nobreak{}\oindex{Innsbruck@\textbf{Innsbruck}, \emph{Verwaltungsgebiet}|pwk}I{[}nnbru{]}\textcolor{gray}{ck}, \textcolor{gray}{13}. {[}7. 1901{]}\nobreak{}«.   2) Stempel: »\nobreak{}\oindex{Pörtschach am Wörthersee@\textbf{Pörtschach am Wörthersee}|pwk}Pörtschach am
                                          {[}See{]}, 14/7 01\nobreak{}«. 
\newline{}Ordnung: mit Bleistift von unbekannter Hand datiert: »14. 7.« }\toendnotes[C]{\smallbreak}\pstart{}{\pb}\strikeout{N.Oe.\oindex{Niederösterreich@\textbf{Niederösterreich}, \emph{Land}|pw}}\pend{}\pstart{}Hrn Dr. \textsc{Rich Beer-Hofmann}\pend{}\pstart{}\textsc{Pörtschach}\oindex{Pörtschach am Wörthersee@\textbf{Pörtschach am Wörthersee}|pw}\pend{}\pstart{}\textsc{Villa Arnstein\oindex{Villa Arnstein@\textbf{Villa Arnstein}, \emph{Wohngebäude}|pw}}\pend{}{\bigskip}
\pstart
           \noindent{}{\pb}\textcolor{gray}{\textbf{INNSBRUCK\oindex{Innsbruck@\textbf{Innsbruck}, \emph{Verwaltungsgebiet}|pw} VON NORD.}}\pend
           \vspace{1em}
\pstart
           \noindent{}{\pb}Herzlichen Dank für das \label{K_L01146-1v}\edtext{Telegramm}{\lemma{\textnormal{\emph{Telegramm}}}\Cendnote{\textnormal{nicht
                  überliefert}}}\label{K_L01146-1}. \label{K_L01146-2v}\edtext{Morgen}{\lemma{\textnormal{\emph{Morgen}}}\Cendnote{\textnormal{Das erlaubt die Datierung noch vor dem
                  Poststempel, der, unsicher gelesen, vom 13. 7. 1901 stammen dürfte,
                  da Schnitzler bereits an diesem Tag in Vahrn\oindex{Vahrn@\textbf{Vahrn}, \emph{Hauptstadt}|pwk} ankam.}}}\label{K_L01146-2} fahr ich hin.\pend
           \pstart Ihr \spacefill\mbox{A.}\pend{}\selectlanguage{ngerman}\endnumbering\briefempfaengerindex{Beer-Hofmann, Richard@\textsc{Beer-Hofmann, Richard}!zzzSchnitzler, Arthur@\emph{von Arthur Schnitzler}!1901-07-121@{1[2?]. 7. 1901}|)be}\mylabel{L01146h}  \newcommand{\dateiname}{L01146}\newcommand{\titel}{Arthur Schnitzler an Richard Beer-Hofmann, 1[2?]. 7. 1901}\newcommand{\editorInnen}{Martin Anton Müller und Gerd-Hermann Susen}%% latex-leseansicht-abspann.tex
%% Abspann für die Leseansicht.
%% Der Schalter \ifkorrekturansicht ist bereits durch den Vorspann gesetzt.

%% latex-abspann.tex
%% Gemeinsamer Abspann für Korrekturansicht und Leseansicht.
%% Setzt den Schalter \ifkorrekturansicht voraus (gesetzt in den
%% einbindenden Dateien latex-korrekturansicht-abspann.tex bzw.
%% latex-leseansicht-abspann.tex).
%% ---------------------------------------------------------------

\normalsize

% Das esempio-Environment wird nur in der Leseansicht benötigt
\ifkorrekturansicht\else
\newenvironment{esempio}[3]%
{
    \vspace{1.5ex}
    \rlap{\underline{#1}}
    \par
    \setlength{\parindent}{0cm}
    \nopagebreak
    \leftskip=#2cm
    \rightskip=#3cm
}
{
    \par
}
\fi

\doendnotes{C}
\bigskip
\vfill

\clearpage

\footnotesize

\ifkorrekturansicht
  \lohead{\textsc{register}}
\fi

% theindex-Environment neu definieren ohne reledmac
\makeatletter
\renewenvironment{theindex}{%
  \ifkorrekturansicht
    \section*{\indexname}%
  \else
    \subsubsection*{Index der erwähnten Entitäten}%
  \fi
  \setlength{\parindent}{0pt}%
  \setlength{\parskip}{0pt plus 0.3pt}%
  \let\item\@idxitem
}{%
  \ifkorrekturansicht\clearpage\fi
}
\makeatother

\IfFileExists{\jobname-pw.ind}{\input{\jobname-pw.ind}}{}

% Quellenangabe nur in der Leseansicht
\ifkorrekturansicht\else
% Fallback-Definitionen, falls die .tex-Datei \titel etc. nicht gesetzt hat
\providecommand{\titel}{}
\providecommand{\editorInnen}{}
\providecommand{\dateiname}{\jobname}

\vspace{3cm}

\vfill

\footnotesize
\textsc{Quelle}: \titel. Herausgegeben von {\editorInnen}. In: \emph{Arthur Schnitzler: Briefwechsel mit Autorinnen und Autoren}.
 Digitale Edition, https://schnitzler-briefe.acdh.oeaw.ac.at/{\dateiname}.html (Stand \today)
\fi

\end{document}


