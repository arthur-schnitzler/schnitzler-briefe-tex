%% latex-leseansicht-vorspann.tex
%% Vorspann für die Leseansicht.
%% Lädt die gemeinsame Datei latex-vorspann.tex mit nicht gesetztem Schalter.

\newif\ifkorrekturansicht
\korrekturansichtfalse

\input{../tex-inputs/latex-vorspann}


\section[ Paul Goldmann an Arthur Schnitzler, 3. 1. {[}1903{]}]{L03360 Paul Goldmann an Arthur Schnitzler,  3. 1. [1903]}
\nopagebreak\mylabel{L03360v}
\rehead{ }\normalsize\beginnumbering\briefempfaengerindex{Schnitzler, Arthur@\textsc{Schnitzler, Arthur}!zzzGoldmann, Paul@\emph{von Paul Goldmann}!1903-01-031@{3. 1. [1903]}|(be}
\toendnotes[C]{\smallbreak\pagebreak[2]}
\correspDesc{Versand  durch Paul Goldmann am 3. 1. [1903] in Frankfurt am Main
\newline{}Erhalt  durch Arthur Schnitzler im Zeitraum [4. 1. 1903
                  – 8. 1. 1903?] in Wien}\toendnotes[C]{\smallbreak}
\Standort{DLA, A:Schnitzler, HS.NZ85.1.3173.}
\physDesc{Brief, 1 Blatt, 3 Seiten, 1324 Zeichen
\newline{}Handschrift: schwarze Tinte, deutsche Kurrent
\newline{}Schnitzler: 1) mit Bleistift das Jahr »903.« vermerkt  2) mit rotem Buntstift eine Unterstreichung}\toendnotes[C]{\smallbreak}
\pstart
           {\pb}\textcolor{gray}{\textbf{\textsc{Telephon \textbf{4167.}}}}\hfill \textcolor{gray}{\textbf{\textsc{Telegramm-Adresse:}}}\pend
           
\pstart
           \textcolor{gray}{\textbf{\textsc{und \textbf{3940.}}}}\hfill \textcolor{gray}{\textbf{\textbf{\textsc{Palast Fürstenhof\oindex{Fürstenhof [Frankfurt am Main]@\textbf{Fürstenhof [Frankfurt am Main]}, \emph{Gebäude}|pw}{ }Frankfurtmain\oindex{Frankfurt am Main@\textbf{Frankfurt am Main}, \emph{Hauptstadt}|pw}.}}}}\pend
           
\pstart
           \centering{}\textcolor{gray}{\textbf{\textsc{\textbf{Palast-Hotel\oindex{Fürstenhof [Frankfurt am Main]@\textbf{Fürstenhof [Frankfurt am Main]}, \emph{Gebäude}|pw}}}}}\pend
           
\pstart
           \centering{}\textcolor{gray}{\textbf{\textsc{Fürstenhof\oindex{Fürstenhof [Frankfurt am Main]@\textbf{Fürstenhof [Frankfurt am Main]}, \emph{Gebäude}|pw}}}}\pend
           
\pstart
           \centering{}\textcolor{gray}{\textbf{\textsc{Louis Bolle-Ritz\pwindex{Bolle-Ritz, Louis @\textsc{Bolle-Ritz, Louis}, \emph{Hotelbesitzer, Restaurateur}|pw}.}}}\pend
           
\pstart
           \centering{}\textcolor{gray}{\textbf{\textsc{(Kaiserstrasse\oindex{Kaiserstraße [Frankfurt am Main]@\textbf{Kaiserstraße [Frankfurt am Main]}, \emph{Straße}|pw} – Kronprinzenstrasse\oindex{Münchener Straße@\textbf{Münchener Straße}, \emph{Straße}|pw})}}}\pend
           
\pstart
           \raggedleft{}\textcolor{gray}{\textbf{Frankfurt \textsuperscript{a/}M.\oindex{Frankfurt am Main@\textbf{Frankfurt am Main}, \emph{Hauptstadt}|pw}}}{ }3. Januar.\pend
           
\pstart\center{}Mein lieber Freund,\pend\vspace{0.5em}
\pstart
           Dank für Deinen lieben und theilnehmenden Brief. Morgen fahre ich zurück. Es waren
               entſetzliche Tage. Geſtern habe ich \label{K_L03360-1v}\edtext{ſie\pwindex{Rottenberg, Theodore 7.\,9.\,1875 – 5.\,4.\,1945 Limburg an der Lahn@\textsc{Rottenberg, Theodore} (7.\,9.\,1875 – 5.\,4.\,1945 Limburg an der Lahn)|pwv}}{\lemma{\textnormal{\emph{sie}}}\Cendnote{\textnormal{Siehe XXXX Auszeichnungsfehler: Dokument L03231 nicht gefunden.
               }}}\label{K_L03360-1}, nach \strikeout{i} inſtändigen Bitten, zum letzten Mal
               geſehen. Ich habe{ }ſie flehentlich gebeten, zu mir zurückzukehren, habe ihr
               verſprochen,{ }ſie zu heirathen. Sie lächelt{ }ſchmerzlich: »zu{ }ſpät«. Sie hat mich nicht
               mehr lieb. Der {\pb}»Andere\pwindex{?? [Partner von Theodore Rottenberg, Ende 1902/Anfang 1903] @\textsc{?? [Partner von Theodore Rottenberg, Ende 1902/Anfang 1903]}|pwv}« exiſtirt. Er iſt ein rückenmarkskranker Millionär.
               Was ſie\pwindex{Rottenberg, Theodore 7.\,9.\,1875 – 5.\,4.\,1945 Limburg an der Lahn@\textsc{Rottenberg, Theodore} (7.\,9.\,1875 – 5.\,4.\,1945 Limburg an der Lahn)|pwv} an ihn feſſelt, iſt
               eine Miſchung von Romantik, Mitleid und Behagen an Geld und Wohlleben. Sie hat ihn
               gern,{ }ſie gefällt{ }ſich in der Rolle der \label{K_L03360-2v}\edtext{»\textsc{Mouche\pwindex{Krinitz, Elise 22.\,3.\,1825 Belgern – 7.\,8.\,1896 Orsay@\textsc{Krinitz, Elise} (22.\,3.\,1825 Belgern – 7.\,8.\,1896 Orsay), \emph{Schriftstellerin}|pwv}\pwindex{Heine, Heinrich 13.\,12.\,1797 Düsseldorf – 17.\,2.\,1856 Paris@\textsc{Heine, Heinrich} (13.\,12.\,1797 Düsseldorf – 17.\,2.\,1856 Paris), \emph{Schriftsteller}!Gedichte an die Mouche@\strich\emph{Gedichte an die Mouche}|pwv}}«}{\lemma{\textnormal{\emph{»Mouche«}}}\Cendnote{\textnormal{»Mouche« war Heinrich Heines\pwindex{Heine, Heinrich 13.\,12.\,1797 Düsseldorf – 17.\,2.\,1856 Paris@\textsc{Heine, Heinrich} (13.\,12.\,1797 Düsseldorf – 17.\,2.\,1856 Paris), \emph{Schriftsteller}|pwk} Kosename für seine letzte Geliebte, Elise Krinitz\pwindex{Krinitz, Elise 22.\,3.\,1825 Belgern – 7.\,8.\,1896 Orsay@\textsc{Krinitz, Elise} (22.\,3.\,1825 Belgern – 7.\,8.\,1896 Orsay), \emph{Schriftstellerin}|pwk}. In Heines\pwindex{Heine, Heinrich 13.\,12.\,1797 Düsseldorf – 17.\,2.\,1856 Paris@\textsc{Heine, Heinrich} (13.\,12.\,1797 Düsseldorf – 17.\,2.\,1856 Paris), \emph{Schriftsteller}|pwk} Nachlass finden sich auch fünf \emph{Gedichte an die Mouche}\pwindex{Heine, Heinrich 13.\,12.\,1797 Düsseldorf – 17.\,2.\,1856 Paris@\textsc{Heine, Heinrich} (13.\,12.\,1797 Düsseldorf – 17.\,2.\,1856 Paris), \emph{Schriftsteller}!Gedichte an die Mouche@\strich\emph{Gedichte an die Mouche}|pwk}.}}}\label{K_L03360-2}, – und{ }ſie iſt glücklich,
               daß er mit ihr nach \textsc{Monte Carlo\oindex{Monte Carlo@\textbf{Monte Carlo}, \emph{Ehemaliger Ort}|pw}} reiſen wird. Alles Wundervolle und alles Gemeine iſt in dieſer Frau\pwindex{Rottenberg, Theodore 7.\,9.\,1875 – 5.\,4.\,1945 Limburg an der Lahn@\textsc{Rottenberg, Theodore} (7.\,9.\,1875 – 5.\,4.\,1945 Limburg an der Lahn)|pwv} gemiſcht. Das gütigſte
               Herz und die{ }ſchamloſeſten dirnenhaften Inſtinkte. Ich müßte, aus moraliſchen und
               Vernunft-Gründen, froh{ }ſein, von ihr loszukommen. Aber was nützen Vernunft und Moral,
               da ich{ }ſie wahnſinnig liebe?\pend
           
\pstart
           Dank für Deine guten Worte! {\pb}Ich glaube nicht, daß
               ich darüber hinwegkommen werde. \strikeout{D\textcolor{gray}{e}r} Was blühend in meinem Leben war, iſt vernichtet, –
               vernichtet durch meine Schuld. Hätte ich erkannt, was ich an ihr beſaß, – hätte ich
               mich ihrer angenommen, – wäre ich nicht ein niederträchtiger Egoiſt geweſen, – ich
               hätte{ }ſie behalten.\pend
           
\pstart
           Adieu, liebſter Freund! Grüße Olga\pwindex{Schnitzler, Olga 17.\,1.\,1882 Wien – 13.\,1.\,1970 Lugano@\textsc{Schnitzler, Olga} (17.\,1.\,1882 Wien – 13.\,1.\,1970 Lugano), \emph{Schauspielerin, Sängerin}|pw} und den
               dicken Buben\pwindex{Schnitzler, Heinrich 9.\,8.\,1902 Hinterbrühl – 12.\,7.\,1982 Wien@\textsc{Schnitzler, Heinrich} (9.\,8.\,1902 Hinterbrühl – 12.\,7.\,1982 Wien), \emph{Regisseur, Schauspieler}|pwv}! {\\[\baselineskip]}Dein
               getreuer {\\[\baselineskip]}\spacefill\mbox{Paul Goldmann}\pend
           \leftskip=0em{}\selectlanguage{ngerman}\endnumbering\briefempfaengerindex{Schnitzler, Arthur@\textsc{Schnitzler, Arthur}!zzzGoldmann, Paul@\emph{von Paul Goldmann}!1903-01-031@{3. 1. [1903]}|)be}\mylabel{L03360h}  \newcommand{\dateiname}{L03360}\newcommand{\titel}{Paul Goldmann an Arthur Schnitzler, 3. 1. [1903]}\newcommand{\editorInnen}{Martin Anton Müller und Laura Untner}%% latex-leseansicht-abspann.tex
%% Abspann für die Leseansicht.
%% Der Schalter \ifkorrekturansicht ist bereits durch den Vorspann gesetzt.

%% latex-abspann.tex
%% Gemeinsamer Abspann für Korrekturansicht und Leseansicht.
%% Setzt den Schalter \ifkorrekturansicht voraus (gesetzt in den
%% einbindenden Dateien latex-korrekturansicht-abspann.tex bzw.
%% latex-leseansicht-abspann.tex).
%% ---------------------------------------------------------------

\normalsize

% Das esempio-Environment wird nur in der Leseansicht benötigt
\ifkorrekturansicht\else
\newenvironment{esempio}[3]%
{
    \vspace{1.5ex}
    \rlap{\underline{#1}}
    \par
    \setlength{\parindent}{0cm}
    \nopagebreak
    \leftskip=#2cm
    \rightskip=#3cm
}
{
    \par
}
\fi

\doendnotes{C}
\bigskip
\vfill

\clearpage

\footnotesize

\ifkorrekturansicht
  \lohead{\textsc{register}}
\fi

% theindex-Environment neu definieren ohne reledmac
\makeatletter
\renewenvironment{theindex}{%
  \ifkorrekturansicht
    \section*{\indexname}%
  \else
    \subsubsection*{Index der erwähnten Entitäten}%
  \fi
  \setlength{\parindent}{0pt}%
  \setlength{\parskip}{0pt plus 0.3pt}%
  \let\item\@idxitem
}{%
  \ifkorrekturansicht\clearpage\fi
}
\makeatother

\IfFileExists{\jobname-pw.ind}{\input{\jobname-pw.ind}}{}

% Quellenangabe nur in der Leseansicht
\ifkorrekturansicht\else
% Fallback-Definitionen, falls die .tex-Datei \titel etc. nicht gesetzt hat
\providecommand{\titel}{}
\providecommand{\editorInnen}{}
\providecommand{\dateiname}{\jobname}

\vspace{3cm}

\vfill

\footnotesize
\textsc{Quelle}: \titel. Herausgegeben von {\editorInnen}. In: \emph{Arthur Schnitzler: Briefwechsel mit Autorinnen und Autoren}.
 Digitale Edition, https://schnitzler-briefe.acdh.oeaw.ac.at/{\dateiname}.html (Stand \today)
\fi

\end{document}


