%% latex-korrekturansicht-vorspann.tex
%% Vorspann für die Korrekturansicht.
%% Lädt die gemeinsame Datei latex-vorspann.tex mit gesetztem Schalter.

\newif\ifkorrekturansicht
\korrekturansichttrue

\input{../tex-inputs/latex-vorspann}


\section[Arthur Schnitzler an Thomas Mann, 6. 11. 1924]{L02420 Arthur Schnitzler an Thomas Mann, 6. 11. 1924}
\nopagebreak\mylabel{L02420v}
\rehead{ }\normalsize\beginnumbering\briefempfaengerindex{Mann, Thomas@\textsc{Mann, Thomas}!zzzSchnitzler, Arthur@\emph{von Arthur Schnitzler}!1924-11-061@{6. 11. 1924}|(be}
\toendnotes[C]{\smallbreak\pagebreak[2]}\Standort{DLA, A:Schnitzler, 85.1.1371,2.}
\physDesc{Brief, Durchschlag1 Blatt, 1 Seite, 1033 Zeichen
\newline{}Schreibmaschine
\newline{}Handschrift: roter Buntstift, deutsche Kurrent (\noindent{}Beschriftung: »K{[}opie{]}«, dies durchgestrichen und durch einen Haken als erledigt
                                 markiert, Unterstreichungen)}
\buchAbdrucke{\weitereDrucke{1) \emph{Modern Austrian Literature}, Jg. 7 (1974) Nr. 1/2, S. 22–23.} \weitereDrucke{2) Arthur Schnitzler: \emph{Briefe 1913–1931}. Frankfurt am Main: \emph{S. Fischer} 1984, S. 372.} \weitereDrucke{3) Hans-Ulrich Lindken: \emph{Arthur Schnitzler. Aspekte und Akzente. Materialien zu Leben
                        und Werk}. Frankfurt am Main, Bern, Göttingen: \emph{Peter Lang} 1984, S. 197–198.} }
\pstart
           \raggedleft{}{\pb}6. 11. 1924.\pend
           
\pstart{}Lieber und verehrter Herr Thomas Mann.\pend\vspace{0.5em}
\pstart
           Die guten Worte, die Sie mir über die »Komödie der
                  Verführung\pwindex{Komoedie der Verfuehrung. In drei Akten@\emph{Komödie der Verführung. In drei Akten}|pw}« sagen, erfreuen mich herzlich. Hier bewährt sich das Stück weiter
               gut im Repertoir; Ihre Befürchtung, dass man es in München\oindex{Muenchen@\textbf{München}, \emph{P.PPLA}|pw} nicht gut genug darstellen würde, ist vorläufig
               unbegründet, da man dort kaum daran denkt es aufzuführen. Bisher hat sich in Deutschland\oindex{Deutschland@\textbf{Deutschland}, \emph{A.PCLI}|pw} nur Wiesbaden\oindex{Wiesbaden@\textbf{Wiesbaden}, \emph{P.PPLA}|pw} daran gewagt mit sehr gutem Erfolg, Köln\oindex{Koeln@\textbf{Köln}, \emph{P.PPLA2}|pw} folgt bald, Hannover\oindex{Hannover@\textbf{Hannover}, \emph{P.PPLA}|pw} und Danzig\oindex{Danzig@\textbf{Danzig}, \emph{A.ADM3}|pw} glaube ich haben es angenommen. Ueber die
               Kritik wollen wir lieber nicht reden. Das frühere Totschlage-Wort von der »grossen
               Zeit« ist nun durch das neue von der »versunkenen Welt« ersetzt worden. Es ist, als
               wäre die Weltgeschichte überhaupt nur dazu da, um den Rezensenten neue falsche
               Massstäbe an die Hand zu geben.\pend
           
\pstart
           Der Zusendung des »Zauberbergs\pwindex{Zauberberg. Roman@\emph{Der Zauberberg. Roman}|pw}« sehe ich mit
               freudiger Ungeduld entgegen. Ich hoffe Sie haben in Sestri-Levante\oindex{Sestri Levante@\textbf{Sestri Levante}, \emph{P.PPLA3}|pw}{ }schöne Tage, – helle, arbeitsfreudige,
               lebensfrohe.\pend
           
\pstart
           Seien Sie sehr herzlich gegrüsst{\\[\baselineskip]}von Ihrem aufrichtig ergebenen\pend
           \leftskip=0em{}
\pstart
           \noindent{}Herrn Thomas Mann{\\}Sestri-Levante\oindex{Sestri Levante@\textbf{Sestri Levante}, \emph{P.PPLA3}|pw},{\\}Italien\oindex{Italien@\textbf{Italien}, \emph{A.PCLI}|pw}.\pend
           \selectlanguage{ngerman}\endnumbering\briefempfaengerindex{Mann, Thomas@\textsc{Mann, Thomas}!zzzSchnitzler, Arthur@\emph{von Arthur Schnitzler}!1924-11-061@{6. 11. 1924}|)be}\mylabel{L02420h}  \normalsize

\doendnotes{C}
\bigskip
\vfill

\clearpage

\footnotesize

\lohead{\textsc{register}}

% Definiere theindex-Environment komplett neu ohne reledmac
\makeatletter
\renewenvironment{theindex}{%
  \section*{\indexname}%
  \setlength{\parindent}{0pt}%
  \setlength{\parskip}{0pt plus 0.3pt}%
  \let\item\@idxitem
}{%
  \clearpage
}
\makeatother

\IfFileExists{\jobname-pw.ind}{\input{\jobname-pw.ind}}{}

\end{document}

      