%% latex-korrekturansicht-vorspann.tex
%% Vorspann für die Korrekturansicht.
%% Lädt die gemeinsame Datei latex-vorspann.tex mit gesetztem Schalter.

\newif\ifkorrekturansicht
\korrekturansichttrue

\input{../tex-inputs/latex-vorspann}


\section[Hermann Bahr an Arthur Schnitzler, 21. 12. {[}1900{]}]{L01086 Hermann Bahr an Arthur Schnitzler, 21. 12. {[}1900{]}}
\nopagebreak\mylabel{L01086v}
\rehead{ }\normalsize\beginnumbering\briefempfaengerindex{Schnitzler, Arthur@\textsc{Schnitzler, Arthur}!zzzBahr, Hermann@\emph{von Hermann Bahr}!1900-12-211@{21. 12. {[}1900{]}}|(be}
\toendnotes[C]{\smallbreak\pagebreak[2]}\Standort{CUL, Schnitzler, B 5b.}
\physDesc{Brief, 1 Blatt, 3 Seiten, 493 Zeichen
\newline{}Handschrift: schwarze Tinte, deutsche Kurrent
\newline{}Schnitzler: mit Bleistift die Jahreszahl »900« ergänzt 
\newline{}Ordnung: mit Bleistift von unbekannter Hand nummeriert:
                                    »71« }
\buchAbdrucke{\weitereDrucke{Hermann Bahr, Arthur Schnitzler: \emph{Briefwechsel, Aufzeichnungen, Dokumente (1891–1931)}. Göttingen: \emph{Wallstein} 2018, S. 191.} }\toendnotes[C]{\smallbreak}
\pstart
           \raggedleft{}{\pb}2\substVorne{}\textsuperscript{2}\substDazwischen{}1\substHinten{}/12\pend
           
\pstart\center{}Lieber Arthur!\pend\vspace{0.5em}
\pstart
           \textsc{Bukovics}\pwindex{Bukovics, Emerich von 28.02.1844 – 04.07.1905@\textsc{Bukovics, Emerich von} (28.02.1844 – 04.07.1905), \emph{Journalist/Journalistin, Theaterleiter/Theaterleiterin}|pw}{ }ſagt mir, es ſei über den \label{K_L01086-1v}\edtext{Volkstheater\oindex{Volkstheater@\textbf{Volkstheater}, \emph{Theater (K.THE)}|pw}abend}{\lemma{\textnormal{\emph{Volkstheaterabend}}}\Cendnote{\textnormal{Ein jährlich stattfindender Abend in einem angemieteten
                  Veranstaltungssaal mit speziellem Programm. 1901 fand er am
                     9. 3. in den Sophiensälen\oindex{Sofiensaele@\textbf{Sofiensäle}, \emph{Veranstaltungsgebäude (K.VSB)}|pwk}
                  statt. Vor der Eröffnung der Tanzfläche wurden Lieder gesungen und das Mimodrama
                     \emph{Die Hand}\pwindex{Hand@\emph{Die Hand}|pwk} von Henri Berény\pwindex{Bereny, Henri 01.01.1871 – 22.03.1932@\textsc{Berény, Henri} (01.01.1871 – 22.03.1932), \emph{Komponist/Komponistin, Theateragent/Theateragentin}|pwk} gegeben.}}}\label{K_L01086-1} dieſes Jahr noch nichts
               beschloſſen. Ich mache Dich nur aufmerkſam, daß bei dem ſpäten Anfang (½ 11), der
               elenden Bühne (meiſtens Ronacher\oindex{Ronacher@\textbf{Ronacher}, \emph{Theater (K.THE)}|pw}) u. der kaum zu
               bändigenden Tanzluſt hier nur {\pb}ganz einfache u.
               rohe Sachen wirken.\pend
           
\pstart
           Für die lieben Worte Deines Briefes danke ich Dir ſehr und bin, Dir das Beſte
               wünſchend,\pend
           
\pstart
           Dein alter{\\[\baselineskip]}\spacefill\mbox{Hermann}\pend
           \leftskip=0em{}
\pstart
           \noindent{}Hofrath \textsc{Burckhard}\pwindex{Burckhard, Max Eugen 14.07.1854 – 16.03.1912@\textsc{Burckhard, Max Eugen} (14.07.1854 – 16.03.1912), \emph{Schriftsteller/Schriftstellerin, Rechtswissenschaftler/Rechtswissenschaftlerin, Theaterleiter/Theaterleiterin}|pw} möchte ſehr gern ein Exemplar der {\pb}Beatrice\pwindex{Schleier der Beatrice. Schauspiel in fuenf Akten@\emph{Der Schleier der Beatrice. Schauspiel in fünf Akten}|pw} haben; kannſt Du ihm nicht eins
                  ſchicken?\pend
           \selectlanguage{ngerman}\endnumbering\briefempfaengerindex{Schnitzler, Arthur@\textsc{Schnitzler, Arthur}!zzzBahr, Hermann@\emph{von Hermann Bahr}!1900-12-211@{21. 12. {[}1900{]}}|)be}\mylabel{L01086h}  \normalsize

\doendnotes{C}
\bigskip
\vfill

\clearpage

\footnotesize

\lohead{\textsc{register}}

% Definiere theindex-Environment komplett neu ohne reledmac
\makeatletter
\renewenvironment{theindex}{%
  \section*{\indexname}%
  \setlength{\parindent}{0pt}%
  \setlength{\parskip}{0pt plus 0.3pt}%
  \let\item\@idxitem
}{%
  \clearpage
}
\makeatother

\IfFileExists{\jobname-pw.ind}{\input{\jobname-pw.ind}}{}

\end{document}

      