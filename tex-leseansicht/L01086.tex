%% latex-leseansicht-vorspann.tex
%% Vorspann für die Leseansicht.
%% Lädt die gemeinsame Datei latex-vorspann.tex mit nicht gesetztem Schalter.

\newif\ifkorrekturansicht
\korrekturansichtfalse

\input{../tex-inputs/latex-vorspann}


         
         \renewcommand{\erwaehntePersonen}{Personen: Hermann Bahr, Henri Berény, Emerich von Bukovics, Max Eugen Burckhard}
         \renewcommand{\erwaehnteOrte}{Orte: Ronacher, Sofiensäle, Volkstheater, Wien}
         \renewcommand{\erwaehnteWerke}{Werke: Der Schleier der Beatrice. Schauspiel in fünf Akten, Die Hand}
               \section[Hermann Bahr an Arthur Schnitzler, 21. 12. {[}1900{]}]{ Hermann Bahr an Arthur Schnitzler, 21. 12. {[}1900{]}}\nopagebreak\mylabel{v}\rehead{ }\begin{ledgroupsized}[t]{13cm}\normalsize\beginnumbering\briefempfaengerindex{Schnitzler, Arthur@\textsc{Schnitzler, Arthur}!zzzBahr, Hermann@\emph{von Hermann Bahr}!1900-12-211@{21. 12. {[}1900{]}}|(be} \toendnotes[C]{\smallbreak\pagebreak[2]} \Standort{CUL, Schnitzler, B 5b.}
\physDesc{Brief, 1 Blatt, 3 Seiten, 493 Zeichen
\newline{}Handschrift: schwarze Tinte, deutsche Kurrent
\newline{}Schnitzler: mit Bleistift die Jahreszahl »900« ergänzt 
\newline{}Ordnung: mit Bleistift von unbekannter Hand nummeriert:
                                    »71« }\buchAbdrucke{\weitereDrucke{Hermann Bahr, Arthur Schnitzler: \emph{Briefwechsel, Aufzeichnungen, Dokumente (1891–1931)}. Hg. Kurt Ifkovits und Martin Anton Müller. Göttingen: \emph{Wallstein} 2018, S. 191.} }\toendnotes[C]{\smallbreak}\pstart
           \raggedleft{}{\pb}2\substVorne{}\textsuperscript{2}\substDazwischen{}1\substHinten{}/12\pend
           \pstart\center{}Lieber Arthur!\pend\pstart
           \textsc{Bukovics}\pwindex{Bukovics, Emerich von 28.02.1844 – 04.07.1905@\textsc{Bukovics, Emerich von} (28.02.1844 – 04.07.1905), \emph{Journalist, Theaterleiter}|pw}{ }ſagt mir, es ſei über den \label{K_L01086-1v}\edtext{Volkstheater\oindex{Volkstheater@\textbf{Volkstheater}|pw}abend}{\lemma{\textnormal{\emph{Volkstheaterabend}}}\Cendnote{\textnormal{Ein jährlich stattfindender Abend in einem angemieteten
                  Veranstaltungssaal mit speziellem Programm. 1901 fand er am
                     9. 3. in den Sophiensälen\oindex{Sofiensaele@\textbf{Sofiensäle}|pwk}
                  statt. Vor der Eröffnung der Tanzfläche wurden Lieder gesungen und das Mimodrama
                     \emph{Die Hand}\pwindex{Bereny, Henri 01.01.1871 – 22.03.1932@\textsc{Berény, Henri} (01.01.1871 – 22.03.1932), \emph{Komponist, Theateragent}!Hand1900@\strich\emph{Die Hand} {[}1900{]}|pwk} von Henri Berény\pwindex{Bereny, Henri 01.01.1871 – 22.03.1932@\textsc{Berény, Henri} (01.01.1871 – 22.03.1932), \emph{Komponist, Theateragent}|pwk} gegeben.}}}\label{K_L01086-1h} dieſes Jahr noch nichts
               beschloſſen. Ich mache Dich nur aufmerkſam, daß bei dem ſpäten Anfang (½ 11), der
               elenden Bühne (meiſtens Ronacher\oindex{Ronacher@\textbf{Ronacher}|pw}) u. der kaum zu
               bändigenden Tanzluſt hier nur {\pb}ganz einfache u.
               rohe Sachen wirken.\pend
           \pstart
           Für die lieben Worte Deines Briefes danke ich Dir ſehr und bin, Dir das Beſte
               wünſchend,\pend
           \pstart
           Dein alter{\\[\baselineskip]}\spacefill\mbox{Hermann}\pend
           \leftskip=0em{}\pstart
           \noindent{}Hofrath \textsc{Burckhard}\pwindex{Burckhard, Max Eugen 14.07.1854 – 16.03.1912@\textsc{Burckhard, Max Eugen} (14.07.1854 – 16.03.1912), \emph{Schriftsteller, Rechtswissenschaftler, Theaterleiter}|pw} möchte ſehr gern ein Exemplar der {\pb}Beatrice\pwindex{Schnitzler, Arthur 15.05.1862 – 21.10.1931@\textsc{Schnitzler, Arthur} (15.05.1862 – 21.10.1931), \emph{Schriftsteller, Mediziner}!Schleier der Beatrice. Schauspiel in fuenf Akten1900-12-01@\strich\emph{Der Schleier der Beatrice. Schauspiel in fünf Akten} {[}1900-12-01{]}|pw} haben; kannſt Du ihm nicht eins
                  ſchicken?\pend
           
         
         \endnumbering\mylabel{h}\end{ledgroupsized}  \newcommand{\dateiname}{L01086}\newcommand{\titel}{Hermann Bahr an Arthur Schnitzler, 21. 12. [1900]}\newcommand{\editorInnen}{ Kurt Ifkovits,  Martin Anton Müller}%% latex-leseansicht-abspann.tex
%% Abspann für die Leseansicht.
%% Der Schalter \ifkorrekturansicht ist bereits durch den Vorspann gesetzt.

%% latex-abspann.tex
%% Gemeinsamer Abspann für Korrekturansicht und Leseansicht.
%% Setzt den Schalter \ifkorrekturansicht voraus (gesetzt in den
%% einbindenden Dateien latex-korrekturansicht-abspann.tex bzw.
%% latex-leseansicht-abspann.tex).
%% ---------------------------------------------------------------

\normalsize

% Das esempio-Environment wird nur in der Leseansicht benötigt
\ifkorrekturansicht\else
\newenvironment{esempio}[3]%
{
    \vspace{1.5ex}
    \rlap{\underline{#1}}
    \par
    \setlength{\parindent}{0cm}
    \nopagebreak
    \leftskip=#2cm
    \rightskip=#3cm
}
{
    \par
}
\fi

\doendnotes{C}
\bigskip
\vfill

\clearpage

\footnotesize

\ifkorrekturansicht
  \lohead{\textsc{register}}
\fi

% theindex-Environment neu definieren ohne reledmac
\makeatletter
\renewenvironment{theindex}{%
  \ifkorrekturansicht
    \section*{\indexname}%
  \else
    \subsubsection*{Index der erwähnten Entitäten}%
  \fi
  \setlength{\parindent}{0pt}%
  \setlength{\parskip}{0pt plus 0.3pt}%
  \let\item\@idxitem
}{%
  \ifkorrekturansicht\clearpage\fi
}
\makeatother

\IfFileExists{\jobname-pw.ind}{\input{\jobname-pw.ind}}{}

% Quellenangabe nur in der Leseansicht
\ifkorrekturansicht\else
% Fallback-Definitionen, falls die .tex-Datei \titel etc. nicht gesetzt hat
\providecommand{\titel}{}
\providecommand{\editorInnen}{}
\providecommand{\dateiname}{\jobname}

\vspace{3cm}

\vfill

\footnotesize
\textsc{Quelle}: \titel. Herausgegeben von {\editorInnen}. In: \emph{Arthur Schnitzler: Briefwechsel mit Autorinnen und Autoren}.
 Digitale Edition, https://schnitzler-briefe.acdh.oeaw.ac.at/{\dateiname}.html (Stand \today)
\fi

\end{document}


      