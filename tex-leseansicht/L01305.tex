%% latex-leseansicht-vorspann.tex
%% Vorspann für die Leseansicht.
%% Lädt die gemeinsame Datei latex-vorspann.tex mit nicht gesetztem Schalter.

\newif\ifkorrekturansicht
\korrekturansichtfalse

\input{../tex-inputs/latex-vorspann}


\section[Arthur Schnitzler an Richard Beer-Hofmann, 28. 7. 1903]{L01305 Arthur Schnitzler an Richard Beer-Hofmann, 28. 7. 1903}
\nopagebreak\mylabel{L01305v}
\rehead{ }\normalsize\beginnumbering\briefempfaengerindex{Beer-Hofmann, Richard@\textsc{Beer-Hofmann, Richard}!zzzSchnitzler, Arthur@\emph{von Arthur Schnitzler}!1903-07-281@{28. 7. 1903}|(be}
\toendnotes[C]{\smallbreak\pagebreak[2]}
\correspDesc{Versand  durch Arthur Schnitzler am 28. 7. 1903 in Hochschneeberg
\newline{}Übermittlung  am 29. 7. 1903 in Hochschneeberg
\newline{}Erhalt  durch Richard Beer-Hofmann am 30. 7. 1903 in Rodaun}\toendnotes[C]{\smallbreak}
\Standort{YCGL, MSS 31.}
\physDesc{Bildpostkarte, 92 Zeichen
\newline{}Handschrift: Bleistift, deutsche Kurrent
\newline{}Versand: 1) Stempel: »\nobreak{}\oindex{Hochschneeberg@\textbf{Hochschneeberg}, \emph{Gebirge}|pwk}Hochschneeberg, 29. 7. \textcolor{gray}{03}, 2–4N\nobreak{}«.   2) Stempel: »\nobreak{}\oindex{Wien@\textbf{Wien}!XXIII., Liesing@\textbf{XXIII., Liesing}!Rodaun@\textbf{Rodaun}, \emph{Region}|pwk}{[}Rod{]}\textcolor{gray}{a}un, 30/7 {[}03{]}\nobreak{}«. 
\newline{}Ordnung: mit Bleistift von unbekannter Hand datiert: »28. 7.« }\toendnotes[C]{\smallbreak}\pstart{}{\pb}Herrn \textsc{Dr Richard
                     Beer-Hofmann}\pend{}\pstart{}\textsc{Rodaun \textsuperscript{b}/ Wien}\oindex{Wien@\textbf{Wien}!XXIII., Liesing@\textbf{XXIII., Liesing}!Rodaun@\textbf{Rodaun}, \emph{Region}|pw}\pend{}\pstart{}\textsc{Liesinger Hptstr 2}\oindex{Liesingerstraße@\textbf{Liesingerstraße}, \emph{Straße}|pw}.\pend{}{\bigskip}
\pstart
           \noindent{}\centering{}{\pb}\textcolor{gray}{\textbf{Fischerhütte am Schneeberg\oindex{Fischerhütte@\textbf{Fischerhütte}, \emph{Beherbergungsgebäude}|pw} mit dem Klosterwappen\oindex{Klosterwappen@\textbf{Klosterwappen}, \emph{Berg}|pw} 2075 Mtr.}}\pend
           \vspace{1em}
\pstart
           \raggedleft{}{\pb}\label{K_L01305-1v}\edtext{28. 7. 903}{\lemma{\textnormal{\emph{28. 7. 903}}}\Cendnote{\textnormal{Das Besondere an dieser Karte ist,
                        dass der Empfänger, unmittelbar nachdem sie abgeschickt wurde, ebenfalls am
                        Berg erschien und über Nacht blieb.}}}\label{K_L01305-1}.\pend
           \vspace{0.5em}
\pstart
           Herzliche Grüße!\pend
           
\pstart
           Ihr{\\[\baselineskip]}\spacefill\mbox{A.}\pend
           \leftskip=0em{}\selectlanguage{ngerman}\endnumbering\briefempfaengerindex{Beer-Hofmann, Richard@\textsc{Beer-Hofmann, Richard}!zzzSchnitzler, Arthur@\emph{von Arthur Schnitzler}!1903-07-281@{28. 7. 1903}|)be}\mylabel{L01305h}  \newcommand{\dateiname}{L01305}\newcommand{\titel}{Arthur Schnitzler an Richard Beer-Hofmann, 28. 7. 1903}\newcommand{\editorInnen}{Martin Anton Müller und Gerd-Hermann Susen}%% latex-leseansicht-abspann.tex
%% Abspann für die Leseansicht.
%% Der Schalter \ifkorrekturansicht ist bereits durch den Vorspann gesetzt.

%% latex-abspann.tex
%% Gemeinsamer Abspann für Korrekturansicht und Leseansicht.
%% Setzt den Schalter \ifkorrekturansicht voraus (gesetzt in den
%% einbindenden Dateien latex-korrekturansicht-abspann.tex bzw.
%% latex-leseansicht-abspann.tex).
%% ---------------------------------------------------------------

\normalsize

% Das esempio-Environment wird nur in der Leseansicht benötigt
\ifkorrekturansicht\else
\newenvironment{esempio}[3]%
{
    \vspace{1.5ex}
    \rlap{\underline{#1}}
    \par
    \setlength{\parindent}{0cm}
    \nopagebreak
    \leftskip=#2cm
    \rightskip=#3cm
}
{
    \par
}
\fi

\doendnotes{C}
\bigskip
\vfill

\clearpage

\footnotesize

\ifkorrekturansicht
  \lohead{\textsc{register}}
\fi

% theindex-Environment neu definieren ohne reledmac
\makeatletter
\renewenvironment{theindex}{%
  \ifkorrekturansicht
    \section*{\indexname}%
  \else
    \subsubsection*{Index der erwähnten Entitäten}%
  \fi
  \setlength{\parindent}{0pt}%
  \setlength{\parskip}{0pt plus 0.3pt}%
  \let\item\@idxitem
}{%
  \ifkorrekturansicht\clearpage\fi
}
\makeatother

\IfFileExists{\jobname-pw.ind}{\input{\jobname-pw.ind}}{}

% Quellenangabe nur in der Leseansicht
\ifkorrekturansicht\else
% Fallback-Definitionen, falls die .tex-Datei \titel etc. nicht gesetzt hat
\providecommand{\titel}{}
\providecommand{\editorInnen}{}
\providecommand{\dateiname}{\jobname}

\vspace{3cm}

\vfill

\footnotesize
\textsc{Quelle}: \titel. Herausgegeben von {\editorInnen}. In: \emph{Arthur Schnitzler: Briefwechsel mit Autorinnen und Autoren}.
 Digitale Edition, https://schnitzler-briefe.acdh.oeaw.ac.at/{\dateiname}.html (Stand \today)
\fi

\end{document}


