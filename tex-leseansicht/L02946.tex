%% latex-korrekturansicht-vorspann.tex
%% Vorspann für die Korrekturansicht.
%% Lädt die gemeinsame Datei latex-vorspann.tex mit gesetztem Schalter.

\newif\ifkorrekturansicht
\korrekturansichttrue

\input{../tex-inputs/latex-vorspann}


\section[ Paul Goldmann an Arthur Schnitzler, 27. 12. {[}1900{]}]{L02946 Paul Goldmann an Arthur Schnitzler, 27. 12. {[}1900{]}}
\nopagebreak\mylabel{L02946v}
\rehead{ }\normalsize\beginnumbering\briefempfaengerindex{Schnitzler, Arthur@\textsc{Schnitzler, Arthur}!zzzGoldmann, Paul@\emph{von Paul Goldmann}!1900-12-271@{27. 12. {[}1900{]}}|(be}
\toendnotes[C]{\smallbreak\pagebreak[2]}\Standort{DLA, A:Schnitzler, HS.NZ85.1.3170.}
\physDesc{Brief, 1 Blatt, 3 Seiten, 1065 Zeichen
\newline{}Handschrift: blaue Tinte, deutsche Kurrent
\newline{}Schnitzler: mit Bleistift das Jahr »900« vermerkt }\toendnotes[C]{\smallbreak}
\pstart
           \noindent{}
\pstart
           {\pb}Frankfurt\oindex{Frankfurt am Main@\textbf{Frankfurt am Main}, \emph{P.PPLA3}|pw}{ }27. December.\pend
           
\pstart
           \raggedleft{}\textcolor{gray}{\textbf{Reuterweg 59\oindex{Reuterweg@\textbf{Reuterweg}, \emph{Straße (K.STR)}|pw}.}}\pend
           \pend
           
\pstart
           \centering{}Mein lieber Freund,\pend
           
\pstart
           Ich hoffe, Du haſt frohe Weihnachten gehabt und ich wünſche Dir ein glückliches neues
                  Jahr.\pend
           
\pstart
           Ich bin dieſe Woche in Frankfurt\oindex{Frankfurt am Main@\textbf{Frankfurt am Main}, \emph{P.PPLA3}|pw}, ruhe mich ein
               wenig aus und laſſe es mir gut gehen.\pend
           
\pstart
           Alle die Meinigen grüßen Dich. Mein Onkel\pwindex{Mamroth, Fedor 21.02.1851 – 25.06.1907@\textsc{Mamroth, Fedor} (21.02.1851 – 25.06.1907), \emph{Journalist/Journalistin, Kritiker/Kritikerin}|pwv} hätte gern den »blinden
                     \textsc{Hironymo}\pwindex{blinde Geronimo und sein Bruder@\emph{Der blinde Geronimo und sein Bruder}|pw}« für die Frankfurter Zeitung\pwindex{Frankfurter Zeitung@\emph{Frankfurter Zeitung}|pw} gehabt und
               läßt Dich bitten, wenn Du wieder einmal eine kurze Novelle fertig haſt, ſie ihm zu
               ſchicken.\pend
           
\pstart
           Die Weihnachtsnummer der N. Fr. Pr.\pwindex{Neue Freie Presse@\emph{Neue Freie Presse}|pw} iſt mir
               nicht zu Geſicht {\pb}gekommen, und ich habe den \label{K_L02946-1v}\edtext{»Lieutenant
                  Guſtl\pwindex{Lieutenant Gustl. Novelle@\emph{Lieutenant Gustl. Novelle}|pw}«}{\lemma{\textnormal{\emph{»Lieutenant
                  Guſtl«}}}\Cendnote{\textnormal{Arthur Schnitzler: \emph{Lieutenant Gustl}\pwindex{Lieutenant Gustl. Novelle@\emph{Lieutenant Gustl. Novelle}|pwk}. In: \emph{Neue Freie Presse}\pwindex{Neue Freie Presse@\emph{Neue Freie Presse}|pwk}, Nr. 13.053, 25. 12. 1900, Morgenblatt, S. 34–41.}}}\label{K_L02946-1} daher noch nicht
               geleſen.\pend
           
\pstart
           Gibſt Du die »\textsc{Beatrice\pwindex{Schleier der Beatrice. Schauspiel in fuenf Akten@\emph{Der Schleier der Beatrice. Schauspiel in fünf Akten}|pw}}« dem \label{K_L02946-2v}\edtext{»Volkstheater\orgindex{Volkstheater@Volkstheater|pw}«}{\lemma{\textnormal{\emph{»Volkstheater«}}}\Cendnote{\textnormal{Siehe Paul Goldmann an Arthur Schnitzler, 21. 6. [1900] und 9. 12. [1900].
               }}}\label{K_L02946-2}? Du ſollteſt es entſchieden thun. Auch mein Onkel\pwindex{Mamroth, Fedor 21.02.1851 – 25.06.1907@\textsc{Mamroth, Fedor} (21.02.1851 – 25.06.1907), \emph{Journalist/Journalistin, Kritiker/Kritikerin}|pwv} iſt der Anſicht.\pend
           
\pstart
           Meine Feuilletons ſammeln? Nie im Leben finde ich einen Verleger. Man weiſt mich mit
               Hohnlachen zurück, wenn ich mit ſo etwas komme.\pend
           
\pstart
           Sei ſo gut und ſchreib mir ein Wort hierher an die obige Adreſſe meines Schwagers \textsc{Dr. Rosengart\pwindex{Rosengart, Josef 1860-02-08 – 1927-08-04@\textsc{Rosengart, Josef} (1860-02-08 – 1927-08-04), \emph{Arzt/Ärztin}|pw}}.\pend
           
\pstart
           Bitte auch Deiner Frau Mutter\pwindex{Schnitzler, Louise 1840-07-08 – 1911-09-09@\textsc{Schnitzler, Louise} (1840-07-08 – 1911-09-09)|pwv}, Deinem Bruder\pwindex{Schnitzler, Julius 13.07.1865 – 29.06.1939@\textsc{Schnitzler, Julius} (13.07.1865 – 29.06.1939), \emph{Chirurg/Chirurgin}|pwv} und Deiner Schwägerin\pwindex{Schnitzler, Helene 16.07.1871 – September 1941@\textsc{Schnitzler, Helene} (16.07.1871 – September 1941)|pwv}, {\pb}Deiner Schweſter\pwindex{Hajek, Gisela 20.12.1867 – 03.02.1953@\textsc{Hajek, Gisela} (20.12.1867 – 03.02.1953)|pwv} und Deinem Schwager\pwindex{Hajek, Gisela 20.12.1867 – 03.02.1953@\textsc{Hajek, Gisela} (20.12.1867 – 03.02.1953)|pwv} meine herzlichſten
               Neujahrs-Glückwünſche zu übermitteln.\pend
           
\pstart
           Viele treue Grüße! {\\[\baselineskip]}Dein {\\[\baselineskip]}\spacefill\mbox{Paul Goldmann.}\pend
           \leftskip=0em{}\selectlanguage{ngerman}\endnumbering\briefempfaengerindex{Schnitzler, Arthur@\textsc{Schnitzler, Arthur}!zzzGoldmann, Paul@\emph{von Paul Goldmann}!1900-12-271@{27. 12. {[}1900{]}}|)be}\mylabel{L02946h}  \normalsize

\doendnotes{C}
\bigskip
\vfill

\clearpage

\footnotesize

\lohead{\textsc{register}}

% Definiere theindex-Environment komplett neu ohne reledmac
\makeatletter
\renewenvironment{theindex}{%
  \section*{\indexname}%
  \setlength{\parindent}{0pt}%
  \setlength{\parskip}{0pt plus 0.3pt}%
  \let\item\@idxitem
}{%
  \clearpage
}
\makeatother

\IfFileExists{\jobname-pw.ind}{\input{\jobname-pw.ind}}{}

\end{document}

      