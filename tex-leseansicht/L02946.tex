%% latex-leseansicht-vorspann.tex
%% Vorspann für die Leseansicht.
%% Lädt die gemeinsame Datei latex-vorspann.tex mit nicht gesetztem Schalter.

\newif\ifkorrekturansicht
\korrekturansichtfalse

\input{../tex-inputs/latex-vorspann}


         
         \renewcommand{\erwaehntePersonen}{Personen: Paul Goldmann, Gisela Hajek, Fedor Mamroth, Josef Rosengart, Louise Schnitzler, Julius Schnitzler, Helene Schnitzler}
         \renewcommand{\erwaehnteInstitutionen}{Institutionen: Volkstheater}
         \renewcommand{\erwaehnteOrte}{Orte: Frankfurt am Main, Reuterweg, Wien}
         \renewcommand{\erwaehnteWerke}{Werke: Der Schleier der Beatrice. Schauspiel in fünf Akten, Der blinde Geronimo und sein Bruder, Frankfurter Zeitung, Lieutenant Gustl. Novelle, Neue Freie Presse}
               \section[ Paul Goldmann an Arthur Schnitzler, 27. 12. {[}1900{]}]{ Paul Goldmann an Arthur Schnitzler, 27. 12. {[}1900{]}}\nopagebreak\mylabel{v}\rehead{ }\begin{ledgroupsized}[t]{13cm}\normalsize\beginnumbering\briefempfaengerindex{Schnitzler, Arthur@\textsc{Schnitzler, Arthur}!zzzGoldmann, Paul@\emph{von Paul Goldmann}!1900-12-271@{27. 12. {[}1900{]}}|(be} \toendnotes[C]{\smallbreak\pagebreak[2]} \Standort{DLA, A:Schnitzler, HS.NZ85.1.3170.}
\physDesc{Brief, 1 Blatt, 3 Seiten, 1065 Zeichen
\newline{}Handschrift: blaue Tinte, deutsche Kurrent
\newline{}Schnitzler: mit Bleistift das Jahr »900« vermerkt }\toendnotes[C]{\smallbreak}\pstart
           \noindent{}{\pb}Frankfurt\oindex{Frankfurt am Main@\textbf{Frankfurt am Main}|pw}{ }27. December.\hfill \textcolor{gray}{\textbf{Reuterweg 59\oindex{Reuterweg@\textbf{Reuterweg}|pw}.}}\pend
           \pstart
           \centering{}Mein lieber Freund,\pend
           \pstart
           \noindent{}Ich hoffe, Du haſt frohe Weihnachten gehabt und ich wünſche Dir ein glückliches neues
                  Jahr.\pend
           \pstart
           Ich bin dieſe Woche in Frankfurt\oindex{Frankfurt am Main@\textbf{Frankfurt am Main}|pw}, ruhe mich ein
               wenig aus und laſſe es mir gut gehen.\pend
           \pstart
           Alle die Meinigen grüßen Dich. Mein Onkel\pwindex{Mamroth, Fedor 21.02.1851 – 25.06.1907@\textsc{Mamroth, Fedor} (21.02.1851 – 25.06.1907), \emph{Journalist, Kritiker}|pwv} hätte gern den »blinden
                     \textsc{Hironymo}\pwindex{Schnitzler, Arthur 15.05.1862 – 21.10.1931@\textsc{Schnitzler, Arthur} (15.05.1862 – 21.10.1931), \emph{Schriftsteller, Mediziner}!blinde Geronimo und sein Bruder22.12.1900 – 12.1.1901@\strich\emph{Der blinde Geronimo und sein Bruder} {[}22.12.1900 – 12.1.1901{]}|pw}« für die Frankfurter Zeitung\pwindex{?? Werk@Nicht ermittelte Verfasserinnen und Verfasser!Frankfurter Zeitung1856 – 1943@\emph{Frankfurter Zeitung} {[}1856 – 1943{]}|pw} gehabt und
               läßt Dich bitten, wenn Du wieder einmal eine kurze Novelle fertig haſt, ſie ihm zu
               ſchicken.\pend
           \pstart
           Die Weihnachtsnummer der N. Fr. Pr.\pwindex{Neue Freie Presse1864 – 1939@\emph{Neue Freie Presse} {[}1864 – 1939{]}|pw} iſt mir
               nicht zu Geſicht {\pb}gekommen, und ich habe den \label{K_L02946-1v}\edtext{»Lieutenant
                  Guſtl\pwindex{Schnitzler, Arthur 15.05.1862 – 21.10.1931@\textsc{Schnitzler, Arthur} (15.05.1862 – 21.10.1931), \emph{Schriftsteller, Mediziner}!Lieutenant Gustl. Novelle1900-12-25@\strich\emph{Lieutenant Gustl. Novelle} {[}1900-12-25{]}|pw}«}{\lemma{\textnormal{\emph{»Lieutenant
                  Guſtl«}}}\Cendnote{\textnormal{Arthur Schnitzler\pwindex{Schnitzler, Arthur 15.05.1862 – 21.10.1931@\textsc{Schnitzler, Arthur} (15.05.1862 – 21.10.1931), \emph{Schriftsteller, Mediziner}|pwk}: \emph{Lieutenant Gustl}\pwindex{Schnitzler, Arthur 15.05.1862 – 21.10.1931@\textsc{Schnitzler, Arthur} (15.05.1862 – 21.10.1931), \emph{Schriftsteller, Mediziner}!Lieutenant Gustl. Novelle1900-12-25@\strich\emph{Lieutenant Gustl. Novelle} {[}1900-12-25{]}|pwk}. In: \emph{Neue Freie Presse}\pwindex{Neue Freie Presse1864 – 1939@\emph{Neue Freie Presse} {[}1864 – 1939{]}|pwk}, Nr. 13.053, 25. 12. 1900, Morgenblatt, S. 34–41.}}}\label{K_L02946-1h} daher noch nicht
               geleſen.\pend
           \pstart
           Gibſt Du die »\textsc{Beatrice\pwindex{Schnitzler, Arthur 15.05.1862 – 21.10.1931@\textsc{Schnitzler, Arthur} (15.05.1862 – 21.10.1931), \emph{Schriftsteller, Mediziner}!Schleier der Beatrice. Schauspiel in fuenf Akten1900-12-01@\strich\emph{Der Schleier der Beatrice. Schauspiel in fünf Akten} {[}1900-12-01{]}|pw}}« dem \label{K_L02946-2v}\edtext{»Volkstheater\orgindex{Volkstheater@Volkstheater|pw}«}{\lemma{\textnormal{\emph{»Volkstheater«}}}\Cendnote{\textnormal{Siehe Paul Goldmann an Arthur Schnitzler, 21. 6. [1900] und 9. 12. [1900].
               }}}\label{K_L02946-2h}? Du ſollteſt es entſchieden thun. Auch mein Onkel\pwindex{Mamroth, Fedor 21.02.1851 – 25.06.1907@\textsc{Mamroth, Fedor} (21.02.1851 – 25.06.1907), \emph{Journalist, Kritiker}|pwv} iſt der Anſicht.\pend
           \pstart
           Meine Feuilletons ſammeln? Nie im Leben finde ich einen Verleger. Man weiſt mich mit
               Hohnlachen zurück, wenn ich mit ſo etwas komme.\pend
           \pstart
           Sei ſo gut und ſchreib mir ein Wort hierher an die obige Adreſſe meines Schwagers \textsc{Dr. Rosengart\pwindex{Rosengart, Josef 1860-02-08 – 1927-08-04@\textsc{Rosengart, Josef} (1860-02-08 – 1927-08-04), \emph{Arzt}|pw}}.\pend
           \pstart
           Bitte auch Deiner Frau Mutter\pwindex{Schnitzler, Louise 1840-07-08 – 1911-09-09@\textsc{Schnitzler, Louise} (1840-07-08 – 1911-09-09)|pwv}, Deinem Bruder\pwindex{Schnitzler, Julius 13.07.1865 – 29.06.1939@\textsc{Schnitzler, Julius} (13.07.1865 – 29.06.1939), \emph{Mediziner}|pwv} und Deiner Schwägerin\pwindex{Schnitzler, Helene 16.07.1871 – September 1941@\textsc{Schnitzler, Helene} (16.07.1871 – September 1941)|pwv}, {\pb}Deiner Schweſter\pwindex{Hajek, Gisela 20.12.1867 – 03.02.1953@\textsc{Hajek, Gisela} (20.12.1867 – 03.02.1953)|pwv} und Deinem Schwager\pwindex{Hajek, Gisela 20.12.1867 – 03.02.1953@\textsc{Hajek, Gisela} (20.12.1867 – 03.02.1953)|pwv} meine herzlichſten
               Neujahrs-Glückwünſche zu übermitteln.\pend
           \pstart
           Viele treue Grüße! {\\[\baselineskip]}Dein {\\[\baselineskip]}\spacefill\mbox{Paul Goldmann.}\pend
           \leftskip=0em{}
         
         \endnumbering\mylabel{h}\end{ledgroupsized}  \newcommand{\dateiname}{L02946}\newcommand{\titel}{Paul Goldmann an Arthur Schnitzler, 27. 12. [1900]}\newcommand{\editorInnen}{Martin Anton Müller und Laura Untner}%% latex-leseansicht-abspann.tex
%% Abspann für die Leseansicht.
%% Der Schalter \ifkorrekturansicht ist bereits durch den Vorspann gesetzt.

%% latex-abspann.tex
%% Gemeinsamer Abspann für Korrekturansicht und Leseansicht.
%% Setzt den Schalter \ifkorrekturansicht voraus (gesetzt in den
%% einbindenden Dateien latex-korrekturansicht-abspann.tex bzw.
%% latex-leseansicht-abspann.tex).
%% ---------------------------------------------------------------

\normalsize

% Das esempio-Environment wird nur in der Leseansicht benötigt
\ifkorrekturansicht\else
\newenvironment{esempio}[3]%
{
    \vspace{1.5ex}
    \rlap{\underline{#1}}
    \par
    \setlength{\parindent}{0cm}
    \nopagebreak
    \leftskip=#2cm
    \rightskip=#3cm
}
{
    \par
}
\fi

\doendnotes{C}
\bigskip
\vfill

\clearpage

\footnotesize

\ifkorrekturansicht
  \lohead{\textsc{register}}
\fi

% theindex-Environment neu definieren ohne reledmac
\makeatletter
\renewenvironment{theindex}{%
  \ifkorrekturansicht
    \section*{\indexname}%
  \else
    \subsubsection*{Index der erwähnten Entitäten}%
  \fi
  \setlength{\parindent}{0pt}%
  \setlength{\parskip}{0pt plus 0.3pt}%
  \let\item\@idxitem
}{%
  \ifkorrekturansicht\clearpage\fi
}
\makeatother

\IfFileExists{\jobname-pw.ind}{\input{\jobname-pw.ind}}{}

% Quellenangabe nur in der Leseansicht
\ifkorrekturansicht\else
% Fallback-Definitionen, falls die .tex-Datei \titel etc. nicht gesetzt hat
\providecommand{\titel}{}
\providecommand{\editorInnen}{}
\providecommand{\dateiname}{\jobname}

\vspace{3cm}

\vfill

\footnotesize
\textsc{Quelle}: \titel. Herausgegeben von {\editorInnen}. In: \emph{Arthur Schnitzler: Briefwechsel mit Autorinnen und Autoren}.
 Digitale Edition, https://schnitzler-briefe.acdh.oeaw.ac.at/{\dateiname}.html (Stand \today)
\fi

\end{document}


      