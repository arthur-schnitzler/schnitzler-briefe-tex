%% latex-korrekturansicht-vorspann.tex
%% Vorspann für die Korrekturansicht.
%% Lädt die gemeinsame Datei latex-vorspann.tex mit gesetztem Schalter.

\newif\ifkorrekturansicht
\korrekturansichttrue

\input{../tex-inputs/latex-vorspann}


\section[ Paul Goldmann an Arthur Schnitzler, 23. 9. {[}1901{]}]{L03085 Paul Goldmann an Arthur Schnitzler, 23. 9. {[}1901{]}}
\nopagebreak\mylabel{L03085v}
\rehead{ }\normalsize\beginnumbering\briefempfaengerindex{Schnitzler, Arthur@\textsc{Schnitzler, Arthur}!zzzGoldmann, Paul@\emph{von Paul Goldmann}!1901-09-231@{23. 9. {[}1901{]}}|(be}
\toendnotes[C]{\smallbreak\pagebreak[2]}\Standort{DLA, A:Schnitzler, HS.NZ85.1.3171.}
\physDesc{Brief, 1 Blatt, 4 Seiten, 1512 Zeichen
\newline{}Handschrift: blaue Tinte, deutsche Kurrent
\newline{}Schnitzler: 1) mit Bleistift das Jahr »901« vermerkt  2) mit rotem Buntstift vier Unterstreichungen}\toendnotes[C]{\smallbreak}
\pstart
           \raggedleft{}{\pb}\textcolor{gray}{\textbf{DESSAUERSTRASSE 19}}\oindex{Dessauer Strasse@\textbf{Dessauer Straße}, \emph{Straße (K.STR)}|pw}\pend
           
\pstart
           Berlin\oindex{Berlin@\textbf{Berlin}, \emph{P.PPLC}|pw}, 23. Sept.\pend
           
\pstart\center{}Mein lieber Freund,\pend\vspace{0.5em}
\pstart
           Die \textsc{Triesch\pwindex{Triesch, Irene 13.04.1877 – 24.11.1964@\textsc{Triesch, Irene} (13.04.1877 – 24.11.1964), \emph{Schauspieler/Schauspielerin}|pw}} hat bereits die \label{K_L03085-1v}\edtext{Rollen in Deinen
                  Stücken\pwindex{Lebendige Stunden. Vier Einakter@\emph{Lebendige Stunden. Vier Einakter}|pwv}}{\lemma{\textnormal{\emph{Rollen in Deinen
                  Stücken}}}\Cendnote{\textnormal{In \emph{Die
                     Frau mit dem Dolche}\pwindex{Frau mit dem Dolche@\emph{Die Frau mit dem Dolche}|pwk} spielte Irene
                     Triesch\pwindex{Triesch, Irene 13.04.1877 – 24.11.1964@\textsc{Triesch, Irene} (13.04.1877 – 24.11.1964), \emph{Schauspieler/Schauspielerin}|pwk} am Deutschen Theater Berlin\oindex{Deutsches Theater Berlin@\textbf{Deutsches Theater Berlin}, \emph{Theater (K.THE)}|pwk} die
                  Rolle der Pauline\pwindex{Frau mit dem Dolche@\emph{Die Frau mit dem Dolche}|pwkv} und in
                     \emph{Literatur}\pwindex{Literatur@\emph{Literatur}|pwk} jene der Margarete\pwindex{Literatur@\emph{Literatur}|pwkv}.}}}\label{K_L03085-1} bekommen und iſt
               namentlich von de\substVorne{}\textsuperscript{m}\substDazwischen{}r\substHinten{}{ }Frau mit dem Dolch\pwindex{Frau mit dem Dolche@\emph{Die Frau mit dem Dolche}|pw} entzückt. Hat ſie ſich auch
               bereits recht hübſch zurechtgelegt\strikeout{.} und ſpricht jeden
               Tag Goethe\pwindex{Goethe, Johann Wolfgang von 1749-08-28 – 1832-03-22@\textsc{Goethe, Johann Wolfgang von} (1749-08-28 – 1832-03-22), \emph{Schriftsteller/Schriftstellerin}|pw}ſche Verſe, um ſich im
               Verſe-Recitiren zu üben. Sie will nach \textsc{München\oindex{Muenchen@\textbf{München}, \emph{P.PPLA}|pw}} fahren und \textsc{Lenbach\pwindex{Lenbach, Franz 1836-10-13 – 1904-05-06@\textsc{Lenbach, Franz} (1836-10-13 – 1904-05-06), \emph{Maler/Malerin}|pw}} oder \textsc{Stuck\pwindex{Stuck, Franz von 1863-02-23 – 1928-08-30@\textsc{Stuck, Franz von} (1863-02-23 – 1928-08-30), \emph{Maler/Malerin}|pw}} bitten, das betreffende \label{K_L03085-2v}\edtext{Bild}{\lemma{\textnormal{\emph{Bild}}}\Cendnote{\textnormal{Franz Lenbach\pwindex{Lenbach, Franz 1836-10-13 – 1904-05-06@\textsc{Lenbach, Franz} (1836-10-13 – 1904-05-06), \emph{Maler/Malerin}|pwk} und Franz von Stuck\pwindex{Stuck, Franz von 1863-02-23 – 1928-08-30@\textsc{Stuck, Franz von} (1863-02-23 – 1928-08-30), \emph{Maler/Malerin}|pwk} waren Münch\oindex{Muenchen@\textbf{München}, \emph{P.PPLA}|pwkv}ner Maler. In \emph{Die
                     Frau mit dem Dolche}\pwindex{Frau mit dem Dolche@\emph{Die Frau mit dem Dolche}|pwk} verbindet ein Renaissance-Bild die Gegenwartshandlung
                  mit einem historischen Teil.}}}\label{K_L03085-2} zu {\pb}entwerfen, was gar nicht übel wäre.\pend
           
\pstart
           \strikeout{Daß} Wann \strikeout{kommſt}{ }\label{K_L03085-3v}\edtext{kommſt}{\lemma{\textnormal{\emph{kommſt}}}\Cendnote{\textnormal{Schnitzler kam für die Uraufführung von \emph{Lebendige Stunden}\pwindex{Lebendige Stunden. Vier Einakter@\emph{Lebendige Stunden. Vier Einakter}|pwk} (4. 1. 1902, \emph{Deutsches Theater}\orgindex{Deutsches Theater Berlin@Deutsches Theater Berlin|pwk}) nach Berlin\oindex{Berlin@\textbf{Berlin}, \emph{P.PPLC}|pwk}. Er blieb vom 28. 12. 1901 bis zum 6. 1. 1902.}}}\label{K_L03085-3} Du?\pend
           
\pstart
           Daß Du mir \textsc{Kerrs\pwindex{Kerr, Alfred 25.12.1867 – 12.10.1948@\textsc{Kerr, Alfred} (25.12.1867 – 12.10.1948), \emph{Schriftsteller/Schriftstellerin, Kritiker/Kritikerin}|pw}} Beſuch in Berlin\oindex{Berlin@\textbf{Berlin}, \emph{P.PPLC}|pw} verſchwiegen haſt, iſt bedauerlich. Immerhin wirſt Du
               bei unſerem nächſten Beiſammenſein behaupten, es mir geſchrieben zu haben.\pend
           
\pstart
           \textsc{Salten\pwindex{Salten, Felix 06.09.1869 – 08.10.1945@\textsc{Salten, Felix} (06.09.1869 – 08.10.1945), \emph{Schriftsteller/Schriftstellerin, Journalist/Journalistin, Chefredakteur/Chefredakteurin}|pw}} iſt morgen bei mir zu Tiſch.\pend
           
\pstart
           \textsc{Peter Dorner\pwindex{Dorner, Peter 17.02.1857 – 01.04.1931@\textsc{Dorner, Peter} (17.02.1857 – 01.04.1931), \emph{Schmied/Schmiedin, Kunsthandwerker/Kunsthandwerkerin, Kunstschmied/Kunstschmiedin}|pw}}, denke Dir!, ſchickte {\pb}mir dieſelben \label{K_L03085-4v}\edtext{Bücher}{\lemma{\textnormal{\emph{Bücher}}}\Cendnote{\textnormal{nicht ermittelt}}}\label{K_L03085-4}, die er Dir geſandt. Ich habe ihm
               ein ſchönes \label{K_L03085-5v}\edtext{Werk über Schmiede\substVorne{}\textsuperscript{\textcolor{gray}{r}}\substDazwischen{}a\substHinten{}rbeit\pwindex{Schmiedekunst nach Originalen des XV. bis XVIII. Jahrhunderts@\emph{Die Schmiedekunst nach Originalen des XV. bis XVIII. Jahrhunderts}|pwuv} mit Nachbildungen alter Meiſterſtücke}{\lemma{\textnormal{\emph{Werk … Meiſterſtücke}}}\Cendnote{\textnormal{Möglicherweise: \emph{Die Schmiedekunst nach Originalen des XV. bis
                        XVIII. Jahrhunderts}\pwindex{Schmiedekunst nach Originalen des XV. bis XVIII. Jahrhunderts@\emph{Die Schmiedekunst nach Originalen des XV. bis XVIII. Jahrhunderts}|pwk}. Berlin\oindex{Berlin@\textbf{Berlin}, \emph{P.PPLC}|pwk}: \emph{Verlag von Ernst Wasmuth}\orgindex{Verlag von Ernst Wasmuth@Verlag von Ernst Wasmuth|pwk}{ }1887.}}}\label{K_L03085-5}, im Betrage von 30 \textsc{MK}, als Gegengeſchenk
               geſandt. Dann gibt es ein noch viel ſchöneres Werk\pwindex{Schmiedekunst nach Originalen des XV. bis XVIII. Jahrhunderts@\emph{Die Schmiedekunst nach Originalen des XV. bis XVIII. Jahrhunderts}|pwuv} derſelben Art, das 44 \textsc{MK} koſtet, betitelt »\label{K_L03085-6v}\edtext{Die deutſche Schmiedekunſt\pwindex{Schmiedekunst seit dem Ende der Renaissance@\emph{Die Schmiedekunst seit dem Ende der Renaissance}|pwu}}{\lemma{\textnormal{\emph{Die … Schmiedekunſt}}}\Cendnote{\textnormal{Vermutlich: Adolf Brüning\pwindex{Bruening, Adolf Oktober 1867 – 1912-02-02@\textsc{Brüning, Adolf} (Oktober 1867 – 1912-02-02), \emph{Museumsdirektor/Museumsdirektorin, Klassischer Philologe/Klassische Philologin, Kunsthistoriker/Kunsthistorikerin}|pwk}: \emph{Die Schmiedekunst seit dem Ende der Renaissance}\pwindex{Schmiedekunst seit dem Ende der Renaissance@\emph{Die Schmiedekunst seit dem Ende der Renaissance}|pwk}.
                     Mit 150 Abbildungen. Leipzig\oindex{Leipzig@\textbf{Leipzig}, \emph{P.PPLA3}|pwk}: \emph{Verlag von Hermann Seemann
                        Nachfolger}\orgindex{Hermann Seemann Nachfolger@Hermann Seemann Nachfolger|pwk} [1901?].}}}\label{K_L03085-6}«. Mir allein
               iſt es zu theuer. Möchteſt Du Dich mit der Hälfte betheiligen? Davon würde der Mann\pwindex{Dorner, Peter 17.02.1857 – 01.04.1931@\textsc{Dorner, Peter} (17.02.1857 – 01.04.1931), \emph{Schmied/Schmiedin, Kunsthandwerker/Kunsthandwerkerin, Kunstschmied/Kunstschmiedin}|pwv} wenigſtens \strikeout{etwas} etwas Ordentliches {\pb}profitiren.\pend
           
\pstart
           Danke den lieben Mädchen\pwindex{Schnitzler, Olga 17.01.1882 – 13.01.1970@\textsc{Schnitzler, Olga} (17.01.1882 – 13.01.1970), \emph{Schauspieler/Schauspielerin, Sänger/Sängerin}|pwv}\pwindex{Steinrueck, Elisabeth 19.11.1885 – 07.04.1920@\textsc{Steinrück, Elisabeth} (19.11.1885 – 07.04.1920)|pwv} in meinem Namen für ihre reizenden Briefe, die mich unendlich
               erfreut haben. Sie ſollen mir nicht böſe ſein, daß ich nicht gleich antworte; aber
               ich ſtecke tief in der Arbeit. Nächſter Tage ſchreibe ich ihnen. Iſt die Adreſſe
               immer noch \textsc{Maximilianplatz\oindex{Rooseveltplatz@\textbf{Rooseveltplatz}, \emph{Platz (K.PLT)}|pw}}?\pend
           
\pstart
           Viele treue Grüße {\\[\baselineskip]}Dein {\\[\baselineskip]}\spacefill\mbox{Paul Goldmnn}\pend
           \leftskip=0em{}
\pstart
           \noindent{}\label{T_L03085-1v}\edtext{{\pb}Lies’ in der letzten »Zeit\pwindex{Zeit. Wiener Wochenschrift@\emph{Die Zeit. Wiener Wochenschrift}|pw}« die ſchöne Geſpenſtergeſchichte \label{K_L03085-7v}\edtext{»Das
                     rothe Zimmer\pwindex{rothe Zimmer@\emph{Das rothe Zimmer}|pw}«}{\lemma{\textnormal{\emph{»Das
                     rothe Zimmer«}}}\Cendnote{\textnormal{H. G. Wells\pwindex{Wells, H. G. 21.09.1866 – 13.08.1946@\textsc{Wells, H. G.} (21.09.1866 – 13.08.1946), \emph{Schriftsteller/Schriftstellerin}|pwk}: \emph{Das rothe Zimmer}\pwindex{rothe Zimmer@\emph{Das rothe Zimmer}|pwk}. Übersetzt von M. v. Berthof\pwindex{Berthof, M. v. @\textsc{Berthof, M. v.}, \emph{Übersetzer/Übersetzerin}|pwk}. In: \emph{Die Zeit. Wiener Wochenschrift}\pwindex{Zeit. Wiener Wochenschrift@\emph{Die Zeit. Wiener Wochenschrift}|pwk}, Bd. 28, Nr. 364, 21. 9. 1901, S. 190–192.}}}\label{K_L03085-7}.{\\}\label{K_L03085-8v}\edtext{\textsc{Chamfort\pwindex{Chamfort, Sebastien Roch Nicolas 06.04.1741 – 13.04.1794@\textsc{Chamfort, Sébastien Roch Nicolas} (06.04.1741 – 13.04.1794), \emph{Schriftsteller/Schriftstellerin}|pw}}}{\lemma{\textnormal{\emph{Chamfort}}}\Cendnote{\textnormal{\emph{Œuvres choisies de N. Chamfort, publiées avec
                        préface, notes et tables}\pwindex{Œuvres choisies de N. Chamfort, publiees avec preface, notes et tables@\emph{Œuvres choisies de N. Chamfort, publiées avec préface, notes et tables}|pwk}. Herausgegeben von Adolphe Mathurin de Lescure\pwindex{Lescure, Adolphe Mathurin de 1833 – 1892-05-06@\textsc{Lescure, Adolphe Mathurin de} (1833 – 1892-05-06), \emph{Journalist/Journalistin, Chefredakteur/Chefredakteurin, Sekretär/Sekretärin}|pwk}. Paris\oindex{Paris@\textbf{Paris}, \emph{P.PPLC}|pwk}: \emph{Flammarion}\orgindex{Flammarion@Flammarion|pwk}{ }1892.}}}\label{K_L03085-8} (\begin{otherlanguage}{french}\textsc{Œuvres choisies}\end{otherlanguage}, in 2 Bänden\pwindex{Œuvres choisies de N. Chamfort, publiees avec preface, notes et tables@\emph{Œuvres choisies de N. Chamfort, publiées avec préface, notes et tables}|pw}) iſt bei \textsc{Flammarion\orgindex{Flammarion@Flammarion|pw}} erſchienen.}{\lemma{\textnormal{\emph{Lies’ … erſchienen.}}}\Cendnote{\textnormal{kopfüber am oberen
                     Rand der ersten Seite}}}\label{T_L03085-1}\pend
           \selectlanguage{ngerman}\endnumbering\briefempfaengerindex{Schnitzler, Arthur@\textsc{Schnitzler, Arthur}!zzzGoldmann, Paul@\emph{von Paul Goldmann}!1901-09-231@{23. 9. {[}1901{]}}|)be}\mylabel{L03085h}  \normalsize

\doendnotes{C}
\bigskip
\vfill

\clearpage

\footnotesize

\lohead{\textsc{register}}

% Definiere theindex-Environment komplett neu ohne reledmac
\makeatletter
\renewenvironment{theindex}{%
  \section*{\indexname}%
  \setlength{\parindent}{0pt}%
  \setlength{\parskip}{0pt plus 0.3pt}%
  \let\item\@idxitem
}{%
  \clearpage
}
\makeatother

\IfFileExists{\jobname-pw.ind}{\input{\jobname-pw.ind}}{}

\end{document}

      