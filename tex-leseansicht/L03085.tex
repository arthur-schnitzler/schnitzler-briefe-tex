%% latex-leseansicht-vorspann.tex
%% Vorspann für die Leseansicht.
%% Lädt die gemeinsame Datei latex-vorspann.tex mit nicht gesetztem Schalter.

\newif\ifkorrekturansicht
\korrekturansichtfalse

\input{../tex-inputs/latex-vorspann}


         
         \renewcommand{\erwaehntePersonen}{Personen: M. v. Berthof, Adolf Brüning, Sébastien Roch Nicolas Chamfort, Peter Dorner, Johann Wolfgang von Goethe, Alfred Kerr, Franz Lenbach, Adolphe Mathurin de Lescure, Felix Salten, Olga Schnitzler, Elisabeth Steinrück, Franz von Stuck, Irene Triesch, H. G. Wells}
         \renewcommand{\erwaehnteInstitutionen}{Institutionen: Deutsches Theater Berlin, Flammarion, Hermann Seemann Nachfolger, Verlag von Ernst Wasmuth}
         \renewcommand{\erwaehnteOrte}{Orte: Berlin, Dessauer Straße, Deutsches Theater Berlin, Leipzig, München, Paris, Rooseveltplatz, Wien}
         \renewcommand{\erwaehnteWerke}{Werke: Das rothe Zimmer, Die Frau mit dem Dolche, Die Schmiedekunst nach Originalen des XV. bis XVIII. Jahrhunderts, Die Schmiedekunst seit dem Ende der Renaissance, Die Zeit. Wiener Wochenschrift, Lebendige Stunden. Vier Einakter, Literatur, Œuvres choisies de N. Chamfort, publiées avec préface, notes et tables}
               \section[ Paul Goldmann an Arthur Schnitzler, 23. 9. {[}1901{]}]{ Paul Goldmann an Arthur Schnitzler, 23. 9. {[}1901{]}}\nopagebreak\mylabel{v}\rehead{ }\begin{ledgroupsized}[t]{13cm}\normalsize\beginnumbering \toendnotes[C]{\smallbreak\pagebreak[2]} \Standort{DLA, A:Schnitzler, HS.NZ85.1.3171.}
\physDesc{Brief, 1 Blatt, 4 Seiten, 1512 Zeichen
\newline{}Handschrift: blaue Tinte, deutsche Kurrent
\newline{}Schnitzler: 1) mit Bleistift das Jahr »{[}1{]}901« vermerkt  2) mit rotem Buntstift vier Unterstreichungen}\toendnotes[C]{\smallbreak}\pstart
           \noindent{}\raggedleft{}{\pb}\textcolor{gray}{\textbf{DESSAUERSTRASSE 19}}\oindex{Dessauer Strasse@\textbf{Dessauer Straße}|pw}\pend
           \pstart
           Berlin\oindex{Berlin@\textbf{Berlin}|pw}, 23. Sept.\pend
           \pstart\center{}Mein lieber Freund,\pend\pstart
           Die \textsc{Triesch\pwindex{Triesch, Irene 13.04.1877 – 24.11.1964@\textsc{Triesch, Irene} (13.04.1877 – 24.11.1964), \emph{Schauspielerin}|pw}} hat bereits die \label{K_L03085-1v}\edtext{Rollen in Deinen
                  Stücken\pwindex{Schnitzler, Arthur 15.05.1862 – 21.10.1931@\textsc{Schnitzler, Arthur} (15.05.1862 – 21.10.1931), \emph{Schriftsteller, Mediziner}!Lebendige Stunden. Vier Einakter1901-12-23@\strich\emph{Lebendige Stunden. Vier Einakter} {[}1901-12-23{]}|pwv}}{\lemma{\textnormal{\emph{Rollen in Deinen
                  Stücken}}}\Cendnote{\textnormal{In \emph{Die
                     Frau mit dem Dolche}\pwindex{Schnitzler, Arthur 15.05.1862 – 21.10.1931@\textsc{Schnitzler, Arthur} (15.05.1862 – 21.10.1931), \emph{Schriftsteller, Mediziner}!Frau mit dem Dolche1901@\strich\emph{Die Frau mit dem Dolche} {[}1901{]}|pwk} spielte Irene
                     Triesch\pwindex{Triesch, Irene 13.04.1877 – 24.11.1964@\textsc{Triesch, Irene} (13.04.1877 – 24.11.1964), \emph{Schauspielerin}|pwk} am Deutschen Theater Berlin\oindex{Deutsches Theater Berlin@\textbf{Deutsches Theater Berlin}|pwk} die
                  Rolle der Pauline\pwindex{Schnitzler, Arthur 15.05.1862 – 21.10.1931@\textsc{Schnitzler, Arthur} (15.05.1862 – 21.10.1931), \emph{Schriftsteller, Mediziner}!Frau mit dem Dolche1901@\strich\emph{Die Frau mit dem Dolche} {[}1901{]}|pwkv} und in
                     \emph{Literatur}\pwindex{Schnitzler, Arthur 15.05.1862 – 21.10.1931@\textsc{Schnitzler, Arthur} (15.05.1862 – 21.10.1931), \emph{Schriftsteller, Mediziner}!Literatur1901@\strich\emph{Literatur} {[}1901{]}|pwk} jene der Margarete\pwindex{Schnitzler, Arthur 15.05.1862 – 21.10.1931@\textsc{Schnitzler, Arthur} (15.05.1862 – 21.10.1931), \emph{Schriftsteller, Mediziner}!Literatur1901@\strich\emph{Literatur} {[}1901{]}|pwkv}.}}}\label{K_L03085-1h} bekommen und iſt
               namentlich von de\substVorne{}\textsuperscript{m}\substDazwischen{}r\substHinten{}{ }Frau mit dem Dolch\pwindex{Schnitzler, Arthur 15.05.1862 – 21.10.1931@\textsc{Schnitzler, Arthur} (15.05.1862 – 21.10.1931), \emph{Schriftsteller, Mediziner}!Frau mit dem Dolche1901@\strich\emph{Die Frau mit dem Dolche} {[}1901{]}|pw} entzückt. Hat ſie ſich auch
               bereits recht hübſch zurechtgelegt\strikeout{.} und ſpricht jeden
               Tag Goethe\pwindex{Goethe, Johann Wolfgang von 1749-08-28 – 1832-03-22@\textsc{Goethe, Johann Wolfgang von} (1749-08-28 – 1832-03-22), \emph{Schriftsteller}|pw}ſche Verſe, um ſich im
               Verſe-Recitiren zu üben. Sie will nach \textsc{München\oindex{Muenchen@\textbf{München}|pw}} fahren und \textsc{Lenbach\pwindex{Lenbach, Franz 1836-10-13 – 1904-05-06@\textsc{Lenbach, Franz} (1836-10-13 – 1904-05-06), \emph{Bildender Künstler}|pw}} oder \textsc{Stuck\pwindex{Stuck, Franz von 1863-02-23 – 1928-08-30@\textsc{Stuck, Franz von} (1863-02-23 – 1928-08-30), \emph{Bildender Künstler}|pw}} bitten, das betreffende \label{K_L03085-2v}\edtext{Bild}{\lemma{\textnormal{\emph{Bild}}}\Cendnote{\textnormal{Franz Lenbach\pwindex{Lenbach, Franz 1836-10-13 – 1904-05-06@\textsc{Lenbach, Franz} (1836-10-13 – 1904-05-06), \emph{Bildender Künstler}|pwk} und Franz von Stuck\pwindex{Stuck, Franz von 1863-02-23 – 1928-08-30@\textsc{Stuck, Franz von} (1863-02-23 – 1928-08-30), \emph{Bildender Künstler}|pwk} waren Münch\oindex{Muenchen@\textbf{München}|pwkv}ner Maler. In \emph{Die
                     Frau mit dem Dolche}\pwindex{Schnitzler, Arthur 15.05.1862 – 21.10.1931@\textsc{Schnitzler, Arthur} (15.05.1862 – 21.10.1931), \emph{Schriftsteller, Mediziner}!Frau mit dem Dolche1901@\strich\emph{Die Frau mit dem Dolche} {[}1901{]}|pwk} verbindet ein Renaissance-Bild die Gegenwartshandlung
                  mit einem historischen Teil.}}}\label{K_L03085-2h} zu {\pb}entwerfen, was gar nicht übel wäre.\pend
           \pstart
           \strikeout{Daß} Wann \strikeout{kommſt}{ }\label{K_L03085-3v}\edtext{kommſt}{\lemma{\textnormal{\emph{kommſt}}}\Cendnote{\textnormal{Schnitzler\pwindex{Schnitzler, Arthur 15.05.1862 – 21.10.1931@\textsc{Schnitzler, Arthur} (15.05.1862 – 21.10.1931), \emph{Schriftsteller, Mediziner}|pwk} kam für die Uraufführung von \emph{Lebendige Stunden}\pwindex{Schnitzler, Arthur 15.05.1862 – 21.10.1931@\textsc{Schnitzler, Arthur} (15.05.1862 – 21.10.1931), \emph{Schriftsteller, Mediziner}!Lebendige Stunden. Vier Einakter1901-12-23@\strich\emph{Lebendige Stunden. Vier Einakter} {[}1901-12-23{]}|pwk} (4. 1. 1902, \emph{Deutsches Theater}\orgindex{Deutsches Theater Berlin@Deutsches Theater Berlin|pwk}) nach Berlin\oindex{Berlin@\textbf{Berlin}|pwk}. Er blieb von 28. 12. 1901 bis 6. 1. 1902.}}}\label{K_L03085-3h} Du?\pend
           \pstart
           Daß Du mir \textsc{Kerr\pwindex{Kerr, Alfred 25.12.1867 – 12.10.1948@\textsc{Kerr, Alfred} (25.12.1867 – 12.10.1948), \emph{Schriftsteller, Kritiker}|pw}s} Beſuch in Berlin\oindex{Berlin@\textbf{Berlin}|pw} verſchwiegen haſt, iſt bedauerlich. Immerhin wirſt Du
               bei unſerem nächſten Beiſammenſein behaupten, es mir geſchrieben zu haben.\pend
           \pstart
           \textsc{Salten\pwindex{Salten, Felix 06.09.1869 – 08.10.1945@\textsc{Salten, Felix} (06.09.1869 – 08.10.1945), \emph{Schriftsteller, Journalist}|pw}} iſt morgen bei mir zu Tiſch.\pend
           \pstart
           \textsc{Peter Dorner\pwindex{Dorner, Peter 17.02.1857 – 01.04.1931@\textsc{Dorner, Peter} (17.02.1857 – 01.04.1931), \emph{Schmied, Kunsthandwerker, Kunstschmied}|pw}}, denke Dir!, ſchickte {\pb}mir dieſelben \label{K_L03085-123v}\edtext{Bücher}{\lemma{\textnormal{\emph{Bücher}}}\Cendnote{\textnormal{nicht ermittelt}}}\label{K_L03085-123h}, die er Dir geſandt. Ich habe ihm
               ein ſchönes \label{K_L03085-1111v}\edtext{Werk über Schmiede\substVorne{}\textsuperscript{\textcolor{gray}{r}}\substDazwischen{}a\substHinten{}rbeit\pwindex{?? Werk@Nicht ermittelte Verfasserinnen und Verfasser!Schmiedekunst nach Originalen des XV. bis XVIII. Jahrhunderts1887@\emph{Die Schmiedekunst nach Originalen des XV. bis XVIII. Jahrhunderts} {[}1887{]}|pwuv} mit Nachbildungen alter Meiſterſtücke}{\lemma{\textnormal{\emph{Werk … Meiſterſtücke}}}\Cendnote{\textnormal{Möglicherweise: \emph{Die Schmiedekunst nach Originalen des XV. bis
                        XVIII. Jahrhunderts}\pwindex{?? Werk@Nicht ermittelte Verfasserinnen und Verfasser!Schmiedekunst nach Originalen des XV. bis XVIII. Jahrhunderts1887@\emph{Die Schmiedekunst nach Originalen des XV. bis XVIII. Jahrhunderts} {[}1887{]}|pwk}. Berlin\oindex{Berlin@\textbf{Berlin}|pwk}: \emph{Verlag von Ernst Wasmuth}\orgindex{Verlag von Ernst Wasmuth@Verlag von Ernst Wasmuth|pwk}{ }1887.}}}\label{K_L03085-1111h}, im Betrage von 30 \textsc{MK}, als Gegengeſchenk
               geſandt. Dann gibt es ein noch viel ſchöneres Werk\pwindex{?? Werk@Nicht ermittelte Verfasserinnen und Verfasser!Schmiedekunst nach Originalen des XV. bis XVIII. Jahrhunderts1887@\emph{Die Schmiedekunst nach Originalen des XV. bis XVIII. Jahrhunderts} {[}1887{]}|pwuv} derſelben Art, das 44 \textsc{MK} koſtet, betitelt »\label{K_L03085-13v}\edtext{Die deutſche Schmiedekunſt\pwindex{Bruening, Adolf Oktober 1867 – 1912-02-02@\textsc{Brüning, Adolf} (Oktober 1867 – 1912-02-02), \emph{Kunsthistoriker, Museumsdirektor, Klassischer Philologe}!Schmiedekunst seit dem Ende der Renaissance1901@\strich\emph{Die Schmiedekunst seit dem Ende der Renaissance} {[}1901{]}|pwu}}{\lemma{\textnormal{\emph{Die … Schmiedekunſt}}}\Cendnote{\textnormal{Vermutlich: Adolf Brüning\pwindex{Bruening, Adolf Oktober 1867 – 1912-02-02@\textsc{Brüning, Adolf} (Oktober 1867 – 1912-02-02), \emph{Kunsthistoriker, Museumsdirektor, Klassischer Philologe}|pwk}: \emph{Die Schmiedekunst seit dem Ende der Renaissance}\pwindex{Bruening, Adolf Oktober 1867 – 1912-02-02@\textsc{Brüning, Adolf} (Oktober 1867 – 1912-02-02), \emph{Kunsthistoriker, Museumsdirektor, Klassischer Philologe}!Schmiedekunst seit dem Ende der Renaissance1901@\strich\emph{Die Schmiedekunst seit dem Ende der Renaissance} {[}1901{]}|pwk}.
                     Mit 150 Abbildungen. Leipzig\oindex{Leipzig@\textbf{Leipzig}|pwk}: \emph{Verlag von Hermann Seemann
                        Nachfolger}\orgindex{Hermann Seemann Nachfolger@Hermann Seemann Nachfolger|pwk} [1901?].}}}\label{K_L03085-13h}«. Mir allein
               iſt es zu theuer. Möchteſt Du Dich mit der Hälfte betheiligen? Davon würde der Mann\pwindex{Dorner, Peter 17.02.1857 – 01.04.1931@\textsc{Dorner, Peter} (17.02.1857 – 01.04.1931), \emph{Schmied, Kunsthandwerker, Kunstschmied}|pwv} wenigſtens \strikeout{etwas} etwas Ordentliches {\pb}profitiren.\pend
           \pstart
           Danke den lieben Mädchen\pwindex{Schnitzler, Olga 17.01.1882 – 13.01.1970@\textsc{Schnitzler, Olga} (17.01.1882 – 13.01.1970), \emph{Schauspielerin, Sängerin}|pwv}\pwindex{Steinrueck, Elisabeth 19.11.1885 – 07.04.1920@\textsc{Steinrück, Elisabeth} (19.11.1885 – 07.04.1920)|pwv} in meinem Namen für ihre reizenden Briefe, die mich unendlich
               erfreut haben. Sie ſollen mir nicht böſe ſein, daß ich nicht gleich antworte; aber
               ich ſtecke tief in der Arbeit. Nächſter Tage ſchreibe ich ihnen. Iſt die Adreſſe
               immer noch \textsc{Maximilianplatz\oindex{Rooseveltplatz@\textbf{Rooseveltplatz}|pw}}?\pend
           \pstart
           Viele treue Grüße {\\[\baselineskip]}Dein {\\[\baselineskip]}\spacefill\mbox{Paul Goldmnn}\pend
           \leftskip=0em{}\pstart
           \noindent{}\label{T_L03085-1v}\edtext{{\pb}Lies’ in der letzten »Zeit\pwindex{Zeit. Wiener Wochenschrift1894 – 1904@\emph{Die Zeit. Wiener Wochenschrift} {[}1894 – 1904{]}|pw}« die ſchöne Geſpenſtergeſchichte \label{K_L03085-99v}\edtext{»Das
                     rothe Zimmer\pwindex{Wells, H. G. 21.09.1866 – 13.08.1946@\textsc{Wells, H. G.} (21.09.1866 – 13.08.1946), \emph{Schriftsteller}!rothe Zimmer1901-09-21@\strich\emph{Das rothe Zimmer} {[}1901-09-21{]}|pw}«}{\lemma{\textnormal{\emph{»Das
                     rothe Zimmer«}}}\Cendnote{\textnormal{H. G. Wells\pwindex{Wells, H. G. 21.09.1866 – 13.08.1946@\textsc{Wells, H. G.} (21.09.1866 – 13.08.1946), \emph{Schriftsteller}|pwk}: \emph{Das rothe Zimmer}\pwindex{Wells, H. G. 21.09.1866 – 13.08.1946@\textsc{Wells, H. G.} (21.09.1866 – 13.08.1946), \emph{Schriftsteller}!rothe Zimmer1901-09-21@\strich\emph{Das rothe Zimmer} {[}1901-09-21{]}|pwk}. Übersetzt von M. v. Berthof\pwindex{Berthof, M. v. @\textsc{Berthof, M. v.}, \emph{Übersetzer/Übersetzerin}|pwk}. In: \emph{Die Zeit. Wiener Wochenschrift}\pwindex{Zeit. Wiener Wochenschrift1894 – 1904@\emph{Die Zeit. Wiener Wochenschrift} {[}1894 – 1904{]}|pwk}, Bd. 28, Nr. 364, 21. 9. 1901, S. 190–192.}}}\label{K_L03085-99h}.{\\}\label{K_L03085-33v}\edtext{\textsc{Chamfort\pwindex{Chamfort, Sebastien Roch Nicolas 06.04.1741 – 13.04.1794@\textsc{Chamfort, Sébastien Roch Nicolas} (06.04.1741 – 13.04.1794), \emph{Schriftsteller}|pw}}}{\lemma{\textnormal{\emph{Chamfort}}}\Cendnote{\textnormal{\emph{Œuvres choisies de N. Chamfort, publiées avec
                        préface, notes et tables}\pwindex{Chamfort, Sebastien Roch Nicolas 06.04.1741 – 13.04.1794@\textsc{Chamfort, Sébastien Roch Nicolas} (06.04.1741 – 13.04.1794), \emph{Schriftsteller}!Œuvres choisies de N. Chamfort, publiees avec preface, notes et tables1892@\strich\emph{Œuvres choisies de N. Chamfort, publiées avec préface, notes et tables} {[}1892{]}|pwk}. 2 Bde. Hg. v. Adolphe Mathurin de Lescure\pwindex{Lescure, Adolphe Mathurin de 1833 – 1892-05-06@\textsc{Lescure, Adolphe Mathurin de} (1833 – 1892-05-06), \emph{Journalist, Redakteur, Sekretär}|pwk}. Paris\oindex{Paris@\textbf{Paris}|pwk}: \emph{Flammarion}\orgindex{Flammarion@Flammarion|pwk}{ }1892.}}}\label{K_L03085-33h} (\begin{otherlanguage}{french}\textsc{Œuvres choisies}\end{otherlanguage}, in 2 Bänden\pwindex{Chamfort, Sebastien Roch Nicolas 06.04.1741 – 13.04.1794@\textsc{Chamfort, Sébastien Roch Nicolas} (06.04.1741 – 13.04.1794), \emph{Schriftsteller}!Œuvres choisies de N. Chamfort, publiees avec preface, notes et tables1892@\strich\emph{Œuvres choisies de N. Chamfort, publiées avec préface, notes et tables} {[}1892{]}|pw}) iſt bei \textsc{Flammarion\orgindex{Flammarion@Flammarion|pw}} erſchienen.}{\lemma{\textnormal{\emph{Lies’ … erſchienen.}}}\Cendnote{\textnormal{kopfüber am oberen
                     Rand der ersten Seite}}}\label{T_L03085-1h}\pend
           
         
         \endnumbering\mylabel{h}\end{ledgroupsized}  \newcommand{\dateiname}{L03085}\newcommand{\titel}{Paul Goldmann an Arthur Schnitzler, 23. 9. [1901]}\newcommand{\editorInnen}{Martin Anton Müller und Laura Untner}%% latex-leseansicht-abspann.tex
%% Abspann für die Leseansicht.
%% Der Schalter \ifkorrekturansicht ist bereits durch den Vorspann gesetzt.

%% latex-abspann.tex
%% Gemeinsamer Abspann für Korrekturansicht und Leseansicht.
%% Setzt den Schalter \ifkorrekturansicht voraus (gesetzt in den
%% einbindenden Dateien latex-korrekturansicht-abspann.tex bzw.
%% latex-leseansicht-abspann.tex).
%% ---------------------------------------------------------------

\normalsize

% Das esempio-Environment wird nur in der Leseansicht benötigt
\ifkorrekturansicht\else
\newenvironment{esempio}[3]%
{
    \vspace{1.5ex}
    \rlap{\underline{#1}}
    \par
    \setlength{\parindent}{0cm}
    \nopagebreak
    \leftskip=#2cm
    \rightskip=#3cm
}
{
    \par
}
\fi

\doendnotes{C}
\bigskip
\vfill

\clearpage

\footnotesize

\ifkorrekturansicht
  \lohead{\textsc{register}}
\fi

% theindex-Environment neu definieren ohne reledmac
\makeatletter
\renewenvironment{theindex}{%
  \ifkorrekturansicht
    \section*{\indexname}%
  \else
    \subsubsection*{Index der erwähnten Entitäten}%
  \fi
  \setlength{\parindent}{0pt}%
  \setlength{\parskip}{0pt plus 0.3pt}%
  \let\item\@idxitem
}{%
  \ifkorrekturansicht\clearpage\fi
}
\makeatother

\IfFileExists{\jobname-pw.ind}{\input{\jobname-pw.ind}}{}

% Quellenangabe nur in der Leseansicht
\ifkorrekturansicht\else
% Fallback-Definitionen, falls die .tex-Datei \titel etc. nicht gesetzt hat
\providecommand{\titel}{}
\providecommand{\editorInnen}{}
\providecommand{\dateiname}{\jobname}

\vspace{3cm}

\vfill

\footnotesize
\textsc{Quelle}: \titel. Herausgegeben von {\editorInnen}. In: \emph{Arthur Schnitzler: Briefwechsel mit Autorinnen und Autoren}.
 Digitale Edition, https://schnitzler-briefe.acdh.oeaw.ac.at/{\dateiname}.html (Stand \today)
\fi

\end{document}


      