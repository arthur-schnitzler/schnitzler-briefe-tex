%% latex-leseansicht-vorspann.tex
%% Vorspann für die Leseansicht.
%% Lädt die gemeinsame Datei latex-vorspann.tex mit nicht gesetztem Schalter.

\newif\ifkorrekturansicht
\korrekturansichtfalse

\input{../tex-inputs/latex-vorspann}


\section[ Paul Goldmann an Arthur Schnitzler, 23. 9. {[}1901{]}]{L03085 Paul Goldmann an Arthur Schnitzler,  23. 9. [1901]}
\nopagebreak\mylabel{L03085v}
\rehead{ }\normalsize\beginnumbering\briefempfaengerindex{Schnitzler, Arthur@\textsc{Schnitzler, Arthur}!zzzGoldmann, Paul@\emph{von Paul Goldmann}!1901-09-231@{23. 9. [1901]}|(be}
\toendnotes[C]{\smallbreak\pagebreak[2]}
\correspDesc{Versand  durch Paul Goldmann am 23. 9. [1901] in Berlin
\newline{}Erhalt  durch Arthur Schnitzler im Zeitraum [24. 9. 1901
                  – 28. 9. 1901?] in Wien}\toendnotes[C]{\smallbreak}
\Standort{DLA, A:Schnitzler, HS.NZ85.1.3171.}
\physDesc{Brief, 1 Blatt, 4 Seiten, 1512 Zeichen
\newline{}Handschrift: blaue Tinte, deutsche Kurrent
\newline{}Schnitzler: 1) mit Bleistift das Jahr »901« vermerkt  2) mit rotem Buntstift vier Unterstreichungen}\toendnotes[C]{\smallbreak}
\pstart
           \raggedleft{}{\pb}\textcolor{gray}{\textbf{DESSAUERSTRASSE 19}}\oindex{Dessauer Straße@\textbf{Dessauer Straße}, \emph{Straße}|pw}\pend
           
\pstart
           Berlin\oindex{Berlin@\textbf{Berlin}, \emph{Hauptstadt}|pw}, 23. Sept.\pend
           
\pstart\center{}Mein lieber Freund,\pend\vspace{0.5em}
\pstart
           Die \textsc{Triesch\pwindex{Triesch, Irene 13.\,4.\,1877 Wien – 24.\,11.\,1964 Basel@\textsc{Triesch, Irene} (13.\,4.\,1877 Wien – 24.\,11.\,1964 Basel), \emph{Schauspielerin}|pw}} hat bereits die \label{K_L03085-1v}\edtext{Rollen in Deinen
                  Stücken\pwindex{Schnitzler, Arthur 15.\,5.\,1862 Wien – 21.\,10.\,1931 ebd.@\textsc{Schnitzler, Arthur} (15.\,5.\,1862 Wien – 21.\,10.\,1931 ebd.), \emph{Schriftsteller, Mediziner}!Lebendige Stunden. Vier Einakter@\strich\emph{Lebendige Stunden. Vier Einakter}|pwv}}{\lemma{\textnormal{\emph{Rollen in Deinen
                  Stücken}}}\Cendnote{\textnormal{In \emph{Die
                     Frau mit dem Dolche}\pwindex{Schnitzler, Arthur 15.\,5.\,1862 Wien – 21.\,10.\,1931 ebd.@\textsc{Schnitzler, Arthur} (15.\,5.\,1862 Wien – 21.\,10.\,1931 ebd.), \emph{Schriftsteller, Mediziner}!Frau mit dem Dolche@\strich\emph{Die Frau mit dem Dolche}|pwk} spielte Irene
                     Triesch\pwindex{Triesch, Irene 13.\,4.\,1877 Wien – 24.\,11.\,1964 Basel@\textsc{Triesch, Irene} (13.\,4.\,1877 Wien – 24.\,11.\,1964 Basel), \emph{Schauspielerin}|pwk} am Deutschen Theater Berlin\oindex{Deutsches Theater Berlin@\textbf{Deutsches Theater Berlin}, \emph{Theater}|pwk} die
                  Rolle der Pauline\pwindex{Schnitzler, Arthur 15.\,5.\,1862 Wien – 21.\,10.\,1931 ebd.@\textsc{Schnitzler, Arthur} (15.\,5.\,1862 Wien – 21.\,10.\,1931 ebd.), \emph{Schriftsteller, Mediziner}!Frau mit dem Dolche@\strich\emph{Die Frau mit dem Dolche}|pwkv} und in
                     \emph{Literatur}\pwindex{Schnitzler, Arthur 15.\,5.\,1862 Wien – 21.\,10.\,1931 ebd.@\textsc{Schnitzler, Arthur} (15.\,5.\,1862 Wien – 21.\,10.\,1931 ebd.), \emph{Schriftsteller, Mediziner}!Literatur@\strich\emph{Literatur}|pwk} jene der Margarete\pwindex{Schnitzler, Arthur 15.\,5.\,1862 Wien – 21.\,10.\,1931 ebd.@\textsc{Schnitzler, Arthur} (15.\,5.\,1862 Wien – 21.\,10.\,1931 ebd.), \emph{Schriftsteller, Mediziner}!Literatur@\strich\emph{Literatur}|pwkv}.}}}\label{K_L03085-1} bekommen und iſt
               namentlich von de\substVorne{}\textsuperscript{m}\substDazwischen{}r\substHinten{}{ }Frau mit dem Dolch\pwindex{Schnitzler, Arthur 15.\,5.\,1862 Wien – 21.\,10.\,1931 ebd.@\textsc{Schnitzler, Arthur} (15.\,5.\,1862 Wien – 21.\,10.\,1931 ebd.), \emph{Schriftsteller, Mediziner}!Frau mit dem Dolche@\strich\emph{Die Frau mit dem Dolche}|pw} entzückt. Hat{ }ſie{ }ſich auch
               bereits recht hübſch zurechtgelegt\strikeout{.} und{ }ſpricht jeden
               Tag Goethe\pwindex{Goethe, Johann Wolfgang von 28.\,8.\,1749 Frankfurt am Main – 22.\,3.\,1832 Weimar@\textsc{Goethe, Johann Wolfgang von} (28.\,8.\,1749 Frankfurt am Main – 22.\,3.\,1832 Weimar), \emph{Schriftsteller}|pw}ſche Verſe, um{ }ſich im
               Verſe-Recitiren zu üben. Sie will nach \textsc{München\oindex{München@\textbf{München}|pw}} fahren und \textsc{Lenbach\pwindex{Lenbach, Franz 13.\,10.\,1836 Schrobenhausen – 6.\,5.\,1904 München@\textsc{Lenbach, Franz} (13.\,10.\,1836 Schrobenhausen – 6.\,5.\,1904 München), \emph{Maler}|pw}} oder \textsc{Stuck\pwindex{Stuck, Franz von 23.\,2.\,1863 Tettenweis – 30.\,8.\,1928 München@\textsc{Stuck, Franz von} (23.\,2.\,1863 Tettenweis – 30.\,8.\,1928 München), \emph{Maler}|pw}} bitten, das betreffende \label{K_L03085-2v}\edtext{Bild}{\lemma{\textnormal{\emph{Bild}}}\Cendnote{\textnormal{Franz Lenbach\pwindex{Lenbach, Franz 13.\,10.\,1836 Schrobenhausen – 6.\,5.\,1904 München@\textsc{Lenbach, Franz} (13.\,10.\,1836 Schrobenhausen – 6.\,5.\,1904 München), \emph{Maler}|pwk} und Franz von Stuck\pwindex{Stuck, Franz von 23.\,2.\,1863 Tettenweis – 30.\,8.\,1928 München@\textsc{Stuck, Franz von} (23.\,2.\,1863 Tettenweis – 30.\,8.\,1928 München), \emph{Maler}|pwk} waren Münch\oindex{München@\textbf{München}|pwkv}ner Maler. In \emph{Die
                     Frau mit dem Dolche}\pwindex{Schnitzler, Arthur 15.\,5.\,1862 Wien – 21.\,10.\,1931 ebd.@\textsc{Schnitzler, Arthur} (15.\,5.\,1862 Wien – 21.\,10.\,1931 ebd.), \emph{Schriftsteller, Mediziner}!Frau mit dem Dolche@\strich\emph{Die Frau mit dem Dolche}|pwk} verbindet ein Renaissance-Bild die Gegenwartshandlung
                  mit einem historischen Teil.}}}\label{K_L03085-2} zu {\pb}entwerfen, was gar nicht übel wäre.\pend
           
\pstart
           \strikeout{Daß} Wann \strikeout{kommſt}{ }\label{K_L03085-3v}\edtext{kommſt}{\lemma{\textnormal{\emph{kommst}}}\Cendnote{\textnormal{Schnitzler kam für die Uraufführung von \emph{Lebendige Stunden}\pwindex{Schnitzler, Arthur 15.\,5.\,1862 Wien – 21.\,10.\,1931 ebd.@\textsc{Schnitzler, Arthur} (15.\,5.\,1862 Wien – 21.\,10.\,1931 ebd.), \emph{Schriftsteller, Mediziner}!Lebendige Stunden. Vier Einakter@\strich\emph{Lebendige Stunden. Vier Einakter}|pwk}\eventindex{Deutsches Theater Berlin@\textbf{Deutsches Theater Berlin}!Uraufführung von Lebendige Stunden, 4.1.1902@Uraufführung von Lebendige Stunden, 4.1.1902|pwk} (4. 1. 1902, \emph{Deutsches Theater}\orgindex{Deutsches Theater Berlin@Deutsches Theater Berlin|pwk}) nach Berlin\oindex{Berlin@\textbf{Berlin}, \emph{Hauptstadt}|pwk}. Er blieb vom 28. 12. 1901 bis zum 6. 1. 1902.}}}\label{K_L03085-3} Du?\pend
           
\pstart
           Daß Du mir \textsc{Kerrs\pwindex{Kerr, Alfred 25.\,12.\,1867 Breslau – 12.\,10.\,1948 Hamburg@\textsc{Kerr, Alfred} (25.\,12.\,1867 Breslau – 12.\,10.\,1948 Hamburg), \emph{Schriftsteller, Kritiker}|pw}} Beſuch in Berlin\oindex{Berlin@\textbf{Berlin}, \emph{Hauptstadt}|pw} verſchwiegen haſt, iſt bedauerlich. Immerhin wirſt Du
               bei unſerem nächſten Beiſammenſein behaupten, es mir geſchrieben zu haben.\pend
           
\pstart
           \textsc{Salten\pwindex{Salten, Felix 6.\,9.\,1869 Budapest – 8.\,10.\,1945 Zürich@\textsc{Salten, Felix} (6.\,9.\,1869 Budapest – 8.\,10.\,1945 Zürich), \emph{Schriftsteller, Journalist, Chefredakteur}|pw}} iſt morgen bei mir zu Tiſch.\pend
           
\pstart
           \textsc{Peter Dorner\pwindex{Dorner, Peter 17.\,2.\,1857 Welsberg-Taisten – 1.\,4.\,1931 ebd.@\textsc{Dorner, Peter} (17.\,2.\,1857 Welsberg-Taisten – 1.\,4.\,1931 ebd.), \emph{Schmied, Kunsthandwerker, Kunstschmied}|pw}}, denke Dir!,{ }ſchickte {\pb}mir dieſelben \label{K_L03085-4v}\edtext{Bücher}{\lemma{\textnormal{\emph{Bücher}}}\Cendnote{\textnormal{nicht ermittelt}}}\label{K_L03085-4}, die er Dir geſandt. Ich habe ihm
               ein{ }ſchönes \label{K_L03085-5v}\edtext{Werk über Schmiede\substVorne{}\textsuperscript{\textcolor{gray}{r}}\substDazwischen{}a\substHinten{}rbeit\pwindex{Schmiedekunst nach Originalen des XV. bis XVIII. Jahrhunderts@\emph{Die Schmiedekunst nach Originalen des XV. bis XVIII. Jahrhunderts}|pwuv} mit Nachbildungen alter Meiſterſtücke}{\lemma{\textnormal{\emph{Werk … Meisterstücke}}}\Cendnote{\textnormal{Möglicherweise: \emph{Die Schmiedekunst nach Originalen des XV. bis
                        XVIII. Jahrhunderts}\pwindex{Schmiedekunst nach Originalen des XV. bis XVIII. Jahrhunderts@\emph{Die Schmiedekunst nach Originalen des XV. bis XVIII. Jahrhunderts}|pwk}. Berlin\oindex{Berlin@\textbf{Berlin}, \emph{Hauptstadt}|pwk}: \emph{Verlag von Ernst Wasmuth}\orgindex{Verlag von Ernst Wasmuth@Verlag von Ernst Wasmuth|pwk}{ }1887.}}}\label{K_L03085-5}, im Betrage von 30 \textsc{MK}, als Gegengeſchenk
               geſandt. Dann gibt es ein noch viel{ }ſchöneres Werk\pwindex{Schmiedekunst nach Originalen des XV. bis XVIII. Jahrhunderts@\emph{Die Schmiedekunst nach Originalen des XV. bis XVIII. Jahrhunderts}|pwuv} derſelben Art, das 44 \textsc{MK} koſtet, betitelt »\label{K_L03085-6v}\edtext{Die deutſche Schmiedekunſt\pwindex{Brüning, Adolf Oktober 1867 Münster – 2.\,2.\,1912 Hannover@\textsc{Brüning, Adolf} (Oktober 1867 Münster – 2.\,2.\,1912 Hannover), \emph{Museumsdirektor, Klassischer Philologe, Kunsthistoriker}!Schmiedekunst seit dem Ende der Renaissance@\strich\emph{Die Schmiedekunst seit dem Ende der Renaissance}|pwu}}{\lemma{\textnormal{\emph{Die … Schmiedekunst}}}\Cendnote{\textnormal{Vermutlich: Adolf Brüning\pwindex{Brüning, Adolf Oktober 1867 Münster – 2.\,2.\,1912 Hannover@\textsc{Brüning, Adolf} (Oktober 1867 Münster – 2.\,2.\,1912 Hannover), \emph{Museumsdirektor, Klassischer Philologe, Kunsthistoriker}|pwk}: \emph{Die Schmiedekunst seit dem Ende der Renaissance}\pwindex{Brüning, Adolf Oktober 1867 Münster – 2.\,2.\,1912 Hannover@\textsc{Brüning, Adolf} (Oktober 1867 Münster – 2.\,2.\,1912 Hannover), \emph{Museumsdirektor, Klassischer Philologe, Kunsthistoriker}!Schmiedekunst seit dem Ende der Renaissance@\strich\emph{Die Schmiedekunst seit dem Ende der Renaissance}|pwk}.
                     Mit 150 Abbildungen. Leipzig\oindex{Leipzig@\textbf{Leipzig}, \emph{Hauptstadt}|pwk}: \emph{Verlag von Hermann Seemann
                        Nachfolger}\orgindex{Hermann Seemann Nachfolger@Hermann Seemann Nachfolger|pwk} [1901?].}}}\label{K_L03085-6}«. Mir allein
               iſt es zu theuer. Möchteſt Du Dich mit der Hälfte betheiligen? Davon würde der Mann\pwindex{Dorner, Peter 17.\,2.\,1857 Welsberg-Taisten – 1.\,4.\,1931 ebd.@\textsc{Dorner, Peter} (17.\,2.\,1857 Welsberg-Taisten – 1.\,4.\,1931 ebd.), \emph{Schmied, Kunsthandwerker, Kunstschmied}|pwv} wenigſtens \strikeout{etwas} etwas Ordentliches {\pb}profitiren.\pend
           
\pstart
           Danke den lieben Mädchen\pwindex{Schnitzler, Olga 17.\,1.\,1882 Wien – 13.\,1.\,1970 Lugano@\textsc{Schnitzler, Olga} (17.\,1.\,1882 Wien – 13.\,1.\,1970 Lugano), \emph{Schauspielerin, Sängerin}|pwv}\pwindex{Steinrück, Elisabeth 19.\,11.\,1885 – 7.\,4.\,1920 Partenkirchen@\textsc{Steinrück, Elisabeth} (19.\,11.\,1885 – 7.\,4.\,1920 Partenkirchen)|pwv} in meinem Namen für ihre reizenden Briefe, die mich unendlich
               erfreut haben. Sie{ }ſollen mir nicht böſe{ }ſein, daß ich nicht gleich antworte; aber
               ich{ }ſtecke tief in der Arbeit. Nächſter Tage{ }ſchreibe ich ihnen. Iſt die Adreſſe
               immer noch \textsc{Maximilianplatz\oindex{Wien@\textbf{Wien}!IX., Alsergrund@\textbf{IX., Alsergrund}!Rooseveltplatz@\textbf{Rooseveltplatz}, \emph{Platz}|pw}}?\pend
           
\pstart
           Viele treue Grüße {\\[\baselineskip]}Dein {\\[\baselineskip]}\spacefill\mbox{Paul Goldmnn}\pend
           \leftskip=0em{}
\pstart
           \noindent{}\label{T_L03085-1v}\edtext{{\pb}Lies’ in der letzten »Zeit\pwindex{Zeit. Wiener Wochenschrift@\emph{Die Zeit. Wiener Wochenschrift}|pw}« die{ }ſchöne Geſpenſtergeſchichte \label{K_L03085-7v}\edtext{»Das
                     rothe Zimmer\pwindex{Wells, H. G. 21.\,9.\,1866 Bromley – 13.\,8.\,1946 London@\textsc{Wells, H. G.} (21.\,9.\,1866 Bromley – 13.\,8.\,1946 London), \emph{Schriftsteller}!rothe Zimmer@\strich\emph{Das rothe Zimmer}|pw}«}{\lemma{\textnormal{\emph{»Das
                     rothe Zimmer«}}}\Cendnote{\textnormal{H. G. Wells\pwindex{Wells, H. G. 21.\,9.\,1866 Bromley – 13.\,8.\,1946 London@\textsc{Wells, H. G.} (21.\,9.\,1866 Bromley – 13.\,8.\,1946 London), \emph{Schriftsteller}|pwk}: \emph{Das rothe Zimmer}\pwindex{Wells, H. G. 21.\,9.\,1866 Bromley – 13.\,8.\,1946 London@\textsc{Wells, H. G.} (21.\,9.\,1866 Bromley – 13.\,8.\,1946 London), \emph{Schriftsteller}!rothe Zimmer@\strich\emph{Das rothe Zimmer}|pwk}. Übersetzt von M. v. Berthof\pwindex{Berthof, M. v. @\textsc{Berthof, M. v.}, \emph{Übersetzer/Übersetzerin}|pwk}. In: \emph{Die Zeit. Wiener Wochenschrift}\pwindex{Zeit. Wiener Wochenschrift@\emph{Die Zeit. Wiener Wochenschrift}|pwk}, Bd. 28, Nr. 364, 21. 9. 1901, S. 190–192.}}}\label{K_L03085-7}.{\\}\label{K_L03085-8v}\edtext{\textsc{Chamfort\pwindex{Chamfort, Sébastien Roch Nicolas 6.\,4.\,1741 Clermont – 13.\,4.\,1794 Paris@\textsc{Chamfort, Sébastien Roch Nicolas} (6.\,4.\,1741 Clermont – 13.\,4.\,1794 Paris), \emph{Schriftsteller}|pw}}}{\lemma{\textnormal{\emph{Chamfort}}}\Cendnote{\textnormal{\emph{Œuvres choisies de N. Chamfort, publiées avec
                        préface, notes et tables}\pwindex{Chamfort, Sébastien Roch Nicolas 6.\,4.\,1741 Clermont – 13.\,4.\,1794 Paris@\textsc{Chamfort, Sébastien Roch Nicolas} (6.\,4.\,1741 Clermont – 13.\,4.\,1794 Paris), \emph{Schriftsteller}!Œuvres choisies de N. Chamfort, publiées avec préface, notes et tables@\strich\emph{Œuvres choisies de N. Chamfort, publiées avec préface, notes et tables}|pwk}. Herausgegeben von Adolphe Mathurin de Lescure\pwindex{Lescure, Adolphe Mathurin de 1833 Bretenoux – 6.\,5.\,1892 Clamart@\textsc{Lescure, Adolphe Mathurin de} (1833 Bretenoux – 6.\,5.\,1892 Clamart), \emph{Journalist, Chefredakteur, Sekretär}|pwk}. Paris\oindex{Paris@\textbf{Paris}, \emph{Hauptstadt}|pwk}: \emph{Flammarion}\orgindex{Flammarion@Flammarion|pwk}{ }1892.}}}\label{K_L03085-8} (\begin{otherlanguage}{french}\textsc{Œuvres choisies}\end{otherlanguage}, in 2 Bänden\pwindex{Chamfort, Sébastien Roch Nicolas 6.\,4.\,1741 Clermont – 13.\,4.\,1794 Paris@\textsc{Chamfort, Sébastien Roch Nicolas} (6.\,4.\,1741 Clermont – 13.\,4.\,1794 Paris), \emph{Schriftsteller}!Œuvres choisies de N. Chamfort, publiées avec préface, notes et tables@\strich\emph{Œuvres choisies de N. Chamfort, publiées avec préface, notes et tables}|pw}) iſt bei \textsc{Flammarion\orgindex{Flammarion@Flammarion|pw}} erſchienen.}{\lemma{\textnormal{\emph{Lies’ … erschienen.}}}\Cendnote{\textnormal{kopfüber am oberen
                     Rand der ersten Seite}}}\label{T_L03085-1}\pend
           \selectlanguage{ngerman}\endnumbering\briefempfaengerindex{Schnitzler, Arthur@\textsc{Schnitzler, Arthur}!zzzGoldmann, Paul@\emph{von Paul Goldmann}!1901-09-231@{23. 9. [1901]}|)be}\mylabel{L03085h}  \newcommand{\dateiname}{L03085}\newcommand{\titel}{Paul Goldmann an Arthur Schnitzler, 23. 9. [1901]}\newcommand{\editorInnen}{Martin Anton Müller und Laura Untner}%% latex-leseansicht-abspann.tex
%% Abspann für die Leseansicht.
%% Der Schalter \ifkorrekturansicht ist bereits durch den Vorspann gesetzt.

%% latex-abspann.tex
%% Gemeinsamer Abspann für Korrekturansicht und Leseansicht.
%% Setzt den Schalter \ifkorrekturansicht voraus (gesetzt in den
%% einbindenden Dateien latex-korrekturansicht-abspann.tex bzw.
%% latex-leseansicht-abspann.tex).
%% ---------------------------------------------------------------

\normalsize

% Das esempio-Environment wird nur in der Leseansicht benötigt
\ifkorrekturansicht\else
\newenvironment{esempio}[3]%
{
    \vspace{1.5ex}
    \rlap{\underline{#1}}
    \par
    \setlength{\parindent}{0cm}
    \nopagebreak
    \leftskip=#2cm
    \rightskip=#3cm
}
{
    \par
}
\fi

\doendnotes{C}
\bigskip
\vfill

\clearpage

\footnotesize

\ifkorrekturansicht
  \lohead{\textsc{register}}
\fi

% theindex-Environment neu definieren ohne reledmac
\makeatletter
\renewenvironment{theindex}{%
  \ifkorrekturansicht
    \section*{\indexname}%
  \else
    \subsubsection*{Index der erwähnten Entitäten}%
  \fi
  \setlength{\parindent}{0pt}%
  \setlength{\parskip}{0pt plus 0.3pt}%
  \let\item\@idxitem
}{%
  \ifkorrekturansicht\clearpage\fi
}
\makeatother

\IfFileExists{\jobname-pw.ind}{\input{\jobname-pw.ind}}{}

% Quellenangabe nur in der Leseansicht
\ifkorrekturansicht\else
% Fallback-Definitionen, falls die .tex-Datei \titel etc. nicht gesetzt hat
\providecommand{\titel}{}
\providecommand{\editorInnen}{}
\providecommand{\dateiname}{\jobname}

\vspace{3cm}

\vfill

\footnotesize
\textsc{Quelle}: \titel. Herausgegeben von {\editorInnen}. In: \emph{Arthur Schnitzler: Briefwechsel mit Autorinnen und Autoren}.
 Digitale Edition, https://schnitzler-briefe.acdh.oeaw.ac.at/{\dateiname}.html (Stand \today)
\fi

\end{document}


