%% latex-leseansicht-vorspann.tex
%% Vorspann für die Leseansicht.
%% Lädt die gemeinsame Datei latex-vorspann.tex mit nicht gesetztem Schalter.

\newif\ifkorrekturansicht
\korrekturansichtfalse

\input{../tex-inputs/latex-vorspann}


\section[Detlev von Liliencron u. a. an Arthur Schnitzler, 14. 4. 1904]{L01391 Detlev von Liliencron u. a. an Arthur Schnitzler, 14. 4. 1904}
\nopagebreak\mylabel{L01391v}
\rehead{ }\normalsize\beginnumbering\briefempfaengerindex{Schnitzler, Arthur@\textsc{Schnitzler, Arthur}!zzzDevrient, Max@\emph{von Max Devrient}!1904-04-141@{14. 4. 1904}|(be}\briefempfaengerindex{Schnitzler, Arthur@\textsc{Schnitzler, Arthur}!zzzWebern, Anton von@\emph{von Anton von Webern}!1904-04-141@{14. 4. 1904}|(be}\briefempfaengerindex{Schnitzler, Arthur@\textsc{Schnitzler, Arthur}!zzzLoos, Lina@\emph{von Lina Loos}!1904-04-141@{14. 4. 1904}|(be}\briefempfaengerindex{Schnitzler, Arthur@\textsc{Schnitzler, Arthur}!zzzWymetal [Musikschriftsteller], Wilhelm@\emph{von Wilhelm Wymetal [Musikschriftsteller]}!1904-04-141@{14. 4. 1904}|(be}\briefempfaengerindex{Schnitzler, Arthur@\textsc{Schnitzler, Arthur}!zzzKraus, Karl@\emph{von Karl Kraus}!1904-04-141@{14. 4. 1904}|(be}\briefempfaengerindex{Schnitzler, Arthur@\textsc{Schnitzler, Arthur}!zzzAltenberg, Peter@\emph{von Peter Altenberg}!1904-04-141@{14. 4. 1904}|(be}\briefempfaengerindex{Schnitzler, Arthur@\textsc{Schnitzler, Arthur}!zzzKahler, Erich@\emph{von Erich Kahler}!1904-04-141@{14. 4. 1904}|(be}\briefempfaengerindex{Schnitzler, Arthur@\textsc{Schnitzler, Arthur}!zzzHorwitz, Charlotte@\emph{von Charlotte Horwitz}!1904-04-141@{14. 4. 1904}|(be}\briefempfaengerindex{Schnitzler, Arthur@\textsc{Schnitzler, Arthur}!zzzStefan, Paul@\emph{von Paul Stefan}!1904-04-141@{14. 4. 1904}|(be}\briefempfaengerindex{Schnitzler, Arthur@\textsc{Schnitzler, Arthur}!zzzLiliencron, Detlev von@\emph{von Detlev von Liliencron}!1904-04-141@{14. 4. 1904}|(be}
\toendnotes[C]{\smallbreak\pagebreak[2]}
\correspDesc{Versand  durch Detlev von Liliencron, Paul Stefan, Charlotte Horwitz, Erich Kahler, Peter Altenberg, Karl Kraus, Wilhelm von Wymetal, Lina Loos, Anton von Webern, Max Devrient am 14. 4. 1904 in Wien
\newline{}Erhalt  durch Arthur Schnitzler im Zeitraum [15. 4. 1904
                  – 18. 4. 1904?] in Wien}\toendnotes[C]{\smallbreak}
\Standort{DLA, A:Schnitzler, HS.NZ85.1.3896, S. 2.}
\physDesc{Karte, maschinenschriftliche Abschrift, 1 Blatt, 1 Seite, 191 Zeichen
\newline{}Schreibmaschine}\toendnotes[C]{\smallbreak}
\pstart
           \raggedleft{}{\pb}Wiener Ansorge-Verein\orgindex{Ansorge-Verein. Verein zur Förderung moderner Kunst@Ansorge-Verein. Verein zur Förderung moderner Kunst|pw},{\\}\label{T_L01391-1v}\edtext{14. 4. 1904\eventindex{Eisenbahnbeamtenclub@\textbf{Eisenbahnbeamtenclub}!2. Liliencron-Abend, 14.4.1904@2. Liliencron-Abend, 14.4.1904|pwv}}{\lemma{\textnormal{\emph{14. 4. 1904}}}\Cendnote{\textnormal{Am 14. 4. 1904 fand ein
                        zweiter Liliencron-Abend\eventindex{Eisenbahnbeamtenclub@\textbf{Eisenbahnbeamtenclub}!2. Liliencron-Abend, 14.4.1904@2. Liliencron-Abend, 14.4.1904|pwk}, statt,
                     nachdem der erste\eventindex{Bösendorfer-Saal@\textbf{Bösendorfer-Saal}!1. Liliencron-Abend, 10.4.1904@1. Liliencron-Abend, 10.4.1904|pwkv} am
                        10. 4. 1904 ein großer Erfolg gewesen war.
                     Organisiert wurden beide vom \emph{Wiener
                        Ansorge-Verein}\orgindex{Ansorge-Verein. Verein zur Förderung moderner Kunst@Ansorge-Verein. Verein zur Förderung moderner Kunst|pwk}. Hatte der erste im Bösendorfer-Saal\oindex{Wien@\textbf{Wien}!I., Innere Stadt@\textbf{I., Innere Stadt}!Bösendorfer-Saal@\textbf{Bösendorfer-Saal}, \emph{Veranstaltungsgebäude}|pwk} stattgefunden, so fand dieser mit modifiziertem
                     Programm im Eisenbahnbeamtenclub\oindex{Wien@\textbf{Wien}!I., Innere Stadt@\textbf{I., Innere Stadt}!Eisenbahnbeamtenclub@\textbf{Eisenbahnbeamtenclub}, \emph{Veranstaltungsgebäude}|pwk} in der
                        Eschenbachgasse 11\oindex{Wien@\textbf{Wien}!I., Innere Stadt@\textbf{I., Innere Stadt}!Eschenbachgasse 11@\textbf{Eschenbachgasse 11}, \emph{Wohngebäude}|pwk} statt.
                  }}}\label{T_L01391-1}.\pend
           \vspace{0.5em}
\pstart
           Wir bedauern, dass Sie nicht in unserer Mitte weilen:\pend
           
\pstart
           Ihr \spacefill\mbox{Liliencron.}{\\[\baselineskip]}\spacefill\mbox{Paul \label{T_L01391-2v}\edtext{Stefan}{\lemma{\textnormal{\emph{Stefan}}}\Cendnote{\textnormal{In der Abschrift steht:
                        »Atefan«.}}}\label{T_L01391-2}}{\\[\baselineskip]}\spacefill\mbox{C. Horwitz}{\\[\baselineskip]}\spacefill\mbox{Erich Kahler}{\\[\baselineskip]}\spacefill\mbox{Peter Altenberg}{\\[\baselineskip]}\spacefill\mbox{Karl Kraus}{\\[\baselineskip]}\spacefill\mbox{Wymetal}{\\[\baselineskip]}\spacefill\mbox{L. Loos}{\\[\baselineskip]}\spacefill\mbox{Webern}{\\[\baselineskip]}\spacefill\mbox{Max Devrient.}\pend
           \leftskip=0em{}\selectlanguage{ngerman}\endnumbering\briefempfaengerindex{Schnitzler, Arthur@\textsc{Schnitzler, Arthur}!zzzDevrient, Max@\emph{von Max Devrient}!1904-04-141@{14. 4. 1904}|)be}\briefempfaengerindex{Schnitzler, Arthur@\textsc{Schnitzler, Arthur}!zzzWebern, Anton von@\emph{von Anton von Webern}!1904-04-141@{14. 4. 1904}|)be}\briefempfaengerindex{Schnitzler, Arthur@\textsc{Schnitzler, Arthur}!zzzLoos, Lina@\emph{von Lina Loos}!1904-04-141@{14. 4. 1904}|)be}\briefempfaengerindex{Schnitzler, Arthur@\textsc{Schnitzler, Arthur}!zzzWymetal [Musikschriftsteller], Wilhelm@\emph{von Wilhelm Wymetal [Musikschriftsteller]}!1904-04-141@{14. 4. 1904}|)be}\briefempfaengerindex{Schnitzler, Arthur@\textsc{Schnitzler, Arthur}!zzzKraus, Karl@\emph{von Karl Kraus}!1904-04-141@{14. 4. 1904}|)be}\briefempfaengerindex{Schnitzler, Arthur@\textsc{Schnitzler, Arthur}!zzzAltenberg, Peter@\emph{von Peter Altenberg}!1904-04-141@{14. 4. 1904}|)be}\briefempfaengerindex{Schnitzler, Arthur@\textsc{Schnitzler, Arthur}!zzzKahler, Erich@\emph{von Erich Kahler}!1904-04-141@{14. 4. 1904}|)be}\briefempfaengerindex{Schnitzler, Arthur@\textsc{Schnitzler, Arthur}!zzzHorwitz, Charlotte@\emph{von Charlotte Horwitz}!1904-04-141@{14. 4. 1904}|)be}\briefempfaengerindex{Schnitzler, Arthur@\textsc{Schnitzler, Arthur}!zzzStefan, Paul@\emph{von Paul Stefan}!1904-04-141@{14. 4. 1904}|)be}\briefempfaengerindex{Schnitzler, Arthur@\textsc{Schnitzler, Arthur}!zzzLiliencron, Detlev von@\emph{von Detlev von Liliencron}!1904-04-141@{14. 4. 1904}|)be}\mylabel{L01391h}  \newcommand{\dateiname}{L01391}\newcommand{\titel}{Detlev von Liliencron u. a. an Arthur Schnitzler, 14. 4. 1904}\newcommand{\editorInnen}{Martin Anton Müller und Gerd-Hermann Susen}%% latex-leseansicht-abspann.tex
%% Abspann für die Leseansicht.
%% Der Schalter \ifkorrekturansicht ist bereits durch den Vorspann gesetzt.

%% latex-abspann.tex
%% Gemeinsamer Abspann für Korrekturansicht und Leseansicht.
%% Setzt den Schalter \ifkorrekturansicht voraus (gesetzt in den
%% einbindenden Dateien latex-korrekturansicht-abspann.tex bzw.
%% latex-leseansicht-abspann.tex).
%% ---------------------------------------------------------------

\normalsize

% Das esempio-Environment wird nur in der Leseansicht benötigt
\ifkorrekturansicht\else
\newenvironment{esempio}[3]%
{
    \vspace{1.5ex}
    \rlap{\underline{#1}}
    \par
    \setlength{\parindent}{0cm}
    \nopagebreak
    \leftskip=#2cm
    \rightskip=#3cm
}
{
    \par
}
\fi

\doendnotes{C}
\bigskip
\vfill

\clearpage

\footnotesize

\ifkorrekturansicht
  \lohead{\textsc{register}}
\fi

% theindex-Environment neu definieren ohne reledmac
\makeatletter
\renewenvironment{theindex}{%
  \ifkorrekturansicht
    \section*{\indexname}%
  \else
    \subsubsection*{Index der erwähnten Entitäten}%
  \fi
  \setlength{\parindent}{0pt}%
  \setlength{\parskip}{0pt plus 0.3pt}%
  \let\item\@idxitem
}{%
  \ifkorrekturansicht\clearpage\fi
}
\makeatother

\IfFileExists{\jobname-pw.ind}{\input{\jobname-pw.ind}}{}

% Quellenangabe nur in der Leseansicht
\ifkorrekturansicht\else
% Fallback-Definitionen, falls die .tex-Datei \titel etc. nicht gesetzt hat
\providecommand{\titel}{}
\providecommand{\editorInnen}{}
\providecommand{\dateiname}{\jobname}

\vspace{3cm}

\vfill

\footnotesize
\textsc{Quelle}: \titel. Herausgegeben von {\editorInnen}. In: \emph{Arthur Schnitzler: Briefwechsel mit Autorinnen und Autoren}.
 Digitale Edition, https://schnitzler-briefe.acdh.oeaw.ac.at/{\dateiname}.html (Stand \today)
\fi

\end{document}


