%% latex-leseansicht-vorspann.tex
%% Vorspann für die Leseansicht.
%% Lädt die gemeinsame Datei latex-vorspann.tex mit nicht gesetztem Schalter.

\newif\ifkorrekturansicht
\korrekturansichtfalse

\input{../tex-inputs/latex-vorspann}


         
         \renewcommand{\erwaehntePersonen}{Personen: Max Devrient, Lina Loos, Anton von Webern, Wilhelm von Wymetal}
         \renewcommand{\erwaehnteInstitutionen}{Institutionen: Ansorge-Verein}
         \renewcommand{\erwaehnteOrte}{Orte: Wien}
         \renewcommand{\erwaehnteWerke}{}
               \section[Detlev von Liliencron u. a. an Arthur Schnitzler, 1{[}0?{]}. 4. 1904]{ Detlev von Liliencron u. a. an Arthur Schnitzler,
               1{[}0?{]}. 4. 1904}\nopagebreak\mylabel{v}\rehead{ }\begin{ledgroupsized}[t]{13cm}\normalsize\beginnumbering \toendnotes[C]{\smallbreak\pagebreak[2]} \Standort{DLA, A:Schnitzler, HS.NZ85.1.3896, S. 2.}
\physDesc{Karte, Maschinenschriftliche Abschrift, 1 Blatt, 1 Seite, 191 Zeichen
\newline{}Schreibmaschine}\toendnotes[C]{\smallbreak}\pstart
           \raggedleft{}{\pb}Wiener Ansorge-Verein\orgindex{Ansorge-Verein@Ansorge-Verein|pw},{\\}\label{T_L01391_1v}\edtext{14. 4. 1904}{\lemma{\textnormal{\emph{14. 4. 1904}}}\Cendnote{\textnormal{Die Veranstaltung – ein »Liliencron\pwindex{Liliencron, Detlev von 03.06.1844 – 22.07.1909@\textsc{Liliencron, Detlev von} (03.06.1844 – 22.07.1909)|pw}-Abend« mit Lesung
                     des Dichters und Devrient\pwindex{Devrient, Max 12.12.1857 – 13.06.1929@\textsc{Devrient, Max} (12.12.1857 – 13.06.1929), \emph{Regisseur, Schauspieler}|pwk}s fand am
                        10. 4. 1904 statt; in der Abschrift dürfte beim Datum ein
                     Fehler unterlaufen sein.}}}\label{T_L01391_1h}.\pend
           \pstart
           Wir bedauern, dass Sie nicht in unserer Mitte weilen:\pend
           \pstart
           Ihr \spacefill\mbox{Liliencron.}{\\[\baselineskip]}\spacefill\mbox{Paul \label{T_L01391_2v}\edtext{Stefan}{\lemma{\textnormal{\emph{Stefan}}}\Cendnote{\textnormal{korrigiert aus:
                     »Atefan«}}}\label{T_L01391_2h}}{\\[\baselineskip]}\spacefill\mbox{C. Horwitz}{\\[\baselineskip]}\spacefill\mbox{Erich Kahler}{\\[\baselineskip]}\spacefill\mbox{Peter Altenberg}{\\[\baselineskip]}\spacefill\mbox{Karl Kraus}{\\[\baselineskip]}\spacefill\mbox{Wymetal}{\\[\baselineskip]}\spacefill\mbox{L. Loos}{\\[\baselineskip]}\spacefill\mbox{Webern}{\\[\baselineskip]}\spacefill\mbox{Max Devrient.}\pend
           \leftskip=0em{}
         
         \endnumbering\mylabel{h}\end{ledgroupsized}  \newcommand{\dateiname}{L01391}\newcommand{\titel}{Detlev von Liliencron u. a. an Arthur Schnitzler, 1[0?]. 4. 1904}\newcommand{\editorInnen}{Martin Anton Müller und Gerd-Hermann Susen}%% latex-leseansicht-abspann.tex
%% Abspann für die Leseansicht.
%% Der Schalter \ifkorrekturansicht ist bereits durch den Vorspann gesetzt.

%% latex-abspann.tex
%% Gemeinsamer Abspann für Korrekturansicht und Leseansicht.
%% Setzt den Schalter \ifkorrekturansicht voraus (gesetzt in den
%% einbindenden Dateien latex-korrekturansicht-abspann.tex bzw.
%% latex-leseansicht-abspann.tex).
%% ---------------------------------------------------------------

\normalsize

% Das esempio-Environment wird nur in der Leseansicht benötigt
\ifkorrekturansicht\else
\newenvironment{esempio}[3]%
{
    \vspace{1.5ex}
    \rlap{\underline{#1}}
    \par
    \setlength{\parindent}{0cm}
    \nopagebreak
    \leftskip=#2cm
    \rightskip=#3cm
}
{
    \par
}
\fi

\doendnotes{C}
\bigskip
\vfill

\clearpage

\footnotesize

\ifkorrekturansicht
  \lohead{\textsc{register}}
\fi

% theindex-Environment neu definieren ohne reledmac
\makeatletter
\renewenvironment{theindex}{%
  \ifkorrekturansicht
    \section*{\indexname}%
  \else
    \subsubsection*{Index der erwähnten Entitäten}%
  \fi
  \setlength{\parindent}{0pt}%
  \setlength{\parskip}{0pt plus 0.3pt}%
  \let\item\@idxitem
}{%
  \ifkorrekturansicht\clearpage\fi
}
\makeatother

\IfFileExists{\jobname-pw.ind}{\input{\jobname-pw.ind}}{}

% Quellenangabe nur in der Leseansicht
\ifkorrekturansicht\else
% Fallback-Definitionen, falls die .tex-Datei \titel etc. nicht gesetzt hat
\providecommand{\titel}{}
\providecommand{\editorInnen}{}
\providecommand{\dateiname}{\jobname}

\vspace{3cm}

\vfill

\footnotesize
\textsc{Quelle}: \titel. Herausgegeben von {\editorInnen}. In: \emph{Arthur Schnitzler: Briefwechsel mit Autorinnen und Autoren}.
 Digitale Edition, https://schnitzler-briefe.acdh.oeaw.ac.at/{\dateiname}.html (Stand \today)
\fi

\end{document}


      