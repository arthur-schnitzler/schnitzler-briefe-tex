%% latex-leseansicht-vorspann.tex
%% Vorspann für die Leseansicht.
%% Lädt die gemeinsame Datei latex-vorspann.tex mit nicht gesetztem Schalter.

\newif\ifkorrekturansicht
\korrekturansichtfalse

\input{../tex-inputs/latex-vorspann}


\section[Hermann Bahr an Arthur Schnitzler, 4. {[}12.{]} 1904]{L01474 Hermann Bahr an Arthur Schnitzler, 4. [12.] 1904}
\nopagebreak\mylabel{L01474v}
\rehead{ }\normalsize\beginnumbering\briefempfaengerindex{Schnitzler, Arthur@\textsc{Schnitzler, Arthur}!zzzBahr, Hermann@\emph{von Hermann Bahr}!1904-12-041@{4. [12.] 1904}|(be}
\toendnotes[C]{\smallbreak\pagebreak[2]}
\correspDesc{Versand  durch Hermann Bahr am 4. [12.] 1904 in Wien
\newline{}Erhalt  durch Arthur Schnitzler im Zeitraum [4. 12. 1904
                  – 8. 12. 1904?] in Wien}\toendnotes[C]{\smallbreak}
\Standort{CUL, Schnitzler, B 5b.}
\physDesc{Brief, 1 Blatt, 2 Seiten, 1379 Zeichen
\newline{}Handschrift: schwarze Tinte, deutsche Kurrent
\newline{}Ordnung: mit Bleistift von unbekannter Hand nummeriert:
                                    »124« }
\buchAbdrucke{\weitereDrucke{Hermann Bahr, Arthur Schnitzler: \emph{Briefwechsel, Aufzeichnungen, Dokumente (1891–1931)}. Herausgegeben von Kurt Ifkovits und Martin Anton Müller. Göttingen: \emph{Wallstein} 2018, S. 326–327.} }\toendnotes[C]{\smallbreak}
\pstart
           \raggedleft{}{\pb}4. \label{T_L01474-1v}\edtext{11.}{\lemma{\textnormal{\emph{11.}}}\Cendnote{\textnormal{Schreibirrtum, durch den Inhalt auf
                        Dezember zu datieren.}}}\label{T_L01474-1} 04\pend
           
\pstart\center{}Lieber Arthur!\pend\vspace{0.5em}
\pstart
           Bitte, kannſt Du mir den »Puppenſpieler\pwindex{Schnitzler, Arthur 15.\,5.\,1862 Wien – 21.\,10.\,1931 ebd.@\textsc{Schnitzler, Arthur} (15.\,5.\,1862 Wien – 21.\,10.\,1931 ebd.), \emph{Schriftsteller, Mediziner}!Puppenspieler. Studie in einem Aufzuge@\strich\emph{Der Puppenspieler. Studie in einem Aufzuge}|pw}«
               gedruckt{ }ſchicken? Ich möchte, wenn es mir zuſammengeht, über den \label{K_L01474-1v}\edtext{Schnitzlerabend}{\lemma{\textnormal{\emph{Schnitzlerabend}}}\Cendnote{\textnormal{Es handelt sich um den am 12. 12. 1904
                  stattfindenden »Arthur-Schnitzler-Abend« im Carl-Theater\oindex{Wien@\textbf{Wien}!II., Leopoldstadt@\textbf{II., Leopoldstadt}!Carl-Theater@\textbf{Carl-Theater}, \emph{Theater}|pwk}. Dieser wurde für das seit 1787 bestehende \emph{Erste öffentliche Kinderkrankeninstitut}\orgindex{Erstes öffentliches Kinderkrankeninstitut@Erstes öffentliches Kinderkrankeninstitut|pwk}
                  abgehalten, dessen Leitung Carl Hochsinger\pwindex{Hochsinger, Carl 12.\,7.\,1860 Wien – 28.\,10.\,1942 Konzentrationslager Theresienstadt@\textsc{Hochsinger, Carl} (12.\,7.\,1860 Wien – 28.\,10.\,1942 Konzentrationslager Theresienstadt), \emph{Pädiater}|pwk}
                  innehatte.}}}\label{K_L01474-1}{ }\label{K_L01474-2v}\edtext{ausführlicher ſchreiben\pwindex{Bahr, Hermann 19.\,7.\,1863 Linz – 15.\,1.\,1934 München@\textsc{Bahr, Hermann} (19.\,7.\,1863 Linz – 15.\,1.\,1934 München), \emph{Schriftsteller, Kritiker}!Puppenspieler. (Studie in einem Aufzuge von Arthur Schnitzler. Zum ersten Mal aufgeführt im Carl-Theater am 12. Dezember 1904)@\strich\emph{Der Puppenspieler. (Studie in einem Aufzuge von Arthur Schnitzler. Zum ersten Mal aufgeführt im Carl-Theater am 12. Dezember 1904)}|pwv}}{\lemma{\textnormal{\emph{ausführlicher schreiben}}}\Cendnote{\textnormal{Siehe Hermann Bahr, Arthur Schnitzler: \emph{Briefwechsel, Aufzeichnungen, Dokumente (1891–1931)}, Hermann Bahr: Der Puppenspieler, 13. 12. 1904.
               }}}\label{K_L01474-2}. Dazu wäre es mir allerdings{ }ſehr lieb, das Buch noch vor
               Donnerſtag zu kriegen. Ja?\pend
           
\pstart
           Sehr gern möchte ich Dich auch endlich wieder{ }ſehen. Allerdings bin ich wenig frei,
               da ich mich nun mit einer gewiß törichten \label{K_L01474-3v}\edtext{Leidenſchaft}{\lemma{\textnormal{\emph{Leidenschaft}}}\Cendnote{\textnormal{die
                  Bekanntschaft mit seiner späteren zweiten Frau, der Opernsängerin Anna von Mildenburg\pwindex{Bahr-Mildenburg, Anna 29.\,11.\,1872 Wien – 27.\,1.\,1947 ebd.@\textsc{Bahr-Mildenburg, Anna} (29.\,11.\,1872 Wien – 27.\,1.\,1947 ebd.), \emph{Sängerin}|pwk}}}}\label{K_L01474-3}, der ich aber momentan so viel unſagbares Glück verdanke, wie ich nie im
               Leben kannte (\label{LL286-1v}vielleicht wird man{ }ſo ganz
                  transparenter Seligkeiten erſt im Angeſicht des Todes fähig\label{LL286-1h}), aufs Hören
               von Musik geworfen habe, wovon ich dann manchmal in einer Ermattung mit {\pb}vollſtändigem Verſagen und Verſiegen jeder Kraft
               zurückbleibe. \textsc{Vita minima}, die auch ihre{ }ſchönen Schauder
               hat. Wie eben jetzt,{ }ſonſt würde ich Dir dieſen Unsinn nicht{ }ſchreiben, \textsc{enfin} ich wollte{ }ſagen: ich möchte Dich gern wiederſehen und
               hoffe bald zu Dir zu kommen. Und was würdeſt Du zu der Idee{ }ſagen: zu Weihnachten uns
               in Lueg\oindex{Lueg@\textbf{Lueg}, \emph{Teil eines besiedelten Ortes}|pw}{ }\introOben{}am Wolfgangſee\oindex{Wolfgangsee@\textbf{Wolfgangsee}, \emph{See}|pw}\introOben{} zu treffen, wo ich ein paar \label{K_L01474-4v}\edtext{Tage beim Burckhard\pwindex{Burckhard, Max Eugen 14.\,7.\,1854 Korneuburg – 16.\,3.\,1912 Wien@\textsc{Burckhard, Max Eugen} (14.\,7.\,1854 Korneuburg – 16.\,3.\,1912 Wien), \emph{Schriftsteller, Rechtswissenschaftler, Theaterleiter}|pw}}{\lemma{\textnormal{\emph{Tage beim Burckhard}}}\Cendnote{\textnormal{Bahr\pwindex{Bahr, Hermann 19.\,7.\,1863 Linz – 15.\,1.\,1934 München@\textsc{Bahr, Hermann} (19.\,7.\,1863 Linz – 15.\,1.\,1934 München), \emph{Schriftsteller, Kritiker}|pwk} fuhr am 24. 12. 1904, blieb
                  bis zum 27. 12. 1904 und verpasste Schnitzler knapp.}}}\label{K_L01474-4} hauſen will? Ich wollte eigentlich nach Athen\oindex{Athen@\textbf{Athen}, \emph{Hauptstadt}|pw}, aber da müßte ich am 20. von Trieſt\oindex{Triest@\textbf{Triest}, \emph{Verwaltungsgebiet}|pw} weg und \label{K_L01474-5v}\edtext{am 22. ist der Triſtan\pwindex{\textcolor{red}{\textsuperscript{XXXX indx1}}!Tristan und Isolde@\strich\emph{Tristan und Isolde}|pw}}{\lemma{\textnormal{\emph{am 22. ist der Tristan}}}\Cendnote{\textnormal{Die Aufführung von \emph{Tristan und Isolde}\pwindex{\textcolor{red}{\textsuperscript{XXXX indx1}}!Tristan und Isolde@\strich\emph{Tristan und Isolde}|pwk} war noch am 8. 12. 1904 für
                  den 22. angesetzt (vgl. Brief Bahrs an Anna Mildenburg\pwindex{Bahr-Mildenburg, Anna 29.\,11.\,1872 Wien – 27.\,1.\,1947 ebd.@\textsc{Bahr-Mildenburg, Anna} (29.\,11.\,1872 Wien – 27.\,1.\,1947 ebd.), \emph{Sängerin}|pwk}, 8. 12. 1904, \emph{Theatermuseum
                        Wien}, AM 43.853 BaM), wurde aber auf den
                     23. 12. 1904 verschoben.}}}\label{K_L01474-5}, der für mich jetzt – ganz real
               und ganz phyſiſch geſprochen – das höchſte Wolſein iſt, mehr als Sonne und Meer.\pend
           
\pstart
           Entſchuldige den verworrenen Ton dieſes Briefes, grüße Frau Olga\pwindex{Schnitzler, Olga 17.\,1.\,1882 Wien – 13.\,1.\,1970 Lugano@\textsc{Schnitzler, Olga} (17.\,1.\,1882 Wien – 13.\,1.\,1970 Lugano), \emph{Schauspielerin, Sängerin}|pw} und den Heinrich\pwindex{Schnitzler, Heinrich 9.\,8.\,1902 Hinterbrühl – 12.\,7.\,1982 Wien@\textsc{Schnitzler, Heinrich} (9.\,8.\,1902 Hinterbrühl – 12.\,7.\,1982 Wien), \emph{Regisseur, Schauspieler}|pw}
               herzlichſt und{ }ſei es{ }ſelbſt von{\\[\baselineskip]}Deinem{\\[\baselineskip]}\spacefill\mbox{Hermann}\pend
           \leftskip=0em{}\selectlanguage{ngerman}\endnumbering\briefempfaengerindex{Schnitzler, Arthur@\textsc{Schnitzler, Arthur}!zzzBahr, Hermann@\emph{von Hermann Bahr}!1904-12-041@{4. [12.] 1904}|)be}\mylabel{L01474h}  \newcommand{\dateiname}{L01474}\newcommand{\titel}{Hermann Bahr an Arthur Schnitzler, 4. [12.] 1904}\newcommand{\editorInnen}{Herausgegeben von Martin Anton Müller}%% latex-leseansicht-abspann.tex
%% Abspann für die Leseansicht.
%% Der Schalter \ifkorrekturansicht ist bereits durch den Vorspann gesetzt.

%% latex-abspann.tex
%% Gemeinsamer Abspann für Korrekturansicht und Leseansicht.
%% Setzt den Schalter \ifkorrekturansicht voraus (gesetzt in den
%% einbindenden Dateien latex-korrekturansicht-abspann.tex bzw.
%% latex-leseansicht-abspann.tex).
%% ---------------------------------------------------------------

\normalsize

% Das esempio-Environment wird nur in der Leseansicht benötigt
\ifkorrekturansicht\else
\newenvironment{esempio}[3]%
{
    \vspace{1.5ex}
    \rlap{\underline{#1}}
    \par
    \setlength{\parindent}{0cm}
    \nopagebreak
    \leftskip=#2cm
    \rightskip=#3cm
}
{
    \par
}
\fi

\doendnotes{C}
\bigskip
\vfill

\clearpage

\footnotesize

\ifkorrekturansicht
  \lohead{\textsc{register}}
\fi

% theindex-Environment neu definieren ohne reledmac
\makeatletter
\renewenvironment{theindex}{%
  \ifkorrekturansicht
    \section*{\indexname}%
  \else
    \subsubsection*{Index der erwähnten Entitäten}%
  \fi
  \setlength{\parindent}{0pt}%
  \setlength{\parskip}{0pt plus 0.3pt}%
  \let\item\@idxitem
}{%
  \ifkorrekturansicht\clearpage\fi
}
\makeatother

\IfFileExists{\jobname-pw.ind}{\input{\jobname-pw.ind}}{}

% Quellenangabe nur in der Leseansicht
\ifkorrekturansicht\else
% Fallback-Definitionen, falls die .tex-Datei \titel etc. nicht gesetzt hat
\providecommand{\titel}{}
\providecommand{\editorInnen}{}
\providecommand{\dateiname}{\jobname}

\vspace{3cm}

\vfill

\footnotesize
\textsc{Quelle}: \titel. Herausgegeben von {\editorInnen}. In: \emph{Arthur Schnitzler: Briefwechsel mit Autorinnen und Autoren}.
 Digitale Edition, https://schnitzler-briefe.acdh.oeaw.ac.at/{\dateiname}.html (Stand \today)
\fi

\end{document}


