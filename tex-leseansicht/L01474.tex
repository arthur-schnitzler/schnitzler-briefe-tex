%% latex-leseansicht-vorspann.tex
%% Vorspann für die Leseansicht.
%% Lädt die gemeinsame Datei latex-vorspann.tex mit nicht gesetztem Schalter.

\newif\ifkorrekturansicht
\korrekturansichtfalse

\input{../tex-inputs/latex-vorspann}


         
         \renewcommand{\erwaehntePersonen}{Personen: Hermann Bahr, Anna Bahr-Mildenburg, Max Eugen Burckhard, Carl Hochsinger, Olga Schnitzler, Heinrich Schnitzler}
         \renewcommand{\erwaehnteInstitutionen}{Institutionen: Erstes öffentliches Kinderkrankeninstitut}
         \renewcommand{\erwaehnteOrte}{Orte: Athen, Carl-Theater, Lueg am Wolfgangsee, Triest, Wien, Wolfgangsee}
         \renewcommand{\erwaehnteWerke}{Werke: Der Puppenspieler. (Studie in einem Aufzuge von Arthur Schnitzler. Zum ersten Mal aufgeführt im Carl-Theater am 12. Dezember 1904), Der Puppenspieler. Studie in einem Aufzuge, Tristan und Isolde}
               \section[Hermann Bahr an Arthur Schnitzler, 4. {[}12.{]} 1904]{ Hermann Bahr an Arthur Schnitzler, 4. {[}12.{]} 1904}\nopagebreak\mylabel{v}\rehead{ }\begin{ledgroupsized}[t]{13cm}\normalsize\beginnumbering\briefempfaengerindex{Schnitzler, Arthur@\textsc{Schnitzler, Arthur}!zzzBahr, Hermann@\emph{von Hermann Bahr}!1904-12-041@{4. {[}12.{]} 1904}|(be} \toendnotes[C]{\smallbreak\pagebreak[2]} \Standort{CUL, Schnitzler, B 5b.}
\physDesc{Brief, 1 Blatt, 2 Seiten, 1379 Zeichen
\newline{}Handschrift: schwarze Tinte, deutsche Kurrent
\newline{}Ordnung: mit Bleistift von unbekannter Hand nummeriert:
                                    »124« }\buchAbdrucke{\weitereDrucke{Hermann Bahr, Arthur Schnitzler: \emph{Briefwechsel, Aufzeichnungen, Dokumente (1891–1931)}. Hg. Kurt Ifkovits und Martin Anton Müller. Göttingen: \emph{Wallstein} 2018, S. 326–327.} }\toendnotes[C]{\smallbreak}\pstart
           \raggedleft{}{\pb}4. \label{T_L01474-1v}\edtext{11.}{\lemma{\textnormal{\emph{11.}}}\Cendnote{\textnormal{Schreibirrtum, durch den Inhalt auf
                        Dezember zu datieren.}}}\label{T_L01474-1h} 04\pend
           \pstart\center{}Lieber Arthur!\pend\pstart
           Bitte, kannſt Du mir den »Puppenſpieler\pwindex{Schnitzler, Arthur 15.05.1862 – 21.10.1931@\textsc{Schnitzler, Arthur} (15.05.1862 – 21.10.1931), \emph{Schriftsteller, Mediziner}!Puppenspieler. Studie in einem Aufzuge31. 05. 1903@\strich\emph{Der Puppenspieler. Studie in einem Aufzuge} {[}31. 05. 1903{]}|pw}«
               gedruckt ſchicken? Ich möchte, wenn es mir zuſammengeht, über den \label{K_L01474-1v}\edtext{Schnitzlerabend}{\lemma{\textnormal{\emph{Schnitzlerabend}}}\Cendnote{\textnormal{Es handelt sich um den am 12. 12. 1904
                  stattfindenden »Arthur-Schnitzler-Abend« im Carl-Theater\oindex{Carl-Theater@\textbf{Carl-Theater}|pwk}. Dieser wurde für das seit 1787 bestehende \emph{Erste öffentliche Kinderkrankeninstitut}\orgindex{Erstes oeffentliches Kinderkrankeninstitut@Erstes öffentliches Kinderkrankeninstitut|pwk}
                  abgehalten, dessen Leitung Carl Hochsinger\pwindex{Hochsinger, Carl 12.07.1860 – 28.10.1942@\textsc{Hochsinger, Carl} (12.07.1860 – 28.10.1942), \emph{Pädiater}|pwk}
                  inne hatte.}}}\label{K_L01474-1h}{ }\label{K_L01474-2v}\edtext{ausführlicher ſchreiben\pwindex{Bahr, Hermann 19.07.1863 – 15.01.1934@\textsc{Bahr, Hermann} (19.07.1863 – 15.01.1934), \emph{Schriftsteller, Kritiker}!Puppenspieler. (Studie in einem Aufzuge von Arthur Schnitzler. Zum ersten Mal aufgefuehrt im Carl-Theater am 12. Dezember 1904)13. 12. 1904@\strich\emph{Der Puppenspieler. (Studie in einem Aufzuge von Arthur Schnitzler. Zum ersten Mal aufgeführt im Carl-Theater am 12. Dezember 1904)} {[}13. 12. 1904{]}|pwv}}{\lemma{\textnormal{\emph{ausführlicher ſchreiben}}}\Cendnote{\textnormal{siehe Bahr/Schnitzler, T030021}}}\label{K_L01474-2h}. Dazu wäre es mir allerdings ſehr lieb, das Buch noch vor
               Donnerſtag zu kriegen. Ja?\pend
           \pstart
           Sehr gern möchte ich Dich auch endlich wieder ſehen. Allerdings bin ich wenig frei,
               da ich mich nun mit einer gewiß törichten \label{K_L01474-3v}\edtext{Leidenſchaft}{\lemma{\textnormal{\emph{Leidenſchaft}}}\Cendnote{\textnormal{die
                  Bekanntschaft mit seiner späteren zweiten Frau, der Opernsängerin Anna von Mildenburg\pwindex{Bahr-Mildenburg, Anna 29.11.1872 – 27.01.1947@\textsc{Bahr-Mildenburg, Anna} (29.11.1872 – 27.01.1947), \emph{Sängerin}|pwk}}}}\label{K_L01474-3h}, der ich aber momentan so viel unſagbares Glück verdanke, wie ich nie im
               Leben kannte (\label{LL286-1v}vielleicht wird man ſo ganz
                  transparenter Seligkeiten erſt im Angeſicht des Todes fähig\label{LL286-1h}), aufs Hören
               von Musik geworfen habe, wovon ich dann manchmal in einer Ermattung mit {\pb}vollſtändigem Verſagen und Verſiegen jeder Kraft
               zurückbleibe. \textsc{Vita minima}, die auch ihre ſchönen Schauder
               hat. Wie eben jetzt, ſonſt würde ich Dir dieſen Unsinn nicht ſchreiben, \textsc{enfin} ich wollte ſagen: ich möchte Dich gern wiederſehen und
               hoffe bald zu Dir zu kommen. Und was würdeſt Du zu der Idee ſagen: zu Weihnachten uns
               in Lueg\oindex{Lueg am Wolfgangsee@\textbf{Lueg am Wolfgangsee}|pw}{ }\introOben{}am Wolfgangſee\oindex{Wolfgangsee@\textbf{Wolfgangsee}|pw}\introOben{} zu treffen, wo ich ein paar \label{K_L01474-4v}\edtext{Tage beim Burckhard\pwindex{Burckhard, Max Eugen 14.07.1854 – 16.03.1912@\textsc{Burckhard, Max Eugen} (14.07.1854 – 16.03.1912), \emph{Schriftsteller, Rechtswissenschaftler, Theaterleiter}|pw}}{\lemma{\textnormal{\emph{Tage beim Burckhard}}}\Cendnote{\textnormal{Bahr\pwindex{Bahr, Hermann 19.07.1863 – 15.01.1934@\textsc{Bahr, Hermann} (19.07.1863 – 15.01.1934), \emph{Schriftsteller, Kritiker}|pwk} fährt am 24. und bleibt
                  bis 27. 12. 1904 und verpasst Schnitzler\pwindex{Schnitzler, Arthur 15.05.1862 – 21.10.1931@\textsc{Schnitzler, Arthur} (15.05.1862 – 21.10.1931), \emph{Schriftsteller, Mediziner}|pwk} knapp.}}}\label{K_L01474-4h} hauſen will? Ich wollte eigentlich nach Athen\oindex{Athen@\textbf{Athen}|pw}, aber da müßte ich am 20. von Trieſt\oindex{Triest@\textbf{Triest}|pw} weg und \label{K_L01474-5v}\edtext{am 22. ist der Triſtan\pwindex{\textcolor{red}{\textsuperscript{XXXX1 indx}}!Tristan und Isolde1865@\strich\emph{Tristan und Isolde} {[}1865{]}|pw}}{\lemma{\textnormal{\emph{am 22. ist der Triſtan}}}\Cendnote{\textnormal{Die Aufführung von \emph{Tristan und Isolde}\pwindex{\textcolor{red}{\textsuperscript{XXXX1 indx}}!Tristan und Isolde1865@\strich\emph{Tristan und Isolde} {[}1865{]}|pwk} war noch am 8. 12. 1904 für
                  den 22. angesetzt (vgl. Brief Bahrs an Anna Mildenburg\pwindex{Bahr-Mildenburg, Anna 29.11.1872 – 27.01.1947@\textsc{Bahr-Mildenburg, Anna} (29.11.1872 – 27.01.1947), \emph{Sängerin}|pwk}, 8. 12. 1904, \emph{Theatermuseum
                        Wien}, AM 43853 BaM), wurde aber auf den
                     23. 12. 1904 verschoben.}}}\label{K_L01474-5h}, der für mich jetzt – ganz real
               und ganz phyſiſch geſprochen – das höchſte Wolſein iſt, mehr als Sonne und Meer.\pend
           \pstart
           Entſchuldige den verworrenen Ton dieſes Briefes, grüße Frau Olga\pwindex{Schnitzler, Olga 17.01.1882 – 13.01.1970@\textsc{Schnitzler, Olga} (17.01.1882 – 13.01.1970), \emph{Schauspielerin, Sängerin}|pw} und den Heinrich\pwindex{Schnitzler, Heinrich 09.08.1902 – 12.07.1982@\textsc{Schnitzler, Heinrich} (09.08.1902 – 12.07.1982), \emph{Regisseur, Schauspieler}|pw}
               herzlichſt und ſei es ſelbſt von{\\[\baselineskip]}Deinem{\\[\baselineskip]}\spacefill\mbox{Hermann}\pend
           \leftskip=0em{}
         
         \endnumbering\mylabel{h}\end{ledgroupsized}  \newcommand{\dateiname}{L01474}\newcommand{\titel}{Hermann Bahr an Arthur Schnitzler, 4. [12.] 1904}\newcommand{\editorInnen}{ Kurt Ifkovits,  Martin Anton Müller}%% latex-leseansicht-abspann.tex
%% Abspann für die Leseansicht.
%% Der Schalter \ifkorrekturansicht ist bereits durch den Vorspann gesetzt.

%% latex-abspann.tex
%% Gemeinsamer Abspann für Korrekturansicht und Leseansicht.
%% Setzt den Schalter \ifkorrekturansicht voraus (gesetzt in den
%% einbindenden Dateien latex-korrekturansicht-abspann.tex bzw.
%% latex-leseansicht-abspann.tex).
%% ---------------------------------------------------------------

\normalsize

% Das esempio-Environment wird nur in der Leseansicht benötigt
\ifkorrekturansicht\else
\newenvironment{esempio}[3]%
{
    \vspace{1.5ex}
    \rlap{\underline{#1}}
    \par
    \setlength{\parindent}{0cm}
    \nopagebreak
    \leftskip=#2cm
    \rightskip=#3cm
}
{
    \par
}
\fi

\doendnotes{C}
\bigskip
\vfill

\clearpage

\footnotesize

\ifkorrekturansicht
  \lohead{\textsc{register}}
\fi

% theindex-Environment neu definieren ohne reledmac
\makeatletter
\renewenvironment{theindex}{%
  \ifkorrekturansicht
    \section*{\indexname}%
  \else
    \subsubsection*{Index der erwähnten Entitäten}%
  \fi
  \setlength{\parindent}{0pt}%
  \setlength{\parskip}{0pt plus 0.3pt}%
  \let\item\@idxitem
}{%
  \ifkorrekturansicht\clearpage\fi
}
\makeatother

\IfFileExists{\jobname-pw.ind}{\input{\jobname-pw.ind}}{}

% Quellenangabe nur in der Leseansicht
\ifkorrekturansicht\else
% Fallback-Definitionen, falls die .tex-Datei \titel etc. nicht gesetzt hat
\providecommand{\titel}{}
\providecommand{\editorInnen}{}
\providecommand{\dateiname}{\jobname}

\vspace{3cm}

\vfill

\footnotesize
\textsc{Quelle}: \titel. Herausgegeben von {\editorInnen}. In: \emph{Arthur Schnitzler: Briefwechsel mit Autorinnen und Autoren}.
 Digitale Edition, https://schnitzler-briefe.acdh.oeaw.ac.at/{\dateiname}.html (Stand \today)
\fi

\end{document}


      