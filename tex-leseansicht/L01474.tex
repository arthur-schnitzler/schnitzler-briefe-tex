%% latex-korrekturansicht-vorspann.tex
%% Vorspann für die Korrekturansicht.
%% Lädt die gemeinsame Datei latex-vorspann.tex mit gesetztem Schalter.

\newif\ifkorrekturansicht
\korrekturansichttrue

\input{../tex-inputs/latex-vorspann}


\section[Hermann Bahr an Arthur Schnitzler, 4. {[}12.{]} 1904]{L01474 Hermann Bahr an Arthur Schnitzler, 4. {[}12.{]} 1904}
\nopagebreak\mylabel{L01474v}
\rehead{ }\normalsize\beginnumbering\briefempfaengerindex{Schnitzler, Arthur@\textsc{Schnitzler, Arthur}!zzzBahr, Hermann@\emph{von Hermann Bahr}!1904-12-041@{4. {[}12.{]} 1904}|(be}
\toendnotes[C]{\smallbreak\pagebreak[2]}\Standort{CUL, Schnitzler, B 5b.}
\physDesc{Brief, 1 Blatt, 2 Seiten, 1379 Zeichen
\newline{}Handschrift: schwarze Tinte, deutsche Kurrent
\newline{}Ordnung: mit Bleistift von unbekannter Hand nummeriert:
                                    »124« }
\buchAbdrucke{\weitereDrucke{Hermann Bahr, Arthur Schnitzler: \emph{Briefwechsel, Aufzeichnungen, Dokumente (1891–1931)}. Göttingen: \emph{Wallstein} 2018, S. 326–327.} }\toendnotes[C]{\smallbreak}
\pstart
           \raggedleft{}{\pb}4. \label{T_L01474-1v}\edtext{11.}{\lemma{\textnormal{\emph{11.}}}\Cendnote{\textnormal{Schreibirrtum, durch den Inhalt auf
                        Dezember zu datieren.}}}\label{T_L01474-1} 04\pend
           
\pstart\center{}Lieber Arthur!\pend\vspace{0.5em}
\pstart
           Bitte, kannſt Du mir den »Puppenſpieler\pwindex{Puppenspieler. Studie in einem Aufzuge@\emph{Der Puppenspieler. Studie in einem Aufzuge}|pw}«
               gedruckt ſchicken? Ich möchte, wenn es mir zuſammengeht, über den \label{K_L01474-1v}\edtext{Schnitzlerabend}{\lemma{\textnormal{\emph{Schnitzlerabend}}}\Cendnote{\textnormal{Es handelt sich um den am 12. 12. 1904
                  stattfindenden »Arthur-Schnitzler-Abend« im Carl-Theater\oindex{Carl-Theater@\textbf{Carl-Theater}, \emph{Theater (K.THE)}|pwk}. Dieser wurde für das seit 1787 bestehende \emph{Erste öffentliche Kinderkrankeninstitut}\orgindex{Erstes oeffentliches Kinderkrankeninstitut@Erstes öffentliches Kinderkrankeninstitut|pwk}
                  abgehalten, dessen Leitung Carl Hochsinger\pwindex{Hochsinger, Carl 12.07.1860 – 28.10.1942@\textsc{Hochsinger, Carl} (12.07.1860 – 28.10.1942), \emph{Pädiater/Pädiaterin}|pwk}
                  innehatte.}}}\label{K_L01474-1}{ }\label{K_L01474-2v}\edtext{ausführlicher ſchreiben\pwindex{Puppenspieler. (Studie in einem Aufzuge von Arthur Schnitzler. Zum ersten Mal aufgefuehrt im Carl-Theater am 12. Dezember 1904)@\emph{Der Puppenspieler. (Studie in einem Aufzuge von Arthur Schnitzler. Zum ersten Mal aufgeführt im Carl-Theater am 12. Dezember 1904)}|pwv}}{\lemma{\textnormal{\emph{ausführlicher ſchreiben}}}\Cendnote{\textnormal{Siehe Hermann Bahr, Arthur Schnitzler: \emph{Briefwechsel, Aufzeichnungen, Dokumente (1891–1931)}, Hermann Bahr: Der Puppenspieler, 13. 12. 1904.
               }}}\label{K_L01474-2}. Dazu wäre es mir allerdings ſehr lieb, das Buch noch vor
               Donnerſtag zu kriegen. Ja?\pend
           
\pstart
           Sehr gern möchte ich Dich auch endlich wieder ſehen. Allerdings bin ich wenig frei,
               da ich mich nun mit einer gewiß törichten \label{K_L01474-3v}\edtext{Leidenſchaft}{\lemma{\textnormal{\emph{Leidenſchaft}}}\Cendnote{\textnormal{die
                  Bekanntschaft mit seiner späteren zweiten Frau, der Opernsängerin Anna von Mildenburg\pwindex{Bahr-Mildenburg, Anna 29.11.1872 – 27.01.1947@\textsc{Bahr-Mildenburg, Anna} (29.11.1872 – 27.01.1947), \emph{Sänger/Sängerin}|pwk}}}}\label{K_L01474-3}, der ich aber momentan so viel unſagbares Glück verdanke, wie ich nie im
               Leben kannte (\label{LL286-1v}vielleicht wird man ſo ganz
                  transparenter Seligkeiten erſt im Angeſicht des Todes fähig\label{LL286-1h}), aufs Hören
               von Musik geworfen habe, wovon ich dann manchmal in einer Ermattung mit {\pb}vollſtändigem Verſagen und Verſiegen jeder Kraft
               zurückbleibe. \textsc{Vita minima}, die auch ihre ſchönen Schauder
               hat. Wie eben jetzt, ſonſt würde ich Dir dieſen Unsinn nicht ſchreiben, \textsc{enfin} ich wollte ſagen: ich möchte Dich gern wiederſehen und
               hoffe bald zu Dir zu kommen. Und was würdeſt Du zu der Idee ſagen: zu Weihnachten uns
               in Lueg\oindex{Lueg@\textbf{Lueg}, \emph{Teil eines besiedelten Ortes (A.BSOX)}|pw}{ }\introOben{}am Wolfgangſee\oindex{Wolfgangsee@\textbf{Wolfgangsee}, \emph{See (N.SEE)}|pw}\introOben{} zu treffen, wo ich ein paar \label{K_L01474-4v}\edtext{Tage beim Burckhard\pwindex{Burckhard, Max Eugen 14.07.1854 – 16.03.1912@\textsc{Burckhard, Max Eugen} (14.07.1854 – 16.03.1912), \emph{Schriftsteller/Schriftstellerin, Rechtswissenschaftler/Rechtswissenschaftlerin, Theaterleiter/Theaterleiterin}|pw}}{\lemma{\textnormal{\emph{Tage beim Burckhard}}}\Cendnote{\textnormal{Bahr\pwindex{Bahr, Hermann 19.07.1863 – 15.01.1934@\textsc{Bahr, Hermann} (19.07.1863 – 15.01.1934), \emph{Schriftsteller/Schriftstellerin, Kritiker/Kritikerin}|pwk} fuhr am 24. 12. 1904, blieb
                  bis zum 27. 12. 1904 und verpasste Schnitzler knapp.}}}\label{K_L01474-4} hauſen will? Ich wollte eigentlich nach Athen\oindex{Athen@\textbf{Athen}, \emph{P.PPLC}|pw}, aber da müßte ich am 20. von Trieſt\oindex{Triest@\textbf{Triest}, \emph{A.ADM3}|pw} weg und \label{K_L01474-5v}\edtext{am 22. ist der Triſtan\pwindex{Tristan und Isolde@\emph{Tristan und Isolde}|pw}}{\lemma{\textnormal{\emph{am 22. ist der Triſtan}}}\Cendnote{\textnormal{Die Aufführung von \emph{Tristan und Isolde}\pwindex{Tristan und Isolde@\emph{Tristan und Isolde}|pwk} war noch am 8. 12. 1904 für
                  den 22. angesetzt (vgl. Brief Bahrs an Anna Mildenburg\pwindex{Bahr-Mildenburg, Anna 29.11.1872 – 27.01.1947@\textsc{Bahr-Mildenburg, Anna} (29.11.1872 – 27.01.1947), \emph{Sänger/Sängerin}|pwk}, 8. 12. 1904, \emph{Theatermuseum
                        Wien}, AM 43.853 BaM), wurde aber auf den
                     23. 12. 1904 verschoben.}}}\label{K_L01474-5}, der für mich jetzt – ganz real
               und ganz phyſiſch geſprochen – das höchſte Wolſein iſt, mehr als Sonne und Meer.\pend
           
\pstart
           Entſchuldige den verworrenen Ton dieſes Briefes, grüße Frau Olga\pwindex{Schnitzler, Olga 17.01.1882 – 13.01.1970@\textsc{Schnitzler, Olga} (17.01.1882 – 13.01.1970), \emph{Schauspieler/Schauspielerin, Sänger/Sängerin}|pw} und den Heinrich\pwindex{Schnitzler, Heinrich 09.08.1902 – 12.07.1982@\textsc{Schnitzler, Heinrich} (09.08.1902 – 12.07.1982), \emph{Regisseur/Regisseurin, Schauspieler/Schauspielerin}|pw}
               herzlichſt und ſei es ſelbſt von{\\[\baselineskip]}Deinem{\\[\baselineskip]}\spacefill\mbox{Hermann}\pend
           \leftskip=0em{}\selectlanguage{ngerman}\endnumbering\briefempfaengerindex{Schnitzler, Arthur@\textsc{Schnitzler, Arthur}!zzzBahr, Hermann@\emph{von Hermann Bahr}!1904-12-041@{4. {[}12.{]} 1904}|)be}\mylabel{L01474h}  \normalsize

\doendnotes{C}
\bigskip
\vfill

\clearpage

\footnotesize

\lohead{\textsc{register}}

% Definiere theindex-Environment komplett neu ohne reledmac
\makeatletter
\renewenvironment{theindex}{%
  \section*{\indexname}%
  \setlength{\parindent}{0pt}%
  \setlength{\parskip}{0pt plus 0.3pt}%
  \let\item\@idxitem
}{%
  \clearpage
}
\makeatother

\IfFileExists{\jobname-pw.ind}{\input{\jobname-pw.ind}}{}

\end{document}

      