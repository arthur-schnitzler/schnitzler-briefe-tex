%% latex-korrekturansicht-vorspann.tex
%% Vorspann für die Korrekturansicht.
%% Lädt die gemeinsame Datei latex-vorspann.tex mit gesetztem Schalter.

\newif\ifkorrekturansicht
\korrekturansichttrue

\input{../tex-inputs/latex-vorspann}


\section[Felix Salten an Arthur Schnitzler, 18. 5. 1891]{L03101 Felix Salten an Arthur Schnitzler, 18. 5. 1891}
\nopagebreak\mylabel{L03101v}
\rehead{ }\normalsize\beginnumbering\briefempfaengerindex{Schnitzler, Arthur@\textsc{Schnitzler, Arthur}!zzzSalten, Felix@\emph{von Felix Salten}!1891-05-181@{18. 5. 1891}|(be}
\toendnotes[C]{\smallbreak\pagebreak[2]}\Standort{CUL, Schnitzler, B 89, A 1.}
\physDesc{Brief, 1 Blatt, 2 Seiten, 566 Zeichen
\newline{}Handschrift: schwarze Tinte, lateinische Kurrent
\newline{}Ordnung: mit Bleistift von unbekannter Hand nummeriert: »2« }\toendnotes[C]{\smallbreak}
\pstart
           \noindent{}{\pb}Verehrtester! Eben habe ich Ihr \label{K_L03101-1v}\edtext{»Denksteine\pwindex{Denksteine@\emph{Denksteine}|pw}« gelesen}{\lemma{\textnormal{\emph{»Denksteine« gelesen}}}\Cendnote{\textnormal{Arthur Schnitzler: \emph{Denksteine}\pwindex{Denksteine@\emph{Denksteine}|pwk}. In: \emph{Moderne
                        Rundschau}\pwindex{Moderne Rundschau@\emph{Moderne Rundschau}|pwk}, Bd. 3, H. 4, 15. 5. 1891,
                     S. 151–154. Siehe auch A. S.: \emph{Tagebuch}, 19. 5. 1891. Mit dem Dialog \emph{Die
                     Einzige}\pwindex{Einzige@\emph{Die Einzige}|pwk} (1902) schuf Salten\pwindex{Salten, Felix 06.09.1869 – 08.10.1945@\textsc{Salten, Felix} (06.09.1869 – 08.10.1945), \emph{Schriftsteller/Schriftstellerin, Journalist/Journalistin, Chefredakteur/Chefredakteurin}|pwk} später eine Variation des Einakters\pwindex{Denksteine@\emph{Denksteine}|pwkv} (vgl. Marcel Atze und
                        Gerhard Hubmann: \emph{»Der schwärmerischste, zärtlichste,
                        unermüdlichste Liebhaber, den ich kenne«. Felix Salten und das
                        Theater}. In: Marcel Atze, unter Mitarbeit von Tanja Gausterer (Herausgeber):
                        \emph{Im Schatten von Bambi. Felix Salten entdeckt die Wiener
                        Moderne. Leben und Werk}.
                     Salzburg/Wien:
                        \emph{Residenz}{ }2020, S. 376–397, hier: S. 393).}}}\label{K_L03101-1}. Ich
                  \uline{muss} es Ihnen sagen, wie entzückt und begeistert
               ich davon bin. Viele zwar werden Sie nicht verstehen, und das sind die Männer, welche
               die Frauen, die wir lieben, zu Fall gebracht und gedankenlos besessen, – und was noch
               schmerzlicher ist – die Weiber selbst.\pend
           
\pstart
           Wer doch auch so ruhig »Dirne« sagen könnte, und sich wegwenden. Ich habe bisher
               gefunden, dass das erste {\pb}leichter war, als das zweite.\pend
           
\pstart
           Noch einmal, das Stück\pwindex{Denksteine@\emph{Denksteine}|pwv} hat mir
               in’s Herz gegriffen, und seien Sie mir bedankt und handgeschüttelt.\pend
           
\pstart
           Ihr{\\[\baselineskip]}\spacefill\mbox{Felix Salten}\pend
           \leftskip=0em{}
\pstart
           \raggedleft{}18/5. 91\pend
           \selectlanguage{ngerman}\endnumbering\briefempfaengerindex{Schnitzler, Arthur@\textsc{Schnitzler, Arthur}!zzzSalten, Felix@\emph{von Felix Salten}!1891-05-181@{18. 5. 1891}|)be}\mylabel{L03101h}  \normalsize

\doendnotes{C}
\bigskip
\vfill

\clearpage

\footnotesize

\lohead{\textsc{register}}

% Definiere theindex-Environment komplett neu ohne reledmac
\makeatletter
\renewenvironment{theindex}{%
  \section*{\indexname}%
  \setlength{\parindent}{0pt}%
  \setlength{\parskip}{0pt plus 0.3pt}%
  \let\item\@idxitem
}{%
  \clearpage
}
\makeatother

\IfFileExists{\jobname-pw.ind}{\input{\jobname-pw.ind}}{}

\end{document}

      