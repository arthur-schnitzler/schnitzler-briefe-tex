%% latex-leseansicht-vorspann.tex
%% Vorspann für die Leseansicht.
%% Lädt die gemeinsame Datei latex-vorspann.tex mit nicht gesetztem Schalter.

\newif\ifkorrekturansicht
\korrekturansichtfalse

\input{../tex-inputs/latex-vorspann}


\section[Felix Salten an Arthur Schnitzler, 18. 5. 1891]{L03101 Felix Salten an Arthur Schnitzler, 18. 5. 1891}
\nopagebreak\mylabel{L03101v}
\rehead{ }\normalsize\beginnumbering\briefempfaengerindex{Schnitzler, Arthur@\textsc{Schnitzler, Arthur}!zzzSalten, Felix@\emph{von Felix Salten}!1891-05-181@{18. 5. 1891}|(be}
\toendnotes[C]{\smallbreak\pagebreak[2]}
\correspDesc{Versand  durch Felix Salten am 18. 5. 1891 in Wien
\newline{}Erhalt  durch Arthur Schnitzler am 19. 5. 1891 in Wien}\toendnotes[C]{\smallbreak}
\Standort{CUL, Schnitzler, B 89, A 1.}
\physDesc{Brief, 1 Blatt, 2 Seiten, 566 Zeichen
\newline{}Handschrift: schwarze Tinte, lateinische Kurrent
\newline{}Ordnung: mit Bleistift von unbekannter Hand nummeriert: »2« }\toendnotes[C]{\smallbreak}
\pstart
           \noindent{}{\pb}Verehrtester! Eben habe ich Ihr \label{K_L03101-1v}\edtext{»Denksteine\pwindex{Schnitzler, Arthur 15.\,5.\,1862 Wien – 21.\,10.\,1931 ebd.@\textsc{Schnitzler, Arthur} (15.\,5.\,1862 Wien – 21.\,10.\,1931 ebd.), \emph{Schriftsteller, Mediziner}!Denksteine@\strich\emph{Denksteine}|pw}« gelesen}{\lemma{\textnormal{\emph{»Denksteine« gelesen}}}\Cendnote{\textnormal{Arthur Schnitzler: \emph{Denksteine}\pwindex{Schnitzler, Arthur 15.\,5.\,1862 Wien – 21.\,10.\,1931 ebd.@\textsc{Schnitzler, Arthur} (15.\,5.\,1862 Wien – 21.\,10.\,1931 ebd.), \emph{Schriftsteller, Mediziner}!Denksteine@\strich\emph{Denksteine}|pwk}. In: \emph{Moderne
                        Rundschau}\pwindex{Moderne Rundschau@\emph{Moderne Rundschau}|pwk}, Bd. 3, H. 4, 15. 5. 1891,
                     S. 151–154. Siehe auch A. S.: \emph{Tagebuch}, 19. 5. 1891. Mit dem Dialog \emph{Die
                     Einzige}\pwindex{Salten, Felix 6.\,9.\,1869 Budapest – 8.\,10.\,1945 Zürich@\textsc{Salten, Felix} (6.\,9.\,1869 Budapest – 8.\,10.\,1945 Zürich), \emph{Schriftsteller, Journalist, Chefredakteur}!Einzige@\strich\emph{Die Einzige}|pwk} (1902) schuf Salten\pwindex{Salten, Felix 6.\,9.\,1869 Budapest – 8.\,10.\,1945 Zürich@\textsc{Salten, Felix} (6.\,9.\,1869 Budapest – 8.\,10.\,1945 Zürich), \emph{Schriftsteller, Journalist, Chefredakteur}|pwk} später eine Variation des Einakters\pwindex{Schnitzler, Arthur 15.\,5.\,1862 Wien – 21.\,10.\,1931 ebd.@\textsc{Schnitzler, Arthur} (15.\,5.\,1862 Wien – 21.\,10.\,1931 ebd.), \emph{Schriftsteller, Mediziner}!Denksteine@\strich\emph{Denksteine}|pwkv} (vgl. Marcel Atze und
                        Gerhard Hubmann: \emph{»Der schwärmerischste, zärtlichste,
                        unermüdlichste Liebhaber, den ich kenne«. Felix Salten und das
                        Theater}. In: Marcel Atze, unter Mitarbeit von Tanja Gausterer (Herausgeber):
                        \emph{Im Schatten von Bambi. Felix Salten entdeckt die Wiener
                        Moderne. Leben und Werk}.
                     Salzburg/Wien:
                        \emph{Residenz}{ }2020, S. 376–397, hier: S. 393).}}}\label{K_L03101-1}. Ich
                  \uline{muss} es Ihnen sagen, wie entzückt und begeistert
               ich davon bin. Viele zwar werden Sie nicht verstehen, und das sind die Männer, welche
               die Frauen, die wir lieben, zu Fall gebracht und gedankenlos besessen, – und was noch
               schmerzlicher ist – die Weiber selbst.\pend
           
\pstart
           Wer doch auch so ruhig »Dirne« sagen könnte, und sich wegwenden. Ich habe bisher
               gefunden, dass das erste {\pb}leichter war, als das zweite.\pend
           
\pstart
           Noch einmal, das Stück\pwindex{Schnitzler, Arthur 15.\,5.\,1862 Wien – 21.\,10.\,1931 ebd.@\textsc{Schnitzler, Arthur} (15.\,5.\,1862 Wien – 21.\,10.\,1931 ebd.), \emph{Schriftsteller, Mediziner}!Denksteine@\strich\emph{Denksteine}|pwv} hat mir
               in’s Herz gegriffen, und seien Sie mir bedankt und handgeschüttelt.\pend
           
\pstart
           Ihr{\\[\baselineskip]}\spacefill\mbox{Felix Salten}\pend
           \leftskip=0em{}
\pstart
           \raggedleft{}18/5. 91\pend
           \selectlanguage{ngerman}\endnumbering\briefempfaengerindex{Schnitzler, Arthur@\textsc{Schnitzler, Arthur}!zzzSalten, Felix@\emph{von Felix Salten}!1891-05-181@{18. 5. 1891}|)be}\mylabel{L03101h}  \newcommand{\dateiname}{L03101}\newcommand{\titel}{Felix Salten an Arthur Schnitzler, 18. 5. 1891}\newcommand{\editorInnen}{Martin Anton Müller und Laura Untner}%% latex-leseansicht-abspann.tex
%% Abspann für die Leseansicht.
%% Der Schalter \ifkorrekturansicht ist bereits durch den Vorspann gesetzt.

%% latex-abspann.tex
%% Gemeinsamer Abspann für Korrekturansicht und Leseansicht.
%% Setzt den Schalter \ifkorrekturansicht voraus (gesetzt in den
%% einbindenden Dateien latex-korrekturansicht-abspann.tex bzw.
%% latex-leseansicht-abspann.tex).
%% ---------------------------------------------------------------

\normalsize

% Das esempio-Environment wird nur in der Leseansicht benötigt
\ifkorrekturansicht\else
\newenvironment{esempio}[3]%
{
    \vspace{1.5ex}
    \rlap{\underline{#1}}
    \par
    \setlength{\parindent}{0cm}
    \nopagebreak
    \leftskip=#2cm
    \rightskip=#3cm
}
{
    \par
}
\fi

\doendnotes{C}
\bigskip
\vfill

\clearpage

\footnotesize

\ifkorrekturansicht
  \lohead{\textsc{register}}
\fi

% theindex-Environment neu definieren ohne reledmac
\makeatletter
\renewenvironment{theindex}{%
  \ifkorrekturansicht
    \section*{\indexname}%
  \else
    \subsubsection*{Index der erwähnten Entitäten}%
  \fi
  \setlength{\parindent}{0pt}%
  \setlength{\parskip}{0pt plus 0.3pt}%
  \let\item\@idxitem
}{%
  \ifkorrekturansicht\clearpage\fi
}
\makeatother

\IfFileExists{\jobname-pw.ind}{\input{\jobname-pw.ind}}{}

% Quellenangabe nur in der Leseansicht
\ifkorrekturansicht\else
% Fallback-Definitionen, falls die .tex-Datei \titel etc. nicht gesetzt hat
\providecommand{\titel}{}
\providecommand{\editorInnen}{}
\providecommand{\dateiname}{\jobname}

\vspace{3cm}

\vfill

\footnotesize
\textsc{Quelle}: \titel. Herausgegeben von {\editorInnen}. In: \emph{Arthur Schnitzler: Briefwechsel mit Autorinnen und Autoren}.
 Digitale Edition, https://schnitzler-briefe.acdh.oeaw.ac.at/{\dateiname}.html (Stand \today)
\fi

\end{document}


