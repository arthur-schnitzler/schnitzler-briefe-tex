%% latex-korrekturansicht-vorspann.tex
%% Vorspann für die Korrekturansicht.
%% Lädt die gemeinsame Datei latex-vorspann.tex mit gesetztem Schalter.

\newif\ifkorrekturansicht
\korrekturansichttrue

\input{../tex-inputs/latex-vorspann}


\section[Hugo von Hofmannsthal an Arthur Schnitzler, {[}9. 5. 1910{]}]{L01928 Hugo von Hofmannsthal an Arthur Schnitzler, {[}9. 5. 1910{]}}
\nopagebreak\mylabel{L01928v}
\rehead{ }\normalsize\beginnumbering\briefempfaengerindex{Schnitzler, Arthur@\textsc{Schnitzler, Arthur}!zzzHofmannsthal, Hugo von@\emph{von Hugo von Hofmannsthal}!1910-05-091@{{[}9. 5. 1910{]}}|(be}
\toendnotes[C]{\smallbreak\pagebreak[2]}\Standort{CUL, Schnitzler, B 43.}
\physDesc{Telegramm, 155 Zeichen
\newline{}maschinell
\newline{}Versand: Stempel des Telegrafenbeamten: »Sedlacek\pwindex{Sedlacek 1910-05-09 – 1910-05-09@\textsc{Sedlaček} (1910-05-09 – 1910-05-09), \emph{Telegrafenbeamter/Telegrafenbeamtin}|pw}« 
\newline{}Schnitzler: mit Bleistift datiert: »9/5 10« 
\newline{}Ordnung: mit Bleistift von unbekannter Hand nummeriert:
                                    »318« }
\buchAbdrucke{\weitereDrucke{Hugo von Hofmannsthal, Arthur Schnitzler: \emph{Briefwechsel}. Frankfurt am Main: \emph{S. Fischer} 1964, S. 249.} }\toendnotes[C]{\smallbreak}
\pstart
           {\pb}budapest\oindex{Budapest@\textbf{Budapest}, \emph{P.PPLC}|pw} 51-786 22/21 9{ }5 50\pend
           \vspace{0.5em}
\pstart
           verzweifelt ueber ungeschicklichkeit versuche telephonisch \label{K_L01928-1v}\edtext{ordnen}{\lemma{\textnormal{\emph{ordnen}}}\Cendnote{\textnormal{Das Schreiben bezieht sich auf die bevorstehende Aufführung von \emph{Cristinas Heimreise}\pwindex{Cristinas Heimreise. Komoedie@\emph{Cristinas Heimreise. Komödie}|pwk} am 13. 5. 1910 in Wien\oindex{Wien@\textbf{Wien}, \emph{A.ADM2}|pwk}, bei der Hofmannsthal\pwindex{Hofmannsthal, Hugo von 1874-02-01 – 1929-07-15@\textsc{Hofmannsthal, Hugo von} (1874-02-01 – 1929-07-15), \emph{Schriftsteller/Schriftstellerin}|pwk} zuerst keine (Frei-)Karten zur Disposition bekommen
                  hatte.}}}\label{K_L01928-1} habe selber anscheinend keinen platz im haus\oindex{Theater an der Wien@\textbf{Theater an der Wien}, \emph{Theater (K.THE)}|pwv}\pend
           \pstart herzlichstes =\spacefill\mbox{hugo .+=}\pend{}\selectlanguage{ngerman}\endnumbering\briefempfaengerindex{Schnitzler, Arthur@\textsc{Schnitzler, Arthur}!zzzHofmannsthal, Hugo von@\emph{von Hugo von Hofmannsthal}!1910-05-091@{{[}9. 5. 1910{]}}|)be}\mylabel{L01928h}  \normalsize

\doendnotes{C}
\bigskip
\vfill

\clearpage

\footnotesize

\lohead{\textsc{register}}

% Definiere theindex-Environment komplett neu ohne reledmac
\makeatletter
\renewenvironment{theindex}{%
  \section*{\indexname}%
  \setlength{\parindent}{0pt}%
  \setlength{\parskip}{0pt plus 0.3pt}%
  \let\item\@idxitem
}{%
  \clearpage
}
\makeatother

\IfFileExists{\jobname-pw.ind}{\input{\jobname-pw.ind}}{}

\end{document}

      