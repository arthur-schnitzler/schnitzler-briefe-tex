%% latex-leseansicht-vorspann.tex
%% Vorspann für die Leseansicht.
%% Lädt die gemeinsame Datei latex-vorspann.tex mit nicht gesetztem Schalter.

\newif\ifkorrekturansicht
\korrekturansichtfalse

\input{../tex-inputs/latex-vorspann}


\section[Hugo von Hofmannsthal an Arthur Schnitzler, {[}9. 5. 1910{]}]{L01928 Hugo von Hofmannsthal an Arthur Schnitzler, {[}9. 5. 1910{]}}
\nopagebreak\mylabel{L01928v}
\rehead{ }\normalsize\beginnumbering\briefempfaengerindex{Schnitzler, Arthur@\textsc{Schnitzler, Arthur}!zzzHofmannsthal, Hugo von@\emph{von Hugo von Hofmannsthal}!1910-05-091@{{[}9. 5. 1910{]}}|(be}
\toendnotes[C]{\smallbreak\pagebreak[2]}
\correspDesc{Versand  durch Hugo von Hofmannsthal am [9. 5. 1910] in Budapest
\newline{}Erhalt  durch Arthur Schnitzler am [9. 5. 1910] in Wien}\toendnotes[C]{\smallbreak}
\Standort{CUL, Schnitzler, B 43.}
\physDesc{Telegramm, 155 Zeichen
\newline{}maschinell
\newline{}Versand: Stempel des Telegrafenbeamten: »Sedlacek\pwindex{Sedlaček 9.\,5.\,1910 – 9.\,5.\,1910@\textsc{Sedlaček} (9.\,5.\,1910 – 9.\,5.\,1910), \emph{Telegrafenbeamter/Telegrafenbeamtin}|pw}« 
\newline{}Schnitzler: mit Bleistift datiert: »9/5 10« 
\newline{}Ordnung: mit Bleistift von unbekannter Hand nummeriert:
                                    »318« }
\buchAbdrucke{\weitereDrucke{Hugo von Hofmannsthal, Arthur Schnitzler: \emph{Briefwechsel}. Herausgegeben von Therese Nickl und Heinrich Schnitzler. Frankfurt am Main: \emph{S. Fischer} 1964, S. 249.} }\toendnotes[C]{\smallbreak}
\pstart
           {\pb}budapest\oindex{Budapest@\textbf{Budapest}, \emph{Hauptstadt}|pw} 51-786 22/21 9{ }5 50\pend
           \vspace{0.5em}
\pstart
           verzweifelt ueber ungeschicklichkeit versuche telephonisch \label{K_L01928-1v}\edtext{ordnen}{\lemma{\textnormal{\emph{ordnen}}}\Cendnote{\textnormal{Das Schreiben bezieht sich auf die bevorstehende Aufführung von \emph{Cristinas Heimreise}\pwindex{Hofmannsthal, Hugo von 1.\,2.\,1874 Wien – 15.\,7.\,1929 Rodaun@\textsc{Hofmannsthal, Hugo von} (1.\,2.\,1874 Wien – 15.\,7.\,1929 Rodaun), \emph{Schriftsteller}!Cristinas Heimreise. Komödie@\strich\emph{Cristinas Heimreise. Komödie}|pwk} am 13. 5. 1910 in Wien\oindex{Wien@\textbf{Wien}, \emph{Verwaltungsgebiet}|pwk}, bei der Hofmannsthal\pwindex{Hofmannsthal, Hugo von 1.\,2.\,1874 Wien – 15.\,7.\,1929 Rodaun@\textsc{Hofmannsthal, Hugo von} (1.\,2.\,1874 Wien – 15.\,7.\,1929 Rodaun), \emph{Schriftsteller}|pwk} zuerst keine (Frei-)Karten zur Disposition bekommen
                  hatte.}}}\label{K_L01928-1} habe selber anscheinend keinen platz im haus\oindex{Wien@\textbf{Wien}!VI., Mariahilf@\textbf{VI., Mariahilf}!Theater an der Wien@\textbf{Theater an der Wien}, \emph{Theater}|pwv}\pend
           \pstart herzlichstes =\spacefill\mbox{hugo .+=}\pend{}\selectlanguage{ngerman}\endnumbering\briefempfaengerindex{Schnitzler, Arthur@\textsc{Schnitzler, Arthur}!zzzHofmannsthal, Hugo von@\emph{von Hugo von Hofmannsthal}!1910-05-091@{{[}9. 5. 1910{]}}|)be}\mylabel{L01928h}  \newcommand{\dateiname}{L01928}\newcommand{\titel}{Hugo von Hofmannsthal an Arthur Schnitzler, [9. 5. 1910]}\newcommand{\editorInnen}{Martin Anton Müller und Gerd-Hermann Susen}%% latex-leseansicht-abspann.tex
%% Abspann für die Leseansicht.
%% Der Schalter \ifkorrekturansicht ist bereits durch den Vorspann gesetzt.

%% latex-abspann.tex
%% Gemeinsamer Abspann für Korrekturansicht und Leseansicht.
%% Setzt den Schalter \ifkorrekturansicht voraus (gesetzt in den
%% einbindenden Dateien latex-korrekturansicht-abspann.tex bzw.
%% latex-leseansicht-abspann.tex).
%% ---------------------------------------------------------------

\normalsize

% Das esempio-Environment wird nur in der Leseansicht benötigt
\ifkorrekturansicht\else
\newenvironment{esempio}[3]%
{
    \vspace{1.5ex}
    \rlap{\underline{#1}}
    \par
    \setlength{\parindent}{0cm}
    \nopagebreak
    \leftskip=#2cm
    \rightskip=#3cm
}
{
    \par
}
\fi

\doendnotes{C}
\bigskip
\vfill

\clearpage

\footnotesize

\ifkorrekturansicht
  \lohead{\textsc{register}}
\fi

% theindex-Environment neu definieren ohne reledmac
\makeatletter
\renewenvironment{theindex}{%
  \ifkorrekturansicht
    \section*{\indexname}%
  \else
    \subsubsection*{Index der erwähnten Entitäten}%
  \fi
  \setlength{\parindent}{0pt}%
  \setlength{\parskip}{0pt plus 0.3pt}%
  \let\item\@idxitem
}{%
  \ifkorrekturansicht\clearpage\fi
}
\makeatother

\IfFileExists{\jobname-pw.ind}{\input{\jobname-pw.ind}}{}

% Quellenangabe nur in der Leseansicht
\ifkorrekturansicht\else
% Fallback-Definitionen, falls die .tex-Datei \titel etc. nicht gesetzt hat
\providecommand{\titel}{}
\providecommand{\editorInnen}{}
\providecommand{\dateiname}{\jobname}

\vspace{3cm}

\vfill

\footnotesize
\textsc{Quelle}: \titel. Herausgegeben von {\editorInnen}. In: \emph{Arthur Schnitzler: Briefwechsel mit Autorinnen und Autoren}.
 Digitale Edition, https://schnitzler-briefe.acdh.oeaw.ac.at/{\dateiname}.html (Stand \today)
\fi

\end{document}


