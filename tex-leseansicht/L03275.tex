%% latex-leseansicht-vorspann.tex
%% Vorspann für die Leseansicht.
%% Lädt die gemeinsame Datei latex-vorspann.tex mit nicht gesetztem Schalter.

\newif\ifkorrekturansicht
\korrekturansichtfalse

\input{../tex-inputs/latex-vorspann}


\section[ Felix Salten an Arthur Schnitzler, {[}30.? 11. 1897{]}]{L03275 Felix Salten an Arthur Schnitzler,  [30.? 11. 1897]}
\nopagebreak\mylabel{L03275v}
\rehead{ }\normalsize\beginnumbering\briefempfaengerindex{Schnitzler, Arthur@\textsc{Schnitzler, Arthur}!zzzSalten, Felix@\emph{von Felix Salten}!1897-11-302@{{[}30.? 11. 1897{]}}|(be}
\toendnotes[C]{\smallbreak\pagebreak[2]}
\correspDesc{Versand  durch Felix Salten am [30.? 11. 1897] in Wien
\newline{}Erhalt  durch Arthur Schnitzler am [30.? 11. 1897] in Wien}\toendnotes[C]{\smallbreak}
\Standort{CUL, Schnitzler, B 89, A 2.}
\physDesc{Brief, 1 Blatt, 1 Seite, 124 Zeichen
\newline{}Handschrift: Bleistift, lateinische Kurrent
\newline{}Schnitzler: mit Bleistift datiert: »Nov 97« 
\newline{}Ordnung: mit Bleistift von unbekannter Hand nummeriert: »98« }\toendnotes[C]{\smallbreak}
\pstart
           \noindent{}{\pb}Lieber, vielleicht können Sie diesen \label{K_L03275-1v}\edtext{Brief}{\lemma{\textnormal{\emph{Brief}}}\Cendnote{\textnormal{Schnitzler datiert auf
                        »Nov{[}ember{]} 97«. In diesem Monat sahen
                  sich die beiden häufig. Das undatierte Korrespondenzstück dürfte vom 30. 11. 1897 stammen.
                  Zu diesem Tag hat Schnitzler im \emph{Tagebuch}\pwindex{Schnitzler, Arthur 15.\,5.\,1862 Wien – 21.\,10.\,1931 ebd.@\textsc{Schnitzler, Arthur} (15.\,5.\,1862 Wien – 21.\,10.\,1931 ebd.), \emph{Schriftsteller, Mediziner}!Tagebuch@\strich\emph{Tagebuch}|pwk} die Übermittlung eines Briefes von
                     Salten\pwindex{Salten, Felix 6.\,9.\,1869 Budapest – 8.\,10.\,1945 Zürich@\textsc{Salten, Felix} (6.\,9.\,1869 Budapest – 8.\,10.\,1945 Zürich), \emph{Schriftsteller, Journalist, Chefredakteur}|pwk} an Anna Loeb\pwindex{Epstein, Anna 6.\,3.\,1877 Wien – 16.\,3.\,1943 Konzentrationslager Theresienstadt@\textsc{Epstein, Anna} (6.\,3.\,1877 Wien – 16.\,3.\,1943 Konzentrationslager Theresienstadt)|pwk} vermerkt: »Bei Loebs\pwindex{Loeb, Louis 29.\,6.\,1842 Mattersdorf – 6.\,6.\,1921 Wien@\textsc{Loeb, Louis} (29.\,6.\,1842 Mattersdorf – 6.\,6.\,1921 Wien), \emph{Bankier}|pw}\pwindex{Loeb, Regina 1850 – 5.\,2.\,1918 Wien@\textsc{Loeb, Regina} (1850 – 5.\,2.\,1918 Wien)|pw}\pwindex{Epstein, Anna 6.\,3.\,1877 Wien – 16.\,3.\,1943 Konzentrationslager Theresienstadt@\textsc{Epstein, Anna} (6.\,3.\,1877 Wien – 16.\,3.\,1943 Konzentrationslager Theresienstadt)|pw}\pwindex{Pollaczek, Clara Katharina 15.\,1.\,1875 Wien – 22.\,7.\,1951 ebd.@\textsc{Pollaczek, Clara Katharina} (15.\,1.\,1875 Wien – 22.\,7.\,1951 ebd.), \emph{Schriftstellerin}|pw}. – Der Anna\pwindex{Epstein, Anna 6.\,3.\,1877 Wien – 16.\,3.\,1943 Konzentrationslager Theresienstadt@\textsc{Epstein, Anna} (6.\,3.\,1877 Wien – 16.\,3.\,1943 Konzentrationslager Theresienstadt)|pw} einen Brief von S.\pwindex{Salten, Felix 6.\,9.\,1869 Budapest – 8.\,10.\,1945 Zürich@\textsc{Salten, Felix} (6.\,9.\,1869 Budapest – 8.\,10.\,1945 Zürich), \emph{Schriftsteller, Journalist, Chefredakteur}|pw} übergeben. – (Rocktasche.)«}}}\label{K_L03275-1} jetzt in
               Ihre Rocktasche stecken?\pend
           
\pstart
           Vielen Dank und wenn möglich auf heut{ }Abend\pend
           \pstart Ihr \spacefill\mbox{Salten}\pend{}\selectlanguage{ngerman}\endnumbering\briefempfaengerindex{Schnitzler, Arthur@\textsc{Schnitzler, Arthur}!zzzSalten, Felix@\emph{von Felix Salten}!1897-11-302@{{[}30.? 11. 1897{]}}|)be}\mylabel{L03275h}  \newcommand{\dateiname}{L03275}\newcommand{\titel}{Felix Salten an Arthur Schnitzler, [30.? 11. 1897]}\newcommand{\editorInnen}{Martin Anton Müller und Laura Untner}%% latex-leseansicht-abspann.tex
%% Abspann für die Leseansicht.
%% Der Schalter \ifkorrekturansicht ist bereits durch den Vorspann gesetzt.

%% latex-abspann.tex
%% Gemeinsamer Abspann für Korrekturansicht und Leseansicht.
%% Setzt den Schalter \ifkorrekturansicht voraus (gesetzt in den
%% einbindenden Dateien latex-korrekturansicht-abspann.tex bzw.
%% latex-leseansicht-abspann.tex).
%% ---------------------------------------------------------------

\normalsize

% Das esempio-Environment wird nur in der Leseansicht benötigt
\ifkorrekturansicht\else
\newenvironment{esempio}[3]%
{
    \vspace{1.5ex}
    \rlap{\underline{#1}}
    \par
    \setlength{\parindent}{0cm}
    \nopagebreak
    \leftskip=#2cm
    \rightskip=#3cm
}
{
    \par
}
\fi

\doendnotes{C}
\bigskip
\vfill

\clearpage

\footnotesize

\ifkorrekturansicht
  \lohead{\textsc{register}}
\fi

% theindex-Environment neu definieren ohne reledmac
\makeatletter
\renewenvironment{theindex}{%
  \ifkorrekturansicht
    \section*{\indexname}%
  \else
    \subsubsection*{Index der erwähnten Entitäten}%
  \fi
  \setlength{\parindent}{0pt}%
  \setlength{\parskip}{0pt plus 0.3pt}%
  \let\item\@idxitem
}{%
  \ifkorrekturansicht\clearpage\fi
}
\makeatother

\IfFileExists{\jobname-pw.ind}{\input{\jobname-pw.ind}}{}

% Quellenangabe nur in der Leseansicht
\ifkorrekturansicht\else
% Fallback-Definitionen, falls die .tex-Datei \titel etc. nicht gesetzt hat
\providecommand{\titel}{}
\providecommand{\editorInnen}{}
\providecommand{\dateiname}{\jobname}

\vspace{3cm}

\vfill

\footnotesize
\textsc{Quelle}: \titel. Herausgegeben von {\editorInnen}. In: \emph{Arthur Schnitzler: Briefwechsel mit Autorinnen und Autoren}.
 Digitale Edition, https://schnitzler-briefe.acdh.oeaw.ac.at/{\dateiname}.html (Stand \today)
\fi

\end{document}


