%% latex-korrekturansicht-vorspann.tex
%% Vorspann für die Korrekturansicht.
%% Lädt die gemeinsame Datei latex-vorspann.tex mit gesetztem Schalter.

\newif\ifkorrekturansicht
\korrekturansichttrue

\input{../tex-inputs/latex-vorspann}


\section[Arthur Schnitzler an Hermann Bahr, 29. 1. 1906]{L01578 Arthur Schnitzler an Hermann Bahr, 29. 1. 1906}
\nopagebreak\mylabel{L01578v}
\rehead{ }\normalsize\beginnumbering\briefempfaengerindex{Bahr, Hermann@\textsc{Bahr, Hermann}!zzzSchnitzler, Arthur@\emph{von Arthur Schnitzler}!1906-01-292@{29. 1. 1906}|(be}
\toendnotes[C]{\smallbreak\pagebreak[2]}\Standort{TMW, HS AM 23378 Ba.}
\physDesc{Brief, 1 Blatt, 2 Seiten, 562 Zeichen
\newline{}Handschrift: schwarze Tinte, deutsche Kurrent
\newline{}Ordnung: Lochung }
\buchAbdrucke{\weitereDrucke{1) Arthur Schnitzler: \emph{The Letters of Arthur Schnitzler to Hermann Bahr}. Chapel Hill: \emph{The University of North Carolina Press} 1978, S. 93.} \weitereDrucke{2) Hermann Bahr, Arthur Schnitzler: \emph{Briefwechsel, Aufzeichnungen, Dokumente (1891–1931)}. Göttingen: \emph{Wallstein} 2018, S. 372.} }\toendnotes[C]{\smallbreak}
\pstart
           {\pb}\textcolor{gray}{\textbf{Dr. Arthur Schnitzler}}\hfill 29. 1. 906.\pend
           
\pstart
           \textcolor{gray}{\textbf{Wien, XVIII. Spoettelgasse 7\oindex{Edmund-Weiss-Gasse 7@\textbf{Edmund-Weiß-Gasse 7}, \emph{Wohngebäude (K.WHS)}|pw}.}}\pend
           
\pstart{}lieber Hermann, \pend\vspace{0.5em}
\pstart
           es thut mir natürlich rieſig leid, daſs man nun auch mein Stück\pwindex{Ruf des Lebens. Schauspiel in drei Akten@\emph{Der Ruf des Lebens. Schauspiel in drei Akten}|pwv} benützt, um dir was unangenehmes
               anzuthun, aber ich bitte dich ja nicht dieſen Fall als Cabinetsfrage zwiſchen dir und
               der Intendanz\orgindex{Nationaltheater Muenchen@Nationaltheater München|pwv} zu behandeln.
               Intereſſiren wird dich unter dieſen Umſtänden vielleicht daſs mir das \label{K_L01578-1v}\edtext{Petersburger \uline{kaiser}{\pb}\uline{liche} Theater\oindex{Alexandrinski-Theater@\textbf{Alexandrinski-Theater}, \emph{Theater (K.THE)}|pw} telegrafiſch tauſend Rubel}{\lemma{\textnormal{\emph{Petersburger … Rubel}}}\Cendnote{\textnormal{Vgl. A. S.: \emph{Tagebuch}, 26. 1. 1906.
               }}}\label{K_L01578-1} Garantie bieten lieſs, wenn ich das Erſcheinen des \uline{Buches}\pwindex{Ruf des Lebens. Schauspiel in drei Akten@\emph{Der Ruf des Lebens. Schauspiel in drei Akten}|pwv}{ }\introOben{}in deutſcher Sprache\introOben{} bis \label{K_L01578-2v}\edtext{Oktober hinausſchieben}{\lemma{\textnormal{\emph{Oktober hinausſchieben}}}\Cendnote{\textnormal{\emph{Der Ruf des Lebens}\pwindex{Ruf des Lebens. Schauspiel in drei Akten@\emph{Der Ruf des Lebens. Schauspiel in drei Akten}|pwk} erschien im März 1906.}}}\label{K_L01578-2} wollte.\pend
           
\pstart
           Herzlichſt dein{\\[\baselineskip]}\spacefill\mbox{A.}\pend
           \leftskip=0em{}\selectlanguage{ngerman}\vspace{1em}
\pstart
           \noindent{}Kann man dich nicht d\damage{oc}h vielleicht einmal ſehen? –\pend
           
\pstart
           Viele Grüße von meiner Frau\pwindex{Schnitzler, Olga 17.01.1882 – 13.01.1970@\textsc{Schnitzler, Olga} (17.01.1882 – 13.01.1970), \emph{Schauspieler/Schauspielerin, Sänger/Sängerin}|pwv}.\pend
           \selectlanguage{ngerman}\endnumbering\briefempfaengerindex{Bahr, Hermann@\textsc{Bahr, Hermann}!zzzSchnitzler, Arthur@\emph{von Arthur Schnitzler}!1906-01-292@{29. 1. 1906}|)be}\mylabel{L01578h}  \normalsize

\doendnotes{C}
\bigskip
\vfill

\clearpage

\footnotesize

\lohead{\textsc{register}}

% Definiere theindex-Environment komplett neu ohne reledmac
\makeatletter
\renewenvironment{theindex}{%
  \section*{\indexname}%
  \setlength{\parindent}{0pt}%
  \setlength{\parskip}{0pt plus 0.3pt}%
  \let\item\@idxitem
}{%
  \clearpage
}
\makeatother

\IfFileExists{\jobname-pw.ind}{\input{\jobname-pw.ind}}{}

\end{document}

      