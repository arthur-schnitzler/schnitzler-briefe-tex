%% latex-leseansicht-vorspann.tex
%% Vorspann für die Leseansicht.
%% Lädt die gemeinsame Datei latex-vorspann.tex mit nicht gesetztem Schalter.

\newif\ifkorrekturansicht
\korrekturansichtfalse

\input{../tex-inputs/latex-vorspann}


         
         \renewcommand{\erwaehntePersonen}{Personen: Alfred Dreyfus, Félix Fénéon, Paul Goldmann, Lucien Millevoye, Leo P. Reckford, Rena R. Schlein, Heinrich Schnitzler, Rowland Strong}
         \renewcommand{\erwaehnteInstitutionen}{Institutionen: Frankfurter Zeitung}
         \renewcommand{\erwaehnteOrte}{Orte: Paris, Wien, rue Feydeau}
         \renewcommand{\erwaehnteWerke}{Werke: Die Affaire Dreyfus, Die Enthüllungen über die Affaire Dreyfus, Dreyfus, die öffentliche Meinung und die deutsche Regierung, Justice{\rufezeichen}, La Patrie. Organe de la défense nationale, Le Petit Parisien. Journal quotidien du soir, Ritterlichkeit, Wiener Zeitung, [Duell Goldmann – Millevoye]}
               \section[Arthur Schnitzler an Paul Goldmann, 21. 11. 1896]{ Arthur Schnitzler an Paul Goldmann, 21. 11. 1896}\nopagebreak\mylabel{v}\rehead{ }\begin{ledgroupsized}[t]{13cm}\normalsize\beginnumbering \toendnotes[C]{\smallbreak\pagebreak[2]} \Standort{DLA, A:Schnitzler, HS85.1.5681.}
\physDesc{Telegramm, Fotokopie, 83 Zeichen
\newline{}maschinell
\newline{}Versand: von unbekannter Hand datiert: »21. 11. 96« 
\newline{}Ordnung: mit blauem Kugelschreiber von unbekannter Hand teilweise den
                                 schwer leserlichen Text nachgezogen 
\newline{}Zusatz: Von den Korrespondenzstücken Schnitzler\pwindex{Schnitzler, Arthur 15.05.1862 – 21.10.1931@\textsc{Schnitzler, Arthur} (15.05.1862 – 21.10.1931), \emph{Schriftsteller, Mediziner}|pw}s an Goldmann\pwindex{Goldmann, Paul 31.01.1865 – 25.09.1935@\textsc{Goldmann, Paul} (31.01.1865 – 25.09.1935), \emph{Schriftsteller, Journalist}|pw} fehlt weitgehend jede Spur. In der Edition von
                                    Ritterlichkeit\pwindex{Schnitzler, Arthur 15.05.1862 – 21.10.1931@\textsc{Schnitzler, Arthur} (15.05.1862 – 21.10.1931), \emph{Schriftsteller, Mediziner}!Ritterlichkeit1977@\strich\emph{Ritterlichkeit} {[}1977{]}|pw}
                                    (1975) schreibt die Herausgeberin Rena R. Schlein\pwindex{Schlein, Rena R. *~1919-06-20@\textsc{Schlein, Rena R.} (*~1919-06-20)|pw}: »Zwei Telegramme
                                    und ein Brief Schnitzler\pwindex{Schnitzler, Arthur 15.05.1862 – 21.10.1931@\textsc{Schnitzler, Arthur} (15.05.1862 – 21.10.1931), \emph{Schriftsteller, Mediziner}|pw}s
                                    an Goldmann\pwindex{Goldmann, Paul 31.01.1865 – 25.09.1935@\textsc{Goldmann, Paul} (31.01.1865 – 25.09.1935), \emph{Schriftsteller, Journalist}|pw} wurden mir
                                    von Dr. Leo P. Reckford\pwindex{Reckford, Leo P. 1903-05-03 – 1988-10-19@\textsc{Reckford, Leo P.} (1903-05-03 – 1988-10-19), \emph{Mediziner}|pw},
                                    der diese Dokumente von der Familie Goldmann\pwindex{Goldmann, Paul 31.01.1865 – 25.09.1935@\textsc{Goldmann, Paul} (31.01.1865 – 25.09.1935), \emph{Schriftsteller, Journalist}|pw}s zum Geschenk bekam, für meine
                                    Arbeit zur Verfügung gestellt« (S. 1). Reckford\pwindex{Reckford, Leo P. 1903-05-03 – 1988-10-19@\textsc{Reckford, Leo P.} (1903-05-03 – 1988-10-19), \emph{Mediziner}|pw} starb 1988, seine
                                 Nachkommen haben keine Kenntnis von diesen (und etwaigen weiteren)
                                 Korrespondenzstücken und sie sind auch nicht auffindbar. Rena R. Schlein\pwindex{Schlein, Rena R. *~1919-06-20@\textsc{Schlein, Rena R.} (*~1919-06-20)|pw} kam
                                    1919 zur Welt. Ein Kontakt konnte nicht hergestellt
                                 werden. Die Kopie des vorliegenden Telegramms dürfte durch Reckford\pwindex{Reckford, Leo P. 1903-05-03 – 1988-10-19@\textsc{Reckford, Leo P.} (1903-05-03 – 1988-10-19), \emph{Mediziner}|pw} oder Schlein\pwindex{Schlein, Rena R. *~1919-06-20@\textsc{Schlein, Rena R.} (*~1919-06-20)|pw} in den Besitz Heinrich Schnitzlers\pwindex{Schnitzler, Heinrich 09.08.1902 – 12.07.1982@\textsc{Schnitzler, Heinrich} (09.08.1902 – 12.07.1982), \emph{Regisseur, Schauspieler}|pw} gelangt
                                 sein. }\buchAbdrucke{\weitereDrucke{\pwindex{Schnitzler, Arthur 15.05.1862 – 21.10.1931@\textsc{Schnitzler, Arthur} (15.05.1862 – 21.10.1931), \emph{Schriftsteller, Mediziner}!Ritterlichkeit1977@\strich\emph{Ritterlichkeit} {[}1977{]}|pwk}Arthur Schnitzler: \emph{Ritterlichkeit. Fragment aus dem Nachlaß}. Bonn: \emph{Bouvier Verlag Herbert Grundmann} 1975, S. 5 (Abhandlungen zur Kunst-, Musik- und
                        Literaturwissenschaft, 176).} }\toendnotes[C]{\smallbreak}\pstart{}{\pb}PAUL GOLDMANN{ }PARIS\oindex{Paris@\textbf{Paris}|pw}\pend{}\pstart{}24 RUE FEYDEAU\oindex{rue Feydeau@\textbf{rue Feydeau}|pw}\pend{}{\bigskip}\pstart
           \noindent{}\centering{}{\pb}FR WIEN\oindex{Wien@\textbf{Wien}|pw} 72\textcolor{gray}{×}\-\textcolor{gray}{×}685\pend
           \pstart
           = SENDE MIR SOFORT \label{K_L02684-1v}\edtext{NACHRICHT}{\lemma{\textnormal{\emph{Nachricht}}}\Cendnote{\textnormal{Entrüstet über Goldmann\pwindex{Goldmann, Paul 31.01.1865 – 25.09.1935@\textsc{Goldmann, Paul} (31.01.1865 – 25.09.1935), \emph{Schriftsteller, Journalist}|pwk}s Berichterstattung über die Dreyfus\pwindex{Dreyfus, Alfred 1859-10-09 – 1935-07-12@\textsc{Dreyfus, Alfred} (1859-10-09 – 1935-07-12), \emph{Militär}|pwk}-Affäre für die \emph{Frankfurter Zeitung}\orgindex{Frankfurter Zeitung@Frankfurter Zeitung|pwk} (G.\pwindex{Goldmann, Paul 31.01.1865 – 25.09.1935@\textsc{Goldmann, Paul} (31.01.1865 – 25.09.1935), \emph{Schriftsteller, Journalist}|pwk} [ = Paul Goldmann\pwindex{Goldmann, Paul 31.01.1865 – 25.09.1935@\textsc{Goldmann, Paul} (31.01.1865 – 25.09.1935), \emph{Schriftsteller, Journalist}|pwk}]: \emph{Die
                           Enthüllungen über die Affaire Dreyfus}\pwindex{Enthuellungen ueber die Affaire Dreyfus1896-09-16@\emph{Die Enthüllungen über die Affaire Dreyfus} {[}1896-09-16{]}|pwk}, Jg. 41, Nr. 258,
                           16. 9. 1896, Erstes Morgenblatt, S. 1. G.\pwindex{Goldmann, Paul 31.01.1865 – 25.09.1935@\textsc{Goldmann, Paul} (31.01.1865 – 25.09.1935), \emph{Schriftsteller, Journalist}|pwk} [ = Paul Goldmann\pwindex{Goldmann, Paul 31.01.1865 – 25.09.1935@\textsc{Goldmann, Paul} (31.01.1865 – 25.09.1935), \emph{Schriftsteller, Journalist}|pwk}]: \emph{Die
                           Affaire Dreyfus}\pwindex{Affaire Dreyfus1896-11-11@\emph{Die Affaire Dreyfus} {[}1896-11-11{]}|pwk}, Jg. 41, Nr. 314, 11. 11. 1896, Zweites
                        Morgenblatt, S. 1. G.\pwindex{Goldmann, Paul 31.01.1865 – 25.09.1935@\textsc{Goldmann, Paul} (31.01.1865 – 25.09.1935), \emph{Schriftsteller, Journalist}|pwk} [ = Paul Goldmann\pwindex{Goldmann, Paul 31.01.1865 – 25.09.1935@\textsc{Goldmann, Paul} (31.01.1865 – 25.09.1935), \emph{Schriftsteller, Journalist}|pwk}]: \emph{Dreyfus, die öffentliche Meinung und die deutsche Regierung}\pwindex{Dreyfus, die oeffentliche Meinung und die deutsche Regierung1896-11-12@\emph{Dreyfus, die öffentliche Meinung und die deutsche Regierung} {[}1896-11-12{]}|pwk},
                        Jg. 41, Nr. 315, 12. 11. 1896, Erstes Morgenblatt,
                     S. 1.), in der für die Wiederaufnahme des Prozesses gegen Dreyfus\pwindex{Dreyfus, Alfred 1859-10-09 – 1935-07-12@\textsc{Dreyfus, Alfred} (1859-10-09 – 1935-07-12), \emph{Militär}|pwk} Partei ergriffen wurde, hatte der
                     antisemitische Chefredakteur Lucien
                        Millevoye\pwindex{Millevoye, Lucien 1850-08-01 – 1918-03-25@\textsc{Millevoye, Lucien} (1850-08-01 – 1918-03-25), \emph{Politiker, Journalist}|pwk} über ihn geschrieben (\emph{Justice!}\pwindex{Millevoye, Lucien 1850-08-01 – 1918-03-25@\textsc{Millevoye, Lucien} (1850-08-01 – 1918-03-25), \emph{Politiker, Journalist}!Justice1896-11-15@\strich\emph{Justice{\rufezeichen}} {[}1896-11-15{]}|pwk} In: \emph{La Patrie}\pwindex{Patrie. Organe de la defense nationale1841 – 1937@\emph{La Patrie. Organe de la défense nationale} {[}1841 – 1937{]}|pwk}, Jg. 56, 15. 11. 1896,
                        S. 1.): »\begin{otherlanguage}{french}Le lâche coquin se croit à l’arbi.\end{otherlanguage}« – Der ungezogene Feigling glaubt sich in Sicherheit. Daraufhin wurde
                     er von Goldmann\pwindex{Goldmann, Paul 31.01.1865 – 25.09.1935@\textsc{Goldmann, Paul} (31.01.1865 – 25.09.1935), \emph{Schriftsteller, Journalist}|pwk} zum Pistolenduell
                     gefordert. Goldmann\pwindex{Goldmann, Paul 31.01.1865 – 25.09.1935@\textsc{Goldmann, Paul} (31.01.1865 – 25.09.1935), \emph{Schriftsteller, Journalist}|pwk}s Sekundanten\pwindex{Feneon, Felix 1861-06-22 – 1944-02-29@\textsc{Fénéon, Félix} (1861-06-22 – 1944-02-29), \emph{Schriftsteller, Journalist, Kritiker}|pwkv}\pwindex{Strong, Rowland 1865-07-10 – 1924-01-05@\textsc{Strong, Rowland} (1865-07-10 – 1924-01-05), \emph{Journalist}|pwkv} waren die
                     Journalisten Félix Fénéon\pwindex{Feneon, Felix 1861-06-22 – 1944-02-29@\textsc{Fénéon, Félix} (1861-06-22 – 1944-02-29), \emph{Schriftsteller, Journalist, Kritiker}|pwk} und Rowland Strong\pwindex{Strong, Rowland 1865-07-10 – 1924-01-05@\textsc{Strong, Rowland} (1865-07-10 – 1924-01-05), \emph{Journalist}|pwk}. Nach zwei Kugelwechseln
                     mit 25 Schritt Abstand war niemand verletzt. vgl. A. S.: \emph{Tagebuch}, 23. 11. 1896, ungezeichnete Notiz\pwindex{?? Werk@Nicht ermittelte Verfasserinnen und Verfasser!Duell Goldmann – Millevoye]1896-11-22@\emph{[Duell Goldmann – Millevoye]} {[}1896-11-22{]}|pwkv} in: \emph{Le Petit Parisien}\pwindex{?? Werk@Nicht ermittelte Verfasserinnen und Verfasser!Le Petit Parisien. Journal quotidien du soir1876 – 1944@\emph{Le Petit Parisien. Journal quotidien du soir} {[}1876 – 1944{]}|pwk}, Jg. 21, Nr. 7.331,
                           22. 11. 1896, S. 2 und \emph{Wiener Zeitung}\pwindex{?? Werk@Nicht ermittelte Verfasserinnen und Verfasser!Wiener Zeitung1703@\emph{Wiener Zeitung} {[}1703{]}|pwk}, Nr. 272,
                           22. 11. 1896, S. 11.}}}\label{K_L02684-1h} DEIN{ }\spacefill\mbox{ARTHUR +}\pend
           
         
         \endnumbering\mylabel{h}\end{ledgroupsized}  \newcommand{\dateiname}{L02684}\newcommand{\titel}{Arthur Schnitzler an Paul Goldmann, 21. 11. 1896}\newcommand{\editorInnen}{Martin Anton Müller und Laura Untner}%% latex-leseansicht-abspann.tex
%% Abspann für die Leseansicht.
%% Der Schalter \ifkorrekturansicht ist bereits durch den Vorspann gesetzt.

%% latex-abspann.tex
%% Gemeinsamer Abspann für Korrekturansicht und Leseansicht.
%% Setzt den Schalter \ifkorrekturansicht voraus (gesetzt in den
%% einbindenden Dateien latex-korrekturansicht-abspann.tex bzw.
%% latex-leseansicht-abspann.tex).
%% ---------------------------------------------------------------

\normalsize

% Das esempio-Environment wird nur in der Leseansicht benötigt
\ifkorrekturansicht\else
\newenvironment{esempio}[3]%
{
    \vspace{1.5ex}
    \rlap{\underline{#1}}
    \par
    \setlength{\parindent}{0cm}
    \nopagebreak
    \leftskip=#2cm
    \rightskip=#3cm
}
{
    \par
}
\fi

\doendnotes{C}
\bigskip
\vfill

\clearpage

\footnotesize

\ifkorrekturansicht
  \lohead{\textsc{register}}
\fi

% theindex-Environment neu definieren ohne reledmac
\makeatletter
\renewenvironment{theindex}{%
  \ifkorrekturansicht
    \section*{\indexname}%
  \else
    \subsubsection*{Index der erwähnten Entitäten}%
  \fi
  \setlength{\parindent}{0pt}%
  \setlength{\parskip}{0pt plus 0.3pt}%
  \let\item\@idxitem
}{%
  \ifkorrekturansicht\clearpage\fi
}
\makeatother

\IfFileExists{\jobname-pw.ind}{\input{\jobname-pw.ind}}{}

% Quellenangabe nur in der Leseansicht
\ifkorrekturansicht\else
% Fallback-Definitionen, falls die .tex-Datei \titel etc. nicht gesetzt hat
\providecommand{\titel}{}
\providecommand{\editorInnen}{}
\providecommand{\dateiname}{\jobname}

\vspace{3cm}

\vfill

\footnotesize
\textsc{Quelle}: \titel. Herausgegeben von {\editorInnen}. In: \emph{Arthur Schnitzler: Briefwechsel mit Autorinnen und Autoren}.
 Digitale Edition, https://schnitzler-briefe.acdh.oeaw.ac.at/{\dateiname}.html (Stand \today)
\fi

\end{document}


      