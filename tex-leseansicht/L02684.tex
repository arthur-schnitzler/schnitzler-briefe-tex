\input{../tex-inputs/latex-pdf-vorspann}
\begin{center}
            \textcolor{red}{ENTWURF. ENTZIFFERUNG NOCH NICHT KORREKTURGELESEN}
                      \end{center}
            
               \section[Arthur Schnitzler an Paul Goldmann, 21. 11. 1896]{ Arthur Schnitzler an Paul Goldmann, 21. 11. 1896}\nopagebreak\mylabel{v}\rehead{ }\begin{ledgroupsized}[t]{13cm}\normalsize\beginnumbering\briefempfaengerindex{Goldmann, Paul@\textsc{Goldmann, Paul}!zzzSchnitzler, Arthur@\emph{von Arthur Schnitzler}!1896-11-211@{21. 11. 1896}|(be} \toendnotes[C]{\smallbreak\pagebreak[2]} \Standort{DLA, A:Schnitzler, HS85.1.5681.}
\physDesc{Telegramm, Fotokopie
\newline{}maschinell\newline{}Ordnung: mit blauem Kugelschreiber von unbekannter Hand teilweise den
                                 schwer leserlichen Text nachgezogen \newline{}Zusatz: Von den Korrespondenzstücken Schnitzler\pwindex{Schnitzler, Arthur 15.05.1862 – 21.10.1931@\textsc{Schnitzler, Arthur} (15.05.1862 – 21.10.1931), \emph{Schriftsteller, Mediziner}|pw}s an Goldmann\pwindex{Goldmann, Paul 31.01.1865 – 25.09.1935@\textsc{Goldmann, Paul} (31.01.1865 – 25.09.1935), \emph{Schriftsteller, Journalist}|pw} fehlt weitgehend jede Spur. In der Edition von
                                    Ritterlichkeit\pwindex{Schnitzler, Arthur 15.05.1862 – 21.10.1931@\textsc{Schnitzler, Arthur} (15.05.1862 – 21.10.1931), \emph{Schriftsteller, Mediziner}!Ritterlichkeit1977@\strich\emph{Ritterlichkeit} {[}1977{]}|pw}
                                    (1975) schreibt die Herausgeberin Rena R. Schlein\pwindex{Schlein, Rena R. *~1919-06-20@\textsc{Schlein, Rena R.} (*~1919-06-20)|pw}: »Zwei Telegramme
                                    und ein Brief Schnitzler\pwindex{Schnitzler, Arthur 15.05.1862 – 21.10.1931@\textsc{Schnitzler, Arthur} (15.05.1862 – 21.10.1931), \emph{Schriftsteller, Mediziner}|pw}s
                                    an Goldmann\pwindex{Goldmann, Paul 31.01.1865 – 25.09.1935@\textsc{Goldmann, Paul} (31.01.1865 – 25.09.1935), \emph{Schriftsteller, Journalist}|pw} wurden mir
                                    von Dr. Leo P. Reckford\pwindex{Reckford, Leo P. 1903-05-03 – 1988-10-19@\textsc{Reckford, Leo P.} (1903-05-03 – 1988-10-19), \emph{Mediziner}|pw},
                                    der diese Dokumente von der Familie Goldmann\pwindex{Goldmann, Paul 31.01.1865 – 25.09.1935@\textsc{Goldmann, Paul} (31.01.1865 – 25.09.1935), \emph{Schriftsteller, Journalist}|pw}s zum Geschenk bekam, für meine
                                    Arbeit zur Verfügung gestellt« (S. 1). Reckford\pwindex{Reckford, Leo P. 1903-05-03 – 1988-10-19@\textsc{Reckford, Leo P.} (1903-05-03 – 1988-10-19), \emph{Mediziner}|pw} starb 1988, seine
                                 Nachkommen haben keine Kenntnis von diesen (und etwaigen weiteren)
                                 Korrespondenzstücken und sie sind auch nicht auffindbar. Rena R. Schlein\pwindex{Schlein, Rena R. *~1919-06-20@\textsc{Schlein, Rena R.} (*~1919-06-20)|pw} wäre, wenn
                                 sie noch leben sollte, deutlich über 100 Jahre alt. Ein Kontakt
                                 konnte nicht hergestellt werden. Eine Kopie des vorliegenden
                                 Telegramms dürfte durch Reckford\pwindex{Reckford, Leo P. 1903-05-03 – 1988-10-19@\textsc{Reckford, Leo P.} (1903-05-03 – 1988-10-19), \emph{Mediziner}|pw} oder Schlein\pwindex{Schlein, Rena R. *~1919-06-20@\textsc{Schlein, Rena R.} (*~1919-06-20)|pw} in den Besitz Heinrich Schnitzlers\pwindex{Schnitzler, Heinrich 09.08.1902 – 12.07.1982@\textsc{Schnitzler, Heinrich} (09.08.1902 – 12.07.1982), \emph{Regisseur, Schauspieler}|pw} gelangt sein. }\buchAbdrucke{\weitereDrucke{\pwindex{Schnitzler, Arthur 15.05.1862 – 21.10.1931@\textsc{Schnitzler, Arthur} (15.05.1862 – 21.10.1931), \emph{Schriftsteller, Mediziner}!Ritterlichkeit1977@\strich\emph{Ritterlichkeit} {[}1977{]}|pwk}Arthur Schnitzler: \emph{Ritterlichkeit. Fragment aus dem Nachlaß}. Bonn: \emph{Bouvier Verlag Herbert Grundmann} 1975, S. 5 (Abhandlungen zur Kunst-, Musik- und
                        Literaturwissenschaft, 176).} }\toendnotes[C]{\smallbreak}\pstart{}{\pb}PAUL GOLDMANN{ }PARIS\oindex{Paris@\textbf{Paris}|pw}\pend{}\pstart{}24 RUE FEYDEAU\oindex{rue Feydeau@\textbf{rue Feydeau}|pw}\pend{}{\bigskip}\pstart
           \noindent{}\centering{}{\pb}FR WIEN\oindex{Wien@\textbf{Wien}|pw} 72\textcolor{gray}{×}\-\textcolor{gray}{×}685\pend
           \pstart
           = SENDE MIR SOFORT \label{K_L02684-1v}\edtext{NACHRICHT}{\lemma{\textnormal{\emph{Nachricht}}}\Cendnote{\textnormal{Entrüstet über Goldmann\pwindex{Goldmann, Paul 31.01.1865 – 25.09.1935@\textsc{Goldmann, Paul} (31.01.1865 – 25.09.1935), \emph{Schriftsteller, Journalist}|pwk}s Berichterstattung über die Dreyfus\pwindex{Dreyfus, Alfred 09.10.1859 – 12.07.1935@\textsc{Dreyfus, Alfred} (09.10.1859 – 12.07.1935), \emph{Militär}|pwk}-Affäre für die \emph{Frankfurter Zeitung}\orgindex{Frankfurter Zeitung@Frankfurter Zeitung|pwk} (\emph{Die Enthüllungen über die Affaire
                           Dreyfus}\pwindex{Goldmann, Paul 31.01.1865 – 25.09.1935@\textsc{Goldmann, Paul} (31.01.1865 – 25.09.1935), \emph{Schriftsteller, Journalist}!Enthuellungen ueber die Affaire Dreyfus1896-09-16@\strich\emph{Die Enthüllungen über die Affaire Dreyfus} {[}1896-09-16{]}|pwk}, Jg. 40, Nr. XXXX, 16. 9. 1896, S. XXXX. \emph{Die Affaire Dreyfus}\pwindex{Goldmann, Paul 31.01.1865 – 25.09.1935@\textsc{Goldmann, Paul} (31.01.1865 – 25.09.1935), \emph{Schriftsteller, Journalist}!Affaire Dreyfus1896-11-11@\strich\emph{Die Affaire Dreyfus} {[}1896-11-11{]}|pwk}, Jg. 40, Nr. XXXX,
                           11. 11. 1896, S. XXXX. \emph{Dreyfus, die öffentliche Meinung und die
                           deutsche Regierung}\pwindex{Goldmann, Paul 31.01.1865 – 25.09.1935@\textsc{Goldmann, Paul} (31.01.1865 – 25.09.1935), \emph{Schriftsteller, Journalist}!Dreyfus, die oeffentliche Meinung und die deutsche Regierung1896-11-12@\strich\emph{Dreyfus, die öffentliche Meinung und die deutsche Regierung} {[}1896-11-12{]}|pwk}, Jg. 40, Nr. XXXX, 12. 11. 1896,
                        S. XXXX) hatte der antisemitische Journalist Lucien Millevoye\pwindex{Millevoye, Lucien 1850-08-01 – 1918-03-25@\textsc{Millevoye, Lucien} (1850-08-01 – 1918-03-25), \emph{Politiker, Journalist}|pwk} ihn »\begin{otherlanguage}{french}lâche coquin\end{otherlanguage}« (ungezogener Feigling) genannt. (\emph{Justice}\pwindex{Millevoye, Lucien 1850-08-01 – 1918-03-25@\textsc{Millevoye, Lucien} (1850-08-01 – 1918-03-25), \emph{Politiker, Journalist}!Justice1896-11-15@\strich\emph{Justice} {[}1896-11-15{]}|pwk}. In: \emph{La Patrie}\pwindex{Patrie. Organe de la defense nationale@\emph{La Patrie. Organe de la défense nationale}|pwk}, 15. 11. 1896.) Daraufhin wurde er von Goldmann\pwindex{Goldmann, Paul 31.01.1865 – 25.09.1935@\textsc{Goldmann, Paul} (31.01.1865 – 25.09.1935), \emph{Schriftsteller, Journalist}|pwk} zum Pistolenduell gefordert. Goldmann\pwindex{Goldmann, Paul 31.01.1865 – 25.09.1935@\textsc{Goldmann, Paul} (31.01.1865 – 25.09.1935), \emph{Schriftsteller, Journalist}|pwk}s Sekundanten\pwindex{Feneon, Felix 1861-06-22 – 1944-02-29@\textsc{Fénéon, Félix} (1861-06-22 – 1944-02-29), \emph{Schriftsteller, Journalist, Kritiker}|pwkv}\pwindex{Strong, Rowland 1865-07-10 – 1924-01-05@\textsc{Strong, Rowland} (1865-07-10 – 1924-01-05), \emph{Journalist}|pwkv} waren die Journalisten Félix Fénéon\pwindex{Feneon, Felix 1861-06-22 – 1944-02-29@\textsc{Fénéon, Félix} (1861-06-22 – 1944-02-29), \emph{Schriftsteller, Journalist, Kritiker}|pwk} und Rowland Strong\pwindex{Strong, Rowland 1865-07-10 – 1924-01-05@\textsc{Strong, Rowland} (1865-07-10 – 1924-01-05), \emph{Journalist}|pwk}. Nach zwei Kugelwechseln mit 25 Schritt
                     Abstand war niemand verletzt. vgl. A. S.: \emph{Tagebuch}, 23. 11. 1896, ungezeichnete Notiz\pwindex{?? Werk@Nicht ermittelte Verfasserinnen und Verfasser!Duell Goldmann – Millevoye]1896-11-22@\emph{[Duell Goldmann – Millevoye]} {[}1896-11-22{]}|pwkv} in: \emph{Le Petit Parisien}\pwindex{Le Petit Parisien. Journal quotidien du soir1876 – 1944@\emph{Le Petit Parisien. Journal quotidien du soir}|pwk}, Jg. 21, Nr. 7.331,
                           22. 11. 1896, S. 2 und \emph{Wiener Zeitung}\pwindex{Wiener Zeitung1703@\emph{Wiener Zeitung}|pwk}, Nr. 272,
                           22. 11. 1896, S. 11.}}}\label{K_L02684-1h} DEIN\pend
           \pstart \spacefill\mbox{ARTHUR +}\pend{}\endnumbering\briefempfaengerindex{Goldmann, Paul@\textsc{Goldmann, Paul}!zzzSchnitzler, Arthur@\emph{von Arthur Schnitzler}!1896-11-211@{21. 11. 1896}|)be}\mylabel{h}\end{ledgroupsized}\begin{anhang}\end{anhang}\newcommand{\dateiname}{L02684}\newcommand{\titel}{Arthur Schnitzler an Paul Goldmann, 21. 11. 1896}\newcommand{\editorInnen}{Martin Anton Müller und Laura Untner}\input{../tex-inputs/latex-pdf-abspann}
      