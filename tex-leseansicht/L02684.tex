%% latex-leseansicht-vorspann.tex
%% Vorspann für die Leseansicht.
%% Lädt die gemeinsame Datei latex-vorspann.tex mit nicht gesetztem Schalter.

\newif\ifkorrekturansicht
\korrekturansichtfalse

\input{../tex-inputs/latex-vorspann}


\section[Arthur Schnitzler an Paul Goldmann, 21. 11. 1896]{L02684 Arthur Schnitzler an Paul Goldmann, 21. 11. 1896}
\nopagebreak\mylabel{L02684v}
\rehead{ }\normalsize\beginnumbering\briefempfaengerindex{Goldmann, Paul@\textsc{Goldmann, Paul}!zzzSchnitzler, Arthur@\emph{von Arthur Schnitzler}!1896-11-211@{21. 11. 1896}|(be}
\toendnotes[C]{\smallbreak\pagebreak[2]}
\correspDesc{Versand  durch Arthur Schnitzler am 21. 11. 1896 in Wien
\newline{}Erhalt  durch Paul Goldmann am 21. 11. 1896 in Paris}\toendnotes[C]{\smallbreak}
\Standort{DLA, A:Schnitzler, HS85.1.5681.}
\physDesc{Telegramm, Fotokopie, 83 Zeichen
\newline{}maschinell
\newline{}Versand: von unbekannter Hand datiert: »21. 11. 96« 
\newline{}Ordnung: mit blauem Kugelschreiber von unbekannter Hand teilweise den
                                 schwer leserlichen Text nachgezogen 
\newline{}Zusatz: Von den Korrespondenzstücken Schnitzlers an Goldmann\pwindex{Goldmann, Paul 31.\,1.\,1865 Breslau – 25.\,9.\,1935 Wien@\textsc{Goldmann, Paul} (31.\,1.\,1865 Breslau – 25.\,9.\,1935 Wien), \emph{Schriftsteller, Journalist}|pw} fehlt weitgehend jede Spur. In der Edition von
                                    Ritterlichkeit\pwindex{Schnitzler, Arthur 15.\,5.\,1862 Wien – 21.\,10.\,1931 ebd.@\textsc{Schnitzler, Arthur} (15.\,5.\,1862 Wien – 21.\,10.\,1931 ebd.), \emph{Schriftsteller, Mediziner}!Ritterlichkeit@\strich\emph{Ritterlichkeit}|pw}
                                    (1975) schreibt die Herausgeberin Rena R. Schlein\pwindex{Schlein, Rena R. *~20.\,6.\,1919 Wien@\textsc{Schlein, Rena R.} (*~20.\,6.\,1919 Wien)|pw}: »Zwei Telegramme
                                    und ein Brief Schnitzlers
                                    an Goldmann\pwindex{Goldmann, Paul 31.\,1.\,1865 Breslau – 25.\,9.\,1935 Wien@\textsc{Goldmann, Paul} (31.\,1.\,1865 Breslau – 25.\,9.\,1935 Wien), \emph{Schriftsteller, Journalist}|pw} wurden mir
                                    von Dr. Leo P. Reckford\pwindex{Reckford, Leo P. 3.\,5.\,1903 Wien – 19.\,10.\,1988 Manhattan@\textsc{Reckford, Leo P.} (3.\,5.\,1903 Wien – 19.\,10.\,1988 Manhattan), \emph{Laryngologe}|pw},
                                    der diese Dokumente von der Familie Goldmanns\pwindex{Goldmann, Paul 31.\,1.\,1865 Breslau – 25.\,9.\,1935 Wien@\textsc{Goldmann, Paul} (31.\,1.\,1865 Breslau – 25.\,9.\,1935 Wien), \emph{Schriftsteller, Journalist}|pw} zum Geschenk bekam, für meine
                                    Arbeit zur Verfügung gestellt« (S. 1). Reckford\pwindex{Reckford, Leo P. 3.\,5.\,1903 Wien – 19.\,10.\,1988 Manhattan@\textsc{Reckford, Leo P.} (3.\,5.\,1903 Wien – 19.\,10.\,1988 Manhattan), \emph{Laryngologe}|pw} starb 1988, seine
                                 Nachkommen haben keine Kenntnis von diesen (und etwaigen weiteren)
                                 Korrespondenzstücken und sie sind auch nicht auffindbar. Rena R. Schlein\pwindex{Schlein, Rena R. *~20.\,6.\,1919 Wien@\textsc{Schlein, Rena R.} (*~20.\,6.\,1919 Wien)|pw} kam
                                    1919 zur Welt. Ein Kontakt konnte nicht hergestellt
                                 werden. Die Kopie des vorliegenden Telegramms dürfte durch Reckford\pwindex{Reckford, Leo P. 3.\,5.\,1903 Wien – 19.\,10.\,1988 Manhattan@\textsc{Reckford, Leo P.} (3.\,5.\,1903 Wien – 19.\,10.\,1988 Manhattan), \emph{Laryngologe}|pw} oder Schlein\pwindex{Schlein, Rena R. *~20.\,6.\,1919 Wien@\textsc{Schlein, Rena R.} (*~20.\,6.\,1919 Wien)|pw} in den Besitz Heinrich Schnitzlers\pwindex{Schnitzler, Heinrich 9.\,8.\,1902 Hinterbrühl – 12.\,7.\,1982 Wien@\textsc{Schnitzler, Heinrich} (9.\,8.\,1902 Hinterbrühl – 12.\,7.\,1982 Wien), \emph{Regisseur, Schauspieler}|pw} gelangt
                                 sein. }
\buchAbdrucke{\weitereDrucke{\pwindex{Schnitzler, Arthur 15.\,5.\,1862 Wien – 21.\,10.\,1931 ebd.@\textsc{Schnitzler, Arthur} (15.\,5.\,1862 Wien – 21.\,10.\,1931 ebd.), \emph{Schriftsteller, Mediziner}!Ritterlichkeit@\strich\emph{Ritterlichkeit}|pwk}Arthur Schnitzler: \emph{Ritterlichkeit. Fragment aus dem Nachlaß}. Bonn: \emph{Bouvier Verlag Herbert Grundmann} 1975, S. 5 (Abhandlungen zur Kunst-, Musik- und
                        Literaturwissenschaft, 176).} }\toendnotes[C]{\smallbreak}\pstart{}{\pb}PAUL GOLDMANN{ }PARIS\oindex{Paris@\textbf{Paris}, \emph{Hauptstadt}|pw}\pend{}\pstart{}24 RUE FEYDEAU\oindex{rue Feydeau@\textbf{rue Feydeau}, \emph{Straße}|pw}\pend{}{\bigskip}\vspace{1em}
\pstart
           \centering{}{\pb}FR WIEN\oindex{Wien@\textbf{Wien}, \emph{Verwaltungsgebiet}|pw} 72\textcolor{gray}{×}\-\textcolor{gray}{×}685\pend
           \vspace{0.5em}
\pstart
           = SENDE MIR SOFORT \label{K_L02684-1v}\edtext{NACHRICHT}{\lemma{\textnormal{\emph{Nachricht}}}\Cendnote{\textnormal{Entrüstet über Goldmanns\pwindex{Goldmann, Paul 31.\,1.\,1865 Breslau – 25.\,9.\,1935 Wien@\textsc{Goldmann, Paul} (31.\,1.\,1865 Breslau – 25.\,9.\,1935 Wien), \emph{Schriftsteller, Journalist}|pwk} Berichterstattung über die Dreyfus\pwindex{Dreyfus, Alfred 9.\,10.\,1859 Mulhouse – 12.\,7.\,1935 Paris@\textsc{Dreyfus, Alfred} (9.\,10.\,1859 Mulhouse – 12.\,7.\,1935 Paris), \emph{Militär}|pwk}-Affäre für die \emph{Frankfurter Zeitung}\orgindex{Frankfurter Zeitung@Frankfurter Zeitung|pwk} (G.\pwindex{Goldmann, Paul 31.\,1.\,1865 Breslau – 25.\,9.\,1935 Wien@\textsc{Goldmann, Paul} (31.\,1.\,1865 Breslau – 25.\,9.\,1935 Wien), \emph{Schriftsteller, Journalist}|pwk} [ = Paul Goldmann\pwindex{Goldmann, Paul 31.\,1.\,1865 Breslau – 25.\,9.\,1935 Wien@\textsc{Goldmann, Paul} (31.\,1.\,1865 Breslau – 25.\,9.\,1935 Wien), \emph{Schriftsteller, Journalist}|pwk}]: \emph{Die
                           Enthüllungen über die Affaire Dreyfus}\pwindex{Goldmann, Paul 31.\,1.\,1865 Breslau – 25.\,9.\,1935 Wien@\textsc{Goldmann, Paul} (31.\,1.\,1865 Breslau – 25.\,9.\,1935 Wien), \emph{Schriftsteller, Journalist}!Enthüllungen über die Affaire Dreyfus@\strich\emph{Die Enthüllungen über die Affaire Dreyfus}|pwk}, Jg. 41, Nr. 258,
                           16. 9. 1896, Erstes Morgenblatt, S. 1; G.\pwindex{Goldmann, Paul 31.\,1.\,1865 Breslau – 25.\,9.\,1935 Wien@\textsc{Goldmann, Paul} (31.\,1.\,1865 Breslau – 25.\,9.\,1935 Wien), \emph{Schriftsteller, Journalist}|pwk} [ = Paul Goldmann\pwindex{Goldmann, Paul 31.\,1.\,1865 Breslau – 25.\,9.\,1935 Wien@\textsc{Goldmann, Paul} (31.\,1.\,1865 Breslau – 25.\,9.\,1935 Wien), \emph{Schriftsteller, Journalist}|pwk}]: \emph{Die
                           Affaire Dreyfus}\pwindex{Goldmann, Paul 31.\,1.\,1865 Breslau – 25.\,9.\,1935 Wien@\textsc{Goldmann, Paul} (31.\,1.\,1865 Breslau – 25.\,9.\,1935 Wien), \emph{Schriftsteller, Journalist}!Affaire Dreyfus@\strich\emph{Die Affaire Dreyfus}|pwk}, Jg. 41, Nr. 314, 11. 11. 1896, Zweites
                        Morgenblatt, S. 1; G.\pwindex{Goldmann, Paul 31.\,1.\,1865 Breslau – 25.\,9.\,1935 Wien@\textsc{Goldmann, Paul} (31.\,1.\,1865 Breslau – 25.\,9.\,1935 Wien), \emph{Schriftsteller, Journalist}|pwk} [ = Paul Goldmann\pwindex{Goldmann, Paul 31.\,1.\,1865 Breslau – 25.\,9.\,1935 Wien@\textsc{Goldmann, Paul} (31.\,1.\,1865 Breslau – 25.\,9.\,1935 Wien), \emph{Schriftsteller, Journalist}|pwk}]: \emph{Dreyfus, die öffentliche Meinung und die deutsche Regierung}\pwindex{Goldmann, Paul 31.\,1.\,1865 Breslau – 25.\,9.\,1935 Wien@\textsc{Goldmann, Paul} (31.\,1.\,1865 Breslau – 25.\,9.\,1935 Wien), \emph{Schriftsteller, Journalist}!Dreyfus, die öffentliche Meinung und die deutsche Regierung@\strich\emph{Dreyfus, die öffentliche Meinung und die deutsche Regierung}|pwk},
                        Jg. 41, Nr. 315, 12. 11. 1896, Erstes Morgenblatt,
                     S. 1.), in der für die Wiederaufnahme des Prozesses gegen Dreyfus\pwindex{Dreyfus, Alfred 9.\,10.\,1859 Mulhouse – 12.\,7.\,1935 Paris@\textsc{Dreyfus, Alfred} (9.\,10.\,1859 Mulhouse – 12.\,7.\,1935 Paris), \emph{Militär}|pwk} Partei ergriffen wurde, hatte der
                     antisemitische Chefredakteur Lucien
                        Millevoye\pwindex{Millevoye, Lucien 1.\,8.\,1850 Grenoble – 25.\,3.\,1918 Paris@\textsc{Millevoye, Lucien} (1.\,8.\,1850 Grenoble – 25.\,3.\,1918 Paris), \emph{Politiker, Journalist}|pwk} über ihn geschrieben (\emph{Justice!}\pwindex{Millevoye, Lucien 1.\,8.\,1850 Grenoble – 25.\,3.\,1918 Paris@\textsc{Millevoye, Lucien} (1.\,8.\,1850 Grenoble – 25.\,3.\,1918 Paris), \emph{Politiker, Journalist}!Justice@\strich\emph{Justice{\rufezeichen}}|pwk} In: \emph{La Patrie}\pwindex{Patrie. Organe de la défense nationale@\emph{La Patrie. Organe de la défense nationale}|pwk}, Jg. 56, 15. 11. 1896,
                        S. 1.): »\begin{otherlanguage}{french}Le lâche coquin se croit à l’arbi.\end{otherlanguage}« – Der ungezogene Feigling glaubt sich in Sicherheit. Daraufhin wurde
                     er von Goldmann\pwindex{Goldmann, Paul 31.\,1.\,1865 Breslau – 25.\,9.\,1935 Wien@\textsc{Goldmann, Paul} (31.\,1.\,1865 Breslau – 25.\,9.\,1935 Wien), \emph{Schriftsteller, Journalist}|pwk} zum Pistolenduell
                     gefordert. Goldmanns\pwindex{Goldmann, Paul 31.\,1.\,1865 Breslau – 25.\,9.\,1935 Wien@\textsc{Goldmann, Paul} (31.\,1.\,1865 Breslau – 25.\,9.\,1935 Wien), \emph{Schriftsteller, Journalist}|pwk}{ }Sekundanten\pwindex{Fénéon, Félix 22.\,6.\,1861 Turin – 29.\,2.\,1944 Châtenay-Malabry@\textsc{Fénéon, Félix} (22.\,6.\,1861 Turin – 29.\,2.\,1944 Châtenay-Malabry), \emph{Schriftsteller, Journalist, Kunstkritiker}|pwkv}\pwindex{Strong, Rowland 10.\,7.\,1865 London – 5.\,1.\,1924 Paris@\textsc{Strong, Rowland} (10.\,7.\,1865 London – 5.\,1.\,1924 Paris), \emph{Journalist}|pwkv} waren die
                     Journalisten Félix Fénéon\pwindex{Fénéon, Félix 22.\,6.\,1861 Turin – 29.\,2.\,1944 Châtenay-Malabry@\textsc{Fénéon, Félix} (22.\,6.\,1861 Turin – 29.\,2.\,1944 Châtenay-Malabry), \emph{Schriftsteller, Journalist, Kunstkritiker}|pwk} und Rowland Strong\pwindex{Strong, Rowland 10.\,7.\,1865 London – 5.\,1.\,1924 Paris@\textsc{Strong, Rowland} (10.\,7.\,1865 London – 5.\,1.\,1924 Paris), \emph{Journalist}|pwk}. Nach zwei Kugelwechseln
                     mit 25 Schritt Abstand war niemand verletzt. Vgl. A. S.: \emph{Tagebuch}, 23. 11. 1896, ungezeichnete Notiz\pwindex{Duell Goldmann – Millevoye]@\emph{[Duell Goldmann – Millevoye]}|pwkv} in: \emph{Le Petit Parisien}\pwindex{Le Petit Parisien.  Journal quotidien du soir@\emph{Le Petit Parisien. Journal quotidien du soir}|pwk}, Jg. 21, Nr. 7331,
                           22. 11. 1896, S. 2 und \emph{Wiener Zeitung}\pwindex{Wiener Zeitung@\emph{Wiener Zeitung}|pwk}, Nr. 272,
                           22. 11. 1896, S. 11.}}}\label{K_L02684-1} DEIN{ }\spacefill\mbox{ARTHUR +}\pend
           \selectlanguage{ngerman}\endnumbering\briefempfaengerindex{Goldmann, Paul@\textsc{Goldmann, Paul}!zzzSchnitzler, Arthur@\emph{von Arthur Schnitzler}!1896-11-211@{21. 11. 1896}|)be}\mylabel{L02684h}  \newcommand{\dateiname}{L02684}\newcommand{\titel}{Arthur Schnitzler an Paul Goldmann, 21. 11. 1896}\newcommand{\editorInnen}{Martin Anton Müller und Laura Untner}%% latex-leseansicht-abspann.tex
%% Abspann für die Leseansicht.
%% Der Schalter \ifkorrekturansicht ist bereits durch den Vorspann gesetzt.

%% latex-abspann.tex
%% Gemeinsamer Abspann für Korrekturansicht und Leseansicht.
%% Setzt den Schalter \ifkorrekturansicht voraus (gesetzt in den
%% einbindenden Dateien latex-korrekturansicht-abspann.tex bzw.
%% latex-leseansicht-abspann.tex).
%% ---------------------------------------------------------------

\normalsize

% Das esempio-Environment wird nur in der Leseansicht benötigt
\ifkorrekturansicht\else
\newenvironment{esempio}[3]%
{
    \vspace{1.5ex}
    \rlap{\underline{#1}}
    \par
    \setlength{\parindent}{0cm}
    \nopagebreak
    \leftskip=#2cm
    \rightskip=#3cm
}
{
    \par
}
\fi

\doendnotes{C}
\bigskip
\vfill

\clearpage

\footnotesize

\ifkorrekturansicht
  \lohead{\textsc{register}}
\fi

% theindex-Environment neu definieren ohne reledmac
\makeatletter
\renewenvironment{theindex}{%
  \ifkorrekturansicht
    \section*{\indexname}%
  \else
    \subsubsection*{Index der erwähnten Entitäten}%
  \fi
  \setlength{\parindent}{0pt}%
  \setlength{\parskip}{0pt plus 0.3pt}%
  \let\item\@idxitem
}{%
  \ifkorrekturansicht\clearpage\fi
}
\makeatother

\IfFileExists{\jobname-pw.ind}{\input{\jobname-pw.ind}}{}

% Quellenangabe nur in der Leseansicht
\ifkorrekturansicht\else
% Fallback-Definitionen, falls die .tex-Datei \titel etc. nicht gesetzt hat
\providecommand{\titel}{}
\providecommand{\editorInnen}{}
\providecommand{\dateiname}{\jobname}

\vspace{3cm}

\vfill

\footnotesize
\textsc{Quelle}: \titel. Herausgegeben von {\editorInnen}. In: \emph{Arthur Schnitzler: Briefwechsel mit Autorinnen und Autoren}.
 Digitale Edition, https://schnitzler-briefe.acdh.oeaw.ac.at/{\dateiname}.html (Stand \today)
\fi

\end{document}


