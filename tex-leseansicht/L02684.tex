%% latex-korrekturansicht-vorspann.tex
%% Vorspann für die Korrekturansicht.
%% Lädt die gemeinsame Datei latex-vorspann.tex mit gesetztem Schalter.

\newif\ifkorrekturansicht
\korrekturansichttrue

\input{../tex-inputs/latex-vorspann}


\section[Arthur Schnitzler an Paul Goldmann, 21. 11. 1896]{L02684 Arthur Schnitzler an Paul Goldmann, 21. 11. 1896}
\nopagebreak\mylabel{L02684v}
\rehead{ }\normalsize\beginnumbering\briefempfaengerindex{Goldmann, Paul@\textsc{Goldmann, Paul}!zzzSchnitzler, Arthur@\emph{von Arthur Schnitzler}!1896-11-211@{21. 11. 1896}|(be}
\toendnotes[C]{\smallbreak\pagebreak[2]}\Standort{DLA, A:Schnitzler, HS85.1.5681.}
\physDesc{Telegramm, Fotokopie83 Zeichen
\newline{}maschinell
\newline{}Versand: von unbekannter Hand datiert: »21. 11. 96« 
\newline{}Ordnung: mit blauem Kugelschreiber von unbekannter Hand teilweise den
                                 schwer leserlichen Text nachgezogen 
\newline{}Zusatz: Von den Korrespondenzstücken Schnitzlers an Goldmann\pwindex{Goldmann, Paul 31.01.1865 – 25.09.1935@\textsc{Goldmann, Paul} (31.01.1865 – 25.09.1935), \emph{Schriftsteller/Schriftstellerin, Journalist/Journalistin}|pw} fehlt weitgehend jede Spur. In der Edition von
                                    Ritterlichkeit\pwindex{Ritterlichkeit@\emph{Ritterlichkeit}|pw}
                                    (1975) schreibt die Herausgeberin Rena R. Schlein\pwindex{Schlein, Rena R. *~1919-06-20@\textsc{Schlein, Rena R.} (*~1919-06-20)|pw}: »Zwei Telegramme
                                    und ein Brief Schnitzlers
                                    an Goldmann\pwindex{Goldmann, Paul 31.01.1865 – 25.09.1935@\textsc{Goldmann, Paul} (31.01.1865 – 25.09.1935), \emph{Schriftsteller/Schriftstellerin, Journalist/Journalistin}|pw} wurden mir
                                    von Dr. Leo P. Reckford\pwindex{Reckford, Leo P. 1903-05-03 – 1988-10-19@\textsc{Reckford, Leo P.} (1903-05-03 – 1988-10-19), \emph{Laryngologe/Laryngologin}|pw},
                                    der diese Dokumente von der Familie Goldmanns\pwindex{Goldmann, Paul 31.01.1865 – 25.09.1935@\textsc{Goldmann, Paul} (31.01.1865 – 25.09.1935), \emph{Schriftsteller/Schriftstellerin, Journalist/Journalistin}|pw} zum Geschenk bekam, für meine
                                    Arbeit zur Verfügung gestellt« (S. 1). Reckford\pwindex{Reckford, Leo P. 1903-05-03 – 1988-10-19@\textsc{Reckford, Leo P.} (1903-05-03 – 1988-10-19), \emph{Laryngologe/Laryngologin}|pw} starb 1988, seine
                                 Nachkommen haben keine Kenntnis von diesen (und etwaigen weiteren)
                                 Korrespondenzstücken und sie sind auch nicht auffindbar. Rena R. Schlein\pwindex{Schlein, Rena R. *~1919-06-20@\textsc{Schlein, Rena R.} (*~1919-06-20)|pw} kam
                                    1919 zur Welt. Ein Kontakt konnte nicht hergestellt
                                 werden. Die Kopie des vorliegenden Telegramms dürfte durch Reckford\pwindex{Reckford, Leo P. 1903-05-03 – 1988-10-19@\textsc{Reckford, Leo P.} (1903-05-03 – 1988-10-19), \emph{Laryngologe/Laryngologin}|pw} oder Schlein\pwindex{Schlein, Rena R. *~1919-06-20@\textsc{Schlein, Rena R.} (*~1919-06-20)|pw} in den Besitz Heinrich Schnitzlers\pwindex{Schnitzler, Heinrich 09.08.1902 – 12.07.1982@\textsc{Schnitzler, Heinrich} (09.08.1902 – 12.07.1982), \emph{Regisseur/Regisseurin, Schauspieler/Schauspielerin}|pw} gelangt
                                 sein. }
\buchAbdrucke{\weitereDrucke{\pwindex{Ritterlichkeit@\emph{Ritterlichkeit}|pwk}Arthur Schnitzler: \emph{Ritterlichkeit. Fragment aus dem Nachlaß}. Bonn: \emph{Bouvier Verlag Herbert Grundmann} 1975, S. 5.} }\toendnotes[C]{\smallbreak}\pstart{}{\pb}PAUL GOLDMANN{ }PARIS\oindex{Paris@\textbf{Paris}, \emph{P.PPLC}|pw}\pend{}\pstart{}24 RUE FEYDEAU\oindex{rue Feydeau@\textbf{rue Feydeau}, \emph{Straße (K.STR)}|pw}\pend{}{\bigskip}\vspace{1em}
\pstart
           \centering{}{\pb}FR WIEN\oindex{Wien@\textbf{Wien}, \emph{A.ADM2}|pw} 72\textcolor{gray}{×}\-\textcolor{gray}{×}685\pend
           \vspace{0.5em}
\pstart
           = SENDE MIR SOFORT \label{K_L02684-1v}\edtext{NACHRICHT}{\lemma{\textnormal{\emph{Nachricht}}}\Cendnote{\textnormal{Entrüstet über Goldmanns\pwindex{Goldmann, Paul 31.01.1865 – 25.09.1935@\textsc{Goldmann, Paul} (31.01.1865 – 25.09.1935), \emph{Schriftsteller/Schriftstellerin, Journalist/Journalistin}|pwk} Berichterstattung über die Dreyfus\pwindex{Dreyfus, Alfred 1859-10-09 – 1935-07-12@\textsc{Dreyfus, Alfred} (1859-10-09 – 1935-07-12), \emph{Militär/Militärin}|pwk}-Affäre für die \emph{Frankfurter Zeitung}\orgindex{Frankfurter Zeitung@Frankfurter Zeitung|pwk} (G.\pwindex{Goldmann, Paul 31.01.1865 – 25.09.1935@\textsc{Goldmann, Paul} (31.01.1865 – 25.09.1935), \emph{Schriftsteller/Schriftstellerin, Journalist/Journalistin}|pwk} [ = Paul Goldmann\pwindex{Goldmann, Paul 31.01.1865 – 25.09.1935@\textsc{Goldmann, Paul} (31.01.1865 – 25.09.1935), \emph{Schriftsteller/Schriftstellerin, Journalist/Journalistin}|pwk}]: \emph{Die
                           Enthüllungen über die Affaire Dreyfus}\pwindex{Enthuellungen ueber die Affaire Dreyfus@\emph{Die Enthüllungen über die Affaire Dreyfus}|pwk}, Jg. 41, Nr. 258,
                           16. 9. 1896, Erstes Morgenblatt, S. 1; G.\pwindex{Goldmann, Paul 31.01.1865 – 25.09.1935@\textsc{Goldmann, Paul} (31.01.1865 – 25.09.1935), \emph{Schriftsteller/Schriftstellerin, Journalist/Journalistin}|pwk} [ = Paul Goldmann\pwindex{Goldmann, Paul 31.01.1865 – 25.09.1935@\textsc{Goldmann, Paul} (31.01.1865 – 25.09.1935), \emph{Schriftsteller/Schriftstellerin, Journalist/Journalistin}|pwk}]: \emph{Die
                           Affaire Dreyfus}\pwindex{Affaire Dreyfus@\emph{Die Affaire Dreyfus}|pwk}, Jg. 41, Nr. 314, 11. 11. 1896, Zweites
                        Morgenblatt, S. 1; G.\pwindex{Goldmann, Paul 31.01.1865 – 25.09.1935@\textsc{Goldmann, Paul} (31.01.1865 – 25.09.1935), \emph{Schriftsteller/Schriftstellerin, Journalist/Journalistin}|pwk} [ = Paul Goldmann\pwindex{Goldmann, Paul 31.01.1865 – 25.09.1935@\textsc{Goldmann, Paul} (31.01.1865 – 25.09.1935), \emph{Schriftsteller/Schriftstellerin, Journalist/Journalistin}|pwk}]: \emph{Dreyfus, die öffentliche Meinung und die deutsche Regierung}\pwindex{Dreyfus, die oeffentliche Meinung und die deutsche Regierung@\emph{Dreyfus, die öffentliche Meinung und die deutsche Regierung}|pwk},
                        Jg. 41, Nr. 315, 12. 11. 1896, Erstes Morgenblatt,
                     S. 1.), in der für die Wiederaufnahme des Prozesses gegen Dreyfus\pwindex{Dreyfus, Alfred 1859-10-09 – 1935-07-12@\textsc{Dreyfus, Alfred} (1859-10-09 – 1935-07-12), \emph{Militär/Militärin}|pwk} Partei ergriffen wurde, hatte der
                     antisemitische Chefredakteur Lucien
                        Millevoye\pwindex{Millevoye, Lucien 1850-08-01 – 1918-03-25@\textsc{Millevoye, Lucien} (1850-08-01 – 1918-03-25), \emph{Politiker/Politikerin, Journalist/Journalistin}|pwk} über ihn geschrieben (\emph{Justice!}\pwindex{Justice@\emph{Justice{\rufezeichen}}|pwk} In: \emph{La Patrie}\pwindex{Patrie. Organe de la defense nationale@\emph{La Patrie. Organe de la défense nationale}|pwk}, Jg. 56, 15. 11. 1896,
                        S. 1.): »\begin{otherlanguage}{french}Le lâche coquin se croit à l’arbi.\end{otherlanguage}« – Der ungezogene Feigling glaubt sich in Sicherheit. Daraufhin wurde
                     er von Goldmann\pwindex{Goldmann, Paul 31.01.1865 – 25.09.1935@\textsc{Goldmann, Paul} (31.01.1865 – 25.09.1935), \emph{Schriftsteller/Schriftstellerin, Journalist/Journalistin}|pwk} zum Pistolenduell
                     gefordert. Goldmanns\pwindex{Goldmann, Paul 31.01.1865 – 25.09.1935@\textsc{Goldmann, Paul} (31.01.1865 – 25.09.1935), \emph{Schriftsteller/Schriftstellerin, Journalist/Journalistin}|pwk}{ }Sekundanten\pwindex{Feneon, Felix 1861-06-22 – 1944-02-29@\textsc{Fénéon, Félix} (1861-06-22 – 1944-02-29), \emph{Schriftsteller/Schriftstellerin, Journalist/Journalistin, Kunstkritiker/Kunstkritikerin}|pwkv}\pwindex{Strong, Rowland 1865-07-10 – 1924-01-05@\textsc{Strong, Rowland} (1865-07-10 – 1924-01-05), \emph{Journalist/Journalistin}|pwkv} waren die
                     Journalisten Félix Fénéon\pwindex{Feneon, Felix 1861-06-22 – 1944-02-29@\textsc{Fénéon, Félix} (1861-06-22 – 1944-02-29), \emph{Schriftsteller/Schriftstellerin, Journalist/Journalistin, Kunstkritiker/Kunstkritikerin}|pwk} und Rowland Strong\pwindex{Strong, Rowland 1865-07-10 – 1924-01-05@\textsc{Strong, Rowland} (1865-07-10 – 1924-01-05), \emph{Journalist/Journalistin}|pwk}. Nach zwei Kugelwechseln
                     mit 25 Schritt Abstand war niemand verletzt. Vgl. A. S.: \emph{Tagebuch}, 23. 11. 1896, ungezeichnete Notiz\pwindex{Duell Goldmann – Millevoye]@\emph{[Duell Goldmann – Millevoye]}|pwkv} in: \emph{Le Petit Parisien}\pwindex{Le Petit Parisien.  Journal quotidien du soir@\emph{Le Petit Parisien. Journal quotidien du soir}|pwk}, Jg. 21, Nr. 7331,
                           22. 11. 1896, S. 2 und \emph{Wiener Zeitung}\pwindex{Wiener Zeitung@\emph{Wiener Zeitung}|pwk}, Nr. 272,
                           22. 11. 1896, S. 11.}}}\label{K_L02684-1} DEIN{ }\spacefill\mbox{ARTHUR +}\pend
           \selectlanguage{ngerman}\endnumbering\briefempfaengerindex{Goldmann, Paul@\textsc{Goldmann, Paul}!zzzSchnitzler, Arthur@\emph{von Arthur Schnitzler}!1896-11-211@{21. 11. 1896}|)be}\mylabel{L02684h}  \normalsize

\doendnotes{C}
\bigskip
\vfill

\clearpage

\footnotesize

\lohead{\textsc{register}}

% Definiere theindex-Environment komplett neu ohne reledmac
\makeatletter
\renewenvironment{theindex}{%
  \section*{\indexname}%
  \setlength{\parindent}{0pt}%
  \setlength{\parskip}{0pt plus 0.3pt}%
  \let\item\@idxitem
}{%
  \clearpage
}
\makeatother

\IfFileExists{\jobname-pw.ind}{\input{\jobname-pw.ind}}{}

\end{document}

      