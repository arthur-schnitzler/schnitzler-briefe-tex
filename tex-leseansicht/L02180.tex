%% latex-leseansicht-vorspann.tex
%% Vorspann für die Leseansicht.
%% Lädt die gemeinsame Datei latex-vorspann.tex mit nicht gesetztem Schalter.

\newif\ifkorrekturansicht
\korrekturansichtfalse

\input{../tex-inputs/latex-vorspann}


         
         \renewcommand{\erwaehntePersonen}{Personen: Richard Beer-Hofmann, Paula Beer-Hofmann,  Rembrandt van Rijn}
         \renewcommand{\erwaehnteInstitutionen}{Institutionen: Königliches Museum der Schönen Künste}
         \renewcommand{\erwaehnteOrte}{Orte: Algiers, Antwerpen, Hasenauerstraße, Wien, XVIII., Währing, Österreich}
         \renewcommand{\erwaehnteWerke}{Werke: Alter Mann}
               \section[Arthur und Olga Schnitzler an Richard und Paula Beer-Hofmann, 21. 5. 1914]{ Arthur und Olga Schnitzler an Richard und Paula Beer-Hofmann,
               21. 5. 1914}\nopagebreak\mylabel{v}\rehead{ }\begin{ledgroupsized}[t]{13cm}\normalsize\beginnumbering \toendnotes[C]{\smallbreak\pagebreak[2]} \Standort{YCGL, MSS 31.}
\physDesc{Bildpostkarte
\newline{}Handschrift Arthur Schnitzler: Bleistift, deutsche Kurrent\newline{}Handschrift Olga Schnitzler: Bleistift, lateinische Kurrent\newline{}Versand: Stempel: »\nobreak{}\oindex{Antwerpen@\textbf{Antwerpen}|pwk}\textcolor{gray}{Antw}erpe\textcolor{gray}{n}
                                       Anvers, 21. 5. 1914, 1\textcolor{gray}{0} 20\nobreak{}«.  }\pstart{}{\pb}Herrn u Frau \textsc{Dr.
                     Rich.}\pend{}\pstart{}\textsc{Beer}\substVorne{}\textsuperscript{h}\substDazwischen{}\textsc{H}\substHinten{}\textsc{ofmann}\pend{}\pstart{}Wien XVIII\oindex{XVIII., Waehring@\textbf{XVIII., Währing}|pw}\pend{}\pstart{}\textsc{Hasenauerstr. 59}\oindex{Hasenauerstrasse@\textbf{Hasenauerstraße}|pw}. \pend{}\pstart{}\textsc{Austria}\oindex{Oesterreich@\textbf{Österreich}|pw}\pend{}{\bigskip}\pstart
           \noindent{}\centering{}{\pb}\textcolor{gray}{\textbf{REMBRANDT\pwindex{Rembrandt van Rijn 15.07.1606 – 04.10.1669@\textsc{Rembrandt van Rijn} (15.07.1606 – 04.10.1669), \emph{Maler}|pw}. – Portrait d’un Vieux Juif\pwindex{\textcolor{red}{\textsuperscript{XXXX1 indx}}!Alter Mann1650 – 1680@\strich\emph{Alter Mann} {[}1650 – 1680{]}|pw}.}}\pend
           \pstart
           \noindent{}\centering{}\textcolor{gray}{\textbf{MUSÉE ROYAL D’ANVERS\orgindex{Koenigliches Museum der Schoenen Kuenste@Königliches Museum der Schönen Künste|pw}}}\pend
           \pstart
           {\pb}Herzliche Grüße!\pend
           \pstart
           \textsc{Antwerpen}\oindex{Antwerpen@\textbf{Antwerpen}|pw}{ }2\substVorne{}\textsuperscript{0}\substDazwischen{}1\substHinten{}/5 914\pend
           \pstart Ihr \spacefill\mbox{Arthur}\pend{}\pstart
           \noindent{}{[}hs. Olga Schnitzler:{]} Es ist hier viel wärmer wie in Algier\oindex{Algiers@\textbf{Algiers}|pw}. Wir hatten immer schönes Wetter, immer gute Fahrt.\pend
           \pstart
           Herzlichst{\\[\baselineskip]}Ihre \spacefill\mbox{O.}\pend
           \leftskip=0em{}
         
         \endnumbering\mylabel{h}\end{ledgroupsized}  \newcommand{\dateiname}{L02180}\newcommand{\titel}{Arthur und Olga Schnitzler an Richard und Paula Beer-Hofmann, 21. 5. 1914}\newcommand{\editorInnen}{Martin Anton Müller und Gerd-Hermann Susen}%% latex-leseansicht-abspann.tex
%% Abspann für die Leseansicht.
%% Der Schalter \ifkorrekturansicht ist bereits durch den Vorspann gesetzt.

%% latex-abspann.tex
%% Gemeinsamer Abspann für Korrekturansicht und Leseansicht.
%% Setzt den Schalter \ifkorrekturansicht voraus (gesetzt in den
%% einbindenden Dateien latex-korrekturansicht-abspann.tex bzw.
%% latex-leseansicht-abspann.tex).
%% ---------------------------------------------------------------

\normalsize

% Das esempio-Environment wird nur in der Leseansicht benötigt
\ifkorrekturansicht\else
\newenvironment{esempio}[3]%
{
    \vspace{1.5ex}
    \rlap{\underline{#1}}
    \par
    \setlength{\parindent}{0cm}
    \nopagebreak
    \leftskip=#2cm
    \rightskip=#3cm
}
{
    \par
}
\fi

\doendnotes{C}
\bigskip
\vfill

\clearpage

\footnotesize

\ifkorrekturansicht
  \lohead{\textsc{register}}
\fi

% theindex-Environment neu definieren ohne reledmac
\makeatletter
\renewenvironment{theindex}{%
  \ifkorrekturansicht
    \section*{\indexname}%
  \else
    \subsubsection*{Index der erwähnten Entitäten}%
  \fi
  \setlength{\parindent}{0pt}%
  \setlength{\parskip}{0pt plus 0.3pt}%
  \let\item\@idxitem
}{%
  \ifkorrekturansicht\clearpage\fi
}
\makeatother

\IfFileExists{\jobname-pw.ind}{\input{\jobname-pw.ind}}{}

% Quellenangabe nur in der Leseansicht
\ifkorrekturansicht\else
% Fallback-Definitionen, falls die .tex-Datei \titel etc. nicht gesetzt hat
\providecommand{\titel}{}
\providecommand{\editorInnen}{}
\providecommand{\dateiname}{\jobname}

\vspace{3cm}

\vfill

\footnotesize
\textsc{Quelle}: \titel. Herausgegeben von {\editorInnen}. In: \emph{Arthur Schnitzler: Briefwechsel mit Autorinnen und Autoren}.
 Digitale Edition, https://schnitzler-briefe.acdh.oeaw.ac.at/{\dateiname}.html (Stand \today)
\fi

\end{document}


      