%% latex-korrekturansicht-vorspann.tex
%% Vorspann für die Korrekturansicht.
%% Lädt die gemeinsame Datei latex-vorspann.tex mit gesetztem Schalter.

\newif\ifkorrekturansicht
\korrekturansichttrue

\input{../tex-inputs/latex-vorspann}


\section[ Felix Salten an Arthur Schnitzler, 29. 1. 1910]{L03544 Felix Salten an Arthur Schnitzler, 29. 1. 1910}
\nopagebreak\mylabel{L03544v}
\rehead{ }\normalsize\beginnumbering\briefempfaengerindex{Schnitzler, Arthur@\textsc{Schnitzler, Arthur}!zzzSalten, Felix@\emph{von Felix Salten}!1910-01-291@{29. 1. 1910}|(be}
\toendnotes[C]{\smallbreak\pagebreak[2]}\Standort{CUL, Schnitzler, B 89, B 2.}
\physDesc{Postkarte, 473 Zeichen
\newline{}Handschrift: schwarze Tinte, lateinische Kurrent
\newline{}Versand: Stempel: »\nobreak{}\oindex{XVIII., Waehring@\textbf{XVIII., Währing}, \emph{A.ADM3}|pwk}18/\textsubscript{1} Wien 111, 29. I. \textcolor{gray}{10}, 4\nobreak{}«.  
\newline{}Ordnung: mit Bleistift von unbekannter Hand nummeriert: »259« und
                                    »2« }\toendnotes[C]{\smallbreak}\pstart{}{\pb}\textcolor{gray}{\textbf{\textit{FELIX SALTEN}}}\pend{}\pstart{}\textcolor{gray}{\textbf{\textit{WIEN, XVIII.}}}\oindex{XVIII., Waehring@\textbf{XVIII., Währing}, \emph{A.ADM3}|pw}\pend{}\pstart{}\textcolor{gray}{\textbf{\textit{COTTAGEGASSE 37}}}\oindex{Cottagegasse@\textbf{Cottagegasse}, \emph{Straße (K.STR)}|pw}\pend{}{\bigskip}\pstart{}Herrn D\textsuperscript{r} Arthur Schnitzler\pend{}\pstart{}Wien\oindex{Wien@\textbf{Wien}, \emph{A.ADM2}|pw}\pend{}\pstart{}\label{T_L03544-1v}\edtext{X\substVorne{}\textsuperscript{IX}\substDazwischen{}VI\substHinten{}II}{\lemma{\textnormal{\emph{XVIII}}}\Cendnote{\textnormal{Zur Verdeutlichung wurde  von Salten\pwindex{Salten, Felix 06.09.1869 – 08.10.1945@\textsc{Salten, Felix} (06.09.1869 – 08.10.1945), \emph{Schriftsteller/Schriftstellerin, Journalist/Journalistin, Chefredakteur/Chefredakteurin}|pwk} »XVIII« seitlich wiederholt.}}}\label{T_L03544-1}. Spöttelgaße 7\oindex{Edmund-Weiss-Gasse 7@\textbf{Edmund-Weiß-Gasse 7}, \emph{Wohngebäude (K.WHS)}|pw}\pend{}{\bigskip}\vspace{1em}
\pstart{}{\pb}Lieber,\pend\vspace{0.5em}
\pstart
           mein Schwager Ludwig\pwindex{Metzl, Ludwig *~1854-03-09@\textsc{Metzl, Ludwig} (*~1854-03-09)|pw} ist unverhofft aus Berlin\oindex{Berlin@\textbf{Berlin}, \emph{P.PPLC}|pw} angekommen und legt mich heute, wie auch morgen,
                  Sonntag, in Beschlag. Ich kann also leider nicht mit Ihnen spazieren gehen.
               Nächster Tage \label{K_L03544-1v}\edtext{Vormittag komme ich einmal zu Ihnen}{\lemma{\textnormal{\emph{Vormittag … Ihnen}}}\Cendnote{\textnormal{Am Dienstag, dem 1. 2. 1910 besuchte
                  Schnitzler{ }Salten\pwindex{Salten, Felix 06.09.1869 – 08.10.1945@\textsc{Salten, Felix} (06.09.1869 – 08.10.1945), \emph{Schriftsteller/Schriftstellerin, Journalist/Journalistin, Chefredakteur/Chefredakteurin}|pwk}. Am
                  2. 2. 1910 fand der Spaziergang statt. 
               }}}\label{K_L03544-1}. Muss Ihnen übrigens auch vom \label{K_L03544-2v}\edtext{Baron B.\pwindex{Berger, Alfred von 30.04.1853 – 24.08.1912@\textsc{Berger, Alfred von} (30.04.1853 – 24.08.1912), \emph{Schriftsteller/Schriftstellerin, Journalist/Journalistin, Theaterleiter/Theaterleiterin}|pw}}{\lemma{\textnormal{\emph{Baron B.}}}\Cendnote{\textnormal{Alfred von Berger\pwindex{Berger, Alfred von 30.04.1853 – 24.08.1912@\textsc{Berger, Alfred von} (30.04.1853 – 24.08.1912), \emph{Schriftsteller/Schriftstellerin, Journalist/Journalistin, Theaterleiter/Theaterleiterin}|pwk}, der neue Direktor des
                     \emph{Burgtheaters}\orgindex{Burgtheater@Burgtheater|pwk}}}}\label{K_L03544-2} erzählen. Er will den
                  \label{K_L03544-3v}\edtext{Medardus\pwindex{junge Medardus. Dramatische Historie in einem Vorspiel und fuenf Aufzuegen@\emph{Der junge Medardus. Dramatische Historie in einem Vorspiel und fünf Aufzügen}|pw}{ }\uline{mit} der Bastei}{\lemma{\textnormal{\emph{Medardus mit der Bastei}}}\Cendnote{\textnormal{Das Stück war
                   durch seinen Textumfang nur mit Kürzungen aufzuführen (vgl. Arthur Schnitzler an Hermann Bahr, 17. 11. 1910). Die 
                  auf dem Festungswall (Bastei) angesiedelten Szenen waren durch die vielen benötigten Statisten besonders 
                  aufwendig zu inszenieren. Vgl. A. S.: \emph{Tagebuch}, 1. 2. 1910.
               }}}\label{K_L03544-3} spielen. Auf Montag oder Dienstag also!\pend
           
\pstart
           Alles Herzliche von uns\pwindex{Salten, Ottilie 07.03.1868 – 22.06.1942@\textsc{Salten, Ottilie} (07.03.1868 – 22.06.1942), \emph{Schauspieler/Schauspielerin}|pwv}
               zu Ihnen{\\[\baselineskip]} Ihr{\\[\baselineskip]}\spacefill\mbox{Salten}\pend
           \leftskip=0em{}
\pstart
           28. I. 10\pend
           \selectlanguage{ngerman}\endnumbering\briefempfaengerindex{Schnitzler, Arthur@\textsc{Schnitzler, Arthur}!zzzSalten, Felix@\emph{von Felix Salten}!1910-01-291@{29. 1. 1910}|)be}\mylabel{L03544h}  \normalsize

\doendnotes{C}
\bigskip
\vfill

\clearpage

\footnotesize

\lohead{\textsc{register}}

% Definiere theindex-Environment komplett neu ohne reledmac
\makeatletter
\renewenvironment{theindex}{%
  \section*{\indexname}%
  \setlength{\parindent}{0pt}%
  \setlength{\parskip}{0pt plus 0.3pt}%
  \let\item\@idxitem
}{%
  \clearpage
}
\makeatother

\IfFileExists{\jobname-pw.ind}{\input{\jobname-pw.ind}}{}

\end{document}

      