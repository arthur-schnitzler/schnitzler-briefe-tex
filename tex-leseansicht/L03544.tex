%% latex-leseansicht-vorspann.tex
%% Vorspann für die Leseansicht.
%% Lädt die gemeinsame Datei latex-vorspann.tex mit nicht gesetztem Schalter.

\newif\ifkorrekturansicht
\korrekturansichtfalse

\input{../tex-inputs/latex-vorspann}


\section[ Felix Salten an Arthur Schnitzler, 29. 1. 1910]{L03544 Felix Salten an Arthur Schnitzler,  29. 1. 1910}
\nopagebreak\mylabel{L03544v}
\rehead{ }\normalsize\beginnumbering\briefempfaengerindex{Schnitzler, Arthur@\textsc{Schnitzler, Arthur}!zzzSalten, Felix@\emph{von Felix Salten}!1910-01-291@{29. 1. 1910}|(be}
\toendnotes[C]{\smallbreak\pagebreak[2]}
\correspDesc{Versand  durch Felix Salten am 29. 1. 1910 in Wien
\newline{}Erhalt  durch Arthur Schnitzler im Zeitraum [29. 1. 1910
                  – 1. 2. 1910?] in Wien}\toendnotes[C]{\smallbreak}
\Standort{CUL, Schnitzler, B 89, B 2.}
\physDesc{Postkarte, 473 Zeichen
\newline{}Handschrift: schwarze Tinte, lateinische Kurrent
\newline{}Versand: Stempel: »\nobreak{}\oindex{XVIII., Währing@\textbf{XVIII., Währing}, \emph{Verwaltungsgebiet}|pwk}18/\textsubscript{1} Wien 111, 29. I. \textcolor{gray}{10}, 4\nobreak{}«.  
\newline{}Ordnung: mit Bleistift von unbekannter Hand nummeriert: »259« und
                                    »2« }\toendnotes[C]{\smallbreak}\pstart{}{\pb}\textcolor{gray}{\textbf{\textit{FELIX SALTEN}}}\pend{}\pstart{}\textcolor{gray}{\textbf{\textit{WIEN, XVIII.}}}\oindex{XVIII., Währing@\textbf{XVIII., Währing}, \emph{Verwaltungsgebiet}|pw}\pend{}\pstart{}\textcolor{gray}{\textbf{\textit{COTTAGEGASSE 37}}}\oindex{Wien@\textbf{Wien}!XVIII., Währing@\textbf{XVIII., Währing}!Cottagegasse@\textbf{Cottagegasse}, \emph{Straße}|pw}\oindex{Wien@\textbf{Wien}!XIX., Döbling@\textbf{XIX., Döbling}!Cottagegasse@\textbf{Cottagegasse}, \emph{Straße}|pw}\pend{}{\bigskip}\pstart{}Herrn D\textsuperscript{r} Arthur Schnitzler\pend{}\pstart{}Wien\oindex{Wien@\textbf{Wien}, \emph{Verwaltungsgebiet}|pw}\pend{}\pstart{}\label{T_L03544-1v}\edtext{X\substVorne{}\textsuperscript{IX}\substDazwischen{}VI\substHinten{}II}{\lemma{\textnormal{\emph{XVIII}}}\Cendnote{\textnormal{Zur Verdeutlichung wurde  von Salten\pwindex{Salten, Felix 6.\,9.\,1869 Budapest – 8.\,10.\,1945 Zürich@\textsc{Salten, Felix} (6.\,9.\,1869 Budapest – 8.\,10.\,1945 Zürich), \emph{Schriftsteller, Journalist, Chefredakteur}|pwk} »XVIII« seitlich wiederholt.}}}\label{T_L03544-1}. Spöttelgaße 7\oindex{Wien@\textbf{Wien}!XVIII., Währing@\textbf{XVIII., Währing}!Edmund-Weiß-Gasse 7@\textbf{Edmund-Weiß-Gasse 7}, \emph{Wohngebäude}|pw}\pend{}{\bigskip}\vspace{1em}
\pstart{}{\pb}Lieber,\pend\vspace{0.5em}
\pstart
           mein Schwager Ludwig\pwindex{Metzl, Ludwig *~9.\,3.\,1854@\textsc{Metzl, Ludwig} (*~9.\,3.\,1854)|pw} ist unverhofft aus Berlin\oindex{Berlin@\textbf{Berlin}, \emph{Hauptstadt}|pw} angekommen und legt mich heute, wie auch morgen, Sonntag, in Beschlag. Ich kann also leider nicht mit Ihnen spazieren gehen.
               Nächster Tage \label{K_L03544-1v}\edtext{Vormittag komme ich einmal zu Ihnen}{\lemma{\textnormal{\emph{Vormittag … Ihnen}}}\Cendnote{\textnormal{Am Dienstag, dem 1. 2. 1910 besuchte
                  Schnitzler{ }Salten\pwindex{Salten, Felix 6.\,9.\,1869 Budapest – 8.\,10.\,1945 Zürich@\textsc{Salten, Felix} (6.\,9.\,1869 Budapest – 8.\,10.\,1945 Zürich), \emph{Schriftsteller, Journalist, Chefredakteur}|pwk}. Am
                  2. 2. 1910 fand der Spaziergang statt. 
               }}}\label{K_L03544-1}. Muss Ihnen übrigens auch vom \label{K_L03544-2v}\edtext{Baron B.\pwindex{Berger, Alfred von 30.\,4.\,1853 Wien – 24.\,8.\,1912 ebd.@\textsc{Berger, Alfred von} (30.\,4.\,1853 Wien – 24.\,8.\,1912 ebd.), \emph{Schriftsteller, Journalist, Theaterleiter}|pw}}{\lemma{\textnormal{\emph{Baron B.}}}\Cendnote{\textnormal{Alfred von Berger\pwindex{Berger, Alfred von 30.\,4.\,1853 Wien – 24.\,8.\,1912 ebd.@\textsc{Berger, Alfred von} (30.\,4.\,1853 Wien – 24.\,8.\,1912 ebd.), \emph{Schriftsteller, Journalist, Theaterleiter}|pwk}, der neue Direktor des
                     \emph{Burgtheaters}\orgindex{Burgtheater@Burgtheater|pwk}}}}\label{K_L03544-2} erzählen. Er will den
                  \label{K_L03544-3v}\edtext{Medardus\pwindex{Schnitzler, Arthur 15.\,5.\,1862 Wien – 21.\,10.\,1931 ebd.@\textsc{Schnitzler, Arthur} (15.\,5.\,1862 Wien – 21.\,10.\,1931 ebd.), \emph{Schriftsteller, Mediziner}!junge Medardus. Dramatische Historie in einem Vorspiel und fünf Aufzügen@\strich\emph{Der junge Medardus. Dramatische Historie in einem Vorspiel und fünf Aufzügen}|pw}{ }\uline{mit} der Bastei}{\lemma{\textnormal{\emph{Medardus mit der Bastei}}}\Cendnote{\textnormal{Das Stück war
                   durch seinen Textumfang nur mit Kürzungen aufzuführen (vgl. XXXX Auszeichnungsfehler: Dokument L01981 nicht gefunden). Die 
                  auf dem Festungswall (Bastei) angesiedelten Szenen waren durch die vielen benötigten Statisten besonders 
                  aufwendig zu inszenieren. Vgl. A. S.: \emph{Tagebuch}, 1. 2. 1910.
               }}}\label{K_L03544-3} spielen. Auf Montag oder Dienstag also!\pend
           
\pstart
           Alles Herzliche von uns\pwindex{Salten, Ottilie 7.\,3.\,1868 Prag – 22.\,6.\,1942 Zürich@\textsc{Salten, Ottilie} (7.\,3.\,1868 Prag – 22.\,6.\,1942 Zürich), \emph{Schauspielerin}|pwv}
               zu Ihnen{\\[\baselineskip]} Ihr{\\[\baselineskip]}\spacefill\mbox{Salten}\pend
           \leftskip=0em{}
\pstart
           28. I. 10\pend
           \selectlanguage{ngerman}\endnumbering\briefempfaengerindex{Schnitzler, Arthur@\textsc{Schnitzler, Arthur}!zzzSalten, Felix@\emph{von Felix Salten}!1910-01-291@{29. 1. 1910}|)be}\mylabel{L03544h}  \newcommand{\dateiname}{L03544}\newcommand{\titel}{Felix Salten an Arthur Schnitzler, 29. 1. 1910}\newcommand{\editorInnen}{Martin Anton Müller und Laura Untner}%% latex-leseansicht-abspann.tex
%% Abspann für die Leseansicht.
%% Der Schalter \ifkorrekturansicht ist bereits durch den Vorspann gesetzt.

%% latex-abspann.tex
%% Gemeinsamer Abspann für Korrekturansicht und Leseansicht.
%% Setzt den Schalter \ifkorrekturansicht voraus (gesetzt in den
%% einbindenden Dateien latex-korrekturansicht-abspann.tex bzw.
%% latex-leseansicht-abspann.tex).
%% ---------------------------------------------------------------

\normalsize

% Das esempio-Environment wird nur in der Leseansicht benötigt
\ifkorrekturansicht\else
\newenvironment{esempio}[3]%
{
    \vspace{1.5ex}
    \rlap{\underline{#1}}
    \par
    \setlength{\parindent}{0cm}
    \nopagebreak
    \leftskip=#2cm
    \rightskip=#3cm
}
{
    \par
}
\fi

\doendnotes{C}
\bigskip
\vfill

\clearpage

\footnotesize

\ifkorrekturansicht
  \lohead{\textsc{register}}
\fi

% theindex-Environment neu definieren ohne reledmac
\makeatletter
\renewenvironment{theindex}{%
  \ifkorrekturansicht
    \section*{\indexname}%
  \else
    \subsubsection*{Index der erwähnten Entitäten}%
  \fi
  \setlength{\parindent}{0pt}%
  \setlength{\parskip}{0pt plus 0.3pt}%
  \let\item\@idxitem
}{%
  \ifkorrekturansicht\clearpage\fi
}
\makeatother

\IfFileExists{\jobname-pw.ind}{\input{\jobname-pw.ind}}{}

% Quellenangabe nur in der Leseansicht
\ifkorrekturansicht\else
% Fallback-Definitionen, falls die .tex-Datei \titel etc. nicht gesetzt hat
\providecommand{\titel}{}
\providecommand{\editorInnen}{}
\providecommand{\dateiname}{\jobname}

\vspace{3cm}

\vfill

\footnotesize
\textsc{Quelle}: \titel. Herausgegeben von {\editorInnen}. In: \emph{Arthur Schnitzler: Briefwechsel mit Autorinnen und Autoren}.
 Digitale Edition, https://schnitzler-briefe.acdh.oeaw.ac.at/{\dateiname}.html (Stand \today)
\fi

\end{document}


