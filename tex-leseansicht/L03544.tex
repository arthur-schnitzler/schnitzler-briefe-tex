%% latex-leseansicht-vorspann.tex
%% Vorspann für die Leseansicht.
%% Lädt die gemeinsame Datei latex-vorspann.tex mit nicht gesetztem Schalter.

\newif\ifkorrekturansicht
\korrekturansichtfalse

\input{../tex-inputs/latex-vorspann}


         
         \renewcommand{\erwaehntePersonen}{Personen: Alfred von Berger, Ludwig Metzl, Felix Salten, Ottilie Salten}
         \renewcommand{\erwaehnteInstitutionen}{Institutionen: Burgtheater}
         \renewcommand{\erwaehnteOrte}{Orte: Berlin, Cottagegasse, Edmund-Weiß-Gasse 7, Wien, XVIII., Währing}
         \renewcommand{\erwaehnteWerke}{Werke: Der junge Medardus. Dramatische Historie in einem Vorspiel und fünf Aufzügen}
               \section[ Felix Salten an Arthur Schnitzler, 29. 1. 1910]{ Felix Salten an Arthur Schnitzler, 29. 1. 1910}\nopagebreak\mylabel{v}\rehead{ }\begin{ledgroupsized}[t]{13cm}\normalsize\beginnumbering\briefempfaengerindex{Schnitzler, Arthur@\textsc{Schnitzler, Arthur}!zzzSalten, Felix@\emph{von Felix Salten}!1910-01-291@{29. 1. 1910}|(be} \toendnotes[C]{\smallbreak\pagebreak[2]} \Standort{CUL, Schnitzler, B 89, B 2.}
\physDesc{Postkarte, 473 Zeichen
\newline{}Handschrift: schwarze Tinte, lateinische Kurrent
\newline{}Versand: Stempel: »\nobreak{}\oindex{XVIII., Waehring@\textbf{XVIII., Währing}|pwk}18/\textsubscript{1} Wien 111, 29. I. \textcolor{gray}{10}, 4\nobreak{}«.  
\newline{}Ordnung: mit Bleistift von unbekannter Hand nummeriert: »259« und
                                    »2« }\toendnotes[C]{\smallbreak}\pstart{}{\pb}\textcolor{gray}{\textbf{\textit{FELIX SALTEN}}}\pend{}\pstart{}\textcolor{gray}{\textbf{\textit{WIEN, XVIII.}}}\oindex{XVIII., Waehring@\textbf{XVIII., Währing}|pw}\pend{}\pstart{}\textcolor{gray}{\textbf{\textit{COTTAGEGASSE 37}}}\oindex{Cottagegasse@\textbf{Cottagegasse}|pw}\pend{}{\bigskip}\pstart{}Herrn D\textsuperscript{r} Arthur Schnitzler\pend{}\pstart{}Wien\oindex{Wien@\textbf{Wien}|pw}\pend{}\pstart{}\label{T_L03544-1v}\edtext{X\substVorne{}\textsuperscript{IX}\substDazwischen{}VI\substHinten{}II}{\lemma{\textnormal{\emph{XVIII}}}\Cendnote{\textnormal{Zur Verdeutlichung wurde  von Salten\pwindex{Salten, Felix 06.09.1869 – 08.10.1945@\textsc{Salten, Felix} (06.09.1869 – 08.10.1945), \emph{Schriftsteller, Journalist, Chefredakteur}|pwk} »XVIII« seitlich wiederholt.}}}\label{T_L03544-1h}. Spöttelgaße 7\oindex{Edmund-Weiss-Gasse 7@\textbf{Edmund-Weiß-Gasse 7}|pw}\pend{}{\bigskip}\pstart{}{\pb}Lieber,\pend\pstart
           mein Schwager Ludwig\pwindex{Metzl, Ludwig *~1854-03-09@\textsc{Metzl, Ludwig} (*~1854-03-09)|pw} ist unverhofft aus Berlin\oindex{Berlin@\textbf{Berlin}|pw} angekommen und legt mich heute, wie auch morgen,
                  Sonntag, in Beschlag. Ich kann also leider nicht mit Ihnen spazieren gehen.
               Nächster Tage \label{K_L03544-1v}\edtext{Vormittag komme ich einmal zu Ihnen}{\lemma{\textnormal{\emph{Vormittag … Ihnen}}}\Cendnote{\textnormal{Am Dienstag, dem 1. 2. 1910 besuchte
                  Schnitzler\pwindex{Schnitzler, Arthur 15.05.1862 – 21.10.1931@\textsc{Schnitzler, Arthur} (15.05.1862 – 21.10.1931), \emph{Schriftsteller, Mediziner}|pwk}{ }Salten\pwindex{Salten, Felix 06.09.1869 – 08.10.1945@\textsc{Salten, Felix} (06.09.1869 – 08.10.1945), \emph{Schriftsteller, Journalist, Chefredakteur}|pwk}. Am
                  2. 2. 1910 fand der Spaziergang statt. 
               }}}\label{K_L03544-1h}. Muss Ihnen übrigens auch vom \label{K_L03544-2v}\edtext{Baron B.\pwindex{Berger, Alfred von 30.04.1853 – 24.08.1912@\textsc{Berger, Alfred von} (30.04.1853 – 24.08.1912), \emph{Schriftsteller, Journalist, Theaterleiter}|pw}}{\lemma{\textnormal{\emph{Baron B.}}}\Cendnote{\textnormal{Alfred von Berger\pwindex{Berger, Alfred von 30.04.1853 – 24.08.1912@\textsc{Berger, Alfred von} (30.04.1853 – 24.08.1912), \emph{Schriftsteller, Journalist, Theaterleiter}|pwk}, der neue Direktor des
                     \emph{Burgtheaters}\orgindex{Burgtheater@Burgtheater|pwk}}}}\label{K_L03544-2h} erzählen. Er will den
                  \label{K_L03544-3v}\edtext{Medardus\pwindex{Schnitzler, Arthur 15.05.1862 – 21.10.1931@\textsc{Schnitzler, Arthur} (15.05.1862 – 21.10.1931), \emph{Schriftsteller, Mediziner}!junge Medardus. Dramatische Historie in einem Vorspiel und fuenf Aufzuegen1910-10-26@\strich\emph{Der junge Medardus. Dramatische Historie in einem Vorspiel und fünf Aufzügen} {[}1910-10-26{]}|pw}{ }\uline{mit} der Bastei}{\lemma{\textnormal{\emph{Medardus mit der Bastei}}}\Cendnote{\textnormal{Das Stück war
                   durch seinen Textumfang nur mit Kürzungen aufzuführen (vgl. Arthur Schnitzler an Hermann Bahr, 17. 11. 1910). Die 
                  auf dem Festungswall (Bastei) angesiedelten Szenen waren durch die vielen benötigten Statisten besonders 
                  aufwendig zu inszenieren. Vgl. A. S.: \emph{Tagebuch}, 1. 2. 1910.
               }}}\label{K_L03544-3h} spielen. Auf Montag oder Dienstag also!\pend
           \pstart
           Alles Herzliche von uns\pwindex{Salten, Ottilie 07.03.1868 – 22.06.1942@\textsc{Salten, Ottilie} (07.03.1868 – 22.06.1942), \emph{Schauspielerin}|pwv}
               zu Ihnen{\\[\baselineskip]} Ihr{\\[\baselineskip]}\spacefill\mbox{Salten}\pend
           \leftskip=0em{}\pstart
           28. I. 10\pend
           
         
         \endnumbering\mylabel{h}\end{ledgroupsized}  \newcommand{\dateiname}{L03544}\newcommand{\titel}{Felix Salten an Arthur Schnitzler, 29. 1. 1910}\newcommand{\editorInnen}{Martin Anton Müller und Laura Untner}%% latex-leseansicht-abspann.tex
%% Abspann für die Leseansicht.
%% Der Schalter \ifkorrekturansicht ist bereits durch den Vorspann gesetzt.

%% latex-abspann.tex
%% Gemeinsamer Abspann für Korrekturansicht und Leseansicht.
%% Setzt den Schalter \ifkorrekturansicht voraus (gesetzt in den
%% einbindenden Dateien latex-korrekturansicht-abspann.tex bzw.
%% latex-leseansicht-abspann.tex).
%% ---------------------------------------------------------------

\normalsize

% Das esempio-Environment wird nur in der Leseansicht benötigt
\ifkorrekturansicht\else
\newenvironment{esempio}[3]%
{
    \vspace{1.5ex}
    \rlap{\underline{#1}}
    \par
    \setlength{\parindent}{0cm}
    \nopagebreak
    \leftskip=#2cm
    \rightskip=#3cm
}
{
    \par
}
\fi

\doendnotes{C}
\bigskip
\vfill

\clearpage

\footnotesize

\ifkorrekturansicht
  \lohead{\textsc{register}}
\fi

% theindex-Environment neu definieren ohne reledmac
\makeatletter
\renewenvironment{theindex}{%
  \ifkorrekturansicht
    \section*{\indexname}%
  \else
    \subsubsection*{Index der erwähnten Entitäten}%
  \fi
  \setlength{\parindent}{0pt}%
  \setlength{\parskip}{0pt plus 0.3pt}%
  \let\item\@idxitem
}{%
  \ifkorrekturansicht\clearpage\fi
}
\makeatother

\IfFileExists{\jobname-pw.ind}{\input{\jobname-pw.ind}}{}

% Quellenangabe nur in der Leseansicht
\ifkorrekturansicht\else
% Fallback-Definitionen, falls die .tex-Datei \titel etc. nicht gesetzt hat
\providecommand{\titel}{}
\providecommand{\editorInnen}{}
\providecommand{\dateiname}{\jobname}

\vspace{3cm}

\vfill

\footnotesize
\textsc{Quelle}: \titel. Herausgegeben von {\editorInnen}. In: \emph{Arthur Schnitzler: Briefwechsel mit Autorinnen und Autoren}.
 Digitale Edition, https://schnitzler-briefe.acdh.oeaw.ac.at/{\dateiname}.html (Stand \today)
\fi

\end{document}


      