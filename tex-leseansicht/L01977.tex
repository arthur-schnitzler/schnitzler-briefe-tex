%% latex-korrekturansicht-vorspann.tex
%% Vorspann für die Korrekturansicht.
%% Lädt die gemeinsame Datei latex-vorspann.tex mit gesetztem Schalter.

\newif\ifkorrekturansicht
\korrekturansichttrue

\input{../tex-inputs/latex-vorspann}


\section[Arthur Schnitzler an Richard Beer-Hofmann, 3. 11. 1910]{L01977 Arthur Schnitzler an Richard Beer-Hofmann, 3. 11. 1910}
\nopagebreak\mylabel{L01977v}
\rehead{ }\normalsize\beginnumbering\briefempfaengerindex{Beer-Hofmann, Richard@\textsc{Beer-Hofmann, Richard}!zzzSchnitzler, Arthur@\emph{von Arthur Schnitzler}!1910-11-032@{3. 11. 1910}|(be}
\toendnotes[C]{\smallbreak\pagebreak[2]}\Standort{YCGL, MSS 31.}
\physDesc{Brief, 1 Blatt, 3 Seiten, Umschlag, 335 Zeichen
\newline{}Handschrift: Bleistift, deutsche Kurrent
\newline{}Versand: ohne postalischen Übermittlungsvermerk }
\buchAbdrucke{\weitereDrucke{Arthur Schnitzler, Richard Beer-Hofmann: \emph{Briefwechsel 1891–1931}. Wien, Zürich: \emph{Europaverlag} 1992, S. 213.} }\toendnotes[C]{\smallbreak}\pstart{}{\pb}\textcolor{gray}{\textbf{Dr. Arthur Schnitzler}}\pend{}\pstart{}\textcolor{gray}{\textbf{Wien XVIII. Sternwartestrasse 71\oindex{Sternwartestrasse 71@\textbf{Sternwartestraße 71}, \emph{Wohngebäude (K.WHS)}|pw}}}\pend{}{\bigskip}\pstart{}{\pb}\textsc{Herrn Dr. Richard Beer Hofmann}\pend{}\pstart{}Wien XVIII\oindex{XVIII., Waehring@\textbf{XVIII., Währing}, \emph{A.ADM3}|pw}\pend{}\pstart{}\textsc{Hasenauerstr 59\oindex{Hasenauerstrasse 59@\textbf{Hasenauerstraße 59}, \emph{Wohngebäude (K.WHS)}|pw}}\pend{}{\bigskip}\vspace{1em}
\pstart
           {\pb}\textcolor{gray}{\textbf{Dr. Arthur Schnitzler}}\hfill 3/11 910\pend
           
\pstart
           \textcolor{gray}{\textbf{Wien XVIII. \label{K_L01977-1v}\edtext{Spoettelgasse 7}{\lemma{\textnormal{\emph{Spoettelgasse 7}}}\Cendnote{\textnormal{Schnitzler verwendet das nicht mehr aktuelle
                           Briefpapier.}}}\label{K_L01977-1}\oindex{Edmund-Weiss-Gasse 7@\textbf{Edmund-Weiß-Gasse 7}, \emph{Wohngebäude (K.WHS)}|pw}.}}\pend
           
\pstart{}lieber Richard, \pend\vspace{0.5em}
\pstart
           für den »Mutterſchutz\orgindex{Bund fuer Mutterschutz@Bund für Mutterschutz|pw}« ka{\geminationn} ich nicht leſen – weil ich ſonſt auch für ſo u ſoviel
               andre Vereine leſen müßte, {\pb}die mich ſchon
               aufgefordert haben u noch auffordern werden; u. weil ich überhaupt in Wien\oindex{Wien@\textbf{Wien}, \emph{A.ADM2}|pw} nicht gern leſe.\pend
           
\pstart
           Leo\pwindex{Van-Jung, Leo 15.10.1866 – 02.07.1939@\textsc{Van-Jung, Leo} (15.10.1866 – 02.07.1939), \emph{Gesangspädagoge/Gesangspädagogin, Mathematiker/Mathematikerin}|pw} ni{\geminationm}ts mir
               gewiſs nicht übel.\pend
           
\pstart
           {\pb}Herzlichſt Ihr{\\[\baselineskip]}\spacefill\mbox{A.}\pend
           \leftskip=0em{}\selectlanguage{ngerman}\endnumbering\briefempfaengerindex{Beer-Hofmann, Richard@\textsc{Beer-Hofmann, Richard}!zzzSchnitzler, Arthur@\emph{von Arthur Schnitzler}!1910-11-032@{3. 11. 1910}|)be}\mylabel{L01977h}  \normalsize

\doendnotes{C}
\bigskip
\vfill

\clearpage

\footnotesize

\lohead{\textsc{register}}

% Definiere theindex-Environment komplett neu ohne reledmac
\makeatletter
\renewenvironment{theindex}{%
  \section*{\indexname}%
  \setlength{\parindent}{0pt}%
  \setlength{\parskip}{0pt plus 0.3pt}%
  \let\item\@idxitem
}{%
  \clearpage
}
\makeatother

\IfFileExists{\jobname-pw.ind}{\input{\jobname-pw.ind}}{}

\end{document}

      