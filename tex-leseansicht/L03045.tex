%% latex-leseansicht-vorspann.tex
%% Vorspann für die Leseansicht.
%% Lädt die gemeinsame Datei latex-vorspann.tex mit nicht gesetztem Schalter.

\newif\ifkorrekturansicht
\korrekturansichtfalse

\input{../tex-inputs/latex-vorspann}


\section[ Felix Salten: Widmungsexemplar Neue Menschen auf alter Erde für Arthur Schnitzler, 30. 4. 1925]{L03045 Felix Salten: Widmungsexemplar Neue Menschen auf alter Erde für Arthur
               Schnitzler,  30. 4. 1925}
\nopagebreak\mylabel{L03045v}
\rehead{ }\normalsize\beginnumbering\briefempfaengerindex{Schnitzler, Arthur@\textsc{Schnitzler, Arthur}!zzzSalten, Felix@\emph{von Felix Salten}!1925-04-301@{30. 4. 1925}|(be}
\toendnotes[C]{\smallbreak\pagebreak[2]}
\correspDesc{Versand  durch Felix Salten am 30. 4. 1925 in Wien
\newline{}Erhalt  durch Arthur Schnitzler im Zeitraum [30. 4. 1925
                  – 4. 5. 1925?] in Wien}\toendnotes[C]{\smallbreak}
\Standort{DLA, G:Schnitzler, Arthur (Sammlung Heinrich Schnitzler).}
\physDesc{Widmung am Schmutztitel, 48 Zeichen
\newline{}Handschrift: schwarze Tinte, lateinische Kurrent}\toendnotes[C]{\smallbreak}
\pstart
           \noindent{}\centering{}{\pb}\textcolor{gray}{\textbf{\so{FELIX SALTEN}}}\pend
           
\pstart
           \centering{}\textcolor{gray}{\textbf{NEUE MENSCHEN\pwindex{Salten, Felix 6.\,9.\,1869 Budapest – 8.\,10.\,1945 Zürich@\textsc{Salten, Felix} (6.\,9.\,1869 Budapest – 8.\,10.\,1945 Zürich), \emph{Schriftsteller, Journalist, Chefredakteur}!Neue Menschen auf alter Erde. Eine Palästinafahrt@\strich\emph{Neue Menschen auf alter Erde. Eine Palästinafahrt}|pw}}}\pend
           
\pstart
           \centering{}\textcolor{gray}{\textbf{AUF ALTER ERDE\pwindex{Salten, Felix 6.\,9.\,1869 Budapest – 8.\,10.\,1945 Zürich@\textsc{Salten, Felix} (6.\,9.\,1869 Budapest – 8.\,10.\,1945 Zürich), \emph{Schriftsteller, Journalist, Chefredakteur}!Neue Menschen auf alter Erde. Eine Palästinafahrt@\strich\emph{Neue Menschen auf alter Erde. Eine Palästinafahrt}|pw}}}\pend
           {\vspace{1\baselineskip}}
\pstart
           Arthur Schnitzler {\\}herzlich{\\}\spacefill\mbox{Felix Salten}\pend
           
\pstart
           \label{K_L03045-1v}\edtext{30⋅4⋅25}{\lemma{\textnormal{\emph{30⋅4⋅25}}}\Cendnote{\textnormal{Schnitzler
                     kannte das Buch in Auszügen seit ein paar Wochen, vgl. A. S.: \emph{Tagebuch}, 17. 2. 1925.
                     Trotz seiner Danksagung an Salten\pwindex{Salten, Felix 6.\,9.\,1869 Budapest – 8.\,10.\,1945 Zürich@\textsc{Salten, Felix} (6.\,9.\,1869 Budapest – 8.\,10.\,1945 Zürich), \emph{Schriftsteller, Journalist, Chefredakteur}|pwk} vom XXXX Auszeichnungsfehler: Dokument L03021 nicht gefunden
                     las er das Buch erst am 25. 7. 1925 und beurteilte es im \emph{Tagebuch}\pwindex{Schnitzler, Arthur 15.\,5.\,1862 Wien – 21.\,10.\,1931 ebd.@\textsc{Schnitzler, Arthur} (15.\,5.\,1862 Wien – 21.\,10.\,1931 ebd.), \emph{Schriftsteller, Mediziner}!Tagebuch@\strich\emph{Tagebuch}|pwk}
                  negativ.}}}\label{K_L03045-1}\pend
           {\vspace{1\baselineskip}}
\pstart
           \centering{}{\pb}\textcolor{gray}{\textbf{\so{FELIX SALTEN}}}\pend
           
\pstart
           \centering{}\textcolor{gray}{\textbf{NEUE MENSCHEN\pwindex{Salten, Felix 6.\,9.\,1869 Budapest – 8.\,10.\,1945 Zürich@\textsc{Salten, Felix} (6.\,9.\,1869 Budapest – 8.\,10.\,1945 Zürich), \emph{Schriftsteller, Journalist, Chefredakteur}!Neue Menschen auf alter Erde. Eine Palästinafahrt@\strich\emph{Neue Menschen auf alter Erde. Eine Palästinafahrt}|pw}}}\pend
           
\pstart
           \centering{}\textcolor{gray}{\textbf{AUF ALTER ERDE\pwindex{Salten, Felix 6.\,9.\,1869 Budapest – 8.\,10.\,1945 Zürich@\textsc{Salten, Felix} (6.\,9.\,1869 Budapest – 8.\,10.\,1945 Zürich), \emph{Schriftsteller, Journalist, Chefredakteur}!Neue Menschen auf alter Erde. Eine Palästinafahrt@\strich\emph{Neue Menschen auf alter Erde. Eine Palästinafahrt}|pw}}}\pend
           {\vspace{1\baselineskip}}
\pstart
           \centering{}\textcolor{gray}{\textbf{EINE PALÄSTINAFAHRT}}\pend
           {\vspace{1\baselineskip}}
\pstart
           \centering{}\textcolor{gray}{\textbf{\so{1925}}}\pend
           
\pstart
           \centering{}\textcolor{gray}{\textbf{\so{PAUL ZSOLNAY VERLAG}\orgindex{Paul Zsolnay Verlag@Paul Zsolnay Verlag|pw}}}\pend
           
\pstart
           \centering{}\textcolor{gray}{\textbf{\so{BERLIN}\oindex{Berlin@\textbf{Berlin}, \emph{Hauptstadt}|pw}{ }–{ }\so{WIEN}\oindex{Wien@\textbf{Wien}, \emph{Verwaltungsgebiet}|pw}{ }–{ }\so{LEIPZIG}\oindex{Leipzig@\textbf{Leipzig}, \emph{Hauptstadt}|pw}}}\pend
           \selectlanguage{ngerman}\endnumbering\briefempfaengerindex{Schnitzler, Arthur@\textsc{Schnitzler, Arthur}!zzzSalten, Felix@\emph{von Felix Salten}!1925-04-301@{30. 4. 1925}|)be}\mylabel{L03045h}  \newcommand{\dateiname}{L03045}\newcommand{\titel}{Felix Salten: Widmungsexemplar Neue Menschen auf alter Erde für Arthur Schnitzler, 30. 4. 1925}\newcommand{\editorInnen}{Martin Anton Müller und Laura Untner}%% latex-leseansicht-abspann.tex
%% Abspann für die Leseansicht.
%% Der Schalter \ifkorrekturansicht ist bereits durch den Vorspann gesetzt.

%% latex-abspann.tex
%% Gemeinsamer Abspann für Korrekturansicht und Leseansicht.
%% Setzt den Schalter \ifkorrekturansicht voraus (gesetzt in den
%% einbindenden Dateien latex-korrekturansicht-abspann.tex bzw.
%% latex-leseansicht-abspann.tex).
%% ---------------------------------------------------------------

\normalsize

% Das esempio-Environment wird nur in der Leseansicht benötigt
\ifkorrekturansicht\else
\newenvironment{esempio}[3]%
{
    \vspace{1.5ex}
    \rlap{\underline{#1}}
    \par
    \setlength{\parindent}{0cm}
    \nopagebreak
    \leftskip=#2cm
    \rightskip=#3cm
}
{
    \par
}
\fi

\doendnotes{C}
\bigskip
\vfill

\clearpage

\footnotesize

\ifkorrekturansicht
  \lohead{\textsc{register}}
\fi

% theindex-Environment neu definieren ohne reledmac
\makeatletter
\renewenvironment{theindex}{%
  \ifkorrekturansicht
    \section*{\indexname}%
  \else
    \subsubsection*{Index der erwähnten Entitäten}%
  \fi
  \setlength{\parindent}{0pt}%
  \setlength{\parskip}{0pt plus 0.3pt}%
  \let\item\@idxitem
}{%
  \ifkorrekturansicht\clearpage\fi
}
\makeatother

\IfFileExists{\jobname-pw.ind}{\input{\jobname-pw.ind}}{}

% Quellenangabe nur in der Leseansicht
\ifkorrekturansicht\else
% Fallback-Definitionen, falls die .tex-Datei \titel etc. nicht gesetzt hat
\providecommand{\titel}{}
\providecommand{\editorInnen}{}
\providecommand{\dateiname}{\jobname}

\vspace{3cm}

\vfill

\footnotesize
\textsc{Quelle}: \titel. Herausgegeben von {\editorInnen}. In: \emph{Arthur Schnitzler: Briefwechsel mit Autorinnen und Autoren}.
 Digitale Edition, https://schnitzler-briefe.acdh.oeaw.ac.at/{\dateiname}.html (Stand \today)
\fi

\end{document}


