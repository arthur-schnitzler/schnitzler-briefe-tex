%% latex-leseansicht-vorspann.tex
%% Vorspann für die Leseansicht.
%% Lädt die gemeinsame Datei latex-vorspann.tex mit nicht gesetztem Schalter.

\newif\ifkorrekturansicht
\korrekturansichtfalse

\input{../tex-inputs/latex-vorspann}

\begin{center}
            \textcolor{red}{ENTWURF, NICHT FERTIG KORRIGIERT}
                      \end{center}
            
         \renewcommand{\erwaehnteOrte}{Orte: Berlin, Wien}
         \renewcommand{\erwaehnteWerke}{}
               \section[Paul Goldmann an Arthur Schnitzler, 7. 7. 1907]{ Paul Goldmann an Arthur Schnitzler, 7. 7. 1907}\nopagebreak\mylabel{v}\rehead{ }\begin{ledgroupsized}[t]{13cm}\normalsize\beginnumbering \toendnotes[C]{\smallbreak\pagebreak[2]} \Standort{DLA, A:Schnitzler, HS.NZ85.1.3175.}
\physDesc{ Blätter,  Seiten
\newline{}Handschrift: , deutsche Kurrent}\toendnotes[C]{\smallbreak}{\pb}\textcolor{gray}{\textbf{ Dessauerstrasse 19}}\textcolor{red}{\textsuperscript{\textbf{KEY}}}\pstart
            7. 7. 07. \pend
           \pstart
           \pend
           \pstart
           \pend
           \pstart{}Lieber Freund,\pend\pstart
           \pend
           \pstart
           Das traurige \label{XXXXv}\edtext{Ereignis\textcolor{red}{\textsuperscript{\textbf{KEY}}}[Kommentar:
                  wohl\u0020Mamroths\u0020Tod]}{\lemma{\textnormal{\emph{XXXX Lemmafehler}}}\Cendnote{\textnormal{}}}\label{XXXX} hat in ſeinem Gefolge eine ſolche Fülle von
               Angelegenheiten gehabt, die erledigt werden mußten, daß ich erſt heut
               dazu komme, Deinen lieben Brief zu beantworten u. Dir, auch im Namen der Meinigen,
               für Deine ſchönen, teilnehmenden Worte zu danken, die uns Alle tief berührt haben.
               \pend
           \pstart
           Mir iſt der Tod zum erſten Mal ganz in die Nähe gekommen, {\pb} u. ich habe
               ihn erkannt, als das, was er iſt: unſinnig u. ſcheußlich. \pend
           \pstart
           Das Schwerſte, das Du mir zu überwinden wünſchſt, waren nicht die Tage in Frankfurt\textcolor{red}{\textsuperscript{\textbf{KEY}}}. Das Schwerſte beginnt jetzt. Es iſt die Leere,
               die das Hinſcheiden eines geliebten Menſchen\textcolor{red}{\textsuperscript{\textbf{KEY}}} im Leben
               des Zurückgebliebenen läßt – es iſt die Sehnſucht, ein teures Geſicht wiederzuſehen,
               eine vetraute Stimme zu hören, die man niemals wiederſehen u. \strikeout{wiederhoh\textcolor{gray}{re}}\strikeout{wird} wiederhören wird, – {\pb} es iſt die
               Unmöglichkeit, ſich Jemanden\textcolor{red}{\textsuperscript{\textbf{KEY}}} als todt (todt!)
               vorzuſtellen, der noch vor Kurzem von Geiſt u. Leben ſprühte u. an dem man mit ganzer
               Seele gehangen hat....... \pend
           \pstart
           Dir u. Deiner Frau\textcolor{red}{\textsuperscript{\textbf{KEY}}} (der ich für ihre Teilnahme vielmals
               zu danken bitte) wünſche ich frohe Sommertage. Schreib’ mir jedenfalls, wo Ihr\textcolor{red}{\textsuperscript{\textbf{KEY}}} ſeid. Freilich iſt die Hoffnung gering, daß ich Euch\textcolor{red}{\textsuperscript{\textbf{KEY}}} in dieſem Sommer ſehen werde, da ich diesmal meine
                  Mutter\textcolor{red}{\textsuperscript{\textbf{KEY}}} nicht allein {\pb} laſſen u. mit ihr
               keine weiten Reiſen machen kann. Wahrſcheinlich gehen wir imAuguſt
               zunächſt noch Marienbad\textcolor{red}{\textsuperscript{\textbf{KEY}}}. \pend
           \pstart
           Mißverſtändniſſe ſollen uns gewiß nicht mehr trennen. Ich bin wenigſtens diesmal
               von Wien\textcolor{red}{\textsuperscript{\textbf{KEY}}} mit dem feſten Vorſatz fortgefahren, Alles \strikeout{zu}, was an mir liegt, zu tun, um eine mir
                  \strikeout{\textcolor{gray}{mein}} eine alte Freundſchaft zu erhalten, deren Wert ich gewiß nicht geringer
               bemeſſe, wie einſt\strikeout{,}.{\\[\baselineskip]}Nimm’ alſo nochmals meinen\pend
           \leftskip=0em{}\pstart
           {\\[\baselineskip]}u. der Meinigen herzlichſten\pend
           \leftskip=0em{}\pstart
           {\\[\baselineskip]}Dank u. ſei, ſammt Deiner\pend
           \leftskip=0em{}\pstart
           {\\[\baselineskip]}Frau\textcolor{red}{\textsuperscript{\textbf{KEY}}}, vielmals gegrüßt von\pend
           \leftskip=0em{}\pstart
           {\\[\baselineskip]}Deinem\pend
           \leftskip=0em{}\pstart
           {\\[\baselineskip]}\spacefill\mbox{Paul Goldmann. }\pend
           \leftskip=0em{}
         
         \endnumbering\mylabel{h}\end{ledgroupsized}\begin{anhang}\end{anhang}\newcommand{\dateiname}{L03254}\newcommand{\titel}{Paul Goldmann an Arthur Schnitzler, 7. 7. 1907}\newcommand{\editorInnen}{Martin Anton Müller und Laura Untner}%% latex-leseansicht-abspann.tex
%% Abspann für die Leseansicht.
%% Der Schalter \ifkorrekturansicht ist bereits durch den Vorspann gesetzt.

%% latex-abspann.tex
%% Gemeinsamer Abspann für Korrekturansicht und Leseansicht.
%% Setzt den Schalter \ifkorrekturansicht voraus (gesetzt in den
%% einbindenden Dateien latex-korrekturansicht-abspann.tex bzw.
%% latex-leseansicht-abspann.tex).
%% ---------------------------------------------------------------

\normalsize

% Das esempio-Environment wird nur in der Leseansicht benötigt
\ifkorrekturansicht\else
\newenvironment{esempio}[3]%
{
    \vspace{1.5ex}
    \rlap{\underline{#1}}
    \par
    \setlength{\parindent}{0cm}
    \nopagebreak
    \leftskip=#2cm
    \rightskip=#3cm
}
{
    \par
}
\fi

\doendnotes{C}
\bigskip
\vfill

\clearpage

\footnotesize

\ifkorrekturansicht
  \lohead{\textsc{register}}
\fi

% theindex-Environment neu definieren ohne reledmac
\makeatletter
\renewenvironment{theindex}{%
  \ifkorrekturansicht
    \section*{\indexname}%
  \else
    \subsubsection*{Index der erwähnten Entitäten}%
  \fi
  \setlength{\parindent}{0pt}%
  \setlength{\parskip}{0pt plus 0.3pt}%
  \let\item\@idxitem
}{%
  \ifkorrekturansicht\clearpage\fi
}
\makeatother

\IfFileExists{\jobname-pw.ind}{\input{\jobname-pw.ind}}{}

% Quellenangabe nur in der Leseansicht
\ifkorrekturansicht\else
% Fallback-Definitionen, falls die .tex-Datei \titel etc. nicht gesetzt hat
\providecommand{\titel}{}
\providecommand{\editorInnen}{}
\providecommand{\dateiname}{\jobname}

\vspace{3cm}

\vfill

\footnotesize
\textsc{Quelle}: \titel. Herausgegeben von {\editorInnen}. In: \emph{Arthur Schnitzler: Briefwechsel mit Autorinnen und Autoren}.
 Digitale Edition, https://schnitzler-briefe.acdh.oeaw.ac.at/{\dateiname}.html (Stand \today)
\fi

\end{document}


      