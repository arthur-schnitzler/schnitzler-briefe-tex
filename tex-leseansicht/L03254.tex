%% latex-leseansicht-vorspann.tex
%% Vorspann für die Leseansicht.
%% Lädt die gemeinsame Datei latex-vorspann.tex mit nicht gesetztem Schalter.

\newif\ifkorrekturansicht
\korrekturansichtfalse

\input{../tex-inputs/latex-vorspann}


\section[ Paul Goldmann an Arthur Schnitzler, 7. 7. 1907]{L03254 Paul Goldmann an Arthur Schnitzler,  7. 7. 1907}
\nopagebreak\mylabel{L03254v}
\rehead{ }\normalsize\beginnumbering\briefempfaengerindex{Schnitzler, Arthur@\textsc{Schnitzler, Arthur}!zzzGoldmann, Paul@\emph{von Paul Goldmann}!1907-07-071@{7. 7. 1907}|(be}
\toendnotes[C]{\smallbreak\pagebreak[2]}
\correspDesc{Versand  durch Paul Goldmann am 7. 7. 1907 in Berlin 
\newline{}Erhalt  durch Arthur Schnitzler im Zeitraum [8. 7. 1907
                  – 12. 7. 1907?] in Wien}\toendnotes[C]{\smallbreak}
\Standort{DLA, A:Schnitzler, HS.NZ85.1.3175.}
\physDesc{Brief, 1 Blatt, 4 Seiten, 1720 Zeichen
\newline{}Handschrift: blaue Tinte, deutsche Kurrent}\toendnotes[C]{\smallbreak}
\pstart
           \raggedleft{}{\pb}\textcolor{gray}{\textbf{DESSAUERSTRASSE 19}}\oindex{Dessauer Straße@\textbf{Dessauer Straße}, \emph{Straße}|pw}\pend
           
\pstart
           7. 7. 07.\pend
           
\pstart{}Lieber Freund,\pend\vspace{0.5em}
\pstart
           Das traurige \label{K_L03254-1v}\edtext{Ereignis}{\lemma{\textnormal{\emph{Ereignis}}}\Cendnote{\textnormal{Goldmanns\pwindex{Goldmann, Paul 31.\,1.\,1865 Breslau – 25.\,9.\,1935 Wien@\textsc{Goldmann, Paul} (31.\,1.\,1865 Breslau – 25.\,9.\,1935 Wien), \emph{Schriftsteller, Journalist}|pwk} Onkel Fedor Mamroth\pwindex{Mamroth, Fedor 21.\,2.\,1851 Breslau – 25.\,6.\,1907 Frankfurt am Main@\textsc{Mamroth, Fedor} (21.\,2.\,1851 Breslau – 25.\,6.\,1907 Frankfurt am Main), \emph{Journalist, Kritiker}|pwk} war am 25. 6. 1907 an den Folgen von Darmkrebs verstorben.}}}\label{K_L03254-1} hat in{ }ſeinem Gefolge eine{ }ſolche Fülle von Angelegenheiten gehabt, die erledigt werden
               mußten, daß ich erſt heut dazu komme, Deinen lieben
               Brief zu beantworten u. Dir, auch im Namen der Meinigen, für Deine{ }ſchönen,
               teilnehmenden Worte zu danken, die uns Alle tief berührt haben.\pend
           
\pstart
           Mir iſt der Tod zum erſten Mal ganz in die Nähe gekommen, {\pb}u. ich habe ihn erkannt, als das, was er iſt:
               unſinnig u.{ }ſcheußlich.\pend
           
\pstart
           Das Schwerſte, das Du mir zu überwinden wünſchſt, waren nicht die Tage in Frankfurt\oindex{Frankfurt am Main@\textbf{Frankfurt am Main}, \emph{Hauptstadt}|pw}. Das Schwerſte beginnt jetzt. Es iſt die
               Leere, die das Hinſcheiden eines geliebten Menſchen\pwindex{Mamroth, Fedor 21.\,2.\,1851 Breslau – 25.\,6.\,1907 Frankfurt am Main@\textsc{Mamroth, Fedor} (21.\,2.\,1851 Breslau – 25.\,6.\,1907 Frankfurt am Main), \emph{Journalist, Kritiker}|pwv} im Leben des Zurückgebliebenen läßt, – es iſt die
               Sehnſucht, ein teures Geſicht wiederzuſehen, eine vertraute Stimme zu hören, die man
               niemals wiederſehen u. \strikeout{wiederhoh\textcolor{gray}{re}}{ }\strikeout{wird} wiederhören wird, – {\pb}es iſt die Unmöglichkeit,{ }ſich Jemanden\pwindex{Mamroth, Fedor 21.\,2.\,1851 Breslau – 25.\,6.\,1907 Frankfurt am Main@\textsc{Mamroth, Fedor} (21.\,2.\,1851 Breslau – 25.\,6.\,1907 Frankfurt am Main), \emph{Journalist, Kritiker}|pwv} als todt (todt!) vorzuſtellen, der
               noch vor Kurzem von Geiſt u. Leben{ }ſprühte u. an dem man mit ganzer Seele gehangen
                  hat{\dotsseven}\pend
           
\pstart
           Dir u. Deiner Frau\pwindex{Schnitzler, Olga 17.\,1.\,1882 Wien – 13.\,1.\,1970 Lugano@\textsc{Schnitzler, Olga} (17.\,1.\,1882 Wien – 13.\,1.\,1970 Lugano), \emph{Schauspielerin, Sängerin}|pwv} (der ich
               für ihre Teilnahme vielmals zu danken bitte) wünſche ich frohe Sommertage. Schreib’
               mir jedenfalls, wo Ihr\pwindex{Schnitzler, Olga 17.\,1.\,1882 Wien – 13.\,1.\,1970 Lugano@\textsc{Schnitzler, Olga} (17.\,1.\,1882 Wien – 13.\,1.\,1970 Lugano), \emph{Schauspielerin, Sängerin}|pwv}{ }ſeid.
               Freilich iſt die Hoffnung gering, daß ich Euch\pwindex{Schnitzler, Olga 17.\,1.\,1882 Wien – 13.\,1.\,1970 Lugano@\textsc{Schnitzler, Olga} (17.\,1.\,1882 Wien – 13.\,1.\,1970 Lugano), \emph{Schauspielerin, Sängerin}|pwv}{ }\label{K_L03254-2v}\edtext{in dieſem Sommer{ }ſehen}{\lemma{\textnormal{\emph{in diesem Sommer sehen}}}\Cendnote{\textnormal{Schnitzler und Goldmann\pwindex{Goldmann, Paul 31.\,1.\,1865 Breslau – 25.\,9.\,1935 Wien@\textsc{Goldmann, Paul} (31.\,1.\,1865 Breslau – 25.\,9.\,1935 Wien), \emph{Schriftsteller, Journalist}|pwk} trafen sich erst am 8. 10. 1907
                  wieder.}}}\label{K_L03254-2} werde, da ich diesmal meine Mutter\pwindex{Goldmann, Clementine 15.\,5.\,1842 Breslau – 24.\,2.\,1924 Frankfurt am Main@\textsc{Goldmann, Clementine} (15.\,5.\,1842 Breslau – 24.\,2.\,1924 Frankfurt am Main)|pw} nicht allein {\pb}laſſen u. mit ihr
               keine weiten Reiſen machen kann. Wahrſcheinlich gehen wir im Auguſt zunächſt nach Marienbad\oindex{Marienbad@\textbf{Marienbad}|pw}.\pend
           
\pstart
           \label{K_L03254-3v}\edtext{Mißverſtändniſſe}{\lemma{\textnormal{\emph{Mißverständnisse}}}\Cendnote{\textnormal{Möglicherweise hatte es beim letzten persönlichen Treffen am
                     2. 6. 1907 eine
                  Auseinandersetzung gegeben. Ansonsten könnte es sich allgemein um die
                  Verstimmungen der letzten Jahre handeln.}}}\label{K_L03254-3}{ }ſollen uns gewiß nicht mehr trennen. Ich bin wenigſtens diesmal von
                  Wien\oindex{Wien@\textbf{Wien}, \emph{Verwaltungsgebiet}|pw} mit dem feſten Vorſatz fortgefahren, Alles
                  \strikeout{zu}, was an mir liegt, zu tun, um mir \substVorne{}\textsuperscript{\textcolor{gray}{me}}\substDazwischen{}eine\substHinten{} alte Freundſchaft zu erhalten, deren Wert ich gewiß nicht geringer bemeſſe,
               wie einſt\substVorne{}\textsuperscript{,}\substDazwischen{}.\substHinten{}\pend
           
\pstart
           Nimm’ alſo nochmals meinen u. der Meinigen herzlichſten Dank u.{ }ſei,{ }ſammt
               Deiner Frau\pwindex{Schnitzler, Olga 17.\,1.\,1882 Wien – 13.\,1.\,1970 Lugano@\textsc{Schnitzler, Olga} (17.\,1.\,1882 Wien – 13.\,1.\,1970 Lugano), \emph{Schauspielerin, Sängerin}|pwv}, vielmals gegrüßt
               von {\\[\baselineskip]}Deinem {\\[\baselineskip]}\spacefill\mbox{Paul Goldmann.}\pend
           \leftskip=0em{}\selectlanguage{ngerman}\endnumbering\briefempfaengerindex{Schnitzler, Arthur@\textsc{Schnitzler, Arthur}!zzzGoldmann, Paul@\emph{von Paul Goldmann}!1907-07-071@{7. 7. 1907}|)be}\mylabel{L03254h}  \newcommand{\dateiname}{L03254}\newcommand{\titel}{Paul Goldmann an Arthur Schnitzler, 7. 7. 1907}\newcommand{\editorInnen}{Martin Anton Müller und Laura Untner}%% latex-leseansicht-abspann.tex
%% Abspann für die Leseansicht.
%% Der Schalter \ifkorrekturansicht ist bereits durch den Vorspann gesetzt.

%% latex-abspann.tex
%% Gemeinsamer Abspann für Korrekturansicht und Leseansicht.
%% Setzt den Schalter \ifkorrekturansicht voraus (gesetzt in den
%% einbindenden Dateien latex-korrekturansicht-abspann.tex bzw.
%% latex-leseansicht-abspann.tex).
%% ---------------------------------------------------------------

\normalsize

% Das esempio-Environment wird nur in der Leseansicht benötigt
\ifkorrekturansicht\else
\newenvironment{esempio}[3]%
{
    \vspace{1.5ex}
    \rlap{\underline{#1}}
    \par
    \setlength{\parindent}{0cm}
    \nopagebreak
    \leftskip=#2cm
    \rightskip=#3cm
}
{
    \par
}
\fi

\doendnotes{C}
\bigskip
\vfill

\clearpage

\footnotesize

\ifkorrekturansicht
  \lohead{\textsc{register}}
\fi

% theindex-Environment neu definieren ohne reledmac
\makeatletter
\renewenvironment{theindex}{%
  \ifkorrekturansicht
    \section*{\indexname}%
  \else
    \subsubsection*{Index der erwähnten Entitäten}%
  \fi
  \setlength{\parindent}{0pt}%
  \setlength{\parskip}{0pt plus 0.3pt}%
  \let\item\@idxitem
}{%
  \ifkorrekturansicht\clearpage\fi
}
\makeatother

\IfFileExists{\jobname-pw.ind}{\input{\jobname-pw.ind}}{}

% Quellenangabe nur in der Leseansicht
\ifkorrekturansicht\else
% Fallback-Definitionen, falls die .tex-Datei \titel etc. nicht gesetzt hat
\providecommand{\titel}{}
\providecommand{\editorInnen}{}
\providecommand{\dateiname}{\jobname}

\vspace{3cm}

\vfill

\footnotesize
\textsc{Quelle}: \titel. Herausgegeben von {\editorInnen}. In: \emph{Arthur Schnitzler: Briefwechsel mit Autorinnen und Autoren}.
 Digitale Edition, https://schnitzler-briefe.acdh.oeaw.ac.at/{\dateiname}.html (Stand \today)
\fi

\end{document}


