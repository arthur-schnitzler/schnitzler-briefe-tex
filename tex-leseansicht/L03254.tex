%% latex-leseansicht-vorspann.tex
%% Vorspann für die Leseansicht.
%% Lädt die gemeinsame Datei latex-vorspann.tex mit nicht gesetztem Schalter.

\newif\ifkorrekturansicht
\korrekturansichtfalse

\input{../tex-inputs/latex-vorspann}


         
         \renewcommand{\erwaehntePersonen}{Personen: Clementine Goldmann, Fedor Mamroth, Olga Schnitzler}
         \renewcommand{\erwaehnteOrte}{Orte: Berlin, Dessauer Straße, Frankfurt am Main, Marienbad, Wien}
         \renewcommand{\erwaehnteWerke}{}
               \section[ Paul Goldmann an Arthur Schnitzler, 7. 7. 1907]{ Paul Goldmann an Arthur Schnitzler, 7. 7. 1907}\nopagebreak\mylabel{v}\rehead{ }\begin{ledgroupsized}[t]{13cm}\normalsize\beginnumbering \toendnotes[C]{\smallbreak\pagebreak[2]} \Standort{DLA, A:Schnitzler, HS.NZ85.1.3175.}
\physDesc{Brief, 1 Blatt, 4 Seiten
\newline{}Handschrift: blaue Tinte, deutsche Kurrent}\toendnotes[C]{\smallbreak}\pstart
           \noindent{}\raggedleft{}{\pb}\textcolor{gray}{\textbf{DESSAUERSTRASSE 19}}\oindex{Dessauer Strasse@\textbf{Dessauer Straße}|pw}\pend
           \pstart
           7. 7. 07.\pend
           \pstart{}Lieber Freund,\pend\pstart
           Das traurige \label{K-L03254-1v}\edtext{Ereignis}{\lemma{\textnormal{\emph{Ereignis}}}\Cendnote{\textnormal{Goldmann\pwindex{Goldmann, Paul 31.01.1865 – 25.09.1935@\textsc{Goldmann, Paul} (31.01.1865 – 25.09.1935), \emph{Schriftsteller, Journalist}|pwk}s Onkel Fedor Mamroth\pwindex{Mamroth, Fedor 21.02.1851 – 25.06.1907@\textsc{Mamroth, Fedor} (21.02.1851 – 25.06.1907), \emph{Journalist, Kritiker}|pwk} war am 25. 6. 1907 an den Folgen von Darmkrebs verstorben.}}}\label{K-L03254-1h} hat in
               ſeinem Gefolge eine ſolche Fülle von Angelegenheiten gehabt, die erledigt werden
               mußten, daß ich erſt heut dazu komme, Deinen lieben
               Brief zu beantworten u. Dir, auch im Namen der Meinigen, für Deine ſchönen,
               teilnehmenden Worte zu danken, die uns Alle tief berührt haben.\pend
           \pstart
           Mir iſt der Tod
               zum erſten Mal ganz in die Nähe gekommen, {\pb}u. ich
               habe ihn erkannt, als das, was er iſt: unſinnig u. ſcheußlich.\pend
           \pstart
           Das Schwerſte, das Du mir zu überwinden wünſchſt, waren nicht die Tage in Frankfurt\oindex{Frankfurt am Main@\textbf{Frankfurt am Main}|pw}. Das Schwerſte beginnt jetzt. Es iſt die
               Leere, die das Hinſcheiden eines geliebten Menſchen\pwindex{Mamroth, Fedor 21.02.1851 – 25.06.1907@\textsc{Mamroth, Fedor} (21.02.1851 – 25.06.1907), \emph{Journalist, Kritiker}|pwv} im Leben des Zurückgebliebenen läßt, – es iſt
               die Sehnſucht, ein teures Geſicht wiederzuſehen, eine vertraute Stimme zu hören, die
               man niemals wiederſehen u. \strikeout{wiederhoh\textcolor{gray}{re}}{ }\strikeout{wird} wiederhören wird, – {\pb}es iſt die Unmöglichkeit, ſich Jemanden\pwindex{Mamroth, Fedor 21.02.1851 – 25.06.1907@\textsc{Mamroth, Fedor} (21.02.1851 – 25.06.1907), \emph{Journalist, Kritiker}|pwv} als todt (todt!) vorzuſtellen, der
               noch vor Kurzem von Geiſt u. Leben ſprühte u. an dem man mit ganzer Seele gehangen
                  hat{\dotsseven}\pend
           \pstart
           Dir u. Deiner Frau\pwindex{Schnitzler, Olga 17.01.1882 – 13.01.1970@\textsc{Schnitzler, Olga} (17.01.1882 – 13.01.1970), \emph{Schauspielerin, Sängerin}|pwv} (der ich
               für ihre Teilnahme vielmals zu danken bitte) wünſche ich frohe Sommertage. Schreib’
               mir jedenfalls, wo Ihr\pwindex{Schnitzler, Olga 17.01.1882 – 13.01.1970@\textsc{Schnitzler, Olga} (17.01.1882 – 13.01.1970), \emph{Schauspielerin, Sängerin}|pwv} ſeid.
               Freilich iſt die Hoffnung gering, daß ich Euch\pwindex{Schnitzler, Olga 17.01.1882 – 13.01.1970@\textsc{Schnitzler, Olga} (17.01.1882 – 13.01.1970), \emph{Schauspielerin, Sängerin}|pwv}{ }\label{K-L03254-2v}\edtext{in dieſem Sommer ſehen}{\lemma{\textnormal{\emph{in dieſem Sommer ſehen}}}\Cendnote{\textnormal{Schnitzler\pwindex{Schnitzler, Arthur 15.05.1862 – 21.10.1931@\textsc{Schnitzler, Arthur} (15.05.1862 – 21.10.1931), \emph{Schriftsteller, Mediziner}|pwk} und Goldmann\pwindex{Goldmann, Paul 31.01.1865 – 25.09.1935@\textsc{Goldmann, Paul} (31.01.1865 – 25.09.1935), \emph{Schriftsteller, Journalist}|pwk} trafen sich erst am 8. 10. 1907
                  wieder.}}}\label{K-L03254-2h} werde, da ich diesmal meine Mutter\pwindex{Goldmann, Clementine 1842-05-15 – 1924-02-24@\textsc{Goldmann, Clementine} (1842-05-15 – 1924-02-24)|pw} nicht allein {\pb}laſſen u. mit ihr
               keine weiten Reiſen machen kann. Wahrſcheinlich gehen wir im Auguſt zunächſt nach Marienbad\oindex{Marienbad@\textbf{Marienbad}|pw}.\pend
           \pstart
           \label{K-03254-3v}\edtext{Mißverſtändniſſe}{\lemma{\textnormal{\emph{Mißverſtändniſſe}}}\Cendnote{\textnormal{Möglicherweise hatte es beim letzten
                  persönlichen Treffen am 2. 6. 1907 eine Auseinandersetzung gegeben. Ansonsten könnte es sich
            allgemein um die Verstimmungen der letzten Jahre handeln, die nie vollständig
            gekittet wurden.}}}\label{K-03254-3h} ſollen uns gewiß
               nicht mehr trennen. Ich bin wenigſtens diesmal von Wien\oindex{Wien@\textbf{Wien}|pw} mit dem feſten Vorſatz fortgefahren, Alles \strikeout{zu}, was an mir liegt, zu tun, um mir \substVorne{}\textsuperscript{\textcolor{gray}{me}}\substDazwischen{}eine\substHinten{} alte Freundſchaft zu erhalten, deren Wert ich gewiß nicht geringer bemeſſe,
               wie einſt\substVorne{}\textsuperscript{,}\substDazwischen{}.\substHinten{}\pend
           \pstart
           Nimm’ alſo nochmals meinen u. der Meinigen herzlichſten Dank u. ſei, ſammt
               Deiner Frau\pwindex{Schnitzler, Olga 17.01.1882 – 13.01.1970@\textsc{Schnitzler, Olga} (17.01.1882 – 13.01.1970), \emph{Schauspielerin, Sängerin}|pwv}, vielmals gegrüßt
               von {\\[\baselineskip]}Deinem {\\[\baselineskip]}\spacefill\mbox{Paul Goldmann.}\pend
           \leftskip=0em{}
         
         \endnumbering\mylabel{h}\end{ledgroupsized}  \newcommand{\dateiname}{L03254}\newcommand{\titel}{Paul Goldmann an Arthur Schnitzler, 7. 7. 1907}\newcommand{\editorInnen}{Martin Anton Müller und Laura Untner}%% latex-leseansicht-abspann.tex
%% Abspann für die Leseansicht.
%% Der Schalter \ifkorrekturansicht ist bereits durch den Vorspann gesetzt.

%% latex-abspann.tex
%% Gemeinsamer Abspann für Korrekturansicht und Leseansicht.
%% Setzt den Schalter \ifkorrekturansicht voraus (gesetzt in den
%% einbindenden Dateien latex-korrekturansicht-abspann.tex bzw.
%% latex-leseansicht-abspann.tex).
%% ---------------------------------------------------------------

\normalsize

% Das esempio-Environment wird nur in der Leseansicht benötigt
\ifkorrekturansicht\else
\newenvironment{esempio}[3]%
{
    \vspace{1.5ex}
    \rlap{\underline{#1}}
    \par
    \setlength{\parindent}{0cm}
    \nopagebreak
    \leftskip=#2cm
    \rightskip=#3cm
}
{
    \par
}
\fi

\doendnotes{C}
\bigskip
\vfill

\clearpage

\footnotesize

\ifkorrekturansicht
  \lohead{\textsc{register}}
\fi

% theindex-Environment neu definieren ohne reledmac
\makeatletter
\renewenvironment{theindex}{%
  \ifkorrekturansicht
    \section*{\indexname}%
  \else
    \subsubsection*{Index der erwähnten Entitäten}%
  \fi
  \setlength{\parindent}{0pt}%
  \setlength{\parskip}{0pt plus 0.3pt}%
  \let\item\@idxitem
}{%
  \ifkorrekturansicht\clearpage\fi
}
\makeatother

\IfFileExists{\jobname-pw.ind}{\input{\jobname-pw.ind}}{}

% Quellenangabe nur in der Leseansicht
\ifkorrekturansicht\else
% Fallback-Definitionen, falls die .tex-Datei \titel etc. nicht gesetzt hat
\providecommand{\titel}{}
\providecommand{\editorInnen}{}
\providecommand{\dateiname}{\jobname}

\vspace{3cm}

\vfill

\footnotesize
\textsc{Quelle}: \titel. Herausgegeben von {\editorInnen}. In: \emph{Arthur Schnitzler: Briefwechsel mit Autorinnen und Autoren}.
 Digitale Edition, https://schnitzler-briefe.acdh.oeaw.ac.at/{\dateiname}.html (Stand \today)
\fi

\end{document}


      